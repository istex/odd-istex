\documentclass[11pt,twoside]{article}\makeatletter

\IfFileExists{xcolor.sty}%
  {\RequirePackage{xcolor}}%
  {\RequirePackage{color}}
\usepackage{colortbl}
\usepackage{wrapfig}
\usepackage{ifxetex}
\ifxetex
  \usepackage{fontspec}
  \usepackage{xunicode}
  \catcode`⃥=\active \def⃥{\textbackslash}
  \catcode`❴=\active \def❴{\{}
  \catcode`❵=\active \def❵{\}}
  \def\textJapanese{\fontspec{IPAMincho}}
  \def\textChinese{\fontspec{HAN NOM A}\XeTeXlinebreaklocale "zh"\XeTeXlinebreakskip = 0pt plus 1pt }
  \def\textKorean{\fontspec{Baekmuk Gulim} }
  \setmonofont{DejaVu Sans Mono}
  
\else
  \IfFileExists{utf8x.def}%
   {\usepackage[utf8x]{inputenc}
      \PrerenderUnicode{–}
    }%
   {\usepackage[utf8]{inputenc}}
  \usepackage[english]{babel}
  \usepackage[T1]{fontenc}
  \usepackage{float}
  \usepackage[]{ucs}
  \uc@dclc{8421}{default}{\textbackslash }
  \uc@dclc{10100}{default}{\{}
  \uc@dclc{10101}{default}{\}}
  \uc@dclc{8491}{default}{\AA{}}
  \uc@dclc{8239}{default}{\,}
  \uc@dclc{20154}{default}{ }
  \uc@dclc{10148}{default}{>}
  \def\textschwa{\rotatebox{-90}{e}}
  \def\textJapanese{}
  \def\textChinese{}
  \IfFileExists{tipa.sty}{\usepackage{tipa}}{}
  \usepackage{times}
\fi
\def\exampleFont{\ttfamily\small}
\DeclareTextSymbol{\textpi}{OML}{25}
\usepackage{relsize}
\RequirePackage{array}
\def\@testpach{\@chclass
 \ifnum \@lastchclass=6 \@ne \@chnum \@ne \else
  \ifnum \@lastchclass=7 5 \else
   \ifnum \@lastchclass=8 \tw@ \else
    \ifnum \@lastchclass=9 \thr@@
   \else \z@
   \ifnum \@lastchclass = 10 \else
   \edef\@nextchar{\expandafter\string\@nextchar}%
   \@chnum
   \if \@nextchar c\z@ \else
    \if \@nextchar l\@ne \else
     \if \@nextchar r\tw@ \else
   \z@ \@chclass
   \if\@nextchar |\@ne \else
    \if \@nextchar !6 \else
     \if \@nextchar @7 \else
      \if \@nextchar (8 \else
       \if \@nextchar )9 \else
  10
  \@chnum
  \if \@nextchar m\thr@@\else
   \if \@nextchar p4 \else
    \if \@nextchar b5 \else
   \z@ \@chclass \z@ \@preamerr \z@ \fi \fi \fi \fi
   \fi \fi  \fi  \fi  \fi  \fi  \fi \fi \fi \fi \fi \fi}
\gdef\arraybackslash{\let\\=\@arraycr}
\def\@textsubscript#1{{\m@th\ensuremath{_{\mbox{\fontsize\sf@size\z@#1}}}}}
\def\Panel#1#2#3#4{\multicolumn{#3}{){\columncolor{#2}}#4}{#1}}
\def\abbr{}
\def\corr{}
\def\expan{}
\def\gap{}
\def\orig{}
\def\reg{}
\def\ref{}
\def\sic{}
\def\persName{}\def\name{}
\def\placeName{}
\def\orgName{}
\def\textcal#1{{\fontspec{Lucida Calligraphy}#1}}
\def\textgothic#1{{\fontspec{Lucida Blackletter}#1}}
\def\textlarge#1{{\large #1}}
\def\textoverbar#1{\ensuremath{\overline{#1}}}
\def\textquoted#1{‘#1’}
\def\textsmall#1{{\small #1}}
\def\textsubscript#1{\@textsubscript{\selectfont#1}}
\def\textxi{\ensuremath{\xi}}
\def\titlem{\itshape}
\newenvironment{biblfree}{}{\ifvmode\par\fi }
\newenvironment{bibl}{}{}
\newenvironment{byline}{\vskip6pt\itshape\fontsize{16pt}{18pt}\selectfont}{\par }
\newenvironment{citbibl}{}{\ifvmode\par\fi }
\newenvironment{docAuthor}{\ifvmode\vskip4pt\fontsize{16pt}{18pt}\selectfont\fi\itshape}{\ifvmode\par\fi }
\newenvironment{docDate}{}{\ifvmode\par\fi }
\newenvironment{docImprint}{\vskip 6pt}{\ifvmode\par\fi }
\newenvironment{docTitle}{\vskip6pt\bfseries\fontsize{18pt}{22pt}\selectfont}{\par }
\newenvironment{msHead}{\vskip 6pt}{\par}
\newenvironment{msItem}{\vskip 6pt}{\par}
\newenvironment{rubric}{}{}
\newenvironment{titlePart}{}{\par }

\newcolumntype{L}[1]{){\raggedright\arraybackslash}p{#1}}
\newcolumntype{C}[1]{){\centering\arraybackslash}p{#1}}
\newcolumntype{R}[1]{){\raggedleft\arraybackslash}p{#1}}
\newcolumntype{P}[1]{){\arraybackslash}p{#1}}
\newcolumntype{B}[1]{){\arraybackslash}b{#1}}
\newcolumntype{M}[1]{){\arraybackslash}m{#1}}
\definecolor{label}{gray}{0.75}
\def\unusedattribute#1{\sout{\textcolor{label}{#1}}}
\DeclareRobustCommand*{\xref}{\hyper@normalise\xref@}
\def\xref@#1#2{\hyper@linkurl{#2}{#1}}
\begingroup
\catcode`\_=\active
\gdef_#1{\ensuremath{\sb{\mathrm{#1}}}}
\endgroup
\mathcode`\_=\string"8000
\catcode`\_=12\relax

\usepackage[a4paper,twoside,lmargin=1in,rmargin=1in,tmargin=1in,bmargin=1in,marginparwidth=0.75in]{geometry}
\usepackage{framed}

\definecolor{shadecolor}{gray}{0.95}
\usepackage{longtable}
\usepackage[normalem]{ulem}
\usepackage{fancyvrb}
\usepackage{fancyhdr}
\usepackage{graphicx}
\usepackage{marginnote}

\renewcommand{\@cite}[1]{#1}


\renewcommand*{\marginfont}{\itshape\footnotesize}

\def\Gin@extensions{.pdf,.png,.jpg,.mps,.tif}

  \pagestyle{fancy}

\usepackage[pdftitle={Schéma TEI pour le projet ISTEX},
 pdfauthor={generated by Roma 4.18}]{hyperref}
\hyperbaseurl{}

	 \paperwidth210mm
	 \paperheight297mm
              
\def\@pnumwidth{1.55em}
\def\@tocrmarg {2.55em}
\def\@dotsep{4.5}
\setcounter{tocdepth}{3}
\clubpenalty=8000
\emergencystretch 3em
\hbadness=4000
\hyphenpenalty=400
\pretolerance=750
\tolerance=2000
\vbadness=4000
\widowpenalty=10000

\renewcommand\section{\@startsection {section}{1}{\z@}%
     {-1.75ex \@plus -0.5ex \@minus -.2ex}%
     {0.5ex \@plus .2ex}%
     {\reset@font\Large\bfseries\sffamily}}
\renewcommand\subsection{\@startsection{subsection}{2}{\z@}%
     {-1.75ex\@plus -0.5ex \@minus- .2ex}%
     {0.5ex \@plus .2ex}%
     {\reset@font\Large\sffamily}}
\renewcommand\subsubsection{\@startsection{subsubsection}{3}{\z@}%
     {-1.5ex\@plus -0.35ex \@minus -.2ex}%
     {0.5ex \@plus .2ex}%
     {\reset@font\large\sffamily}}
\renewcommand\paragraph{\@startsection{paragraph}{4}{\z@}%
     {-1ex \@plus-0.35ex \@minus -0.2ex}%
     {0.5ex \@plus .2ex}%
     {\reset@font\normalsize\sffamily}}
\renewcommand\subparagraph{\@startsection{subparagraph}{5}{\parindent}%
     {1.5ex \@plus1ex \@minus .2ex}%
     {-1em}%
     {\reset@font\normalsize\bfseries}}


\def\l@section#1#2{\addpenalty{\@secpenalty} \addvspace{1.0em plus 1pt}
 \@tempdima 1.5em \begingroup
 \parindent \z@ \rightskip \@pnumwidth 
 \parfillskip -\@pnumwidth 
 \bfseries \leavevmode #1\hfil \hbox to\@pnumwidth{\hss #2}\par
 \endgroup}
\def\l@subsection{\@dottedtocline{2}{1.5em}{2.3em}}
\def\l@subsubsection{\@dottedtocline{3}{3.8em}{3.2em}}
\def\l@paragraph{\@dottedtocline{4}{7.0em}{4.1em}}
\def\l@subparagraph{\@dottedtocline{5}{10em}{5em}}
\@ifundefined{c@section}{\newcounter{section}}{}
\@ifundefined{c@chapter}{\newcounter{chapter}}{}
\newif\if@mainmatter 
\@mainmattertrue
\def\chaptername{Chapter}
\def\frontmatter{%
  \pagenumbering{roman}
  \def\thechapter{\@roman\c@chapter}
  \def\theHchapter{\roman{chapter}}
  \def\thesection{\@roman\c@section}
  \def\theHsection{\roman{section}}
  \def\@chapapp{}%
}
\def\mainmatter{%
  \cleardoublepage
  \def\thechapter{\@arabic\c@chapter}
  \setcounter{chapter}{0}
  \setcounter{section}{0}
  \pagenumbering{arabic}
  \setcounter{secnumdepth}{6}
  \def\@chapapp{\chaptername}%
  \def\theHchapter{\arabic{chapter}}
  \def\thesection{\@arabic\c@section}
  \def\theHsection{\arabic{section}}
}
\def\backmatter{%
  \cleardoublepage
  \setcounter{chapter}{0}
  \setcounter{section}{0}
  \setcounter{secnumdepth}{2}
  \def\@chapapp{\appendixname}%
  \def\thechapter{\@Alph\c@chapter}
  \def\theHchapter{\Alph{chapter}}
  \appendix
}
\newenvironment{bibitemlist}[1]{%
   \list{\@biblabel{\@arabic\c@enumiv}}%
       {\settowidth\labelwidth{\@biblabel{#1}}%
        \leftmargin\labelwidth
        \advance\leftmargin\labelsep
        \@openbib@code
        \usecounter{enumiv}%
        \let\p@enumiv\@empty
        \renewcommand\theenumiv{\@arabic\c@enumiv}%
	}%
  \sloppy
  \clubpenalty4000
  \@clubpenalty \clubpenalty
  \widowpenalty4000%
  \sfcode`\.\@m}%
  {\def\@noitemerr
    {\@latex@warning{Empty `bibitemlist' environment}}%
    \endlist}

\def\tableofcontents{\section*{\contentsname}\@starttoc{toc}}
\parskip0pt
\parindent1em
\def\Panel#1#2#3#4{\multicolumn{#3}{){\columncolor{#2}}#4}{#1}}
\newenvironment{reflist}{%
  \begin{raggedright}\begin{list}{}
  {%
   \setlength{\topsep}{0pt}%
   \setlength{\rightmargin}{0.25in}%
   \setlength{\itemsep}{0pt}%
   \setlength{\itemindent}{0pt}%
   \setlength{\parskip}{0pt}%
   \setlength{\parsep}{2pt}%
   \def\makelabel##1{\itshape ##1}}%
  }
  {\end{list}\end{raggedright}}
\newenvironment{sansreflist}{%
  \begin{raggedright}\begin{list}{}
  {%
   \setlength{\topsep}{0pt}%
   \setlength{\rightmargin}{0.25in}%
   \setlength{\itemindent}{0pt}%
   \setlength{\parskip}{0pt}%
   \setlength{\itemsep}{0pt}%
   \setlength{\parsep}{2pt}%
   \def\makelabel##1{\upshape\sffamily ##1}}%
  }
  {\end{list}\end{raggedright}}
\newenvironment{specHead}[2]%
 {\vspace{20pt}\hrule\vspace{10pt}%
  \phantomsection\label{#1}\markright{#2}%

  \pdfbookmark[2]{#2}{#1}%
  \hspace{-0.75in}{\bfseries\fontsize{16pt}{18pt}\selectfont#2}%
  }{}
      \def\TheFullDate{2019-07-27 (revised: 2019-07-27T22:47:23Z)}
\def\TheID{\makeatother }
\def\TheDate{2019-07-27}
\title{Schéma TEI pour le projet ISTEX}
\author{generated by Roma 4.18}\makeatletter 
\makeatletter
\newcommand*{\cleartoleftpage}{%
  \clearpage
    \if@twoside
    \ifodd\c@page
      \hbox{}\newpage
      \if@twocolumn
        \hbox{}\newpage
      \fi
    \fi
  \fi
}
\makeatother
\makeatletter
\thispagestyle{empty}
\markright{\@title}\markboth{\@title}{\@author}
\renewcommand\small{\@setfontsize\small{9pt}{11pt}\abovedisplayskip 8.5\p@ plus3\p@ minus4\p@
\belowdisplayskip \abovedisplayskip
\abovedisplayshortskip \z@ plus2\p@
\belowdisplayshortskip 4\p@ plus2\p@ minus2\p@
\def\@listi{\leftmargin\leftmargini
               \topsep 2\p@ plus1\p@ minus1\p@
               \parsep 2\p@ plus\p@ minus\p@
               \itemsep 1pt}
}
\makeatother
\fvset{frame=single,numberblanklines=false,xleftmargin=5mm,xrightmargin=5mm}
\fancyhf{} 
\setlength{\headheight}{14pt}
\fancyhead[LE]{\bfseries\leftmark} 
\fancyhead[RO]{\bfseries\rightmark} 
\fancyfoot[RO]{}
\fancyfoot[CO]{\thepage}
\fancyfoot[LO]{\TheID}
\fancyfoot[LE]{}
\fancyfoot[CE]{\thepage}
\fancyfoot[RE]{\TheID}
\hypersetup{citebordercolor=0.75 0.75 0.75,linkbordercolor=0.75 0.75 0.75,urlbordercolor=0.75 0.75 0.75,bookmarksnumbered=true}
\fancypagestyle{plain}{\fancyhead{}\renewcommand{\headrulewidth}{0pt}}\makeatother 
\begin{document}

\makeatletter
\noindent\parbox[b]{.75\textwidth}{\fontsize{14pt}{16pt}\bfseries\raggedright\sffamily\selectfont \@title}
\vskip20pt
\par\noindent{\fontsize{11pt}{13pt}\sffamily\itshape\raggedright\selectfont\@author\hfill\TheDate}
\vspace{18pt}
\makeatother
\let\tabcellsep&\frontmatter 
     \tableofcontents
   \mainmatter \par
Spécification d'un schéma ISTEX pour les sorties Tei
\section[{Elements}]{Elements}\index{TEI=<TEI>|oddindex}\index{version=@version!<TEI>|oddindex}
\begin{reflist}
\item[]\begin{specHead}{TEI.TEI}{<TEI> }(document TEI) contient un seul document conforme à la TEI, qui comprend un en-tête TEI et un texte, soit de façon isolée, soit comme partie d’un élément \hyperref[TEI.teiCorpus]{<teiCorpus>}. [\xref{http://www.tei-c.org/release/doc/tei-p5-doc/en/html/DS.html\#DS}{4. Default Text Structure} \xref{http://www.tei-c.org/release/doc/tei-p5-doc/en/html/CC.html\#CCDEF}{15.1. Varieties of Composite Text}]\end{specHead} 
    \item[{Namespace}]
  
    \item[{Module}]
  textstructure
    \item[{Attributs}]
  Attributs \hyperref[TEI.att.global]{att.global} (\textit{@xml:id}, \textit{@n}, \textit{@xml:lang}, \textit{@xml:base}, \textit{@xml:space})  (\hyperref[TEI.att.global.rendition]{att.global.rendition} (\textit{@rend}, \textit{@style}, \textit{@rendition})) (\hyperref[TEI.att.global.linking]{att.global.linking} (\textit{@corresp}, \textit{@synch}, \textit{@sameAs}, \textit{@copyOf}, \textit{@next}, \textit{@prev}, \textit{@exclude}, \textit{@select})) (\hyperref[TEI.att.global.analytic]{att.global.analytic} (\textit{@ana})) (\hyperref[TEI.att.global.facs]{att.global.facs} (\textit{@facs})) (\hyperref[TEI.att.global.change]{att.global.change} (\textit{@change})) (\hyperref[TEI.att.global.responsibility]{att.global.responsibility} (\textit{@cert}, \textit{@resp})) (\hyperref[TEI.att.global.source]{att.global.source} (\textit{@source})) \hyperref[TEI.att.typed]{att.typed} (\textit{@type}, \textit{@subtype}) \hfil\\[-10pt]\begin{sansreflist}
    \item[@version]
  la version majeure du schéma TEI
\begin{reflist}
    \item[{Statut}]
  Optionel
    \item[{Type de données}]
  \hyperref[TEI.teidata.version]{teidata.version}
\end{reflist}  
\end{sansreflist}  
    \item[{Contenu dans}]
  
    \item[core: ]
   \hyperref[TEI.teiCorpus]{teiCorpus}
    \item[{Peut contenir}]
  
    \item[header: ]
   \hyperref[TEI.teiHeader]{teiHeader}\par 
    \item[iso-fs: ]
   \hyperref[TEI.fsdDecl]{fsdDecl}\par 
    \item[standOff: ]
   \hyperref[TEI.standOff]{standOff}\par 
    \item[textstructure: ]
   \hyperref[TEI.text]{text}\par 
    \item[transcr: ]
   \hyperref[TEI.facsimile]{facsimile} \hyperref[TEI.sourceDoc]{sourceDoc}
    \item[{Note}]
  \par
Cet élément est obligatoire.
    \item[{Exemple}]
  \leavevmode\bgroup\exampleFont \begin{shaded}\noindent\mbox{}{<\textbf{TEI}\hspace*{6pt}{version}="{5.0}" xmlns="http://www.tei-c.org/ns/1.0">}\mbox{}\newline 
\hspace*{6pt}{<\textbf{teiHeader}>}\mbox{}\newline 
\hspace*{6pt}\hspace*{6pt}{<\textbf{fileDesc}>}\mbox{}\newline 
\hspace*{6pt}\hspace*{6pt}\hspace*{6pt}{<\textbf{titleStmt}>}\mbox{}\newline 
\hspace*{6pt}\hspace*{6pt}\hspace*{6pt}\hspace*{6pt}{<\textbf{title}>}Le document TEI le plus court possible.{</\textbf{title}>}\mbox{}\newline 
\hspace*{6pt}\hspace*{6pt}\hspace*{6pt}{</\textbf{titleStmt}>}\mbox{}\newline 
\hspace*{6pt}\hspace*{6pt}\hspace*{6pt}{<\textbf{publicationStmt}>}\mbox{}\newline 
\hspace*{6pt}\hspace*{6pt}\hspace*{6pt}\hspace*{6pt}{<\textbf{p}>}D'abord publié comme faisant partie de la TEI P2.{</\textbf{p}>}\mbox{}\newline 
\hspace*{6pt}\hspace*{6pt}\hspace*{6pt}{</\textbf{publicationStmt}>}\mbox{}\newline 
\hspace*{6pt}\hspace*{6pt}\hspace*{6pt}{<\textbf{sourceDesc}>}\mbox{}\newline 
\hspace*{6pt}\hspace*{6pt}\hspace*{6pt}\hspace*{6pt}{<\textbf{p}>}Aucune source : il s'agit d'un document original.{</\textbf{p}>}\mbox{}\newline 
\hspace*{6pt}\hspace*{6pt}\hspace*{6pt}{</\textbf{sourceDesc}>}\mbox{}\newline 
\hspace*{6pt}\hspace*{6pt}{</\textbf{fileDesc}>}\mbox{}\newline 
\hspace*{6pt}{</\textbf{teiHeader}>}\mbox{}\newline 
\hspace*{6pt}{<\textbf{text}>}\mbox{}\newline 
\hspace*{6pt}\hspace*{6pt}{<\textbf{body}>}\mbox{}\newline 
\hspace*{6pt}\hspace*{6pt}\hspace*{6pt}{<\textbf{p}>}A peu pres, le document TEI le plus court envisageable.{</\textbf{p}>}\mbox{}\newline 
\hspace*{6pt}\hspace*{6pt}{</\textbf{body}>}\mbox{}\newline 
\hspace*{6pt}{</\textbf{text}>}\mbox{}\newline 
{</\textbf{TEI}>}\end{shaded}\egroup 


    \item[{Schematron}]
   <s:ns prefix="tei"  uri="http://www.tei-c.org/ns/1.0"/> <s:ns prefix="xs"  uri="http://www.w3.org/2001/XMLSchema"/>
    \item[{Schematron}]
   <s:ns prefix="rng"  uri="http://relaxng.org/ns/structure/1.0"/>
    \item[{Modèle de contenu}]
  \mbox{}\hfill\\[-10pt]\begin{Verbatim}[fontsize=\small]
<content>
 <sequence maxOccurs="1" minOccurs="1">
  <elementRef key="teiHeader"/>
  <classRef key="model.resourceLike"
   maxOccurs="unbounded" minOccurs="0"/>
 </sequence>
</content>
    
\end{Verbatim}

    \item[{Schéma Declaration}]
  \mbox{}\hfill\\[-10pt]\begin{Verbatim}[fontsize=\small]
element TEI
{
   tei_att.global.attributes,
   tei_att.typed.attributes,
   attribute version { text }?,
   ( tei_teiHeader, tei_model.resourceLike* )
}
\end{Verbatim}

\end{reflist}  \index{ab=<ab>|oddindex}
\begin{reflist}
\item[]\begin{specHead}{TEI.ab}{<ab> }(bloc anonyme) contient une unité de texte quelconque, de niveau "composant", faisant office de contenant anonyme pour une expression ou des éléments de niveau intermédiaire, analogue à un paragraphe mais sans sa portée sémantique. [\xref{http://www.tei-c.org/release/doc/tei-p5-doc/en/html/SA.html\#SASE}{16.3. Blocks, Segments, and Anchors}]\end{specHead} 
    \item[{Module}]
  linking
    \item[{Attributs}]
  Attributs \hyperref[TEI.att.global]{att.global} (\textit{@xml:id}, \textit{@n}, \textit{@xml:lang}, \textit{@xml:base}, \textit{@xml:space})  (\hyperref[TEI.att.global.rendition]{att.global.rendition} (\textit{@rend}, \textit{@style}, \textit{@rendition})) (\hyperref[TEI.att.global.linking]{att.global.linking} (\textit{@corresp}, \textit{@synch}, \textit{@sameAs}, \textit{@copyOf}, \textit{@next}, \textit{@prev}, \textit{@exclude}, \textit{@select})) (\hyperref[TEI.att.global.analytic]{att.global.analytic} (\textit{@ana})) (\hyperref[TEI.att.global.facs]{att.global.facs} (\textit{@facs})) (\hyperref[TEI.att.global.change]{att.global.change} (\textit{@change})) (\hyperref[TEI.att.global.responsibility]{att.global.responsibility} (\textit{@cert}, \textit{@resp})) (\hyperref[TEI.att.global.source]{att.global.source} (\textit{@source})) \hyperref[TEI.att.typed]{att.typed} (\textit{@type}, \textit{@subtype}) \hyperref[TEI.att.declaring]{att.declaring} (\textit{@decls}) \hyperref[TEI.att.fragmentable]{att.fragmentable} (\textit{@part}) \hyperref[TEI.att.written]{att.written} (\textit{@hand}) 
    \item[{Membre du}]
  \hyperref[TEI.model.pLike]{model.pLike}
    \item[{Contenu dans}]
  
    \item[core: ]
   \hyperref[TEI.item]{item} \hyperref[TEI.note]{note} \hyperref[TEI.q]{q} \hyperref[TEI.quote]{quote} \hyperref[TEI.said]{said} \hyperref[TEI.sp]{sp} \hyperref[TEI.stage]{stage}\par 
    \item[figures: ]
   \hyperref[TEI.cell]{cell} \hyperref[TEI.figure]{figure}\par 
    \item[header: ]
   \hyperref[TEI.abstract]{abstract} \hyperref[TEI.application]{application} \hyperref[TEI.availability]{availability} \hyperref[TEI.change]{change} \hyperref[TEI.correction]{correction} \hyperref[TEI.editionStmt]{editionStmt} \hyperref[TEI.encodingDesc]{encodingDesc} \hyperref[TEI.langUsage]{langUsage} \hyperref[TEI.licence]{licence} \hyperref[TEI.publicationStmt]{publicationStmt} \hyperref[TEI.seriesStmt]{seriesStmt} \hyperref[TEI.sourceDesc]{sourceDesc}\par 
    \item[msdescription: ]
   \hyperref[TEI.accMat]{accMat} \hyperref[TEI.acquisition]{acquisition} \hyperref[TEI.additions]{additions} \hyperref[TEI.binding]{binding} \hyperref[TEI.bindingDesc]{bindingDesc} \hyperref[TEI.collation]{collation} \hyperref[TEI.condition]{condition} \hyperref[TEI.custEvent]{custEvent} \hyperref[TEI.custodialHist]{custodialHist} \hyperref[TEI.decoDesc]{decoDesc} \hyperref[TEI.decoNote]{decoNote} \hyperref[TEI.filiation]{filiation} \hyperref[TEI.foliation]{foliation} \hyperref[TEI.handDesc]{handDesc} \hyperref[TEI.history]{history} \hyperref[TEI.layout]{layout} \hyperref[TEI.layoutDesc]{layoutDesc} \hyperref[TEI.msContents]{msContents} \hyperref[TEI.msDesc]{msDesc} \hyperref[TEI.msFrag]{msFrag} \hyperref[TEI.msItem]{msItem} \hyperref[TEI.msItemStruct]{msItemStruct} \hyperref[TEI.msPart]{msPart} \hyperref[TEI.musicNotation]{musicNotation} \hyperref[TEI.objectDesc]{objectDesc} \hyperref[TEI.origin]{origin} \hyperref[TEI.physDesc]{physDesc} \hyperref[TEI.provenance]{provenance} \hyperref[TEI.recordHist]{recordHist} \hyperref[TEI.scriptDesc]{scriptDesc} \hyperref[TEI.seal]{seal} \hyperref[TEI.sealDesc]{sealDesc} \hyperref[TEI.signatures]{signatures} \hyperref[TEI.source]{source} \hyperref[TEI.summary]{summary} \hyperref[TEI.support]{support} \hyperref[TEI.supportDesc]{supportDesc} \hyperref[TEI.surrogates]{surrogates} \hyperref[TEI.typeDesc]{typeDesc} \hyperref[TEI.typeNote]{typeNote}\par 
    \item[namesdates: ]
   \hyperref[TEI.event]{event} \hyperref[TEI.org]{org} \hyperref[TEI.person]{person} \hyperref[TEI.personGrp]{personGrp} \hyperref[TEI.persona]{persona} \hyperref[TEI.place]{place} \hyperref[TEI.state]{state}\par 
    \item[textstructure: ]
   \hyperref[TEI.back]{back} \hyperref[TEI.body]{body} \hyperref[TEI.div]{div} \hyperref[TEI.front]{front}\par 
    \item[transcr: ]
   \hyperref[TEI.metamark]{metamark}
    \item[{Peut contenir}]
  
    \item[analysis: ]
   \hyperref[TEI.c]{c} \hyperref[TEI.cl]{cl} \hyperref[TEI.interp]{interp} \hyperref[TEI.interpGrp]{interpGrp} \hyperref[TEI.m]{m} \hyperref[TEI.pc]{pc} \hyperref[TEI.phr]{phr} \hyperref[TEI.s]{s} \hyperref[TEI.span]{span} \hyperref[TEI.spanGrp]{spanGrp} \hyperref[TEI.w]{w}\par 
    \item[core: ]
   \hyperref[TEI.abbr]{abbr} \hyperref[TEI.add]{add} \hyperref[TEI.address]{address} \hyperref[TEI.bibl]{bibl} \hyperref[TEI.biblStruct]{biblStruct} \hyperref[TEI.binaryObject]{binaryObject} \hyperref[TEI.cb]{cb} \hyperref[TEI.choice]{choice} \hyperref[TEI.cit]{cit} \hyperref[TEI.corr]{corr} \hyperref[TEI.date]{date} \hyperref[TEI.del]{del} \hyperref[TEI.desc]{desc} \hyperref[TEI.distinct]{distinct} \hyperref[TEI.email]{email} \hyperref[TEI.emph]{emph} \hyperref[TEI.expan]{expan} \hyperref[TEI.foreign]{foreign} \hyperref[TEI.gap]{gap} \hyperref[TEI.gb]{gb} \hyperref[TEI.gloss]{gloss} \hyperref[TEI.graphic]{graphic} \hyperref[TEI.hi]{hi} \hyperref[TEI.index]{index} \hyperref[TEI.l]{l} \hyperref[TEI.label]{label} \hyperref[TEI.lb]{lb} \hyperref[TEI.lg]{lg} \hyperref[TEI.list]{list} \hyperref[TEI.listBibl]{listBibl} \hyperref[TEI.measure]{measure} \hyperref[TEI.measureGrp]{measureGrp} \hyperref[TEI.media]{media} \hyperref[TEI.mentioned]{mentioned} \hyperref[TEI.milestone]{milestone} \hyperref[TEI.name]{name} \hyperref[TEI.note]{note} \hyperref[TEI.num]{num} \hyperref[TEI.orig]{orig} \hyperref[TEI.pb]{pb} \hyperref[TEI.ptr]{ptr} \hyperref[TEI.q]{q} \hyperref[TEI.quote]{quote} \hyperref[TEI.ref]{ref} \hyperref[TEI.reg]{reg} \hyperref[TEI.rs]{rs} \hyperref[TEI.said]{said} \hyperref[TEI.sic]{sic} \hyperref[TEI.soCalled]{soCalled} \hyperref[TEI.stage]{stage} \hyperref[TEI.term]{term} \hyperref[TEI.time]{time} \hyperref[TEI.title]{title} \hyperref[TEI.unclear]{unclear}\par 
    \item[derived-module-tei.istex: ]
   \hyperref[TEI.math]{math} \hyperref[TEI.mrow]{mrow}\par 
    \item[figures: ]
   \hyperref[TEI.figure]{figure} \hyperref[TEI.formula]{formula} \hyperref[TEI.notatedMusic]{notatedMusic} \hyperref[TEI.table]{table}\par 
    \item[header: ]
   \hyperref[TEI.biblFull]{biblFull} \hyperref[TEI.idno]{idno}\par 
    \item[iso-fs: ]
   \hyperref[TEI.fLib]{fLib} \hyperref[TEI.fs]{fs} \hyperref[TEI.fvLib]{fvLib}\par 
    \item[linking: ]
   \hyperref[TEI.alt]{alt} \hyperref[TEI.altGrp]{altGrp} \hyperref[TEI.anchor]{anchor} \hyperref[TEI.join]{join} \hyperref[TEI.joinGrp]{joinGrp} \hyperref[TEI.link]{link} \hyperref[TEI.linkGrp]{linkGrp} \hyperref[TEI.seg]{seg} \hyperref[TEI.timeline]{timeline}\par 
    \item[msdescription: ]
   \hyperref[TEI.catchwords]{catchwords} \hyperref[TEI.depth]{depth} \hyperref[TEI.dim]{dim} \hyperref[TEI.dimensions]{dimensions} \hyperref[TEI.height]{height} \hyperref[TEI.heraldry]{heraldry} \hyperref[TEI.locus]{locus} \hyperref[TEI.locusGrp]{locusGrp} \hyperref[TEI.material]{material} \hyperref[TEI.msDesc]{msDesc} \hyperref[TEI.objectType]{objectType} \hyperref[TEI.origDate]{origDate} \hyperref[TEI.origPlace]{origPlace} \hyperref[TEI.secFol]{secFol} \hyperref[TEI.signatures]{signatures} \hyperref[TEI.source]{source} \hyperref[TEI.stamp]{stamp} \hyperref[TEI.watermark]{watermark} \hyperref[TEI.width]{width}\par 
    \item[namesdates: ]
   \hyperref[TEI.addName]{addName} \hyperref[TEI.affiliation]{affiliation} \hyperref[TEI.country]{country} \hyperref[TEI.forename]{forename} \hyperref[TEI.genName]{genName} \hyperref[TEI.geogName]{geogName} \hyperref[TEI.listOrg]{listOrg} \hyperref[TEI.listPlace]{listPlace} \hyperref[TEI.location]{location} \hyperref[TEI.nameLink]{nameLink} \hyperref[TEI.orgName]{orgName} \hyperref[TEI.persName]{persName} \hyperref[TEI.placeName]{placeName} \hyperref[TEI.region]{region} \hyperref[TEI.roleName]{roleName} \hyperref[TEI.settlement]{settlement} \hyperref[TEI.state]{state} \hyperref[TEI.surname]{surname}\par 
    \item[spoken: ]
   \hyperref[TEI.annotationBlock]{annotationBlock}\par 
    \item[textstructure: ]
   \hyperref[TEI.floatingText]{floatingText}\par 
    \item[transcr: ]
   \hyperref[TEI.addSpan]{addSpan} \hyperref[TEI.am]{am} \hyperref[TEI.damage]{damage} \hyperref[TEI.damageSpan]{damageSpan} \hyperref[TEI.delSpan]{delSpan} \hyperref[TEI.ex]{ex} \hyperref[TEI.fw]{fw} \hyperref[TEI.handShift]{handShift} \hyperref[TEI.listTranspose]{listTranspose} \hyperref[TEI.metamark]{metamark} \hyperref[TEI.mod]{mod} \hyperref[TEI.redo]{redo} \hyperref[TEI.restore]{restore} \hyperref[TEI.retrace]{retrace} \hyperref[TEI.secl]{secl} \hyperref[TEI.space]{space} \hyperref[TEI.subst]{subst} \hyperref[TEI.substJoin]{substJoin} \hyperref[TEI.supplied]{supplied} \hyperref[TEI.surplus]{surplus} \hyperref[TEI.undo]{undo}\par des données textuelles
    \item[{Note}]
  \par
L'élément \hyperref[TEI.ab]{<ab>} peut être utilisé à la discrétion de l'encodeur pour marquer dans un texte tout élément de niveau composant pour lequel aucune méthode appropriée de balisage plus spécifique n'est définie.
    \item[{Exemple}]
  \leavevmode\bgroup\exampleFont \begin{shaded}\noindent\mbox{}{<\textbf{div}\hspace*{6pt}{n}="{Genesis}"\hspace*{6pt}{type}="{book}">}\mbox{}\newline 
\hspace*{6pt}{<\textbf{div}\hspace*{6pt}{n}="{1}"\hspace*{6pt}{type}="{chapter}">}\mbox{}\newline 
\hspace*{6pt}\hspace*{6pt}{<\textbf{ab}>}In the beginning God created the heaven and the earth.{</\textbf{ab}>}\mbox{}\newline 
\hspace*{6pt}\hspace*{6pt}{<\textbf{ab}>}And the earth was without form, and void; and\mbox{}\newline 
\hspace*{6pt}\hspace*{6pt}\hspace*{6pt}\hspace*{6pt} darkness was upon the face of the deep. And the\mbox{}\newline 
\hspace*{6pt}\hspace*{6pt}\hspace*{6pt}\hspace*{6pt} spirit of God moved upon the face of the waters.{</\textbf{ab}>}\mbox{}\newline 
\hspace*{6pt}\hspace*{6pt}{<\textbf{ab}>}And God said, Let there be light: and there was light.{</\textbf{ab}>}\mbox{}\newline 
\textit{<!-- ...-->}\mbox{}\newline 
\hspace*{6pt}{</\textbf{div}>}\mbox{}\newline 
{</\textbf{div}>}\end{shaded}\egroup 


    \item[{Schematron}]
   <s:report test="not(ancestor::floatingText) and (ancestor::tei:p or ancestor::tei:ab)   and not(parent::tei:exemplum |parent::tei:item |parent::tei:note |parent::tei:q   |parent::tei:quote |parent::tei:remarks |parent::tei:said |parent::tei:sp   |parent::tei:stage |parent::tei:cell |parent::tei:figure)"> Abstract model violation: ab may not contain paragraphs or other ab elements. </s:report>
    \item[{Schematron}]
   <s:report test="ancestor::tei:l or ancestor::tei:lg"> Abstract model violation: Lines may not contain higher-level divisions such as p or ab. </s:report>
    \item[{Modèle de contenu}]
  \mbox{}\hfill\\[-10pt]\begin{Verbatim}[fontsize=\small]
<content>
 <macroRef key="macro.paraContent"/>
</content>
    
\end{Verbatim}

    \item[{Schéma Declaration}]
  \mbox{}\hfill\\[-10pt]\begin{Verbatim}[fontsize=\small]
element ab
{
   tei_att.global.attributes,
   tei_att.typed.attributes,
   tei_att.declaring.attributes,
   tei_att.fragmentable.attributes,
   tei_att.written.attributes,
   tei_macro.paraContent}
\end{Verbatim}

\end{reflist}  \index{abbr=<abbr>|oddindex}\index{type=@type!<abbr>|oddindex}
\begin{reflist}
\item[]\begin{specHead}{TEI.abbr}{<abbr> }(abréviation) contient une abréviation quelconque. [\xref{http://www.tei-c.org/release/doc/tei-p5-doc/en/html/CO.html\#CONAAB}{3.5.5. Abbreviations and Their Expansions}]\end{specHead} 
    \item[{Module}]
  core
    \item[{Attributs}]
  Attributs \hyperref[TEI.att.global]{att.global} (\textit{@xml:id}, \textit{@n}, \textit{@xml:lang}, \textit{@xml:base}, \textit{@xml:space})  (\hyperref[TEI.att.global.rendition]{att.global.rendition} (\textit{@rend}, \textit{@style}, \textit{@rendition})) (\hyperref[TEI.att.global.linking]{att.global.linking} (\textit{@corresp}, \textit{@synch}, \textit{@sameAs}, \textit{@copyOf}, \textit{@next}, \textit{@prev}, \textit{@exclude}, \textit{@select})) (\hyperref[TEI.att.global.analytic]{att.global.analytic} (\textit{@ana})) (\hyperref[TEI.att.global.facs]{att.global.facs} (\textit{@facs})) (\hyperref[TEI.att.global.change]{att.global.change} (\textit{@change})) (\hyperref[TEI.att.global.responsibility]{att.global.responsibility} (\textit{@cert}, \textit{@resp})) (\hyperref[TEI.att.global.source]{att.global.source} (\textit{@source})) \hyperref[TEI.att.typed]{att.typed} (\unusedattribute{type}, @subtype) \hfil\\[-10pt]\begin{sansreflist}
    \item[@type]
  permet à l'encodeur de caractériser l'abréviation selon une typologie adéquate
\begin{reflist}
    \item[{Dérivé de}]
  \hyperref[TEI.att.typed]{att.typed}
    \item[{Statut}]
  Optionel
    \item[{Type de données}]
  \hyperref[TEI.teidata.enumerated]{teidata.enumerated}
    \item[{Exemple de valeurs possibles:}]
  \begin{description}

\item[{suspension}]l'abréviation donne la première lettre lettre du mot ou de l'expression et omet le reste.
\item[{contraction}]l'abréviation omet une ou plusieurs lettres au milieu.
\item[{brevigraph}]l'abréviation comprend un symbole spécial ou une marque.
\item[{superscription}]l'abréviation inclut ce qui est écrit au-dessus de la ligne.
\item[{acronym}]l'abréviation comprend les initiales des mots d'une expression.
\item[{title}]l'abréviation recouvre une identité sociale (Dr., Mme, M., …)
\item[{organization}]l'abréviation recouvre le nom d'un organisme.
\item[{geographic}]l'abréviation recouvre un nom géographique.
\end{description} 
    \item[{Note}]
  \par
L'attribut {\itshape type} est donné si on souhaite typer les abréviations à l'endroit où elles apparaissent ; cela peut être utile dans certaines circonstances bien qu'une abréviation conserve la même signification dans toutes ses occurrences. Comme les échantillons des valeurs le montrent, les abréviations peuvent être typées selon la méthode utilisée pour leur construction, pour leur écriture, ou le référent du terme abrégé ; la typologie utilisée dépend de l'encodeur et doit être pensée soigneusement afin de correspondre aux attentes. Pour une typologie des abréviations concernant le Moyen Anglais, voir PETTY.
\end{reflist}  
\end{sansreflist}  
    \item[{Membre du}]
  \hyperref[TEI.model.choicePart]{model.choicePart} \hyperref[TEI.model.pPart.editorial]{model.pPart.editorial}
    \item[{Contenu dans}]
  
    \item[analysis: ]
   \hyperref[TEI.cl]{cl} \hyperref[TEI.pc]{pc} \hyperref[TEI.phr]{phr} \hyperref[TEI.s]{s} \hyperref[TEI.span]{span} \hyperref[TEI.w]{w}\par 
    \item[core: ]
   \hyperref[TEI.abbr]{abbr} \hyperref[TEI.add]{add} \hyperref[TEI.addrLine]{addrLine} \hyperref[TEI.author]{author} \hyperref[TEI.bibl]{bibl} \hyperref[TEI.biblScope]{biblScope} \hyperref[TEI.choice]{choice} \hyperref[TEI.citedRange]{citedRange} \hyperref[TEI.corr]{corr} \hyperref[TEI.date]{date} \hyperref[TEI.del]{del} \hyperref[TEI.desc]{desc} \hyperref[TEI.distinct]{distinct} \hyperref[TEI.editor]{editor} \hyperref[TEI.email]{email} \hyperref[TEI.emph]{emph} \hyperref[TEI.expan]{expan} \hyperref[TEI.foreign]{foreign} \hyperref[TEI.gloss]{gloss} \hyperref[TEI.head]{head} \hyperref[TEI.headItem]{headItem} \hyperref[TEI.headLabel]{headLabel} \hyperref[TEI.hi]{hi} \hyperref[TEI.item]{item} \hyperref[TEI.l]{l} \hyperref[TEI.label]{label} \hyperref[TEI.measure]{measure} \hyperref[TEI.meeting]{meeting} \hyperref[TEI.mentioned]{mentioned} \hyperref[TEI.name]{name} \hyperref[TEI.note]{note} \hyperref[TEI.num]{num} \hyperref[TEI.orig]{orig} \hyperref[TEI.p]{p} \hyperref[TEI.pubPlace]{pubPlace} \hyperref[TEI.publisher]{publisher} \hyperref[TEI.q]{q} \hyperref[TEI.quote]{quote} \hyperref[TEI.ref]{ref} \hyperref[TEI.reg]{reg} \hyperref[TEI.resp]{resp} \hyperref[TEI.rs]{rs} \hyperref[TEI.said]{said} \hyperref[TEI.sic]{sic} \hyperref[TEI.soCalled]{soCalled} \hyperref[TEI.speaker]{speaker} \hyperref[TEI.stage]{stage} \hyperref[TEI.street]{street} \hyperref[TEI.term]{term} \hyperref[TEI.textLang]{textLang} \hyperref[TEI.time]{time} \hyperref[TEI.title]{title} \hyperref[TEI.unclear]{unclear}\par 
    \item[figures: ]
   \hyperref[TEI.cell]{cell} \hyperref[TEI.figDesc]{figDesc}\par 
    \item[header: ]
   \hyperref[TEI.authority]{authority} \hyperref[TEI.change]{change} \hyperref[TEI.classCode]{classCode} \hyperref[TEI.creation]{creation} \hyperref[TEI.distributor]{distributor} \hyperref[TEI.edition]{edition} \hyperref[TEI.extent]{extent} \hyperref[TEI.funder]{funder} \hyperref[TEI.language]{language} \hyperref[TEI.licence]{licence} \hyperref[TEI.rendition]{rendition}\par 
    \item[iso-fs: ]
   \hyperref[TEI.fDescr]{fDescr} \hyperref[TEI.fsDescr]{fsDescr}\par 
    \item[linking: ]
   \hyperref[TEI.ab]{ab} \hyperref[TEI.seg]{seg}\par 
    \item[msdescription: ]
   \hyperref[TEI.accMat]{accMat} \hyperref[TEI.acquisition]{acquisition} \hyperref[TEI.additions]{additions} \hyperref[TEI.catchwords]{catchwords} \hyperref[TEI.collation]{collation} \hyperref[TEI.colophon]{colophon} \hyperref[TEI.condition]{condition} \hyperref[TEI.custEvent]{custEvent} \hyperref[TEI.decoNote]{decoNote} \hyperref[TEI.explicit]{explicit} \hyperref[TEI.filiation]{filiation} \hyperref[TEI.finalRubric]{finalRubric} \hyperref[TEI.foliation]{foliation} \hyperref[TEI.heraldry]{heraldry} \hyperref[TEI.incipit]{incipit} \hyperref[TEI.layout]{layout} \hyperref[TEI.material]{material} \hyperref[TEI.musicNotation]{musicNotation} \hyperref[TEI.objectType]{objectType} \hyperref[TEI.origDate]{origDate} \hyperref[TEI.origPlace]{origPlace} \hyperref[TEI.origin]{origin} \hyperref[TEI.provenance]{provenance} \hyperref[TEI.rubric]{rubric} \hyperref[TEI.secFol]{secFol} \hyperref[TEI.signatures]{signatures} \hyperref[TEI.source]{source} \hyperref[TEI.stamp]{stamp} \hyperref[TEI.summary]{summary} \hyperref[TEI.support]{support} \hyperref[TEI.surrogates]{surrogates} \hyperref[TEI.typeNote]{typeNote} \hyperref[TEI.watermark]{watermark}\par 
    \item[namesdates: ]
   \hyperref[TEI.addName]{addName} \hyperref[TEI.affiliation]{affiliation} \hyperref[TEI.country]{country} \hyperref[TEI.forename]{forename} \hyperref[TEI.genName]{genName} \hyperref[TEI.geogName]{geogName} \hyperref[TEI.nameLink]{nameLink} \hyperref[TEI.orgName]{orgName} \hyperref[TEI.persName]{persName} \hyperref[TEI.placeName]{placeName} \hyperref[TEI.region]{region} \hyperref[TEI.roleName]{roleName} \hyperref[TEI.settlement]{settlement} \hyperref[TEI.surname]{surname}\par 
    \item[textstructure: ]
   \hyperref[TEI.docAuthor]{docAuthor} \hyperref[TEI.docDate]{docDate} \hyperref[TEI.docEdition]{docEdition} \hyperref[TEI.titlePart]{titlePart}\par 
    \item[transcr: ]
   \hyperref[TEI.damage]{damage} \hyperref[TEI.fw]{fw} \hyperref[TEI.metamark]{metamark} \hyperref[TEI.mod]{mod} \hyperref[TEI.restore]{restore} \hyperref[TEI.retrace]{retrace} \hyperref[TEI.secl]{secl} \hyperref[TEI.supplied]{supplied} \hyperref[TEI.surplus]{surplus}
    \item[{Peut contenir}]
  
    \item[analysis: ]
   \hyperref[TEI.c]{c} \hyperref[TEI.cl]{cl} \hyperref[TEI.interp]{interp} \hyperref[TEI.interpGrp]{interpGrp} \hyperref[TEI.m]{m} \hyperref[TEI.pc]{pc} \hyperref[TEI.phr]{phr} \hyperref[TEI.s]{s} \hyperref[TEI.span]{span} \hyperref[TEI.spanGrp]{spanGrp} \hyperref[TEI.w]{w}\par 
    \item[core: ]
   \hyperref[TEI.abbr]{abbr} \hyperref[TEI.add]{add} \hyperref[TEI.address]{address} \hyperref[TEI.binaryObject]{binaryObject} \hyperref[TEI.cb]{cb} \hyperref[TEI.choice]{choice} \hyperref[TEI.corr]{corr} \hyperref[TEI.date]{date} \hyperref[TEI.del]{del} \hyperref[TEI.distinct]{distinct} \hyperref[TEI.email]{email} \hyperref[TEI.emph]{emph} \hyperref[TEI.expan]{expan} \hyperref[TEI.foreign]{foreign} \hyperref[TEI.gap]{gap} \hyperref[TEI.gb]{gb} \hyperref[TEI.gloss]{gloss} \hyperref[TEI.graphic]{graphic} \hyperref[TEI.hi]{hi} \hyperref[TEI.index]{index} \hyperref[TEI.lb]{lb} \hyperref[TEI.measure]{measure} \hyperref[TEI.measureGrp]{measureGrp} \hyperref[TEI.media]{media} \hyperref[TEI.mentioned]{mentioned} \hyperref[TEI.milestone]{milestone} \hyperref[TEI.name]{name} \hyperref[TEI.note]{note} \hyperref[TEI.num]{num} \hyperref[TEI.orig]{orig} \hyperref[TEI.pb]{pb} \hyperref[TEI.ptr]{ptr} \hyperref[TEI.ref]{ref} \hyperref[TEI.reg]{reg} \hyperref[TEI.rs]{rs} \hyperref[TEI.sic]{sic} \hyperref[TEI.soCalled]{soCalled} \hyperref[TEI.term]{term} \hyperref[TEI.time]{time} \hyperref[TEI.title]{title} \hyperref[TEI.unclear]{unclear}\par 
    \item[derived-module-tei.istex: ]
   \hyperref[TEI.math]{math} \hyperref[TEI.mrow]{mrow}\par 
    \item[figures: ]
   \hyperref[TEI.figure]{figure} \hyperref[TEI.formula]{formula} \hyperref[TEI.notatedMusic]{notatedMusic}\par 
    \item[header: ]
   \hyperref[TEI.idno]{idno}\par 
    \item[iso-fs: ]
   \hyperref[TEI.fLib]{fLib} \hyperref[TEI.fs]{fs} \hyperref[TEI.fvLib]{fvLib}\par 
    \item[linking: ]
   \hyperref[TEI.alt]{alt} \hyperref[TEI.altGrp]{altGrp} \hyperref[TEI.anchor]{anchor} \hyperref[TEI.join]{join} \hyperref[TEI.joinGrp]{joinGrp} \hyperref[TEI.link]{link} \hyperref[TEI.linkGrp]{linkGrp} \hyperref[TEI.seg]{seg} \hyperref[TEI.timeline]{timeline}\par 
    \item[msdescription: ]
   \hyperref[TEI.catchwords]{catchwords} \hyperref[TEI.depth]{depth} \hyperref[TEI.dim]{dim} \hyperref[TEI.dimensions]{dimensions} \hyperref[TEI.height]{height} \hyperref[TEI.heraldry]{heraldry} \hyperref[TEI.locus]{locus} \hyperref[TEI.locusGrp]{locusGrp} \hyperref[TEI.material]{material} \hyperref[TEI.objectType]{objectType} \hyperref[TEI.origDate]{origDate} \hyperref[TEI.origPlace]{origPlace} \hyperref[TEI.secFol]{secFol} \hyperref[TEI.signatures]{signatures} \hyperref[TEI.source]{source} \hyperref[TEI.stamp]{stamp} \hyperref[TEI.watermark]{watermark} \hyperref[TEI.width]{width}\par 
    \item[namesdates: ]
   \hyperref[TEI.addName]{addName} \hyperref[TEI.affiliation]{affiliation} \hyperref[TEI.country]{country} \hyperref[TEI.forename]{forename} \hyperref[TEI.genName]{genName} \hyperref[TEI.geogName]{geogName} \hyperref[TEI.location]{location} \hyperref[TEI.nameLink]{nameLink} \hyperref[TEI.orgName]{orgName} \hyperref[TEI.persName]{persName} \hyperref[TEI.placeName]{placeName} \hyperref[TEI.region]{region} \hyperref[TEI.roleName]{roleName} \hyperref[TEI.settlement]{settlement} \hyperref[TEI.state]{state} \hyperref[TEI.surname]{surname}\par 
    \item[spoken: ]
   \hyperref[TEI.annotationBlock]{annotationBlock}\par 
    \item[transcr: ]
   \hyperref[TEI.addSpan]{addSpan} \hyperref[TEI.am]{am} \hyperref[TEI.damage]{damage} \hyperref[TEI.damageSpan]{damageSpan} \hyperref[TEI.delSpan]{delSpan} \hyperref[TEI.ex]{ex} \hyperref[TEI.fw]{fw} \hyperref[TEI.handShift]{handShift} \hyperref[TEI.listTranspose]{listTranspose} \hyperref[TEI.metamark]{metamark} \hyperref[TEI.mod]{mod} \hyperref[TEI.redo]{redo} \hyperref[TEI.restore]{restore} \hyperref[TEI.retrace]{retrace} \hyperref[TEI.secl]{secl} \hyperref[TEI.space]{space} \hyperref[TEI.subst]{subst} \hyperref[TEI.substJoin]{substJoin} \hyperref[TEI.supplied]{supplied} \hyperref[TEI.surplus]{surplus} \hyperref[TEI.undo]{undo}\par des données textuelles
    \item[{Note}]
  \par
La balise \hyperref[TEI.abbr]{<abbr>} n'est pas obligatoire. Si c'est pertinent, l'encodeur peut transcrire les abréviations du texte source sans les commenter ni les baliser. Si les abréviations ne sont pas transcrites directement mais \textit{développées} sans commentaires, alors l'en-tête TEI doit le mentionner.
    \item[{Exemple}]
  \leavevmode\bgroup\exampleFont \begin{shaded}\noindent\mbox{}{<\textbf{choice}>}\mbox{}\newline 
\hspace*{6pt}{<\textbf{expan}>}North Atlantic Treaty Organization{</\textbf{expan}>}\mbox{}\newline 
\hspace*{6pt}{<\textbf{abbr}\hspace*{6pt}{cert}="{low}">}NorATO{</\textbf{abbr}>}\mbox{}\newline 
\hspace*{6pt}{<\textbf{abbr}\hspace*{6pt}{cert}="{high}">}NATO{</\textbf{abbr}>}\mbox{}\newline 
\hspace*{6pt}{<\textbf{abbr}\hspace*{6pt}{cert}="{high}"\hspace*{6pt}{xml:lang}="{fr}">}OTAN{</\textbf{abbr}>}\mbox{}\newline 
{</\textbf{choice}>}\end{shaded}\egroup 


    \item[{Exemple}]
  \leavevmode\bgroup\exampleFont \begin{shaded}\noindent\mbox{}{<\textbf{choice}>}\mbox{}\newline 
\hspace*{6pt}{<\textbf{abbr}>}SPQR{</\textbf{abbr}>}\mbox{}\newline 
\hspace*{6pt}{<\textbf{expan}\hspace*{6pt}{xml:lang}="{la}">}senatus populusque romanorum{</\textbf{expan}>}\mbox{}\newline 
{</\textbf{choice}>}\end{shaded}\egroup 


    \item[{Exemple}]
  \leavevmode\bgroup\exampleFont \begin{shaded}\noindent\mbox{}{<\textbf{choice}>}\mbox{}\newline 
\hspace*{6pt}{<\textbf{abbr}>}SPQR{</\textbf{abbr}>}\mbox{}\newline 
\hspace*{6pt}{<\textbf{expan}>}senatus populusque romanorum{</\textbf{expan}>}\mbox{}\newline 
{</\textbf{choice}>}\end{shaded}\egroup 


    \item[{Modèle de contenu}]
  \mbox{}\hfill\\[-10pt]\begin{Verbatim}[fontsize=\small]
<content>
 <macroRef key="macro.phraseSeq"/>
</content>
    
\end{Verbatim}

    \item[{Schéma Declaration}]
  \mbox{}\hfill\\[-10pt]\begin{Verbatim}[fontsize=\small]
element abbr
{
   tei_att.global.attributes,
   tei_att.typed.attribute.subtype,
   attribute type { text }?,
   tei_macro.phraseSeq}
\end{Verbatim}

\end{reflist}  \index{abstract=<abstract>|oddindex}
\begin{reflist}
\item[]\begin{specHead}{TEI.abstract}{<abstract> }contains a summary or formal abstract prefixed to an existing source document by the encoder. [\xref{http://www.tei-c.org/release/doc/tei-p5-doc/en/html/HD.html\#HD4ABS}{2.4.4. Abstracts}]\end{specHead} 
    \item[{Module}]
  header
    \item[{Attributs}]
  Attributs \hyperref[TEI.att.global]{att.global} (\textit{@xml:id}, \textit{@n}, \textit{@xml:lang}, \textit{@xml:base}, \textit{@xml:space})  (\hyperref[TEI.att.global.rendition]{att.global.rendition} (\textit{@rend}, \textit{@style}, \textit{@rendition})) (\hyperref[TEI.att.global.linking]{att.global.linking} (\textit{@corresp}, \textit{@synch}, \textit{@sameAs}, \textit{@copyOf}, \textit{@next}, \textit{@prev}, \textit{@exclude}, \textit{@select})) (\hyperref[TEI.att.global.analytic]{att.global.analytic} (\textit{@ana})) (\hyperref[TEI.att.global.facs]{att.global.facs} (\textit{@facs})) (\hyperref[TEI.att.global.change]{att.global.change} (\textit{@change})) (\hyperref[TEI.att.global.responsibility]{att.global.responsibility} (\textit{@cert}, \textit{@resp})) (\hyperref[TEI.att.global.source]{att.global.source} (\textit{@source}))
    \item[{Membre du}]
  \hyperref[TEI.model.profileDescPart]{model.profileDescPart}
    \item[{Contenu dans}]
  
    \item[header: ]
   \hyperref[TEI.profileDesc]{profileDesc}
    \item[{Peut contenir}]
  
    \item[core: ]
   \hyperref[TEI.head]{head} \hyperref[TEI.list]{list} \hyperref[TEI.p]{p}\par 
    \item[figures: ]
   \hyperref[TEI.table]{table}\par 
    \item[linking: ]
   \hyperref[TEI.ab]{ab}\par 
    \item[namesdates: ]
   \hyperref[TEI.listOrg]{listOrg} \hyperref[TEI.listPlace]{listPlace}
    \item[{Exemple}]
  Cas où un abstract contient un titre\leavevmode\bgroup\exampleFont \begin{shaded}\noindent\mbox{}{<\textbf{abstract}>}\mbox{}\newline 
\hspace*{6pt}{<\textbf{head}>}Méthode{</\textbf{head}>}\mbox{}\newline 
\hspace*{6pt}{<\textbf{p}>}Contenu de la\mbox{}\newline 
\hspace*{6pt}\hspace*{6pt} méthode.{</\textbf{p}>}\mbox{}\newline 
\hspace*{6pt}{<\textbf{head}>}Expérience{</\textbf{head}>}\mbox{}\newline 
\hspace*{6pt}{<\textbf{p}>}Contenu de\mbox{}\newline 
\hspace*{6pt}\hspace*{6pt} l'expérience.{</\textbf{p}>}\mbox{}\newline 
{</\textbf{abstract}>}\end{shaded}\egroup 


    \item[{Modèle de contenu}]
  \mbox{}\hfill\\[-10pt]\begin{Verbatim}[fontsize=\small]
<content>
 <alternate maxOccurs="unbounded"
  minOccurs="1">
  <classRef key="model.pLike"/>
  <classRef key="model.listLike"/>
  <classRef key="model.headLike"/>
 </alternate>
</content>
    
\end{Verbatim}

    \item[{Schéma Declaration}]
  \mbox{}\hfill\\[-10pt]\begin{Verbatim}[fontsize=\small]
element abstract
{
   tei_att.global.attributes,
   ( tei_model.pLike | tei_model.listLike | tei_model.headLike )+
}
\end{Verbatim}

\end{reflist}  \index{accMat=<accMat>|oddindex}
\begin{reflist}
\item[]\begin{specHead}{TEI.accMat}{<accMat> }(matériel d'accompagnement) donne des détails sur tout matériel d'accompagnement étroitement associé au manuscrit, tel que documents non contemporains ou fragments reliés avec le manuscrit à une époque antérieure. [\xref{http://www.tei-c.org/release/doc/tei-p5-doc/en/html/MS.html\#msadac}{10.7.3.3. Accompanying Material}]\end{specHead} 
    \item[{Module}]
  msdescription
    \item[{Attributs}]
  Attributs \hyperref[TEI.att.global]{att.global} (\textit{@xml:id}, \textit{@n}, \textit{@xml:lang}, \textit{@xml:base}, \textit{@xml:space})  (\hyperref[TEI.att.global.rendition]{att.global.rendition} (\textit{@rend}, \textit{@style}, \textit{@rendition})) (\hyperref[TEI.att.global.linking]{att.global.linking} (\textit{@corresp}, \textit{@synch}, \textit{@sameAs}, \textit{@copyOf}, \textit{@next}, \textit{@prev}, \textit{@exclude}, \textit{@select})) (\hyperref[TEI.att.global.analytic]{att.global.analytic} (\textit{@ana})) (\hyperref[TEI.att.global.facs]{att.global.facs} (\textit{@facs})) (\hyperref[TEI.att.global.change]{att.global.change} (\textit{@change})) (\hyperref[TEI.att.global.responsibility]{att.global.responsibility} (\textit{@cert}, \textit{@resp})) (\hyperref[TEI.att.global.source]{att.global.source} (\textit{@source})) \hyperref[TEI.att.typed]{att.typed} (\textit{@type}, \textit{@subtype}) 
    \item[{Membre du}]
  \hyperref[TEI.model.physDescPart]{model.physDescPart}
    \item[{Contenu dans}]
  
    \item[msdescription: ]
   \hyperref[TEI.physDesc]{physDesc}
    \item[{Peut contenir}]
  
    \item[analysis: ]
   \hyperref[TEI.c]{c} \hyperref[TEI.cl]{cl} \hyperref[TEI.interp]{interp} \hyperref[TEI.interpGrp]{interpGrp} \hyperref[TEI.m]{m} \hyperref[TEI.pc]{pc} \hyperref[TEI.phr]{phr} \hyperref[TEI.s]{s} \hyperref[TEI.span]{span} \hyperref[TEI.spanGrp]{spanGrp} \hyperref[TEI.w]{w}\par 
    \item[core: ]
   \hyperref[TEI.abbr]{abbr} \hyperref[TEI.add]{add} \hyperref[TEI.address]{address} \hyperref[TEI.bibl]{bibl} \hyperref[TEI.biblStruct]{biblStruct} \hyperref[TEI.binaryObject]{binaryObject} \hyperref[TEI.cb]{cb} \hyperref[TEI.choice]{choice} \hyperref[TEI.cit]{cit} \hyperref[TEI.corr]{corr} \hyperref[TEI.date]{date} \hyperref[TEI.del]{del} \hyperref[TEI.desc]{desc} \hyperref[TEI.distinct]{distinct} \hyperref[TEI.email]{email} \hyperref[TEI.emph]{emph} \hyperref[TEI.expan]{expan} \hyperref[TEI.foreign]{foreign} \hyperref[TEI.gap]{gap} \hyperref[TEI.gb]{gb} \hyperref[TEI.gloss]{gloss} \hyperref[TEI.graphic]{graphic} \hyperref[TEI.hi]{hi} \hyperref[TEI.index]{index} \hyperref[TEI.l]{l} \hyperref[TEI.label]{label} \hyperref[TEI.lb]{lb} \hyperref[TEI.lg]{lg} \hyperref[TEI.list]{list} \hyperref[TEI.listBibl]{listBibl} \hyperref[TEI.measure]{measure} \hyperref[TEI.measureGrp]{measureGrp} \hyperref[TEI.media]{media} \hyperref[TEI.mentioned]{mentioned} \hyperref[TEI.milestone]{milestone} \hyperref[TEI.name]{name} \hyperref[TEI.note]{note} \hyperref[TEI.num]{num} \hyperref[TEI.orig]{orig} \hyperref[TEI.p]{p} \hyperref[TEI.pb]{pb} \hyperref[TEI.ptr]{ptr} \hyperref[TEI.q]{q} \hyperref[TEI.quote]{quote} \hyperref[TEI.ref]{ref} \hyperref[TEI.reg]{reg} \hyperref[TEI.rs]{rs} \hyperref[TEI.said]{said} \hyperref[TEI.sic]{sic} \hyperref[TEI.soCalled]{soCalled} \hyperref[TEI.sp]{sp} \hyperref[TEI.stage]{stage} \hyperref[TEI.term]{term} \hyperref[TEI.time]{time} \hyperref[TEI.title]{title} \hyperref[TEI.unclear]{unclear}\par 
    \item[derived-module-tei.istex: ]
   \hyperref[TEI.math]{math} \hyperref[TEI.mrow]{mrow}\par 
    \item[figures: ]
   \hyperref[TEI.figure]{figure} \hyperref[TEI.formula]{formula} \hyperref[TEI.notatedMusic]{notatedMusic} \hyperref[TEI.table]{table}\par 
    \item[header: ]
   \hyperref[TEI.biblFull]{biblFull} \hyperref[TEI.idno]{idno}\par 
    \item[iso-fs: ]
   \hyperref[TEI.fLib]{fLib} \hyperref[TEI.fs]{fs} \hyperref[TEI.fvLib]{fvLib}\par 
    \item[linking: ]
   \hyperref[TEI.ab]{ab} \hyperref[TEI.alt]{alt} \hyperref[TEI.altGrp]{altGrp} \hyperref[TEI.anchor]{anchor} \hyperref[TEI.join]{join} \hyperref[TEI.joinGrp]{joinGrp} \hyperref[TEI.link]{link} \hyperref[TEI.linkGrp]{linkGrp} \hyperref[TEI.seg]{seg} \hyperref[TEI.timeline]{timeline}\par 
    \item[msdescription: ]
   \hyperref[TEI.catchwords]{catchwords} \hyperref[TEI.depth]{depth} \hyperref[TEI.dim]{dim} \hyperref[TEI.dimensions]{dimensions} \hyperref[TEI.height]{height} \hyperref[TEI.heraldry]{heraldry} \hyperref[TEI.locus]{locus} \hyperref[TEI.locusGrp]{locusGrp} \hyperref[TEI.material]{material} \hyperref[TEI.msDesc]{msDesc} \hyperref[TEI.objectType]{objectType} \hyperref[TEI.origDate]{origDate} \hyperref[TEI.origPlace]{origPlace} \hyperref[TEI.secFol]{secFol} \hyperref[TEI.signatures]{signatures} \hyperref[TEI.source]{source} \hyperref[TEI.stamp]{stamp} \hyperref[TEI.watermark]{watermark} \hyperref[TEI.width]{width}\par 
    \item[namesdates: ]
   \hyperref[TEI.addName]{addName} \hyperref[TEI.affiliation]{affiliation} \hyperref[TEI.country]{country} \hyperref[TEI.forename]{forename} \hyperref[TEI.genName]{genName} \hyperref[TEI.geogName]{geogName} \hyperref[TEI.listOrg]{listOrg} \hyperref[TEI.listPlace]{listPlace} \hyperref[TEI.location]{location} \hyperref[TEI.nameLink]{nameLink} \hyperref[TEI.orgName]{orgName} \hyperref[TEI.persName]{persName} \hyperref[TEI.placeName]{placeName} \hyperref[TEI.region]{region} \hyperref[TEI.roleName]{roleName} \hyperref[TEI.settlement]{settlement} \hyperref[TEI.state]{state} \hyperref[TEI.surname]{surname}\par 
    \item[spoken: ]
   \hyperref[TEI.annotationBlock]{annotationBlock}\par 
    \item[textstructure: ]
   \hyperref[TEI.floatingText]{floatingText}\par 
    \item[transcr: ]
   \hyperref[TEI.addSpan]{addSpan} \hyperref[TEI.am]{am} \hyperref[TEI.damage]{damage} \hyperref[TEI.damageSpan]{damageSpan} \hyperref[TEI.delSpan]{delSpan} \hyperref[TEI.ex]{ex} \hyperref[TEI.fw]{fw} \hyperref[TEI.handShift]{handShift} \hyperref[TEI.listTranspose]{listTranspose} \hyperref[TEI.metamark]{metamark} \hyperref[TEI.mod]{mod} \hyperref[TEI.redo]{redo} \hyperref[TEI.restore]{restore} \hyperref[TEI.retrace]{retrace} \hyperref[TEI.secl]{secl} \hyperref[TEI.space]{space} \hyperref[TEI.subst]{subst} \hyperref[TEI.substJoin]{substJoin} \hyperref[TEI.supplied]{supplied} \hyperref[TEI.surplus]{surplus} \hyperref[TEI.undo]{undo}\par des données textuelles
    \item[{Exemple}]
  \leavevmode\bgroup\exampleFont \begin{shaded}\noindent\mbox{}{<\textbf{accMat}>}A copy of a tax form from 1947 is included in the envelope with the letter. It is\mbox{}\newline 
 not catalogued separately.{</\textbf{accMat}>}\end{shaded}\egroup 


    \item[{Modèle de contenu}]
  \mbox{}\hfill\\[-10pt]\begin{Verbatim}[fontsize=\small]
<content>
 <macroRef key="macro.specialPara"/>
</content>
    
\end{Verbatim}

    \item[{Schéma Declaration}]
  \mbox{}\hfill\\[-10pt]\begin{Verbatim}[fontsize=\small]
element accMat
{
   tei_att.global.attributes,
   tei_att.typed.attributes,
   tei_macro.specialPara}
\end{Verbatim}

\end{reflist}  \index{acquisition=<acquisition>|oddindex}
\begin{reflist}
\item[]\begin{specHead}{TEI.acquisition}{<acquisition> }(acquisition) contient des informations sur les modalités et circonstances de l'entrée du manuscrit ou de la partie du manuscrit dans l'institution qui le détient [\xref{http://www.tei-c.org/release/doc/tei-p5-doc/en/html/MS.html\#mshy}{10.8. History}]\end{specHead} 
    \item[{Module}]
  msdescription
    \item[{Attributs}]
  Attributs \hyperref[TEI.att.global]{att.global} (\textit{@xml:id}, \textit{@n}, \textit{@xml:lang}, \textit{@xml:base}, \textit{@xml:space})  (\hyperref[TEI.att.global.rendition]{att.global.rendition} (\textit{@rend}, \textit{@style}, \textit{@rendition})) (\hyperref[TEI.att.global.linking]{att.global.linking} (\textit{@corresp}, \textit{@synch}, \textit{@sameAs}, \textit{@copyOf}, \textit{@next}, \textit{@prev}, \textit{@exclude}, \textit{@select})) (\hyperref[TEI.att.global.analytic]{att.global.analytic} (\textit{@ana})) (\hyperref[TEI.att.global.facs]{att.global.facs} (\textit{@facs})) (\hyperref[TEI.att.global.change]{att.global.change} (\textit{@change})) (\hyperref[TEI.att.global.responsibility]{att.global.responsibility} (\textit{@cert}, \textit{@resp})) (\hyperref[TEI.att.global.source]{att.global.source} (\textit{@source})) \hyperref[TEI.att.datable]{att.datable} (\textit{@calendar}, \textit{@period})  (\hyperref[TEI.att.datable.w3c]{att.datable.w3c} (\textit{@when}, \textit{@notBefore}, \textit{@notAfter}, \textit{@from}, \textit{@to})) (\hyperref[TEI.att.datable.iso]{att.datable.iso} (\textit{@when-iso}, \textit{@notBefore-iso}, \textit{@notAfter-iso}, \textit{@from-iso}, \textit{@to-iso})) (\hyperref[TEI.att.datable.custom]{att.datable.custom} (\textit{@when-custom}, \textit{@notBefore-custom}, \textit{@notAfter-custom}, \textit{@from-custom}, \textit{@to-custom}, \textit{@datingPoint}, \textit{@datingMethod}))
    \item[{Contenu dans}]
  
    \item[msdescription: ]
   \hyperref[TEI.history]{history}
    \item[{Peut contenir}]
  
    \item[analysis: ]
   \hyperref[TEI.c]{c} \hyperref[TEI.cl]{cl} \hyperref[TEI.interp]{interp} \hyperref[TEI.interpGrp]{interpGrp} \hyperref[TEI.m]{m} \hyperref[TEI.pc]{pc} \hyperref[TEI.phr]{phr} \hyperref[TEI.s]{s} \hyperref[TEI.span]{span} \hyperref[TEI.spanGrp]{spanGrp} \hyperref[TEI.w]{w}\par 
    \item[core: ]
   \hyperref[TEI.abbr]{abbr} \hyperref[TEI.add]{add} \hyperref[TEI.address]{address} \hyperref[TEI.bibl]{bibl} \hyperref[TEI.biblStruct]{biblStruct} \hyperref[TEI.binaryObject]{binaryObject} \hyperref[TEI.cb]{cb} \hyperref[TEI.choice]{choice} \hyperref[TEI.cit]{cit} \hyperref[TEI.corr]{corr} \hyperref[TEI.date]{date} \hyperref[TEI.del]{del} \hyperref[TEI.desc]{desc} \hyperref[TEI.distinct]{distinct} \hyperref[TEI.email]{email} \hyperref[TEI.emph]{emph} \hyperref[TEI.expan]{expan} \hyperref[TEI.foreign]{foreign} \hyperref[TEI.gap]{gap} \hyperref[TEI.gb]{gb} \hyperref[TEI.gloss]{gloss} \hyperref[TEI.graphic]{graphic} \hyperref[TEI.hi]{hi} \hyperref[TEI.index]{index} \hyperref[TEI.l]{l} \hyperref[TEI.label]{label} \hyperref[TEI.lb]{lb} \hyperref[TEI.lg]{lg} \hyperref[TEI.list]{list} \hyperref[TEI.listBibl]{listBibl} \hyperref[TEI.measure]{measure} \hyperref[TEI.measureGrp]{measureGrp} \hyperref[TEI.media]{media} \hyperref[TEI.mentioned]{mentioned} \hyperref[TEI.milestone]{milestone} \hyperref[TEI.name]{name} \hyperref[TEI.note]{note} \hyperref[TEI.num]{num} \hyperref[TEI.orig]{orig} \hyperref[TEI.p]{p} \hyperref[TEI.pb]{pb} \hyperref[TEI.ptr]{ptr} \hyperref[TEI.q]{q} \hyperref[TEI.quote]{quote} \hyperref[TEI.ref]{ref} \hyperref[TEI.reg]{reg} \hyperref[TEI.rs]{rs} \hyperref[TEI.said]{said} \hyperref[TEI.sic]{sic} \hyperref[TEI.soCalled]{soCalled} \hyperref[TEI.sp]{sp} \hyperref[TEI.stage]{stage} \hyperref[TEI.term]{term} \hyperref[TEI.time]{time} \hyperref[TEI.title]{title} \hyperref[TEI.unclear]{unclear}\par 
    \item[derived-module-tei.istex: ]
   \hyperref[TEI.math]{math} \hyperref[TEI.mrow]{mrow}\par 
    \item[figures: ]
   \hyperref[TEI.figure]{figure} \hyperref[TEI.formula]{formula} \hyperref[TEI.notatedMusic]{notatedMusic} \hyperref[TEI.table]{table}\par 
    \item[header: ]
   \hyperref[TEI.biblFull]{biblFull} \hyperref[TEI.idno]{idno}\par 
    \item[iso-fs: ]
   \hyperref[TEI.fLib]{fLib} \hyperref[TEI.fs]{fs} \hyperref[TEI.fvLib]{fvLib}\par 
    \item[linking: ]
   \hyperref[TEI.ab]{ab} \hyperref[TEI.alt]{alt} \hyperref[TEI.altGrp]{altGrp} \hyperref[TEI.anchor]{anchor} \hyperref[TEI.join]{join} \hyperref[TEI.joinGrp]{joinGrp} \hyperref[TEI.link]{link} \hyperref[TEI.linkGrp]{linkGrp} \hyperref[TEI.seg]{seg} \hyperref[TEI.timeline]{timeline}\par 
    \item[msdescription: ]
   \hyperref[TEI.catchwords]{catchwords} \hyperref[TEI.depth]{depth} \hyperref[TEI.dim]{dim} \hyperref[TEI.dimensions]{dimensions} \hyperref[TEI.height]{height} \hyperref[TEI.heraldry]{heraldry} \hyperref[TEI.locus]{locus} \hyperref[TEI.locusGrp]{locusGrp} \hyperref[TEI.material]{material} \hyperref[TEI.msDesc]{msDesc} \hyperref[TEI.objectType]{objectType} \hyperref[TEI.origDate]{origDate} \hyperref[TEI.origPlace]{origPlace} \hyperref[TEI.secFol]{secFol} \hyperref[TEI.signatures]{signatures} \hyperref[TEI.source]{source} \hyperref[TEI.stamp]{stamp} \hyperref[TEI.watermark]{watermark} \hyperref[TEI.width]{width}\par 
    \item[namesdates: ]
   \hyperref[TEI.addName]{addName} \hyperref[TEI.affiliation]{affiliation} \hyperref[TEI.country]{country} \hyperref[TEI.forename]{forename} \hyperref[TEI.genName]{genName} \hyperref[TEI.geogName]{geogName} \hyperref[TEI.listOrg]{listOrg} \hyperref[TEI.listPlace]{listPlace} \hyperref[TEI.location]{location} \hyperref[TEI.nameLink]{nameLink} \hyperref[TEI.orgName]{orgName} \hyperref[TEI.persName]{persName} \hyperref[TEI.placeName]{placeName} \hyperref[TEI.region]{region} \hyperref[TEI.roleName]{roleName} \hyperref[TEI.settlement]{settlement} \hyperref[TEI.state]{state} \hyperref[TEI.surname]{surname}\par 
    \item[spoken: ]
   \hyperref[TEI.annotationBlock]{annotationBlock}\par 
    \item[textstructure: ]
   \hyperref[TEI.floatingText]{floatingText}\par 
    \item[transcr: ]
   \hyperref[TEI.addSpan]{addSpan} \hyperref[TEI.am]{am} \hyperref[TEI.damage]{damage} \hyperref[TEI.damageSpan]{damageSpan} \hyperref[TEI.delSpan]{delSpan} \hyperref[TEI.ex]{ex} \hyperref[TEI.fw]{fw} \hyperref[TEI.handShift]{handShift} \hyperref[TEI.listTranspose]{listTranspose} \hyperref[TEI.metamark]{metamark} \hyperref[TEI.mod]{mod} \hyperref[TEI.redo]{redo} \hyperref[TEI.restore]{restore} \hyperref[TEI.retrace]{retrace} \hyperref[TEI.secl]{secl} \hyperref[TEI.space]{space} \hyperref[TEI.subst]{subst} \hyperref[TEI.substJoin]{substJoin} \hyperref[TEI.supplied]{supplied} \hyperref[TEI.surplus]{surplus} \hyperref[TEI.undo]{undo}\par des données textuelles
    \item[{Exemple}]
  \leavevmode\bgroup\exampleFont \begin{shaded}\noindent\mbox{}{<\textbf{acquisition}>}Left to the {<\textbf{name}\hspace*{6pt}{type}="{place}">}Bodleian{</\textbf{name}>} by{<\textbf{name}\hspace*{6pt}{type}="{person}">}Richard\mbox{}\newline 
\hspace*{6pt}\hspace*{6pt} Rawlinson{</\textbf{name}>} in 1755.{</\textbf{acquisition}>}\end{shaded}\egroup 


    \item[{Modèle de contenu}]
  \mbox{}\hfill\\[-10pt]\begin{Verbatim}[fontsize=\small]
<content>
 <macroRef key="macro.specialPara"/>
</content>
    
\end{Verbatim}

    \item[{Schéma Declaration}]
  \mbox{}\hfill\\[-10pt]\begin{Verbatim}[fontsize=\small]
element acquisition
{
   tei_att.global.attributes,
   tei_att.datable.attributes,
   tei_macro.specialPara}
\end{Verbatim}

\end{reflist}  \index{add=<add>|oddindex}
\begin{reflist}
\item[]\begin{specHead}{TEI.add}{<add> }(ajout) contient des lettres, des mots ou des phrases insérés dans le texte par un auteur, un copiste, un annotateur ou un correcteur. [\xref{http://www.tei-c.org/release/doc/tei-p5-doc/en/html/CO.html\#COEDADD}{3.4.3. Additions, Deletions, and Omissions}]\end{specHead} 
    \item[{Module}]
  core
    \item[{Attributs}]
  Attributs \hyperref[TEI.att.global]{att.global} (\textit{@xml:id}, \textit{@n}, \textit{@xml:lang}, \textit{@xml:base}, \textit{@xml:space})  (\hyperref[TEI.att.global.rendition]{att.global.rendition} (\textit{@rend}, \textit{@style}, \textit{@rendition})) (\hyperref[TEI.att.global.linking]{att.global.linking} (\textit{@corresp}, \textit{@synch}, \textit{@sameAs}, \textit{@copyOf}, \textit{@next}, \textit{@prev}, \textit{@exclude}, \textit{@select})) (\hyperref[TEI.att.global.analytic]{att.global.analytic} (\textit{@ana})) (\hyperref[TEI.att.global.facs]{att.global.facs} (\textit{@facs})) (\hyperref[TEI.att.global.change]{att.global.change} (\textit{@change})) (\hyperref[TEI.att.global.responsibility]{att.global.responsibility} (\textit{@cert}, \textit{@resp})) (\hyperref[TEI.att.global.source]{att.global.source} (\textit{@source})) \hyperref[TEI.att.transcriptional]{att.transcriptional} (\textit{@status}, \textit{@cause}, \textit{@seq})  (\hyperref[TEI.att.editLike]{att.editLike} (\textit{@evidence}, \textit{@instant}) (\hyperref[TEI.att.dimensions]{att.dimensions} (\textit{@unit}, \textit{@quantity}, \textit{@extent}, \textit{@precision}, \textit{@scope}) (\hyperref[TEI.att.ranging]{att.ranging} (\textit{@atLeast}, \textit{@atMost}, \textit{@min}, \textit{@max}, \textit{@confidence})) ) ) (\hyperref[TEI.att.written]{att.written} (\textit{@hand})) \hyperref[TEI.att.placement]{att.placement} (\textit{@place}) \hyperref[TEI.att.typed]{att.typed} (\textit{@type}, \textit{@subtype}) 
    \item[{Membre du}]
  \hyperref[TEI.model.linePart]{model.linePart} \hyperref[TEI.model.pPart.transcriptional]{model.pPart.transcriptional} 
    \item[{Contenu dans}]
  
    \item[analysis: ]
   \hyperref[TEI.cl]{cl} \hyperref[TEI.pc]{pc} \hyperref[TEI.phr]{phr} \hyperref[TEI.s]{s} \hyperref[TEI.w]{w}\par 
    \item[core: ]
   \hyperref[TEI.abbr]{abbr} \hyperref[TEI.add]{add} \hyperref[TEI.addrLine]{addrLine} \hyperref[TEI.author]{author} \hyperref[TEI.bibl]{bibl} \hyperref[TEI.biblScope]{biblScope} \hyperref[TEI.citedRange]{citedRange} \hyperref[TEI.corr]{corr} \hyperref[TEI.date]{date} \hyperref[TEI.del]{del} \hyperref[TEI.distinct]{distinct} \hyperref[TEI.editor]{editor} \hyperref[TEI.email]{email} \hyperref[TEI.emph]{emph} \hyperref[TEI.expan]{expan} \hyperref[TEI.foreign]{foreign} \hyperref[TEI.gloss]{gloss} \hyperref[TEI.head]{head} \hyperref[TEI.headItem]{headItem} \hyperref[TEI.headLabel]{headLabel} \hyperref[TEI.hi]{hi} \hyperref[TEI.item]{item} \hyperref[TEI.l]{l} \hyperref[TEI.label]{label} \hyperref[TEI.measure]{measure} \hyperref[TEI.mentioned]{mentioned} \hyperref[TEI.name]{name} \hyperref[TEI.note]{note} \hyperref[TEI.num]{num} \hyperref[TEI.orig]{orig} \hyperref[TEI.p]{p} \hyperref[TEI.pubPlace]{pubPlace} \hyperref[TEI.publisher]{publisher} \hyperref[TEI.q]{q} \hyperref[TEI.quote]{quote} \hyperref[TEI.ref]{ref} \hyperref[TEI.reg]{reg} \hyperref[TEI.rs]{rs} \hyperref[TEI.said]{said} \hyperref[TEI.sic]{sic} \hyperref[TEI.soCalled]{soCalled} \hyperref[TEI.speaker]{speaker} \hyperref[TEI.stage]{stage} \hyperref[TEI.street]{street} \hyperref[TEI.term]{term} \hyperref[TEI.textLang]{textLang} \hyperref[TEI.time]{time} \hyperref[TEI.title]{title} \hyperref[TEI.unclear]{unclear}\par 
    \item[figures: ]
   \hyperref[TEI.cell]{cell}\par 
    \item[header: ]
   \hyperref[TEI.change]{change} \hyperref[TEI.distributor]{distributor} \hyperref[TEI.edition]{edition} \hyperref[TEI.extent]{extent} \hyperref[TEI.licence]{licence}\par 
    \item[linking: ]
   \hyperref[TEI.ab]{ab} \hyperref[TEI.seg]{seg}\par 
    \item[msdescription: ]
   \hyperref[TEI.accMat]{accMat} \hyperref[TEI.acquisition]{acquisition} \hyperref[TEI.additions]{additions} \hyperref[TEI.catchwords]{catchwords} \hyperref[TEI.collation]{collation} \hyperref[TEI.colophon]{colophon} \hyperref[TEI.condition]{condition} \hyperref[TEI.custEvent]{custEvent} \hyperref[TEI.decoNote]{decoNote} \hyperref[TEI.explicit]{explicit} \hyperref[TEI.filiation]{filiation} \hyperref[TEI.finalRubric]{finalRubric} \hyperref[TEI.foliation]{foliation} \hyperref[TEI.heraldry]{heraldry} \hyperref[TEI.incipit]{incipit} \hyperref[TEI.layout]{layout} \hyperref[TEI.material]{material} \hyperref[TEI.musicNotation]{musicNotation} \hyperref[TEI.objectType]{objectType} \hyperref[TEI.origDate]{origDate} \hyperref[TEI.origPlace]{origPlace} \hyperref[TEI.origin]{origin} \hyperref[TEI.provenance]{provenance} \hyperref[TEI.rubric]{rubric} \hyperref[TEI.secFol]{secFol} \hyperref[TEI.signatures]{signatures} \hyperref[TEI.source]{source} \hyperref[TEI.stamp]{stamp} \hyperref[TEI.summary]{summary} \hyperref[TEI.support]{support} \hyperref[TEI.surrogates]{surrogates} \hyperref[TEI.typeNote]{typeNote} \hyperref[TEI.watermark]{watermark}\par 
    \item[namesdates: ]
   \hyperref[TEI.addName]{addName} \hyperref[TEI.affiliation]{affiliation} \hyperref[TEI.country]{country} \hyperref[TEI.forename]{forename} \hyperref[TEI.genName]{genName} \hyperref[TEI.geogName]{geogName} \hyperref[TEI.nameLink]{nameLink} \hyperref[TEI.orgName]{orgName} \hyperref[TEI.persName]{persName} \hyperref[TEI.placeName]{placeName} \hyperref[TEI.region]{region} \hyperref[TEI.roleName]{roleName} \hyperref[TEI.settlement]{settlement} \hyperref[TEI.surname]{surname}\par 
    \item[textstructure: ]
   \hyperref[TEI.docAuthor]{docAuthor} \hyperref[TEI.docDate]{docDate} \hyperref[TEI.docEdition]{docEdition} \hyperref[TEI.titlePart]{titlePart}\par 
    \item[transcr: ]
   \hyperref[TEI.am]{am} \hyperref[TEI.damage]{damage} \hyperref[TEI.fw]{fw} \hyperref[TEI.line]{line} \hyperref[TEI.metamark]{metamark} \hyperref[TEI.mod]{mod} \hyperref[TEI.restore]{restore} \hyperref[TEI.retrace]{retrace} \hyperref[TEI.secl]{secl} \hyperref[TEI.subst]{subst} \hyperref[TEI.supplied]{supplied} \hyperref[TEI.surplus]{surplus} \hyperref[TEI.zone]{zone}
    \item[{Peut contenir}]
  
    \item[analysis: ]
   \hyperref[TEI.c]{c} \hyperref[TEI.cl]{cl} \hyperref[TEI.interp]{interp} \hyperref[TEI.interpGrp]{interpGrp} \hyperref[TEI.m]{m} \hyperref[TEI.pc]{pc} \hyperref[TEI.phr]{phr} \hyperref[TEI.s]{s} \hyperref[TEI.span]{span} \hyperref[TEI.spanGrp]{spanGrp} \hyperref[TEI.w]{w}\par 
    \item[core: ]
   \hyperref[TEI.abbr]{abbr} \hyperref[TEI.add]{add} \hyperref[TEI.address]{address} \hyperref[TEI.bibl]{bibl} \hyperref[TEI.biblStruct]{biblStruct} \hyperref[TEI.binaryObject]{binaryObject} \hyperref[TEI.cb]{cb} \hyperref[TEI.choice]{choice} \hyperref[TEI.cit]{cit} \hyperref[TEI.corr]{corr} \hyperref[TEI.date]{date} \hyperref[TEI.del]{del} \hyperref[TEI.desc]{desc} \hyperref[TEI.distinct]{distinct} \hyperref[TEI.email]{email} \hyperref[TEI.emph]{emph} \hyperref[TEI.expan]{expan} \hyperref[TEI.foreign]{foreign} \hyperref[TEI.gap]{gap} \hyperref[TEI.gb]{gb} \hyperref[TEI.gloss]{gloss} \hyperref[TEI.graphic]{graphic} \hyperref[TEI.hi]{hi} \hyperref[TEI.index]{index} \hyperref[TEI.l]{l} \hyperref[TEI.label]{label} \hyperref[TEI.lb]{lb} \hyperref[TEI.lg]{lg} \hyperref[TEI.list]{list} \hyperref[TEI.listBibl]{listBibl} \hyperref[TEI.measure]{measure} \hyperref[TEI.measureGrp]{measureGrp} \hyperref[TEI.media]{media} \hyperref[TEI.mentioned]{mentioned} \hyperref[TEI.milestone]{milestone} \hyperref[TEI.name]{name} \hyperref[TEI.note]{note} \hyperref[TEI.num]{num} \hyperref[TEI.orig]{orig} \hyperref[TEI.pb]{pb} \hyperref[TEI.ptr]{ptr} \hyperref[TEI.q]{q} \hyperref[TEI.quote]{quote} \hyperref[TEI.ref]{ref} \hyperref[TEI.reg]{reg} \hyperref[TEI.rs]{rs} \hyperref[TEI.said]{said} \hyperref[TEI.sic]{sic} \hyperref[TEI.soCalled]{soCalled} \hyperref[TEI.stage]{stage} \hyperref[TEI.term]{term} \hyperref[TEI.time]{time} \hyperref[TEI.title]{title} \hyperref[TEI.unclear]{unclear}\par 
    \item[derived-module-tei.istex: ]
   \hyperref[TEI.math]{math} \hyperref[TEI.mrow]{mrow}\par 
    \item[figures: ]
   \hyperref[TEI.figure]{figure} \hyperref[TEI.formula]{formula} \hyperref[TEI.notatedMusic]{notatedMusic} \hyperref[TEI.table]{table}\par 
    \item[header: ]
   \hyperref[TEI.biblFull]{biblFull} \hyperref[TEI.idno]{idno}\par 
    \item[iso-fs: ]
   \hyperref[TEI.fLib]{fLib} \hyperref[TEI.fs]{fs} \hyperref[TEI.fvLib]{fvLib}\par 
    \item[linking: ]
   \hyperref[TEI.alt]{alt} \hyperref[TEI.altGrp]{altGrp} \hyperref[TEI.anchor]{anchor} \hyperref[TEI.join]{join} \hyperref[TEI.joinGrp]{joinGrp} \hyperref[TEI.link]{link} \hyperref[TEI.linkGrp]{linkGrp} \hyperref[TEI.seg]{seg} \hyperref[TEI.timeline]{timeline}\par 
    \item[msdescription: ]
   \hyperref[TEI.catchwords]{catchwords} \hyperref[TEI.depth]{depth} \hyperref[TEI.dim]{dim} \hyperref[TEI.dimensions]{dimensions} \hyperref[TEI.height]{height} \hyperref[TEI.heraldry]{heraldry} \hyperref[TEI.locus]{locus} \hyperref[TEI.locusGrp]{locusGrp} \hyperref[TEI.material]{material} \hyperref[TEI.msDesc]{msDesc} \hyperref[TEI.objectType]{objectType} \hyperref[TEI.origDate]{origDate} \hyperref[TEI.origPlace]{origPlace} \hyperref[TEI.secFol]{secFol} \hyperref[TEI.signatures]{signatures} \hyperref[TEI.source]{source} \hyperref[TEI.stamp]{stamp} \hyperref[TEI.watermark]{watermark} \hyperref[TEI.width]{width}\par 
    \item[namesdates: ]
   \hyperref[TEI.addName]{addName} \hyperref[TEI.affiliation]{affiliation} \hyperref[TEI.country]{country} \hyperref[TEI.forename]{forename} \hyperref[TEI.genName]{genName} \hyperref[TEI.geogName]{geogName} \hyperref[TEI.listOrg]{listOrg} \hyperref[TEI.listPlace]{listPlace} \hyperref[TEI.location]{location} \hyperref[TEI.nameLink]{nameLink} \hyperref[TEI.orgName]{orgName} \hyperref[TEI.persName]{persName} \hyperref[TEI.placeName]{placeName} \hyperref[TEI.region]{region} \hyperref[TEI.roleName]{roleName} \hyperref[TEI.settlement]{settlement} \hyperref[TEI.state]{state} \hyperref[TEI.surname]{surname}\par 
    \item[spoken: ]
   \hyperref[TEI.annotationBlock]{annotationBlock}\par 
    \item[textstructure: ]
   \hyperref[TEI.floatingText]{floatingText}\par 
    \item[transcr: ]
   \hyperref[TEI.addSpan]{addSpan} \hyperref[TEI.am]{am} \hyperref[TEI.damage]{damage} \hyperref[TEI.damageSpan]{damageSpan} \hyperref[TEI.delSpan]{delSpan} \hyperref[TEI.ex]{ex} \hyperref[TEI.fw]{fw} \hyperref[TEI.handShift]{handShift} \hyperref[TEI.listTranspose]{listTranspose} \hyperref[TEI.metamark]{metamark} \hyperref[TEI.mod]{mod} \hyperref[TEI.redo]{redo} \hyperref[TEI.restore]{restore} \hyperref[TEI.retrace]{retrace} \hyperref[TEI.secl]{secl} \hyperref[TEI.space]{space} \hyperref[TEI.subst]{subst} \hyperref[TEI.substJoin]{substJoin} \hyperref[TEI.supplied]{supplied} \hyperref[TEI.surplus]{surplus} \hyperref[TEI.undo]{undo}\par des données textuelles
    \item[{Note}]
  \par
Dans une édition diplomatique souhaitant representer une source originalle, l'élément \hyperref[TEI.add]{<add>} ne sera pas utilisé pour les ajouts effectués par les éditeurs ou les encodeurs. Dans ce cas, on va préféra soit l'élément \hyperref[TEI.corr]{<corr>} soit l'élément \hyperref[TEI.supplied]{<supplied>}.
    \item[{Exemple}]
  \leavevmode\bgroup\exampleFont \begin{shaded}\noindent\mbox{}The story I am\mbox{}\newline 
 going to relate is true as to its main facts, and as to the\mbox{}\newline 
 consequences {<\textbf{add}\hspace*{6pt}{place}="{above}">}of these facts{</\textbf{add}>} from which\mbox{}\newline 
 this tale takes its title.\end{shaded}\egroup 


    \item[{Modèle de contenu}]
  \mbox{}\hfill\\[-10pt]\begin{Verbatim}[fontsize=\small]
<content>
 <macroRef key="macro.paraContent"/>
</content>
    
\end{Verbatim}

    \item[{Schéma Declaration}]
  \mbox{}\hfill\\[-10pt]\begin{Verbatim}[fontsize=\small]
element add
{
   tei_att.global.attributes,
   tei_att.transcriptional.attributes,
   tei_att.placement.attributes,
   tei_att.typed.attributes,
   tei_macro.paraContent}
\end{Verbatim}

\end{reflist}  \index{addName=<addName>|oddindex}
\begin{reflist}
\item[]\begin{specHead}{TEI.addName}{<addName> }(nom additionnel) contient une composante de nom additionnelle, comme un surnom, une épithète, un alias ou toute autre expression descriptive utilisée dans un nom de personne. [\xref{http://www.tei-c.org/release/doc/tei-p5-doc/en/html/ND.html\#NDPER}{13.2.1. Personal Names}]\end{specHead} 
    \item[{Module}]
  namesdates
    \item[{Attributs}]
  Attributs \hyperref[TEI.att.global]{att.global} (\textit{@xml:id}, \textit{@n}, \textit{@xml:lang}, \textit{@xml:base}, \textit{@xml:space})  (\hyperref[TEI.att.global.rendition]{att.global.rendition} (\textit{@rend}, \textit{@style}, \textit{@rendition})) (\hyperref[TEI.att.global.linking]{att.global.linking} (\textit{@corresp}, \textit{@synch}, \textit{@sameAs}, \textit{@copyOf}, \textit{@next}, \textit{@prev}, \textit{@exclude}, \textit{@select})) (\hyperref[TEI.att.global.analytic]{att.global.analytic} (\textit{@ana})) (\hyperref[TEI.att.global.facs]{att.global.facs} (\textit{@facs})) (\hyperref[TEI.att.global.change]{att.global.change} (\textit{@change})) (\hyperref[TEI.att.global.responsibility]{att.global.responsibility} (\textit{@cert}, \textit{@resp})) (\hyperref[TEI.att.global.source]{att.global.source} (\textit{@source})) \hyperref[TEI.att.personal]{att.personal} (\textit{@full}, \textit{@sort})  (\hyperref[TEI.att.naming]{att.naming} (\textit{@role}, \textit{@nymRef}) (\hyperref[TEI.att.canonical]{att.canonical} (\textit{@key}, \textit{@ref})) ) \hyperref[TEI.att.typed]{att.typed} (\textit{@type}, \textit{@subtype}) 
    \item[{Membre du}]
  \hyperref[TEI.model.persNamePart]{model.persNamePart}
    \item[{Contenu dans}]
  
    \item[analysis: ]
   \hyperref[TEI.cl]{cl} \hyperref[TEI.phr]{phr} \hyperref[TEI.s]{s} \hyperref[TEI.span]{span}\par 
    \item[core: ]
   \hyperref[TEI.abbr]{abbr} \hyperref[TEI.add]{add} \hyperref[TEI.addrLine]{addrLine} \hyperref[TEI.address]{address} \hyperref[TEI.author]{author} \hyperref[TEI.bibl]{bibl} \hyperref[TEI.biblScope]{biblScope} \hyperref[TEI.citedRange]{citedRange} \hyperref[TEI.corr]{corr} \hyperref[TEI.date]{date} \hyperref[TEI.del]{del} \hyperref[TEI.desc]{desc} \hyperref[TEI.distinct]{distinct} \hyperref[TEI.editor]{editor} \hyperref[TEI.email]{email} \hyperref[TEI.emph]{emph} \hyperref[TEI.expan]{expan} \hyperref[TEI.foreign]{foreign} \hyperref[TEI.gloss]{gloss} \hyperref[TEI.head]{head} \hyperref[TEI.headItem]{headItem} \hyperref[TEI.headLabel]{headLabel} \hyperref[TEI.hi]{hi} \hyperref[TEI.item]{item} \hyperref[TEI.l]{l} \hyperref[TEI.label]{label} \hyperref[TEI.measure]{measure} \hyperref[TEI.meeting]{meeting} \hyperref[TEI.mentioned]{mentioned} \hyperref[TEI.name]{name} \hyperref[TEI.note]{note} \hyperref[TEI.num]{num} \hyperref[TEI.orig]{orig} \hyperref[TEI.p]{p} \hyperref[TEI.pubPlace]{pubPlace} \hyperref[TEI.publisher]{publisher} \hyperref[TEI.q]{q} \hyperref[TEI.quote]{quote} \hyperref[TEI.ref]{ref} \hyperref[TEI.reg]{reg} \hyperref[TEI.resp]{resp} \hyperref[TEI.rs]{rs} \hyperref[TEI.said]{said} \hyperref[TEI.sic]{sic} \hyperref[TEI.soCalled]{soCalled} \hyperref[TEI.speaker]{speaker} \hyperref[TEI.stage]{stage} \hyperref[TEI.street]{street} \hyperref[TEI.term]{term} \hyperref[TEI.textLang]{textLang} \hyperref[TEI.time]{time} \hyperref[TEI.title]{title} \hyperref[TEI.unclear]{unclear}\par 
    \item[figures: ]
   \hyperref[TEI.cell]{cell} \hyperref[TEI.figDesc]{figDesc}\par 
    \item[header: ]
   \hyperref[TEI.authority]{authority} \hyperref[TEI.change]{change} \hyperref[TEI.classCode]{classCode} \hyperref[TEI.creation]{creation} \hyperref[TEI.distributor]{distributor} \hyperref[TEI.edition]{edition} \hyperref[TEI.extent]{extent} \hyperref[TEI.funder]{funder} \hyperref[TEI.language]{language} \hyperref[TEI.licence]{licence} \hyperref[TEI.rendition]{rendition}\par 
    \item[iso-fs: ]
   \hyperref[TEI.fDescr]{fDescr} \hyperref[TEI.fsDescr]{fsDescr}\par 
    \item[linking: ]
   \hyperref[TEI.ab]{ab} \hyperref[TEI.seg]{seg}\par 
    \item[msdescription: ]
   \hyperref[TEI.accMat]{accMat} \hyperref[TEI.acquisition]{acquisition} \hyperref[TEI.additions]{additions} \hyperref[TEI.catchwords]{catchwords} \hyperref[TEI.collation]{collation} \hyperref[TEI.colophon]{colophon} \hyperref[TEI.condition]{condition} \hyperref[TEI.custEvent]{custEvent} \hyperref[TEI.decoNote]{decoNote} \hyperref[TEI.explicit]{explicit} \hyperref[TEI.filiation]{filiation} \hyperref[TEI.finalRubric]{finalRubric} \hyperref[TEI.foliation]{foliation} \hyperref[TEI.heraldry]{heraldry} \hyperref[TEI.incipit]{incipit} \hyperref[TEI.layout]{layout} \hyperref[TEI.material]{material} \hyperref[TEI.musicNotation]{musicNotation} \hyperref[TEI.objectType]{objectType} \hyperref[TEI.origDate]{origDate} \hyperref[TEI.origPlace]{origPlace} \hyperref[TEI.origin]{origin} \hyperref[TEI.provenance]{provenance} \hyperref[TEI.rubric]{rubric} \hyperref[TEI.secFol]{secFol} \hyperref[TEI.signatures]{signatures} \hyperref[TEI.source]{source} \hyperref[TEI.stamp]{stamp} \hyperref[TEI.summary]{summary} \hyperref[TEI.support]{support} \hyperref[TEI.surrogates]{surrogates} \hyperref[TEI.typeNote]{typeNote} \hyperref[TEI.watermark]{watermark}\par 
    \item[namesdates: ]
   \hyperref[TEI.addName]{addName} \hyperref[TEI.affiliation]{affiliation} \hyperref[TEI.country]{country} \hyperref[TEI.forename]{forename} \hyperref[TEI.genName]{genName} \hyperref[TEI.geogName]{geogName} \hyperref[TEI.nameLink]{nameLink} \hyperref[TEI.org]{org} \hyperref[TEI.orgName]{orgName} \hyperref[TEI.persName]{persName} \hyperref[TEI.placeName]{placeName} \hyperref[TEI.region]{region} \hyperref[TEI.roleName]{roleName} \hyperref[TEI.settlement]{settlement} \hyperref[TEI.surname]{surname}\par 
    \item[spoken: ]
   \hyperref[TEI.annotationBlock]{annotationBlock}\par 
    \item[standOff: ]
   \hyperref[TEI.listAnnotation]{listAnnotation}\par 
    \item[textstructure: ]
   \hyperref[TEI.docAuthor]{docAuthor} \hyperref[TEI.docDate]{docDate} \hyperref[TEI.docEdition]{docEdition} \hyperref[TEI.titlePart]{titlePart}\par 
    \item[transcr: ]
   \hyperref[TEI.damage]{damage} \hyperref[TEI.fw]{fw} \hyperref[TEI.metamark]{metamark} \hyperref[TEI.mod]{mod} \hyperref[TEI.restore]{restore} \hyperref[TEI.retrace]{retrace} \hyperref[TEI.secl]{secl} \hyperref[TEI.supplied]{supplied} \hyperref[TEI.surplus]{surplus}
    \item[{Peut contenir}]
  
    \item[analysis: ]
   \hyperref[TEI.c]{c} \hyperref[TEI.cl]{cl} \hyperref[TEI.interp]{interp} \hyperref[TEI.interpGrp]{interpGrp} \hyperref[TEI.m]{m} \hyperref[TEI.pc]{pc} \hyperref[TEI.phr]{phr} \hyperref[TEI.s]{s} \hyperref[TEI.span]{span} \hyperref[TEI.spanGrp]{spanGrp} \hyperref[TEI.w]{w}\par 
    \item[core: ]
   \hyperref[TEI.abbr]{abbr} \hyperref[TEI.add]{add} \hyperref[TEI.address]{address} \hyperref[TEI.binaryObject]{binaryObject} \hyperref[TEI.cb]{cb} \hyperref[TEI.choice]{choice} \hyperref[TEI.corr]{corr} \hyperref[TEI.date]{date} \hyperref[TEI.del]{del} \hyperref[TEI.distinct]{distinct} \hyperref[TEI.email]{email} \hyperref[TEI.emph]{emph} \hyperref[TEI.expan]{expan} \hyperref[TEI.foreign]{foreign} \hyperref[TEI.gap]{gap} \hyperref[TEI.gb]{gb} \hyperref[TEI.gloss]{gloss} \hyperref[TEI.graphic]{graphic} \hyperref[TEI.hi]{hi} \hyperref[TEI.index]{index} \hyperref[TEI.lb]{lb} \hyperref[TEI.measure]{measure} \hyperref[TEI.measureGrp]{measureGrp} \hyperref[TEI.media]{media} \hyperref[TEI.mentioned]{mentioned} \hyperref[TEI.milestone]{milestone} \hyperref[TEI.name]{name} \hyperref[TEI.note]{note} \hyperref[TEI.num]{num} \hyperref[TEI.orig]{orig} \hyperref[TEI.pb]{pb} \hyperref[TEI.ptr]{ptr} \hyperref[TEI.ref]{ref} \hyperref[TEI.reg]{reg} \hyperref[TEI.rs]{rs} \hyperref[TEI.sic]{sic} \hyperref[TEI.soCalled]{soCalled} \hyperref[TEI.term]{term} \hyperref[TEI.time]{time} \hyperref[TEI.title]{title} \hyperref[TEI.unclear]{unclear}\par 
    \item[derived-module-tei.istex: ]
   \hyperref[TEI.math]{math} \hyperref[TEI.mrow]{mrow}\par 
    \item[figures: ]
   \hyperref[TEI.figure]{figure} \hyperref[TEI.formula]{formula} \hyperref[TEI.notatedMusic]{notatedMusic}\par 
    \item[header: ]
   \hyperref[TEI.idno]{idno}\par 
    \item[iso-fs: ]
   \hyperref[TEI.fLib]{fLib} \hyperref[TEI.fs]{fs} \hyperref[TEI.fvLib]{fvLib}\par 
    \item[linking: ]
   \hyperref[TEI.alt]{alt} \hyperref[TEI.altGrp]{altGrp} \hyperref[TEI.anchor]{anchor} \hyperref[TEI.join]{join} \hyperref[TEI.joinGrp]{joinGrp} \hyperref[TEI.link]{link} \hyperref[TEI.linkGrp]{linkGrp} \hyperref[TEI.seg]{seg} \hyperref[TEI.timeline]{timeline}\par 
    \item[msdescription: ]
   \hyperref[TEI.catchwords]{catchwords} \hyperref[TEI.depth]{depth} \hyperref[TEI.dim]{dim} \hyperref[TEI.dimensions]{dimensions} \hyperref[TEI.height]{height} \hyperref[TEI.heraldry]{heraldry} \hyperref[TEI.locus]{locus} \hyperref[TEI.locusGrp]{locusGrp} \hyperref[TEI.material]{material} \hyperref[TEI.objectType]{objectType} \hyperref[TEI.origDate]{origDate} \hyperref[TEI.origPlace]{origPlace} \hyperref[TEI.secFol]{secFol} \hyperref[TEI.signatures]{signatures} \hyperref[TEI.source]{source} \hyperref[TEI.stamp]{stamp} \hyperref[TEI.watermark]{watermark} \hyperref[TEI.width]{width}\par 
    \item[namesdates: ]
   \hyperref[TEI.addName]{addName} \hyperref[TEI.affiliation]{affiliation} \hyperref[TEI.country]{country} \hyperref[TEI.forename]{forename} \hyperref[TEI.genName]{genName} \hyperref[TEI.geogName]{geogName} \hyperref[TEI.location]{location} \hyperref[TEI.nameLink]{nameLink} \hyperref[TEI.orgName]{orgName} \hyperref[TEI.persName]{persName} \hyperref[TEI.placeName]{placeName} \hyperref[TEI.region]{region} \hyperref[TEI.roleName]{roleName} \hyperref[TEI.settlement]{settlement} \hyperref[TEI.state]{state} \hyperref[TEI.surname]{surname}\par 
    \item[spoken: ]
   \hyperref[TEI.annotationBlock]{annotationBlock}\par 
    \item[transcr: ]
   \hyperref[TEI.addSpan]{addSpan} \hyperref[TEI.am]{am} \hyperref[TEI.damage]{damage} \hyperref[TEI.damageSpan]{damageSpan} \hyperref[TEI.delSpan]{delSpan} \hyperref[TEI.ex]{ex} \hyperref[TEI.fw]{fw} \hyperref[TEI.handShift]{handShift} \hyperref[TEI.listTranspose]{listTranspose} \hyperref[TEI.metamark]{metamark} \hyperref[TEI.mod]{mod} \hyperref[TEI.redo]{redo} \hyperref[TEI.restore]{restore} \hyperref[TEI.retrace]{retrace} \hyperref[TEI.secl]{secl} \hyperref[TEI.space]{space} \hyperref[TEI.subst]{subst} \hyperref[TEI.substJoin]{substJoin} \hyperref[TEI.supplied]{supplied} \hyperref[TEI.surplus]{surplus} \hyperref[TEI.undo]{undo}\par des données textuelles
    \item[{Exemple}]
  \leavevmode\bgroup\exampleFont \begin{shaded}\noindent\mbox{}{<\textbf{persName}>}\mbox{}\newline 
\hspace*{6pt}{<\textbf{forename}>}Catherine{</\textbf{forename}>}\mbox{}\newline 
\hspace*{6pt}{<\textbf{genName}>}II{</\textbf{genName}>}, {<\textbf{addName}\hspace*{6pt}{type}="{epithet}">} la\mbox{}\newline 
\hspace*{6pt}\hspace*{6pt} Grande{</\textbf{addName}>}, {<\textbf{roleName}>}impératrice de Russie{</\textbf{roleName}>}\mbox{}\newline 
{</\textbf{persName}>}\end{shaded}\egroup 


    \item[{Modèle de contenu}]
  \mbox{}\hfill\\[-10pt]\begin{Verbatim}[fontsize=\small]
<content>
 <macroRef key="macro.phraseSeq"/>
</content>
    
\end{Verbatim}

    \item[{Schéma Declaration}]
  \mbox{}\hfill\\[-10pt]\begin{Verbatim}[fontsize=\small]
element addName
{
   tei_att.global.attributes,
   tei_att.personal.attributes,
   tei_att.typed.attributes,
   tei_macro.phraseSeq}
\end{Verbatim}

\end{reflist}  \index{addSpan=<addSpan>|oddindex}
\begin{reflist}
\item[]\begin{specHead}{TEI.addSpan}{<addSpan> }(partie de texte ajoutée) marque le début d'une longue partie de texte ajoutée par un auteur, un copiste, un annotateur ou un correcteur (voir aussi \hyperref[TEI.add]{<add>}). [\xref{http://www.tei-c.org/release/doc/tei-p5-doc/en/html/PH.html\#PHAD}{11.3.1.4. Additions and Deletions}]\end{specHead} 
    \item[{Module}]
  transcr
    \item[{Attributs}]
  Attributs \hyperref[TEI.att.global]{att.global} (\textit{@xml:id}, \textit{@n}, \textit{@xml:lang}, \textit{@xml:base}, \textit{@xml:space})  (\hyperref[TEI.att.global.rendition]{att.global.rendition} (\textit{@rend}, \textit{@style}, \textit{@rendition})) (\hyperref[TEI.att.global.linking]{att.global.linking} (\textit{@corresp}, \textit{@synch}, \textit{@sameAs}, \textit{@copyOf}, \textit{@next}, \textit{@prev}, \textit{@exclude}, \textit{@select})) (\hyperref[TEI.att.global.analytic]{att.global.analytic} (\textit{@ana})) (\hyperref[TEI.att.global.facs]{att.global.facs} (\textit{@facs})) (\hyperref[TEI.att.global.change]{att.global.change} (\textit{@change})) (\hyperref[TEI.att.global.responsibility]{att.global.responsibility} (\textit{@cert}, \textit{@resp})) (\hyperref[TEI.att.global.source]{att.global.source} (\textit{@source})) \hyperref[TEI.att.transcriptional]{att.transcriptional} (\textit{@status}, \textit{@cause}, \textit{@seq})  (\hyperref[TEI.att.editLike]{att.editLike} (\textit{@evidence}, \textit{@instant}) (\hyperref[TEI.att.dimensions]{att.dimensions} (\textit{@unit}, \textit{@quantity}, \textit{@extent}, \textit{@precision}, \textit{@scope}) (\hyperref[TEI.att.ranging]{att.ranging} (\textit{@atLeast}, \textit{@atMost}, \textit{@min}, \textit{@max}, \textit{@confidence})) ) ) (\hyperref[TEI.att.written]{att.written} (\textit{@hand})) \hyperref[TEI.att.placement]{att.placement} (\textit{@place}) \hyperref[TEI.att.typed]{att.typed} (\textit{@type}, \textit{@subtype}) \hyperref[TEI.att.spanning]{att.spanning} (\textit{@spanTo}) 
    \item[{Membre du}]
  \hyperref[TEI.model.global.edit]{model.global.edit}
    \item[{Contenu dans}]
  
    \item[analysis: ]
   \hyperref[TEI.cl]{cl} \hyperref[TEI.m]{m} \hyperref[TEI.phr]{phr} \hyperref[TEI.s]{s} \hyperref[TEI.span]{span} \hyperref[TEI.w]{w}\par 
    \item[core: ]
   \hyperref[TEI.abbr]{abbr} \hyperref[TEI.add]{add} \hyperref[TEI.addrLine]{addrLine} \hyperref[TEI.address]{address} \hyperref[TEI.author]{author} \hyperref[TEI.bibl]{bibl} \hyperref[TEI.biblScope]{biblScope} \hyperref[TEI.cit]{cit} \hyperref[TEI.citedRange]{citedRange} \hyperref[TEI.corr]{corr} \hyperref[TEI.date]{date} \hyperref[TEI.del]{del} \hyperref[TEI.distinct]{distinct} \hyperref[TEI.editor]{editor} \hyperref[TEI.email]{email} \hyperref[TEI.emph]{emph} \hyperref[TEI.expan]{expan} \hyperref[TEI.foreign]{foreign} \hyperref[TEI.gloss]{gloss} \hyperref[TEI.head]{head} \hyperref[TEI.headItem]{headItem} \hyperref[TEI.headLabel]{headLabel} \hyperref[TEI.hi]{hi} \hyperref[TEI.imprint]{imprint} \hyperref[TEI.item]{item} \hyperref[TEI.l]{l} \hyperref[TEI.label]{label} \hyperref[TEI.lg]{lg} \hyperref[TEI.list]{list} \hyperref[TEI.measure]{measure} \hyperref[TEI.mentioned]{mentioned} \hyperref[TEI.name]{name} \hyperref[TEI.note]{note} \hyperref[TEI.num]{num} \hyperref[TEI.orig]{orig} \hyperref[TEI.p]{p} \hyperref[TEI.pubPlace]{pubPlace} \hyperref[TEI.publisher]{publisher} \hyperref[TEI.q]{q} \hyperref[TEI.quote]{quote} \hyperref[TEI.ref]{ref} \hyperref[TEI.reg]{reg} \hyperref[TEI.resp]{resp} \hyperref[TEI.rs]{rs} \hyperref[TEI.said]{said} \hyperref[TEI.series]{series} \hyperref[TEI.sic]{sic} \hyperref[TEI.soCalled]{soCalled} \hyperref[TEI.sp]{sp} \hyperref[TEI.speaker]{speaker} \hyperref[TEI.stage]{stage} \hyperref[TEI.street]{street} \hyperref[TEI.term]{term} \hyperref[TEI.textLang]{textLang} \hyperref[TEI.time]{time} \hyperref[TEI.title]{title} \hyperref[TEI.unclear]{unclear}\par 
    \item[figures: ]
   \hyperref[TEI.cell]{cell} \hyperref[TEI.figure]{figure} \hyperref[TEI.table]{table}\par 
    \item[header: ]
   \hyperref[TEI.authority]{authority} \hyperref[TEI.change]{change} \hyperref[TEI.classCode]{classCode} \hyperref[TEI.distributor]{distributor} \hyperref[TEI.edition]{edition} \hyperref[TEI.extent]{extent} \hyperref[TEI.funder]{funder} \hyperref[TEI.language]{language} \hyperref[TEI.licence]{licence}\par 
    \item[linking: ]
   \hyperref[TEI.ab]{ab} \hyperref[TEI.seg]{seg}\par 
    \item[msdescription: ]
   \hyperref[TEI.accMat]{accMat} \hyperref[TEI.acquisition]{acquisition} \hyperref[TEI.additions]{additions} \hyperref[TEI.catchwords]{catchwords} \hyperref[TEI.collation]{collation} \hyperref[TEI.colophon]{colophon} \hyperref[TEI.condition]{condition} \hyperref[TEI.custEvent]{custEvent} \hyperref[TEI.decoNote]{decoNote} \hyperref[TEI.explicit]{explicit} \hyperref[TEI.filiation]{filiation} \hyperref[TEI.finalRubric]{finalRubric} \hyperref[TEI.foliation]{foliation} \hyperref[TEI.heraldry]{heraldry} \hyperref[TEI.incipit]{incipit} \hyperref[TEI.layout]{layout} \hyperref[TEI.material]{material} \hyperref[TEI.msItem]{msItem} \hyperref[TEI.musicNotation]{musicNotation} \hyperref[TEI.objectType]{objectType} \hyperref[TEI.origDate]{origDate} \hyperref[TEI.origPlace]{origPlace} \hyperref[TEI.origin]{origin} \hyperref[TEI.provenance]{provenance} \hyperref[TEI.rubric]{rubric} \hyperref[TEI.secFol]{secFol} \hyperref[TEI.signatures]{signatures} \hyperref[TEI.source]{source} \hyperref[TEI.stamp]{stamp} \hyperref[TEI.summary]{summary} \hyperref[TEI.support]{support} \hyperref[TEI.surrogates]{surrogates} \hyperref[TEI.typeNote]{typeNote} \hyperref[TEI.watermark]{watermark}\par 
    \item[namesdates: ]
   \hyperref[TEI.addName]{addName} \hyperref[TEI.affiliation]{affiliation} \hyperref[TEI.country]{country} \hyperref[TEI.forename]{forename} \hyperref[TEI.genName]{genName} \hyperref[TEI.geogName]{geogName} \hyperref[TEI.nameLink]{nameLink} \hyperref[TEI.orgName]{orgName} \hyperref[TEI.persName]{persName} \hyperref[TEI.person]{person} \hyperref[TEI.personGrp]{personGrp} \hyperref[TEI.persona]{persona} \hyperref[TEI.placeName]{placeName} \hyperref[TEI.region]{region} \hyperref[TEI.roleName]{roleName} \hyperref[TEI.settlement]{settlement} \hyperref[TEI.surname]{surname}\par 
    \item[textstructure: ]
   \hyperref[TEI.back]{back} \hyperref[TEI.body]{body} \hyperref[TEI.div]{div} \hyperref[TEI.docAuthor]{docAuthor} \hyperref[TEI.docDate]{docDate} \hyperref[TEI.docEdition]{docEdition} \hyperref[TEI.docTitle]{docTitle} \hyperref[TEI.floatingText]{floatingText} \hyperref[TEI.front]{front} \hyperref[TEI.group]{group} \hyperref[TEI.text]{text} \hyperref[TEI.titlePage]{titlePage} \hyperref[TEI.titlePart]{titlePart}\par 
    \item[transcr: ]
   \hyperref[TEI.damage]{damage} \hyperref[TEI.fw]{fw} \hyperref[TEI.line]{line} \hyperref[TEI.metamark]{metamark} \hyperref[TEI.mod]{mod} \hyperref[TEI.restore]{restore} \hyperref[TEI.retrace]{retrace} \hyperref[TEI.secl]{secl} \hyperref[TEI.sourceDoc]{sourceDoc} \hyperref[TEI.supplied]{supplied} \hyperref[TEI.surface]{surface} \hyperref[TEI.surfaceGrp]{surfaceGrp} \hyperref[TEI.surplus]{surplus} \hyperref[TEI.zone]{zone}
    \item[{Peut contenir}]
  Elément vide
    \item[{Note}]
  \par
Le début et la fin de la partie de texte ajoutée doivent être marqués ; le début, par l'élément \hyperref[TEI.addSpan]{<addSpan>} lui-même, la fin, par l'attribut {\itshape spanTo}.
    \item[{Exemple}]
  \leavevmode\bgroup\exampleFont \begin{shaded}\noindent\mbox{}{<\textbf{handNote}\hspace*{6pt}{scribe}="{HelgiÓlafsson}"\mbox{}\newline 
\hspace*{6pt}{xml:id}="{HEOL}"/>}\mbox{}\newline 
\textit{<!-- ... -->}\mbox{}\newline 
{<\textbf{body}>}\mbox{}\newline 
\hspace*{6pt}{<\textbf{div}>}\mbox{}\newline 
\textit{<!-- text here -->}\mbox{}\newline 
\hspace*{6pt}{</\textbf{div}>}\mbox{}\newline 
\hspace*{6pt}{<\textbf{addSpan}\hspace*{6pt}{hand}="{\#HEOL}"\hspace*{6pt}{n}="{added gathering}"\mbox{}\newline 
\hspace*{6pt}\hspace*{6pt}{spanTo}="{\#P025}"/>}\mbox{}\newline 
\hspace*{6pt}{<\textbf{div}>}\mbox{}\newline 
\textit{<!-- text of first added poem here -->}\mbox{}\newline 
\hspace*{6pt}{</\textbf{div}>}\mbox{}\newline 
\hspace*{6pt}{<\textbf{div}>}\mbox{}\newline 
\textit{<!-- text of second added poem here -->}\mbox{}\newline 
\hspace*{6pt}{</\textbf{div}>}\mbox{}\newline 
\hspace*{6pt}{<\textbf{div}>}\mbox{}\newline 
\textit{<!-- text of third added poem here -->}\mbox{}\newline 
\hspace*{6pt}{</\textbf{div}>}\mbox{}\newline 
\hspace*{6pt}{<\textbf{div}>}\mbox{}\newline 
\textit{<!-- text of fourth added poem here -->}\mbox{}\newline 
\hspace*{6pt}{</\textbf{div}>}\mbox{}\newline 
\hspace*{6pt}{<\textbf{anchor}\hspace*{6pt}{xml:id}="{P025}"/>}\mbox{}\newline 
\hspace*{6pt}{<\textbf{div}>}\mbox{}\newline 
\textit{<!-- more text here -->}\mbox{}\newline 
\hspace*{6pt}{</\textbf{div}>}\mbox{}\newline 
{</\textbf{body}>}\end{shaded}\egroup 


    \item[{Schematron}]
   <sch:assert test="@spanTo">The @spanTo attribute of <sch:name/> is required.</sch:assert>
    \item[{Schematron}]
   <sch:assert test="@spanTo">L'attribut spanTo est requis.</sch:assert>
    \item[{Modèle de contenu}]
  \fbox{\ttfamily <content/>\newline
    } 
    \item[{Schéma Declaration}]
  \mbox{}\hfill\\[-10pt]\begin{Verbatim}[fontsize=\small]
element addSpan
{
   tei_att.global.attributes,
   tei_att.transcriptional.attributes,
   tei_att.placement.attributes,
   tei_att.typed.attributes,
   tei_att.spanning.attributes,
   empty
}
\end{Verbatim}

\end{reflist}  \index{additional=<additional>|oddindex}
\begin{reflist}
\item[]\begin{specHead}{TEI.additional}{<additional> }(informations complémentaires) regroupe les informations complémentaires sur le manuscrit, incluant une bibliographie, des indications sur ses reproductions, ou des informations sur sa conservation et sur sa gestion [\xref{http://www.tei-c.org/release/doc/tei-p5-doc/en/html/MS.html\#msad}{10.9. Additional Information}]\end{specHead} 
    \item[{Module}]
  msdescription
    \item[{Attributs}]
  Attributs \hyperref[TEI.att.global]{att.global} (\textit{@xml:id}, \textit{@n}, \textit{@xml:lang}, \textit{@xml:base}, \textit{@xml:space})  (\hyperref[TEI.att.global.rendition]{att.global.rendition} (\textit{@rend}, \textit{@style}, \textit{@rendition})) (\hyperref[TEI.att.global.linking]{att.global.linking} (\textit{@corresp}, \textit{@synch}, \textit{@sameAs}, \textit{@copyOf}, \textit{@next}, \textit{@prev}, \textit{@exclude}, \textit{@select})) (\hyperref[TEI.att.global.analytic]{att.global.analytic} (\textit{@ana})) (\hyperref[TEI.att.global.facs]{att.global.facs} (\textit{@facs})) (\hyperref[TEI.att.global.change]{att.global.change} (\textit{@change})) (\hyperref[TEI.att.global.responsibility]{att.global.responsibility} (\textit{@cert}, \textit{@resp})) (\hyperref[TEI.att.global.source]{att.global.source} (\textit{@source}))
    \item[{Contenu dans}]
  
    \item[msdescription: ]
   \hyperref[TEI.msDesc]{msDesc} \hyperref[TEI.msFrag]{msFrag} \hyperref[TEI.msPart]{msPart}
    \item[{Peut contenir}]
  
    \item[core: ]
   \hyperref[TEI.listBibl]{listBibl}\par 
    \item[msdescription: ]
   \hyperref[TEI.adminInfo]{adminInfo} \hyperref[TEI.surrogates]{surrogates}
    \item[{Exemple}]
  \leavevmode\bgroup\exampleFont \begin{shaded}\noindent\mbox{}{<\textbf{additional}>}\mbox{}\newline 
\hspace*{6pt}{<\textbf{adminInfo}>}\mbox{}\newline 
\hspace*{6pt}\hspace*{6pt}{<\textbf{recordHist}>}\mbox{}\newline 
\hspace*{6pt}\hspace*{6pt}\hspace*{6pt}{<\textbf{p}>}\mbox{}\newline 
\textit{<!-- ... -->}\mbox{}\newline 
\hspace*{6pt}\hspace*{6pt}\hspace*{6pt}{</\textbf{p}>}\mbox{}\newline 
\hspace*{6pt}\hspace*{6pt}{</\textbf{recordHist}>}\mbox{}\newline 
\hspace*{6pt}\hspace*{6pt}{<\textbf{custodialHist}>}\mbox{}\newline 
\hspace*{6pt}\hspace*{6pt}\hspace*{6pt}{<\textbf{p}>}\mbox{}\newline 
\textit{<!-- ... -->}\mbox{}\newline 
\hspace*{6pt}\hspace*{6pt}\hspace*{6pt}{</\textbf{p}>}\mbox{}\newline 
\hspace*{6pt}\hspace*{6pt}{</\textbf{custodialHist}>}\mbox{}\newline 
\hspace*{6pt}{</\textbf{adminInfo}>}\mbox{}\newline 
\hspace*{6pt}{<\textbf{surrogates}>}\mbox{}\newline 
\hspace*{6pt}\hspace*{6pt}{<\textbf{p}>}\mbox{}\newline 
\textit{<!-- ... -->}\mbox{}\newline 
\hspace*{6pt}\hspace*{6pt}{</\textbf{p}>}\mbox{}\newline 
\hspace*{6pt}{</\textbf{surrogates}>}\mbox{}\newline 
\hspace*{6pt}{<\textbf{listBibl}>}\mbox{}\newline 
\hspace*{6pt}\hspace*{6pt}{<\textbf{bibl}>}\mbox{}\newline 
\textit{<!-- ... -->}\mbox{}\newline 
\hspace*{6pt}\hspace*{6pt}{</\textbf{bibl}>}\mbox{}\newline 
\hspace*{6pt}{</\textbf{listBibl}>}\mbox{}\newline 
{</\textbf{additional}>}\end{shaded}\egroup 


    \item[{Modèle de contenu}]
  \mbox{}\hfill\\[-10pt]\begin{Verbatim}[fontsize=\small]
<content>
 <sequence maxOccurs="1" minOccurs="1">
  <elementRef key="adminInfo" minOccurs="0"/>
  <elementRef key="surrogates"
   minOccurs="0"/>
  <elementRef key="listBibl" minOccurs="0"/>
 </sequence>
</content>
    
\end{Verbatim}

    \item[{Schéma Declaration}]
  \mbox{}\hfill\\[-10pt]\begin{Verbatim}[fontsize=\small]
element additional
{
   tei_att.global.attributes,
   ( tei_adminInfo?, tei_surrogates?, tei_listBibl? )
}
\end{Verbatim}

\end{reflist}  \index{additions=<additions>|oddindex}
\begin{reflist}
\item[]\begin{specHead}{TEI.additions}{<additions> }(ajouts) contient la description des ajouts significatifs trouvés dans un manuscrit, tels que gloses marginales ou autres annotations. [\xref{http://www.tei-c.org/release/doc/tei-p5-doc/en/html/MS.html\#msph2}{10.7.2. Writing, Decoration, and Other Notations}]\end{specHead} 
    \item[{Module}]
  msdescription
    \item[{Attributs}]
  Attributs \hyperref[TEI.att.global]{att.global} (\textit{@xml:id}, \textit{@n}, \textit{@xml:lang}, \textit{@xml:base}, \textit{@xml:space})  (\hyperref[TEI.att.global.rendition]{att.global.rendition} (\textit{@rend}, \textit{@style}, \textit{@rendition})) (\hyperref[TEI.att.global.linking]{att.global.linking} (\textit{@corresp}, \textit{@synch}, \textit{@sameAs}, \textit{@copyOf}, \textit{@next}, \textit{@prev}, \textit{@exclude}, \textit{@select})) (\hyperref[TEI.att.global.analytic]{att.global.analytic} (\textit{@ana})) (\hyperref[TEI.att.global.facs]{att.global.facs} (\textit{@facs})) (\hyperref[TEI.att.global.change]{att.global.change} (\textit{@change})) (\hyperref[TEI.att.global.responsibility]{att.global.responsibility} (\textit{@cert}, \textit{@resp})) (\hyperref[TEI.att.global.source]{att.global.source} (\textit{@source}))
    \item[{Membre du}]
  \hyperref[TEI.model.physDescPart]{model.physDescPart}
    \item[{Contenu dans}]
  
    \item[msdescription: ]
   \hyperref[TEI.physDesc]{physDesc}
    \item[{Peut contenir}]
  
    \item[analysis: ]
   \hyperref[TEI.c]{c} \hyperref[TEI.cl]{cl} \hyperref[TEI.interp]{interp} \hyperref[TEI.interpGrp]{interpGrp} \hyperref[TEI.m]{m} \hyperref[TEI.pc]{pc} \hyperref[TEI.phr]{phr} \hyperref[TEI.s]{s} \hyperref[TEI.span]{span} \hyperref[TEI.spanGrp]{spanGrp} \hyperref[TEI.w]{w}\par 
    \item[core: ]
   \hyperref[TEI.abbr]{abbr} \hyperref[TEI.add]{add} \hyperref[TEI.address]{address} \hyperref[TEI.bibl]{bibl} \hyperref[TEI.biblStruct]{biblStruct} \hyperref[TEI.binaryObject]{binaryObject} \hyperref[TEI.cb]{cb} \hyperref[TEI.choice]{choice} \hyperref[TEI.cit]{cit} \hyperref[TEI.corr]{corr} \hyperref[TEI.date]{date} \hyperref[TEI.del]{del} \hyperref[TEI.desc]{desc} \hyperref[TEI.distinct]{distinct} \hyperref[TEI.email]{email} \hyperref[TEI.emph]{emph} \hyperref[TEI.expan]{expan} \hyperref[TEI.foreign]{foreign} \hyperref[TEI.gap]{gap} \hyperref[TEI.gb]{gb} \hyperref[TEI.gloss]{gloss} \hyperref[TEI.graphic]{graphic} \hyperref[TEI.hi]{hi} \hyperref[TEI.index]{index} \hyperref[TEI.l]{l} \hyperref[TEI.label]{label} \hyperref[TEI.lb]{lb} \hyperref[TEI.lg]{lg} \hyperref[TEI.list]{list} \hyperref[TEI.listBibl]{listBibl} \hyperref[TEI.measure]{measure} \hyperref[TEI.measureGrp]{measureGrp} \hyperref[TEI.media]{media} \hyperref[TEI.mentioned]{mentioned} \hyperref[TEI.milestone]{milestone} \hyperref[TEI.name]{name} \hyperref[TEI.note]{note} \hyperref[TEI.num]{num} \hyperref[TEI.orig]{orig} \hyperref[TEI.p]{p} \hyperref[TEI.pb]{pb} \hyperref[TEI.ptr]{ptr} \hyperref[TEI.q]{q} \hyperref[TEI.quote]{quote} \hyperref[TEI.ref]{ref} \hyperref[TEI.reg]{reg} \hyperref[TEI.rs]{rs} \hyperref[TEI.said]{said} \hyperref[TEI.sic]{sic} \hyperref[TEI.soCalled]{soCalled} \hyperref[TEI.sp]{sp} \hyperref[TEI.stage]{stage} \hyperref[TEI.term]{term} \hyperref[TEI.time]{time} \hyperref[TEI.title]{title} \hyperref[TEI.unclear]{unclear}\par 
    \item[derived-module-tei.istex: ]
   \hyperref[TEI.math]{math} \hyperref[TEI.mrow]{mrow}\par 
    \item[figures: ]
   \hyperref[TEI.figure]{figure} \hyperref[TEI.formula]{formula} \hyperref[TEI.notatedMusic]{notatedMusic} \hyperref[TEI.table]{table}\par 
    \item[header: ]
   \hyperref[TEI.biblFull]{biblFull} \hyperref[TEI.idno]{idno}\par 
    \item[iso-fs: ]
   \hyperref[TEI.fLib]{fLib} \hyperref[TEI.fs]{fs} \hyperref[TEI.fvLib]{fvLib}\par 
    \item[linking: ]
   \hyperref[TEI.ab]{ab} \hyperref[TEI.alt]{alt} \hyperref[TEI.altGrp]{altGrp} \hyperref[TEI.anchor]{anchor} \hyperref[TEI.join]{join} \hyperref[TEI.joinGrp]{joinGrp} \hyperref[TEI.link]{link} \hyperref[TEI.linkGrp]{linkGrp} \hyperref[TEI.seg]{seg} \hyperref[TEI.timeline]{timeline}\par 
    \item[msdescription: ]
   \hyperref[TEI.catchwords]{catchwords} \hyperref[TEI.depth]{depth} \hyperref[TEI.dim]{dim} \hyperref[TEI.dimensions]{dimensions} \hyperref[TEI.height]{height} \hyperref[TEI.heraldry]{heraldry} \hyperref[TEI.locus]{locus} \hyperref[TEI.locusGrp]{locusGrp} \hyperref[TEI.material]{material} \hyperref[TEI.msDesc]{msDesc} \hyperref[TEI.objectType]{objectType} \hyperref[TEI.origDate]{origDate} \hyperref[TEI.origPlace]{origPlace} \hyperref[TEI.secFol]{secFol} \hyperref[TEI.signatures]{signatures} \hyperref[TEI.source]{source} \hyperref[TEI.stamp]{stamp} \hyperref[TEI.watermark]{watermark} \hyperref[TEI.width]{width}\par 
    \item[namesdates: ]
   \hyperref[TEI.addName]{addName} \hyperref[TEI.affiliation]{affiliation} \hyperref[TEI.country]{country} \hyperref[TEI.forename]{forename} \hyperref[TEI.genName]{genName} \hyperref[TEI.geogName]{geogName} \hyperref[TEI.listOrg]{listOrg} \hyperref[TEI.listPlace]{listPlace} \hyperref[TEI.location]{location} \hyperref[TEI.nameLink]{nameLink} \hyperref[TEI.orgName]{orgName} \hyperref[TEI.persName]{persName} \hyperref[TEI.placeName]{placeName} \hyperref[TEI.region]{region} \hyperref[TEI.roleName]{roleName} \hyperref[TEI.settlement]{settlement} \hyperref[TEI.state]{state} \hyperref[TEI.surname]{surname}\par 
    \item[spoken: ]
   \hyperref[TEI.annotationBlock]{annotationBlock}\par 
    \item[textstructure: ]
   \hyperref[TEI.floatingText]{floatingText}\par 
    \item[transcr: ]
   \hyperref[TEI.addSpan]{addSpan} \hyperref[TEI.am]{am} \hyperref[TEI.damage]{damage} \hyperref[TEI.damageSpan]{damageSpan} \hyperref[TEI.delSpan]{delSpan} \hyperref[TEI.ex]{ex} \hyperref[TEI.fw]{fw} \hyperref[TEI.handShift]{handShift} \hyperref[TEI.listTranspose]{listTranspose} \hyperref[TEI.metamark]{metamark} \hyperref[TEI.mod]{mod} \hyperref[TEI.redo]{redo} \hyperref[TEI.restore]{restore} \hyperref[TEI.retrace]{retrace} \hyperref[TEI.secl]{secl} \hyperref[TEI.space]{space} \hyperref[TEI.subst]{subst} \hyperref[TEI.substJoin]{substJoin} \hyperref[TEI.supplied]{supplied} \hyperref[TEI.surplus]{surplus} \hyperref[TEI.undo]{undo}\par des données textuelles
    \item[{Exemple}]
  \leavevmode\bgroup\exampleFont \begin{shaded}\noindent\mbox{}{<\textbf{additions}>}\mbox{}\newline 
\hspace*{6pt}{<\textbf{p}>}There are several marginalia in this manuscript. Some consist of single characters and\mbox{}\newline 
\hspace*{6pt}\hspace*{6pt} others are figurative. On 8v is to be found a drawing of a mans head wearing a hat. At\mbox{}\newline 
\hspace*{6pt}\hspace*{6pt} times sentences occurs: On 5v:{<\textbf{q}\hspace*{6pt}{xml:lang}="{is}">}Her er skrif andres isslendin{</\textbf{q}>}, on\mbox{}\newline 
\hspace*{6pt}\hspace*{6pt} 19r: {<\textbf{q}\hspace*{6pt}{xml:lang}="{is}">}þeim go{</\textbf{q}>}, on 21r: {<\textbf{q}\hspace*{6pt}{xml:lang}="{is}">}amen med aund ok munn halla\mbox{}\newline 
\hspace*{6pt}\hspace*{6pt}\hspace*{6pt}\hspace*{6pt} rei knar hofud summu all huad batar þad mælgi ok mal{</\textbf{q}>}, On 21v: some runic letters\mbox{}\newline 
\hspace*{6pt}\hspace*{6pt} and the sentence {<\textbf{q}\hspace*{6pt}{xml:lang}="{la}">}aue maria gracia plena dominus{</\textbf{q}>}.{</\textbf{p}>}\mbox{}\newline 
{</\textbf{additions}>}\end{shaded}\egroup 


    \item[{Modèle de contenu}]
  \mbox{}\hfill\\[-10pt]\begin{Verbatim}[fontsize=\small]
<content>
 <macroRef key="macro.specialPara"/>
</content>
    
\end{Verbatim}

    \item[{Schéma Declaration}]
  \mbox{}\hfill\\[-10pt]\begin{Verbatim}[fontsize=\small]
element additions { tei_att.global.attributes, tei_macro.specialPara }
\end{Verbatim}

\end{reflist}  \index{addrLine=<addrLine>|oddindex}
\begin{reflist}
\item[]\begin{specHead}{TEI.addrLine}{<addrLine> }(ligne d'adresse) contient une ligne d'adresse postale. [\xref{http://www.tei-c.org/release/doc/tei-p5-doc/en/html/CO.html\#CONAAD}{3.5.2. Addresses} \xref{http://www.tei-c.org/release/doc/tei-p5-doc/en/html/HD.html\#HD24}{2.2.4. Publication, Distribution, Licensing, etc.} \xref{http://www.tei-c.org/release/doc/tei-p5-doc/en/html/CO.html\#COBICOI}{3.11.2.4. Imprint, Size of a Document, and Reprint Information}]\end{specHead} 
    \item[{Module}]
  core
    \item[{Attributs}]
  Attributs \hyperref[TEI.att.global]{att.global} (\textit{@xml:id}, \textit{@n}, \textit{@xml:lang}, \textit{@xml:base}, \textit{@xml:space})  (\hyperref[TEI.att.global.rendition]{att.global.rendition} (\textit{@rend}, \textit{@style}, \textit{@rendition})) (\hyperref[TEI.att.global.linking]{att.global.linking} (\textit{@corresp}, \textit{@synch}, \textit{@sameAs}, \textit{@copyOf}, \textit{@next}, \textit{@prev}, \textit{@exclude}, \textit{@select})) (\hyperref[TEI.att.global.analytic]{att.global.analytic} (\textit{@ana})) (\hyperref[TEI.att.global.facs]{att.global.facs} (\textit{@facs})) (\hyperref[TEI.att.global.change]{att.global.change} (\textit{@change})) (\hyperref[TEI.att.global.responsibility]{att.global.responsibility} (\textit{@cert}, \textit{@resp})) (\hyperref[TEI.att.global.source]{att.global.source} (\textit{@source}))
    \item[{Membre du}]
  \hyperref[TEI.model.addrPart]{model.addrPart}
    \item[{Contenu dans}]
  
    \item[core: ]
   \hyperref[TEI.address]{address}
    \item[{Peut contenir}]
  
    \item[analysis: ]
   \hyperref[TEI.c]{c} \hyperref[TEI.cl]{cl} \hyperref[TEI.interp]{interp} \hyperref[TEI.interpGrp]{interpGrp} \hyperref[TEI.m]{m} \hyperref[TEI.pc]{pc} \hyperref[TEI.phr]{phr} \hyperref[TEI.s]{s} \hyperref[TEI.span]{span} \hyperref[TEI.spanGrp]{spanGrp} \hyperref[TEI.w]{w}\par 
    \item[core: ]
   \hyperref[TEI.abbr]{abbr} \hyperref[TEI.add]{add} \hyperref[TEI.address]{address} \hyperref[TEI.binaryObject]{binaryObject} \hyperref[TEI.cb]{cb} \hyperref[TEI.choice]{choice} \hyperref[TEI.corr]{corr} \hyperref[TEI.date]{date} \hyperref[TEI.del]{del} \hyperref[TEI.distinct]{distinct} \hyperref[TEI.email]{email} \hyperref[TEI.emph]{emph} \hyperref[TEI.expan]{expan} \hyperref[TEI.foreign]{foreign} \hyperref[TEI.gap]{gap} \hyperref[TEI.gb]{gb} \hyperref[TEI.gloss]{gloss} \hyperref[TEI.graphic]{graphic} \hyperref[TEI.hi]{hi} \hyperref[TEI.index]{index} \hyperref[TEI.lb]{lb} \hyperref[TEI.measure]{measure} \hyperref[TEI.measureGrp]{measureGrp} \hyperref[TEI.media]{media} \hyperref[TEI.mentioned]{mentioned} \hyperref[TEI.milestone]{milestone} \hyperref[TEI.name]{name} \hyperref[TEI.note]{note} \hyperref[TEI.num]{num} \hyperref[TEI.orig]{orig} \hyperref[TEI.pb]{pb} \hyperref[TEI.ptr]{ptr} \hyperref[TEI.ref]{ref} \hyperref[TEI.reg]{reg} \hyperref[TEI.rs]{rs} \hyperref[TEI.sic]{sic} \hyperref[TEI.soCalled]{soCalled} \hyperref[TEI.term]{term} \hyperref[TEI.time]{time} \hyperref[TEI.title]{title} \hyperref[TEI.unclear]{unclear}\par 
    \item[derived-module-tei.istex: ]
   \hyperref[TEI.math]{math} \hyperref[TEI.mrow]{mrow}\par 
    \item[figures: ]
   \hyperref[TEI.figure]{figure} \hyperref[TEI.formula]{formula} \hyperref[TEI.notatedMusic]{notatedMusic}\par 
    \item[header: ]
   \hyperref[TEI.idno]{idno}\par 
    \item[iso-fs: ]
   \hyperref[TEI.fLib]{fLib} \hyperref[TEI.fs]{fs} \hyperref[TEI.fvLib]{fvLib}\par 
    \item[linking: ]
   \hyperref[TEI.alt]{alt} \hyperref[TEI.altGrp]{altGrp} \hyperref[TEI.anchor]{anchor} \hyperref[TEI.join]{join} \hyperref[TEI.joinGrp]{joinGrp} \hyperref[TEI.link]{link} \hyperref[TEI.linkGrp]{linkGrp} \hyperref[TEI.seg]{seg} \hyperref[TEI.timeline]{timeline}\par 
    \item[msdescription: ]
   \hyperref[TEI.catchwords]{catchwords} \hyperref[TEI.depth]{depth} \hyperref[TEI.dim]{dim} \hyperref[TEI.dimensions]{dimensions} \hyperref[TEI.height]{height} \hyperref[TEI.heraldry]{heraldry} \hyperref[TEI.locus]{locus} \hyperref[TEI.locusGrp]{locusGrp} \hyperref[TEI.material]{material} \hyperref[TEI.objectType]{objectType} \hyperref[TEI.origDate]{origDate} \hyperref[TEI.origPlace]{origPlace} \hyperref[TEI.secFol]{secFol} \hyperref[TEI.signatures]{signatures} \hyperref[TEI.source]{source} \hyperref[TEI.stamp]{stamp} \hyperref[TEI.watermark]{watermark} \hyperref[TEI.width]{width}\par 
    \item[namesdates: ]
   \hyperref[TEI.addName]{addName} \hyperref[TEI.affiliation]{affiliation} \hyperref[TEI.country]{country} \hyperref[TEI.forename]{forename} \hyperref[TEI.genName]{genName} \hyperref[TEI.geogName]{geogName} \hyperref[TEI.location]{location} \hyperref[TEI.nameLink]{nameLink} \hyperref[TEI.orgName]{orgName} \hyperref[TEI.persName]{persName} \hyperref[TEI.placeName]{placeName} \hyperref[TEI.region]{region} \hyperref[TEI.roleName]{roleName} \hyperref[TEI.settlement]{settlement} \hyperref[TEI.state]{state} \hyperref[TEI.surname]{surname}\par 
    \item[spoken: ]
   \hyperref[TEI.annotationBlock]{annotationBlock}\par 
    \item[transcr: ]
   \hyperref[TEI.addSpan]{addSpan} \hyperref[TEI.am]{am} \hyperref[TEI.damage]{damage} \hyperref[TEI.damageSpan]{damageSpan} \hyperref[TEI.delSpan]{delSpan} \hyperref[TEI.ex]{ex} \hyperref[TEI.fw]{fw} \hyperref[TEI.handShift]{handShift} \hyperref[TEI.listTranspose]{listTranspose} \hyperref[TEI.metamark]{metamark} \hyperref[TEI.mod]{mod} \hyperref[TEI.redo]{redo} \hyperref[TEI.restore]{restore} \hyperref[TEI.retrace]{retrace} \hyperref[TEI.secl]{secl} \hyperref[TEI.space]{space} \hyperref[TEI.subst]{subst} \hyperref[TEI.substJoin]{substJoin} \hyperref[TEI.supplied]{supplied} \hyperref[TEI.surplus]{surplus} \hyperref[TEI.undo]{undo}\par des données textuelles
    \item[{Note}]
  \par
Les adresses peuvent être encodées soit comme une suite de lignes, soit en utilisant un jeu d'éléments de la classe \textsf{model.addrPart}. Les types d'adresses autres que l'adresse postale, tels que les numéros de téléphone, les courriels, ne doivent pas être inclus directement à l'intérieur d'un élément \hyperref[TEI.address]{<address>} mais peuvent être contenus dans un élément \hyperref[TEI.addrLine]{<addrLine>} s'ils font partie de l'adresse imprimée dans un texte source.
    \item[{Exemple}]
  \leavevmode\bgroup\exampleFont \begin{shaded}\noindent\mbox{}{<\textbf{address}>}\mbox{}\newline 
\hspace*{6pt}{<\textbf{addrLine}>}44, avenue de la Libération{</\textbf{addrLine}>}\mbox{}\newline 
\hspace*{6pt}{<\textbf{addrLine}>}B.P. 30687{</\textbf{addrLine}>}\mbox{}\newline 
\hspace*{6pt}{<\textbf{addrLine}>}F 54063 NANCY CEDEX{</\textbf{addrLine}>}\mbox{}\newline 
\hspace*{6pt}{<\textbf{addrLine}>}FRANCE{</\textbf{addrLine}>}\mbox{}\newline 
{</\textbf{address}>}\end{shaded}\egroup 


    \item[{Modèle de contenu}]
  \mbox{}\hfill\\[-10pt]\begin{Verbatim}[fontsize=\small]
<content>
 <macroRef key="macro.phraseSeq"/>
</content>
    
\end{Verbatim}

    \item[{Schéma Declaration}]
  \mbox{}\hfill\\[-10pt]\begin{Verbatim}[fontsize=\small]
element addrLine { tei_att.global.attributes, tei_macro.phraseSeq }
\end{Verbatim}

\end{reflist}  \index{address=<address>|oddindex}
\begin{reflist}
\item[]\begin{specHead}{TEI.address}{<address> }contient une adresse postale ou d'un autre type, par exemple l'adresse d'un éditeur, d'un organisme ou d'une personne. [\xref{http://www.tei-c.org/release/doc/tei-p5-doc/en/html/CO.html\#CONAAD}{3.5.2. Addresses} \xref{http://www.tei-c.org/release/doc/tei-p5-doc/en/html/HD.html\#HD24}{2.2.4. Publication, Distribution, Licensing, etc.} \xref{http://www.tei-c.org/release/doc/tei-p5-doc/en/html/CO.html\#COBICOI}{3.11.2.4. Imprint, Size of a Document, and Reprint Information}]\end{specHead} 
    \item[{Module}]
  core
    \item[{Attributs}]
  Attributs \hyperref[TEI.att.global]{att.global} (\textit{@xml:id}, \textit{@n}, \textit{@xml:lang}, \textit{@xml:base}, \textit{@xml:space})  (\hyperref[TEI.att.global.rendition]{att.global.rendition} (\textit{@rend}, \textit{@style}, \textit{@rendition})) (\hyperref[TEI.att.global.linking]{att.global.linking} (\textit{@corresp}, \textit{@synch}, \textit{@sameAs}, \textit{@copyOf}, \textit{@next}, \textit{@prev}, \textit{@exclude}, \textit{@select})) (\hyperref[TEI.att.global.analytic]{att.global.analytic} (\textit{@ana})) (\hyperref[TEI.att.global.facs]{att.global.facs} (\textit{@facs})) (\hyperref[TEI.att.global.change]{att.global.change} (\textit{@change})) (\hyperref[TEI.att.global.responsibility]{att.global.responsibility} (\textit{@cert}, \textit{@resp})) (\hyperref[TEI.att.global.source]{att.global.source} (\textit{@source}))
    \item[{Membre du}]
  \hyperref[TEI.model.addressLike]{model.addressLike} \hyperref[TEI.model.publicationStmtPart.detail]{model.publicationStmtPart.detail}
    \item[{Contenu dans}]
  
    \item[analysis: ]
   \hyperref[TEI.cl]{cl} \hyperref[TEI.phr]{phr} \hyperref[TEI.s]{s} \hyperref[TEI.span]{span}\par 
    \item[core: ]
   \hyperref[TEI.abbr]{abbr} \hyperref[TEI.add]{add} \hyperref[TEI.addrLine]{addrLine} \hyperref[TEI.author]{author} \hyperref[TEI.bibl]{bibl} \hyperref[TEI.biblScope]{biblScope} \hyperref[TEI.citedRange]{citedRange} \hyperref[TEI.corr]{corr} \hyperref[TEI.date]{date} \hyperref[TEI.del]{del} \hyperref[TEI.desc]{desc} \hyperref[TEI.distinct]{distinct} \hyperref[TEI.editor]{editor} \hyperref[TEI.email]{email} \hyperref[TEI.emph]{emph} \hyperref[TEI.expan]{expan} \hyperref[TEI.foreign]{foreign} \hyperref[TEI.gloss]{gloss} \hyperref[TEI.head]{head} \hyperref[TEI.headItem]{headItem} \hyperref[TEI.headLabel]{headLabel} \hyperref[TEI.hi]{hi} \hyperref[TEI.item]{item} \hyperref[TEI.l]{l} \hyperref[TEI.label]{label} \hyperref[TEI.measure]{measure} \hyperref[TEI.meeting]{meeting} \hyperref[TEI.mentioned]{mentioned} \hyperref[TEI.name]{name} \hyperref[TEI.note]{note} \hyperref[TEI.num]{num} \hyperref[TEI.orig]{orig} \hyperref[TEI.p]{p} \hyperref[TEI.pubPlace]{pubPlace} \hyperref[TEI.publisher]{publisher} \hyperref[TEI.q]{q} \hyperref[TEI.quote]{quote} \hyperref[TEI.ref]{ref} \hyperref[TEI.reg]{reg} \hyperref[TEI.resp]{resp} \hyperref[TEI.rs]{rs} \hyperref[TEI.said]{said} \hyperref[TEI.sic]{sic} \hyperref[TEI.soCalled]{soCalled} \hyperref[TEI.speaker]{speaker} \hyperref[TEI.stage]{stage} \hyperref[TEI.street]{street} \hyperref[TEI.term]{term} \hyperref[TEI.textLang]{textLang} \hyperref[TEI.time]{time} \hyperref[TEI.title]{title} \hyperref[TEI.unclear]{unclear}\par 
    \item[figures: ]
   \hyperref[TEI.cell]{cell} \hyperref[TEI.figDesc]{figDesc}\par 
    \item[header: ]
   \hyperref[TEI.authority]{authority} \hyperref[TEI.change]{change} \hyperref[TEI.classCode]{classCode} \hyperref[TEI.creation]{creation} \hyperref[TEI.distributor]{distributor} \hyperref[TEI.edition]{edition} \hyperref[TEI.extent]{extent} \hyperref[TEI.funder]{funder} \hyperref[TEI.language]{language} \hyperref[TEI.licence]{licence} \hyperref[TEI.publicationStmt]{publicationStmt} \hyperref[TEI.rendition]{rendition}\par 
    \item[iso-fs: ]
   \hyperref[TEI.fDescr]{fDescr} \hyperref[TEI.fsDescr]{fsDescr}\par 
    \item[linking: ]
   \hyperref[TEI.ab]{ab} \hyperref[TEI.seg]{seg}\par 
    \item[msdescription: ]
   \hyperref[TEI.accMat]{accMat} \hyperref[TEI.acquisition]{acquisition} \hyperref[TEI.additions]{additions} \hyperref[TEI.catchwords]{catchwords} \hyperref[TEI.collation]{collation} \hyperref[TEI.colophon]{colophon} \hyperref[TEI.condition]{condition} \hyperref[TEI.custEvent]{custEvent} \hyperref[TEI.decoNote]{decoNote} \hyperref[TEI.explicit]{explicit} \hyperref[TEI.filiation]{filiation} \hyperref[TEI.finalRubric]{finalRubric} \hyperref[TEI.foliation]{foliation} \hyperref[TEI.heraldry]{heraldry} \hyperref[TEI.incipit]{incipit} \hyperref[TEI.layout]{layout} \hyperref[TEI.material]{material} \hyperref[TEI.musicNotation]{musicNotation} \hyperref[TEI.objectType]{objectType} \hyperref[TEI.origDate]{origDate} \hyperref[TEI.origPlace]{origPlace} \hyperref[TEI.origin]{origin} \hyperref[TEI.provenance]{provenance} \hyperref[TEI.rubric]{rubric} \hyperref[TEI.secFol]{secFol} \hyperref[TEI.signatures]{signatures} \hyperref[TEI.source]{source} \hyperref[TEI.stamp]{stamp} \hyperref[TEI.summary]{summary} \hyperref[TEI.support]{support} \hyperref[TEI.surrogates]{surrogates} \hyperref[TEI.typeNote]{typeNote} \hyperref[TEI.watermark]{watermark}\par 
    \item[namesdates: ]
   \hyperref[TEI.addName]{addName} \hyperref[TEI.affiliation]{affiliation} \hyperref[TEI.country]{country} \hyperref[TEI.forename]{forename} \hyperref[TEI.genName]{genName} \hyperref[TEI.geogName]{geogName} \hyperref[TEI.location]{location} \hyperref[TEI.nameLink]{nameLink} \hyperref[TEI.orgName]{orgName} \hyperref[TEI.persName]{persName} \hyperref[TEI.placeName]{placeName} \hyperref[TEI.region]{region} \hyperref[TEI.roleName]{roleName} \hyperref[TEI.settlement]{settlement} \hyperref[TEI.surname]{surname}\par 
    \item[textstructure: ]
   \hyperref[TEI.docAuthor]{docAuthor} \hyperref[TEI.docDate]{docDate} \hyperref[TEI.docEdition]{docEdition} \hyperref[TEI.titlePart]{titlePart}\par 
    \item[transcr: ]
   \hyperref[TEI.damage]{damage} \hyperref[TEI.fw]{fw} \hyperref[TEI.metamark]{metamark} \hyperref[TEI.mod]{mod} \hyperref[TEI.restore]{restore} \hyperref[TEI.retrace]{retrace} \hyperref[TEI.secl]{secl} \hyperref[TEI.supplied]{supplied} \hyperref[TEI.surplus]{surplus}
    \item[{Peut contenir}]
  
    \item[analysis: ]
   \hyperref[TEI.interp]{interp} \hyperref[TEI.interpGrp]{interpGrp} \hyperref[TEI.span]{span} \hyperref[TEI.spanGrp]{spanGrp}\par 
    \item[core: ]
   \hyperref[TEI.addrLine]{addrLine} \hyperref[TEI.cb]{cb} \hyperref[TEI.gap]{gap} \hyperref[TEI.gb]{gb} \hyperref[TEI.index]{index} \hyperref[TEI.lb]{lb} \hyperref[TEI.milestone]{milestone} \hyperref[TEI.name]{name} \hyperref[TEI.note]{note} \hyperref[TEI.pb]{pb} \hyperref[TEI.postBox]{postBox} \hyperref[TEI.postCode]{postCode} \hyperref[TEI.rs]{rs} \hyperref[TEI.street]{street}\par 
    \item[figures: ]
   \hyperref[TEI.figure]{figure} \hyperref[TEI.notatedMusic]{notatedMusic}\par 
    \item[header: ]
   \hyperref[TEI.idno]{idno}\par 
    \item[iso-fs: ]
   \hyperref[TEI.fLib]{fLib} \hyperref[TEI.fs]{fs} \hyperref[TEI.fvLib]{fvLib}\par 
    \item[linking: ]
   \hyperref[TEI.alt]{alt} \hyperref[TEI.altGrp]{altGrp} \hyperref[TEI.anchor]{anchor} \hyperref[TEI.join]{join} \hyperref[TEI.joinGrp]{joinGrp} \hyperref[TEI.link]{link} \hyperref[TEI.linkGrp]{linkGrp} \hyperref[TEI.timeline]{timeline}\par 
    \item[msdescription: ]
   \hyperref[TEI.source]{source}\par 
    \item[namesdates: ]
   \hyperref[TEI.addName]{addName} \hyperref[TEI.country]{country} \hyperref[TEI.forename]{forename} \hyperref[TEI.genName]{genName} \hyperref[TEI.geogName]{geogName} \hyperref[TEI.location]{location} \hyperref[TEI.nameLink]{nameLink} \hyperref[TEI.orgName]{orgName} \hyperref[TEI.persName]{persName} \hyperref[TEI.placeName]{placeName} \hyperref[TEI.region]{region} \hyperref[TEI.roleName]{roleName} \hyperref[TEI.settlement]{settlement} \hyperref[TEI.state]{state} \hyperref[TEI.surname]{surname}\par 
    \item[transcr: ]
   \hyperref[TEI.addSpan]{addSpan} \hyperref[TEI.damageSpan]{damageSpan} \hyperref[TEI.delSpan]{delSpan} \hyperref[TEI.fw]{fw} \hyperref[TEI.listTranspose]{listTranspose} \hyperref[TEI.metamark]{metamark} \hyperref[TEI.space]{space} \hyperref[TEI.substJoin]{substJoin}
    \item[{Note}]
  \par
Cet élément ne doit être utilisé que pour donner une adresse postale. A l'intérieur de cet élément, l'élément générique \hyperref[TEI.addrLine]{<addrLine>} peut être utilisé comme élément alternatif aux éléments plus spécialisés de la classe \textsf{model.addrPart} class, tels que \hyperref[TEI.street]{<street>}, \hyperref[TEI.postCode]{<postCode>} etc.
    \item[{Exemple}]
  \leavevmode\bgroup\exampleFont \begin{shaded}\noindent\mbox{}{<\textbf{address}>}\mbox{}\newline 
\hspace*{6pt}{<\textbf{addrLine}>}Centre d'Études Supérieures de la Renaissance{</\textbf{addrLine}>}\mbox{}\newline 
\hspace*{6pt}{<\textbf{addrLine}>}59, rue Néricault-Destouches{</\textbf{addrLine}>}\mbox{}\newline 
\hspace*{6pt}{<\textbf{addrLine}>} 37013\mbox{}\newline 
\hspace*{6pt}\hspace*{6pt} TOURS{</\textbf{addrLine}>}\mbox{}\newline 
\hspace*{6pt}{<\textbf{addrLine}>}France{</\textbf{addrLine}>}\mbox{}\newline 
{</\textbf{address}>}\end{shaded}\egroup 


    \item[{Exemple}]
  \leavevmode\bgroup\exampleFont \begin{shaded}\noindent\mbox{}{<\textbf{address}>}\mbox{}\newline 
\hspace*{6pt}{<\textbf{country}\hspace*{6pt}{key}="{FR}"/>}\mbox{}\newline 
\hspace*{6pt}{<\textbf{settlement}\hspace*{6pt}{type}="{city}">}Lyon{</\textbf{settlement}>}\mbox{}\newline 
\hspace*{6pt}{<\textbf{postCode}>}69002{</\textbf{postCode}>}\mbox{}\newline 
\hspace*{6pt}{<\textbf{district}\hspace*{6pt}{type}="{arrondissement}">}IIème{</\textbf{district}>}\mbox{}\newline 
\hspace*{6pt}{<\textbf{district}\hspace*{6pt}{type}="{quartier}">}Perrache{</\textbf{district}>}\mbox{}\newline 
\hspace*{6pt}{<\textbf{street}>}\mbox{}\newline 
\hspace*{6pt}\hspace*{6pt}{<\textbf{num}>}30{</\textbf{num}>}, Cours de Verdun{</\textbf{street}>}\mbox{}\newline 
{</\textbf{address}>}\end{shaded}\egroup 


    \item[{Modèle de contenu}]
  \mbox{}\hfill\\[-10pt]\begin{Verbatim}[fontsize=\small]
<content>
 <sequence maxOccurs="1" minOccurs="1">
  <classRef key="model.global"
   maxOccurs="unbounded" minOccurs="0"/>
  <sequence maxOccurs="unbounded"
   minOccurs="1">
   <classRef key="model.addrPart"/>
   <classRef key="model.global"
    maxOccurs="unbounded" minOccurs="0"/>
  </sequence>
 </sequence>
</content>
    
\end{Verbatim}

    \item[{Schéma Declaration}]
  \mbox{}\hfill\\[-10pt]\begin{Verbatim}[fontsize=\small]
element address
{
   tei_att.global.attributes,
   ( tei_model.global*, ( tei_model.addrPart, tei_model.global* )+ )
}
\end{Verbatim}

\end{reflist}  \index{adminInfo=<adminInfo>|oddindex}
\begin{reflist}
\item[]\begin{specHead}{TEI.adminInfo}{<adminInfo> }(informations administratives) contient, pour le manuscrit en cours de description, les informations sur son détenteur actuel, sur ses conditions d'accès et sur les modalités de sa description. [\xref{http://www.tei-c.org/release/doc/tei-p5-doc/en/html/MS.html\#msadad}{10.9.1. Administrative Information}]\end{specHead} 
    \item[{Module}]
  msdescription
    \item[{Attributs}]
  Attributs \hyperref[TEI.att.global]{att.global} (\textit{@xml:id}, \textit{@n}, \textit{@xml:lang}, \textit{@xml:base}, \textit{@xml:space})  (\hyperref[TEI.att.global.rendition]{att.global.rendition} (\textit{@rend}, \textit{@style}, \textit{@rendition})) (\hyperref[TEI.att.global.linking]{att.global.linking} (\textit{@corresp}, \textit{@synch}, \textit{@sameAs}, \textit{@copyOf}, \textit{@next}, \textit{@prev}, \textit{@exclude}, \textit{@select})) (\hyperref[TEI.att.global.analytic]{att.global.analytic} (\textit{@ana})) (\hyperref[TEI.att.global.facs]{att.global.facs} (\textit{@facs})) (\hyperref[TEI.att.global.change]{att.global.change} (\textit{@change})) (\hyperref[TEI.att.global.responsibility]{att.global.responsibility} (\textit{@cert}, \textit{@resp})) (\hyperref[TEI.att.global.source]{att.global.source} (\textit{@source}))
    \item[{Contenu dans}]
  
    \item[msdescription: ]
   \hyperref[TEI.additional]{additional}
    \item[{Peut contenir}]
  
    \item[core: ]
   \hyperref[TEI.note]{note}\par 
    \item[header: ]
   \hyperref[TEI.availability]{availability}\par 
    \item[msdescription: ]
   \hyperref[TEI.custodialHist]{custodialHist} \hyperref[TEI.recordHist]{recordHist}
    \item[{Exemple}]
  \leavevmode\bgroup\exampleFont \begin{shaded}\noindent\mbox{}{<\textbf{adminInfo}>}\mbox{}\newline 
\hspace*{6pt}{<\textbf{recordHist}>}\mbox{}\newline 
\hspace*{6pt}\hspace*{6pt}{<\textbf{source}>}Notice établie à partir du document original{</\textbf{source}>}\mbox{}\newline 
\hspace*{6pt}\hspace*{6pt}{<\textbf{change}\hspace*{6pt}{when}="{2009-10-05}"\hspace*{6pt}{who}="{Markova}">}Description mise à jour le {<\textbf{date}\hspace*{6pt}{type}="{crea}">}5\mbox{}\newline 
\hspace*{6pt}\hspace*{6pt}\hspace*{6pt}\hspace*{6pt}\hspace*{6pt}\hspace*{6pt} octobre 2009 {</\textbf{date}>}en vue de l'encodage en TEI des descriptions des reliure de la\mbox{}\newline 
\hspace*{6pt}\hspace*{6pt}\hspace*{6pt}\hspace*{6pt} Réserve des livres rares{</\textbf{change}>}\mbox{}\newline 
\hspace*{6pt}\hspace*{6pt}{<\textbf{change}\hspace*{6pt}{when}="{2009-06-01}"\hspace*{6pt}{who}="{Le Bars}">}Description revue le {<\textbf{date}\hspace*{6pt}{type}="{maj}">}1er juin\mbox{}\newline 
\hspace*{6pt}\hspace*{6pt}\hspace*{6pt}\hspace*{6pt}\hspace*{6pt}\hspace*{6pt} 2009 {</\textbf{date}>} par Fabienne Le Bars{</\textbf{change}>}\mbox{}\newline 
\hspace*{6pt}\hspace*{6pt}{<\textbf{change}\hspace*{6pt}{when}="{2009-06-25}"\hspace*{6pt}{who}="{Le Bars}">}Description validée le{<\textbf{date}\hspace*{6pt}{type}="{valid}">}25\mbox{}\newline 
\hspace*{6pt}\hspace*{6pt}\hspace*{6pt}\hspace*{6pt}\hspace*{6pt}\hspace*{6pt} juin 2009{</\textbf{date}>}par Fabienne Le Bars{</\textbf{change}>}\mbox{}\newline 
\hspace*{6pt}{</\textbf{recordHist}>}\mbox{}\newline 
{</\textbf{adminInfo}>}\end{shaded}\egroup 


    \item[{Modèle de contenu}]
  \mbox{}\hfill\\[-10pt]\begin{Verbatim}[fontsize=\small]
<content>
 <sequence maxOccurs="1" minOccurs="1">
  <elementRef key="recordHist"
   minOccurs="0"/>
  <elementRef key="availability"
   minOccurs="0"/>
  <elementRef key="custodialHist"
   minOccurs="0"/>
  <classRef key="model.noteLike"
   minOccurs="0"/>
 </sequence>
</content>
    
\end{Verbatim}

    \item[{Schéma Declaration}]
  \mbox{}\hfill\\[-10pt]\begin{Verbatim}[fontsize=\small]
element adminInfo
{
   tei_att.global.attributes,
   (
      tei_recordHist?,
      tei_availability?,
      tei_custodialHist?,
      tei_model.noteLike?
   )
}
\end{Verbatim}

\end{reflist}  \index{affiliation=<affiliation>|oddindex}\index{type=@type!<affiliation>|oddindex}
\begin{reflist}
\item[]\begin{specHead}{TEI.affiliation}{<affiliation> }(affiliation) contient une description non formalisée portant sur l'affiliation présente ou passée d'une personne à une organisation, par exemple un employeur ou un sponsor. [\xref{http://www.tei-c.org/release/doc/tei-p5-doc/en/html/CC.html\#CCAHPA}{15.2.2. The Participant Description}]\end{specHead} 
    \item[{Module}]
  namesdates
    \item[{Attributs}]
  Attributs \hyperref[TEI.att.global]{att.global} (\textit{@xml:id}, \textit{@n}, \textit{@xml:lang}, \textit{@xml:base}, \textit{@xml:space})  (\hyperref[TEI.att.global.rendition]{att.global.rendition} (\textit{@rend}, \textit{@style}, \textit{@rendition})) (\hyperref[TEI.att.global.linking]{att.global.linking} (\textit{@corresp}, \textit{@synch}, \textit{@sameAs}, \textit{@copyOf}, \textit{@next}, \textit{@prev}, \textit{@exclude}, \textit{@select})) (\hyperref[TEI.att.global.analytic]{att.global.analytic} (\textit{@ana})) (\hyperref[TEI.att.global.facs]{att.global.facs} (\textit{@facs})) (\hyperref[TEI.att.global.change]{att.global.change} (\textit{@change})) (\hyperref[TEI.att.global.responsibility]{att.global.responsibility} (\textit{@cert}, \textit{@resp})) (\hyperref[TEI.att.global.source]{att.global.source} (\textit{@source})) \hyperref[TEI.att.editLike]{att.editLike} (\textit{@evidence}, \textit{@instant})  (\hyperref[TEI.att.dimensions]{att.dimensions} (\textit{@unit}, \textit{@quantity}, \textit{@extent}, \textit{@precision}, \textit{@scope}) (\hyperref[TEI.att.ranging]{att.ranging} (\textit{@atLeast}, \textit{@atMost}, \textit{@min}, \textit{@max}, \textit{@confidence})) ) \hyperref[TEI.att.datable]{att.datable} (\textit{@calendar}, \textit{@period})  (\hyperref[TEI.att.datable.w3c]{att.datable.w3c} (\textit{@when}, \textit{@notBefore}, \textit{@notAfter}, \textit{@from}, \textit{@to})) (\hyperref[TEI.att.datable.iso]{att.datable.iso} (\textit{@when-iso}, \textit{@notBefore-iso}, \textit{@notAfter-iso}, \textit{@from-iso}, \textit{@to-iso})) (\hyperref[TEI.att.datable.custom]{att.datable.custom} (\textit{@when-custom}, \textit{@notBefore-custom}, \textit{@notAfter-custom}, \textit{@from-custom}, \textit{@to-custom}, \textit{@datingPoint}, \textit{@datingMethod})) \hyperref[TEI.att.naming]{att.naming} (\textit{@role}, \textit{@nymRef})  (\hyperref[TEI.att.canonical]{att.canonical} (\textit{@key}, \textit{@ref})) \hyperref[TEI.att.typed]{att.typed} (\unusedattribute{type}, @subtype) \hfil\\[-10pt]\begin{sansreflist}
    \item[@type]
  caractérise l'élément en utilisant n'importe quel système ou typologie de classification approprié.
\begin{reflist}
    \item[{Dérivé de}]
  \hyperref[TEI.att.typed]{att.typed}
    \item[{Statut}]
  Optionel
    \item[{Type de données}]
  \hyperref[TEI.teidata.enumerated]{teidata.enumerated}
    \item[{Exemple de valeurs possibles:}]
  \begin{description}

\item[{sponsor}]
\item[{recommend}]
\item[{discredit}]
\item[{pledged}]
\end{description} 
\end{reflist}  
\end{sansreflist}  
    \item[{Membre du}]
  \hyperref[TEI.model.addressLike]{model.addressLike} \hyperref[TEI.model.persStateLike]{model.persStateLike}
    \item[{Contenu dans}]
  
    \item[analysis: ]
   \hyperref[TEI.cl]{cl} \hyperref[TEI.phr]{phr} \hyperref[TEI.s]{s} \hyperref[TEI.span]{span}\par 
    \item[core: ]
   \hyperref[TEI.abbr]{abbr} \hyperref[TEI.add]{add} \hyperref[TEI.addrLine]{addrLine} \hyperref[TEI.author]{author} \hyperref[TEI.bibl]{bibl} \hyperref[TEI.biblScope]{biblScope} \hyperref[TEI.citedRange]{citedRange} \hyperref[TEI.corr]{corr} \hyperref[TEI.date]{date} \hyperref[TEI.del]{del} \hyperref[TEI.desc]{desc} \hyperref[TEI.distinct]{distinct} \hyperref[TEI.editor]{editor} \hyperref[TEI.email]{email} \hyperref[TEI.emph]{emph} \hyperref[TEI.expan]{expan} \hyperref[TEI.foreign]{foreign} \hyperref[TEI.gloss]{gloss} \hyperref[TEI.head]{head} \hyperref[TEI.headItem]{headItem} \hyperref[TEI.headLabel]{headLabel} \hyperref[TEI.hi]{hi} \hyperref[TEI.item]{item} \hyperref[TEI.l]{l} \hyperref[TEI.label]{label} \hyperref[TEI.measure]{measure} \hyperref[TEI.meeting]{meeting} \hyperref[TEI.mentioned]{mentioned} \hyperref[TEI.name]{name} \hyperref[TEI.note]{note} \hyperref[TEI.num]{num} \hyperref[TEI.orig]{orig} \hyperref[TEI.p]{p} \hyperref[TEI.pubPlace]{pubPlace} \hyperref[TEI.publisher]{publisher} \hyperref[TEI.q]{q} \hyperref[TEI.quote]{quote} \hyperref[TEI.ref]{ref} \hyperref[TEI.reg]{reg} \hyperref[TEI.resp]{resp} \hyperref[TEI.rs]{rs} \hyperref[TEI.said]{said} \hyperref[TEI.sic]{sic} \hyperref[TEI.soCalled]{soCalled} \hyperref[TEI.speaker]{speaker} \hyperref[TEI.stage]{stage} \hyperref[TEI.street]{street} \hyperref[TEI.term]{term} \hyperref[TEI.textLang]{textLang} \hyperref[TEI.time]{time} \hyperref[TEI.title]{title} \hyperref[TEI.unclear]{unclear}\par 
    \item[figures: ]
   \hyperref[TEI.cell]{cell} \hyperref[TEI.figDesc]{figDesc}\par 
    \item[header: ]
   \hyperref[TEI.authority]{authority} \hyperref[TEI.change]{change} \hyperref[TEI.classCode]{classCode} \hyperref[TEI.creation]{creation} \hyperref[TEI.distributor]{distributor} \hyperref[TEI.edition]{edition} \hyperref[TEI.extent]{extent} \hyperref[TEI.funder]{funder} \hyperref[TEI.language]{language} \hyperref[TEI.licence]{licence} \hyperref[TEI.rendition]{rendition}\par 
    \item[iso-fs: ]
   \hyperref[TEI.fDescr]{fDescr} \hyperref[TEI.fsDescr]{fsDescr}\par 
    \item[linking: ]
   \hyperref[TEI.ab]{ab} \hyperref[TEI.seg]{seg}\par 
    \item[msdescription: ]
   \hyperref[TEI.accMat]{accMat} \hyperref[TEI.acquisition]{acquisition} \hyperref[TEI.additions]{additions} \hyperref[TEI.catchwords]{catchwords} \hyperref[TEI.collation]{collation} \hyperref[TEI.colophon]{colophon} \hyperref[TEI.condition]{condition} \hyperref[TEI.custEvent]{custEvent} \hyperref[TEI.decoNote]{decoNote} \hyperref[TEI.explicit]{explicit} \hyperref[TEI.filiation]{filiation} \hyperref[TEI.finalRubric]{finalRubric} \hyperref[TEI.foliation]{foliation} \hyperref[TEI.heraldry]{heraldry} \hyperref[TEI.incipit]{incipit} \hyperref[TEI.layout]{layout} \hyperref[TEI.material]{material} \hyperref[TEI.musicNotation]{musicNotation} \hyperref[TEI.objectType]{objectType} \hyperref[TEI.origDate]{origDate} \hyperref[TEI.origPlace]{origPlace} \hyperref[TEI.origin]{origin} \hyperref[TEI.provenance]{provenance} \hyperref[TEI.rubric]{rubric} \hyperref[TEI.secFol]{secFol} \hyperref[TEI.signatures]{signatures} \hyperref[TEI.source]{source} \hyperref[TEI.stamp]{stamp} \hyperref[TEI.summary]{summary} \hyperref[TEI.support]{support} \hyperref[TEI.surrogates]{surrogates} \hyperref[TEI.typeNote]{typeNote} \hyperref[TEI.watermark]{watermark}\par 
    \item[namesdates: ]
   \hyperref[TEI.addName]{addName} \hyperref[TEI.affiliation]{affiliation} \hyperref[TEI.country]{country} \hyperref[TEI.forename]{forename} \hyperref[TEI.genName]{genName} \hyperref[TEI.geogName]{geogName} \hyperref[TEI.location]{location} \hyperref[TEI.nameLink]{nameLink} \hyperref[TEI.orgName]{orgName} \hyperref[TEI.persName]{persName} \hyperref[TEI.person]{person} \hyperref[TEI.personGrp]{personGrp} \hyperref[TEI.persona]{persona} \hyperref[TEI.placeName]{placeName} \hyperref[TEI.region]{region} \hyperref[TEI.roleName]{roleName} \hyperref[TEI.settlement]{settlement} \hyperref[TEI.surname]{surname}\par 
    \item[textstructure: ]
   \hyperref[TEI.docAuthor]{docAuthor} \hyperref[TEI.docDate]{docDate} \hyperref[TEI.docEdition]{docEdition} \hyperref[TEI.titlePart]{titlePart}\par 
    \item[transcr: ]
   \hyperref[TEI.damage]{damage} \hyperref[TEI.fw]{fw} \hyperref[TEI.metamark]{metamark} \hyperref[TEI.mod]{mod} \hyperref[TEI.restore]{restore} \hyperref[TEI.retrace]{retrace} \hyperref[TEI.secl]{secl} \hyperref[TEI.supplied]{supplied} \hyperref[TEI.surplus]{surplus}
    \item[{Peut contenir}]
  
    \item[analysis: ]
   \hyperref[TEI.c]{c} \hyperref[TEI.cl]{cl} \hyperref[TEI.interp]{interp} \hyperref[TEI.interpGrp]{interpGrp} \hyperref[TEI.m]{m} \hyperref[TEI.pc]{pc} \hyperref[TEI.phr]{phr} \hyperref[TEI.s]{s} \hyperref[TEI.span]{span} \hyperref[TEI.spanGrp]{spanGrp} \hyperref[TEI.w]{w}\par 
    \item[core: ]
   \hyperref[TEI.abbr]{abbr} \hyperref[TEI.add]{add} \hyperref[TEI.address]{address} \hyperref[TEI.binaryObject]{binaryObject} \hyperref[TEI.cb]{cb} \hyperref[TEI.choice]{choice} \hyperref[TEI.corr]{corr} \hyperref[TEI.date]{date} \hyperref[TEI.del]{del} \hyperref[TEI.distinct]{distinct} \hyperref[TEI.email]{email} \hyperref[TEI.emph]{emph} \hyperref[TEI.expan]{expan} \hyperref[TEI.foreign]{foreign} \hyperref[TEI.gap]{gap} \hyperref[TEI.gb]{gb} \hyperref[TEI.gloss]{gloss} \hyperref[TEI.graphic]{graphic} \hyperref[TEI.hi]{hi} \hyperref[TEI.index]{index} \hyperref[TEI.lb]{lb} \hyperref[TEI.measure]{measure} \hyperref[TEI.measureGrp]{measureGrp} \hyperref[TEI.media]{media} \hyperref[TEI.mentioned]{mentioned} \hyperref[TEI.milestone]{milestone} \hyperref[TEI.name]{name} \hyperref[TEI.note]{note} \hyperref[TEI.num]{num} \hyperref[TEI.orig]{orig} \hyperref[TEI.pb]{pb} \hyperref[TEI.ptr]{ptr} \hyperref[TEI.ref]{ref} \hyperref[TEI.reg]{reg} \hyperref[TEI.rs]{rs} \hyperref[TEI.sic]{sic} \hyperref[TEI.soCalled]{soCalled} \hyperref[TEI.term]{term} \hyperref[TEI.time]{time} \hyperref[TEI.title]{title} \hyperref[TEI.unclear]{unclear}\par 
    \item[derived-module-tei.istex: ]
   \hyperref[TEI.math]{math} \hyperref[TEI.mrow]{mrow}\par 
    \item[figures: ]
   \hyperref[TEI.figure]{figure} \hyperref[TEI.formula]{formula} \hyperref[TEI.notatedMusic]{notatedMusic}\par 
    \item[header: ]
   \hyperref[TEI.idno]{idno}\par 
    \item[iso-fs: ]
   \hyperref[TEI.fLib]{fLib} \hyperref[TEI.fs]{fs} \hyperref[TEI.fvLib]{fvLib}\par 
    \item[linking: ]
   \hyperref[TEI.alt]{alt} \hyperref[TEI.altGrp]{altGrp} \hyperref[TEI.anchor]{anchor} \hyperref[TEI.join]{join} \hyperref[TEI.joinGrp]{joinGrp} \hyperref[TEI.link]{link} \hyperref[TEI.linkGrp]{linkGrp} \hyperref[TEI.seg]{seg} \hyperref[TEI.timeline]{timeline}\par 
    \item[msdescription: ]
   \hyperref[TEI.catchwords]{catchwords} \hyperref[TEI.depth]{depth} \hyperref[TEI.dim]{dim} \hyperref[TEI.dimensions]{dimensions} \hyperref[TEI.height]{height} \hyperref[TEI.heraldry]{heraldry} \hyperref[TEI.locus]{locus} \hyperref[TEI.locusGrp]{locusGrp} \hyperref[TEI.material]{material} \hyperref[TEI.objectType]{objectType} \hyperref[TEI.origDate]{origDate} \hyperref[TEI.origPlace]{origPlace} \hyperref[TEI.secFol]{secFol} \hyperref[TEI.signatures]{signatures} \hyperref[TEI.source]{source} \hyperref[TEI.stamp]{stamp} \hyperref[TEI.watermark]{watermark} \hyperref[TEI.width]{width}\par 
    \item[namesdates: ]
   \hyperref[TEI.addName]{addName} \hyperref[TEI.affiliation]{affiliation} \hyperref[TEI.country]{country} \hyperref[TEI.forename]{forename} \hyperref[TEI.genName]{genName} \hyperref[TEI.geogName]{geogName} \hyperref[TEI.location]{location} \hyperref[TEI.nameLink]{nameLink} \hyperref[TEI.orgName]{orgName} \hyperref[TEI.persName]{persName} \hyperref[TEI.placeName]{placeName} \hyperref[TEI.region]{region} \hyperref[TEI.roleName]{roleName} \hyperref[TEI.settlement]{settlement} \hyperref[TEI.state]{state} \hyperref[TEI.surname]{surname}\par 
    \item[spoken: ]
   \hyperref[TEI.annotationBlock]{annotationBlock}\par 
    \item[transcr: ]
   \hyperref[TEI.addSpan]{addSpan} \hyperref[TEI.am]{am} \hyperref[TEI.damage]{damage} \hyperref[TEI.damageSpan]{damageSpan} \hyperref[TEI.delSpan]{delSpan} \hyperref[TEI.ex]{ex} \hyperref[TEI.fw]{fw} \hyperref[TEI.handShift]{handShift} \hyperref[TEI.listTranspose]{listTranspose} \hyperref[TEI.metamark]{metamark} \hyperref[TEI.mod]{mod} \hyperref[TEI.redo]{redo} \hyperref[TEI.restore]{restore} \hyperref[TEI.retrace]{retrace} \hyperref[TEI.secl]{secl} \hyperref[TEI.space]{space} \hyperref[TEI.subst]{subst} \hyperref[TEI.substJoin]{substJoin} \hyperref[TEI.supplied]{supplied} \hyperref[TEI.surplus]{surplus} \hyperref[TEI.undo]{undo}\par des données textuelles
    \item[{Note}]
  \par
S'il est présent, le nom d'une organisation peut être balisé en utilisant soit l'élément \hyperref[TEI.name]{<name>} comme ci-dessus, soit l'élément plus spécifique \hyperref[TEI.orgName]{<orgName>}.
    \item[{Exemple}]
  \leavevmode\bgroup\exampleFont \begin{shaded}\noindent\mbox{}{<\textbf{affiliation}>}associé étranger de {<\textbf{name}\hspace*{6pt}{type}="{org}">}l'Académie des Inscriptions et\mbox{}\newline 
\hspace*{6pt}\hspace*{6pt} Belles-Lettres{</\textbf{name}>}\mbox{}\newline 
{</\textbf{affiliation}>}\mbox{}\newline 
{<\textbf{affiliation}\hspace*{6pt}{notAfter}="{1960-01-01}"\mbox{}\newline 
\hspace*{6pt}{notBefore}="{1957-02-28}">}Chargé de cours, puis professeur\mbox{}\newline 
 d’archéologie (1949-1981) et doyen (1958-1961) {<\textbf{orgName}>}à la Faculté des lettres\mbox{}\newline 
\hspace*{6pt}\hspace*{6pt} d’Ankara{</\textbf{orgName}>}.{</\textbf{affiliation}>}\end{shaded}\egroup 


    \item[{Modèle de contenu}]
  \mbox{}\hfill\\[-10pt]\begin{Verbatim}[fontsize=\small]
<content>
 <macroRef key="macro.phraseSeq"/>
</content>
    
\end{Verbatim}

    \item[{Schéma Declaration}]
  \mbox{}\hfill\\[-10pt]\begin{Verbatim}[fontsize=\small]
element affiliation
{
   tei_att.global.attributes,
   tei_att.editLike.attributes,
   tei_att.datable.attributes,
   tei_att.naming.attributes,
   tei_att.typed.attribute.subtype,
   attribute type { text }?,
   tei_macro.phraseSeq}
\end{Verbatim}

\end{reflist}  \index{alt=<alt>|oddindex}\index{target=@target!<alt>|oddindex}\index{mode=@mode!<alt>|oddindex}\index{weights=@weights!<alt>|oddindex}
\begin{reflist}
\item[]\begin{specHead}{TEI.alt}{<alt> }(alternative) identifie une alternative ou un ensemble d'options entre des éléments ou des passages. [\xref{http://www.tei-c.org/release/doc/tei-p5-doc/en/html/SA.html\#SAAT}{16.8. Alternation}]\end{specHead} 
    \item[{Module}]
  linking
    \item[{Attributs}]
  Attributs \hyperref[TEI.att.global]{att.global} (\textit{@xml:id}, \textit{@n}, \textit{@xml:lang}, \textit{@xml:base}, \textit{@xml:space})  (\hyperref[TEI.att.global.rendition]{att.global.rendition} (\textit{@rend}, \textit{@style}, \textit{@rendition})) (\hyperref[TEI.att.global.linking]{att.global.linking} (\textit{@corresp}, \textit{@synch}, \textit{@sameAs}, \textit{@copyOf}, \textit{@next}, \textit{@prev}, \textit{@exclude}, \textit{@select})) (\hyperref[TEI.att.global.analytic]{att.global.analytic} (\textit{@ana})) (\hyperref[TEI.att.global.facs]{att.global.facs} (\textit{@facs})) (\hyperref[TEI.att.global.change]{att.global.change} (\textit{@change})) (\hyperref[TEI.att.global.responsibility]{att.global.responsibility} (\textit{@cert}, \textit{@resp})) (\hyperref[TEI.att.global.source]{att.global.source} (\textit{@source})) \hyperref[TEI.att.typed]{att.typed} (\textit{@type}, \textit{@subtype}) \hyperref[TEI.att.pointing]{att.pointing} (\unusedattribute{target}, @targetLang, @evaluate) \hfil\\[-10pt]\begin{sansreflist}
    \item[@target]
  précise la cible de la référence en donnant une ou plusieurs références URI
\begin{reflist}
    \item[{Dérivé de}]
  \hyperref[TEI.att.pointing]{att.pointing}
    \item[{Statut}]
  Optionel
    \item[{Type de données}]
  2–∞ occurrences de \hyperref[TEI.teidata.pointer]{teidata.pointer} séparé par un espace
\end{reflist}  
    \item[@mode]
  établit si les alternatives rassemblées dans cette collection sont exclusives ou non.
\begin{reflist}
    \item[{Statut}]
  Recommendé
    \item[{Type de données}]
  \hyperref[TEI.teidata.enumerated]{teidata.enumerated}
    \item[{Les valeurs autorisées sont:}]
  \begin{description}

\item[{excl}](exclusif) indique que l'alternative est exclusive, c'est-à-dire, qu'une seule des options proposées est possible.
\item[{incl}](non exclusif) Indique que cette alternative n'est pas exclusive, c'est-à-dire qu'une option au moins est vraie.
\end{description} 
\end{reflist}  
    \item[@weights]
  Si l'attribut {\itshape mode} a la valeur \texttt{excl}, chacune des valeurs de l'attribut {\itshape weights} établit la probabilité que l'option correspondante soit vraie. Si l'attribut {\itshape mode} a la valeur \texttt{incl}, chacune des valeurs de l'attribut {\itshape weights} établit la probabilité que l'option correspondante soit vraie, étant posé qu'au moins une des autres options l'est aussi.
\begin{reflist}
    \item[{Statut}]
  Optionel
    \item[{Type de données}]
  2–∞ occurrences de \hyperref[TEI.teidata.probability]{teidata.probability} séparé par un espace
    \item[{Note}]
  \par
Si l'attribut {\itshape mode} a la valeur \texttt{excl}, la somme des poids doit être égale à 1. Si l'attribut {\itshape mode} a la valeur \texttt{incl}, la somme des poids doit se situer entre 0 et le nombre des alternants.
\end{reflist}  
\end{sansreflist}  
    \item[{Membre du}]
  \hyperref[TEI.model.global.meta]{model.global.meta}
    \item[{Contenu dans}]
  
    \item[analysis: ]
   \hyperref[TEI.cl]{cl} \hyperref[TEI.m]{m} \hyperref[TEI.phr]{phr} \hyperref[TEI.s]{s} \hyperref[TEI.span]{span} \hyperref[TEI.w]{w}\par 
    \item[core: ]
   \hyperref[TEI.abbr]{abbr} \hyperref[TEI.add]{add} \hyperref[TEI.addrLine]{addrLine} \hyperref[TEI.address]{address} \hyperref[TEI.author]{author} \hyperref[TEI.bibl]{bibl} \hyperref[TEI.biblScope]{biblScope} \hyperref[TEI.cit]{cit} \hyperref[TEI.citedRange]{citedRange} \hyperref[TEI.corr]{corr} \hyperref[TEI.date]{date} \hyperref[TEI.del]{del} \hyperref[TEI.distinct]{distinct} \hyperref[TEI.editor]{editor} \hyperref[TEI.email]{email} \hyperref[TEI.emph]{emph} \hyperref[TEI.expan]{expan} \hyperref[TEI.foreign]{foreign} \hyperref[TEI.gloss]{gloss} \hyperref[TEI.head]{head} \hyperref[TEI.headItem]{headItem} \hyperref[TEI.headLabel]{headLabel} \hyperref[TEI.hi]{hi} \hyperref[TEI.imprint]{imprint} \hyperref[TEI.item]{item} \hyperref[TEI.l]{l} \hyperref[TEI.label]{label} \hyperref[TEI.lg]{lg} \hyperref[TEI.list]{list} \hyperref[TEI.measure]{measure} \hyperref[TEI.mentioned]{mentioned} \hyperref[TEI.name]{name} \hyperref[TEI.note]{note} \hyperref[TEI.num]{num} \hyperref[TEI.orig]{orig} \hyperref[TEI.p]{p} \hyperref[TEI.pubPlace]{pubPlace} \hyperref[TEI.publisher]{publisher} \hyperref[TEI.q]{q} \hyperref[TEI.quote]{quote} \hyperref[TEI.ref]{ref} \hyperref[TEI.reg]{reg} \hyperref[TEI.resp]{resp} \hyperref[TEI.rs]{rs} \hyperref[TEI.said]{said} \hyperref[TEI.series]{series} \hyperref[TEI.sic]{sic} \hyperref[TEI.soCalled]{soCalled} \hyperref[TEI.sp]{sp} \hyperref[TEI.speaker]{speaker} \hyperref[TEI.stage]{stage} \hyperref[TEI.street]{street} \hyperref[TEI.term]{term} \hyperref[TEI.textLang]{textLang} \hyperref[TEI.time]{time} \hyperref[TEI.title]{title} \hyperref[TEI.unclear]{unclear}\par 
    \item[figures: ]
   \hyperref[TEI.cell]{cell} \hyperref[TEI.figure]{figure} \hyperref[TEI.table]{table}\par 
    \item[header: ]
   \hyperref[TEI.authority]{authority} \hyperref[TEI.change]{change} \hyperref[TEI.classCode]{classCode} \hyperref[TEI.distributor]{distributor} \hyperref[TEI.edition]{edition} \hyperref[TEI.extent]{extent} \hyperref[TEI.funder]{funder} \hyperref[TEI.language]{language} \hyperref[TEI.licence]{licence}\par 
    \item[linking: ]
   \hyperref[TEI.ab]{ab} \hyperref[TEI.altGrp]{altGrp} \hyperref[TEI.seg]{seg}\par 
    \item[msdescription: ]
   \hyperref[TEI.accMat]{accMat} \hyperref[TEI.acquisition]{acquisition} \hyperref[TEI.additions]{additions} \hyperref[TEI.catchwords]{catchwords} \hyperref[TEI.collation]{collation} \hyperref[TEI.colophon]{colophon} \hyperref[TEI.condition]{condition} \hyperref[TEI.custEvent]{custEvent} \hyperref[TEI.decoNote]{decoNote} \hyperref[TEI.explicit]{explicit} \hyperref[TEI.filiation]{filiation} \hyperref[TEI.finalRubric]{finalRubric} \hyperref[TEI.foliation]{foliation} \hyperref[TEI.heraldry]{heraldry} \hyperref[TEI.incipit]{incipit} \hyperref[TEI.layout]{layout} \hyperref[TEI.material]{material} \hyperref[TEI.msItem]{msItem} \hyperref[TEI.musicNotation]{musicNotation} \hyperref[TEI.objectType]{objectType} \hyperref[TEI.origDate]{origDate} \hyperref[TEI.origPlace]{origPlace} \hyperref[TEI.origin]{origin} \hyperref[TEI.provenance]{provenance} \hyperref[TEI.rubric]{rubric} \hyperref[TEI.secFol]{secFol} \hyperref[TEI.signatures]{signatures} \hyperref[TEI.source]{source} \hyperref[TEI.stamp]{stamp} \hyperref[TEI.summary]{summary} \hyperref[TEI.support]{support} \hyperref[TEI.surrogates]{surrogates} \hyperref[TEI.typeNote]{typeNote} \hyperref[TEI.watermark]{watermark}\par 
    \item[namesdates: ]
   \hyperref[TEI.addName]{addName} \hyperref[TEI.affiliation]{affiliation} \hyperref[TEI.country]{country} \hyperref[TEI.forename]{forename} \hyperref[TEI.genName]{genName} \hyperref[TEI.geogName]{geogName} \hyperref[TEI.nameLink]{nameLink} \hyperref[TEI.orgName]{orgName} \hyperref[TEI.persName]{persName} \hyperref[TEI.person]{person} \hyperref[TEI.personGrp]{personGrp} \hyperref[TEI.persona]{persona} \hyperref[TEI.placeName]{placeName} \hyperref[TEI.region]{region} \hyperref[TEI.roleName]{roleName} \hyperref[TEI.settlement]{settlement} \hyperref[TEI.surname]{surname}\par 
    \item[spoken: ]
   \hyperref[TEI.annotationBlock]{annotationBlock}\par 
    \item[standOff: ]
   \hyperref[TEI.listAnnotation]{listAnnotation}\par 
    \item[textstructure: ]
   \hyperref[TEI.back]{back} \hyperref[TEI.body]{body} \hyperref[TEI.div]{div} \hyperref[TEI.docAuthor]{docAuthor} \hyperref[TEI.docDate]{docDate} \hyperref[TEI.docEdition]{docEdition} \hyperref[TEI.docTitle]{docTitle} \hyperref[TEI.floatingText]{floatingText} \hyperref[TEI.front]{front} \hyperref[TEI.group]{group} \hyperref[TEI.text]{text} \hyperref[TEI.titlePage]{titlePage} \hyperref[TEI.titlePart]{titlePart}\par 
    \item[transcr: ]
   \hyperref[TEI.damage]{damage} \hyperref[TEI.fw]{fw} \hyperref[TEI.line]{line} \hyperref[TEI.metamark]{metamark} \hyperref[TEI.mod]{mod} \hyperref[TEI.restore]{restore} \hyperref[TEI.retrace]{retrace} \hyperref[TEI.secl]{secl} \hyperref[TEI.sourceDoc]{sourceDoc} \hyperref[TEI.supplied]{supplied} \hyperref[TEI.surface]{surface} \hyperref[TEI.surfaceGrp]{surfaceGrp} \hyperref[TEI.surplus]{surplus} \hyperref[TEI.zone]{zone}
    \item[{Peut contenir}]
  Elément vide
    \item[{Exemple}]
  \leavevmode\bgroup\exampleFont \begin{shaded}\noindent\mbox{}{<\textbf{alt}\hspace*{6pt}{mode}="{excl}"\hspace*{6pt}{target}="{\#we.fun \#we.sun}"\mbox{}\newline 
\hspace*{6pt}{weights}="{0.5 0.5}"/>}\end{shaded}\egroup 


    \item[{Exemple}]
  \leavevmode\bgroup\exampleFont \begin{shaded}\noindent\mbox{}{<\textbf{alt}\hspace*{6pt}{mode}="{excl}"\mbox{}\newline 
\hspace*{6pt}{target}="{\#fr\textunderscore we.fun \#fr\textunderscore we.sun}"\hspace*{6pt}{weights}="{0.5 0.5}"/>}\end{shaded}\egroup 


    \item[{Modèle de contenu}]
  \fbox{\ttfamily <content>\newline
</content>\newline
    } 
    \item[{Schéma Declaration}]
  \mbox{}\hfill\\[-10pt]\begin{Verbatim}[fontsize=\small]
element alt
{
   tei_att.global.attributes,
   tei_att.pointing.attribute.targetLang,
   tei_att.pointing.attribute.evaluate,
   tei_att.typed.attributes,
   attribute target { list { * } }?,
   attribute mode { "excl" | "incl" }?,
   attribute weights { list { * } }?,
   empty
}
\end{Verbatim}

\end{reflist}  \index{altGrp=<altGrp>|oddindex}\index{mode=@mode!<altGrp>|oddindex}
\begin{reflist}
\item[]\begin{specHead}{TEI.altGrp}{<altGrp> }(groupe d'alternatives) regroupe une collection d'éléments \hyperref[TEI.alt]{<alt>} et, éventuellement, de pointeurs. [\xref{http://www.tei-c.org/release/doc/tei-p5-doc/en/html/SA.html\#SAAT}{16.8. Alternation}]\end{specHead} 
    \item[{Module}]
  linking
    \item[{Attributs}]
  Attributs \hyperref[TEI.att.global]{att.global} (\textit{@xml:id}, \textit{@n}, \textit{@xml:lang}, \textit{@xml:base}, \textit{@xml:space})  (\hyperref[TEI.att.global.rendition]{att.global.rendition} (\textit{@rend}, \textit{@style}, \textit{@rendition})) (\hyperref[TEI.att.global.linking]{att.global.linking} (\textit{@corresp}, \textit{@synch}, \textit{@sameAs}, \textit{@copyOf}, \textit{@next}, \textit{@prev}, \textit{@exclude}, \textit{@select})) (\hyperref[TEI.att.global.analytic]{att.global.analytic} (\textit{@ana})) (\hyperref[TEI.att.global.facs]{att.global.facs} (\textit{@facs})) (\hyperref[TEI.att.global.change]{att.global.change} (\textit{@change})) (\hyperref[TEI.att.global.responsibility]{att.global.responsibility} (\textit{@cert}, \textit{@resp})) (\hyperref[TEI.att.global.source]{att.global.source} (\textit{@source})) \hyperref[TEI.att.pointing.group]{att.pointing.group} (\textit{@domains}, \textit{@targFunc})  (\hyperref[TEI.att.pointing]{att.pointing} (\textit{@targetLang}, \textit{@target}, \textit{@evaluate})) (\hyperref[TEI.att.typed]{att.typed} (\textit{@type}, \textit{@subtype})) \hfil\\[-10pt]\begin{sansreflist}
    \item[@mode]
  établit si les alternatives rassemblées dans cette collection sont exclusives ou non.
\begin{reflist}
    \item[{Statut}]
  Optionel
    \item[{Type de données}]
  \hyperref[TEI.teidata.enumerated]{teidata.enumerated}
    \item[{Les valeurs autorisées sont:}]
  \begin{description}

\item[{excl}](exclusif) indique que l'alternative est exclusive, c'est-à-dire qu'une seule des alternatives proposées est possible.{[Valeur par défaut] }
\item[{incl}](non exclusif) indique que l'alternative n'est pas exclusive, c'est-à-dire qu'une alternative au moins est vraie.
\end{description} 
\end{reflist}  
\end{sansreflist}  
    \item[{Membre du}]
  \hyperref[TEI.model.global.meta]{model.global.meta}
    \item[{Contenu dans}]
  
    \item[analysis: ]
   \hyperref[TEI.cl]{cl} \hyperref[TEI.m]{m} \hyperref[TEI.phr]{phr} \hyperref[TEI.s]{s} \hyperref[TEI.span]{span} \hyperref[TEI.w]{w}\par 
    \item[core: ]
   \hyperref[TEI.abbr]{abbr} \hyperref[TEI.add]{add} \hyperref[TEI.addrLine]{addrLine} \hyperref[TEI.address]{address} \hyperref[TEI.author]{author} \hyperref[TEI.bibl]{bibl} \hyperref[TEI.biblScope]{biblScope} \hyperref[TEI.cit]{cit} \hyperref[TEI.citedRange]{citedRange} \hyperref[TEI.corr]{corr} \hyperref[TEI.date]{date} \hyperref[TEI.del]{del} \hyperref[TEI.distinct]{distinct} \hyperref[TEI.editor]{editor} \hyperref[TEI.email]{email} \hyperref[TEI.emph]{emph} \hyperref[TEI.expan]{expan} \hyperref[TEI.foreign]{foreign} \hyperref[TEI.gloss]{gloss} \hyperref[TEI.head]{head} \hyperref[TEI.headItem]{headItem} \hyperref[TEI.headLabel]{headLabel} \hyperref[TEI.hi]{hi} \hyperref[TEI.imprint]{imprint} \hyperref[TEI.item]{item} \hyperref[TEI.l]{l} \hyperref[TEI.label]{label} \hyperref[TEI.lg]{lg} \hyperref[TEI.list]{list} \hyperref[TEI.measure]{measure} \hyperref[TEI.mentioned]{mentioned} \hyperref[TEI.name]{name} \hyperref[TEI.note]{note} \hyperref[TEI.num]{num} \hyperref[TEI.orig]{orig} \hyperref[TEI.p]{p} \hyperref[TEI.pubPlace]{pubPlace} \hyperref[TEI.publisher]{publisher} \hyperref[TEI.q]{q} \hyperref[TEI.quote]{quote} \hyperref[TEI.ref]{ref} \hyperref[TEI.reg]{reg} \hyperref[TEI.resp]{resp} \hyperref[TEI.rs]{rs} \hyperref[TEI.said]{said} \hyperref[TEI.series]{series} \hyperref[TEI.sic]{sic} \hyperref[TEI.soCalled]{soCalled} \hyperref[TEI.sp]{sp} \hyperref[TEI.speaker]{speaker} \hyperref[TEI.stage]{stage} \hyperref[TEI.street]{street} \hyperref[TEI.term]{term} \hyperref[TEI.textLang]{textLang} \hyperref[TEI.time]{time} \hyperref[TEI.title]{title} \hyperref[TEI.unclear]{unclear}\par 
    \item[figures: ]
   \hyperref[TEI.cell]{cell} \hyperref[TEI.figure]{figure} \hyperref[TEI.table]{table}\par 
    \item[header: ]
   \hyperref[TEI.authority]{authority} \hyperref[TEI.change]{change} \hyperref[TEI.classCode]{classCode} \hyperref[TEI.distributor]{distributor} \hyperref[TEI.edition]{edition} \hyperref[TEI.extent]{extent} \hyperref[TEI.funder]{funder} \hyperref[TEI.language]{language} \hyperref[TEI.licence]{licence}\par 
    \item[linking: ]
   \hyperref[TEI.ab]{ab} \hyperref[TEI.seg]{seg}\par 
    \item[msdescription: ]
   \hyperref[TEI.accMat]{accMat} \hyperref[TEI.acquisition]{acquisition} \hyperref[TEI.additions]{additions} \hyperref[TEI.catchwords]{catchwords} \hyperref[TEI.collation]{collation} \hyperref[TEI.colophon]{colophon} \hyperref[TEI.condition]{condition} \hyperref[TEI.custEvent]{custEvent} \hyperref[TEI.decoNote]{decoNote} \hyperref[TEI.explicit]{explicit} \hyperref[TEI.filiation]{filiation} \hyperref[TEI.finalRubric]{finalRubric} \hyperref[TEI.foliation]{foliation} \hyperref[TEI.heraldry]{heraldry} \hyperref[TEI.incipit]{incipit} \hyperref[TEI.layout]{layout} \hyperref[TEI.material]{material} \hyperref[TEI.msItem]{msItem} \hyperref[TEI.musicNotation]{musicNotation} \hyperref[TEI.objectType]{objectType} \hyperref[TEI.origDate]{origDate} \hyperref[TEI.origPlace]{origPlace} \hyperref[TEI.origin]{origin} \hyperref[TEI.provenance]{provenance} \hyperref[TEI.rubric]{rubric} \hyperref[TEI.secFol]{secFol} \hyperref[TEI.signatures]{signatures} \hyperref[TEI.source]{source} \hyperref[TEI.stamp]{stamp} \hyperref[TEI.summary]{summary} \hyperref[TEI.support]{support} \hyperref[TEI.surrogates]{surrogates} \hyperref[TEI.typeNote]{typeNote} \hyperref[TEI.watermark]{watermark}\par 
    \item[namesdates: ]
   \hyperref[TEI.addName]{addName} \hyperref[TEI.affiliation]{affiliation} \hyperref[TEI.country]{country} \hyperref[TEI.forename]{forename} \hyperref[TEI.genName]{genName} \hyperref[TEI.geogName]{geogName} \hyperref[TEI.nameLink]{nameLink} \hyperref[TEI.orgName]{orgName} \hyperref[TEI.persName]{persName} \hyperref[TEI.person]{person} \hyperref[TEI.personGrp]{personGrp} \hyperref[TEI.persona]{persona} \hyperref[TEI.placeName]{placeName} \hyperref[TEI.region]{region} \hyperref[TEI.roleName]{roleName} \hyperref[TEI.settlement]{settlement} \hyperref[TEI.surname]{surname}\par 
    \item[spoken: ]
   \hyperref[TEI.annotationBlock]{annotationBlock}\par 
    \item[standOff: ]
   \hyperref[TEI.listAnnotation]{listAnnotation}\par 
    \item[textstructure: ]
   \hyperref[TEI.back]{back} \hyperref[TEI.body]{body} \hyperref[TEI.div]{div} \hyperref[TEI.docAuthor]{docAuthor} \hyperref[TEI.docDate]{docDate} \hyperref[TEI.docEdition]{docEdition} \hyperref[TEI.docTitle]{docTitle} \hyperref[TEI.floatingText]{floatingText} \hyperref[TEI.front]{front} \hyperref[TEI.group]{group} \hyperref[TEI.text]{text} \hyperref[TEI.titlePage]{titlePage} \hyperref[TEI.titlePart]{titlePart}\par 
    \item[transcr: ]
   \hyperref[TEI.damage]{damage} \hyperref[TEI.fw]{fw} \hyperref[TEI.line]{line} \hyperref[TEI.metamark]{metamark} \hyperref[TEI.mod]{mod} \hyperref[TEI.restore]{restore} \hyperref[TEI.retrace]{retrace} \hyperref[TEI.secl]{secl} \hyperref[TEI.sourceDoc]{sourceDoc} \hyperref[TEI.supplied]{supplied} \hyperref[TEI.surface]{surface} \hyperref[TEI.surfaceGrp]{surfaceGrp} \hyperref[TEI.surplus]{surplus} \hyperref[TEI.zone]{zone}
    \item[{Peut contenir}]
  
    \item[core: ]
   \hyperref[TEI.ptr]{ptr}\par 
    \item[linking: ]
   \hyperref[TEI.alt]{alt}
    \item[{Note}]
  \par
Un nombre quelconque d'éléments alternatifs, de pointeurs et de pointeurs étendus.
    \item[{Exemple}]
  \leavevmode\bgroup\exampleFont \begin{shaded}\noindent\mbox{}{<\textbf{altGrp}\hspace*{6pt}{mode}="{excl}">}\mbox{}\newline 
\hspace*{6pt}{<\textbf{alt}\hspace*{6pt}{target}="{\#dm \#lt \#bb}"\mbox{}\newline 
\hspace*{6pt}\hspace*{6pt}{weights}="{0.5 0.25 0.25}"/>}\mbox{}\newline 
\hspace*{6pt}{<\textbf{alt}\hspace*{6pt}{target}="{\#rl \#db}"\hspace*{6pt}{weights}="{0.5 0.5}"/>}\mbox{}\newline 
{</\textbf{altGrp}>}\end{shaded}\egroup 


    \item[{Exemple}]
  \leavevmode\bgroup\exampleFont \begin{shaded}\noindent\mbox{}{<\textbf{altGrp}\hspace*{6pt}{mode}="{excl}">}\mbox{}\newline 
\hspace*{6pt}{<\textbf{alt}\hspace*{6pt}{target}="{\#fr\textunderscore dm \#fr\textunderscore lt \#fr\textunderscore bb}"\mbox{}\newline 
\hspace*{6pt}\hspace*{6pt}{weights}="{0.5 0.25 0.25}"/>}\mbox{}\newline 
\hspace*{6pt}{<\textbf{alt}\hspace*{6pt}{target}="{\#fr\textunderscore rl \#fr\textunderscore db}"\mbox{}\newline 
\hspace*{6pt}\hspace*{6pt}{weights}="{0.5 0.5}"/>}\mbox{}\newline 
{</\textbf{altGrp}>}\end{shaded}\egroup 


    \item[{Exemple}]
  \leavevmode\bgroup\exampleFont \begin{shaded}\noindent\mbox{}{<\textbf{altGrp}\hspace*{6pt}{mode}="{incl}">}\mbox{}\newline 
\hspace*{6pt}{<\textbf{alt}\hspace*{6pt}{target}="{\#fr\textunderscore dm \#fr\textunderscore rl}"\mbox{}\newline 
\hspace*{6pt}\hspace*{6pt}{weights}="{0.90 0.90}"/>}\mbox{}\newline 
\hspace*{6pt}{<\textbf{alt}\hspace*{6pt}{target}="{\#fr\textunderscore lt \#fr\textunderscore rl}"\mbox{}\newline 
\hspace*{6pt}\hspace*{6pt}{weights}="{0.5 0.5}"/>}\mbox{}\newline 
\hspace*{6pt}{<\textbf{alt}\hspace*{6pt}{target}="{\#fr\textunderscore bb \#fr\textunderscore rl}"\mbox{}\newline 
\hspace*{6pt}\hspace*{6pt}{weights}="{0.5 0.5}"/>}\mbox{}\newline 
\hspace*{6pt}{<\textbf{alt}\hspace*{6pt}{target}="{\#fr\textunderscore dm \#fr\textunderscore db}"\mbox{}\newline 
\hspace*{6pt}\hspace*{6pt}{weights}="{0.10 0.10}"/>}\mbox{}\newline 
\hspace*{6pt}{<\textbf{alt}\hspace*{6pt}{target}="{\#fr\textunderscore lt \#fr\textunderscore db}"\mbox{}\newline 
\hspace*{6pt}\hspace*{6pt}{weights}="{0.45 0.90}"/>}\mbox{}\newline 
\hspace*{6pt}{<\textbf{alt}\hspace*{6pt}{target}="{\#fr\textunderscore bb \#fr\textunderscore db}"\mbox{}\newline 
\hspace*{6pt}\hspace*{6pt}{weights}="{0.45 0.90}"/>}\mbox{}\newline 
{</\textbf{altGrp}>}\end{shaded}\egroup 


    \item[{Exemple}]
  \leavevmode\bgroup\exampleFont \begin{shaded}\noindent\mbox{}{<\textbf{altGrp}\hspace*{6pt}{mode}="{incl}">}\mbox{}\newline 
\hspace*{6pt}{<\textbf{alt}\hspace*{6pt}{target}="{\#dm \#rl}"\hspace*{6pt}{weights}="{0.90 0.90}"/>}\mbox{}\newline 
\hspace*{6pt}{<\textbf{alt}\hspace*{6pt}{target}="{\#lt \#rl}"\hspace*{6pt}{weights}="{0.5 0.5}"/>}\mbox{}\newline 
\hspace*{6pt}{<\textbf{alt}\hspace*{6pt}{target}="{\#bb \#rl}"\hspace*{6pt}{weights}="{0.5 0.5}"/>}\mbox{}\newline 
\hspace*{6pt}{<\textbf{alt}\hspace*{6pt}{target}="{\#dm \#db}"\hspace*{6pt}{weights}="{0.10 0.10}"/>}\mbox{}\newline 
\hspace*{6pt}{<\textbf{alt}\hspace*{6pt}{target}="{\#lt \#db}"\hspace*{6pt}{weights}="{0.45 0.90}"/>}\mbox{}\newline 
\hspace*{6pt}{<\textbf{alt}\hspace*{6pt}{target}="{\#bb \#db}"\hspace*{6pt}{weights}="{0.45 0.90}"/>}\mbox{}\newline 
{</\textbf{altGrp}>}\end{shaded}\egroup 


    \item[{Modèle de contenu}]
  \mbox{}\hfill\\[-10pt]\begin{Verbatim}[fontsize=\small]
<content>
 <alternate maxOccurs="unbounded"
  minOccurs="0">
  <elementRef key="alt"/>
  <elementRef key="ptr"/>
 </alternate>
</content>
    
\end{Verbatim}

    \item[{Schéma Declaration}]
  \mbox{}\hfill\\[-10pt]\begin{Verbatim}[fontsize=\small]
element altGrp
{
   tei_att.global.attributes,
   tei_att.pointing.group.attributes,
   attribute mode { "excl" | "incl" }?,
   ( tei_alt | tei_ptr )*
}
\end{Verbatim}

\end{reflist}  \index{altIdentifier=<altIdentifier>|oddindex}
\begin{reflist}
\item[]\begin{specHead}{TEI.altIdentifier}{<altIdentifier> }(autre identifiant) Contient un autre ou un ancien identifiant pour un manuscrit, par exemple un numéro anciennement utilisé dans un catalogue. [\xref{http://www.tei-c.org/release/doc/tei-p5-doc/en/html/MS.html\#msid}{10.4. The Manuscript Identifier}]\end{specHead} 
    \item[{Module}]
  msdescription
    \item[{Attributs}]
  Attributs \hyperref[TEI.att.global]{att.global} (\textit{@xml:id}, \textit{@n}, \textit{@xml:lang}, \textit{@xml:base}, \textit{@xml:space})  (\hyperref[TEI.att.global.rendition]{att.global.rendition} (\textit{@rend}, \textit{@style}, \textit{@rendition})) (\hyperref[TEI.att.global.linking]{att.global.linking} (\textit{@corresp}, \textit{@synch}, \textit{@sameAs}, \textit{@copyOf}, \textit{@next}, \textit{@prev}, \textit{@exclude}, \textit{@select})) (\hyperref[TEI.att.global.analytic]{att.global.analytic} (\textit{@ana})) (\hyperref[TEI.att.global.facs]{att.global.facs} (\textit{@facs})) (\hyperref[TEI.att.global.change]{att.global.change} (\textit{@change})) (\hyperref[TEI.att.global.responsibility]{att.global.responsibility} (\textit{@cert}, \textit{@resp})) (\hyperref[TEI.att.global.source]{att.global.source} (\textit{@source})) \hyperref[TEI.att.typed]{att.typed} (\textit{@type}, \textit{@subtype}) 
    \item[{Contenu dans}]
  
    \item[msdescription: ]
   \hyperref[TEI.msFrag]{msFrag} \hyperref[TEI.msIdentifier]{msIdentifier}
    \item[{Peut contenir}]
  
    \item[core: ]
   \hyperref[TEI.note]{note}\par 
    \item[header: ]
   \hyperref[TEI.idno]{idno}\par 
    \item[msdescription: ]
   \hyperref[TEI.collection]{collection} \hyperref[TEI.institution]{institution} \hyperref[TEI.repository]{repository}\par 
    \item[namesdates: ]
   \hyperref[TEI.country]{country} \hyperref[TEI.geogName]{geogName} \hyperref[TEI.placeName]{placeName} \hyperref[TEI.region]{region} \hyperref[TEI.settlement]{settlement}
    \item[{Note}]
  \par
Un numéro identifiant quelconque doit être fourni s'il est connu ; si on ne le connaît pas, cela devrait être signalé.
    \item[{Exemple}]
  \leavevmode\bgroup\exampleFont \begin{shaded}\noindent\mbox{}{<\textbf{altIdentifier}>}\mbox{}\newline 
\hspace*{6pt}{<\textbf{idno}>}B 106{</\textbf{idno}>}\mbox{}\newline 
\hspace*{6pt}{<\textbf{note}>}Cote de la Bibliothèque royale au XVIIIe siècle.{</\textbf{note}>}\mbox{}\newline 
{</\textbf{altIdentifier}>}\end{shaded}\egroup 


    \item[{Modèle de contenu}]
  \mbox{}\hfill\\[-10pt]\begin{Verbatim}[fontsize=\small]
<content>
 <sequence maxOccurs="1" minOccurs="1">
  <classRef expand="sequenceOptional"
   key="model.placeNamePart"/>
  <elementRef key="institution"
   minOccurs="0"/>
  <elementRef key="repository"
   minOccurs="0"/>
  <elementRef key="collection"
   minOccurs="0"/>
  <elementRef key="idno"/>
  <elementRef key="note" minOccurs="0"/>
 </sequence>
</content>
    
\end{Verbatim}

    \item[{Schéma Declaration}]
  \mbox{}\hfill\\[-10pt]\begin{Verbatim}[fontsize=\small]
element altIdentifier
{
   tei_att.global.attributes,
   tei_att.typed.attributes,
   (
      tei_placeName?,
      tei_country?,
      tei_region?,
      tei_settlement?,
      tei_geogName?,
      tei_institution?,
      tei_repository?,
      tei_collection?,
      tei_idno,
      tei_note?
   )
}
\end{Verbatim}

\end{reflist}  \index{am=<am>|oddindex}
\begin{reflist}
\item[]\begin{specHead}{TEI.am}{<am> }(marqueur d'abréviation) contient une succession de lettres ou de signes présents dans une abréviation mais omis ou remplacés dans la forme développée de l'abréviation [\xref{http://www.tei-c.org/release/doc/tei-p5-doc/en/html/PH.html\#PHAB}{11.3.1.2. Abbreviation and Expansion}]\end{specHead} 
    \item[{Module}]
  transcr
    \item[{Attributs}]
  Attributs \hyperref[TEI.att.global]{att.global} (\textit{@xml:id}, \textit{@n}, \textit{@xml:lang}, \textit{@xml:base}, \textit{@xml:space})  (\hyperref[TEI.att.global.rendition]{att.global.rendition} (\textit{@rend}, \textit{@style}, \textit{@rendition})) (\hyperref[TEI.att.global.linking]{att.global.linking} (\textit{@corresp}, \textit{@synch}, \textit{@sameAs}, \textit{@copyOf}, \textit{@next}, \textit{@prev}, \textit{@exclude}, \textit{@select})) (\hyperref[TEI.att.global.analytic]{att.global.analytic} (\textit{@ana})) (\hyperref[TEI.att.global.facs]{att.global.facs} (\textit{@facs})) (\hyperref[TEI.att.global.change]{att.global.change} (\textit{@change})) (\hyperref[TEI.att.global.responsibility]{att.global.responsibility} (\textit{@cert}, \textit{@resp})) (\hyperref[TEI.att.global.source]{att.global.source} (\textit{@source})) \hyperref[TEI.att.typed]{att.typed} (\textit{@type}, \textit{@subtype}) \hyperref[TEI.att.editLike]{att.editLike} (\textit{@evidence}, \textit{@instant})  (\hyperref[TEI.att.dimensions]{att.dimensions} (\textit{@unit}, \textit{@quantity}, \textit{@extent}, \textit{@precision}, \textit{@scope}) (\hyperref[TEI.att.ranging]{att.ranging} (\textit{@atLeast}, \textit{@atMost}, \textit{@min}, \textit{@max}, \textit{@confidence})) )
    \item[{Membre du}]
  \hyperref[TEI.model.choicePart]{model.choicePart} \hyperref[TEI.model.pPart.editorial]{model.pPart.editorial}
    \item[{Contenu dans}]
  
    \item[analysis: ]
   \hyperref[TEI.cl]{cl} \hyperref[TEI.pc]{pc} \hyperref[TEI.phr]{phr} \hyperref[TEI.s]{s} \hyperref[TEI.span]{span} \hyperref[TEI.w]{w}\par 
    \item[core: ]
   \hyperref[TEI.abbr]{abbr} \hyperref[TEI.add]{add} \hyperref[TEI.addrLine]{addrLine} \hyperref[TEI.author]{author} \hyperref[TEI.bibl]{bibl} \hyperref[TEI.biblScope]{biblScope} \hyperref[TEI.choice]{choice} \hyperref[TEI.citedRange]{citedRange} \hyperref[TEI.corr]{corr} \hyperref[TEI.date]{date} \hyperref[TEI.del]{del} \hyperref[TEI.desc]{desc} \hyperref[TEI.distinct]{distinct} \hyperref[TEI.editor]{editor} \hyperref[TEI.email]{email} \hyperref[TEI.emph]{emph} \hyperref[TEI.expan]{expan} \hyperref[TEI.foreign]{foreign} \hyperref[TEI.gloss]{gloss} \hyperref[TEI.head]{head} \hyperref[TEI.headItem]{headItem} \hyperref[TEI.headLabel]{headLabel} \hyperref[TEI.hi]{hi} \hyperref[TEI.item]{item} \hyperref[TEI.l]{l} \hyperref[TEI.label]{label} \hyperref[TEI.measure]{measure} \hyperref[TEI.meeting]{meeting} \hyperref[TEI.mentioned]{mentioned} \hyperref[TEI.name]{name} \hyperref[TEI.note]{note} \hyperref[TEI.num]{num} \hyperref[TEI.orig]{orig} \hyperref[TEI.p]{p} \hyperref[TEI.pubPlace]{pubPlace} \hyperref[TEI.publisher]{publisher} \hyperref[TEI.q]{q} \hyperref[TEI.quote]{quote} \hyperref[TEI.ref]{ref} \hyperref[TEI.reg]{reg} \hyperref[TEI.resp]{resp} \hyperref[TEI.rs]{rs} \hyperref[TEI.said]{said} \hyperref[TEI.sic]{sic} \hyperref[TEI.soCalled]{soCalled} \hyperref[TEI.speaker]{speaker} \hyperref[TEI.stage]{stage} \hyperref[TEI.street]{street} \hyperref[TEI.term]{term} \hyperref[TEI.textLang]{textLang} \hyperref[TEI.time]{time} \hyperref[TEI.title]{title} \hyperref[TEI.unclear]{unclear}\par 
    \item[figures: ]
   \hyperref[TEI.cell]{cell} \hyperref[TEI.figDesc]{figDesc}\par 
    \item[header: ]
   \hyperref[TEI.authority]{authority} \hyperref[TEI.change]{change} \hyperref[TEI.classCode]{classCode} \hyperref[TEI.creation]{creation} \hyperref[TEI.distributor]{distributor} \hyperref[TEI.edition]{edition} \hyperref[TEI.extent]{extent} \hyperref[TEI.funder]{funder} \hyperref[TEI.language]{language} \hyperref[TEI.licence]{licence} \hyperref[TEI.rendition]{rendition}\par 
    \item[iso-fs: ]
   \hyperref[TEI.fDescr]{fDescr} \hyperref[TEI.fsDescr]{fsDescr}\par 
    \item[linking: ]
   \hyperref[TEI.ab]{ab} \hyperref[TEI.seg]{seg}\par 
    \item[msdescription: ]
   \hyperref[TEI.accMat]{accMat} \hyperref[TEI.acquisition]{acquisition} \hyperref[TEI.additions]{additions} \hyperref[TEI.catchwords]{catchwords} \hyperref[TEI.collation]{collation} \hyperref[TEI.colophon]{colophon} \hyperref[TEI.condition]{condition} \hyperref[TEI.custEvent]{custEvent} \hyperref[TEI.decoNote]{decoNote} \hyperref[TEI.explicit]{explicit} \hyperref[TEI.filiation]{filiation} \hyperref[TEI.finalRubric]{finalRubric} \hyperref[TEI.foliation]{foliation} \hyperref[TEI.heraldry]{heraldry} \hyperref[TEI.incipit]{incipit} \hyperref[TEI.layout]{layout} \hyperref[TEI.material]{material} \hyperref[TEI.musicNotation]{musicNotation} \hyperref[TEI.objectType]{objectType} \hyperref[TEI.origDate]{origDate} \hyperref[TEI.origPlace]{origPlace} \hyperref[TEI.origin]{origin} \hyperref[TEI.provenance]{provenance} \hyperref[TEI.rubric]{rubric} \hyperref[TEI.secFol]{secFol} \hyperref[TEI.signatures]{signatures} \hyperref[TEI.source]{source} \hyperref[TEI.stamp]{stamp} \hyperref[TEI.summary]{summary} \hyperref[TEI.support]{support} \hyperref[TEI.surrogates]{surrogates} \hyperref[TEI.typeNote]{typeNote} \hyperref[TEI.watermark]{watermark}\par 
    \item[namesdates: ]
   \hyperref[TEI.addName]{addName} \hyperref[TEI.affiliation]{affiliation} \hyperref[TEI.country]{country} \hyperref[TEI.forename]{forename} \hyperref[TEI.genName]{genName} \hyperref[TEI.geogName]{geogName} \hyperref[TEI.nameLink]{nameLink} \hyperref[TEI.orgName]{orgName} \hyperref[TEI.persName]{persName} \hyperref[TEI.placeName]{placeName} \hyperref[TEI.region]{region} \hyperref[TEI.roleName]{roleName} \hyperref[TEI.settlement]{settlement} \hyperref[TEI.surname]{surname}\par 
    \item[textstructure: ]
   \hyperref[TEI.docAuthor]{docAuthor} \hyperref[TEI.docDate]{docDate} \hyperref[TEI.docEdition]{docEdition} \hyperref[TEI.titlePart]{titlePart}\par 
    \item[transcr: ]
   \hyperref[TEI.damage]{damage} \hyperref[TEI.fw]{fw} \hyperref[TEI.metamark]{metamark} \hyperref[TEI.mod]{mod} \hyperref[TEI.restore]{restore} \hyperref[TEI.retrace]{retrace} \hyperref[TEI.secl]{secl} \hyperref[TEI.supplied]{supplied} \hyperref[TEI.surplus]{surplus}
    \item[{Peut contenir}]
  
    \item[core: ]
   \hyperref[TEI.add]{add} \hyperref[TEI.corr]{corr} \hyperref[TEI.del]{del} \hyperref[TEI.orig]{orig} \hyperref[TEI.reg]{reg} \hyperref[TEI.sic]{sic} \hyperref[TEI.unclear]{unclear}\par 
    \item[transcr: ]
   \hyperref[TEI.damage]{damage} \hyperref[TEI.handShift]{handShift} \hyperref[TEI.mod]{mod} \hyperref[TEI.redo]{redo} \hyperref[TEI.restore]{restore} \hyperref[TEI.retrace]{retrace} \hyperref[TEI.secl]{secl} \hyperref[TEI.supplied]{supplied} \hyperref[TEI.surplus]{surplus} \hyperref[TEI.undo]{undo}\par des données textuelles
    \item[{Exemple}]
  \leavevmode\bgroup\exampleFont \begin{shaded}\noindent\mbox{} Le {<\textbf{abbr}>}Dr{<\textbf{am}>}.{</\textbf{am}>}\mbox{}\newline 
{</\textbf{abbr}>}\mbox{}\newline 
 Pasquier se prit à marcher de long en large, les mains glissées dans la\mbox{}\newline 
 ceinture de sa blouse.\mbox{}\newline 
\end{shaded}\egroup 


    \item[{Modèle de contenu}]
  \mbox{}\hfill\\[-10pt]\begin{Verbatim}[fontsize=\small]
<content>
 <alternate maxOccurs="unbounded"
  minOccurs="0">
  <textNode/>
  <classRef key="model.gLike"/>
  <classRef key="model.pPart.transcriptional"/>
 </alternate>
</content>
    
\end{Verbatim}

    \item[{Schéma Declaration}]
  \mbox{}\hfill\\[-10pt]\begin{Verbatim}[fontsize=\small]
element am
{
   tei_att.global.attributes,
   tei_att.typed.attributes,
   tei_att.editLike.attributes,
   ( text | tei_model.gLike | tei_model.pPart.transcriptional )*
}
\end{Verbatim}

\end{reflist}  \index{analytic=<analytic>|oddindex}
\begin{reflist}
\item[]\begin{specHead}{TEI.analytic}{<analytic> }(niveau analytique) contient des éléments descriptifs qui décrivent la bibliographie d'une ressource (par exemple un poème ou un article de revue) publiée à l'intérieur d'une monographie ou d'une ressource et non publiée de façon indépendante. [\xref{http://www.tei-c.org/release/doc/tei-p5-doc/en/html/CO.html\#COBICOL}{3.11.2.1. Analytic, Monographic, and Series Levels}]\end{specHead} 
    \item[{Module}]
  core
    \item[{Attributs}]
  Attributs \hyperref[TEI.att.global]{att.global} (\textit{@xml:id}, \textit{@n}, \textit{@xml:lang}, \textit{@xml:base}, \textit{@xml:space})  (\hyperref[TEI.att.global.rendition]{att.global.rendition} (\textit{@rend}, \textit{@style}, \textit{@rendition})) (\hyperref[TEI.att.global.linking]{att.global.linking} (\textit{@corresp}, \textit{@synch}, \textit{@sameAs}, \textit{@copyOf}, \textit{@next}, \textit{@prev}, \textit{@exclude}, \textit{@select})) (\hyperref[TEI.att.global.analytic]{att.global.analytic} (\textit{@ana})) (\hyperref[TEI.att.global.facs]{att.global.facs} (\textit{@facs})) (\hyperref[TEI.att.global.change]{att.global.change} (\textit{@change})) (\hyperref[TEI.att.global.responsibility]{att.global.responsibility} (\textit{@cert}, \textit{@resp})) (\hyperref[TEI.att.global.source]{att.global.source} (\textit{@source}))
    \item[{Contenu dans}]
  
    \item[core: ]
   \hyperref[TEI.biblStruct]{biblStruct}
    \item[{Peut contenir}]
  
    \item[core: ]
   \hyperref[TEI.author]{author} \hyperref[TEI.date]{date} \hyperref[TEI.editor]{editor} \hyperref[TEI.ptr]{ptr} \hyperref[TEI.ref]{ref} \hyperref[TEI.respStmt]{respStmt} \hyperref[TEI.textLang]{textLang} \hyperref[TEI.title]{title}\par 
    \item[header: ]
   \hyperref[TEI.availability]{availability} \hyperref[TEI.idno]{idno}
    \item[{Note}]
  \par
Cet élément peut contenir des titres et des mentions de responsabilité (auteur, éditeur scientifique, ou autre), et cela dans n'importe quel ordre.\par
L'élément \hyperref[TEI.analytic]{<analytic>} n'est disponible qu'à l'intérieur de l'élément \hyperref[TEI.biblStruct]{<biblStruct>}, où il faut l'utiliser pour encoder la description bibliographique d'une partie composante.
    \item[{Exemple}]
  \leavevmode\bgroup\exampleFont \begin{shaded}\noindent\mbox{}{<\textbf{biblStruct}>}\mbox{}\newline 
\hspace*{6pt}{<\textbf{analytic}>}\mbox{}\newline 
\hspace*{6pt}\hspace*{6pt}{<\textbf{author}>}Chesnutt, David{</\textbf{author}>}\mbox{}\newline 
\hspace*{6pt}\hspace*{6pt}{<\textbf{title}>}Historical Editions in the States{</\textbf{title}>}\mbox{}\newline 
\hspace*{6pt}{</\textbf{analytic}>}\mbox{}\newline 
\hspace*{6pt}{<\textbf{monogr}>}\mbox{}\newline 
\hspace*{6pt}\hspace*{6pt}{<\textbf{title}\hspace*{6pt}{level}="{j}">}Computers and the Humanities{</\textbf{title}>}\mbox{}\newline 
\hspace*{6pt}\hspace*{6pt}{<\textbf{imprint}>}\mbox{}\newline 
\hspace*{6pt}\hspace*{6pt}\hspace*{6pt}{<\textbf{date}\hspace*{6pt}{when}="{1991-12}">}(December, 1991):{</\textbf{date}>}\mbox{}\newline 
\hspace*{6pt}\hspace*{6pt}{</\textbf{imprint}>}\mbox{}\newline 
\hspace*{6pt}\hspace*{6pt}{<\textbf{biblScope}>}25.6{</\textbf{biblScope}>}\mbox{}\newline 
\hspace*{6pt}\hspace*{6pt}{<\textbf{biblScope}>}377–380{</\textbf{biblScope}>}\mbox{}\newline 
\hspace*{6pt}{</\textbf{monogr}>}\mbox{}\newline 
{</\textbf{biblStruct}>}\end{shaded}\egroup 


    \item[{Modèle de contenu}]
  \mbox{}\hfill\\[-10pt]\begin{Verbatim}[fontsize=\small]
<content>
 <alternate maxOccurs="unbounded"
  minOccurs="0">
  <elementRef key="author"/>
  <elementRef key="editor"/>
  <elementRef key="respStmt"/>
  <elementRef key="title"/>
  <classRef key="model.ptrLike"/>
  <elementRef key="date"/>
  <elementRef key="textLang"/>
  <elementRef key="idno"/>
  <elementRef key="availability"/>
 </alternate>
</content>
    
\end{Verbatim}

    \item[{Schéma Declaration}]
  \mbox{}\hfill\\[-10pt]\begin{Verbatim}[fontsize=\small]
element analytic
{
   tei_att.global.attributes,
   (
      tei_author    | tei_editor    | tei_respStmt    | tei_title    | tei_model.ptrLike    | tei_date    | tei_textLang    | tei_idno    | tei_availability   )*
}
\end{Verbatim}

\end{reflist}  \index{anchor=<anchor>|oddindex}
\begin{reflist}
\item[]\begin{specHead}{TEI.anchor}{<anchor> }(point d'ancrage) attache un identifiant à un point du texte, que ce point corresponde ou non à un élément textuel. [\xref{http://www.tei-c.org/release/doc/tei-p5-doc/en/html/TS.html\#TSSAPA}{8.4.2. Synchronization and Overlap} \xref{http://www.tei-c.org/release/doc/tei-p5-doc/en/html/SA.html\#SACS}{16.5. Correspondence and Alignment}]\end{specHead} 
    \item[{Module}]
  linking
    \item[{Attributs}]
  Attributs \hyperref[TEI.att.global]{att.global} (\textit{@xml:id}, \textit{@n}, \textit{@xml:lang}, \textit{@xml:base}, \textit{@xml:space})  (\hyperref[TEI.att.global.rendition]{att.global.rendition} (\textit{@rend}, \textit{@style}, \textit{@rendition})) (\hyperref[TEI.att.global.linking]{att.global.linking} (\textit{@corresp}, \textit{@synch}, \textit{@sameAs}, \textit{@copyOf}, \textit{@next}, \textit{@prev}, \textit{@exclude}, \textit{@select})) (\hyperref[TEI.att.global.analytic]{att.global.analytic} (\textit{@ana})) (\hyperref[TEI.att.global.facs]{att.global.facs} (\textit{@facs})) (\hyperref[TEI.att.global.change]{att.global.change} (\textit{@change})) (\hyperref[TEI.att.global.responsibility]{att.global.responsibility} (\textit{@cert}, \textit{@resp})) (\hyperref[TEI.att.global.source]{att.global.source} (\textit{@source})) \hyperref[TEI.att.typed]{att.typed} (\textit{@type}, \textit{@subtype}) 
    \item[{Membre du}]
  \hyperref[TEI.model.milestoneLike]{model.milestoneLike}
    \item[{Contenu dans}]
  
    \item[analysis: ]
   \hyperref[TEI.cl]{cl} \hyperref[TEI.m]{m} \hyperref[TEI.phr]{phr} \hyperref[TEI.s]{s} \hyperref[TEI.span]{span} \hyperref[TEI.w]{w}\par 
    \item[core: ]
   \hyperref[TEI.abbr]{abbr} \hyperref[TEI.add]{add} \hyperref[TEI.addrLine]{addrLine} \hyperref[TEI.address]{address} \hyperref[TEI.author]{author} \hyperref[TEI.bibl]{bibl} \hyperref[TEI.biblScope]{biblScope} \hyperref[TEI.cit]{cit} \hyperref[TEI.citedRange]{citedRange} \hyperref[TEI.corr]{corr} \hyperref[TEI.date]{date} \hyperref[TEI.del]{del} \hyperref[TEI.distinct]{distinct} \hyperref[TEI.editor]{editor} \hyperref[TEI.email]{email} \hyperref[TEI.emph]{emph} \hyperref[TEI.expan]{expan} \hyperref[TEI.foreign]{foreign} \hyperref[TEI.gloss]{gloss} \hyperref[TEI.head]{head} \hyperref[TEI.headItem]{headItem} \hyperref[TEI.headLabel]{headLabel} \hyperref[TEI.hi]{hi} \hyperref[TEI.imprint]{imprint} \hyperref[TEI.item]{item} \hyperref[TEI.l]{l} \hyperref[TEI.label]{label} \hyperref[TEI.lg]{lg} \hyperref[TEI.list]{list} \hyperref[TEI.listBibl]{listBibl} \hyperref[TEI.measure]{measure} \hyperref[TEI.mentioned]{mentioned} \hyperref[TEI.name]{name} \hyperref[TEI.note]{note} \hyperref[TEI.num]{num} \hyperref[TEI.orig]{orig} \hyperref[TEI.p]{p} \hyperref[TEI.pubPlace]{pubPlace} \hyperref[TEI.publisher]{publisher} \hyperref[TEI.q]{q} \hyperref[TEI.quote]{quote} \hyperref[TEI.ref]{ref} \hyperref[TEI.reg]{reg} \hyperref[TEI.resp]{resp} \hyperref[TEI.rs]{rs} \hyperref[TEI.said]{said} \hyperref[TEI.series]{series} \hyperref[TEI.sic]{sic} \hyperref[TEI.soCalled]{soCalled} \hyperref[TEI.sp]{sp} \hyperref[TEI.speaker]{speaker} \hyperref[TEI.stage]{stage} \hyperref[TEI.street]{street} \hyperref[TEI.term]{term} \hyperref[TEI.textLang]{textLang} \hyperref[TEI.time]{time} \hyperref[TEI.title]{title} \hyperref[TEI.unclear]{unclear}\par 
    \item[figures: ]
   \hyperref[TEI.cell]{cell} \hyperref[TEI.figure]{figure} \hyperref[TEI.table]{table}\par 
    \item[header: ]
   \hyperref[TEI.authority]{authority} \hyperref[TEI.change]{change} \hyperref[TEI.classCode]{classCode} \hyperref[TEI.distributor]{distributor} \hyperref[TEI.edition]{edition} \hyperref[TEI.extent]{extent} \hyperref[TEI.funder]{funder} \hyperref[TEI.language]{language} \hyperref[TEI.licence]{licence}\par 
    \item[linking: ]
   \hyperref[TEI.ab]{ab} \hyperref[TEI.seg]{seg}\par 
    \item[msdescription: ]
   \hyperref[TEI.accMat]{accMat} \hyperref[TEI.acquisition]{acquisition} \hyperref[TEI.additions]{additions} \hyperref[TEI.catchwords]{catchwords} \hyperref[TEI.collation]{collation} \hyperref[TEI.colophon]{colophon} \hyperref[TEI.condition]{condition} \hyperref[TEI.custEvent]{custEvent} \hyperref[TEI.decoNote]{decoNote} \hyperref[TEI.explicit]{explicit} \hyperref[TEI.filiation]{filiation} \hyperref[TEI.finalRubric]{finalRubric} \hyperref[TEI.foliation]{foliation} \hyperref[TEI.heraldry]{heraldry} \hyperref[TEI.incipit]{incipit} \hyperref[TEI.layout]{layout} \hyperref[TEI.material]{material} \hyperref[TEI.msItem]{msItem} \hyperref[TEI.musicNotation]{musicNotation} \hyperref[TEI.objectType]{objectType} \hyperref[TEI.origDate]{origDate} \hyperref[TEI.origPlace]{origPlace} \hyperref[TEI.origin]{origin} \hyperref[TEI.provenance]{provenance} \hyperref[TEI.rubric]{rubric} \hyperref[TEI.secFol]{secFol} \hyperref[TEI.signatures]{signatures} \hyperref[TEI.source]{source} \hyperref[TEI.stamp]{stamp} \hyperref[TEI.summary]{summary} \hyperref[TEI.support]{support} \hyperref[TEI.surrogates]{surrogates} \hyperref[TEI.typeNote]{typeNote} \hyperref[TEI.watermark]{watermark}\par 
    \item[namesdates: ]
   \hyperref[TEI.addName]{addName} \hyperref[TEI.affiliation]{affiliation} \hyperref[TEI.country]{country} \hyperref[TEI.forename]{forename} \hyperref[TEI.genName]{genName} \hyperref[TEI.geogName]{geogName} \hyperref[TEI.nameLink]{nameLink} \hyperref[TEI.org]{org} \hyperref[TEI.orgName]{orgName} \hyperref[TEI.persName]{persName} \hyperref[TEI.person]{person} \hyperref[TEI.personGrp]{personGrp} \hyperref[TEI.persona]{persona} \hyperref[TEI.placeName]{placeName} \hyperref[TEI.region]{region} \hyperref[TEI.roleName]{roleName} \hyperref[TEI.settlement]{settlement} \hyperref[TEI.surname]{surname}\par 
    \item[textstructure: ]
   \hyperref[TEI.back]{back} \hyperref[TEI.body]{body} \hyperref[TEI.div]{div} \hyperref[TEI.docAuthor]{docAuthor} \hyperref[TEI.docDate]{docDate} \hyperref[TEI.docEdition]{docEdition} \hyperref[TEI.docTitle]{docTitle} \hyperref[TEI.floatingText]{floatingText} \hyperref[TEI.front]{front} \hyperref[TEI.group]{group} \hyperref[TEI.text]{text} \hyperref[TEI.titlePage]{titlePage} \hyperref[TEI.titlePart]{titlePart}\par 
    \item[transcr: ]
   \hyperref[TEI.damage]{damage} \hyperref[TEI.fw]{fw} \hyperref[TEI.line]{line} \hyperref[TEI.metamark]{metamark} \hyperref[TEI.mod]{mod} \hyperref[TEI.restore]{restore} \hyperref[TEI.retrace]{retrace} \hyperref[TEI.secl]{secl} \hyperref[TEI.sourceDoc]{sourceDoc} \hyperref[TEI.subst]{subst} \hyperref[TEI.supplied]{supplied} \hyperref[TEI.surface]{surface} \hyperref[TEI.surfaceGrp]{surfaceGrp} \hyperref[TEI.surplus]{surplus} \hyperref[TEI.zone]{zone}
    \item[{Peut contenir}]
  Elément vide
    \item[{Note}]
  \par
Il faut donner à cet élément un attribut global {\itshape xml:id} afin de spécifier un identifiant pour le point où l'élément intervient dans un document TEI. La valeur utilisée peut être choisie librement, pourvu qu'elle soit unique dans le document TEI et que le nom soit syntaxiquement valide. Les valeurs contenant des nombres ne doivent pas nécessairement former une séquence.
    \item[{Exemple}]
  \leavevmode\bgroup\exampleFont \begin{shaded}\noindent\mbox{}{<\textbf{s}>}L'ancre est i{<\textbf{anchor}\hspace*{6pt}{xml:id}="{fr\textunderscore A234}"/>}ci quelque part.{</\textbf{s}>}\mbox{}\newline 
{<\textbf{s}>}Aidez-moi à la trouver.{<\textbf{ptr}\hspace*{6pt}{target}="{\#fr\textunderscore A234}"/>}\mbox{}\newline 
{</\textbf{s}>}\end{shaded}\egroup 


    \item[{Modèle de contenu}]
  \fbox{\ttfamily <content>\newline
</content>\newline
    } 
    \item[{Schéma Declaration}]
  \mbox{}\hfill\\[-10pt]\begin{Verbatim}[fontsize=\small]
element anchor { tei_att.global.attributes, tei_att.typed.attributes, empty }
\end{Verbatim}

\end{reflist}  \index{annotation=<annotation>|oddindex}\index{encoding=@encoding!<annotation>|oddindex}
\begin{reflist}
\item[]\begin{specHead}{TEI.annotation}{<annotation> }\end{specHead} 
    \item[{Namespace}]
  http://www.w3.org/1998/Math/MathML
    \item[{Module}]
  derived-module-tei.istex
    \item[{Attributs}]
  Attributs\hfil\\[-10pt]\begin{sansreflist}
    \item[@encoding]
  
\begin{reflist}
    \item[{Statut}]
  Optionel
    \item[{Type de données}]
  \xref{https://www.w3.org/TR/xmlschema-2/\#}{}
\end{reflist}  
\end{sansreflist}  
    \item[{Contenu dans}]
  
    \item[derived-module-tei.istex: ]
   \hyperref[TEI.semantics]{semantics}
    \item[{Peut contenir}]
  Des données textuelles uniquement
    \item[{Modèle de contenu}]
  \fbox{\ttfamily <content>\newline
 <textNode/>\newline
</content>\newline
    } 
    \item[{Schéma Declaration}]
  \mbox{}\hfill\\[-10pt]\begin{Verbatim}[fontsize=\small]
element annotation { attribute encoding { encoding }?, text }
\end{Verbatim}

\end{reflist}  \index{annotationBlock=<annotationBlock>|oddindex}
\begin{reflist}
\item[]\begin{specHead}{TEI.annotationBlock}{<annotationBlock> }groups together various annotations, e.g. for parallel interpretations of a spoken segment. [\xref{http://www.tei-c.org/release/doc/tei-p5-doc/en/html/TS.html\#TSTPAC}{8.4.6. Analytic Coding}]\end{specHead} 
    \item[{Module}]
  spoken
    \item[{Attributs}]
  Attributs \hyperref[TEI.att.notated]{att.notated} (\textit{@notation}) \hyperref[TEI.att.typed]{att.typed} (\textit{@type}, \textit{@subtype}) \hyperref[TEI.att.ascribed]{att.ascribed} (\textit{@who}) \hyperref[TEI.att.timed]{att.timed} (\textit{@start}, \textit{@end})  (\hyperref[TEI.att.duration]{att.duration} (\hyperref[TEI.att.duration.w3c]{att.duration.w3c} (\textit{@dur})) (\hyperref[TEI.att.duration.iso]{att.duration.iso} (\textit{@dur-iso})) ) \hyperref[TEI.att.global]{att.global} (\textit{@xml:id}, \textit{@n}, \textit{@xml:lang}, \textit{@xml:base}, \textit{@xml:space})  (\hyperref[TEI.att.global.rendition]{att.global.rendition} (\textit{@rend}, \textit{@style}, \textit{@rendition})) (\hyperref[TEI.att.global.linking]{att.global.linking} (\textit{@corresp}, \textit{@synch}, \textit{@sameAs}, \textit{@copyOf}, \textit{@next}, \textit{@prev}, \textit{@exclude}, \textit{@select})) (\hyperref[TEI.att.global.analytic]{att.global.analytic} (\textit{@ana})) (\hyperref[TEI.att.global.facs]{att.global.facs} (\textit{@facs})) (\hyperref[TEI.att.global.change]{att.global.change} (\textit{@change})) (\hyperref[TEI.att.global.responsibility]{att.global.responsibility} (\textit{@cert}, \textit{@resp})) (\hyperref[TEI.att.global.source]{att.global.source} (\textit{@source}))
    \item[{Membre du}]
  \hyperref[TEI.model.annotation]{model.annotation} \hyperref[TEI.model.divPart.spoken]{model.divPart.spoken} \hyperref[TEI.model.segLike]{model.segLike}
    \item[{Contenu dans}]
  
    \item[analysis: ]
   \hyperref[TEI.cl]{cl} \hyperref[TEI.phr]{phr} \hyperref[TEI.s]{s}\par 
    \item[core: ]
   \hyperref[TEI.abbr]{abbr} \hyperref[TEI.add]{add} \hyperref[TEI.addrLine]{addrLine} \hyperref[TEI.author]{author} \hyperref[TEI.bibl]{bibl} \hyperref[TEI.biblScope]{biblScope} \hyperref[TEI.citedRange]{citedRange} \hyperref[TEI.corr]{corr} \hyperref[TEI.date]{date} \hyperref[TEI.del]{del} \hyperref[TEI.distinct]{distinct} \hyperref[TEI.editor]{editor} \hyperref[TEI.email]{email} \hyperref[TEI.emph]{emph} \hyperref[TEI.expan]{expan} \hyperref[TEI.foreign]{foreign} \hyperref[TEI.gloss]{gloss} \hyperref[TEI.head]{head} \hyperref[TEI.headItem]{headItem} \hyperref[TEI.headLabel]{headLabel} \hyperref[TEI.hi]{hi} \hyperref[TEI.item]{item} \hyperref[TEI.l]{l} \hyperref[TEI.label]{label} \hyperref[TEI.measure]{measure} \hyperref[TEI.mentioned]{mentioned} \hyperref[TEI.name]{name} \hyperref[TEI.note]{note} \hyperref[TEI.num]{num} \hyperref[TEI.orig]{orig} \hyperref[TEI.p]{p} \hyperref[TEI.pubPlace]{pubPlace} \hyperref[TEI.publisher]{publisher} \hyperref[TEI.q]{q} \hyperref[TEI.quote]{quote} \hyperref[TEI.ref]{ref} \hyperref[TEI.reg]{reg} \hyperref[TEI.rs]{rs} \hyperref[TEI.said]{said} \hyperref[TEI.sic]{sic} \hyperref[TEI.soCalled]{soCalled} \hyperref[TEI.speaker]{speaker} \hyperref[TEI.stage]{stage} \hyperref[TEI.street]{street} \hyperref[TEI.term]{term} \hyperref[TEI.textLang]{textLang} \hyperref[TEI.time]{time} \hyperref[TEI.title]{title} \hyperref[TEI.unclear]{unclear}\par 
    \item[figures: ]
   \hyperref[TEI.cell]{cell} \hyperref[TEI.figure]{figure}\par 
    \item[header: ]
   \hyperref[TEI.change]{change} \hyperref[TEI.distributor]{distributor} \hyperref[TEI.edition]{edition} \hyperref[TEI.extent]{extent} \hyperref[TEI.licence]{licence}\par 
    \item[linking: ]
   \hyperref[TEI.ab]{ab} \hyperref[TEI.seg]{seg}\par 
    \item[msdescription: ]
   \hyperref[TEI.accMat]{accMat} \hyperref[TEI.acquisition]{acquisition} \hyperref[TEI.additions]{additions} \hyperref[TEI.catchwords]{catchwords} \hyperref[TEI.collation]{collation} \hyperref[TEI.colophon]{colophon} \hyperref[TEI.condition]{condition} \hyperref[TEI.custEvent]{custEvent} \hyperref[TEI.decoNote]{decoNote} \hyperref[TEI.explicit]{explicit} \hyperref[TEI.filiation]{filiation} \hyperref[TEI.finalRubric]{finalRubric} \hyperref[TEI.foliation]{foliation} \hyperref[TEI.heraldry]{heraldry} \hyperref[TEI.incipit]{incipit} \hyperref[TEI.layout]{layout} \hyperref[TEI.material]{material} \hyperref[TEI.musicNotation]{musicNotation} \hyperref[TEI.objectType]{objectType} \hyperref[TEI.origDate]{origDate} \hyperref[TEI.origPlace]{origPlace} \hyperref[TEI.origin]{origin} \hyperref[TEI.provenance]{provenance} \hyperref[TEI.rubric]{rubric} \hyperref[TEI.secFol]{secFol} \hyperref[TEI.signatures]{signatures} \hyperref[TEI.source]{source} \hyperref[TEI.stamp]{stamp} \hyperref[TEI.summary]{summary} \hyperref[TEI.support]{support} \hyperref[TEI.surrogates]{surrogates} \hyperref[TEI.typeNote]{typeNote} \hyperref[TEI.watermark]{watermark}\par 
    \item[namesdates: ]
   \hyperref[TEI.addName]{addName} \hyperref[TEI.affiliation]{affiliation} \hyperref[TEI.country]{country} \hyperref[TEI.forename]{forename} \hyperref[TEI.genName]{genName} \hyperref[TEI.geogName]{geogName} \hyperref[TEI.nameLink]{nameLink} \hyperref[TEI.orgName]{orgName} \hyperref[TEI.persName]{persName} \hyperref[TEI.placeName]{placeName} \hyperref[TEI.region]{region} \hyperref[TEI.roleName]{roleName} \hyperref[TEI.settlement]{settlement} \hyperref[TEI.surname]{surname}\par 
    \item[spoken: ]
   \hyperref[TEI.annotationBlock]{annotationBlock}\par 
    \item[standOff: ]
   \hyperref[TEI.listAnnotation]{listAnnotation}\par 
    \item[textstructure: ]
   \hyperref[TEI.body]{body} \hyperref[TEI.div]{div} \hyperref[TEI.docAuthor]{docAuthor} \hyperref[TEI.docDate]{docDate} \hyperref[TEI.docEdition]{docEdition} \hyperref[TEI.titlePart]{titlePart}\par 
    \item[transcr: ]
   \hyperref[TEI.damage]{damage} \hyperref[TEI.fw]{fw} \hyperref[TEI.metamark]{metamark} \hyperref[TEI.mod]{mod} \hyperref[TEI.restore]{restore} \hyperref[TEI.retrace]{retrace} \hyperref[TEI.secl]{secl} \hyperref[TEI.supplied]{supplied} \hyperref[TEI.surplus]{surplus}
    \item[{Peut contenir}]
  
    \item[analysis: ]
   \hyperref[TEI.interp]{interp} \hyperref[TEI.interpGrp]{interpGrp} \hyperref[TEI.span]{span} \hyperref[TEI.spanGrp]{spanGrp}\par 
    \item[core: ]
   \hyperref[TEI.author]{author} \hyperref[TEI.bibl]{bibl} \hyperref[TEI.biblStruct]{biblStruct} \hyperref[TEI.date]{date} \hyperref[TEI.index]{index} \hyperref[TEI.list]{list} \hyperref[TEI.listBibl]{listBibl} \hyperref[TEI.measure]{measure} \hyperref[TEI.name]{name} \hyperref[TEI.note]{note} \hyperref[TEI.ptr]{ptr} \hyperref[TEI.ref]{ref} \hyperref[TEI.rs]{rs} \hyperref[TEI.time]{time}\par 
    \item[figures: ]
   \hyperref[TEI.table]{table}\par 
    \item[header: ]
   \hyperref[TEI.biblFull]{biblFull} \hyperref[TEI.idno]{idno} \hyperref[TEI.keywords]{keywords}\par 
    \item[iso-fs: ]
   \hyperref[TEI.fLib]{fLib} \hyperref[TEI.fs]{fs} \hyperref[TEI.fvLib]{fvLib}\par 
    \item[linking: ]
   \hyperref[TEI.alt]{alt} \hyperref[TEI.altGrp]{altGrp} \hyperref[TEI.join]{join} \hyperref[TEI.joinGrp]{joinGrp} \hyperref[TEI.link]{link} \hyperref[TEI.linkGrp]{linkGrp} \hyperref[TEI.seg]{seg} \hyperref[TEI.timeline]{timeline}\par 
    \item[msdescription: ]
   \hyperref[TEI.msDesc]{msDesc} \hyperref[TEI.source]{source}\par 
    \item[namesdates: ]
   \hyperref[TEI.addName]{addName} \hyperref[TEI.country]{country} \hyperref[TEI.forename]{forename} \hyperref[TEI.genName]{genName} \hyperref[TEI.geogName]{geogName} \hyperref[TEI.listOrg]{listOrg} \hyperref[TEI.listPlace]{listPlace} \hyperref[TEI.location]{location} \hyperref[TEI.nameLink]{nameLink} \hyperref[TEI.org]{org} \hyperref[TEI.orgName]{orgName} \hyperref[TEI.persName]{persName} \hyperref[TEI.person]{person} \hyperref[TEI.place]{place} \hyperref[TEI.placeName]{placeName} \hyperref[TEI.region]{region} \hyperref[TEI.roleName]{roleName} \hyperref[TEI.settlement]{settlement} \hyperref[TEI.state]{state} \hyperref[TEI.surname]{surname}\par 
    \item[spoken: ]
   \hyperref[TEI.annotationBlock]{annotationBlock}\par 
    \item[standOff: ]
   \hyperref[TEI.listAnnotation]{listAnnotation}\par 
    \item[textstructure: ]
   \hyperref[TEI.text]{text}\par 
    \item[transcr: ]
   \hyperref[TEI.listTranspose]{listTranspose} \hyperref[TEI.substJoin]{substJoin} \hyperref[TEI.zone]{zone}
    \item[{Exemple}]
  StandOff entité nommée\leavevmode\bgroup\exampleFont \begin{shaded}\noindent\mbox{}{<\textbf{annotationBlock}\hspace*{6pt}{corresp}="{text}">}\mbox{}\newline 
\hspace*{6pt}{<\textbf{date}\hspace*{6pt}{change}="{\#Unitex-3.2.0-alpha}"\mbox{}\newline 
\hspace*{6pt}\hspace*{6pt}{resp}="{istex}"\mbox{}\newline 
\hspace*{6pt}\hspace*{6pt}{scheme}="{https://date-entity.data.istex.fr}">}\mbox{}\newline 
\hspace*{6pt}\hspace*{6pt}{<\textbf{term}>}1997{</\textbf{term}>}\mbox{}\newline 
\hspace*{6pt}\hspace*{6pt}{<\textbf{fs}\hspace*{6pt}{type}="{statistics}">}\mbox{}\newline 
\hspace*{6pt}\hspace*{6pt}\hspace*{6pt}{<\textbf{f}\hspace*{6pt}{name}="{frequency}">}\mbox{}\newline 
\hspace*{6pt}\hspace*{6pt}\hspace*{6pt}\hspace*{6pt}{<\textbf{numeric}\hspace*{6pt}{value}="{7}"/>}\mbox{}\newline 
\hspace*{6pt}\hspace*{6pt}\hspace*{6pt}{</\textbf{f}>}\mbox{}\newline 
\hspace*{6pt}\hspace*{6pt}{</\textbf{fs}>}\mbox{}\newline 
\hspace*{6pt}{</\textbf{date}>}\mbox{}\newline 
{</\textbf{annotationBlock}>}\end{shaded}\egroup 


    \item[{Exemple}]
  StandOff classification alignement avec le hub de métadonnées\leavevmode\bgroup\exampleFont \begin{shaded}\noindent\mbox{}{<\textbf{annotationBlock}\hspace*{6pt}{corresp}="{\#subject-01}">}\mbox{}\newline 
\hspace*{6pt}{<\textbf{keywords}\hspace*{6pt}{change}="{\#istex-01}"\hspace*{6pt}{resp}="{\#istex}">}\mbox{}\newline 
\hspace*{6pt}\hspace*{6pt}{<\textbf{term}>}Pharmacologie{</\textbf{term}>}\mbox{}\newline 
\hspace*{6pt}{</\textbf{keywords}>}\mbox{}\newline 
{</\textbf{annotationBlock}>}\end{shaded}\egroup 


    \item[{Modèle de contenu}]
  \mbox{}\hfill\\[-10pt]\begin{Verbatim}[fontsize=\small]
<content>
 <alternate maxOccurs="1" minOccurs="1">
  <classRef key="model.annotation"
   maxOccurs="unbounded" minOccurs="1"/>
  <sequence maxOccurs="1" minOccurs="1">
   <classRef key="model.linguisticSegment"
    maxOccurs="1" minOccurs="0"/>
   <classRef key="model.interlinearAnnotation"
    maxOccurs="unbounded" minOccurs="0"/>
  </sequence>
  <sequence maxOccurs="1" minOccurs="1">
   <classRef key="model.OABody"
    maxOccurs="unbounded" minOccurs="0"/>
   <classRef key="model.OAAnnotation"
    maxOccurs="1" minOccurs="1"/>
   <classRef key="model.OATarget"
    maxOccurs="unbounded" minOccurs="0"/>
  </sequence>
 </alternate>
</content>
    
\end{Verbatim}

    \item[{Schéma Declaration}]
  \mbox{}\hfill\\[-10pt]\begin{Verbatim}[fontsize=\small]
element annotationBlock
{
   tei_att.notated.attributes,
   tei_att.typed.attributes,
   tei_att.ascribed.attributes,
   tei_att.timed.attributes,
   tei_att.global.attributes,
   (
      tei_model.annotation+
    | ( model.linguisticSegment?, model.interlinearAnnotation* )
    | ( tei_model.OABody*, tei_model.OAAnnotation, tei_model.OATarget* )
   )
}
\end{Verbatim}

\end{reflist}  \index{appInfo=<appInfo>|oddindex}
\begin{reflist}
\item[]\begin{specHead}{TEI.appInfo}{<appInfo> }(informations d'application) enregistre des informations sur l'application qui a été utilisée pour traiter le fichier TEI. [\xref{http://www.tei-c.org/release/doc/tei-p5-doc/en/html/HD.html\#HDAPP}{2.3.10. The Application Information Element}]\end{specHead} 
    \item[{Module}]
  header
    \item[{Attributs}]
  Attributs \hyperref[TEI.att.global]{att.global} (\textit{@xml:id}, \textit{@n}, \textit{@xml:lang}, \textit{@xml:base}, \textit{@xml:space})  (\hyperref[TEI.att.global.rendition]{att.global.rendition} (\textit{@rend}, \textit{@style}, \textit{@rendition})) (\hyperref[TEI.att.global.linking]{att.global.linking} (\textit{@corresp}, \textit{@synch}, \textit{@sameAs}, \textit{@copyOf}, \textit{@next}, \textit{@prev}, \textit{@exclude}, \textit{@select})) (\hyperref[TEI.att.global.analytic]{att.global.analytic} (\textit{@ana})) (\hyperref[TEI.att.global.facs]{att.global.facs} (\textit{@facs})) (\hyperref[TEI.att.global.change]{att.global.change} (\textit{@change})) (\hyperref[TEI.att.global.responsibility]{att.global.responsibility} (\textit{@cert}, \textit{@resp})) (\hyperref[TEI.att.global.source]{att.global.source} (\textit{@source}))
    \item[{Membre du}]
  \hyperref[TEI.model.encodingDescPart]{model.encodingDescPart}
    \item[{Contenu dans}]
  
    \item[header: ]
   \hyperref[TEI.encodingDesc]{encodingDesc}
    \item[{Peut contenir}]
  
    \item[header: ]
   \hyperref[TEI.application]{application}
    \item[{Exemple}]
  \leavevmode\bgroup\exampleFont \begin{shaded}\noindent\mbox{}{<\textbf{appInfo}>}\mbox{}\newline 
\hspace*{6pt}{<\textbf{application}\hspace*{6pt}{ident}="{Xaira}"\hspace*{6pt}{version}="{1.24}">}\mbox{}\newline 
\hspace*{6pt}\hspace*{6pt}{<\textbf{label}>}XAIRA Indexer{</\textbf{label}>}\mbox{}\newline 
\hspace*{6pt}\hspace*{6pt}{<\textbf{ptr}\hspace*{6pt}{target}="{\#P1}"/>}\mbox{}\newline 
\hspace*{6pt}{</\textbf{application}>}\mbox{}\newline 
{</\textbf{appInfo}>}\end{shaded}\egroup 


    \item[{Modèle de contenu}]
  \mbox{}\hfill\\[-10pt]\begin{Verbatim}[fontsize=\small]
<content>
 <classRef key="model.applicationLike"
  maxOccurs="unbounded" minOccurs="1"/>
</content>
    
\end{Verbatim}

    \item[{Schéma Declaration}]
  \mbox{}\hfill\\[-10pt]\begin{Verbatim}[fontsize=\small]
element appInfo { tei_att.global.attributes, tei_model.applicationLike+ }
\end{Verbatim}

\end{reflist}  \index{application=<application>|oddindex}\index{ident=@ident!<application>|oddindex}\index{version=@version!<application>|oddindex}
\begin{reflist}
\item[]\begin{specHead}{TEI.application}{<application> }fournit des informations sur une application qui a été utilisée pour le traitement du document. [\xref{http://www.tei-c.org/release/doc/tei-p5-doc/en/html/HD.html\#HDAPP}{2.3.10. The Application Information Element}]\end{specHead} 
    \item[{Module}]
  header
    \item[{Attributs}]
  Attributs \hyperref[TEI.att.global]{att.global} (\textit{@xml:id}, \textit{@n}, \textit{@xml:lang}, \textit{@xml:base}, \textit{@xml:space})  (\hyperref[TEI.att.global.rendition]{att.global.rendition} (\textit{@rend}, \textit{@style}, \textit{@rendition})) (\hyperref[TEI.att.global.linking]{att.global.linking} (\textit{@corresp}, \textit{@synch}, \textit{@sameAs}, \textit{@copyOf}, \textit{@next}, \textit{@prev}, \textit{@exclude}, \textit{@select})) (\hyperref[TEI.att.global.analytic]{att.global.analytic} (\textit{@ana})) (\hyperref[TEI.att.global.facs]{att.global.facs} (\textit{@facs})) (\hyperref[TEI.att.global.change]{att.global.change} (\textit{@change})) (\hyperref[TEI.att.global.responsibility]{att.global.responsibility} (\textit{@cert}, \textit{@resp})) (\hyperref[TEI.att.global.source]{att.global.source} (\textit{@source})) \hyperref[TEI.att.typed]{att.typed} (\textit{@type}, \textit{@subtype}) \hyperref[TEI.att.datable]{att.datable} (\textit{@calendar}, \textit{@period})  (\hyperref[TEI.att.datable.w3c]{att.datable.w3c} (\textit{@when}, \textit{@notBefore}, \textit{@notAfter}, \textit{@from}, \textit{@to})) (\hyperref[TEI.att.datable.iso]{att.datable.iso} (\textit{@when-iso}, \textit{@notBefore-iso}, \textit{@notAfter-iso}, \textit{@from-iso}, \textit{@to-iso})) (\hyperref[TEI.att.datable.custom]{att.datable.custom} (\textit{@when-custom}, \textit{@notBefore-custom}, \textit{@notAfter-custom}, \textit{@from-custom}, \textit{@to-custom}, \textit{@datingPoint}, \textit{@datingMethod})) \hfil\\[-10pt]\begin{sansreflist}
    \item[@ident]
  fournit un identifiant pour l'application, indépendamment de son numéro de version ou du nom affiché.
\begin{reflist}
    \item[{Statut}]
  Requis
    \item[{Type de données}]
  \hyperref[TEI.teidata.name]{teidata.name}
\end{reflist}  
    \item[@version]
  fournit un numéro de version pour l'application, indépendamment de son identifiant ou du nom affiché.
\begin{reflist}
    \item[{Statut}]
  Optionel
    \item[{Type de données}]
  \hyperref[TEI.teidata.versionNumber]{teidata.versionNumber}
\end{reflist}  
\end{sansreflist}  
    \item[{Membre du}]
  \hyperref[TEI.model.applicationLike]{model.applicationLike}
    \item[{Contenu dans}]
  
    \item[header: ]
   \hyperref[TEI.appInfo]{appInfo}
    \item[{Peut contenir}]
  
    \item[core: ]
   \hyperref[TEI.desc]{desc} \hyperref[TEI.label]{label} \hyperref[TEI.p]{p} \hyperref[TEI.ptr]{ptr} \hyperref[TEI.ref]{ref}\par 
    \item[linking: ]
   \hyperref[TEI.ab]{ab}
    \item[{Exemple}]
  \leavevmode\bgroup\exampleFont \begin{shaded}\noindent\mbox{}{<\textbf{appInfo}>}\mbox{}\newline 
\hspace*{6pt}{<\textbf{application}\hspace*{6pt}{ident}="{Xaira}"\hspace*{6pt}{version}="{1.24}">}\mbox{}\newline 
\hspace*{6pt}\hspace*{6pt}{<\textbf{label}>}XAIRA Indexer{</\textbf{label}>}\mbox{}\newline 
\hspace*{6pt}\hspace*{6pt}{<\textbf{ptr}\hspace*{6pt}{target}="{\#fr\textunderscore HD}"/>}\mbox{}\newline 
\hspace*{6pt}{</\textbf{application}>}\mbox{}\newline 
{</\textbf{appInfo}>}\end{shaded}\egroup 


    \item[{Modèle de contenu}]
  \mbox{}\hfill\\[-10pt]\begin{Verbatim}[fontsize=\small]
<content>
 <sequence maxOccurs="1" minOccurs="1">
  <classRef key="model.labelLike"
   maxOccurs="unbounded" minOccurs="1"/>
  <alternate maxOccurs="1" minOccurs="1">
   <classRef key="model.ptrLike"
    maxOccurs="unbounded" minOccurs="0"/>
   <classRef key="model.pLike"
    maxOccurs="unbounded" minOccurs="0"/>
  </alternate>
 </sequence>
</content>
    
\end{Verbatim}

    \item[{Schéma Declaration}]
  \mbox{}\hfill\\[-10pt]\begin{Verbatim}[fontsize=\small]
element application
{
   tei_att.global.attributes,
   tei_att.typed.attributes,
   tei_att.datable.attributes,
   attribute ident { text },
   attribute version { text }?,
   ( tei_model.labelLike+, ( tei_model.ptrLike* | tei_model.pLike* ) )
}
\end{Verbatim}

\end{reflist}  \index{author=<author>|oddindex}
\begin{reflist}
\item[]\begin{specHead}{TEI.author}{<author> }(auteur) dans une référence bibliographique contient le nom de la (des) personne(s) physique(s) ou du collectif, auteur(s) d'une oeuvre ; par exemple dans la même forme que celle utilisée par une référence bibliographique reconnue. [\xref{http://www.tei-c.org/release/doc/tei-p5-doc/en/html/CO.html\#COBICOR}{3.11.2.2. Titles, Authors, and Editors} \xref{http://www.tei-c.org/release/doc/tei-p5-doc/en/html/HD.html\#HD21}{2.2.1. The Title Statement}]\end{specHead} 
    \item[{Module}]
  core
    \item[{Attributs}]
  Attributs \hyperref[TEI.att.global]{att.global} (\textit{@xml:id}, \textit{@n}, \textit{@xml:lang}, \textit{@xml:base}, \textit{@xml:space})  (\hyperref[TEI.att.global.rendition]{att.global.rendition} (\textit{@rend}, \textit{@style}, \textit{@rendition})) (\hyperref[TEI.att.global.linking]{att.global.linking} (\textit{@corresp}, \textit{@synch}, \textit{@sameAs}, \textit{@copyOf}, \textit{@next}, \textit{@prev}, \textit{@exclude}, \textit{@select})) (\hyperref[TEI.att.global.analytic]{att.global.analytic} (\textit{@ana})) (\hyperref[TEI.att.global.facs]{att.global.facs} (\textit{@facs})) (\hyperref[TEI.att.global.change]{att.global.change} (\textit{@change})) (\hyperref[TEI.att.global.responsibility]{att.global.responsibility} (\textit{@cert}, \textit{@resp})) (\hyperref[TEI.att.global.source]{att.global.source} (\textit{@source})) \hyperref[TEI.att.naming]{att.naming} (\textit{@role}, \textit{@nymRef})  (\hyperref[TEI.att.canonical]{att.canonical} (\textit{@key}, \textit{@ref}))
    \item[{Membre du}]
  \hyperref[TEI.model.annotation]{model.annotation} \hyperref[TEI.model.respLike]{model.respLike} 
    \item[{Contenu dans}]
  
    \item[core: ]
   \hyperref[TEI.analytic]{analytic} \hyperref[TEI.bibl]{bibl} \hyperref[TEI.monogr]{monogr}\par 
    \item[header: ]
   \hyperref[TEI.editionStmt]{editionStmt} \hyperref[TEI.titleStmt]{titleStmt}\par 
    \item[msdescription: ]
   \hyperref[TEI.msItem]{msItem} \hyperref[TEI.msItemStruct]{msItemStruct}\par 
    \item[spoken: ]
   \hyperref[TEI.annotationBlock]{annotationBlock}\par 
    \item[standOff: ]
   \hyperref[TEI.listAnnotation]{listAnnotation}
    \item[{Peut contenir}]
  
    \item[analysis: ]
   \hyperref[TEI.c]{c} \hyperref[TEI.cl]{cl} \hyperref[TEI.interp]{interp} \hyperref[TEI.interpGrp]{interpGrp} \hyperref[TEI.m]{m} \hyperref[TEI.pc]{pc} \hyperref[TEI.phr]{phr} \hyperref[TEI.s]{s} \hyperref[TEI.span]{span} \hyperref[TEI.spanGrp]{spanGrp} \hyperref[TEI.w]{w}\par 
    \item[core: ]
   \hyperref[TEI.abbr]{abbr} \hyperref[TEI.add]{add} \hyperref[TEI.address]{address} \hyperref[TEI.binaryObject]{binaryObject} \hyperref[TEI.cb]{cb} \hyperref[TEI.choice]{choice} \hyperref[TEI.corr]{corr} \hyperref[TEI.date]{date} \hyperref[TEI.del]{del} \hyperref[TEI.distinct]{distinct} \hyperref[TEI.email]{email} \hyperref[TEI.emph]{emph} \hyperref[TEI.expan]{expan} \hyperref[TEI.foreign]{foreign} \hyperref[TEI.gap]{gap} \hyperref[TEI.gb]{gb} \hyperref[TEI.gloss]{gloss} \hyperref[TEI.graphic]{graphic} \hyperref[TEI.hi]{hi} \hyperref[TEI.index]{index} \hyperref[TEI.lb]{lb} \hyperref[TEI.measure]{measure} \hyperref[TEI.measureGrp]{measureGrp} \hyperref[TEI.media]{media} \hyperref[TEI.mentioned]{mentioned} \hyperref[TEI.milestone]{milestone} \hyperref[TEI.name]{name} \hyperref[TEI.note]{note} \hyperref[TEI.num]{num} \hyperref[TEI.orig]{orig} \hyperref[TEI.pb]{pb} \hyperref[TEI.ptr]{ptr} \hyperref[TEI.ref]{ref} \hyperref[TEI.reg]{reg} \hyperref[TEI.rs]{rs} \hyperref[TEI.sic]{sic} \hyperref[TEI.soCalled]{soCalled} \hyperref[TEI.term]{term} \hyperref[TEI.time]{time} \hyperref[TEI.title]{title} \hyperref[TEI.unclear]{unclear}\par 
    \item[derived-module-tei.istex: ]
   \hyperref[TEI.math]{math} \hyperref[TEI.mrow]{mrow}\par 
    \item[figures: ]
   \hyperref[TEI.figure]{figure} \hyperref[TEI.formula]{formula} \hyperref[TEI.notatedMusic]{notatedMusic}\par 
    \item[header: ]
   \hyperref[TEI.idno]{idno}\par 
    \item[iso-fs: ]
   \hyperref[TEI.fLib]{fLib} \hyperref[TEI.fs]{fs} \hyperref[TEI.fvLib]{fvLib}\par 
    \item[linking: ]
   \hyperref[TEI.alt]{alt} \hyperref[TEI.altGrp]{altGrp} \hyperref[TEI.anchor]{anchor} \hyperref[TEI.join]{join} \hyperref[TEI.joinGrp]{joinGrp} \hyperref[TEI.link]{link} \hyperref[TEI.linkGrp]{linkGrp} \hyperref[TEI.seg]{seg} \hyperref[TEI.timeline]{timeline}\par 
    \item[msdescription: ]
   \hyperref[TEI.catchwords]{catchwords} \hyperref[TEI.depth]{depth} \hyperref[TEI.dim]{dim} \hyperref[TEI.dimensions]{dimensions} \hyperref[TEI.height]{height} \hyperref[TEI.heraldry]{heraldry} \hyperref[TEI.locus]{locus} \hyperref[TEI.locusGrp]{locusGrp} \hyperref[TEI.material]{material} \hyperref[TEI.objectType]{objectType} \hyperref[TEI.origDate]{origDate} \hyperref[TEI.origPlace]{origPlace} \hyperref[TEI.secFol]{secFol} \hyperref[TEI.signatures]{signatures} \hyperref[TEI.source]{source} \hyperref[TEI.stamp]{stamp} \hyperref[TEI.watermark]{watermark} \hyperref[TEI.width]{width}\par 
    \item[namesdates: ]
   \hyperref[TEI.addName]{addName} \hyperref[TEI.affiliation]{affiliation} \hyperref[TEI.country]{country} \hyperref[TEI.forename]{forename} \hyperref[TEI.genName]{genName} \hyperref[TEI.geogName]{geogName} \hyperref[TEI.location]{location} \hyperref[TEI.nameLink]{nameLink} \hyperref[TEI.orgName]{orgName} \hyperref[TEI.persName]{persName} \hyperref[TEI.placeName]{placeName} \hyperref[TEI.region]{region} \hyperref[TEI.roleName]{roleName} \hyperref[TEI.settlement]{settlement} \hyperref[TEI.state]{state} \hyperref[TEI.surname]{surname}\par 
    \item[spoken: ]
   \hyperref[TEI.annotationBlock]{annotationBlock}\par 
    \item[transcr: ]
   \hyperref[TEI.addSpan]{addSpan} \hyperref[TEI.am]{am} \hyperref[TEI.damage]{damage} \hyperref[TEI.damageSpan]{damageSpan} \hyperref[TEI.delSpan]{delSpan} \hyperref[TEI.ex]{ex} \hyperref[TEI.fw]{fw} \hyperref[TEI.handShift]{handShift} \hyperref[TEI.listTranspose]{listTranspose} \hyperref[TEI.metamark]{metamark} \hyperref[TEI.mod]{mod} \hyperref[TEI.redo]{redo} \hyperref[TEI.restore]{restore} \hyperref[TEI.retrace]{retrace} \hyperref[TEI.secl]{secl} \hyperref[TEI.space]{space} \hyperref[TEI.subst]{subst} \hyperref[TEI.substJoin]{substJoin} \hyperref[TEI.supplied]{supplied} \hyperref[TEI.surplus]{surplus} \hyperref[TEI.undo]{undo}\par des données textuelles
    \item[{Exemple}]
  StandOff enrichissement auteur hub de métadonnées Abes\leavevmode\bgroup\exampleFont \begin{shaded}\noindent\mbox{}{<\textbf{author}>}\mbox{}\newline 
\hspace*{6pt}{<\textbf{name}\hspace*{6pt}{change}="{\#author-01}"\hspace*{6pt}{resp}="{\#abes}"\mbox{}\newline 
\hspace*{6pt}\hspace*{6pt}{type}="{person}">}Nom\mbox{}\newline 
\hspace*{6pt}\hspace*{6pt} Prénom{</\textbf{name}>}\mbox{}\newline 
\hspace*{6pt}{<\textbf{persName}>}\mbox{}\newline 
\hspace*{6pt}\hspace*{6pt}{<\textbf{forename}\hspace*{6pt}{change}="{\#author-01}"\mbox{}\newline 
\hspace*{6pt}\hspace*{6pt}\hspace*{6pt}{resp}="{\#abes}"\hspace*{6pt}{type}="{first}">}Prénom{</\textbf{forename}>}\mbox{}\newline 
\hspace*{6pt}\hspace*{6pt}{<\textbf{surname}\hspace*{6pt}{change}="{\#author-01}"\hspace*{6pt}{resp}="{\#abes}">}Nom{</\textbf{surname}>}\mbox{}\newline 
\hspace*{6pt}{</\textbf{persName}>}\mbox{}\newline 
{</\textbf{author}>}\end{shaded}\egroup 


    \item[{Modèle de contenu}]
  \mbox{}\hfill\\[-10pt]\begin{Verbatim}[fontsize=\small]
<content>
 <macroRef key="macro.phraseSeq"/>
</content>
    
\end{Verbatim}

    \item[{Schéma Declaration}]
  \mbox{}\hfill\\[-10pt]\begin{Verbatim}[fontsize=\small]
element author
{
   tei_att.global.attributes,
   tei_att.naming.attributes,
   tei_macro.phraseSeq}
\end{Verbatim}

\end{reflist}  \index{authority=<authority>|oddindex}
\begin{reflist}
\item[]\begin{specHead}{TEI.authority}{<authority> }(responsable de la publication.) donne le nom de la personne ou de l'organisme responsable de la publication d’un fichier électronique, autre qu’un éditeur ou un distributeur. [\xref{http://www.tei-c.org/release/doc/tei-p5-doc/en/html/HD.html\#HD24}{2.2.4. Publication, Distribution, Licensing, etc.}]\end{specHead} 
    \item[{Module}]
  header
    \item[{Attributs}]
  Attributs \hyperref[TEI.att.global]{att.global} (\textit{@xml:id}, \textit{@n}, \textit{@xml:lang}, \textit{@xml:base}, \textit{@xml:space})  (\hyperref[TEI.att.global.rendition]{att.global.rendition} (\textit{@rend}, \textit{@style}, \textit{@rendition})) (\hyperref[TEI.att.global.linking]{att.global.linking} (\textit{@corresp}, \textit{@synch}, \textit{@sameAs}, \textit{@copyOf}, \textit{@next}, \textit{@prev}, \textit{@exclude}, \textit{@select})) (\hyperref[TEI.att.global.analytic]{att.global.analytic} (\textit{@ana})) (\hyperref[TEI.att.global.facs]{att.global.facs} (\textit{@facs})) (\hyperref[TEI.att.global.change]{att.global.change} (\textit{@change})) (\hyperref[TEI.att.global.responsibility]{att.global.responsibility} (\textit{@cert}, \textit{@resp})) (\hyperref[TEI.att.global.source]{att.global.source} (\textit{@source}))
    \item[{Membre du}]
  \hyperref[TEI.model.publicationStmtPart.agency]{model.publicationStmtPart.agency} 
    \item[{Contenu dans}]
  
    \item[core: ]
   \hyperref[TEI.monogr]{monogr}\par 
    \item[header: ]
   \hyperref[TEI.publicationStmt]{publicationStmt}
    \item[{Peut contenir}]
  
    \item[analysis: ]
   \hyperref[TEI.interp]{interp} \hyperref[TEI.interpGrp]{interpGrp} \hyperref[TEI.span]{span} \hyperref[TEI.spanGrp]{spanGrp}\par 
    \item[core: ]
   \hyperref[TEI.abbr]{abbr} \hyperref[TEI.address]{address} \hyperref[TEI.cb]{cb} \hyperref[TEI.choice]{choice} \hyperref[TEI.date]{date} \hyperref[TEI.distinct]{distinct} \hyperref[TEI.email]{email} \hyperref[TEI.emph]{emph} \hyperref[TEI.expan]{expan} \hyperref[TEI.foreign]{foreign} \hyperref[TEI.gap]{gap} \hyperref[TEI.gb]{gb} \hyperref[TEI.gloss]{gloss} \hyperref[TEI.hi]{hi} \hyperref[TEI.index]{index} \hyperref[TEI.lb]{lb} \hyperref[TEI.measure]{measure} \hyperref[TEI.measureGrp]{measureGrp} \hyperref[TEI.mentioned]{mentioned} \hyperref[TEI.milestone]{milestone} \hyperref[TEI.name]{name} \hyperref[TEI.note]{note} \hyperref[TEI.num]{num} \hyperref[TEI.pb]{pb} \hyperref[TEI.ptr]{ptr} \hyperref[TEI.ref]{ref} \hyperref[TEI.rs]{rs} \hyperref[TEI.soCalled]{soCalled} \hyperref[TEI.term]{term} \hyperref[TEI.time]{time} \hyperref[TEI.title]{title}\par 
    \item[figures: ]
   \hyperref[TEI.figure]{figure} \hyperref[TEI.notatedMusic]{notatedMusic}\par 
    \item[header: ]
   \hyperref[TEI.idno]{idno}\par 
    \item[iso-fs: ]
   \hyperref[TEI.fLib]{fLib} \hyperref[TEI.fs]{fs} \hyperref[TEI.fvLib]{fvLib}\par 
    \item[linking: ]
   \hyperref[TEI.alt]{alt} \hyperref[TEI.altGrp]{altGrp} \hyperref[TEI.anchor]{anchor} \hyperref[TEI.join]{join} \hyperref[TEI.joinGrp]{joinGrp} \hyperref[TEI.link]{link} \hyperref[TEI.linkGrp]{linkGrp} \hyperref[TEI.timeline]{timeline}\par 
    \item[msdescription: ]
   \hyperref[TEI.catchwords]{catchwords} \hyperref[TEI.depth]{depth} \hyperref[TEI.dim]{dim} \hyperref[TEI.dimensions]{dimensions} \hyperref[TEI.height]{height} \hyperref[TEI.heraldry]{heraldry} \hyperref[TEI.locus]{locus} \hyperref[TEI.locusGrp]{locusGrp} \hyperref[TEI.material]{material} \hyperref[TEI.objectType]{objectType} \hyperref[TEI.origDate]{origDate} \hyperref[TEI.origPlace]{origPlace} \hyperref[TEI.secFol]{secFol} \hyperref[TEI.signatures]{signatures} \hyperref[TEI.source]{source} \hyperref[TEI.stamp]{stamp} \hyperref[TEI.watermark]{watermark} \hyperref[TEI.width]{width}\par 
    \item[namesdates: ]
   \hyperref[TEI.addName]{addName} \hyperref[TEI.affiliation]{affiliation} \hyperref[TEI.country]{country} \hyperref[TEI.forename]{forename} \hyperref[TEI.genName]{genName} \hyperref[TEI.geogName]{geogName} \hyperref[TEI.location]{location} \hyperref[TEI.nameLink]{nameLink} \hyperref[TEI.orgName]{orgName} \hyperref[TEI.persName]{persName} \hyperref[TEI.placeName]{placeName} \hyperref[TEI.region]{region} \hyperref[TEI.roleName]{roleName} \hyperref[TEI.settlement]{settlement} \hyperref[TEI.state]{state} \hyperref[TEI.surname]{surname}\par 
    \item[transcr: ]
   \hyperref[TEI.addSpan]{addSpan} \hyperref[TEI.am]{am} \hyperref[TEI.damageSpan]{damageSpan} \hyperref[TEI.delSpan]{delSpan} \hyperref[TEI.ex]{ex} \hyperref[TEI.fw]{fw} \hyperref[TEI.listTranspose]{listTranspose} \hyperref[TEI.metamark]{metamark} \hyperref[TEI.space]{space} \hyperref[TEI.subst]{subst} \hyperref[TEI.substJoin]{substJoin}\par des données textuelles
    \item[{Exemple}]
  \leavevmode\bgroup\exampleFont \begin{shaded}\noindent\mbox{}{<\textbf{authority}>}A. D.{</\textbf{authority}>}\end{shaded}\egroup 


    \item[{Modèle de contenu}]
  \mbox{}\hfill\\[-10pt]\begin{Verbatim}[fontsize=\small]
<content>
 <macroRef key="macro.phraseSeq.limited"/>
</content>
    
\end{Verbatim}

    \item[{Schéma Declaration}]
  \mbox{}\hfill\\[-10pt]\begin{Verbatim}[fontsize=\small]
element authority { tei_att.global.attributes, tei_macro.phraseSeq.limited }
\end{Verbatim}

\end{reflist}  \index{availability=<availability>|oddindex}\index{status=@status!<availability>|oddindex}
\begin{reflist}
\item[]\begin{specHead}{TEI.availability}{<availability> }(disponibilité) renseigne sur la disponibilité du texte, par exemple sur toutes restrictions quant à son usage ou sa diffusion, son copyright, etc. [\xref{http://www.tei-c.org/release/doc/tei-p5-doc/en/html/HD.html\#HD24}{2.2.4. Publication, Distribution, Licensing, etc.}]\end{specHead} 
    \item[{Module}]
  header
    \item[{Attributs}]
  Attributs \hyperref[TEI.att.global]{att.global} (\textit{@xml:id}, \textit{@n}, \textit{@xml:lang}, \textit{@xml:base}, \textit{@xml:space})  (\hyperref[TEI.att.global.rendition]{att.global.rendition} (\textit{@rend}, \textit{@style}, \textit{@rendition})) (\hyperref[TEI.att.global.linking]{att.global.linking} (\textit{@corresp}, \textit{@synch}, \textit{@sameAs}, \textit{@copyOf}, \textit{@next}, \textit{@prev}, \textit{@exclude}, \textit{@select})) (\hyperref[TEI.att.global.analytic]{att.global.analytic} (\textit{@ana})) (\hyperref[TEI.att.global.facs]{att.global.facs} (\textit{@facs})) (\hyperref[TEI.att.global.change]{att.global.change} (\textit{@change})) (\hyperref[TEI.att.global.responsibility]{att.global.responsibility} (\textit{@cert}, \textit{@resp})) (\hyperref[TEI.att.global.source]{att.global.source} (\textit{@source})) \hyperref[TEI.att.declarable]{att.declarable} (\textit{@default}) \hfil\\[-10pt]\begin{sansreflist}
    \item[@status]
  donne un code caractérisant la disponibilité actuelle d’un texte.
\begin{reflist}
    \item[{Statut}]
  Optionel
    \item[{Type de données}]
  \hyperref[TEI.teidata.enumerated]{teidata.enumerated}
    \item[{Les valeurs autorisées sont:}]
  \begin{description}

\item[{free}]Le texte est libre de droits.
\item[{unknown}]Le statut du texte est inconnu.{[Valeur par défaut] \xref{http://www.tei-c.org/Activities/Council/Working/tcw27.xml}{Deprecated}. The value will no longer be a default after 2017-09-05.}
\item[{restricted}]le texte est sous droits.
\end{description} 
\end{reflist}  
\end{sansreflist}  
    \item[{Membre du}]
  \hyperref[TEI.model.biblPart]{model.biblPart} \hyperref[TEI.model.publicationStmtPart.detail]{model.publicationStmtPart.detail} 
    \item[{Contenu dans}]
  
    \item[core: ]
   \hyperref[TEI.analytic]{analytic} \hyperref[TEI.bibl]{bibl} \hyperref[TEI.monogr]{monogr} \hyperref[TEI.series]{series}\par 
    \item[header: ]
   \hyperref[TEI.publicationStmt]{publicationStmt}\par 
    \item[msdescription: ]
   \hyperref[TEI.adminInfo]{adminInfo}
    \item[{Peut contenir}]
  
    \item[core: ]
   \hyperref[TEI.p]{p}\par 
    \item[header: ]
   \hyperref[TEI.licence]{licence}\par 
    \item[linking: ]
   \hyperref[TEI.ab]{ab}
    \item[{Note}]
  \par
On devrait adopter un format de codage reconnu.
    \item[{Exemple}]
  \leavevmode\bgroup\exampleFont \begin{shaded}\noindent\mbox{}{<\textbf{availability}\hspace*{6pt}{status}="{restricted}">}\mbox{}\newline 
\hspace*{6pt}{<\textbf{p}>}L' ABES a adopté le système Créative Commons pour permettre à tous ceux qui le\mbox{}\newline 
\hspace*{6pt}\hspace*{6pt} souhaitent, de reproduire tout ou partie des rubriques du site de l'ABES sur support\mbox{}\newline 
\hspace*{6pt}\hspace*{6pt} papier ou support électronique.{</\textbf{p}>}\mbox{}\newline 
{</\textbf{availability}>}\end{shaded}\egroup 


    \item[{Modèle de contenu}]
  \mbox{}\hfill\\[-10pt]\begin{Verbatim}[fontsize=\small]
<content>
 <alternate maxOccurs="unbounded"
  minOccurs="1">
  <classRef key="model.availabilityPart"/>
  <classRef key="model.pLike"/>
 </alternate>
</content>
    
\end{Verbatim}

    \item[{Schéma Declaration}]
  \mbox{}\hfill\\[-10pt]\begin{Verbatim}[fontsize=\small]
element availability
{
   tei_att.global.attributes,
   tei_att.declarable.attributes,
   attribute status { "free" | "unknown" | "restricted" }?,
   ( tei_model.availabilityPart | tei_model.pLike )+
}
\end{Verbatim}

\end{reflist}  \index{back=<back>|oddindex}
\begin{reflist}
\item[]\begin{specHead}{TEI.back}{<back> }(texte annexe) contient tout supplément placé après la partie principale d'un texte : appendice, etc. [\xref{http://www.tei-c.org/release/doc/tei-p5-doc/en/html/DS.html\#DSBACK}{4.7. Back Matter} \xref{http://www.tei-c.org/release/doc/tei-p5-doc/en/html/DS.html\#DS}{4. Default Text Structure}]\end{specHead} 
    \item[{Module}]
  textstructure
    \item[{Attributs}]
  Attributs \hyperref[TEI.att.global]{att.global} (\textit{@xml:id}, \textit{@n}, \textit{@xml:lang}, \textit{@xml:base}, \textit{@xml:space})  (\hyperref[TEI.att.global.rendition]{att.global.rendition} (\textit{@rend}, \textit{@style}, \textit{@rendition})) (\hyperref[TEI.att.global.linking]{att.global.linking} (\textit{@corresp}, \textit{@synch}, \textit{@sameAs}, \textit{@copyOf}, \textit{@next}, \textit{@prev}, \textit{@exclude}, \textit{@select})) (\hyperref[TEI.att.global.analytic]{att.global.analytic} (\textit{@ana})) (\hyperref[TEI.att.global.facs]{att.global.facs} (\textit{@facs})) (\hyperref[TEI.att.global.change]{att.global.change} (\textit{@change})) (\hyperref[TEI.att.global.responsibility]{att.global.responsibility} (\textit{@cert}, \textit{@resp})) (\hyperref[TEI.att.global.source]{att.global.source} (\textit{@source})) \hyperref[TEI.att.declaring]{att.declaring} (\textit{@decls}) 
    \item[{Contenu dans}]
  
    \item[textstructure: ]
   \hyperref[TEI.floatingText]{floatingText} \hyperref[TEI.text]{text}\par 
    \item[transcr: ]
   \hyperref[TEI.facsimile]{facsimile}
    \item[{Peut contenir}]
  
    \item[analysis: ]
   \hyperref[TEI.interp]{interp} \hyperref[TEI.interpGrp]{interpGrp} \hyperref[TEI.span]{span} \hyperref[TEI.spanGrp]{spanGrp}\par 
    \item[core: ]
   \hyperref[TEI.cb]{cb} \hyperref[TEI.divGen]{divGen} \hyperref[TEI.gap]{gap} \hyperref[TEI.gb]{gb} \hyperref[TEI.head]{head} \hyperref[TEI.index]{index} \hyperref[TEI.lb]{lb} \hyperref[TEI.list]{list} \hyperref[TEI.listBibl]{listBibl} \hyperref[TEI.milestone]{milestone} \hyperref[TEI.note]{note} \hyperref[TEI.p]{p} \hyperref[TEI.pb]{pb}\par 
    \item[figures: ]
   \hyperref[TEI.figure]{figure} \hyperref[TEI.notatedMusic]{notatedMusic} \hyperref[TEI.table]{table}\par 
    \item[iso-fs: ]
   \hyperref[TEI.fLib]{fLib} \hyperref[TEI.fs]{fs} \hyperref[TEI.fvLib]{fvLib}\par 
    \item[linking: ]
   \hyperref[TEI.ab]{ab} \hyperref[TEI.alt]{alt} \hyperref[TEI.altGrp]{altGrp} \hyperref[TEI.anchor]{anchor} \hyperref[TEI.join]{join} \hyperref[TEI.joinGrp]{joinGrp} \hyperref[TEI.link]{link} \hyperref[TEI.linkGrp]{linkGrp} \hyperref[TEI.timeline]{timeline}\par 
    \item[msdescription: ]
   \hyperref[TEI.source]{source}\par 
    \item[namesdates: ]
   \hyperref[TEI.listOrg]{listOrg} \hyperref[TEI.listPlace]{listPlace}\par 
    \item[textstructure: ]
   \hyperref[TEI.div]{div} \hyperref[TEI.docAuthor]{docAuthor} \hyperref[TEI.docDate]{docDate} \hyperref[TEI.docEdition]{docEdition} \hyperref[TEI.docTitle]{docTitle} \hyperref[TEI.titlePage]{titlePage} \hyperref[TEI.titlePart]{titlePart}\par 
    \item[transcr: ]
   \hyperref[TEI.addSpan]{addSpan} \hyperref[TEI.damageSpan]{damageSpan} \hyperref[TEI.delSpan]{delSpan} \hyperref[TEI.fw]{fw} \hyperref[TEI.listTranspose]{listTranspose} \hyperref[TEI.metamark]{metamark} \hyperref[TEI.space]{space} \hyperref[TEI.substJoin]{substJoin}
    \item[{Note}]
  \par
Le modèle de contenu de l'élément \hyperref[TEI.back]{<back>} est identique à celui de l'élément \hyperref[TEI.front]{<front>}, ce qui permet de rendre compte de pratiques éditoriales qui ont évolué avec l'histoire culturelle.
    \item[{Exemple}]
  \leavevmode\bgroup\exampleFont \begin{shaded}\noindent\mbox{}{<\textbf{back}>}\mbox{}\newline 
\hspace*{6pt}{<\textbf{div}\hspace*{6pt}{n}="{1}"\hspace*{6pt}{type}="{appendice}">}\mbox{}\newline 
\hspace*{6pt}\hspace*{6pt}{<\textbf{head}>}APPENDICE I {</\textbf{head}>}\mbox{}\newline 
\hspace*{6pt}\hspace*{6pt}{<\textbf{head}>}CHAPITRE XV bis {</\textbf{head}>}\mbox{}\newline 
\hspace*{6pt}\hspace*{6pt}{<\textbf{p}>}Des cruautez exercées par les Turcs, et autres peuples : et nommément par les\mbox{}\newline 
\hspace*{6pt}\hspace*{6pt}\hspace*{6pt}\hspace*{6pt} Espagnols, beaucoup plus barbares que les Sauvages mesmes {</\textbf{p}>}\mbox{}\newline 
\hspace*{6pt}\hspace*{6pt}{<\textbf{p}>}Premierement Chalcondile en son histoire de la decadence de l'Empire des Grecs, ...{</\textbf{p}>}\mbox{}\newline 
\hspace*{6pt}{</\textbf{div}>}\mbox{}\newline 
\hspace*{6pt}{<\textbf{div}\hspace*{6pt}{n}="{2}"\hspace*{6pt}{type}="{appendice}">}\mbox{}\newline 
\hspace*{6pt}\hspace*{6pt}{<\textbf{head}>} Appendice 2{</\textbf{head}>}\mbox{}\newline 
\hspace*{6pt}\hspace*{6pt}{<\textbf{head}>}Advertissement de l'autheur{</\textbf{head}>}\mbox{}\newline 
\hspace*{6pt}\hspace*{6pt}{<\textbf{p}>}Outre les augmentations bien amples, et la revision beaucoup plus exacte que je n'avoye\mbox{}\newline 
\hspace*{6pt}\hspace*{6pt}\hspace*{6pt}\hspace*{6pt} fait és precedentes Editions, j'ai pour le contentement des Lecteurs, plusieurs endroits\mbox{}\newline 
\hspace*{6pt}\hspace*{6pt}\hspace*{6pt}\hspace*{6pt} de ceste quatrieme et derniere monstré ...{</\textbf{p}>}\mbox{}\newline 
\hspace*{6pt}{</\textbf{div}>}\mbox{}\newline 
{</\textbf{back}>}\end{shaded}\egroup 


    \item[{Modèle de contenu}]
  \mbox{}\hfill\\[-10pt]\begin{Verbatim}[fontsize=\small]
<content>
 <sequence maxOccurs="1" minOccurs="1">
  <alternate maxOccurs="unbounded"
   minOccurs="0">
   <classRef key="model.frontPart"/>
   <classRef key="model.pLike.front"/>
   <classRef key="model.pLike"/>
   <classRef key="model.listLike"/>
   <classRef key="model.global"/>
  </alternate>
  <alternate maxOccurs="1" minOccurs="0">
   <sequence maxOccurs="1" minOccurs="1">
    <classRef key="model.div1Like"/>
    <alternate maxOccurs="unbounded"
     minOccurs="0">
     <classRef key="model.frontPart"/>
     <classRef key="model.div1Like"/>
     <classRef key="model.global"/>
    </alternate>
   </sequence>
   <sequence maxOccurs="1" minOccurs="1">
    <classRef key="model.divLike"/>
    <alternate maxOccurs="unbounded"
     minOccurs="0">
     <classRef key="model.frontPart"/>
     <classRef key="model.divLike"/>
     <classRef key="model.global"/>
    </alternate>
   </sequence>
  </alternate>
  <sequence maxOccurs="1" minOccurs="0">
   <classRef key="model.divBottomPart"/>
   <alternate maxOccurs="unbounded"
    minOccurs="0">
    <classRef key="model.divBottomPart"/>
    <classRef key="model.global"/>
   </alternate>
  </sequence>
 </sequence>
</content>
    
\end{Verbatim}

    \item[{Schéma Declaration}]
  \mbox{}\hfill\\[-10pt]\begin{Verbatim}[fontsize=\small]
element back
{
   tei_att.global.attributes,
   tei_att.declaring.attributes,
   (
      (
         tei_model.frontPart       | tei_model.pLike.front       | tei_model.pLike       | tei_model.listLike       | tei_model.global      )*,
      (
         (
            tei_model.div1Like,
            ( tei_model.frontPart | tei_model.div1Like | tei_model.global )*
         )
       | (
            tei_model.divLike,
            ( tei_model.frontPart | tei_model.divLike | tei_model.global )*
         )
      )?,
      (
         tei_model.divBottomPart,
         ( tei_model.divBottomPart | tei_model.global )*
      )?
   )
}
\end{Verbatim}

\end{reflist}  \index{bibl=<bibl>|oddindex}\index{scheme=@scheme!<bibl>|oddindex}
\begin{reflist}
\item[]\begin{specHead}{TEI.bibl}{<bibl> }(référence bibliographique.) contient une référence bibliographique faiblement structurée dans laquelle les sous-composants peuvent ou non être explicitement balisés. [\xref{http://www.tei-c.org/release/doc/tei-p5-doc/en/html/CO.html\#COBITY}{3.11.1. Methods of Encoding Bibliographic References and Lists of References} \xref{http://www.tei-c.org/release/doc/tei-p5-doc/en/html/HD.html\#HD3}{2.2.7. The Source Description} \xref{http://www.tei-c.org/release/doc/tei-p5-doc/en/html/CC.html\#CCAS2}{15.3.2. Declarable Elements}]\end{specHead} 
    \item[{Module}]
  core
    \item[{Attributs}]
  Attributs \hyperref[TEI.att.declarable]{att.declarable} (\textit{@default}) \hyperref[TEI.att.docStatus]{att.docStatus} (\textit{@status}) \hyperref[TEI.att.global]{att.global} (\textit{@xml:id}, \textit{@n}, \textit{@xml:lang}, \textit{@xml:base}, \textit{@xml:space})  (\hyperref[TEI.att.global.rendition]{att.global.rendition} (\textit{@rend}, \textit{@style}, \textit{@rendition})) (\hyperref[TEI.att.global.linking]{att.global.linking} (\textit{@corresp}, \textit{@synch}, \textit{@sameAs}, \textit{@copyOf}, \textit{@next}, \textit{@prev}, \textit{@exclude}, \textit{@select})) (\hyperref[TEI.att.global.analytic]{att.global.analytic} (\textit{@ana})) (\hyperref[TEI.att.global.facs]{att.global.facs} (\textit{@facs})) (\hyperref[TEI.att.global.change]{att.global.change} (\textit{@change})) (\hyperref[TEI.att.global.responsibility]{att.global.responsibility} (\textit{@cert}, \textit{@resp})) (\hyperref[TEI.att.global.source]{att.global.source} (\textit{@source})) \hyperref[TEI.att.sortable]{att.sortable} (\textit{@sortKey}) \hyperref[TEI.att.timed]{att.timed} (\textit{@start}, \textit{@end})  (\hyperref[TEI.att.duration]{att.duration} (\hyperref[TEI.att.duration.w3c]{att.duration.w3c} (\textit{@dur})) (\hyperref[TEI.att.duration.iso]{att.duration.iso} (\textit{@dur-iso})) ) \hyperref[TEI.att.typed]{att.typed} (\textit{@type}, \textit{@subtype}) \hfil\\[-10pt]\begin{sansreflist}
    \item[@scheme]
  désigne la liste des ontologies dans lequel l'ensemble des termes concernés sont définis.
\begin{reflist}
    \item[{Statut}]
  Optionel
    \item[{Type de données}]
  \hyperref[TEI.teidata.pointer]{teidata.pointer}
\end{reflist}  
\end{sansreflist}  
    \item[{Membre du}]
  \hyperref[TEI.model.biblLike]{model.biblLike} \hyperref[TEI.model.biblPart]{model.biblPart} 
    \item[{Contenu dans}]
  
    \item[core: ]
   \hyperref[TEI.add]{add} \hyperref[TEI.bibl]{bibl} \hyperref[TEI.cit]{cit} \hyperref[TEI.corr]{corr} \hyperref[TEI.del]{del} \hyperref[TEI.desc]{desc} \hyperref[TEI.emph]{emph} \hyperref[TEI.head]{head} \hyperref[TEI.hi]{hi} \hyperref[TEI.item]{item} \hyperref[TEI.l]{l} \hyperref[TEI.listBibl]{listBibl} \hyperref[TEI.meeting]{meeting} \hyperref[TEI.note]{note} \hyperref[TEI.orig]{orig} \hyperref[TEI.p]{p} \hyperref[TEI.q]{q} \hyperref[TEI.quote]{quote} \hyperref[TEI.ref]{ref} \hyperref[TEI.reg]{reg} \hyperref[TEI.relatedItem]{relatedItem} \hyperref[TEI.said]{said} \hyperref[TEI.sic]{sic} \hyperref[TEI.stage]{stage} \hyperref[TEI.title]{title} \hyperref[TEI.unclear]{unclear}\par 
    \item[figures: ]
   \hyperref[TEI.cell]{cell} \hyperref[TEI.figDesc]{figDesc} \hyperref[TEI.figure]{figure}\par 
    \item[header: ]
   \hyperref[TEI.change]{change} \hyperref[TEI.licence]{licence} \hyperref[TEI.rendition]{rendition} \hyperref[TEI.sourceDesc]{sourceDesc} \hyperref[TEI.taxonomy]{taxonomy}\par 
    \item[iso-fs: ]
   \hyperref[TEI.fDescr]{fDescr} \hyperref[TEI.fsDescr]{fsDescr}\par 
    \item[linking: ]
   \hyperref[TEI.ab]{ab} \hyperref[TEI.seg]{seg}\par 
    \item[msdescription: ]
   \hyperref[TEI.accMat]{accMat} \hyperref[TEI.acquisition]{acquisition} \hyperref[TEI.additions]{additions} \hyperref[TEI.collation]{collation} \hyperref[TEI.condition]{condition} \hyperref[TEI.custEvent]{custEvent} \hyperref[TEI.decoNote]{decoNote} \hyperref[TEI.filiation]{filiation} \hyperref[TEI.foliation]{foliation} \hyperref[TEI.layout]{layout} \hyperref[TEI.msItem]{msItem} \hyperref[TEI.msItemStruct]{msItemStruct} \hyperref[TEI.musicNotation]{musicNotation} \hyperref[TEI.origin]{origin} \hyperref[TEI.provenance]{provenance} \hyperref[TEI.signatures]{signatures} \hyperref[TEI.source]{source} \hyperref[TEI.summary]{summary} \hyperref[TEI.support]{support} \hyperref[TEI.surrogates]{surrogates} \hyperref[TEI.typeNote]{typeNote}\par 
    \item[namesdates: ]
   \hyperref[TEI.event]{event} \hyperref[TEI.location]{location} \hyperref[TEI.org]{org} \hyperref[TEI.person]{person} \hyperref[TEI.personGrp]{personGrp} \hyperref[TEI.persona]{persona} \hyperref[TEI.place]{place} \hyperref[TEI.state]{state}\par 
    \item[spoken: ]
   \hyperref[TEI.annotationBlock]{annotationBlock}\par 
    \item[standOff: ]
   \hyperref[TEI.listAnnotation]{listAnnotation}\par 
    \item[textstructure: ]
   \hyperref[TEI.body]{body} \hyperref[TEI.div]{div} \hyperref[TEI.docEdition]{docEdition} \hyperref[TEI.titlePart]{titlePart}\par 
    \item[transcr: ]
   \hyperref[TEI.damage]{damage} \hyperref[TEI.metamark]{metamark} \hyperref[TEI.mod]{mod} \hyperref[TEI.restore]{restore} \hyperref[TEI.retrace]{retrace} \hyperref[TEI.secl]{secl} \hyperref[TEI.supplied]{supplied} \hyperref[TEI.surplus]{surplus}
    \item[{Peut contenir}]
  
    \item[analysis: ]
   \hyperref[TEI.c]{c} \hyperref[TEI.cl]{cl} \hyperref[TEI.interp]{interp} \hyperref[TEI.interpGrp]{interpGrp} \hyperref[TEI.m]{m} \hyperref[TEI.pc]{pc} \hyperref[TEI.phr]{phr} \hyperref[TEI.s]{s} \hyperref[TEI.span]{span} \hyperref[TEI.spanGrp]{spanGrp} \hyperref[TEI.w]{w}\par 
    \item[core: ]
   \hyperref[TEI.abbr]{abbr} \hyperref[TEI.add]{add} \hyperref[TEI.address]{address} \hyperref[TEI.author]{author} \hyperref[TEI.bibl]{bibl} \hyperref[TEI.biblScope]{biblScope} \hyperref[TEI.cb]{cb} \hyperref[TEI.choice]{choice} \hyperref[TEI.citedRange]{citedRange} \hyperref[TEI.corr]{corr} \hyperref[TEI.date]{date} \hyperref[TEI.del]{del} \hyperref[TEI.distinct]{distinct} \hyperref[TEI.editor]{editor} \hyperref[TEI.email]{email} \hyperref[TEI.emph]{emph} \hyperref[TEI.expan]{expan} \hyperref[TEI.foreign]{foreign} \hyperref[TEI.gap]{gap} \hyperref[TEI.gb]{gb} \hyperref[TEI.gloss]{gloss} \hyperref[TEI.hi]{hi} \hyperref[TEI.index]{index} \hyperref[TEI.lb]{lb} \hyperref[TEI.measure]{measure} \hyperref[TEI.measureGrp]{measureGrp} \hyperref[TEI.meeting]{meeting} \hyperref[TEI.mentioned]{mentioned} \hyperref[TEI.milestone]{milestone} \hyperref[TEI.name]{name} \hyperref[TEI.note]{note} \hyperref[TEI.num]{num} \hyperref[TEI.orig]{orig} \hyperref[TEI.pb]{pb} \hyperref[TEI.ptr]{ptr} \hyperref[TEI.pubPlace]{pubPlace} \hyperref[TEI.publisher]{publisher} \hyperref[TEI.ref]{ref} \hyperref[TEI.reg]{reg} \hyperref[TEI.relatedItem]{relatedItem} \hyperref[TEI.respStmt]{respStmt} \hyperref[TEI.rs]{rs} \hyperref[TEI.series]{series} \hyperref[TEI.sic]{sic} \hyperref[TEI.soCalled]{soCalled} \hyperref[TEI.term]{term} \hyperref[TEI.textLang]{textLang} \hyperref[TEI.time]{time} \hyperref[TEI.title]{title} \hyperref[TEI.unclear]{unclear}\par 
    \item[figures: ]
   \hyperref[TEI.figure]{figure} \hyperref[TEI.notatedMusic]{notatedMusic}\par 
    \item[header: ]
   \hyperref[TEI.availability]{availability} \hyperref[TEI.distributor]{distributor} \hyperref[TEI.edition]{edition} \hyperref[TEI.extent]{extent} \hyperref[TEI.funder]{funder} \hyperref[TEI.idno]{idno}\par 
    \item[iso-fs: ]
   \hyperref[TEI.fLib]{fLib} \hyperref[TEI.fs]{fs} \hyperref[TEI.fvLib]{fvLib}\par 
    \item[linking: ]
   \hyperref[TEI.alt]{alt} \hyperref[TEI.altGrp]{altGrp} \hyperref[TEI.anchor]{anchor} \hyperref[TEI.join]{join} \hyperref[TEI.joinGrp]{joinGrp} \hyperref[TEI.link]{link} \hyperref[TEI.linkGrp]{linkGrp} \hyperref[TEI.seg]{seg} \hyperref[TEI.timeline]{timeline}\par 
    \item[msdescription: ]
   \hyperref[TEI.depth]{depth} \hyperref[TEI.dim]{dim} \hyperref[TEI.height]{height} \hyperref[TEI.msIdentifier]{msIdentifier} \hyperref[TEI.source]{source} \hyperref[TEI.width]{width}\par 
    \item[namesdates: ]
   \hyperref[TEI.addName]{addName} \hyperref[TEI.affiliation]{affiliation} \hyperref[TEI.country]{country} \hyperref[TEI.forename]{forename} \hyperref[TEI.genName]{genName} \hyperref[TEI.geogName]{geogName} \hyperref[TEI.location]{location} \hyperref[TEI.nameLink]{nameLink} \hyperref[TEI.orgName]{orgName} \hyperref[TEI.persName]{persName} \hyperref[TEI.placeName]{placeName} \hyperref[TEI.region]{region} \hyperref[TEI.roleName]{roleName} \hyperref[TEI.settlement]{settlement} \hyperref[TEI.state]{state} \hyperref[TEI.surname]{surname}\par 
    \item[spoken: ]
   \hyperref[TEI.annotationBlock]{annotationBlock}\par 
    \item[transcr: ]
   \hyperref[TEI.addSpan]{addSpan} \hyperref[TEI.am]{am} \hyperref[TEI.damage]{damage} \hyperref[TEI.damageSpan]{damageSpan} \hyperref[TEI.delSpan]{delSpan} \hyperref[TEI.ex]{ex} \hyperref[TEI.fw]{fw} \hyperref[TEI.handShift]{handShift} \hyperref[TEI.listTranspose]{listTranspose} \hyperref[TEI.metamark]{metamark} \hyperref[TEI.mod]{mod} \hyperref[TEI.redo]{redo} \hyperref[TEI.restore]{restore} \hyperref[TEI.retrace]{retrace} \hyperref[TEI.secl]{secl} \hyperref[TEI.space]{space} \hyperref[TEI.subst]{subst} \hyperref[TEI.substJoin]{substJoin} \hyperref[TEI.supplied]{supplied} \hyperref[TEI.surplus]{surplus} \hyperref[TEI.undo]{undo}\par des données textuelles
    \item[{Exemple}]
  StandOff enrichissement entité nommée bibl\leavevmode\bgroup\exampleFont \begin{shaded}\noindent\mbox{}{<\textbf{annotationBlock}\hspace*{6pt}{corresp}="{text}">}\mbox{}\newline 
\hspace*{6pt}{<\textbf{bibl}\hspace*{6pt}{change}="{\#Unitex-3.2.0-alpha}"\mbox{}\newline 
\hspace*{6pt}\hspace*{6pt}{resp}="{istex}"\mbox{}\newline 
\hspace*{6pt}\hspace*{6pt}{scheme}="{https://bibl-entity.data.istex.fr}">}\mbox{}\newline 
\hspace*{6pt}\hspace*{6pt}{<\textbf{term}>}GIRODET, Jean (2007). Dictionnaire des pièges et\mbox{}\newline 
\hspace*{6pt}\hspace*{6pt}\hspace*{6pt}\hspace*{6pt} difficults de la langue française, Paris : Bordas{</\textbf{term}>}\mbox{}\newline 
\hspace*{6pt}\hspace*{6pt}{<\textbf{fs}\hspace*{6pt}{type}="{statistics}">}\mbox{}\newline 
\hspace*{6pt}\hspace*{6pt}\hspace*{6pt}{<\textbf{f}\hspace*{6pt}{name}="{frequency}">}\mbox{}\newline 
\hspace*{6pt}\hspace*{6pt}\hspace*{6pt}\hspace*{6pt}{<\textbf{numeric}\hspace*{6pt}{value}="{1}"/>}\mbox{}\newline 
\hspace*{6pt}\hspace*{6pt}\hspace*{6pt}{</\textbf{f}>}\mbox{}\newline 
\hspace*{6pt}\hspace*{6pt}{</\textbf{fs}>}\mbox{}\newline 
\hspace*{6pt}{</\textbf{bibl}>}\mbox{}\newline 
{</\textbf{annotationBlock}>}\end{shaded}\egroup 


    \item[{Modèle de contenu}]
  \mbox{}\hfill\\[-10pt]\begin{Verbatim}[fontsize=\small]
<content>
 <alternate maxOccurs="unbounded"
  minOccurs="0">
  <textNode/>
  <classRef key="model.gLike"/>
  <classRef key="model.highlighted"/>
  <classRef key="model.pPart.data"/>
  <classRef key="model.pPart.edit"/>
  <classRef key="model.segLike"/>
  <classRef key="model.ptrLike"/>
  <classRef key="model.biblPart"/>
  <classRef key="model.global"/>
 </alternate>
</content>
    
\end{Verbatim}

    \item[{Schéma Declaration}]
  \mbox{}\hfill\\[-10pt]\begin{Verbatim}[fontsize=\small]
element bibl
{
   tei_att.declarable.attributes,
   tei_att.docStatus.attributes,
   tei_att.global.attributes,
   tei_att.sortable.attributes,
   tei_att.timed.attributes,
   tei_att.typed.attributes,
   attribute scheme { text }?,
   (
      text
    | tei_model.gLike    | tei_model.highlighted    | tei_model.pPart.data    | tei_model.pPart.edit    | tei_model.segLike    | tei_model.ptrLike    | tei_model.biblPart    | tei_model.global   )*
}
\end{Verbatim}

\end{reflist}  \index{biblFull=<biblFull>|oddindex}
\begin{reflist}
\item[]\begin{specHead}{TEI.biblFull}{<biblFull> }(référence bibliographique totalement structurée) contient une référence bibliographique totalement structurée : tous les composants de la description du fichier TEI y sont présents. [\xref{http://www.tei-c.org/release/doc/tei-p5-doc/en/html/CO.html\#COBITY}{3.11.1. Methods of Encoding Bibliographic References and Lists of References} \xref{http://www.tei-c.org/release/doc/tei-p5-doc/en/html/HD.html\#HD2}{2.2. The File Description} \xref{http://www.tei-c.org/release/doc/tei-p5-doc/en/html/HD.html\#HD3}{2.2.7. The Source Description} \xref{http://www.tei-c.org/release/doc/tei-p5-doc/en/html/CC.html\#CCAS2}{15.3.2. Declarable Elements}]\end{specHead} 
    \item[{Module}]
  header
    \item[{Attributs}]
  Attributs \hyperref[TEI.att.global]{att.global} (\textit{@xml:id}, \textit{@n}, \textit{@xml:lang}, \textit{@xml:base}, \textit{@xml:space})  (\hyperref[TEI.att.global.rendition]{att.global.rendition} (\textit{@rend}, \textit{@style}, \textit{@rendition})) (\hyperref[TEI.att.global.linking]{att.global.linking} (\textit{@corresp}, \textit{@synch}, \textit{@sameAs}, \textit{@copyOf}, \textit{@next}, \textit{@prev}, \textit{@exclude}, \textit{@select})) (\hyperref[TEI.att.global.analytic]{att.global.analytic} (\textit{@ana})) (\hyperref[TEI.att.global.facs]{att.global.facs} (\textit{@facs})) (\hyperref[TEI.att.global.change]{att.global.change} (\textit{@change})) (\hyperref[TEI.att.global.responsibility]{att.global.responsibility} (\textit{@cert}, \textit{@resp})) (\hyperref[TEI.att.global.source]{att.global.source} (\textit{@source})) \hyperref[TEI.att.declarable]{att.declarable} (\textit{@default}) \hyperref[TEI.att.sortable]{att.sortable} (\textit{@sortKey}) \hyperref[TEI.att.docStatus]{att.docStatus} (\textit{@status}) 
    \item[{Membre du}]
  \hyperref[TEI.model.biblLike]{model.biblLike}
    \item[{Contenu dans}]
  
    \item[core: ]
   \hyperref[TEI.add]{add} \hyperref[TEI.cit]{cit} \hyperref[TEI.corr]{corr} \hyperref[TEI.del]{del} \hyperref[TEI.desc]{desc} \hyperref[TEI.emph]{emph} \hyperref[TEI.head]{head} \hyperref[TEI.hi]{hi} \hyperref[TEI.item]{item} \hyperref[TEI.l]{l} \hyperref[TEI.listBibl]{listBibl} \hyperref[TEI.meeting]{meeting} \hyperref[TEI.note]{note} \hyperref[TEI.orig]{orig} \hyperref[TEI.p]{p} \hyperref[TEI.q]{q} \hyperref[TEI.quote]{quote} \hyperref[TEI.ref]{ref} \hyperref[TEI.reg]{reg} \hyperref[TEI.relatedItem]{relatedItem} \hyperref[TEI.said]{said} \hyperref[TEI.sic]{sic} \hyperref[TEI.stage]{stage} \hyperref[TEI.title]{title} \hyperref[TEI.unclear]{unclear}\par 
    \item[figures: ]
   \hyperref[TEI.cell]{cell} \hyperref[TEI.figDesc]{figDesc} \hyperref[TEI.figure]{figure}\par 
    \item[header: ]
   \hyperref[TEI.change]{change} \hyperref[TEI.licence]{licence} \hyperref[TEI.rendition]{rendition} \hyperref[TEI.sourceDesc]{sourceDesc} \hyperref[TEI.taxonomy]{taxonomy}\par 
    \item[iso-fs: ]
   \hyperref[TEI.fDescr]{fDescr} \hyperref[TEI.fsDescr]{fsDescr}\par 
    \item[linking: ]
   \hyperref[TEI.ab]{ab} \hyperref[TEI.seg]{seg}\par 
    \item[msdescription: ]
   \hyperref[TEI.accMat]{accMat} \hyperref[TEI.acquisition]{acquisition} \hyperref[TEI.additions]{additions} \hyperref[TEI.collation]{collation} \hyperref[TEI.condition]{condition} \hyperref[TEI.custEvent]{custEvent} \hyperref[TEI.decoNote]{decoNote} \hyperref[TEI.filiation]{filiation} \hyperref[TEI.foliation]{foliation} \hyperref[TEI.layout]{layout} \hyperref[TEI.msItem]{msItem} \hyperref[TEI.musicNotation]{musicNotation} \hyperref[TEI.origin]{origin} \hyperref[TEI.provenance]{provenance} \hyperref[TEI.signatures]{signatures} \hyperref[TEI.source]{source} \hyperref[TEI.summary]{summary} \hyperref[TEI.support]{support} \hyperref[TEI.surrogates]{surrogates} \hyperref[TEI.typeNote]{typeNote}\par 
    \item[namesdates: ]
   \hyperref[TEI.event]{event} \hyperref[TEI.location]{location} \hyperref[TEI.org]{org} \hyperref[TEI.person]{person} \hyperref[TEI.personGrp]{personGrp} \hyperref[TEI.persona]{persona} \hyperref[TEI.place]{place} \hyperref[TEI.state]{state}\par 
    \item[spoken: ]
   \hyperref[TEI.annotationBlock]{annotationBlock}\par 
    \item[standOff: ]
   \hyperref[TEI.listAnnotation]{listAnnotation}\par 
    \item[textstructure: ]
   \hyperref[TEI.body]{body} \hyperref[TEI.div]{div} \hyperref[TEI.docEdition]{docEdition} \hyperref[TEI.titlePart]{titlePart}\par 
    \item[transcr: ]
   \hyperref[TEI.damage]{damage} \hyperref[TEI.metamark]{metamark} \hyperref[TEI.mod]{mod} \hyperref[TEI.restore]{restore} \hyperref[TEI.retrace]{retrace} \hyperref[TEI.secl]{secl} \hyperref[TEI.supplied]{supplied} \hyperref[TEI.surplus]{surplus}
    \item[{Peut contenir}]
  
    \item[header: ]
   \hyperref[TEI.editionStmt]{editionStmt} \hyperref[TEI.extent]{extent} \hyperref[TEI.fileDesc]{fileDesc} \hyperref[TEI.notesStmt]{notesStmt} \hyperref[TEI.profileDesc]{profileDesc} \hyperref[TEI.publicationStmt]{publicationStmt} \hyperref[TEI.seriesStmt]{seriesStmt} \hyperref[TEI.sourceDesc]{sourceDesc} \hyperref[TEI.titleStmt]{titleStmt}
    \item[{Exemple}]
  \leavevmode\bgroup\exampleFont \begin{shaded}\noindent\mbox{}{<\textbf{biblFull}>}\mbox{}\newline 
\hspace*{6pt}{<\textbf{titleStmt}>}\mbox{}\newline 
\hspace*{6pt}\hspace*{6pt}{<\textbf{title}>}Hydraulique fluviale. Tome 16, Écoulement et phénomènes de transport dans les\mbox{}\newline 
\hspace*{6pt}\hspace*{6pt}\hspace*{6pt}\hspace*{6pt} canaux à géométrie simple {</\textbf{title}>}\mbox{}\newline 
\hspace*{6pt}\hspace*{6pt}{<\textbf{editor}>} Mustafa Siddik Altinakar{</\textbf{editor}>}\mbox{}\newline 
\hspace*{6pt}\hspace*{6pt}{<\textbf{editor}>} René Walther{</\textbf{editor}>}\mbox{}\newline 
\hspace*{6pt}{</\textbf{titleStmt}>}\mbox{}\newline 
\hspace*{6pt}{<\textbf{editionStmt}>}\mbox{}\newline 
\hspace*{6pt}\hspace*{6pt}{<\textbf{edition}>}2e édition corrigée{</\textbf{edition}>}\mbox{}\newline 
\hspace*{6pt}{</\textbf{editionStmt}>}\mbox{}\newline 
\hspace*{6pt}{<\textbf{extent}>}627 p.{</\textbf{extent}>}\mbox{}\newline 
\hspace*{6pt}{<\textbf{publicationStmt}>}\mbox{}\newline 
\hspace*{6pt}\hspace*{6pt}{<\textbf{publisher}>}Presses polytechniques et universitaires romandes{</\textbf{publisher}>}\mbox{}\newline 
\hspace*{6pt}\hspace*{6pt}{<\textbf{pubPlace}>}Lausanne{</\textbf{pubPlace}>}\mbox{}\newline 
\hspace*{6pt}\hspace*{6pt}{<\textbf{date}>}2008{</\textbf{date}>}\mbox{}\newline 
\hspace*{6pt}{</\textbf{publicationStmt}>}\mbox{}\newline 
\hspace*{6pt}{<\textbf{sourceDesc}>}\mbox{}\newline 
\hspace*{6pt}\hspace*{6pt}{<\textbf{p}>}Pas de source : il s'agit d'un document original{</\textbf{p}>}\mbox{}\newline 
\hspace*{6pt}{</\textbf{sourceDesc}>}\mbox{}\newline 
{</\textbf{biblFull}>}\end{shaded}\egroup 


    \item[{Modèle de contenu}]
  \mbox{}\hfill\\[-10pt]\begin{Verbatim}[fontsize=\small]
<content>
 <alternate maxOccurs="1" minOccurs="1">
  <sequence maxOccurs="1" minOccurs="1">
   <sequence maxOccurs="1" minOccurs="1">
    <elementRef key="titleStmt"/>
    <elementRef key="editionStmt"
     minOccurs="0"/>
    <elementRef key="extent" minOccurs="0"/>
    <elementRef key="publicationStmt"/>
    <elementRef key="seriesStmt"
     minOccurs="0"/>
    <elementRef key="notesStmt"
     minOccurs="0"/>
   </sequence>
   <elementRef key="sourceDesc"
    maxOccurs="unbounded" minOccurs="0"/>
  </sequence>
  <sequence maxOccurs="1" minOccurs="1">
   <elementRef key="fileDesc"/>
   <elementRef key="profileDesc"/>
  </sequence>
 </alternate>
</content>
    
\end{Verbatim}

    \item[{Schéma Declaration}]
  \mbox{}\hfill\\[-10pt]\begin{Verbatim}[fontsize=\small]
element biblFull
{
   tei_att.global.attributes,
   tei_att.declarable.attributes,
   tei_att.sortable.attributes,
   tei_att.docStatus.attributes,
   (
      (
         (
            tei_titleStmt,
            tei_editionStmt?,
            tei_extent?,
            tei_publicationStmt,
            tei_seriesStmt?,
            tei_notesStmt?
         ),
         tei_sourceDesc*
      )
    | ( tei_fileDesc, tei_profileDesc )
   )
}
\end{Verbatim}

\end{reflist}  \index{biblScope=<biblScope>|oddindex}
\begin{reflist}
\item[]\begin{specHead}{TEI.biblScope}{<biblScope> }(extension d'une référence bibliographique) définit l'extension d'une référence bibliographique, comme par exemple une liste de numéros de page, ou le nom d'une subdivision d'une oeuvre plus grande. [\xref{http://www.tei-c.org/release/doc/tei-p5-doc/en/html/CO.html\#COBICOB}{3.11.2.5. Scopes and Ranges in Bibliographic Citations}]\end{specHead} 
    \item[{Module}]
  core
    \item[{Attributs}]
  Attributs \hyperref[TEI.att.global]{att.global} (\textit{@xml:id}, \textit{@n}, \textit{@xml:lang}, \textit{@xml:base}, \textit{@xml:space})  (\hyperref[TEI.att.global.rendition]{att.global.rendition} (\textit{@rend}, \textit{@style}, \textit{@rendition})) (\hyperref[TEI.att.global.linking]{att.global.linking} (\textit{@corresp}, \textit{@synch}, \textit{@sameAs}, \textit{@copyOf}, \textit{@next}, \textit{@prev}, \textit{@exclude}, \textit{@select})) (\hyperref[TEI.att.global.analytic]{att.global.analytic} (\textit{@ana})) (\hyperref[TEI.att.global.facs]{att.global.facs} (\textit{@facs})) (\hyperref[TEI.att.global.change]{att.global.change} (\textit{@change})) (\hyperref[TEI.att.global.responsibility]{att.global.responsibility} (\textit{@cert}, \textit{@resp})) (\hyperref[TEI.att.global.source]{att.global.source} (\textit{@source})) \hyperref[TEI.att.citing]{att.citing} (\textit{@unit}, \textit{@from}, \textit{@to}) 
    \item[{Membre du}]
  \hyperref[TEI.model.imprintPart]{model.imprintPart} 
    \item[{Contenu dans}]
  
    \item[core: ]
   \hyperref[TEI.bibl]{bibl} \hyperref[TEI.imprint]{imprint} \hyperref[TEI.monogr]{monogr} \hyperref[TEI.series]{series}\par 
    \item[header: ]
   \hyperref[TEI.seriesStmt]{seriesStmt}
    \item[{Peut contenir}]
  
    \item[analysis: ]
   \hyperref[TEI.c]{c} \hyperref[TEI.cl]{cl} \hyperref[TEI.interp]{interp} \hyperref[TEI.interpGrp]{interpGrp} \hyperref[TEI.m]{m} \hyperref[TEI.pc]{pc} \hyperref[TEI.phr]{phr} \hyperref[TEI.s]{s} \hyperref[TEI.span]{span} \hyperref[TEI.spanGrp]{spanGrp} \hyperref[TEI.w]{w}\par 
    \item[core: ]
   \hyperref[TEI.abbr]{abbr} \hyperref[TEI.add]{add} \hyperref[TEI.address]{address} \hyperref[TEI.binaryObject]{binaryObject} \hyperref[TEI.cb]{cb} \hyperref[TEI.choice]{choice} \hyperref[TEI.corr]{corr} \hyperref[TEI.date]{date} \hyperref[TEI.del]{del} \hyperref[TEI.distinct]{distinct} \hyperref[TEI.email]{email} \hyperref[TEI.emph]{emph} \hyperref[TEI.expan]{expan} \hyperref[TEI.foreign]{foreign} \hyperref[TEI.gap]{gap} \hyperref[TEI.gb]{gb} \hyperref[TEI.gloss]{gloss} \hyperref[TEI.graphic]{graphic} \hyperref[TEI.hi]{hi} \hyperref[TEI.index]{index} \hyperref[TEI.lb]{lb} \hyperref[TEI.measure]{measure} \hyperref[TEI.measureGrp]{measureGrp} \hyperref[TEI.media]{media} \hyperref[TEI.mentioned]{mentioned} \hyperref[TEI.milestone]{milestone} \hyperref[TEI.name]{name} \hyperref[TEI.note]{note} \hyperref[TEI.num]{num} \hyperref[TEI.orig]{orig} \hyperref[TEI.pb]{pb} \hyperref[TEI.ptr]{ptr} \hyperref[TEI.ref]{ref} \hyperref[TEI.reg]{reg} \hyperref[TEI.rs]{rs} \hyperref[TEI.sic]{sic} \hyperref[TEI.soCalled]{soCalled} \hyperref[TEI.term]{term} \hyperref[TEI.time]{time} \hyperref[TEI.title]{title} \hyperref[TEI.unclear]{unclear}\par 
    \item[derived-module-tei.istex: ]
   \hyperref[TEI.math]{math} \hyperref[TEI.mrow]{mrow}\par 
    \item[figures: ]
   \hyperref[TEI.figure]{figure} \hyperref[TEI.formula]{formula} \hyperref[TEI.notatedMusic]{notatedMusic}\par 
    \item[header: ]
   \hyperref[TEI.idno]{idno}\par 
    \item[iso-fs: ]
   \hyperref[TEI.fLib]{fLib} \hyperref[TEI.fs]{fs} \hyperref[TEI.fvLib]{fvLib}\par 
    \item[linking: ]
   \hyperref[TEI.alt]{alt} \hyperref[TEI.altGrp]{altGrp} \hyperref[TEI.anchor]{anchor} \hyperref[TEI.join]{join} \hyperref[TEI.joinGrp]{joinGrp} \hyperref[TEI.link]{link} \hyperref[TEI.linkGrp]{linkGrp} \hyperref[TEI.seg]{seg} \hyperref[TEI.timeline]{timeline}\par 
    \item[msdescription: ]
   \hyperref[TEI.catchwords]{catchwords} \hyperref[TEI.depth]{depth} \hyperref[TEI.dim]{dim} \hyperref[TEI.dimensions]{dimensions} \hyperref[TEI.height]{height} \hyperref[TEI.heraldry]{heraldry} \hyperref[TEI.locus]{locus} \hyperref[TEI.locusGrp]{locusGrp} \hyperref[TEI.material]{material} \hyperref[TEI.objectType]{objectType} \hyperref[TEI.origDate]{origDate} \hyperref[TEI.origPlace]{origPlace} \hyperref[TEI.secFol]{secFol} \hyperref[TEI.signatures]{signatures} \hyperref[TEI.source]{source} \hyperref[TEI.stamp]{stamp} \hyperref[TEI.watermark]{watermark} \hyperref[TEI.width]{width}\par 
    \item[namesdates: ]
   \hyperref[TEI.addName]{addName} \hyperref[TEI.affiliation]{affiliation} \hyperref[TEI.country]{country} \hyperref[TEI.forename]{forename} \hyperref[TEI.genName]{genName} \hyperref[TEI.geogName]{geogName} \hyperref[TEI.location]{location} \hyperref[TEI.nameLink]{nameLink} \hyperref[TEI.orgName]{orgName} \hyperref[TEI.persName]{persName} \hyperref[TEI.placeName]{placeName} \hyperref[TEI.region]{region} \hyperref[TEI.roleName]{roleName} \hyperref[TEI.settlement]{settlement} \hyperref[TEI.state]{state} \hyperref[TEI.surname]{surname}\par 
    \item[spoken: ]
   \hyperref[TEI.annotationBlock]{annotationBlock}\par 
    \item[transcr: ]
   \hyperref[TEI.addSpan]{addSpan} \hyperref[TEI.am]{am} \hyperref[TEI.damage]{damage} \hyperref[TEI.damageSpan]{damageSpan} \hyperref[TEI.delSpan]{delSpan} \hyperref[TEI.ex]{ex} \hyperref[TEI.fw]{fw} \hyperref[TEI.handShift]{handShift} \hyperref[TEI.listTranspose]{listTranspose} \hyperref[TEI.metamark]{metamark} \hyperref[TEI.mod]{mod} \hyperref[TEI.redo]{redo} \hyperref[TEI.restore]{restore} \hyperref[TEI.retrace]{retrace} \hyperref[TEI.secl]{secl} \hyperref[TEI.space]{space} \hyperref[TEI.subst]{subst} \hyperref[TEI.substJoin]{substJoin} \hyperref[TEI.supplied]{supplied} \hyperref[TEI.surplus]{surplus} \hyperref[TEI.undo]{undo}\par des données textuelles
    \item[{Note}]
  \par
When a single page is being cited, use the {\itshape from} and {\itshape to} attributes with an identical value. When no clear endpoint is provided, the {\itshape from} attribute may be used without {\itshape to}; for example a citation such as ‘p. 3ff’ might be encoded \texttt{<biblScope from="3">p. 3ff<biblScope>}.\par
It is now considered good practice to supply this element as a sibling (rather than a child) of \hyperref[TEI.imprint]{<imprint>}, since it supplies information which does not constitute part of the imprint.
    \item[{Exemple}]
  \leavevmode\bgroup\exampleFont \begin{shaded}\noindent\mbox{}{<\textbf{biblScope}>}pp 12–34{</\textbf{biblScope}>}\mbox{}\newline 
{<\textbf{biblScope}\hspace*{6pt}{from}="{12}"\hspace*{6pt}{to}="{34}"\hspace*{6pt}{unit}="{page}"/>}\mbox{}\newline 
{<\textbf{biblScope}\hspace*{6pt}{unit}="{volume}">}II{</\textbf{biblScope}>}\mbox{}\newline 
{<\textbf{biblScope}\hspace*{6pt}{unit}="{page}">}12{</\textbf{biblScope}>}\end{shaded}\egroup 


    \item[{Modèle de contenu}]
  \mbox{}\hfill\\[-10pt]\begin{Verbatim}[fontsize=\small]
<content>
 <macroRef key="macro.phraseSeq"/>
</content>
    
\end{Verbatim}

    \item[{Schéma Declaration}]
  \mbox{}\hfill\\[-10pt]\begin{Verbatim}[fontsize=\small]
element biblScope
{
   tei_att.global.attributes,
   tei_att.citing.attributes,
   tei_macro.phraseSeq}
\end{Verbatim}

\end{reflist}  \index{biblStruct=<biblStruct>|oddindex}
\begin{reflist}
\item[]\begin{specHead}{TEI.biblStruct}{<biblStruct> }(référence bibliographique structurée) contient une référence bibliographique dans laquelle seuls des sous-éléments bibliographiques apparaissent et cela, selon un ordre déterminé. [\xref{http://www.tei-c.org/release/doc/tei-p5-doc/en/html/CO.html\#COBITY}{3.11.1. Methods of Encoding Bibliographic References and Lists of References} \xref{http://www.tei-c.org/release/doc/tei-p5-doc/en/html/HD.html\#HD3}{2.2.7. The Source Description} \xref{http://www.tei-c.org/release/doc/tei-p5-doc/en/html/CC.html\#CCAS2}{15.3.2. Declarable Elements}]\end{specHead} 
    \item[{Module}]
  core
    \item[{Attributs}]
  Attributs \hyperref[TEI.att.global]{att.global} (\textit{@xml:id}, \textit{@n}, \textit{@xml:lang}, \textit{@xml:base}, \textit{@xml:space})  (\hyperref[TEI.att.global.rendition]{att.global.rendition} (\textit{@rend}, \textit{@style}, \textit{@rendition})) (\hyperref[TEI.att.global.linking]{att.global.linking} (\textit{@corresp}, \textit{@synch}, \textit{@sameAs}, \textit{@copyOf}, \textit{@next}, \textit{@prev}, \textit{@exclude}, \textit{@select})) (\hyperref[TEI.att.global.analytic]{att.global.analytic} (\textit{@ana})) (\hyperref[TEI.att.global.facs]{att.global.facs} (\textit{@facs})) (\hyperref[TEI.att.global.change]{att.global.change} (\textit{@change})) (\hyperref[TEI.att.global.responsibility]{att.global.responsibility} (\textit{@cert}, \textit{@resp})) (\hyperref[TEI.att.global.source]{att.global.source} (\textit{@source})) \hyperref[TEI.att.declarable]{att.declarable} (\textit{@default}) \hyperref[TEI.att.typed]{att.typed} (\textit{@type}, \textit{@subtype}) \hyperref[TEI.att.sortable]{att.sortable} (\textit{@sortKey}) \hyperref[TEI.att.docStatus]{att.docStatus} (\textit{@status}) 
    \item[{Membre du}]
  \hyperref[TEI.model.biblLike]{model.biblLike} 
    \item[{Contenu dans}]
  
    \item[core: ]
   \hyperref[TEI.add]{add} \hyperref[TEI.cit]{cit} \hyperref[TEI.corr]{corr} \hyperref[TEI.del]{del} \hyperref[TEI.desc]{desc} \hyperref[TEI.emph]{emph} \hyperref[TEI.head]{head} \hyperref[TEI.hi]{hi} \hyperref[TEI.item]{item} \hyperref[TEI.l]{l} \hyperref[TEI.listBibl]{listBibl} \hyperref[TEI.meeting]{meeting} \hyperref[TEI.note]{note} \hyperref[TEI.orig]{orig} \hyperref[TEI.p]{p} \hyperref[TEI.q]{q} \hyperref[TEI.quote]{quote} \hyperref[TEI.ref]{ref} \hyperref[TEI.reg]{reg} \hyperref[TEI.relatedItem]{relatedItem} \hyperref[TEI.said]{said} \hyperref[TEI.sic]{sic} \hyperref[TEI.stage]{stage} \hyperref[TEI.title]{title} \hyperref[TEI.unclear]{unclear}\par 
    \item[figures: ]
   \hyperref[TEI.cell]{cell} \hyperref[TEI.figDesc]{figDesc} \hyperref[TEI.figure]{figure}\par 
    \item[header: ]
   \hyperref[TEI.change]{change} \hyperref[TEI.licence]{licence} \hyperref[TEI.rendition]{rendition} \hyperref[TEI.sourceDesc]{sourceDesc} \hyperref[TEI.taxonomy]{taxonomy}\par 
    \item[iso-fs: ]
   \hyperref[TEI.fDescr]{fDescr} \hyperref[TEI.fsDescr]{fsDescr}\par 
    \item[linking: ]
   \hyperref[TEI.ab]{ab} \hyperref[TEI.seg]{seg}\par 
    \item[msdescription: ]
   \hyperref[TEI.accMat]{accMat} \hyperref[TEI.acquisition]{acquisition} \hyperref[TEI.additions]{additions} \hyperref[TEI.collation]{collation} \hyperref[TEI.condition]{condition} \hyperref[TEI.custEvent]{custEvent} \hyperref[TEI.decoNote]{decoNote} \hyperref[TEI.filiation]{filiation} \hyperref[TEI.foliation]{foliation} \hyperref[TEI.layout]{layout} \hyperref[TEI.msItem]{msItem} \hyperref[TEI.msItemStruct]{msItemStruct} \hyperref[TEI.musicNotation]{musicNotation} \hyperref[TEI.origin]{origin} \hyperref[TEI.provenance]{provenance} \hyperref[TEI.signatures]{signatures} \hyperref[TEI.source]{source} \hyperref[TEI.summary]{summary} \hyperref[TEI.support]{support} \hyperref[TEI.surrogates]{surrogates} \hyperref[TEI.typeNote]{typeNote}\par 
    \item[namesdates: ]
   \hyperref[TEI.event]{event} \hyperref[TEI.location]{location} \hyperref[TEI.org]{org} \hyperref[TEI.person]{person} \hyperref[TEI.personGrp]{personGrp} \hyperref[TEI.persona]{persona} \hyperref[TEI.place]{place} \hyperref[TEI.state]{state}\par 
    \item[spoken: ]
   \hyperref[TEI.annotationBlock]{annotationBlock}\par 
    \item[standOff: ]
   \hyperref[TEI.listAnnotation]{listAnnotation}\par 
    \item[textstructure: ]
   \hyperref[TEI.body]{body} \hyperref[TEI.div]{div} \hyperref[TEI.docEdition]{docEdition} \hyperref[TEI.titlePart]{titlePart}\par 
    \item[transcr: ]
   \hyperref[TEI.damage]{damage} \hyperref[TEI.metamark]{metamark} \hyperref[TEI.mod]{mod} \hyperref[TEI.restore]{restore} \hyperref[TEI.retrace]{retrace} \hyperref[TEI.secl]{secl} \hyperref[TEI.supplied]{supplied} \hyperref[TEI.surplus]{surplus}
    \item[{Peut contenir}]
  
    \item[core: ]
   \hyperref[TEI.analytic]{analytic} \hyperref[TEI.citedRange]{citedRange} \hyperref[TEI.monogr]{monogr} \hyperref[TEI.note]{note} \hyperref[TEI.ptr]{ptr} \hyperref[TEI.ref]{ref} \hyperref[TEI.relatedItem]{relatedItem} \hyperref[TEI.series]{series}
    \item[{Exemple}]
  \leavevmode\bgroup\exampleFont \begin{shaded}\noindent\mbox{}{<\textbf{biblStruct}>}\mbox{}\newline 
\hspace*{6pt}{<\textbf{monogr}>}\mbox{}\newline 
\hspace*{6pt}\hspace*{6pt}{<\textbf{author}>}Anouilh, Jean{</\textbf{author}>}\mbox{}\newline 
\hspace*{6pt}\hspace*{6pt}{<\textbf{title}>}Antigone{</\textbf{title}>}\mbox{}\newline 
\hspace*{6pt}\hspace*{6pt}{<\textbf{edition}>}première édition{</\textbf{edition}>}\mbox{}\newline 
\hspace*{6pt}\hspace*{6pt}{<\textbf{imprint}>}\mbox{}\newline 
\hspace*{6pt}\hspace*{6pt}\hspace*{6pt}{<\textbf{publisher}>}in Nouvelles pièces noires, La Table ronde{</\textbf{publisher}>}\mbox{}\newline 
\hspace*{6pt}\hspace*{6pt}\hspace*{6pt}{<\textbf{pubPlace}>}Paris{</\textbf{pubPlace}>}\mbox{}\newline 
\hspace*{6pt}\hspace*{6pt}\hspace*{6pt}{<\textbf{date}>}1955{</\textbf{date}>}\mbox{}\newline 
\hspace*{6pt}\hspace*{6pt}{</\textbf{imprint}>}\mbox{}\newline 
\hspace*{6pt}{</\textbf{monogr}>}\mbox{}\newline 
{</\textbf{biblStruct}>}\end{shaded}\egroup 


    \item[{Modèle de contenu}]
  \mbox{}\hfill\\[-10pt]\begin{Verbatim}[fontsize=\small]
<content>
 <sequence maxOccurs="1" minOccurs="1">
  <elementRef key="analytic"
   maxOccurs="unbounded" minOccurs="0"/>
  <sequence maxOccurs="unbounded"
   minOccurs="1">
   <elementRef key="monogr"/>
   <elementRef key="series"
    maxOccurs="unbounded" minOccurs="0"/>
  </sequence>
  <alternate maxOccurs="unbounded"
   minOccurs="0">
   <classRef key="model.noteLike"/>
   <classRef key="model.ptrLike"/>
   <elementRef key="relatedItem"/>
   <elementRef key="citedRange"/>
  </alternate>
 </sequence>
</content>
    
\end{Verbatim}

    \item[{Schéma Declaration}]
  \mbox{}\hfill\\[-10pt]\begin{Verbatim}[fontsize=\small]
element biblStruct
{
   tei_att.global.attributes,
   tei_att.declarable.attributes,
   tei_att.typed.attributes,
   tei_att.sortable.attributes,
   tei_att.docStatus.attributes,
   (
      tei_analytic*,
      ( tei_monogr, tei_series* )+,
      (
         tei_model.noteLike       | tei_model.ptrLike       | tei_relatedItem       | tei_citedRange      )*
   )
}
\end{Verbatim}

\end{reflist}  \index{bicond=<bicond>|oddindex}
\begin{reflist}
\item[]\begin{specHead}{TEI.bicond}{<bicond> }(contrainte bi-conditionnelle de structure de traits) définit une contrainte bi-conditionnelle de structure de traits ; la conséquence et l'antécédent sont tous deux indiqués comme structures de traits ou comme groupes de structures de traits ; la contrainte est satisfaite si chacun des deux englobe une structure de traits donnée, ou si aucun ne le fait [\xref{http://www.tei-c.org/release/doc/tei-p5-doc/en/html/FS.html\#FD}{18.11. Feature System Declaration}]\end{specHead} 
    \item[{Module}]
  iso-fs
    \item[{Attributs}]
  Attributs \hyperref[TEI.att.global]{att.global} (\textit{@xml:id}, \textit{@n}, \textit{@xml:lang}, \textit{@xml:base}, \textit{@xml:space})  (\hyperref[TEI.att.global.rendition]{att.global.rendition} (\textit{@rend}, \textit{@style}, \textit{@rendition})) (\hyperref[TEI.att.global.linking]{att.global.linking} (\textit{@corresp}, \textit{@synch}, \textit{@sameAs}, \textit{@copyOf}, \textit{@next}, \textit{@prev}, \textit{@exclude}, \textit{@select})) (\hyperref[TEI.att.global.analytic]{att.global.analytic} (\textit{@ana})) (\hyperref[TEI.att.global.facs]{att.global.facs} (\textit{@facs})) (\hyperref[TEI.att.global.change]{att.global.change} (\textit{@change})) (\hyperref[TEI.att.global.responsibility]{att.global.responsibility} (\textit{@cert}, \textit{@resp})) (\hyperref[TEI.att.global.source]{att.global.source} (\textit{@source}))
    \item[{Contenu dans}]
  
    \item[iso-fs: ]
   \hyperref[TEI.fsConstraints]{fsConstraints}
    \item[{Peut contenir}]
  
    \item[iso-fs: ]
   \hyperref[TEI.f]{f} \hyperref[TEI.fs]{fs} \hyperref[TEI.iff]{iff}
    \item[{Exemple}]
  \leavevmode\bgroup\exampleFont \begin{shaded}\noindent\mbox{}{<\textbf{bicond}>}\mbox{}\newline 
\hspace*{6pt}{<\textbf{fs}>}\mbox{}\newline 
\hspace*{6pt}\hspace*{6pt}{<\textbf{f}\hspace*{6pt}{name}="{FOO}">}\mbox{}\newline 
\hspace*{6pt}\hspace*{6pt}\hspace*{6pt}{<\textbf{symbol}\hspace*{6pt}{value}="{42}"/>}\mbox{}\newline 
\hspace*{6pt}\hspace*{6pt}{</\textbf{f}>}\mbox{}\newline 
\hspace*{6pt}{</\textbf{fs}>}\mbox{}\newline 
\hspace*{6pt}{<\textbf{iff}/>}\mbox{}\newline 
\hspace*{6pt}{<\textbf{fs}>}\mbox{}\newline 
\hspace*{6pt}\hspace*{6pt}{<\textbf{f}\hspace*{6pt}{name}="{BAR}">}\mbox{}\newline 
\hspace*{6pt}\hspace*{6pt}\hspace*{6pt}{<\textbf{binary}\hspace*{6pt}{value}="{true}"/>}\mbox{}\newline 
\hspace*{6pt}\hspace*{6pt}{</\textbf{f}>}\mbox{}\newline 
\hspace*{6pt}{</\textbf{fs}>}\mbox{}\newline 
{</\textbf{bicond}>}\end{shaded}\egroup 


    \item[{Modèle de contenu}]
  \mbox{}\hfill\\[-10pt]\begin{Verbatim}[fontsize=\small]
<content>
 <sequence maxOccurs="1" minOccurs="1">
  <alternate maxOccurs="1" minOccurs="1">
   <elementRef key="fs"/>
   <elementRef key="f"/>
  </alternate>
  <elementRef key="iff"/>
  <alternate maxOccurs="1" minOccurs="1">
   <elementRef key="fs"/>
   <elementRef key="f"/>
  </alternate>
 </sequence>
</content>
    
\end{Verbatim}

    \item[{Schéma Declaration}]
  \mbox{}\hfill\\[-10pt]\begin{Verbatim}[fontsize=\small]
element bicond
{
   tei_att.global.attributes,
   ( ( tei_fs | tei_f ), tei_iff, ( tei_fs | tei_f ) )
}
\end{Verbatim}

\end{reflist}  \index{binary=<binary>|oddindex}\index{value=@value!<binary>|oddindex}
\begin{reflist}
\item[]\begin{specHead}{TEI.binary}{<binary> }(valeur binaire) représente la partie ‘valeur’ d'une spécification trait-valeur qui peut contenir l'une ou l'autre des deux valeurs possibles. [\xref{http://www.tei-c.org/release/doc/tei-p5-doc/en/html/FS.html\#FSBI}{18.2. Elementary Feature Structures and the Binary Feature Value}]\end{specHead} 
    \item[{Module}]
  iso-fs
    \item[{Attributs}]
  Attributs \hyperref[TEI.att.global]{att.global} (\textit{@xml:id}, \textit{@n}, \textit{@xml:lang}, \textit{@xml:base}, \textit{@xml:space})  (\hyperref[TEI.att.global.rendition]{att.global.rendition} (\textit{@rend}, \textit{@style}, \textit{@rendition})) (\hyperref[TEI.att.global.linking]{att.global.linking} (\textit{@corresp}, \textit{@synch}, \textit{@sameAs}, \textit{@copyOf}, \textit{@next}, \textit{@prev}, \textit{@exclude}, \textit{@select})) (\hyperref[TEI.att.global.analytic]{att.global.analytic} (\textit{@ana})) (\hyperref[TEI.att.global.facs]{att.global.facs} (\textit{@facs})) (\hyperref[TEI.att.global.change]{att.global.change} (\textit{@change})) (\hyperref[TEI.att.global.responsibility]{att.global.responsibility} (\textit{@cert}, \textit{@resp})) (\hyperref[TEI.att.global.source]{att.global.source} (\textit{@source})) \hyperref[TEI.att.datcat]{att.datcat} (\textit{@datcat}, \textit{@valueDatcat}) \hfil\\[-10pt]\begin{sansreflist}
    \item[@value]
  fournit une valeur binaire.
\begin{reflist}
    \item[{Statut}]
  Requis
    \item[{Type de données}]
  \hyperref[TEI.teidata.truthValue]{teidata.truthValue}
\end{reflist}  
\end{sansreflist}  
    \item[{Membre du}]
  \hyperref[TEI.model.featureVal.single]{model.featureVal.single}
    \item[{Contenu dans}]
  
    \item[iso-fs: ]
   \hyperref[TEI.f]{f} \hyperref[TEI.fvLib]{fvLib} \hyperref[TEI.if]{if} \hyperref[TEI.vAlt]{vAlt} \hyperref[TEI.vColl]{vColl} \hyperref[TEI.vDefault]{vDefault} \hyperref[TEI.vLabel]{vLabel} \hyperref[TEI.vMerge]{vMerge} \hyperref[TEI.vNot]{vNot} \hyperref[TEI.vRange]{vRange}
    \item[{Peut contenir}]
  Elément vide
    \item[{Note}]
  \par
L'attribut {\itshape value} peut prendre n'importe quelle valeur acceptée pour les attributs de type booléen dans les recommandations du W3C : cela inclut par exemple les chaînes \texttt{true} ou \texttt{1} qui sont équivalentes.
    \item[{Exemple}]
  \leavevmode\bgroup\exampleFont \begin{shaded}\noindent\mbox{}{<\textbf{f}\hspace*{6pt}{name}="{strident}">}\mbox{}\newline 
\hspace*{6pt}{<\textbf{binary}\hspace*{6pt}{value}="{true}"/>}\mbox{}\newline 
{</\textbf{f}>}\mbox{}\newline 
{<\textbf{f}\hspace*{6pt}{name}="{exclusive}">}\mbox{}\newline 
\hspace*{6pt}{<\textbf{binary}\hspace*{6pt}{value}="{false}"/>}\mbox{}\newline 
{</\textbf{f}>}\end{shaded}\egroup 


    \item[{Modèle de contenu}]
  \fbox{\ttfamily <content>\newline
</content>\newline
    } 
    \item[{Schéma Declaration}]
  \mbox{}\hfill\\[-10pt]\begin{Verbatim}[fontsize=\small]
element binary
{
   tei_att.global.attributes,
   tei_att.datcat.attributes,
   attribute value { text },
   empty
}
\end{Verbatim}

\end{reflist}  \index{binaryObject=<binaryObject>|oddindex}\index{encoding=@encoding!<binaryObject>|oddindex}
\begin{reflist}
\item[]\begin{specHead}{TEI.binaryObject}{<binaryObject> }fournit des données binaires encodées qui représentent une image ou un autre objet insérés dans le texte ou dans un autre objet. [\xref{http://www.tei-c.org/release/doc/tei-p5-doc/en/html/CO.html\#COGR}{3.9. Graphics and Other Non-textual Components}]\end{specHead} 
    \item[{Module}]
  core
    \item[{Attributs}]
  Attributs \hyperref[TEI.att.global]{att.global} (\textit{@xml:id}, \textit{@n}, \textit{@xml:lang}, \textit{@xml:base}, \textit{@xml:space})  (\hyperref[TEI.att.global.rendition]{att.global.rendition} (\textit{@rend}, \textit{@style}, \textit{@rendition})) (\hyperref[TEI.att.global.linking]{att.global.linking} (\textit{@corresp}, \textit{@synch}, \textit{@sameAs}, \textit{@copyOf}, \textit{@next}, \textit{@prev}, \textit{@exclude}, \textit{@select})) (\hyperref[TEI.att.global.analytic]{att.global.analytic} (\textit{@ana})) (\hyperref[TEI.att.global.facs]{att.global.facs} (\textit{@facs})) (\hyperref[TEI.att.global.change]{att.global.change} (\textit{@change})) (\hyperref[TEI.att.global.responsibility]{att.global.responsibility} (\textit{@cert}, \textit{@resp})) (\hyperref[TEI.att.global.source]{att.global.source} (\textit{@source})) \hyperref[TEI.att.media]{att.media} (\textit{@width}, \textit{@height}, \textit{@scale})  (\hyperref[TEI.att.internetMedia]{att.internetMedia} (\textit{@mimeType})) \hyperref[TEI.att.timed]{att.timed} (\textit{@start}, \textit{@end})  (\hyperref[TEI.att.duration]{att.duration} (\hyperref[TEI.att.duration.w3c]{att.duration.w3c} (\textit{@dur})) (\hyperref[TEI.att.duration.iso]{att.duration.iso} (\textit{@dur-iso})) ) \hyperref[TEI.att.typed]{att.typed} (\textit{@type}, \textit{@subtype}) \hfil\\[-10pt]\begin{sansreflist}
    \item[@encoding]
  l'encodage utilisé pour encoder les données binaires. Sans autre spécification il est supposé s'agir de \xref{http://en.wikipedia.org/wiki/Base64}{Base64}.
\begin{reflist}
    \item[{Statut}]
  Optionel
    \item[{Type de données}]
  1–∞ occurrences de \hyperref[TEI.teidata.word]{teidata.word} séparé par un espace
\end{reflist}  
\end{sansreflist}  
    \item[{Membre du}]
  \hyperref[TEI.model.graphicLike]{model.graphicLike} \hyperref[TEI.model.titlepagePart]{model.titlepagePart} 
    \item[{Contenu dans}]
  
    \item[analysis: ]
   \hyperref[TEI.cl]{cl} \hyperref[TEI.phr]{phr} \hyperref[TEI.s]{s}\par 
    \item[core: ]
   \hyperref[TEI.abbr]{abbr} \hyperref[TEI.add]{add} \hyperref[TEI.addrLine]{addrLine} \hyperref[TEI.author]{author} \hyperref[TEI.biblScope]{biblScope} \hyperref[TEI.citedRange]{citedRange} \hyperref[TEI.corr]{corr} \hyperref[TEI.date]{date} \hyperref[TEI.del]{del} \hyperref[TEI.distinct]{distinct} \hyperref[TEI.editor]{editor} \hyperref[TEI.email]{email} \hyperref[TEI.emph]{emph} \hyperref[TEI.expan]{expan} \hyperref[TEI.foreign]{foreign} \hyperref[TEI.gloss]{gloss} \hyperref[TEI.head]{head} \hyperref[TEI.headItem]{headItem} \hyperref[TEI.headLabel]{headLabel} \hyperref[TEI.hi]{hi} \hyperref[TEI.item]{item} \hyperref[TEI.l]{l} \hyperref[TEI.label]{label} \hyperref[TEI.measure]{measure} \hyperref[TEI.mentioned]{mentioned} \hyperref[TEI.name]{name} \hyperref[TEI.note]{note} \hyperref[TEI.num]{num} \hyperref[TEI.orig]{orig} \hyperref[TEI.p]{p} \hyperref[TEI.pubPlace]{pubPlace} \hyperref[TEI.publisher]{publisher} \hyperref[TEI.q]{q} \hyperref[TEI.quote]{quote} \hyperref[TEI.ref]{ref} \hyperref[TEI.reg]{reg} \hyperref[TEI.rs]{rs} \hyperref[TEI.said]{said} \hyperref[TEI.sic]{sic} \hyperref[TEI.soCalled]{soCalled} \hyperref[TEI.speaker]{speaker} \hyperref[TEI.stage]{stage} \hyperref[TEI.street]{street} \hyperref[TEI.term]{term} \hyperref[TEI.textLang]{textLang} \hyperref[TEI.time]{time} \hyperref[TEI.title]{title} \hyperref[TEI.unclear]{unclear}\par 
    \item[figures: ]
   \hyperref[TEI.cell]{cell} \hyperref[TEI.figDesc]{figDesc} \hyperref[TEI.figure]{figure} \hyperref[TEI.formula]{formula} \hyperref[TEI.notatedMusic]{notatedMusic} \hyperref[TEI.table]{table}\par 
    \item[header: ]
   \hyperref[TEI.change]{change} \hyperref[TEI.distributor]{distributor} \hyperref[TEI.edition]{edition} \hyperref[TEI.extent]{extent} \hyperref[TEI.licence]{licence}\par 
    \item[linking: ]
   \hyperref[TEI.ab]{ab} \hyperref[TEI.seg]{seg}\par 
    \item[msdescription: ]
   \hyperref[TEI.accMat]{accMat} \hyperref[TEI.acquisition]{acquisition} \hyperref[TEI.additions]{additions} \hyperref[TEI.catchwords]{catchwords} \hyperref[TEI.collation]{collation} \hyperref[TEI.colophon]{colophon} \hyperref[TEI.condition]{condition} \hyperref[TEI.custEvent]{custEvent} \hyperref[TEI.decoNote]{decoNote} \hyperref[TEI.explicit]{explicit} \hyperref[TEI.filiation]{filiation} \hyperref[TEI.finalRubric]{finalRubric} \hyperref[TEI.foliation]{foliation} \hyperref[TEI.heraldry]{heraldry} \hyperref[TEI.incipit]{incipit} \hyperref[TEI.layout]{layout} \hyperref[TEI.material]{material} \hyperref[TEI.msItem]{msItem} \hyperref[TEI.musicNotation]{musicNotation} \hyperref[TEI.objectType]{objectType} \hyperref[TEI.origDate]{origDate} \hyperref[TEI.origPlace]{origPlace} \hyperref[TEI.origin]{origin} \hyperref[TEI.provenance]{provenance} \hyperref[TEI.rubric]{rubric} \hyperref[TEI.secFol]{secFol} \hyperref[TEI.signatures]{signatures} \hyperref[TEI.source]{source} \hyperref[TEI.stamp]{stamp} \hyperref[TEI.summary]{summary} \hyperref[TEI.support]{support} \hyperref[TEI.surrogates]{surrogates} \hyperref[TEI.typeNote]{typeNote} \hyperref[TEI.watermark]{watermark}\par 
    \item[namesdates: ]
   \hyperref[TEI.addName]{addName} \hyperref[TEI.affiliation]{affiliation} \hyperref[TEI.country]{country} \hyperref[TEI.forename]{forename} \hyperref[TEI.genName]{genName} \hyperref[TEI.geogName]{geogName} \hyperref[TEI.nameLink]{nameLink} \hyperref[TEI.orgName]{orgName} \hyperref[TEI.persName]{persName} \hyperref[TEI.placeName]{placeName} \hyperref[TEI.region]{region} \hyperref[TEI.roleName]{roleName} \hyperref[TEI.settlement]{settlement} \hyperref[TEI.surname]{surname}\par 
    \item[textstructure: ]
   \hyperref[TEI.docAuthor]{docAuthor} \hyperref[TEI.docDate]{docDate} \hyperref[TEI.docEdition]{docEdition} \hyperref[TEI.titlePage]{titlePage} \hyperref[TEI.titlePart]{titlePart}\par 
    \item[transcr: ]
   \hyperref[TEI.damage]{damage} \hyperref[TEI.facsimile]{facsimile} \hyperref[TEI.fw]{fw} \hyperref[TEI.metamark]{metamark} \hyperref[TEI.mod]{mod} \hyperref[TEI.restore]{restore} \hyperref[TEI.retrace]{retrace} \hyperref[TEI.secl]{secl} \hyperref[TEI.sourceDoc]{sourceDoc} \hyperref[TEI.supplied]{supplied} \hyperref[TEI.surface]{surface} \hyperref[TEI.surplus]{surplus} \hyperref[TEI.zone]{zone}
    \item[{Peut contenir}]
  Des données textuelles uniquement
    \item[{Exemple}]
  \leavevmode\bgroup\exampleFont \begin{shaded}\noindent\mbox{}{<\textbf{binaryObject}\hspace*{6pt}{mimeType}="{image/gif}">} R0lGODdhMAAwAPAAAAAAAP///ywAAAAAMAAwAAAC8IyPqcvt3wCcDkiLc7C0qwy\mbox{}\newline 
 GHhSWpjQu5yqmCYsapyuvUUlvONmOZtfzgFzByTB10QgxOR0TqBQejhRNzOfkVJ\mbox{}\newline 
 +5YiUqrXF5Y5lKh/DeuNcP5yLWGsEbtLiOSpa/TPg7JpJHxyendzWTBfX0cxOnK\mbox{}\newline 
 PjgBzi4diinWGdkF8kjdfnycQZXZeYGejmJlZeGl9i2icVqaNVailT6F5iJ90m6\mbox{}\newline 
 mvuTS4OK05M0vDk0Q4XUtwvKOzrcd3iq9uisF81M1OIcR7lEewwcLp7tuNNkM3u\mbox{}\newline 
 Nna3F2JQFo97Vriy/Xl4/f1cf5VWzXyym7PH hhx4dbgYKAAA7{</\textbf{binaryObject}>}\end{shaded}\egroup 


    \item[{Modèle de contenu}]
  \fbox{\ttfamily <content>\newline
 <textNode/>\newline
</content>\newline
    } 
    \item[{Schéma Declaration}]
  \mbox{}\hfill\\[-10pt]\begin{Verbatim}[fontsize=\small]
element binaryObject
{
   tei_att.global.attributes,
   tei_att.media.attributes,
   tei_att.timed.attributes,
   tei_att.typed.attributes,
   attribute encoding { list { + } }?,
   text
}
\end{Verbatim}

\end{reflist}  \index{binding=<binding>|oddindex}\index{contemporary=@contemporary!<binding>|oddindex}
\begin{reflist}
\item[]\begin{specHead}{TEI.binding}{<binding> }(reliure) contient la description d'une reliure, i.e. du type de couverture, d'ais, etc., rencontrés. [\xref{http://www.tei-c.org/release/doc/tei-p5-doc/en/html/MS.html\#msphbi}{10.7.3.1. Binding Descriptions}]\end{specHead} 
    \item[{Module}]
  msdescription
    \item[{Attributs}]
  Attributs \hyperref[TEI.att.global]{att.global} (\textit{@xml:id}, \textit{@n}, \textit{@xml:lang}, \textit{@xml:base}, \textit{@xml:space})  (\hyperref[TEI.att.global.rendition]{att.global.rendition} (\textit{@rend}, \textit{@style}, \textit{@rendition})) (\hyperref[TEI.att.global.linking]{att.global.linking} (\textit{@corresp}, \textit{@synch}, \textit{@sameAs}, \textit{@copyOf}, \textit{@next}, \textit{@prev}, \textit{@exclude}, \textit{@select})) (\hyperref[TEI.att.global.analytic]{att.global.analytic} (\textit{@ana})) (\hyperref[TEI.att.global.facs]{att.global.facs} (\textit{@facs})) (\hyperref[TEI.att.global.change]{att.global.change} (\textit{@change})) (\hyperref[TEI.att.global.responsibility]{att.global.responsibility} (\textit{@cert}, \textit{@resp})) (\hyperref[TEI.att.global.source]{att.global.source} (\textit{@source})) \hyperref[TEI.att.datable]{att.datable} (\textit{@calendar}, \textit{@period})  (\hyperref[TEI.att.datable.w3c]{att.datable.w3c} (\textit{@when}, \textit{@notBefore}, \textit{@notAfter}, \textit{@from}, \textit{@to})) (\hyperref[TEI.att.datable.iso]{att.datable.iso} (\textit{@when-iso}, \textit{@notBefore-iso}, \textit{@notAfter-iso}, \textit{@from-iso}, \textit{@to-iso})) (\hyperref[TEI.att.datable.custom]{att.datable.custom} (\textit{@when-custom}, \textit{@notBefore-custom}, \textit{@notAfter-custom}, \textit{@from-custom}, \textit{@to-custom}, \textit{@datingPoint}, \textit{@datingMethod})) \hfil\\[-10pt]\begin{sansreflist}
    \item[@contemporary]
  (contemporaine) spécifie si la reliure est contemporaine ou non de l'essentiel du contenu du manuscrit.
\begin{reflist}
    \item[{Statut}]
  Optionel
    \item[{Type de données}]
  \hyperref[TEI.teidata.xTruthValue]{teidata.xTruthValue}
    \item[{Note}]
  \par
La valeur true indique que la reliure est contemporaine de son contenu ; la valeur false qu'elle ne l'est pas. La valeur unknown est employée quand la date de la reliure ou du manuscrit est inconnue.
\end{reflist}  
\end{sansreflist}  
    \item[{Contenu dans}]
  
    \item[msdescription: ]
   \hyperref[TEI.bindingDesc]{bindingDesc}
    \item[{Peut contenir}]
  
    \item[core: ]
   \hyperref[TEI.p]{p}\par 
    \item[linking: ]
   \hyperref[TEI.ab]{ab}\par 
    \item[msdescription: ]
   \hyperref[TEI.condition]{condition} \hyperref[TEI.decoNote]{decoNote}
    \item[{Exemple}]
  \leavevmode\bgroup\exampleFont \begin{shaded}\noindent\mbox{}{<\textbf{binding}\hspace*{6pt}{contemporary}="{true}">}\mbox{}\newline 
\hspace*{6pt}{<\textbf{p}>}\mbox{}\newline 
\hspace*{6pt}\hspace*{6pt}{<\textbf{index}\hspace*{6pt}{indexName}="{typo\textunderscore reliure}">}\mbox{}\newline 
\hspace*{6pt}\hspace*{6pt}\hspace*{6pt}{<\textbf{term}>}Reliure à la grecque, sur ais{</\textbf{term}>}\mbox{}\newline 
\hspace*{6pt}\hspace*{6pt}{</\textbf{index}>}\mbox{}\newline 
\hspace*{6pt}\hspace*{6pt}{<\textbf{index}\hspace*{6pt}{indexName}="{typo\textunderscore decor}">}\mbox{}\newline 
\hspace*{6pt}\hspace*{6pt}\hspace*{6pt}{<\textbf{term}>}Décor de rinceaux{</\textbf{term}>}\mbox{}\newline 
\hspace*{6pt}\hspace*{6pt}{</\textbf{index}>} Reliure à la grecque en {<\textbf{material}>}maroquin{</\textbf{material}>} orange{</\textbf{p}>}\mbox{}\newline 
\hspace*{6pt}{<\textbf{decoNote}\hspace*{6pt}{type}="{plats}">} aux armes de Henri II dorées sur une pièce de maroquin olive\mbox{}\newline 
\hspace*{6pt}\hspace*{6pt} découpée à la forme exacte des armes (104 mm), mosaïquée dans un rectangle central aux\mbox{}\newline 
\hspace*{6pt}\hspace*{6pt} angles orné d'un léger motif de rinceaux peints en noir, le tout encadré d'une large\mbox{}\newline 
\hspace*{6pt}\hspace*{6pt} bordure mosaïquée de maroquin rouge, à plein décor de rinceaux dorés (incluant un\mbox{}\newline 
\hspace*{6pt}\hspace*{6pt} croissant dans les angles) dessinés en réserve sur un fond pointillé doré.{</\textbf{decoNote}>}\mbox{}\newline 
\hspace*{6pt}{<\textbf{decoNote}\hspace*{6pt}{type}="{plat\textunderscore sup}">}Au plat supérieur, titre {<\textbf{q}>}i • schonerii • opera •{</\textbf{q}>} doré\mbox{}\newline 
\hspace*{6pt}\hspace*{6pt} au-dessus du bloc armorial.{</\textbf{decoNote}>}\mbox{}\newline 
\hspace*{6pt}{<\textbf{decoNote}\hspace*{6pt}{type}="{plat\textunderscore inf}"/>}\mbox{}\newline 
\hspace*{6pt}{<\textbf{decoNote}\hspace*{6pt}{type}="{dos}">}Dos long à décor analogue avec pièces losangées de maroquin rouge et\mbox{}\newline 
\hspace*{6pt}\hspace*{6pt} brun mosaïquées, respectivement au centre et aux deux extrémités du dos, ornées d'un\mbox{}\newline 
\hspace*{6pt}\hspace*{6pt} décor de rinceaux doré en réserve sur un fond doré pointillé, avec fer azuré au chapeau\mbox{}\newline 
\hspace*{6pt}\hspace*{6pt} à chaque extrémité ; chaque pièce de maroquin est redessinée par un encadrement argenté,\mbox{}\newline 
\hspace*{6pt}\hspace*{6pt} lui-même complété de rinceaux sur les côtés et relevé par des traits tracés de plume à\mbox{}\newline 
\hspace*{6pt}\hspace*{6pt} effet de rayures.{</\textbf{decoNote}>}\mbox{}\newline 
\hspace*{6pt}{<\textbf{decoNote}\hspace*{6pt}{type}="{tranchefiles}">}Tranchefiles doubles bicolores : points droits sur chevrons,\mbox{}\newline 
\hspace*{6pt}\hspace*{6pt} bleus et jaunes.{</\textbf{decoNote}>}\mbox{}\newline 
\hspace*{6pt}{<\textbf{decoNote}\hspace*{6pt}{type}="{coupes}">}Chants des ais rainurés.{</\textbf{decoNote}>}\mbox{}\newline 
\hspace*{6pt}{<\textbf{decoNote}\hspace*{6pt}{type}="{annexes}">}Traces de petits boulons aux angles du rectangle intérieur ;\mbox{}\newline 
\hspace*{6pt}\hspace*{6pt} traces des quatre lanières tressées d'origine sur les deux plats ; pas de traces de\mbox{}\newline 
\hspace*{6pt}\hspace*{6pt} sabots.{</\textbf{decoNote}>}\mbox{}\newline 
\hspace*{6pt}{<\textbf{decoNote}\hspace*{6pt}{type}="{tranches}">}Tranches dorées, ciselées et peintes (teinte rosée), à décor de\mbox{}\newline 
\hspace*{6pt}\hspace*{6pt} rinceaux incluant des éléments de l'héraldique royale : triple croissant en tête, grand\mbox{}\newline 
\hspace*{6pt}\hspace*{6pt} H couronné associé à un croissant en gouttière, chiffre HH en queue.{</\textbf{decoNote}>}\mbox{}\newline 
\hspace*{6pt}{<\textbf{decoNote}\hspace*{6pt}{type}="{contreplats}"/>}\mbox{}\newline 
\hspace*{6pt}{<\textbf{decoNote}\hspace*{6pt}{type}="{chasses}">}Absence de chasses.{</\textbf{decoNote}>}\mbox{}\newline 
\textit{<!-- Description des gardes : gardes blanches ; gardes couleurs (marbrées, gaufrées, peintes, dominotées, etc.) généralement suivies de gardes blanches ; dans tous les cas, spécifier le nombre de gardes (début + fin du volume)-->}\mbox{}\newline 
\hspace*{6pt}{<\textbf{decoNote}\hspace*{6pt}{type}="{gardes}">}Gardes (3+2), filigrane {<\textbf{watermark}>}B{</\textbf{watermark}>}. {</\textbf{decoNote}>}\mbox{}\newline 
\textit{<!-- Élément qui inclut aussi bien des remarques sur la couture que les charnières, claies ou modes d'attaches des plats : tous éléments de la structure dont la description est jugée utile à la description et l'identification de la reliure-->}\mbox{}\newline 
\hspace*{6pt}{<\textbf{decoNote}\hspace*{6pt}{type}="{structure}"/>}\mbox{}\newline 
\hspace*{6pt}{<\textbf{condition}/>}\mbox{}\newline 
{</\textbf{binding}>}\end{shaded}\egroup 


    \item[{Exemple}]
  \leavevmode\bgroup\exampleFont \begin{shaded}\noindent\mbox{}{<\textbf{bindingDesc}>}\mbox{}\newline 
\hspace*{6pt}{<\textbf{binding}\hspace*{6pt}{contemporary}="{true}">}\mbox{}\newline 
\hspace*{6pt}\hspace*{6pt}{<\textbf{p}>}\mbox{}\newline 
\hspace*{6pt}\hspace*{6pt}\hspace*{6pt}{<\textbf{index}\hspace*{6pt}{indexName}="{typo\textunderscore reliure}">}\mbox{}\newline 
\hspace*{6pt}\hspace*{6pt}\hspace*{6pt}\hspace*{6pt}{<\textbf{term}>}Reliure à décor{</\textbf{term}>}\mbox{}\newline 
\hspace*{6pt}\hspace*{6pt}\hspace*{6pt}{</\textbf{index}>}\mbox{}\newline 
\hspace*{6pt}\hspace*{6pt}\hspace*{6pt}{<\textbf{index}\hspace*{6pt}{indexName}="{typo\textunderscore decor}">}\mbox{}\newline 
\hspace*{6pt}\hspace*{6pt}\hspace*{6pt}\hspace*{6pt}{<\textbf{term}>}Compartiments espacés{</\textbf{term}>}\mbox{}\newline 
\hspace*{6pt}\hspace*{6pt}\hspace*{6pt}{</\textbf{index}>} Reliure en {<\textbf{material}>}maroquin{</\textbf{material}>} rouge sombre{</\textbf{p}>}\mbox{}\newline 
\hspace*{6pt}\hspace*{6pt}{<\textbf{decoNote}\hspace*{6pt}{type}="{plats}">} aux armes du chancelier Pierre Séguier, à décor de compartiments\mbox{}\newline 
\hspace*{6pt}\hspace*{6pt}\hspace*{6pt}\hspace*{6pt} complétés de fers filigranés, parmi lesquels un fer à la petite tête (type B).{</\textbf{decoNote}>}\mbox{}\newline 
\hspace*{6pt}\hspace*{6pt}{<\textbf{decoNote}\hspace*{6pt}{type}="{plat\textunderscore sup}"/>}\mbox{}\newline 
\hspace*{6pt}\hspace*{6pt}{<\textbf{decoNote}\hspace*{6pt}{type}="{plat\textunderscore inf}"/>}\mbox{}\newline 
\hspace*{6pt}\hspace*{6pt}{<\textbf{decoNote}\hspace*{6pt}{type}="{dos}">}Dos à 6 nerfs, à décor filigrané analogue ; palette ornée sur les\mbox{}\newline 
\hspace*{6pt}\hspace*{6pt}\hspace*{6pt}\hspace*{6pt} nerfs et en tête et queue du dos ; titrage dans le 2e caisson.{</\textbf{decoNote}>}\mbox{}\newline 
\hspace*{6pt}\hspace*{6pt}{<\textbf{decoNote}\hspace*{6pt}{type}="{tranchefiles}">}Tranchefiles à chapiteau tricolore (bleu, blanc et rose).{</\textbf{decoNote}>}\mbox{}\newline 
\hspace*{6pt}\hspace*{6pt}{<\textbf{decoNote}\hspace*{6pt}{type}="{coupes}">}Coupes ornées.{</\textbf{decoNote}>}\mbox{}\newline 
\hspace*{6pt}\hspace*{6pt}{<\textbf{decoNote}\hspace*{6pt}{type}="{annexes}"/>}\mbox{}\newline 
\hspace*{6pt}\hspace*{6pt}{<\textbf{decoNote}\hspace*{6pt}{type}="{tranches}">}Tranches dorées.{</\textbf{decoNote}>}\mbox{}\newline 
\hspace*{6pt}\hspace*{6pt}{<\textbf{decoNote}\hspace*{6pt}{type}="{contreplats}">}Contregardes en papier marbré à petit peigne, dans les tons\mbox{}\newline 
\hspace*{6pt}\hspace*{6pt}\hspace*{6pt}\hspace*{6pt} bleu, blanc, jaune, rouge et blanc.{</\textbf{decoNote}>}\mbox{}\newline 
\hspace*{6pt}\hspace*{6pt}{<\textbf{decoNote}\hspace*{6pt}{type}="{chasses}">}Chasses ornées.{</\textbf{decoNote}>}\mbox{}\newline 
\textit{<!-- Description des gardes : gardes blanches ; gardes couleurs (marbrées, gaufrées, peintes, dominotées, etc.) généralement suivies de gardes blanches ; dans tous les cas, spécifier le nombre de gardes (début + fin du volume)-->}\mbox{}\newline 
\hspace*{6pt}\hspace*{6pt}{<\textbf{decoNote}\hspace*{6pt}{type}="{gardes}">}\mbox{}\newline 
\hspace*{6pt}\hspace*{6pt}\hspace*{6pt}{<\textbf{watermark}/>}\mbox{}\newline 
\hspace*{6pt}\hspace*{6pt}{</\textbf{decoNote}>}\mbox{}\newline 
\textit{<!-- Élément qui inclut aussi bien des remarques sur la couture que les charnières, claies ou modes d'attaches des plats : tous éléments de la structure dont la description est jugée utile à la description et l'identification de la reliure-->}\mbox{}\newline 
\hspace*{6pt}\hspace*{6pt}{<\textbf{decoNote}\hspace*{6pt}{type}="{structure}"/>}\mbox{}\newline 
\hspace*{6pt}\hspace*{6pt}{<\textbf{condition}>}Quelques taches sombres \mbox{}\newline 
\textit{<!--surla -->} sur le plat supérieur et larges\mbox{}\newline 
\hspace*{6pt}\hspace*{6pt}\hspace*{6pt}\hspace*{6pt} éraflures du cuir au plat inférieur. Restauration en queue du mors inférieur (bande de\mbox{}\newline 
\hspace*{6pt}\hspace*{6pt}\hspace*{6pt}\hspace*{6pt} cuir).{</\textbf{condition}>}\mbox{}\newline 
\hspace*{6pt}{</\textbf{binding}>}\mbox{}\newline 
{</\textbf{bindingDesc}>}\end{shaded}\egroup 


    \item[{Modèle de contenu}]
  \mbox{}\hfill\\[-10pt]\begin{Verbatim}[fontsize=\small]
<content>
 <alternate maxOccurs="unbounded"
  minOccurs="1">
  <classRef key="model.pLike"/>
  <elementRef key="condition"/>
  <elementRef key="decoNote"/>
 </alternate>
</content>
    
\end{Verbatim}

    \item[{Schéma Declaration}]
  \mbox{}\hfill\\[-10pt]\begin{Verbatim}[fontsize=\small]
element binding
{
   tei_att.global.attributes,
   tei_att.datable.attributes,
   attribute contemporary { text }?,
   ( tei_model.pLike | tei_condition | tei_decoNote )+
}
\end{Verbatim}

\end{reflist}  \index{bindingDesc=<bindingDesc>|oddindex}
\begin{reflist}
\item[]\begin{specHead}{TEI.bindingDesc}{<bindingDesc> }(description de la reliure) décrit les reliures actuelles et anciennes d'un manuscrit, soit en une série de paragraphes \textit{p}, soit sous la forme d'une série d'éléments \hyperref[TEI.binding]{<binding>}, un pour chaque reliure [\xref{http://www.tei-c.org/release/doc/tei-p5-doc/en/html/MS.html\#msphbi}{10.7.3.1. Binding Descriptions}]\end{specHead} 
    \item[{Module}]
  msdescription
    \item[{Attributs}]
  Attributs \hyperref[TEI.att.global]{att.global} (\textit{@xml:id}, \textit{@n}, \textit{@xml:lang}, \textit{@xml:base}, \textit{@xml:space})  (\hyperref[TEI.att.global.rendition]{att.global.rendition} (\textit{@rend}, \textit{@style}, \textit{@rendition})) (\hyperref[TEI.att.global.linking]{att.global.linking} (\textit{@corresp}, \textit{@synch}, \textit{@sameAs}, \textit{@copyOf}, \textit{@next}, \textit{@prev}, \textit{@exclude}, \textit{@select})) (\hyperref[TEI.att.global.analytic]{att.global.analytic} (\textit{@ana})) (\hyperref[TEI.att.global.facs]{att.global.facs} (\textit{@facs})) (\hyperref[TEI.att.global.change]{att.global.change} (\textit{@change})) (\hyperref[TEI.att.global.responsibility]{att.global.responsibility} (\textit{@cert}, \textit{@resp})) (\hyperref[TEI.att.global.source]{att.global.source} (\textit{@source}))
    \item[{Membre du}]
  \hyperref[TEI.model.physDescPart]{model.physDescPart}
    \item[{Contenu dans}]
  
    \item[msdescription: ]
   \hyperref[TEI.physDesc]{physDesc}
    \item[{Peut contenir}]
  
    \item[core: ]
   \hyperref[TEI.p]{p}\par 
    \item[linking: ]
   \hyperref[TEI.ab]{ab}\par 
    \item[msdescription: ]
   \hyperref[TEI.binding]{binding} \hyperref[TEI.condition]{condition} \hyperref[TEI.decoNote]{decoNote}
    \item[{Exemple}]
  \leavevmode\bgroup\exampleFont \begin{shaded}\noindent\mbox{}{<\textbf{bindingDesc}>}\mbox{}\newline 
\hspace*{6pt}{<\textbf{binding}\hspace*{6pt}{contemporary}="{true}">}\mbox{}\newline 
\hspace*{6pt}\hspace*{6pt}{<\textbf{p}>}\mbox{}\newline 
\hspace*{6pt}\hspace*{6pt}\hspace*{6pt}{<\textbf{index}\hspace*{6pt}{indexName}="{typo\textunderscore reliure}">}\mbox{}\newline 
\hspace*{6pt}\hspace*{6pt}\hspace*{6pt}\hspace*{6pt}{<\textbf{term}>}Reliure à décor{</\textbf{term}>}\mbox{}\newline 
\hspace*{6pt}\hspace*{6pt}\hspace*{6pt}{</\textbf{index}>}\mbox{}\newline 
\hspace*{6pt}\hspace*{6pt}\hspace*{6pt}{<\textbf{index}\hspace*{6pt}{indexName}="{typo\textunderscore decor}">}\mbox{}\newline 
\hspace*{6pt}\hspace*{6pt}\hspace*{6pt}\hspace*{6pt}{<\textbf{term}>}Décor mosaïqué, avec formes géométriques à répétition{</\textbf{term}>}\mbox{}\newline 
\hspace*{6pt}\hspace*{6pt}\hspace*{6pt}{</\textbf{index}>} Reliure en {<\textbf{material}>}maroquin{</\textbf{material}>} citron{</\textbf{p}>}\mbox{}\newline 
\hspace*{6pt}\hspace*{6pt}{<\textbf{decoNote}\hspace*{6pt}{type}="{plats}">}à décor mosaïqué dit à répétition, dont l’effet de dallage est\mbox{}\newline 
\hspace*{6pt}\hspace*{6pt}\hspace*{6pt}\hspace*{6pt} obtenu par des pièces polylobées de maroquin noir, ornée chacune d’une composition de\mbox{}\newline 
\hspace*{6pt}\hspace*{6pt}\hspace*{6pt}\hspace*{6pt} petits fers plein or, cantonnées de petits disques de maroquin rouge ponctué chacun\mbox{}\newline 
\hspace*{6pt}\hspace*{6pt}\hspace*{6pt}\hspace*{6pt} d’un cercle plein or, le tout serti de filets dorés courbes.{</\textbf{decoNote}>}\mbox{}\newline 
\hspace*{6pt}\hspace*{6pt}{<\textbf{decoNote}\hspace*{6pt}{type}="{plat\textunderscore sup}"/>}\mbox{}\newline 
\hspace*{6pt}\hspace*{6pt}{<\textbf{decoNote}\hspace*{6pt}{type}="{plat\textunderscore inf}"/>}\mbox{}\newline 
\hspace*{6pt}\hspace*{6pt}{<\textbf{decoNote}\hspace*{6pt}{type}="{dos}">}Dos à 5 nerfs à décor analogue (pièce polylobée de maroquin noir\mbox{}\newline 
\hspace*{6pt}\hspace*{6pt}\hspace*{6pt}\hspace*{6pt} avec composition identique, cantonnée de quatre disques rouges, ponctués du même\mbox{}\newline 
\hspace*{6pt}\hspace*{6pt}\hspace*{6pt}\hspace*{6pt} cercle plein or) ; filets dorés sur les nerfs ; pièce de titre rouge dans le 2e\mbox{}\newline 
\hspace*{6pt}\hspace*{6pt}\hspace*{6pt}\hspace*{6pt} caisson, soulignée de deux lignes de points dorés identiques à celle portée en tête et\mbox{}\newline 
\hspace*{6pt}\hspace*{6pt}\hspace*{6pt}\hspace*{6pt} queue du dos, sur une bande de maroquin noir.{</\textbf{decoNote}>}\mbox{}\newline 
\hspace*{6pt}\hspace*{6pt}{<\textbf{decoNote}\hspace*{6pt}{type}="{tranchefiles}">}Tranchefiles simples droites, tricolores (noir, bleu et\mbox{}\newline 
\hspace*{6pt}\hspace*{6pt}\hspace*{6pt}\hspace*{6pt} rose).{</\textbf{decoNote}>}\mbox{}\newline 
\hspace*{6pt}\hspace*{6pt}{<\textbf{decoNote}\hspace*{6pt}{type}="{coupes}">}Coupes dorées, proposant en alternance un filet simple et une\mbox{}\newline 
\hspace*{6pt}\hspace*{6pt}\hspace*{6pt}\hspace*{6pt} succession de traits obliques.{</\textbf{decoNote}>}\mbox{}\newline 
\hspace*{6pt}\hspace*{6pt}{<\textbf{decoNote}\hspace*{6pt}{type}="{annexes}">}Signet de soie rose.{</\textbf{decoNote}>}\mbox{}\newline 
\hspace*{6pt}\hspace*{6pt}{<\textbf{decoNote}\hspace*{6pt}{type}="{tranches}">}Tranches dorées sur marbrure à motif caillouté, dans les tons\mbox{}\newline 
\hspace*{6pt}\hspace*{6pt}\hspace*{6pt}\hspace*{6pt} bleu et rose.{</\textbf{decoNote}>}\mbox{}\newline 
\hspace*{6pt}\hspace*{6pt}{<\textbf{decoNote}\hspace*{6pt}{type}="{contreplats}"/>}\mbox{}\newline 
\hspace*{6pt}\hspace*{6pt}{<\textbf{decoNote}\hspace*{6pt}{type}="{chasses}">}Chasses ornées d’une roulette à motif de zigzag.{</\textbf{decoNote}>}\mbox{}\newline 
\textit{<!-- Description des gardes : gardes blanches ; gardes couleurs (marbrées, gaufrées, peintes, dominotées, etc.) généralement suivies de gardes blanches ; dans tous les cas, spécifier le nombre de gardes (début + fin du volume)-->}\mbox{}\newline 
\hspace*{6pt}\hspace*{6pt}{<\textbf{decoNote}\hspace*{6pt}{type}="{gardes}">}Gardes en papier plein or et gardes blanches (1 + 1), sans\mbox{}\newline 
\hspace*{6pt}\hspace*{6pt}\hspace*{6pt}\hspace*{6pt} filigrane.{<\textbf{watermark}/>}\mbox{}\newline 
\hspace*{6pt}\hspace*{6pt}{</\textbf{decoNote}>}\mbox{}\newline 
\textit{<!-- Élément qui inclut aussi bien des remarques sur la couture que les charnières, claies ou modes d'attaches des plats : tous éléments de la structure dont la description est jugée utile à la description et l'identification de la reliure-->}\mbox{}\newline 
\hspace*{6pt}\hspace*{6pt}{<\textbf{decoNote}\hspace*{6pt}{type}="{structure}"/>}\mbox{}\newline 
\hspace*{6pt}\hspace*{6pt}{<\textbf{condition}/>}\mbox{}\newline 
\hspace*{6pt}{</\textbf{binding}>}\mbox{}\newline 
{</\textbf{bindingDesc}>}\end{shaded}\egroup 


    \item[{Modèle de contenu}]
  \mbox{}\hfill\\[-10pt]\begin{Verbatim}[fontsize=\small]
<content>
 <alternate maxOccurs="1" minOccurs="1">
  <alternate maxOccurs="unbounded"
   minOccurs="1">
   <classRef key="model.pLike"/>
   <elementRef key="decoNote"/>
   <elementRef key="condition"/>
  </alternate>
  <elementRef key="binding"
   maxOccurs="unbounded" minOccurs="1"/>
 </alternate>
</content>
    
\end{Verbatim}

    \item[{Schéma Declaration}]
  \mbox{}\hfill\\[-10pt]\begin{Verbatim}[fontsize=\small]
element bindingDesc
{
   tei_att.global.attributes,
   ( ( tei_model.pLike | tei_decoNote | tei_condition )+ | tei_binding+ )
}
\end{Verbatim}

\end{reflist}  \index{body=<body>|oddindex}
\begin{reflist}
\item[]\begin{specHead}{TEI.body}{<body> }(corps du texte) contient la totalité du corps d’un seul texte simple, à l’exclusion de toute partie pré- ou post-liminaire. [\xref{http://www.tei-c.org/release/doc/tei-p5-doc/en/html/DS.html\#DS}{4. Default Text Structure}]\end{specHead} 
    \item[{Module}]
  textstructure
    \item[{Attributs}]
  Attributs \hyperref[TEI.att.global]{att.global} (\textit{@xml:id}, \textit{@n}, \textit{@xml:lang}, \textit{@xml:base}, \textit{@xml:space})  (\hyperref[TEI.att.global.rendition]{att.global.rendition} (\textit{@rend}, \textit{@style}, \textit{@rendition})) (\hyperref[TEI.att.global.linking]{att.global.linking} (\textit{@corresp}, \textit{@synch}, \textit{@sameAs}, \textit{@copyOf}, \textit{@next}, \textit{@prev}, \textit{@exclude}, \textit{@select})) (\hyperref[TEI.att.global.analytic]{att.global.analytic} (\textit{@ana})) (\hyperref[TEI.att.global.facs]{att.global.facs} (\textit{@facs})) (\hyperref[TEI.att.global.change]{att.global.change} (\textit{@change})) (\hyperref[TEI.att.global.responsibility]{att.global.responsibility} (\textit{@cert}, \textit{@resp})) (\hyperref[TEI.att.global.source]{att.global.source} (\textit{@source})) \hyperref[TEI.att.declaring]{att.declaring} (\textit{@decls}) 
    \item[{Contenu dans}]
  
    \item[textstructure: ]
   \hyperref[TEI.floatingText]{floatingText} \hyperref[TEI.text]{text}
    \item[{Peut contenir}]
  
    \item[analysis: ]
   \hyperref[TEI.interp]{interp} \hyperref[TEI.interpGrp]{interpGrp} \hyperref[TEI.span]{span} \hyperref[TEI.spanGrp]{spanGrp}\par 
    \item[core: ]
   \hyperref[TEI.bibl]{bibl} \hyperref[TEI.biblStruct]{biblStruct} \hyperref[TEI.cb]{cb} \hyperref[TEI.cit]{cit} \hyperref[TEI.desc]{desc} \hyperref[TEI.divGen]{divGen} \hyperref[TEI.gap]{gap} \hyperref[TEI.gb]{gb} \hyperref[TEI.head]{head} \hyperref[TEI.index]{index} \hyperref[TEI.l]{l} \hyperref[TEI.label]{label} \hyperref[TEI.lb]{lb} \hyperref[TEI.lg]{lg} \hyperref[TEI.list]{list} \hyperref[TEI.listBibl]{listBibl} \hyperref[TEI.meeting]{meeting} \hyperref[TEI.milestone]{milestone} \hyperref[TEI.note]{note} \hyperref[TEI.p]{p} \hyperref[TEI.pb]{pb} \hyperref[TEI.q]{q} \hyperref[TEI.quote]{quote} \hyperref[TEI.said]{said} \hyperref[TEI.sp]{sp} \hyperref[TEI.stage]{stage}\par 
    \item[figures: ]
   \hyperref[TEI.figure]{figure} \hyperref[TEI.notatedMusic]{notatedMusic} \hyperref[TEI.table]{table}\par 
    \item[header: ]
   \hyperref[TEI.biblFull]{biblFull}\par 
    \item[iso-fs: ]
   \hyperref[TEI.fLib]{fLib} \hyperref[TEI.fs]{fs} \hyperref[TEI.fvLib]{fvLib}\par 
    \item[linking: ]
   \hyperref[TEI.ab]{ab} \hyperref[TEI.alt]{alt} \hyperref[TEI.altGrp]{altGrp} \hyperref[TEI.anchor]{anchor} \hyperref[TEI.join]{join} \hyperref[TEI.joinGrp]{joinGrp} \hyperref[TEI.link]{link} \hyperref[TEI.linkGrp]{linkGrp} \hyperref[TEI.timeline]{timeline}\par 
    \item[msdescription: ]
   \hyperref[TEI.msDesc]{msDesc} \hyperref[TEI.source]{source}\par 
    \item[namesdates: ]
   \hyperref[TEI.listOrg]{listOrg} \hyperref[TEI.listPlace]{listPlace}\par 
    \item[spoken: ]
   \hyperref[TEI.annotationBlock]{annotationBlock}\par 
    \item[textstructure: ]
   \hyperref[TEI.div]{div} \hyperref[TEI.docAuthor]{docAuthor} \hyperref[TEI.docDate]{docDate} \hyperref[TEI.floatingText]{floatingText}\par 
    \item[transcr: ]
   \hyperref[TEI.addSpan]{addSpan} \hyperref[TEI.damageSpan]{damageSpan} \hyperref[TEI.delSpan]{delSpan} \hyperref[TEI.fw]{fw} \hyperref[TEI.listTranspose]{listTranspose} \hyperref[TEI.metamark]{metamark} \hyperref[TEI.space]{space} \hyperref[TEI.substJoin]{substJoin}
    \item[{Exemple}]
  \leavevmode\bgroup\exampleFont \begin{shaded}\noindent\mbox{}{<\textbf{body}>}\mbox{}\newline 
\hspace*{6pt}{<\textbf{l}>}Nu scylun hergan hefaenricaes uard{</\textbf{l}>}\mbox{}\newline 
\hspace*{6pt}{<\textbf{l}>}metudæs maecti end his modgidanc{</\textbf{l}>}\mbox{}\newline 
\hspace*{6pt}{<\textbf{l}>}uerc uuldurfadur sue he uundra gihuaes{</\textbf{l}>}\mbox{}\newline 
\hspace*{6pt}{<\textbf{l}>}eci dryctin or astelidæ{</\textbf{l}>}\mbox{}\newline 
\hspace*{6pt}{<\textbf{l}>}he aerist scop aelda barnum{</\textbf{l}>}\mbox{}\newline 
\hspace*{6pt}{<\textbf{l}>}heben til hrofe haleg scepen.{</\textbf{l}>}\mbox{}\newline 
\hspace*{6pt}{<\textbf{l}>}tha middungeard moncynnæs uard{</\textbf{l}>}\mbox{}\newline 
\hspace*{6pt}{<\textbf{l}>}eci dryctin æfter tiadæ{</\textbf{l}>}\mbox{}\newline 
\hspace*{6pt}{<\textbf{l}>}firum foldu frea allmectig{</\textbf{l}>}\mbox{}\newline 
\hspace*{6pt}{<\textbf{trailer}>}primo cantauit Cædmon istud carmen.{</\textbf{trailer}>}\mbox{}\newline 
{</\textbf{body}>}\end{shaded}\egroup 


    \item[{Modèle de contenu}]
  \mbox{}\hfill\\[-10pt]\begin{Verbatim}[fontsize=\small]
<content>
 <sequence maxOccurs="1" minOccurs="1">
  <classRef key="model.global"
   maxOccurs="unbounded" minOccurs="0"/>
  <sequence maxOccurs="1" minOccurs="0">
   <classRef key="model.divTop"/>
   <alternate maxOccurs="unbounded"
    minOccurs="0">
    <classRef key="model.global"/>
    <classRef key="model.divTop"/>
   </alternate>
  </sequence>
  <sequence maxOccurs="1" minOccurs="0">
   <classRef key="model.divGenLike"/>
   <alternate maxOccurs="unbounded"
    minOccurs="0">
    <classRef key="model.global"/>
    <classRef key="model.divGenLike"/>
   </alternate>
  </sequence>
  <alternate maxOccurs="1" minOccurs="1">
   <sequence maxOccurs="unbounded"
    minOccurs="1">
    <classRef key="model.divLike"/>
    <alternate maxOccurs="unbounded"
     minOccurs="0">
     <classRef key="model.global"/>
     <classRef key="model.divGenLike"/>
    </alternate>
   </sequence>
   <sequence maxOccurs="unbounded"
    minOccurs="1">
    <classRef key="model.div1Like"/>
    <alternate maxOccurs="unbounded"
     minOccurs="0">
     <classRef key="model.global"/>
     <classRef key="model.divGenLike"/>
    </alternate>
   </sequence>
   <sequence maxOccurs="1" minOccurs="1">
    <sequence maxOccurs="unbounded"
     minOccurs="1">
     <classRef key="model.common"/>
     <classRef key="model.global"
      maxOccurs="unbounded" minOccurs="0"/>
    </sequence>
    <alternate maxOccurs="1" minOccurs="0">
     <sequence maxOccurs="unbounded"
      minOccurs="1">
      <classRef key="model.divLike"/>
      <alternate maxOccurs="unbounded"
       minOccurs="0">
       <classRef key="model.global"/>
       <classRef key="model.divGenLike"/>
      </alternate>
     </sequence>
     <sequence maxOccurs="unbounded"
      minOccurs="1">
      <classRef key="model.div1Like"/>
      <alternate maxOccurs="unbounded"
       minOccurs="0">
       <classRef key="model.global"/>
       <classRef key="model.divGenLike"/>
      </alternate>
     </sequence>
    </alternate>
   </sequence>
  </alternate>
  <sequence maxOccurs="unbounded"
   minOccurs="0">
   <classRef key="model.divBottom"/>
   <classRef key="model.global"
    maxOccurs="unbounded" minOccurs="0"/>
  </sequence>
 </sequence>
</content>
    
\end{Verbatim}

    \item[{Schéma Declaration}]
  \mbox{}\hfill\\[-10pt]\begin{Verbatim}[fontsize=\small]
element body
{
   tei_att.global.attributes,
   tei_att.declaring.attributes,
   (
      tei_model.global*,
      ( tei_model.divTop, ( tei_model.global | tei_model.divTop )* )?,
      ( tei_model.divGenLike, ( tei_model.global | tei_model.divGenLike )* )?,
      (
         ( tei_model.divLike, ( tei_model.global | tei_model.divGenLike )* )+
       | ( tei_model.div1Like, ( tei_model.global | tei_model.divGenLike )* )+
       | (
            ( tei_model.common, tei_model.global* )+,
            (
               (
                  tei_model.divLike,
                  ( tei_model.global | tei_model.divGenLike )*
               )+
             | (
                  tei_model.div1Like,
                  ( tei_model.global | tei_model.divGenLike )*
               )+
            )?
         )
      ),
      ( tei_model.divBottom, tei_model.global* )*
   )
}
\end{Verbatim}

\end{reflist}  \index{c=<c>|oddindex}
\begin{reflist}
\item[]\begin{specHead}{TEI.c}{<c> }(caractère) représente un caractère [\xref{http://www.tei-c.org/release/doc/tei-p5-doc/en/html/AI.html\#AILC}{17.1. Linguistic Segment Categories}]\end{specHead} 
    \item[{Module}]
  analysis
    \item[{Attributs}]
  Attributs \hyperref[TEI.att.global]{att.global} (\textit{@xml:id}, \textit{@n}, \textit{@xml:lang}, \textit{@xml:base}, \textit{@xml:space})  (\hyperref[TEI.att.global.rendition]{att.global.rendition} (\textit{@rend}, \textit{@style}, \textit{@rendition})) (\hyperref[TEI.att.global.linking]{att.global.linking} (\textit{@corresp}, \textit{@synch}, \textit{@sameAs}, \textit{@copyOf}, \textit{@next}, \textit{@prev}, \textit{@exclude}, \textit{@select})) (\hyperref[TEI.att.global.analytic]{att.global.analytic} (\textit{@ana})) (\hyperref[TEI.att.global.facs]{att.global.facs} (\textit{@facs})) (\hyperref[TEI.att.global.change]{att.global.change} (\textit{@change})) (\hyperref[TEI.att.global.responsibility]{att.global.responsibility} (\textit{@cert}, \textit{@resp})) (\hyperref[TEI.att.global.source]{att.global.source} (\textit{@source})) \hyperref[TEI.att.segLike]{att.segLike} (\textit{@function})  (\hyperref[TEI.att.datcat]{att.datcat} (\textit{@datcat}, \textit{@valueDatcat})) (\hyperref[TEI.att.fragmentable]{att.fragmentable} (\textit{@part})) \hyperref[TEI.att.typed]{att.typed} (\textit{@type}, \textit{@subtype}) 
    \item[{Membre du}]
  \hyperref[TEI.model.linePart]{model.linePart} \hyperref[TEI.model.segLike]{model.segLike} 
    \item[{Contenu dans}]
  
    \item[analysis: ]
   \hyperref[TEI.cl]{cl} \hyperref[TEI.m]{m} \hyperref[TEI.pc]{pc} \hyperref[TEI.phr]{phr} \hyperref[TEI.s]{s} \hyperref[TEI.w]{w}\par 
    \item[core: ]
   \hyperref[TEI.abbr]{abbr} \hyperref[TEI.add]{add} \hyperref[TEI.addrLine]{addrLine} \hyperref[TEI.author]{author} \hyperref[TEI.bibl]{bibl} \hyperref[TEI.biblScope]{biblScope} \hyperref[TEI.citedRange]{citedRange} \hyperref[TEI.corr]{corr} \hyperref[TEI.date]{date} \hyperref[TEI.del]{del} \hyperref[TEI.distinct]{distinct} \hyperref[TEI.editor]{editor} \hyperref[TEI.email]{email} \hyperref[TEI.emph]{emph} \hyperref[TEI.expan]{expan} \hyperref[TEI.foreign]{foreign} \hyperref[TEI.gloss]{gloss} \hyperref[TEI.head]{head} \hyperref[TEI.headItem]{headItem} \hyperref[TEI.headLabel]{headLabel} \hyperref[TEI.hi]{hi} \hyperref[TEI.item]{item} \hyperref[TEI.l]{l} \hyperref[TEI.label]{label} \hyperref[TEI.measure]{measure} \hyperref[TEI.mentioned]{mentioned} \hyperref[TEI.name]{name} \hyperref[TEI.note]{note} \hyperref[TEI.num]{num} \hyperref[TEI.orig]{orig} \hyperref[TEI.p]{p} \hyperref[TEI.pubPlace]{pubPlace} \hyperref[TEI.publisher]{publisher} \hyperref[TEI.q]{q} \hyperref[TEI.quote]{quote} \hyperref[TEI.ref]{ref} \hyperref[TEI.reg]{reg} \hyperref[TEI.rs]{rs} \hyperref[TEI.said]{said} \hyperref[TEI.sic]{sic} \hyperref[TEI.soCalled]{soCalled} \hyperref[TEI.speaker]{speaker} \hyperref[TEI.stage]{stage} \hyperref[TEI.street]{street} \hyperref[TEI.term]{term} \hyperref[TEI.textLang]{textLang} \hyperref[TEI.time]{time} \hyperref[TEI.title]{title} \hyperref[TEI.unclear]{unclear}\par 
    \item[figures: ]
   \hyperref[TEI.cell]{cell}\par 
    \item[header: ]
   \hyperref[TEI.change]{change} \hyperref[TEI.distributor]{distributor} \hyperref[TEI.edition]{edition} \hyperref[TEI.extent]{extent} \hyperref[TEI.licence]{licence}\par 
    \item[linking: ]
   \hyperref[TEI.ab]{ab} \hyperref[TEI.seg]{seg}\par 
    \item[msdescription: ]
   \hyperref[TEI.accMat]{accMat} \hyperref[TEI.acquisition]{acquisition} \hyperref[TEI.additions]{additions} \hyperref[TEI.catchwords]{catchwords} \hyperref[TEI.collation]{collation} \hyperref[TEI.colophon]{colophon} \hyperref[TEI.condition]{condition} \hyperref[TEI.custEvent]{custEvent} \hyperref[TEI.decoNote]{decoNote} \hyperref[TEI.explicit]{explicit} \hyperref[TEI.filiation]{filiation} \hyperref[TEI.finalRubric]{finalRubric} \hyperref[TEI.foliation]{foliation} \hyperref[TEI.heraldry]{heraldry} \hyperref[TEI.incipit]{incipit} \hyperref[TEI.layout]{layout} \hyperref[TEI.material]{material} \hyperref[TEI.musicNotation]{musicNotation} \hyperref[TEI.objectType]{objectType} \hyperref[TEI.origDate]{origDate} \hyperref[TEI.origPlace]{origPlace} \hyperref[TEI.origin]{origin} \hyperref[TEI.provenance]{provenance} \hyperref[TEI.rubric]{rubric} \hyperref[TEI.secFol]{secFol} \hyperref[TEI.signatures]{signatures} \hyperref[TEI.source]{source} \hyperref[TEI.stamp]{stamp} \hyperref[TEI.summary]{summary} \hyperref[TEI.support]{support} \hyperref[TEI.surrogates]{surrogates} \hyperref[TEI.typeNote]{typeNote} \hyperref[TEI.watermark]{watermark}\par 
    \item[namesdates: ]
   \hyperref[TEI.addName]{addName} \hyperref[TEI.affiliation]{affiliation} \hyperref[TEI.country]{country} \hyperref[TEI.forename]{forename} \hyperref[TEI.genName]{genName} \hyperref[TEI.geogName]{geogName} \hyperref[TEI.nameLink]{nameLink} \hyperref[TEI.orgName]{orgName} \hyperref[TEI.persName]{persName} \hyperref[TEI.placeName]{placeName} \hyperref[TEI.region]{region} \hyperref[TEI.roleName]{roleName} \hyperref[TEI.settlement]{settlement} \hyperref[TEI.surname]{surname}\par 
    \item[textstructure: ]
   \hyperref[TEI.docAuthor]{docAuthor} \hyperref[TEI.docDate]{docDate} \hyperref[TEI.docEdition]{docEdition} \hyperref[TEI.titlePart]{titlePart}\par 
    \item[transcr: ]
   \hyperref[TEI.damage]{damage} \hyperref[TEI.fw]{fw} \hyperref[TEI.line]{line} \hyperref[TEI.metamark]{metamark} \hyperref[TEI.mod]{mod} \hyperref[TEI.restore]{restore} \hyperref[TEI.retrace]{retrace} \hyperref[TEI.secl]{secl} \hyperref[TEI.supplied]{supplied} \hyperref[TEI.surplus]{surplus} \hyperref[TEI.zone]{zone}
    \item[{Peut contenir}]
  Des données textuelles uniquement
    \item[{Note}]
  \par
Contient un seul caractère, un élément \texttt{<g>} ou une suite de graphèmes à traiter comme un seul caractère. L'attribut {\itshape type} est utilisé pour indiquer la fonction de cette segmentation, avec des valeurs telles que letter, punctuation, ou digit, etc.
    \item[{Exemple}]
  \leavevmode\bgroup\exampleFont \begin{shaded}\noindent\mbox{}{<\textbf{c}\hspace*{6pt}{type}="{punctuation}">}?{</\textbf{c}>}\end{shaded}\egroup 


    \item[{Modèle de contenu}]
  \fbox{\ttfamily <content>\newline
 <macroRef key="macro.xtext"/>\newline
</content>\newline
    } 
    \item[{Schéma Declaration}]
  \mbox{}\hfill\\[-10pt]\begin{Verbatim}[fontsize=\small]
element c
{
   tei_att.global.attributes,
   tei_att.segLike.attributes,
   tei_att.typed.attributes,
   tei_macro.xtext}
\end{Verbatim}

\end{reflist}  \index{catchwords=<catchwords>|oddindex}
\begin{reflist}
\item[]\begin{specHead}{TEI.catchwords}{<catchwords> }(réclames) décrit le système utilisé pour s'assurer que les cahiers formant un manuscrit ou un incunable sont dans le bon ordre, typiquement au moyen d'annotations en bas de page. [\xref{http://www.tei-c.org/release/doc/tei-p5-doc/en/html/MS.html\#msmisc}{10.3.7. Catchwords, Signatures, Secundo Folio}]\end{specHead} 
    \item[{Module}]
  msdescription
    \item[{Attributs}]
  Attributs \hyperref[TEI.att.global]{att.global} (\textit{@xml:id}, \textit{@n}, \textit{@xml:lang}, \textit{@xml:base}, \textit{@xml:space})  (\hyperref[TEI.att.global.rendition]{att.global.rendition} (\textit{@rend}, \textit{@style}, \textit{@rendition})) (\hyperref[TEI.att.global.linking]{att.global.linking} (\textit{@corresp}, \textit{@synch}, \textit{@sameAs}, \textit{@copyOf}, \textit{@next}, \textit{@prev}, \textit{@exclude}, \textit{@select})) (\hyperref[TEI.att.global.analytic]{att.global.analytic} (\textit{@ana})) (\hyperref[TEI.att.global.facs]{att.global.facs} (\textit{@facs})) (\hyperref[TEI.att.global.change]{att.global.change} (\textit{@change})) (\hyperref[TEI.att.global.responsibility]{att.global.responsibility} (\textit{@cert}, \textit{@resp})) (\hyperref[TEI.att.global.source]{att.global.source} (\textit{@source}))
    \item[{Membre du}]
  \hyperref[TEI.model.pPart.msdesc]{model.pPart.msdesc}
    \item[{Contenu dans}]
  
    \item[analysis: ]
   \hyperref[TEI.cl]{cl} \hyperref[TEI.phr]{phr} \hyperref[TEI.s]{s} \hyperref[TEI.span]{span}\par 
    \item[core: ]
   \hyperref[TEI.abbr]{abbr} \hyperref[TEI.add]{add} \hyperref[TEI.addrLine]{addrLine} \hyperref[TEI.author]{author} \hyperref[TEI.biblScope]{biblScope} \hyperref[TEI.citedRange]{citedRange} \hyperref[TEI.corr]{corr} \hyperref[TEI.date]{date} \hyperref[TEI.del]{del} \hyperref[TEI.desc]{desc} \hyperref[TEI.distinct]{distinct} \hyperref[TEI.editor]{editor} \hyperref[TEI.email]{email} \hyperref[TEI.emph]{emph} \hyperref[TEI.expan]{expan} \hyperref[TEI.foreign]{foreign} \hyperref[TEI.gloss]{gloss} \hyperref[TEI.head]{head} \hyperref[TEI.headItem]{headItem} \hyperref[TEI.headLabel]{headLabel} \hyperref[TEI.hi]{hi} \hyperref[TEI.item]{item} \hyperref[TEI.l]{l} \hyperref[TEI.label]{label} \hyperref[TEI.measure]{measure} \hyperref[TEI.meeting]{meeting} \hyperref[TEI.mentioned]{mentioned} \hyperref[TEI.name]{name} \hyperref[TEI.note]{note} \hyperref[TEI.num]{num} \hyperref[TEI.orig]{orig} \hyperref[TEI.p]{p} \hyperref[TEI.pubPlace]{pubPlace} \hyperref[TEI.publisher]{publisher} \hyperref[TEI.q]{q} \hyperref[TEI.quote]{quote} \hyperref[TEI.ref]{ref} \hyperref[TEI.reg]{reg} \hyperref[TEI.resp]{resp} \hyperref[TEI.rs]{rs} \hyperref[TEI.said]{said} \hyperref[TEI.sic]{sic} \hyperref[TEI.soCalled]{soCalled} \hyperref[TEI.speaker]{speaker} \hyperref[TEI.stage]{stage} \hyperref[TEI.street]{street} \hyperref[TEI.term]{term} \hyperref[TEI.textLang]{textLang} \hyperref[TEI.time]{time} \hyperref[TEI.title]{title} \hyperref[TEI.unclear]{unclear}\par 
    \item[figures: ]
   \hyperref[TEI.cell]{cell} \hyperref[TEI.figDesc]{figDesc}\par 
    \item[header: ]
   \hyperref[TEI.authority]{authority} \hyperref[TEI.change]{change} \hyperref[TEI.classCode]{classCode} \hyperref[TEI.creation]{creation} \hyperref[TEI.distributor]{distributor} \hyperref[TEI.edition]{edition} \hyperref[TEI.extent]{extent} \hyperref[TEI.funder]{funder} \hyperref[TEI.language]{language} \hyperref[TEI.licence]{licence} \hyperref[TEI.rendition]{rendition}\par 
    \item[iso-fs: ]
   \hyperref[TEI.fDescr]{fDescr} \hyperref[TEI.fsDescr]{fsDescr}\par 
    \item[linking: ]
   \hyperref[TEI.ab]{ab} \hyperref[TEI.seg]{seg}\par 
    \item[msdescription: ]
   \hyperref[TEI.accMat]{accMat} \hyperref[TEI.acquisition]{acquisition} \hyperref[TEI.additions]{additions} \hyperref[TEI.catchwords]{catchwords} \hyperref[TEI.collation]{collation} \hyperref[TEI.colophon]{colophon} \hyperref[TEI.condition]{condition} \hyperref[TEI.custEvent]{custEvent} \hyperref[TEI.decoNote]{decoNote} \hyperref[TEI.explicit]{explicit} \hyperref[TEI.filiation]{filiation} \hyperref[TEI.finalRubric]{finalRubric} \hyperref[TEI.foliation]{foliation} \hyperref[TEI.heraldry]{heraldry} \hyperref[TEI.incipit]{incipit} \hyperref[TEI.layout]{layout} \hyperref[TEI.material]{material} \hyperref[TEI.musicNotation]{musicNotation} \hyperref[TEI.objectType]{objectType} \hyperref[TEI.origDate]{origDate} \hyperref[TEI.origPlace]{origPlace} \hyperref[TEI.origin]{origin} \hyperref[TEI.provenance]{provenance} \hyperref[TEI.rubric]{rubric} \hyperref[TEI.secFol]{secFol} \hyperref[TEI.signatures]{signatures} \hyperref[TEI.source]{source} \hyperref[TEI.stamp]{stamp} \hyperref[TEI.summary]{summary} \hyperref[TEI.support]{support} \hyperref[TEI.surrogates]{surrogates} \hyperref[TEI.typeNote]{typeNote} \hyperref[TEI.watermark]{watermark}\par 
    \item[namesdates: ]
   \hyperref[TEI.addName]{addName} \hyperref[TEI.affiliation]{affiliation} \hyperref[TEI.country]{country} \hyperref[TEI.forename]{forename} \hyperref[TEI.genName]{genName} \hyperref[TEI.geogName]{geogName} \hyperref[TEI.nameLink]{nameLink} \hyperref[TEI.orgName]{orgName} \hyperref[TEI.persName]{persName} \hyperref[TEI.placeName]{placeName} \hyperref[TEI.region]{region} \hyperref[TEI.roleName]{roleName} \hyperref[TEI.settlement]{settlement} \hyperref[TEI.surname]{surname}\par 
    \item[textstructure: ]
   \hyperref[TEI.docAuthor]{docAuthor} \hyperref[TEI.docDate]{docDate} \hyperref[TEI.docEdition]{docEdition} \hyperref[TEI.titlePart]{titlePart}\par 
    \item[transcr: ]
   \hyperref[TEI.damage]{damage} \hyperref[TEI.fw]{fw} \hyperref[TEI.metamark]{metamark} \hyperref[TEI.mod]{mod} \hyperref[TEI.restore]{restore} \hyperref[TEI.retrace]{retrace} \hyperref[TEI.secl]{secl} \hyperref[TEI.supplied]{supplied} \hyperref[TEI.surplus]{surplus}
    \item[{Peut contenir}]
  
    \item[analysis: ]
   \hyperref[TEI.c]{c} \hyperref[TEI.cl]{cl} \hyperref[TEI.interp]{interp} \hyperref[TEI.interpGrp]{interpGrp} \hyperref[TEI.m]{m} \hyperref[TEI.pc]{pc} \hyperref[TEI.phr]{phr} \hyperref[TEI.s]{s} \hyperref[TEI.span]{span} \hyperref[TEI.spanGrp]{spanGrp} \hyperref[TEI.w]{w}\par 
    \item[core: ]
   \hyperref[TEI.abbr]{abbr} \hyperref[TEI.add]{add} \hyperref[TEI.address]{address} \hyperref[TEI.binaryObject]{binaryObject} \hyperref[TEI.cb]{cb} \hyperref[TEI.choice]{choice} \hyperref[TEI.corr]{corr} \hyperref[TEI.date]{date} \hyperref[TEI.del]{del} \hyperref[TEI.distinct]{distinct} \hyperref[TEI.email]{email} \hyperref[TEI.emph]{emph} \hyperref[TEI.expan]{expan} \hyperref[TEI.foreign]{foreign} \hyperref[TEI.gap]{gap} \hyperref[TEI.gb]{gb} \hyperref[TEI.gloss]{gloss} \hyperref[TEI.graphic]{graphic} \hyperref[TEI.hi]{hi} \hyperref[TEI.index]{index} \hyperref[TEI.lb]{lb} \hyperref[TEI.measure]{measure} \hyperref[TEI.measureGrp]{measureGrp} \hyperref[TEI.media]{media} \hyperref[TEI.mentioned]{mentioned} \hyperref[TEI.milestone]{milestone} \hyperref[TEI.name]{name} \hyperref[TEI.note]{note} \hyperref[TEI.num]{num} \hyperref[TEI.orig]{orig} \hyperref[TEI.pb]{pb} \hyperref[TEI.ptr]{ptr} \hyperref[TEI.ref]{ref} \hyperref[TEI.reg]{reg} \hyperref[TEI.rs]{rs} \hyperref[TEI.sic]{sic} \hyperref[TEI.soCalled]{soCalled} \hyperref[TEI.term]{term} \hyperref[TEI.time]{time} \hyperref[TEI.title]{title} \hyperref[TEI.unclear]{unclear}\par 
    \item[derived-module-tei.istex: ]
   \hyperref[TEI.math]{math} \hyperref[TEI.mrow]{mrow}\par 
    \item[figures: ]
   \hyperref[TEI.figure]{figure} \hyperref[TEI.formula]{formula} \hyperref[TEI.notatedMusic]{notatedMusic}\par 
    \item[header: ]
   \hyperref[TEI.idno]{idno}\par 
    \item[iso-fs: ]
   \hyperref[TEI.fLib]{fLib} \hyperref[TEI.fs]{fs} \hyperref[TEI.fvLib]{fvLib}\par 
    \item[linking: ]
   \hyperref[TEI.alt]{alt} \hyperref[TEI.altGrp]{altGrp} \hyperref[TEI.anchor]{anchor} \hyperref[TEI.join]{join} \hyperref[TEI.joinGrp]{joinGrp} \hyperref[TEI.link]{link} \hyperref[TEI.linkGrp]{linkGrp} \hyperref[TEI.seg]{seg} \hyperref[TEI.timeline]{timeline}\par 
    \item[msdescription: ]
   \hyperref[TEI.catchwords]{catchwords} \hyperref[TEI.depth]{depth} \hyperref[TEI.dim]{dim} \hyperref[TEI.dimensions]{dimensions} \hyperref[TEI.height]{height} \hyperref[TEI.heraldry]{heraldry} \hyperref[TEI.locus]{locus} \hyperref[TEI.locusGrp]{locusGrp} \hyperref[TEI.material]{material} \hyperref[TEI.objectType]{objectType} \hyperref[TEI.origDate]{origDate} \hyperref[TEI.origPlace]{origPlace} \hyperref[TEI.secFol]{secFol} \hyperref[TEI.signatures]{signatures} \hyperref[TEI.source]{source} \hyperref[TEI.stamp]{stamp} \hyperref[TEI.watermark]{watermark} \hyperref[TEI.width]{width}\par 
    \item[namesdates: ]
   \hyperref[TEI.addName]{addName} \hyperref[TEI.affiliation]{affiliation} \hyperref[TEI.country]{country} \hyperref[TEI.forename]{forename} \hyperref[TEI.genName]{genName} \hyperref[TEI.geogName]{geogName} \hyperref[TEI.location]{location} \hyperref[TEI.nameLink]{nameLink} \hyperref[TEI.orgName]{orgName} \hyperref[TEI.persName]{persName} \hyperref[TEI.placeName]{placeName} \hyperref[TEI.region]{region} \hyperref[TEI.roleName]{roleName} \hyperref[TEI.settlement]{settlement} \hyperref[TEI.state]{state} \hyperref[TEI.surname]{surname}\par 
    \item[spoken: ]
   \hyperref[TEI.annotationBlock]{annotationBlock}\par 
    \item[transcr: ]
   \hyperref[TEI.addSpan]{addSpan} \hyperref[TEI.am]{am} \hyperref[TEI.damage]{damage} \hyperref[TEI.damageSpan]{damageSpan} \hyperref[TEI.delSpan]{delSpan} \hyperref[TEI.ex]{ex} \hyperref[TEI.fw]{fw} \hyperref[TEI.handShift]{handShift} \hyperref[TEI.listTranspose]{listTranspose} \hyperref[TEI.metamark]{metamark} \hyperref[TEI.mod]{mod} \hyperref[TEI.redo]{redo} \hyperref[TEI.restore]{restore} \hyperref[TEI.retrace]{retrace} \hyperref[TEI.secl]{secl} \hyperref[TEI.space]{space} \hyperref[TEI.subst]{subst} \hyperref[TEI.substJoin]{substJoin} \hyperref[TEI.supplied]{supplied} \hyperref[TEI.surplus]{surplus} \hyperref[TEI.undo]{undo}\par des données textuelles
    \item[{Exemple}]
  \leavevmode\bgroup\exampleFont \begin{shaded}\noindent\mbox{}{<\textbf{catchwords}>}Vertical catchwords in the hand of the scribe placed along\mbox{}\newline 
 the inner bounding line, reading from top to bottom.{</\textbf{catchwords}>}\end{shaded}\egroup 


    \item[{Schematron}]
   <sch:assert role="nonfatal"  test="ancestor::tei:msDesc">WARNING: deprecated use of element — The <sch:name/> element will not be allowed outside of msDesc as of 2018-10-01.</sch:assert>
    \item[{Modèle de contenu}]
  \mbox{}\hfill\\[-10pt]\begin{Verbatim}[fontsize=\small]
<content>
 <macroRef key="macro.phraseSeq"/>
</content>
    
\end{Verbatim}

    \item[{Schéma Declaration}]
  \mbox{}\hfill\\[-10pt]\begin{Verbatim}[fontsize=\small]
element catchwords { tei_att.global.attributes, tei_macro.phraseSeq }
\end{Verbatim}

\end{reflist}  \index{category=<category>|oddindex}
\begin{reflist}
\item[]\begin{specHead}{TEI.category}{<category> }(catégorie) contient une catégorie descriptive particulière, éventuellement intégrée dans une catégorie de niveau supérieur, à l’intérieur d’une taxinomie définie par l’utilisateur. [\xref{http://www.tei-c.org/release/doc/tei-p5-doc/en/html/HD.html\#HD55}{2.3.7. The Classification Declaration}]\end{specHead} 
    \item[{Module}]
  header
    \item[{Attributs}]
  Attributs \hyperref[TEI.att.global]{att.global} (\textit{@xml:id}, \textit{@n}, \textit{@xml:lang}, \textit{@xml:base}, \textit{@xml:space})  (\hyperref[TEI.att.global.rendition]{att.global.rendition} (\textit{@rend}, \textit{@style}, \textit{@rendition})) (\hyperref[TEI.att.global.linking]{att.global.linking} (\textit{@corresp}, \textit{@synch}, \textit{@sameAs}, \textit{@copyOf}, \textit{@next}, \textit{@prev}, \textit{@exclude}, \textit{@select})) (\hyperref[TEI.att.global.analytic]{att.global.analytic} (\textit{@ana})) (\hyperref[TEI.att.global.facs]{att.global.facs} (\textit{@facs})) (\hyperref[TEI.att.global.change]{att.global.change} (\textit{@change})) (\hyperref[TEI.att.global.responsibility]{att.global.responsibility} (\textit{@cert}, \textit{@resp})) (\hyperref[TEI.att.global.source]{att.global.source} (\textit{@source}))
    \item[{Contenu dans}]
  
    \item[header: ]
   \hyperref[TEI.category]{category} \hyperref[TEI.taxonomy]{taxonomy}
    \item[{Peut contenir}]
  
    \item[core: ]
   \hyperref[TEI.desc]{desc} \hyperref[TEI.gloss]{gloss}\par 
    \item[header: ]
   \hyperref[TEI.category]{category}
    \item[{Exemple}]
  \leavevmode\bgroup\exampleFont \begin{shaded}\noindent\mbox{}{<\textbf{category}\hspace*{6pt}{xml:id}="{fr\textunderscore tax.a.d2}">}\mbox{}\newline 
\hspace*{6pt}{<\textbf{catDesc}>}Récits de voyage{</\textbf{catDesc}>}\mbox{}\newline 
{</\textbf{category}>}\mbox{}\newline 
{<\textbf{bibl}>}indexation selon le système d'indexation RAMEAU, géré par la Bibliothèque nationale de\mbox{}\newline 
 France{</\textbf{bibl}>}\end{shaded}\egroup 


    \item[{Exemple}]
  \leavevmode\bgroup\exampleFont \begin{shaded}\noindent\mbox{}{<\textbf{category}\hspace*{6pt}{xml:id}="{fr\textunderscore b1}">}\mbox{}\newline 
\hspace*{6pt}{<\textbf{catDesc}>}Devinettes et énigmes {</\textbf{catDesc}>}\mbox{}\newline 
\hspace*{6pt}{<\textbf{category}\hspace*{6pt}{xml:id}="{fr\textunderscore b11}">}\mbox{}\newline 
\hspace*{6pt}\hspace*{6pt}{<\textbf{catDesc}>}Anagrammes {</\textbf{catDesc}>}\mbox{}\newline 
\hspace*{6pt}{</\textbf{category}>}\mbox{}\newline 
{</\textbf{category}>}\end{shaded}\egroup 


    \item[{Exemple}]
  \leavevmode\bgroup\exampleFont \begin{shaded}\noindent\mbox{}{<\textbf{category}\hspace*{6pt}{xml:id}="{LIT}">}\mbox{}\newline 
\hspace*{6pt}{<\textbf{catDesc}\hspace*{6pt}{xml:lang}="{pl}">}literatura piękna{</\textbf{catDesc}>}\mbox{}\newline 
\hspace*{6pt}{<\textbf{catDesc}\hspace*{6pt}{xml:lang}="{en}">}fiction{</\textbf{catDesc}>}\mbox{}\newline 
\hspace*{6pt}{<\textbf{category}\hspace*{6pt}{xml:id}="{LPROSE}">}\mbox{}\newline 
\hspace*{6pt}\hspace*{6pt}{<\textbf{catDesc}\hspace*{6pt}{xml:lang}="{pl}">}proza{</\textbf{catDesc}>}\mbox{}\newline 
\hspace*{6pt}\hspace*{6pt}{<\textbf{catDesc}\hspace*{6pt}{xml:lang}="{en}">}prose{</\textbf{catDesc}>}\mbox{}\newline 
\hspace*{6pt}{</\textbf{category}>}\mbox{}\newline 
\hspace*{6pt}{<\textbf{category}\hspace*{6pt}{xml:id}="{LPOETRY}">}\mbox{}\newline 
\hspace*{6pt}\hspace*{6pt}{<\textbf{catDesc}\hspace*{6pt}{xml:lang}="{pl}">}poezja{</\textbf{catDesc}>}\mbox{}\newline 
\hspace*{6pt}\hspace*{6pt}{<\textbf{catDesc}\hspace*{6pt}{xml:lang}="{en}">}poetry{</\textbf{catDesc}>}\mbox{}\newline 
\hspace*{6pt}{</\textbf{category}>}\mbox{}\newline 
\hspace*{6pt}{<\textbf{category}\hspace*{6pt}{xml:id}="{LDRAMA}">}\mbox{}\newline 
\hspace*{6pt}\hspace*{6pt}{<\textbf{catDesc}\hspace*{6pt}{xml:lang}="{pl}">}dramat{</\textbf{catDesc}>}\mbox{}\newline 
\hspace*{6pt}\hspace*{6pt}{<\textbf{catDesc}\hspace*{6pt}{xml:lang}="{en}">}drama{</\textbf{catDesc}>}\mbox{}\newline 
\hspace*{6pt}{</\textbf{category}>}\mbox{}\newline 
{</\textbf{category}>}\end{shaded}\egroup 


    \item[{Modèle de contenu}]
  \mbox{}\hfill\\[-10pt]\begin{Verbatim}[fontsize=\small]
<content>
 <sequence maxOccurs="1" minOccurs="1">
  <alternate maxOccurs="1" minOccurs="1">
   <elementRef key="catDesc"
    maxOccurs="unbounded" minOccurs="1"/>
   <alternate maxOccurs="unbounded"
    minOccurs="0">
    <classRef key="model.descLike"/>
    <classRef key="model.glossLike"/>
   </alternate>
  </alternate>
  <elementRef key="category"
   maxOccurs="unbounded" minOccurs="0"/>
 </sequence>
</content>
    
\end{Verbatim}

    \item[{Schéma Declaration}]
  \mbox{}\hfill\\[-10pt]\begin{Verbatim}[fontsize=\small]
element category
{
   tei_att.global.attributes,
   (
      ( catDesc+ | ( tei_model.descLike | tei_model.glossLike )* ),
      tei_category*
   )
}
\end{Verbatim}

\end{reflist}  \index{cb=<cb>|oddindex}
\begin{reflist}
\item[]\begin{specHead}{TEI.cb}{<cb> }(saut de colonne) marque le début d'une nouvelle colonne de texte sur une page multi-colonne. [\xref{http://www.tei-c.org/release/doc/tei-p5-doc/en/html/CO.html\#CORS5}{3.10.3. Milestone Elements}]\end{specHead} 
    \item[{Module}]
  core
    \item[{Attributs}]
  Attributs \hyperref[TEI.att.global]{att.global} (\textit{@xml:id}, \textit{@n}, \textit{@xml:lang}, \textit{@xml:base}, \textit{@xml:space})  (\hyperref[TEI.att.global.rendition]{att.global.rendition} (\textit{@rend}, \textit{@style}, \textit{@rendition})) (\hyperref[TEI.att.global.linking]{att.global.linking} (\textit{@corresp}, \textit{@synch}, \textit{@sameAs}, \textit{@copyOf}, \textit{@next}, \textit{@prev}, \textit{@exclude}, \textit{@select})) (\hyperref[TEI.att.global.analytic]{att.global.analytic} (\textit{@ana})) (\hyperref[TEI.att.global.facs]{att.global.facs} (\textit{@facs})) (\hyperref[TEI.att.global.change]{att.global.change} (\textit{@change})) (\hyperref[TEI.att.global.responsibility]{att.global.responsibility} (\textit{@cert}, \textit{@resp})) (\hyperref[TEI.att.global.source]{att.global.source} (\textit{@source})) \hyperref[TEI.att.typed]{att.typed} (\textit{@type}, \textit{@subtype}) \hyperref[TEI.att.edition]{att.edition} (\textit{@ed}, \textit{@edRef}) \hyperref[TEI.att.spanning]{att.spanning} (\textit{@spanTo}) \hyperref[TEI.att.breaking]{att.breaking} (\textit{@break}) 
    \item[{Membre du}]
  \hyperref[TEI.model.milestoneLike]{model.milestoneLike}
    \item[{Contenu dans}]
  
    \item[analysis: ]
   \hyperref[TEI.cl]{cl} \hyperref[TEI.m]{m} \hyperref[TEI.phr]{phr} \hyperref[TEI.s]{s} \hyperref[TEI.span]{span} \hyperref[TEI.w]{w}\par 
    \item[core: ]
   \hyperref[TEI.abbr]{abbr} \hyperref[TEI.add]{add} \hyperref[TEI.addrLine]{addrLine} \hyperref[TEI.address]{address} \hyperref[TEI.author]{author} \hyperref[TEI.bibl]{bibl} \hyperref[TEI.biblScope]{biblScope} \hyperref[TEI.cit]{cit} \hyperref[TEI.citedRange]{citedRange} \hyperref[TEI.corr]{corr} \hyperref[TEI.date]{date} \hyperref[TEI.del]{del} \hyperref[TEI.distinct]{distinct} \hyperref[TEI.editor]{editor} \hyperref[TEI.email]{email} \hyperref[TEI.emph]{emph} \hyperref[TEI.expan]{expan} \hyperref[TEI.foreign]{foreign} \hyperref[TEI.gloss]{gloss} \hyperref[TEI.head]{head} \hyperref[TEI.headItem]{headItem} \hyperref[TEI.headLabel]{headLabel} \hyperref[TEI.hi]{hi} \hyperref[TEI.imprint]{imprint} \hyperref[TEI.item]{item} \hyperref[TEI.l]{l} \hyperref[TEI.label]{label} \hyperref[TEI.lg]{lg} \hyperref[TEI.list]{list} \hyperref[TEI.listBibl]{listBibl} \hyperref[TEI.measure]{measure} \hyperref[TEI.mentioned]{mentioned} \hyperref[TEI.name]{name} \hyperref[TEI.note]{note} \hyperref[TEI.num]{num} \hyperref[TEI.orig]{orig} \hyperref[TEI.p]{p} \hyperref[TEI.pubPlace]{pubPlace} \hyperref[TEI.publisher]{publisher} \hyperref[TEI.q]{q} \hyperref[TEI.quote]{quote} \hyperref[TEI.ref]{ref} \hyperref[TEI.reg]{reg} \hyperref[TEI.resp]{resp} \hyperref[TEI.rs]{rs} \hyperref[TEI.said]{said} \hyperref[TEI.series]{series} \hyperref[TEI.sic]{sic} \hyperref[TEI.soCalled]{soCalled} \hyperref[TEI.sp]{sp} \hyperref[TEI.speaker]{speaker} \hyperref[TEI.stage]{stage} \hyperref[TEI.street]{street} \hyperref[TEI.term]{term} \hyperref[TEI.textLang]{textLang} \hyperref[TEI.time]{time} \hyperref[TEI.title]{title} \hyperref[TEI.unclear]{unclear}\par 
    \item[figures: ]
   \hyperref[TEI.cell]{cell} \hyperref[TEI.figure]{figure} \hyperref[TEI.table]{table}\par 
    \item[header: ]
   \hyperref[TEI.authority]{authority} \hyperref[TEI.change]{change} \hyperref[TEI.classCode]{classCode} \hyperref[TEI.distributor]{distributor} \hyperref[TEI.edition]{edition} \hyperref[TEI.extent]{extent} \hyperref[TEI.funder]{funder} \hyperref[TEI.language]{language} \hyperref[TEI.licence]{licence}\par 
    \item[linking: ]
   \hyperref[TEI.ab]{ab} \hyperref[TEI.seg]{seg}\par 
    \item[msdescription: ]
   \hyperref[TEI.accMat]{accMat} \hyperref[TEI.acquisition]{acquisition} \hyperref[TEI.additions]{additions} \hyperref[TEI.catchwords]{catchwords} \hyperref[TEI.collation]{collation} \hyperref[TEI.colophon]{colophon} \hyperref[TEI.condition]{condition} \hyperref[TEI.custEvent]{custEvent} \hyperref[TEI.decoNote]{decoNote} \hyperref[TEI.explicit]{explicit} \hyperref[TEI.filiation]{filiation} \hyperref[TEI.finalRubric]{finalRubric} \hyperref[TEI.foliation]{foliation} \hyperref[TEI.heraldry]{heraldry} \hyperref[TEI.incipit]{incipit} \hyperref[TEI.layout]{layout} \hyperref[TEI.material]{material} \hyperref[TEI.msItem]{msItem} \hyperref[TEI.musicNotation]{musicNotation} \hyperref[TEI.objectType]{objectType} \hyperref[TEI.origDate]{origDate} \hyperref[TEI.origPlace]{origPlace} \hyperref[TEI.origin]{origin} \hyperref[TEI.provenance]{provenance} \hyperref[TEI.rubric]{rubric} \hyperref[TEI.secFol]{secFol} \hyperref[TEI.signatures]{signatures} \hyperref[TEI.source]{source} \hyperref[TEI.stamp]{stamp} \hyperref[TEI.summary]{summary} \hyperref[TEI.support]{support} \hyperref[TEI.surrogates]{surrogates} \hyperref[TEI.typeNote]{typeNote} \hyperref[TEI.watermark]{watermark}\par 
    \item[namesdates: ]
   \hyperref[TEI.addName]{addName} \hyperref[TEI.affiliation]{affiliation} \hyperref[TEI.country]{country} \hyperref[TEI.forename]{forename} \hyperref[TEI.genName]{genName} \hyperref[TEI.geogName]{geogName} \hyperref[TEI.nameLink]{nameLink} \hyperref[TEI.org]{org} \hyperref[TEI.orgName]{orgName} \hyperref[TEI.persName]{persName} \hyperref[TEI.person]{person} \hyperref[TEI.personGrp]{personGrp} \hyperref[TEI.persona]{persona} \hyperref[TEI.placeName]{placeName} \hyperref[TEI.region]{region} \hyperref[TEI.roleName]{roleName} \hyperref[TEI.settlement]{settlement} \hyperref[TEI.surname]{surname}\par 
    \item[textstructure: ]
   \hyperref[TEI.back]{back} \hyperref[TEI.body]{body} \hyperref[TEI.div]{div} \hyperref[TEI.docAuthor]{docAuthor} \hyperref[TEI.docDate]{docDate} \hyperref[TEI.docEdition]{docEdition} \hyperref[TEI.docTitle]{docTitle} \hyperref[TEI.floatingText]{floatingText} \hyperref[TEI.front]{front} \hyperref[TEI.group]{group} \hyperref[TEI.text]{text} \hyperref[TEI.titlePage]{titlePage} \hyperref[TEI.titlePart]{titlePart}\par 
    \item[transcr: ]
   \hyperref[TEI.damage]{damage} \hyperref[TEI.fw]{fw} \hyperref[TEI.line]{line} \hyperref[TEI.metamark]{metamark} \hyperref[TEI.mod]{mod} \hyperref[TEI.restore]{restore} \hyperref[TEI.retrace]{retrace} \hyperref[TEI.secl]{secl} \hyperref[TEI.sourceDoc]{sourceDoc} \hyperref[TEI.subst]{subst} \hyperref[TEI.supplied]{supplied} \hyperref[TEI.surface]{surface} \hyperref[TEI.surfaceGrp]{surfaceGrp} \hyperref[TEI.surplus]{surplus} \hyperref[TEI.zone]{zone}
    \item[{Peut contenir}]
  Elément vide
    \item[{Note}]
  \par
L'attribut global {\itshape n} donne un nouveau numéro ou une autre valeur à la colonne qui suit l'élément \hyperref[TEI.cb]{<cb>}. Les encodeurs doivent faire un choix clair, et s'y tenir, entre l'option consistant à se fonder sur la séquence physique des colonnes dans le texte entier, et celle qui consiste à se fonder sur la numérotation des colonnes à l'intérieur de la page. L'élément \hyperref[TEI.cb]{<cb>} apparaît en haut de la colonne à laquelle il se rapporte.
    \item[{Exemple}]
  Markup of an early English dictionary printed in two columns:\leavevmode\bgroup\exampleFont \begin{shaded}\noindent\mbox{}{<\textbf{pb}/>}\mbox{}\newline 
{<\textbf{cb}\hspace*{6pt}{n}="{1}"/>}\mbox{}\newline 
{<\textbf{entryFree}>}\mbox{}\newline 
\hspace*{6pt}{<\textbf{form}>}Well{</\textbf{form}>}, {<\textbf{sense}>}a Pit to hold Spring-Water{</\textbf{sense}>}:\mbox{}\newline 
{<\textbf{sense}>}In the Art of {<\textbf{hi}\hspace*{6pt}{rend}="{italic}">}War{</\textbf{hi}>}, a Depth the Miner\mbox{}\newline 
\hspace*{6pt}\hspace*{6pt} sinks into the Ground, to find out and disappoint the Enemies Mines,\mbox{}\newline 
\hspace*{6pt}\hspace*{6pt} or to prepare one{</\textbf{sense}>}.\mbox{}\newline 
{</\textbf{entryFree}>}\mbox{}\newline 
{<\textbf{entryFree}>}To {<\textbf{form}>}Welter{</\textbf{form}>}, {<\textbf{sense}>}to wallow{</\textbf{sense}>}, or\mbox{}\newline 
{<\textbf{sense}>}lie groveling{</\textbf{sense}>}.{</\textbf{entryFree}>}\mbox{}\newline 
\textit{<!-- remainder of column -->}\mbox{}\newline 
{<\textbf{cb}\hspace*{6pt}{n}="{2}"/>}\mbox{}\newline 
{<\textbf{entryFree}>}\mbox{}\newline 
\hspace*{6pt}{<\textbf{form}>}Wey{</\textbf{form}>}, {<\textbf{sense}>}the greatest Measure for dry Things,\mbox{}\newline 
\hspace*{6pt}\hspace*{6pt} containing five Chaldron{</\textbf{sense}>}.\mbox{}\newline 
{</\textbf{entryFree}>}\mbox{}\newline 
{<\textbf{entryFree}>}\mbox{}\newline 
\hspace*{6pt}{<\textbf{form}>}Whale{</\textbf{form}>}, {<\textbf{sense}>}the greatest of\mbox{}\newline 
\hspace*{6pt}\hspace*{6pt} Sea-Fishes{</\textbf{sense}>}.\mbox{}\newline 
{</\textbf{entryFree}>}\end{shaded}\egroup 


    \item[{Modèle de contenu}]
  \fbox{\ttfamily <content>\newline
</content>\newline
    } 
    \item[{Schéma Declaration}]
  \mbox{}\hfill\\[-10pt]\begin{Verbatim}[fontsize=\small]
element cb
{
   tei_att.global.attributes,
   tei_att.typed.attributes,
   tei_att.edition.attributes,
   tei_att.spanning.attributes,
   tei_att.breaking.attributes,
   empty
}
\end{Verbatim}

\end{reflist}  \index{cell=<cell>|oddindex}
\begin{reflist}
\item[]\begin{specHead}{TEI.cell}{<cell> }(cellule) contient une cellule d'un tableau. [\xref{http://www.tei-c.org/release/doc/tei-p5-doc/en/html/FT.html\#FTTAB1}{14.1.1. TEI Tables}]\end{specHead} 
    \item[{Module}]
  figures
    \item[{Attributs}]
  Attributs \hyperref[TEI.att.global]{att.global} (\textit{@xml:id}, \textit{@n}, \textit{@xml:lang}, \textit{@xml:base}, \textit{@xml:space})  (\hyperref[TEI.att.global.rendition]{att.global.rendition} (\textit{@rend}, \textit{@style}, \textit{@rendition})) (\hyperref[TEI.att.global.linking]{att.global.linking} (\textit{@corresp}, \textit{@synch}, \textit{@sameAs}, \textit{@copyOf}, \textit{@next}, \textit{@prev}, \textit{@exclude}, \textit{@select})) (\hyperref[TEI.att.global.analytic]{att.global.analytic} (\textit{@ana})) (\hyperref[TEI.att.global.facs]{att.global.facs} (\textit{@facs})) (\hyperref[TEI.att.global.change]{att.global.change} (\textit{@change})) (\hyperref[TEI.att.global.responsibility]{att.global.responsibility} (\textit{@cert}, \textit{@resp})) (\hyperref[TEI.att.global.source]{att.global.source} (\textit{@source})) \hyperref[TEI.att.tableDecoration]{att.tableDecoration} (\textit{@role}, \textit{@rows}, \textit{@cols}) 
    \item[{Contenu dans}]
  
    \item[figures: ]
   \hyperref[TEI.row]{row}
    \item[{Peut contenir}]
  
    \item[analysis: ]
   \hyperref[TEI.c]{c} \hyperref[TEI.cl]{cl} \hyperref[TEI.interp]{interp} \hyperref[TEI.interpGrp]{interpGrp} \hyperref[TEI.m]{m} \hyperref[TEI.pc]{pc} \hyperref[TEI.phr]{phr} \hyperref[TEI.s]{s} \hyperref[TEI.span]{span} \hyperref[TEI.spanGrp]{spanGrp} \hyperref[TEI.w]{w}\par 
    \item[core: ]
   \hyperref[TEI.abbr]{abbr} \hyperref[TEI.add]{add} \hyperref[TEI.address]{address} \hyperref[TEI.bibl]{bibl} \hyperref[TEI.biblStruct]{biblStruct} \hyperref[TEI.binaryObject]{binaryObject} \hyperref[TEI.cb]{cb} \hyperref[TEI.choice]{choice} \hyperref[TEI.cit]{cit} \hyperref[TEI.corr]{corr} \hyperref[TEI.date]{date} \hyperref[TEI.del]{del} \hyperref[TEI.desc]{desc} \hyperref[TEI.distinct]{distinct} \hyperref[TEI.email]{email} \hyperref[TEI.emph]{emph} \hyperref[TEI.expan]{expan} \hyperref[TEI.foreign]{foreign} \hyperref[TEI.gap]{gap} \hyperref[TEI.gb]{gb} \hyperref[TEI.gloss]{gloss} \hyperref[TEI.graphic]{graphic} \hyperref[TEI.hi]{hi} \hyperref[TEI.index]{index} \hyperref[TEI.l]{l} \hyperref[TEI.label]{label} \hyperref[TEI.lb]{lb} \hyperref[TEI.lg]{lg} \hyperref[TEI.list]{list} \hyperref[TEI.listBibl]{listBibl} \hyperref[TEI.measure]{measure} \hyperref[TEI.measureGrp]{measureGrp} \hyperref[TEI.media]{media} \hyperref[TEI.mentioned]{mentioned} \hyperref[TEI.milestone]{milestone} \hyperref[TEI.name]{name} \hyperref[TEI.note]{note} \hyperref[TEI.num]{num} \hyperref[TEI.orig]{orig} \hyperref[TEI.p]{p} \hyperref[TEI.pb]{pb} \hyperref[TEI.ptr]{ptr} \hyperref[TEI.q]{q} \hyperref[TEI.quote]{quote} \hyperref[TEI.ref]{ref} \hyperref[TEI.reg]{reg} \hyperref[TEI.rs]{rs} \hyperref[TEI.said]{said} \hyperref[TEI.sic]{sic} \hyperref[TEI.soCalled]{soCalled} \hyperref[TEI.sp]{sp} \hyperref[TEI.stage]{stage} \hyperref[TEI.term]{term} \hyperref[TEI.time]{time} \hyperref[TEI.title]{title} \hyperref[TEI.unclear]{unclear}\par 
    \item[derived-module-tei.istex: ]
   \hyperref[TEI.math]{math} \hyperref[TEI.mrow]{mrow}\par 
    \item[figures: ]
   \hyperref[TEI.figure]{figure} \hyperref[TEI.formula]{formula} \hyperref[TEI.notatedMusic]{notatedMusic} \hyperref[TEI.table]{table}\par 
    \item[header: ]
   \hyperref[TEI.biblFull]{biblFull} \hyperref[TEI.idno]{idno}\par 
    \item[iso-fs: ]
   \hyperref[TEI.fLib]{fLib} \hyperref[TEI.fs]{fs} \hyperref[TEI.fvLib]{fvLib}\par 
    \item[linking: ]
   \hyperref[TEI.ab]{ab} \hyperref[TEI.alt]{alt} \hyperref[TEI.altGrp]{altGrp} \hyperref[TEI.anchor]{anchor} \hyperref[TEI.join]{join} \hyperref[TEI.joinGrp]{joinGrp} \hyperref[TEI.link]{link} \hyperref[TEI.linkGrp]{linkGrp} \hyperref[TEI.seg]{seg} \hyperref[TEI.timeline]{timeline}\par 
    \item[msdescription: ]
   \hyperref[TEI.catchwords]{catchwords} \hyperref[TEI.depth]{depth} \hyperref[TEI.dim]{dim} \hyperref[TEI.dimensions]{dimensions} \hyperref[TEI.height]{height} \hyperref[TEI.heraldry]{heraldry} \hyperref[TEI.locus]{locus} \hyperref[TEI.locusGrp]{locusGrp} \hyperref[TEI.material]{material} \hyperref[TEI.msDesc]{msDesc} \hyperref[TEI.objectType]{objectType} \hyperref[TEI.origDate]{origDate} \hyperref[TEI.origPlace]{origPlace} \hyperref[TEI.secFol]{secFol} \hyperref[TEI.signatures]{signatures} \hyperref[TEI.source]{source} \hyperref[TEI.stamp]{stamp} \hyperref[TEI.watermark]{watermark} \hyperref[TEI.width]{width}\par 
    \item[namesdates: ]
   \hyperref[TEI.addName]{addName} \hyperref[TEI.affiliation]{affiliation} \hyperref[TEI.country]{country} \hyperref[TEI.forename]{forename} \hyperref[TEI.genName]{genName} \hyperref[TEI.geogName]{geogName} \hyperref[TEI.listOrg]{listOrg} \hyperref[TEI.listPlace]{listPlace} \hyperref[TEI.location]{location} \hyperref[TEI.nameLink]{nameLink} \hyperref[TEI.orgName]{orgName} \hyperref[TEI.persName]{persName} \hyperref[TEI.placeName]{placeName} \hyperref[TEI.region]{region} \hyperref[TEI.roleName]{roleName} \hyperref[TEI.settlement]{settlement} \hyperref[TEI.state]{state} \hyperref[TEI.surname]{surname}\par 
    \item[spoken: ]
   \hyperref[TEI.annotationBlock]{annotationBlock}\par 
    \item[textstructure: ]
   \hyperref[TEI.floatingText]{floatingText}\par 
    \item[transcr: ]
   \hyperref[TEI.addSpan]{addSpan} \hyperref[TEI.am]{am} \hyperref[TEI.damage]{damage} \hyperref[TEI.damageSpan]{damageSpan} \hyperref[TEI.delSpan]{delSpan} \hyperref[TEI.ex]{ex} \hyperref[TEI.fw]{fw} \hyperref[TEI.handShift]{handShift} \hyperref[TEI.listTranspose]{listTranspose} \hyperref[TEI.metamark]{metamark} \hyperref[TEI.mod]{mod} \hyperref[TEI.redo]{redo} \hyperref[TEI.restore]{restore} \hyperref[TEI.retrace]{retrace} \hyperref[TEI.secl]{secl} \hyperref[TEI.space]{space} \hyperref[TEI.subst]{subst} \hyperref[TEI.substJoin]{substJoin} \hyperref[TEI.supplied]{supplied} \hyperref[TEI.surplus]{surplus} \hyperref[TEI.undo]{undo}\par des données textuelles
    \item[{Exemple}]
  \leavevmode\bgroup\exampleFont \begin{shaded}\noindent\mbox{}{<\textbf{row}>}\mbox{}\newline 
\hspace*{6pt}{<\textbf{cell}\hspace*{6pt}{role}="{label}">}Comportement général{</\textbf{cell}>}\mbox{}\newline 
\hspace*{6pt}{<\textbf{cell}\hspace*{6pt}{role}="{data}">}Non satisfaisant, à cause de son inexactitude et de son\mbox{}\newline 
\hspace*{6pt}\hspace*{6pt} inconséquence{</\textbf{cell}>}\mbox{}\newline 
{</\textbf{row}>}\end{shaded}\egroup 


    \item[{Modèle de contenu}]
  \mbox{}\hfill\\[-10pt]\begin{Verbatim}[fontsize=\small]
<content>
 <macroRef key="macro.specialPara"/>
</content>
    
\end{Verbatim}

    \item[{Schéma Declaration}]
  \mbox{}\hfill\\[-10pt]\begin{Verbatim}[fontsize=\small]
element cell
{
   tei_att.global.attributes,
   tei_att.tableDecoration.attributes,
   tei_macro.specialPara}
\end{Verbatim}

\end{reflist}  \index{change=<change>|oddindex}\index{target=@target!<change>|oddindex}
\begin{reflist}
\item[]\begin{specHead}{TEI.change}{<change> }résume une modification ou une correction apportée à une version particulière d’un texte électronique partagé entre plusieurs chercheurs. [\xref{http://www.tei-c.org/release/doc/tei-p5-doc/en/html/HD.html\#HD6}{2.6. The Revision Description} \xref{http://www.tei-c.org/release/doc/tei-p5-doc/en/html/HD.html\#HD4C}{2.4.1. Creation} \xref{http://www.tei-c.org/release/doc/tei-p5-doc/en/html/PH.html\#PH-changes}{11.7. Identifying Changes and Revisions}]\end{specHead} 
    \item[{Module}]
  header
    \item[{Attributs}]
  Attributs \hyperref[TEI.att.ascribed]{att.ascribed} (\textit{@who}) \hyperref[TEI.att.datable]{att.datable} (\textit{@calendar}, \textit{@period})  (\hyperref[TEI.att.datable.w3c]{att.datable.w3c} (\textit{@when}, \textit{@notBefore}, \textit{@notAfter}, \textit{@from}, \textit{@to})) (\hyperref[TEI.att.datable.iso]{att.datable.iso} (\textit{@when-iso}, \textit{@notBefore-iso}, \textit{@notAfter-iso}, \textit{@from-iso}, \textit{@to-iso})) (\hyperref[TEI.att.datable.custom]{att.datable.custom} (\textit{@when-custom}, \textit{@notBefore-custom}, \textit{@notAfter-custom}, \textit{@from-custom}, \textit{@to-custom}, \textit{@datingPoint}, \textit{@datingMethod})) \hyperref[TEI.att.docStatus]{att.docStatus} (\textit{@status}) \hyperref[TEI.att.global]{att.global} (\textit{@xml:id}, \textit{@n}, \textit{@xml:lang}, \textit{@xml:base}, \textit{@xml:space})  (\hyperref[TEI.att.global.rendition]{att.global.rendition} (\textit{@rend}, \textit{@style}, \textit{@rendition})) (\hyperref[TEI.att.global.linking]{att.global.linking} (\textit{@corresp}, \textit{@synch}, \textit{@sameAs}, \textit{@copyOf}, \textit{@next}, \textit{@prev}, \textit{@exclude}, \textit{@select})) (\hyperref[TEI.att.global.analytic]{att.global.analytic} (\textit{@ana})) (\hyperref[TEI.att.global.facs]{att.global.facs} (\textit{@facs})) (\hyperref[TEI.att.global.change]{att.global.change} (\textit{@change})) (\hyperref[TEI.att.global.responsibility]{att.global.responsibility} (\textit{@cert}, \textit{@resp})) (\hyperref[TEI.att.global.source]{att.global.source} (\textit{@source})) \hyperref[TEI.att.typed]{att.typed} (\textit{@type}, \textit{@subtype}) \hfil\\[-10pt]\begin{sansreflist}
    \item[@target]
  points to one or more elements that belong to this change.
\begin{reflist}
    \item[{Statut}]
  Optionel
    \item[{Type de données}]
  1–∞ occurrences de \hyperref[TEI.teidata.pointer]{teidata.pointer} séparé par un espace
\end{reflist}  
\end{sansreflist}  
    \item[{Contenu dans}]
  
    \item[header: ]
   \hyperref[TEI.revisionDesc]{revisionDesc}\par 
    \item[msdescription: ]
   \hyperref[TEI.recordHist]{recordHist}
    \item[{Peut contenir}]
  
    \item[analysis: ]
   \hyperref[TEI.c]{c} \hyperref[TEI.cl]{cl} \hyperref[TEI.interp]{interp} \hyperref[TEI.interpGrp]{interpGrp} \hyperref[TEI.m]{m} \hyperref[TEI.pc]{pc} \hyperref[TEI.phr]{phr} \hyperref[TEI.s]{s} \hyperref[TEI.span]{span} \hyperref[TEI.spanGrp]{spanGrp} \hyperref[TEI.w]{w}\par 
    \item[core: ]
   \hyperref[TEI.abbr]{abbr} \hyperref[TEI.add]{add} \hyperref[TEI.address]{address} \hyperref[TEI.bibl]{bibl} \hyperref[TEI.biblStruct]{biblStruct} \hyperref[TEI.binaryObject]{binaryObject} \hyperref[TEI.cb]{cb} \hyperref[TEI.choice]{choice} \hyperref[TEI.cit]{cit} \hyperref[TEI.corr]{corr} \hyperref[TEI.date]{date} \hyperref[TEI.del]{del} \hyperref[TEI.desc]{desc} \hyperref[TEI.distinct]{distinct} \hyperref[TEI.email]{email} \hyperref[TEI.emph]{emph} \hyperref[TEI.expan]{expan} \hyperref[TEI.foreign]{foreign} \hyperref[TEI.gap]{gap} \hyperref[TEI.gb]{gb} \hyperref[TEI.gloss]{gloss} \hyperref[TEI.graphic]{graphic} \hyperref[TEI.hi]{hi} \hyperref[TEI.index]{index} \hyperref[TEI.l]{l} \hyperref[TEI.label]{label} \hyperref[TEI.lb]{lb} \hyperref[TEI.lg]{lg} \hyperref[TEI.list]{list} \hyperref[TEI.listBibl]{listBibl} \hyperref[TEI.measure]{measure} \hyperref[TEI.measureGrp]{measureGrp} \hyperref[TEI.media]{media} \hyperref[TEI.mentioned]{mentioned} \hyperref[TEI.milestone]{milestone} \hyperref[TEI.name]{name} \hyperref[TEI.note]{note} \hyperref[TEI.num]{num} \hyperref[TEI.orig]{orig} \hyperref[TEI.p]{p} \hyperref[TEI.pb]{pb} \hyperref[TEI.ptr]{ptr} \hyperref[TEI.q]{q} \hyperref[TEI.quote]{quote} \hyperref[TEI.ref]{ref} \hyperref[TEI.reg]{reg} \hyperref[TEI.rs]{rs} \hyperref[TEI.said]{said} \hyperref[TEI.sic]{sic} \hyperref[TEI.soCalled]{soCalled} \hyperref[TEI.sp]{sp} \hyperref[TEI.stage]{stage} \hyperref[TEI.term]{term} \hyperref[TEI.time]{time} \hyperref[TEI.title]{title} \hyperref[TEI.unclear]{unclear}\par 
    \item[derived-module-tei.istex: ]
   \hyperref[TEI.math]{math} \hyperref[TEI.mrow]{mrow}\par 
    \item[figures: ]
   \hyperref[TEI.figure]{figure} \hyperref[TEI.formula]{formula} \hyperref[TEI.notatedMusic]{notatedMusic} \hyperref[TEI.table]{table}\par 
    \item[header: ]
   \hyperref[TEI.biblFull]{biblFull} \hyperref[TEI.idno]{idno}\par 
    \item[iso-fs: ]
   \hyperref[TEI.fLib]{fLib} \hyperref[TEI.fs]{fs} \hyperref[TEI.fvLib]{fvLib}\par 
    \item[linking: ]
   \hyperref[TEI.ab]{ab} \hyperref[TEI.alt]{alt} \hyperref[TEI.altGrp]{altGrp} \hyperref[TEI.anchor]{anchor} \hyperref[TEI.join]{join} \hyperref[TEI.joinGrp]{joinGrp} \hyperref[TEI.link]{link} \hyperref[TEI.linkGrp]{linkGrp} \hyperref[TEI.seg]{seg} \hyperref[TEI.timeline]{timeline}\par 
    \item[msdescription: ]
   \hyperref[TEI.catchwords]{catchwords} \hyperref[TEI.depth]{depth} \hyperref[TEI.dim]{dim} \hyperref[TEI.dimensions]{dimensions} \hyperref[TEI.height]{height} \hyperref[TEI.heraldry]{heraldry} \hyperref[TEI.locus]{locus} \hyperref[TEI.locusGrp]{locusGrp} \hyperref[TEI.material]{material} \hyperref[TEI.msDesc]{msDesc} \hyperref[TEI.objectType]{objectType} \hyperref[TEI.origDate]{origDate} \hyperref[TEI.origPlace]{origPlace} \hyperref[TEI.secFol]{secFol} \hyperref[TEI.signatures]{signatures} \hyperref[TEI.source]{source} \hyperref[TEI.stamp]{stamp} \hyperref[TEI.watermark]{watermark} \hyperref[TEI.width]{width}\par 
    \item[namesdates: ]
   \hyperref[TEI.addName]{addName} \hyperref[TEI.affiliation]{affiliation} \hyperref[TEI.country]{country} \hyperref[TEI.forename]{forename} \hyperref[TEI.genName]{genName} \hyperref[TEI.geogName]{geogName} \hyperref[TEI.listOrg]{listOrg} \hyperref[TEI.listPlace]{listPlace} \hyperref[TEI.location]{location} \hyperref[TEI.nameLink]{nameLink} \hyperref[TEI.orgName]{orgName} \hyperref[TEI.persName]{persName} \hyperref[TEI.placeName]{placeName} \hyperref[TEI.region]{region} \hyperref[TEI.roleName]{roleName} \hyperref[TEI.settlement]{settlement} \hyperref[TEI.state]{state} \hyperref[TEI.surname]{surname}\par 
    \item[spoken: ]
   \hyperref[TEI.annotationBlock]{annotationBlock}\par 
    \item[textstructure: ]
   \hyperref[TEI.floatingText]{floatingText}\par 
    \item[transcr: ]
   \hyperref[TEI.addSpan]{addSpan} \hyperref[TEI.am]{am} \hyperref[TEI.damage]{damage} \hyperref[TEI.damageSpan]{damageSpan} \hyperref[TEI.delSpan]{delSpan} \hyperref[TEI.ex]{ex} \hyperref[TEI.fw]{fw} \hyperref[TEI.handShift]{handShift} \hyperref[TEI.listTranspose]{listTranspose} \hyperref[TEI.metamark]{metamark} \hyperref[TEI.mod]{mod} \hyperref[TEI.redo]{redo} \hyperref[TEI.restore]{restore} \hyperref[TEI.retrace]{retrace} \hyperref[TEI.secl]{secl} \hyperref[TEI.space]{space} \hyperref[TEI.subst]{subst} \hyperref[TEI.substJoin]{substJoin} \hyperref[TEI.supplied]{supplied} \hyperref[TEI.surplus]{surplus} \hyperref[TEI.undo]{undo}\par des données textuelles
    \item[{Note}]
  \par
Les modifications doivent être enregistrées dans l'ordre, par exemple la plus récente en premier.
    \item[{Exemple}]
  \leavevmode\bgroup\exampleFont \begin{shaded}\noindent\mbox{}{<\textbf{titleStmt}>}\mbox{}\newline 
\hspace*{6pt}{<\textbf{title}>} ... {</\textbf{title}>}\mbox{}\newline 
\hspace*{6pt}{<\textbf{editor}\hspace*{6pt}{xml:id}="{fr\textunderscore GK}">}G.K.{</\textbf{editor}>}\mbox{}\newline 
\hspace*{6pt}{<\textbf{respStmt}\hspace*{6pt}{xml:id}="{fr\textunderscore RT}">}\mbox{}\newline 
\hspace*{6pt}\hspace*{6pt}{<\textbf{resp}>}copie{</\textbf{resp}>}\mbox{}\newline 
\hspace*{6pt}\hspace*{6pt}{<\textbf{name}>}R.T.{</\textbf{name}>}\mbox{}\newline 
\hspace*{6pt}{</\textbf{respStmt}>}\mbox{}\newline 
{</\textbf{titleStmt}>}\mbox{}\newline 
{<\textbf{revisionDesc}>}\mbox{}\newline 
\hspace*{6pt}{<\textbf{change}\hspace*{6pt}{when}="{2008-02-02}"\hspace*{6pt}{who}="{\#fr\textunderscore RT}">}Fin du chapitre 23{</\textbf{change}>}\mbox{}\newline 
\hspace*{6pt}{<\textbf{change}\hspace*{6pt}{when}="{2008-01-02}"\hspace*{6pt}{who}="{\#fr\textunderscore RT}">}FIn du chapitre 2{</\textbf{change}>}\mbox{}\newline 
\hspace*{6pt}{<\textbf{change}\hspace*{6pt}{n}="{P2.2}"\hspace*{6pt}{when}="{1991-12-21}"\mbox{}\newline 
\hspace*{6pt}\hspace*{6pt}{who}="{\#fr\textunderscore GK}">}Ajout d'exemples dans la section 3{</\textbf{change}>}\mbox{}\newline 
\hspace*{6pt}{<\textbf{change}\hspace*{6pt}{when}="{1991-11-11}"\hspace*{6pt}{who}="{\#fr\textunderscore MSM}">}Suppression du chapitre 10{</\textbf{change}>}\mbox{}\newline 
{</\textbf{revisionDesc}>}\end{shaded}\egroup 


    \item[{Modèle de contenu}]
  \mbox{}\hfill\\[-10pt]\begin{Verbatim}[fontsize=\small]
<content>
 <macroRef key="macro.specialPara"/>
</content>
    
\end{Verbatim}

    \item[{Schéma Declaration}]
  \mbox{}\hfill\\[-10pt]\begin{Verbatim}[fontsize=\small]
element change
{
   tei_att.ascribed.attributes,
   tei_att.datable.attributes,
   tei_att.docStatus.attributes,
   tei_att.global.attributes,
   tei_att.typed.attributes,
   attribute target { list { + } }?,
   tei_macro.specialPara}
\end{Verbatim}

\end{reflist}  \index{choice=<choice>|oddindex}
\begin{reflist}
\item[]\begin{specHead}{TEI.choice}{<choice> }(choix) regroupe un certain nombre de balisages alternatifs possibles pour un même endroit dans un texte. [\xref{http://www.tei-c.org/release/doc/tei-p5-doc/en/html/CO.html\#COED}{3.4. Simple Editorial Changes}]\end{specHead} 
    \item[{Module}]
  core
    \item[{Attributs}]
  Attributs \hyperref[TEI.att.global]{att.global} (\textit{@xml:id}, \textit{@n}, \textit{@xml:lang}, \textit{@xml:base}, \textit{@xml:space})  (\hyperref[TEI.att.global.rendition]{att.global.rendition} (\textit{@rend}, \textit{@style}, \textit{@rendition})) (\hyperref[TEI.att.global.linking]{att.global.linking} (\textit{@corresp}, \textit{@synch}, \textit{@sameAs}, \textit{@copyOf}, \textit{@next}, \textit{@prev}, \textit{@exclude}, \textit{@select})) (\hyperref[TEI.att.global.analytic]{att.global.analytic} (\textit{@ana})) (\hyperref[TEI.att.global.facs]{att.global.facs} (\textit{@facs})) (\hyperref[TEI.att.global.change]{att.global.change} (\textit{@change})) (\hyperref[TEI.att.global.responsibility]{att.global.responsibility} (\textit{@cert}, \textit{@resp})) (\hyperref[TEI.att.global.source]{att.global.source} (\textit{@source}))
    \item[{Membre du}]
  \hyperref[TEI.model.linePart]{model.linePart} \hyperref[TEI.model.pPart.editorial]{model.pPart.editorial}
    \item[{Contenu dans}]
  
    \item[analysis: ]
   \hyperref[TEI.cl]{cl} \hyperref[TEI.pc]{pc} \hyperref[TEI.phr]{phr} \hyperref[TEI.s]{s} \hyperref[TEI.span]{span} \hyperref[TEI.w]{w}\par 
    \item[core: ]
   \hyperref[TEI.abbr]{abbr} \hyperref[TEI.add]{add} \hyperref[TEI.addrLine]{addrLine} \hyperref[TEI.author]{author} \hyperref[TEI.bibl]{bibl} \hyperref[TEI.biblScope]{biblScope} \hyperref[TEI.choice]{choice} \hyperref[TEI.citedRange]{citedRange} \hyperref[TEI.corr]{corr} \hyperref[TEI.date]{date} \hyperref[TEI.del]{del} \hyperref[TEI.desc]{desc} \hyperref[TEI.distinct]{distinct} \hyperref[TEI.editor]{editor} \hyperref[TEI.email]{email} \hyperref[TEI.emph]{emph} \hyperref[TEI.expan]{expan} \hyperref[TEI.foreign]{foreign} \hyperref[TEI.gloss]{gloss} \hyperref[TEI.head]{head} \hyperref[TEI.headItem]{headItem} \hyperref[TEI.headLabel]{headLabel} \hyperref[TEI.hi]{hi} \hyperref[TEI.item]{item} \hyperref[TEI.l]{l} \hyperref[TEI.label]{label} \hyperref[TEI.measure]{measure} \hyperref[TEI.meeting]{meeting} \hyperref[TEI.mentioned]{mentioned} \hyperref[TEI.name]{name} \hyperref[TEI.note]{note} \hyperref[TEI.num]{num} \hyperref[TEI.orig]{orig} \hyperref[TEI.p]{p} \hyperref[TEI.pubPlace]{pubPlace} \hyperref[TEI.publisher]{publisher} \hyperref[TEI.q]{q} \hyperref[TEI.quote]{quote} \hyperref[TEI.ref]{ref} \hyperref[TEI.reg]{reg} \hyperref[TEI.resp]{resp} \hyperref[TEI.rs]{rs} \hyperref[TEI.said]{said} \hyperref[TEI.sic]{sic} \hyperref[TEI.soCalled]{soCalled} \hyperref[TEI.speaker]{speaker} \hyperref[TEI.stage]{stage} \hyperref[TEI.street]{street} \hyperref[TEI.term]{term} \hyperref[TEI.textLang]{textLang} \hyperref[TEI.time]{time} \hyperref[TEI.title]{title} \hyperref[TEI.unclear]{unclear}\par 
    \item[figures: ]
   \hyperref[TEI.cell]{cell} \hyperref[TEI.figDesc]{figDesc}\par 
    \item[header: ]
   \hyperref[TEI.authority]{authority} \hyperref[TEI.change]{change} \hyperref[TEI.classCode]{classCode} \hyperref[TEI.creation]{creation} \hyperref[TEI.distributor]{distributor} \hyperref[TEI.edition]{edition} \hyperref[TEI.extent]{extent} \hyperref[TEI.funder]{funder} \hyperref[TEI.language]{language} \hyperref[TEI.licence]{licence} \hyperref[TEI.rendition]{rendition}\par 
    \item[iso-fs: ]
   \hyperref[TEI.fDescr]{fDescr} \hyperref[TEI.fsDescr]{fsDescr}\par 
    \item[linking: ]
   \hyperref[TEI.ab]{ab} \hyperref[TEI.seg]{seg}\par 
    \item[msdescription: ]
   \hyperref[TEI.accMat]{accMat} \hyperref[TEI.acquisition]{acquisition} \hyperref[TEI.additions]{additions} \hyperref[TEI.catchwords]{catchwords} \hyperref[TEI.collation]{collation} \hyperref[TEI.colophon]{colophon} \hyperref[TEI.condition]{condition} \hyperref[TEI.custEvent]{custEvent} \hyperref[TEI.decoNote]{decoNote} \hyperref[TEI.explicit]{explicit} \hyperref[TEI.filiation]{filiation} \hyperref[TEI.finalRubric]{finalRubric} \hyperref[TEI.foliation]{foliation} \hyperref[TEI.heraldry]{heraldry} \hyperref[TEI.incipit]{incipit} \hyperref[TEI.layout]{layout} \hyperref[TEI.material]{material} \hyperref[TEI.musicNotation]{musicNotation} \hyperref[TEI.objectType]{objectType} \hyperref[TEI.origDate]{origDate} \hyperref[TEI.origPlace]{origPlace} \hyperref[TEI.origin]{origin} \hyperref[TEI.provenance]{provenance} \hyperref[TEI.rubric]{rubric} \hyperref[TEI.secFol]{secFol} \hyperref[TEI.signatures]{signatures} \hyperref[TEI.source]{source} \hyperref[TEI.stamp]{stamp} \hyperref[TEI.summary]{summary} \hyperref[TEI.support]{support} \hyperref[TEI.surrogates]{surrogates} \hyperref[TEI.typeNote]{typeNote} \hyperref[TEI.watermark]{watermark}\par 
    \item[namesdates: ]
   \hyperref[TEI.addName]{addName} \hyperref[TEI.affiliation]{affiliation} \hyperref[TEI.country]{country} \hyperref[TEI.forename]{forename} \hyperref[TEI.genName]{genName} \hyperref[TEI.geogName]{geogName} \hyperref[TEI.nameLink]{nameLink} \hyperref[TEI.orgName]{orgName} \hyperref[TEI.persName]{persName} \hyperref[TEI.placeName]{placeName} \hyperref[TEI.region]{region} \hyperref[TEI.roleName]{roleName} \hyperref[TEI.settlement]{settlement} \hyperref[TEI.surname]{surname}\par 
    \item[textstructure: ]
   \hyperref[TEI.docAuthor]{docAuthor} \hyperref[TEI.docDate]{docDate} \hyperref[TEI.docEdition]{docEdition} \hyperref[TEI.titlePart]{titlePart}\par 
    \item[transcr: ]
   \hyperref[TEI.damage]{damage} \hyperref[TEI.fw]{fw} \hyperref[TEI.line]{line} \hyperref[TEI.metamark]{metamark} \hyperref[TEI.mod]{mod} \hyperref[TEI.restore]{restore} \hyperref[TEI.retrace]{retrace} \hyperref[TEI.secl]{secl} \hyperref[TEI.supplied]{supplied} \hyperref[TEI.surplus]{surplus} \hyperref[TEI.zone]{zone}
    \item[{Peut contenir}]
  
    \item[core: ]
   \hyperref[TEI.abbr]{abbr} \hyperref[TEI.choice]{choice} \hyperref[TEI.corr]{corr} \hyperref[TEI.expan]{expan} \hyperref[TEI.orig]{orig} \hyperref[TEI.reg]{reg} \hyperref[TEI.sic]{sic} \hyperref[TEI.unclear]{unclear}\par 
    \item[linking: ]
   \hyperref[TEI.seg]{seg}\par 
    \item[transcr: ]
   \hyperref[TEI.am]{am} \hyperref[TEI.ex]{ex} \hyperref[TEI.supplied]{supplied}
    \item[{Note}]
  \par
Parce que les éléments contenus par un élément \hyperref[TEI.choice]{<choice>} correspondent tous à des solutions possibles pour encoder la même séquence, il est naturel de les envisager comme exclusifs les uns des autres. Toutefois il peut y avoir des cas où la pleine représentation d'un texte requiert de considérer ces encodages alternatifs comme parallèles.\par
A Noter aussi que les \hyperref[TEI.choice]{<choice>} peuvent s'imbriquer.\par
Pour une version de \hyperref[TEI.choice]{<choice>} spécialisée pour l'encodage de témoins multiples d'une même oeuvre, l'élément \texttt{<app>} peut etre préférable : voir la section \xref{http://www.tei-c.org/release/doc/tei-p5-doc/en/html/TC.html\#TCAPLL}{12.1. The Apparatus Entry, Readings, and Witnesses}.
    \item[{Exemple}]
  L'encodage d'une édition des \textit{Essais} pourra faire état à la fois des formes originales et des formes modernisées correspondantes:.\leavevmode\bgroup\exampleFont \begin{shaded}\noindent\mbox{}{<\textbf{p}>}Ainsi lecteur, je suis{<\textbf{choice}>}\mbox{}\newline 
\hspace*{6pt}\hspace*{6pt}{<\textbf{orig}>}moy-mesmes{</\textbf{orig}>}\mbox{}\newline 
\hspace*{6pt}\hspace*{6pt}{<\textbf{reg}>}moi-même{</\textbf{reg}>}\mbox{}\newline 
\hspace*{6pt}{</\textbf{choice}>} la matière de mon livre : ce n'est pas raison que tu emploies ton loisir en un {<\textbf{choice}>}\mbox{}\newline 
\hspace*{6pt}\hspace*{6pt}{<\textbf{orig}>}subject{</\textbf{orig}>}\mbox{}\newline 
\hspace*{6pt}\hspace*{6pt}{<\textbf{reg}>}sujet{</\textbf{reg}>}\mbox{}\newline 
\hspace*{6pt}{</\textbf{choice}>}si frivole et si vain.{</\textbf{p}>}\end{shaded}\egroup 


    \item[{Modèle de contenu}]
  \mbox{}\hfill\\[-10pt]\begin{Verbatim}[fontsize=\small]
<content>
 <alternate maxOccurs="unbounded"
  minOccurs="0">
  <classRef key="model.choicePart"/>
  <elementRef key="choice"/>
 </alternate>
</content>
    
\end{Verbatim}

    \item[{Schéma Declaration}]
  \mbox{}\hfill\\[-10pt]\begin{Verbatim}[fontsize=\small]
element choice
{
   tei_att.global.attributes,
   ( tei_model.choicePart | tei_choice )*
}
\end{Verbatim}

\end{reflist}  \index{cit=<cit>|oddindex}
\begin{reflist}
\item[]\begin{specHead}{TEI.cit}{<cit> }(citation) citation provenant d'un autre document comprenant la référence bibliographique de sa source. Dans un dictionnaire il peut contenir un exemple avec au moins une occurrence du mot employé dans l’acception qui est décrite, ou une traduction du mot-clé, ou un exemple. [\xref{http://www.tei-c.org/release/doc/tei-p5-doc/en/html/CO.html\#COHQQ}{3.3.3. Quotation} \xref{http://www.tei-c.org/release/doc/tei-p5-doc/en/html/DS.html\#DSGRP}{4.3.1. Grouped Texts} \xref{http://www.tei-c.org/release/doc/tei-p5-doc/en/html/DI.html\#DITPEG}{9.3.5.1. Examples}]\end{specHead} 
    \item[{Module}]
  core
    \item[{Attributs}]
  Attributs \hyperref[TEI.att.global]{att.global} (\textit{@xml:id}, \textit{@n}, \textit{@xml:lang}, \textit{@xml:base}, \textit{@xml:space})  (\hyperref[TEI.att.global.rendition]{att.global.rendition} (\textit{@rend}, \textit{@style}, \textit{@rendition})) (\hyperref[TEI.att.global.linking]{att.global.linking} (\textit{@corresp}, \textit{@synch}, \textit{@sameAs}, \textit{@copyOf}, \textit{@next}, \textit{@prev}, \textit{@exclude}, \textit{@select})) (\hyperref[TEI.att.global.analytic]{att.global.analytic} (\textit{@ana})) (\hyperref[TEI.att.global.facs]{att.global.facs} (\textit{@facs})) (\hyperref[TEI.att.global.change]{att.global.change} (\textit{@change})) (\hyperref[TEI.att.global.responsibility]{att.global.responsibility} (\textit{@cert}, \textit{@resp})) (\hyperref[TEI.att.global.source]{att.global.source} (\textit{@source})) \hyperref[TEI.att.typed]{att.typed} (\textit{@type}, \textit{@subtype}) 
    \item[{Membre du}]
  \hyperref[TEI.model.quoteLike]{model.quoteLike}
    \item[{Contenu dans}]
  
    \item[core: ]
   \hyperref[TEI.add]{add} \hyperref[TEI.cit]{cit} \hyperref[TEI.corr]{corr} \hyperref[TEI.del]{del} \hyperref[TEI.desc]{desc} \hyperref[TEI.emph]{emph} \hyperref[TEI.head]{head} \hyperref[TEI.hi]{hi} \hyperref[TEI.item]{item} \hyperref[TEI.l]{l} \hyperref[TEI.meeting]{meeting} \hyperref[TEI.note]{note} \hyperref[TEI.orig]{orig} \hyperref[TEI.p]{p} \hyperref[TEI.q]{q} \hyperref[TEI.quote]{quote} \hyperref[TEI.ref]{ref} \hyperref[TEI.reg]{reg} \hyperref[TEI.said]{said} \hyperref[TEI.sic]{sic} \hyperref[TEI.sp]{sp} \hyperref[TEI.stage]{stage} \hyperref[TEI.title]{title} \hyperref[TEI.unclear]{unclear}\par 
    \item[figures: ]
   \hyperref[TEI.cell]{cell} \hyperref[TEI.figDesc]{figDesc} \hyperref[TEI.figure]{figure}\par 
    \item[header: ]
   \hyperref[TEI.change]{change} \hyperref[TEI.licence]{licence} \hyperref[TEI.rendition]{rendition}\par 
    \item[iso-fs: ]
   \hyperref[TEI.fDescr]{fDescr} \hyperref[TEI.fsDescr]{fsDescr}\par 
    \item[linking: ]
   \hyperref[TEI.ab]{ab} \hyperref[TEI.seg]{seg}\par 
    \item[msdescription: ]
   \hyperref[TEI.accMat]{accMat} \hyperref[TEI.acquisition]{acquisition} \hyperref[TEI.additions]{additions} \hyperref[TEI.collation]{collation} \hyperref[TEI.condition]{condition} \hyperref[TEI.custEvent]{custEvent} \hyperref[TEI.decoNote]{decoNote} \hyperref[TEI.filiation]{filiation} \hyperref[TEI.foliation]{foliation} \hyperref[TEI.layout]{layout} \hyperref[TEI.msItem]{msItem} \hyperref[TEI.musicNotation]{musicNotation} \hyperref[TEI.origin]{origin} \hyperref[TEI.provenance]{provenance} \hyperref[TEI.signatures]{signatures} \hyperref[TEI.source]{source} \hyperref[TEI.summary]{summary} \hyperref[TEI.support]{support} \hyperref[TEI.surrogates]{surrogates} \hyperref[TEI.typeNote]{typeNote}\par 
    \item[textstructure: ]
   \hyperref[TEI.body]{body} \hyperref[TEI.div]{div} \hyperref[TEI.docEdition]{docEdition} \hyperref[TEI.titlePart]{titlePart}\par 
    \item[transcr: ]
   \hyperref[TEI.damage]{damage} \hyperref[TEI.metamark]{metamark} \hyperref[TEI.mod]{mod} \hyperref[TEI.restore]{restore} \hyperref[TEI.retrace]{retrace} \hyperref[TEI.secl]{secl} \hyperref[TEI.supplied]{supplied} \hyperref[TEI.surplus]{surplus}
    \item[{Peut contenir}]
  
    \item[analysis: ]
   \hyperref[TEI.interp]{interp} \hyperref[TEI.interpGrp]{interpGrp} \hyperref[TEI.span]{span} \hyperref[TEI.spanGrp]{spanGrp}\par 
    \item[core: ]
   \hyperref[TEI.bibl]{bibl} \hyperref[TEI.biblStruct]{biblStruct} \hyperref[TEI.cb]{cb} \hyperref[TEI.cit]{cit} \hyperref[TEI.gap]{gap} \hyperref[TEI.gb]{gb} \hyperref[TEI.index]{index} \hyperref[TEI.lb]{lb} \hyperref[TEI.listBibl]{listBibl} \hyperref[TEI.milestone]{milestone} \hyperref[TEI.note]{note} \hyperref[TEI.pb]{pb} \hyperref[TEI.ptr]{ptr} \hyperref[TEI.q]{q} \hyperref[TEI.quote]{quote} \hyperref[TEI.ref]{ref} \hyperref[TEI.said]{said}\par 
    \item[figures: ]
   \hyperref[TEI.figure]{figure} \hyperref[TEI.notatedMusic]{notatedMusic}\par 
    \item[header: ]
   \hyperref[TEI.biblFull]{biblFull}\par 
    \item[iso-fs: ]
   \hyperref[TEI.fLib]{fLib} \hyperref[TEI.fs]{fs} \hyperref[TEI.fvLib]{fvLib}\par 
    \item[linking: ]
   \hyperref[TEI.alt]{alt} \hyperref[TEI.altGrp]{altGrp} \hyperref[TEI.anchor]{anchor} \hyperref[TEI.join]{join} \hyperref[TEI.joinGrp]{joinGrp} \hyperref[TEI.link]{link} \hyperref[TEI.linkGrp]{linkGrp} \hyperref[TEI.timeline]{timeline}\par 
    \item[msdescription: ]
   \hyperref[TEI.msDesc]{msDesc} \hyperref[TEI.source]{source}\par 
    \item[textstructure: ]
   \hyperref[TEI.floatingText]{floatingText}\par 
    \item[transcr: ]
   \hyperref[TEI.addSpan]{addSpan} \hyperref[TEI.damageSpan]{damageSpan} \hyperref[TEI.delSpan]{delSpan} \hyperref[TEI.fw]{fw} \hyperref[TEI.listTranspose]{listTranspose} \hyperref[TEI.metamark]{metamark} \hyperref[TEI.space]{space} \hyperref[TEI.substJoin]{substJoin}
    \item[{Exemple}]
  \leavevmode\bgroup\exampleFont \begin{shaded}\noindent\mbox{}{<\textbf{cit}>}\mbox{}\newline 
\hspace*{6pt}{<\textbf{quote}>}Regarde de tous tes yeux, regarde{</\textbf{quote}>}\mbox{}\newline 
\hspace*{6pt}{<\textbf{bibl}>}Jules Verne, Michel Strogof{</\textbf{bibl}>}\mbox{}\newline 
{</\textbf{cit}>}\end{shaded}\egroup 


    \item[{Exemple}]
  \leavevmode\bgroup\exampleFont \begin{shaded}\noindent\mbox{}{<\textbf{entry}>}\mbox{}\newline 
\hspace*{6pt}{<\textbf{form}>}\mbox{}\newline 
\hspace*{6pt}\hspace*{6pt}{<\textbf{orth}>}to horrify{</\textbf{orth}>}\mbox{}\newline 
\hspace*{6pt}{</\textbf{form}>}\mbox{}\newline 
\hspace*{6pt}{<\textbf{cit}\hspace*{6pt}{type}="{translation}"\hspace*{6pt}{xml:lang}="{en}">}\mbox{}\newline 
\hspace*{6pt}\hspace*{6pt}{<\textbf{quote}>}horrifier{</\textbf{quote}>}\mbox{}\newline 
\hspace*{6pt}{</\textbf{cit}>}\mbox{}\newline 
\hspace*{6pt}{<\textbf{cit}\hspace*{6pt}{type}="{example}">}\mbox{}\newline 
\hspace*{6pt}\hspace*{6pt}{<\textbf{quote}>}she was horrified at the expense.{</\textbf{quote}>}\mbox{}\newline 
\hspace*{6pt}\hspace*{6pt}{<\textbf{cit}\hspace*{6pt}{type}="{translation}"\hspace*{6pt}{xml:lang}="{en}">}\mbox{}\newline 
\hspace*{6pt}\hspace*{6pt}\hspace*{6pt}{<\textbf{quote}>}elle était horrifiée par la dépense.{</\textbf{quote}>}\mbox{}\newline 
\hspace*{6pt}\hspace*{6pt}{</\textbf{cit}>}\mbox{}\newline 
\hspace*{6pt}{</\textbf{cit}>}\mbox{}\newline 
{</\textbf{entry}>}\end{shaded}\egroup 


    \item[{Modèle de contenu}]
  \mbox{}\hfill\\[-10pt]\begin{Verbatim}[fontsize=\small]
<content>
 <alternate maxOccurs="unbounded"
  minOccurs="1">
  <classRef key="model.qLike"/>
  <classRef key="model.egLike"/>
  <classRef key="model.biblLike"/>
  <classRef key="model.ptrLike"/>
  <classRef key="model.global"/>
  <classRef key="model.entryPart"/>
 </alternate>
</content>
    
\end{Verbatim}

    \item[{Schéma Declaration}]
  \mbox{}\hfill\\[-10pt]\begin{Verbatim}[fontsize=\small]
element cit
{
   tei_att.global.attributes,
   tei_att.typed.attributes,
   (
      tei_model.qLike    | tei_model.egLike    | tei_model.biblLike    | tei_model.ptrLike    | tei_model.global    | tei_model.entryPart   )+
}
\end{Verbatim}

\end{reflist}  \index{citedRange=<citedRange>|oddindex}
\begin{reflist}
\item[]\begin{specHead}{TEI.citedRange}{<citedRange> }(cited range) defines the range of cited content, often represented by pages or other units [\xref{http://www.tei-c.org/release/doc/tei-p5-doc/en/html/CO.html\#COBICOB}{3.11.2.5. Scopes and Ranges in Bibliographic Citations}]\end{specHead} 
    \item[{Module}]
  core
    \item[{Attributs}]
  Attributs \hyperref[TEI.att.global]{att.global} (\textit{@xml:id}, \textit{@n}, \textit{@xml:lang}, \textit{@xml:base}, \textit{@xml:space})  (\hyperref[TEI.att.global.rendition]{att.global.rendition} (\textit{@rend}, \textit{@style}, \textit{@rendition})) (\hyperref[TEI.att.global.linking]{att.global.linking} (\textit{@corresp}, \textit{@synch}, \textit{@sameAs}, \textit{@copyOf}, \textit{@next}, \textit{@prev}, \textit{@exclude}, \textit{@select})) (\hyperref[TEI.att.global.analytic]{att.global.analytic} (\textit{@ana})) (\hyperref[TEI.att.global.facs]{att.global.facs} (\textit{@facs})) (\hyperref[TEI.att.global.change]{att.global.change} (\textit{@change})) (\hyperref[TEI.att.global.responsibility]{att.global.responsibility} (\textit{@cert}, \textit{@resp})) (\hyperref[TEI.att.global.source]{att.global.source} (\textit{@source})) \hyperref[TEI.att.pointing]{att.pointing} (\textit{@targetLang}, \textit{@target}, \textit{@evaluate}) \hyperref[TEI.att.citing]{att.citing} (\textit{@unit}, \textit{@from}, \textit{@to}) 
    \item[{Membre du}]
  \hyperref[TEI.model.biblPart]{model.biblPart}
    \item[{Contenu dans}]
  
    \item[core: ]
   \hyperref[TEI.bibl]{bibl} \hyperref[TEI.biblStruct]{biblStruct}
    \item[{Peut contenir}]
  
    \item[analysis: ]
   \hyperref[TEI.c]{c} \hyperref[TEI.cl]{cl} \hyperref[TEI.interp]{interp} \hyperref[TEI.interpGrp]{interpGrp} \hyperref[TEI.m]{m} \hyperref[TEI.pc]{pc} \hyperref[TEI.phr]{phr} \hyperref[TEI.s]{s} \hyperref[TEI.span]{span} \hyperref[TEI.spanGrp]{spanGrp} \hyperref[TEI.w]{w}\par 
    \item[core: ]
   \hyperref[TEI.abbr]{abbr} \hyperref[TEI.add]{add} \hyperref[TEI.address]{address} \hyperref[TEI.binaryObject]{binaryObject} \hyperref[TEI.cb]{cb} \hyperref[TEI.choice]{choice} \hyperref[TEI.corr]{corr} \hyperref[TEI.date]{date} \hyperref[TEI.del]{del} \hyperref[TEI.distinct]{distinct} \hyperref[TEI.email]{email} \hyperref[TEI.emph]{emph} \hyperref[TEI.expan]{expan} \hyperref[TEI.foreign]{foreign} \hyperref[TEI.gap]{gap} \hyperref[TEI.gb]{gb} \hyperref[TEI.gloss]{gloss} \hyperref[TEI.graphic]{graphic} \hyperref[TEI.hi]{hi} \hyperref[TEI.index]{index} \hyperref[TEI.lb]{lb} \hyperref[TEI.measure]{measure} \hyperref[TEI.measureGrp]{measureGrp} \hyperref[TEI.media]{media} \hyperref[TEI.mentioned]{mentioned} \hyperref[TEI.milestone]{milestone} \hyperref[TEI.name]{name} \hyperref[TEI.note]{note} \hyperref[TEI.num]{num} \hyperref[TEI.orig]{orig} \hyperref[TEI.pb]{pb} \hyperref[TEI.ptr]{ptr} \hyperref[TEI.ref]{ref} \hyperref[TEI.reg]{reg} \hyperref[TEI.rs]{rs} \hyperref[TEI.sic]{sic} \hyperref[TEI.soCalled]{soCalled} \hyperref[TEI.term]{term} \hyperref[TEI.time]{time} \hyperref[TEI.title]{title} \hyperref[TEI.unclear]{unclear}\par 
    \item[derived-module-tei.istex: ]
   \hyperref[TEI.math]{math} \hyperref[TEI.mrow]{mrow}\par 
    \item[figures: ]
   \hyperref[TEI.figure]{figure} \hyperref[TEI.formula]{formula} \hyperref[TEI.notatedMusic]{notatedMusic}\par 
    \item[header: ]
   \hyperref[TEI.idno]{idno}\par 
    \item[iso-fs: ]
   \hyperref[TEI.fLib]{fLib} \hyperref[TEI.fs]{fs} \hyperref[TEI.fvLib]{fvLib}\par 
    \item[linking: ]
   \hyperref[TEI.alt]{alt} \hyperref[TEI.altGrp]{altGrp} \hyperref[TEI.anchor]{anchor} \hyperref[TEI.join]{join} \hyperref[TEI.joinGrp]{joinGrp} \hyperref[TEI.link]{link} \hyperref[TEI.linkGrp]{linkGrp} \hyperref[TEI.seg]{seg} \hyperref[TEI.timeline]{timeline}\par 
    \item[msdescription: ]
   \hyperref[TEI.catchwords]{catchwords} \hyperref[TEI.depth]{depth} \hyperref[TEI.dim]{dim} \hyperref[TEI.dimensions]{dimensions} \hyperref[TEI.height]{height} \hyperref[TEI.heraldry]{heraldry} \hyperref[TEI.locus]{locus} \hyperref[TEI.locusGrp]{locusGrp} \hyperref[TEI.material]{material} \hyperref[TEI.objectType]{objectType} \hyperref[TEI.origDate]{origDate} \hyperref[TEI.origPlace]{origPlace} \hyperref[TEI.secFol]{secFol} \hyperref[TEI.signatures]{signatures} \hyperref[TEI.source]{source} \hyperref[TEI.stamp]{stamp} \hyperref[TEI.watermark]{watermark} \hyperref[TEI.width]{width}\par 
    \item[namesdates: ]
   \hyperref[TEI.addName]{addName} \hyperref[TEI.affiliation]{affiliation} \hyperref[TEI.country]{country} \hyperref[TEI.forename]{forename} \hyperref[TEI.genName]{genName} \hyperref[TEI.geogName]{geogName} \hyperref[TEI.location]{location} \hyperref[TEI.nameLink]{nameLink} \hyperref[TEI.orgName]{orgName} \hyperref[TEI.persName]{persName} \hyperref[TEI.placeName]{placeName} \hyperref[TEI.region]{region} \hyperref[TEI.roleName]{roleName} \hyperref[TEI.settlement]{settlement} \hyperref[TEI.state]{state} \hyperref[TEI.surname]{surname}\par 
    \item[spoken: ]
   \hyperref[TEI.annotationBlock]{annotationBlock}\par 
    \item[transcr: ]
   \hyperref[TEI.addSpan]{addSpan} \hyperref[TEI.am]{am} \hyperref[TEI.damage]{damage} \hyperref[TEI.damageSpan]{damageSpan} \hyperref[TEI.delSpan]{delSpan} \hyperref[TEI.ex]{ex} \hyperref[TEI.fw]{fw} \hyperref[TEI.handShift]{handShift} \hyperref[TEI.listTranspose]{listTranspose} \hyperref[TEI.metamark]{metamark} \hyperref[TEI.mod]{mod} \hyperref[TEI.redo]{redo} \hyperref[TEI.restore]{restore} \hyperref[TEI.retrace]{retrace} \hyperref[TEI.secl]{secl} \hyperref[TEI.space]{space} \hyperref[TEI.subst]{subst} \hyperref[TEI.substJoin]{substJoin} \hyperref[TEI.supplied]{supplied} \hyperref[TEI.surplus]{surplus} \hyperref[TEI.undo]{undo}\par des données textuelles
    \item[{Note}]
  \par
When a single page is being cited, use the {\itshape from} and {\itshape to} attributes with an identical value. When no clear endpoint is provided, the {\itshape from} attribute may be used without {\itshape to}; for example a citation such as ‘p. 3ff’ might be encoded \texttt{<biblScope from="3">p. 3ff<biblScope>}.
    \item[{Exemple}]
  \leavevmode\bgroup\exampleFont \begin{shaded}\noindent\mbox{}{<\textbf{citedRange}>}pp 12–13{</\textbf{citedRange}>}\mbox{}\newline 
{<\textbf{citedRange}\hspace*{6pt}{from}="{12}"\hspace*{6pt}{to}="{13}"\hspace*{6pt}{unit}="{page}"/>}\mbox{}\newline 
{<\textbf{citedRange}\hspace*{6pt}{unit}="{volume}">}II{</\textbf{citedRange}>}\mbox{}\newline 
{<\textbf{citedRange}\hspace*{6pt}{unit}="{page}">}12{</\textbf{citedRange}>}\end{shaded}\egroup 


    \item[{Exemple}]
  \leavevmode\bgroup\exampleFont \begin{shaded}\noindent\mbox{}{<\textbf{bibl}>}\mbox{}\newline 
\hspace*{6pt}{<\textbf{ptr}\hspace*{6pt}{target}="{\#mueller01}"/>}, {<\textbf{citedRange}\hspace*{6pt}{target}="{http://example.com/mueller3.xml\#page4}">}vol. 3, pp.\mbox{}\newline 
\hspace*{6pt}\hspace*{6pt} 4-5{</\textbf{citedRange}>}\mbox{}\newline 
{</\textbf{bibl}>}\end{shaded}\egroup 


    \item[{Modèle de contenu}]
  \mbox{}\hfill\\[-10pt]\begin{Verbatim}[fontsize=\small]
<content>
 <macroRef key="macro.phraseSeq"/>
</content>
    
\end{Verbatim}

    \item[{Schéma Declaration}]
  \mbox{}\hfill\\[-10pt]\begin{Verbatim}[fontsize=\small]
element citedRange
{
   tei_att.global.attributes,
   tei_att.pointing.attributes,
   tei_att.citing.attributes,
   tei_macro.phraseSeq}
\end{Verbatim}

\end{reflist}  \index{cl=<cl>|oddindex}
\begin{reflist}
\item[]\begin{specHead}{TEI.cl}{<cl> }(proposition) représente une proposition grammaticale [\xref{http://www.tei-c.org/release/doc/tei-p5-doc/en/html/AI.html\#AILC}{17.1. Linguistic Segment Categories}]\end{specHead} 
    \item[{Module}]
  analysis
    \item[{Attributs}]
  Attributs \hyperref[TEI.att.global]{att.global} (\textit{@xml:id}, \textit{@n}, \textit{@xml:lang}, \textit{@xml:base}, \textit{@xml:space})  (\hyperref[TEI.att.global.rendition]{att.global.rendition} (\textit{@rend}, \textit{@style}, \textit{@rendition})) (\hyperref[TEI.att.global.linking]{att.global.linking} (\textit{@corresp}, \textit{@synch}, \textit{@sameAs}, \textit{@copyOf}, \textit{@next}, \textit{@prev}, \textit{@exclude}, \textit{@select})) (\hyperref[TEI.att.global.analytic]{att.global.analytic} (\textit{@ana})) (\hyperref[TEI.att.global.facs]{att.global.facs} (\textit{@facs})) (\hyperref[TEI.att.global.change]{att.global.change} (\textit{@change})) (\hyperref[TEI.att.global.responsibility]{att.global.responsibility} (\textit{@cert}, \textit{@resp})) (\hyperref[TEI.att.global.source]{att.global.source} (\textit{@source})) \hyperref[TEI.att.segLike]{att.segLike} (\textit{@function})  (\hyperref[TEI.att.datcat]{att.datcat} (\textit{@datcat}, \textit{@valueDatcat})) (\hyperref[TEI.att.fragmentable]{att.fragmentable} (\textit{@part})) \hyperref[TEI.att.typed]{att.typed} (\textit{@type}, \textit{@subtype}) 
    \item[{Membre du}]
  \hyperref[TEI.model.segLike]{model.segLike}
    \item[{Contenu dans}]
  
    \item[analysis: ]
   \hyperref[TEI.cl]{cl} \hyperref[TEI.phr]{phr} \hyperref[TEI.s]{s}\par 
    \item[core: ]
   \hyperref[TEI.abbr]{abbr} \hyperref[TEI.add]{add} \hyperref[TEI.addrLine]{addrLine} \hyperref[TEI.author]{author} \hyperref[TEI.bibl]{bibl} \hyperref[TEI.biblScope]{biblScope} \hyperref[TEI.citedRange]{citedRange} \hyperref[TEI.corr]{corr} \hyperref[TEI.date]{date} \hyperref[TEI.del]{del} \hyperref[TEI.distinct]{distinct} \hyperref[TEI.editor]{editor} \hyperref[TEI.email]{email} \hyperref[TEI.emph]{emph} \hyperref[TEI.expan]{expan} \hyperref[TEI.foreign]{foreign} \hyperref[TEI.gloss]{gloss} \hyperref[TEI.head]{head} \hyperref[TEI.headItem]{headItem} \hyperref[TEI.headLabel]{headLabel} \hyperref[TEI.hi]{hi} \hyperref[TEI.item]{item} \hyperref[TEI.l]{l} \hyperref[TEI.label]{label} \hyperref[TEI.measure]{measure} \hyperref[TEI.mentioned]{mentioned} \hyperref[TEI.name]{name} \hyperref[TEI.note]{note} \hyperref[TEI.num]{num} \hyperref[TEI.orig]{orig} \hyperref[TEI.p]{p} \hyperref[TEI.pubPlace]{pubPlace} \hyperref[TEI.publisher]{publisher} \hyperref[TEI.q]{q} \hyperref[TEI.quote]{quote} \hyperref[TEI.ref]{ref} \hyperref[TEI.reg]{reg} \hyperref[TEI.rs]{rs} \hyperref[TEI.said]{said} \hyperref[TEI.sic]{sic} \hyperref[TEI.soCalled]{soCalled} \hyperref[TEI.speaker]{speaker} \hyperref[TEI.stage]{stage} \hyperref[TEI.street]{street} \hyperref[TEI.term]{term} \hyperref[TEI.textLang]{textLang} \hyperref[TEI.time]{time} \hyperref[TEI.title]{title} \hyperref[TEI.unclear]{unclear}\par 
    \item[figures: ]
   \hyperref[TEI.cell]{cell}\par 
    \item[header: ]
   \hyperref[TEI.change]{change} \hyperref[TEI.distributor]{distributor} \hyperref[TEI.edition]{edition} \hyperref[TEI.extent]{extent} \hyperref[TEI.licence]{licence}\par 
    \item[linking: ]
   \hyperref[TEI.ab]{ab} \hyperref[TEI.seg]{seg}\par 
    \item[msdescription: ]
   \hyperref[TEI.accMat]{accMat} \hyperref[TEI.acquisition]{acquisition} \hyperref[TEI.additions]{additions} \hyperref[TEI.catchwords]{catchwords} \hyperref[TEI.collation]{collation} \hyperref[TEI.colophon]{colophon} \hyperref[TEI.condition]{condition} \hyperref[TEI.custEvent]{custEvent} \hyperref[TEI.decoNote]{decoNote} \hyperref[TEI.explicit]{explicit} \hyperref[TEI.filiation]{filiation} \hyperref[TEI.finalRubric]{finalRubric} \hyperref[TEI.foliation]{foliation} \hyperref[TEI.heraldry]{heraldry} \hyperref[TEI.incipit]{incipit} \hyperref[TEI.layout]{layout} \hyperref[TEI.material]{material} \hyperref[TEI.musicNotation]{musicNotation} \hyperref[TEI.objectType]{objectType} \hyperref[TEI.origDate]{origDate} \hyperref[TEI.origPlace]{origPlace} \hyperref[TEI.origin]{origin} \hyperref[TEI.provenance]{provenance} \hyperref[TEI.rubric]{rubric} \hyperref[TEI.secFol]{secFol} \hyperref[TEI.signatures]{signatures} \hyperref[TEI.source]{source} \hyperref[TEI.stamp]{stamp} \hyperref[TEI.summary]{summary} \hyperref[TEI.support]{support} \hyperref[TEI.surrogates]{surrogates} \hyperref[TEI.typeNote]{typeNote} \hyperref[TEI.watermark]{watermark}\par 
    \item[namesdates: ]
   \hyperref[TEI.addName]{addName} \hyperref[TEI.affiliation]{affiliation} \hyperref[TEI.country]{country} \hyperref[TEI.forename]{forename} \hyperref[TEI.genName]{genName} \hyperref[TEI.geogName]{geogName} \hyperref[TEI.nameLink]{nameLink} \hyperref[TEI.orgName]{orgName} \hyperref[TEI.persName]{persName} \hyperref[TEI.placeName]{placeName} \hyperref[TEI.region]{region} \hyperref[TEI.roleName]{roleName} \hyperref[TEI.settlement]{settlement} \hyperref[TEI.surname]{surname}\par 
    \item[textstructure: ]
   \hyperref[TEI.docAuthor]{docAuthor} \hyperref[TEI.docDate]{docDate} \hyperref[TEI.docEdition]{docEdition} \hyperref[TEI.titlePart]{titlePart}\par 
    \item[transcr: ]
   \hyperref[TEI.damage]{damage} \hyperref[TEI.fw]{fw} \hyperref[TEI.metamark]{metamark} \hyperref[TEI.mod]{mod} \hyperref[TEI.restore]{restore} \hyperref[TEI.retrace]{retrace} \hyperref[TEI.secl]{secl} \hyperref[TEI.supplied]{supplied} \hyperref[TEI.surplus]{surplus}
    \item[{Peut contenir}]
  
    \item[analysis: ]
   \hyperref[TEI.c]{c} \hyperref[TEI.cl]{cl} \hyperref[TEI.interp]{interp} \hyperref[TEI.interpGrp]{interpGrp} \hyperref[TEI.m]{m} \hyperref[TEI.pc]{pc} \hyperref[TEI.phr]{phr} \hyperref[TEI.s]{s} \hyperref[TEI.span]{span} \hyperref[TEI.spanGrp]{spanGrp} \hyperref[TEI.w]{w}\par 
    \item[core: ]
   \hyperref[TEI.abbr]{abbr} \hyperref[TEI.add]{add} \hyperref[TEI.address]{address} \hyperref[TEI.binaryObject]{binaryObject} \hyperref[TEI.cb]{cb} \hyperref[TEI.choice]{choice} \hyperref[TEI.corr]{corr} \hyperref[TEI.date]{date} \hyperref[TEI.del]{del} \hyperref[TEI.distinct]{distinct} \hyperref[TEI.email]{email} \hyperref[TEI.emph]{emph} \hyperref[TEI.expan]{expan} \hyperref[TEI.foreign]{foreign} \hyperref[TEI.gap]{gap} \hyperref[TEI.gb]{gb} \hyperref[TEI.gloss]{gloss} \hyperref[TEI.graphic]{graphic} \hyperref[TEI.hi]{hi} \hyperref[TEI.index]{index} \hyperref[TEI.lb]{lb} \hyperref[TEI.measure]{measure} \hyperref[TEI.measureGrp]{measureGrp} \hyperref[TEI.media]{media} \hyperref[TEI.mentioned]{mentioned} \hyperref[TEI.milestone]{milestone} \hyperref[TEI.name]{name} \hyperref[TEI.note]{note} \hyperref[TEI.num]{num} \hyperref[TEI.orig]{orig} \hyperref[TEI.pb]{pb} \hyperref[TEI.ptr]{ptr} \hyperref[TEI.ref]{ref} \hyperref[TEI.reg]{reg} \hyperref[TEI.rs]{rs} \hyperref[TEI.sic]{sic} \hyperref[TEI.soCalled]{soCalled} \hyperref[TEI.term]{term} \hyperref[TEI.time]{time} \hyperref[TEI.title]{title} \hyperref[TEI.unclear]{unclear}\par 
    \item[derived-module-tei.istex: ]
   \hyperref[TEI.math]{math} \hyperref[TEI.mrow]{mrow}\par 
    \item[figures: ]
   \hyperref[TEI.figure]{figure} \hyperref[TEI.formula]{formula} \hyperref[TEI.notatedMusic]{notatedMusic}\par 
    \item[header: ]
   \hyperref[TEI.idno]{idno}\par 
    \item[iso-fs: ]
   \hyperref[TEI.fLib]{fLib} \hyperref[TEI.fs]{fs} \hyperref[TEI.fvLib]{fvLib}\par 
    \item[linking: ]
   \hyperref[TEI.alt]{alt} \hyperref[TEI.altGrp]{altGrp} \hyperref[TEI.anchor]{anchor} \hyperref[TEI.join]{join} \hyperref[TEI.joinGrp]{joinGrp} \hyperref[TEI.link]{link} \hyperref[TEI.linkGrp]{linkGrp} \hyperref[TEI.seg]{seg} \hyperref[TEI.timeline]{timeline}\par 
    \item[msdescription: ]
   \hyperref[TEI.catchwords]{catchwords} \hyperref[TEI.depth]{depth} \hyperref[TEI.dim]{dim} \hyperref[TEI.dimensions]{dimensions} \hyperref[TEI.height]{height} \hyperref[TEI.heraldry]{heraldry} \hyperref[TEI.locus]{locus} \hyperref[TEI.locusGrp]{locusGrp} \hyperref[TEI.material]{material} \hyperref[TEI.objectType]{objectType} \hyperref[TEI.origDate]{origDate} \hyperref[TEI.origPlace]{origPlace} \hyperref[TEI.secFol]{secFol} \hyperref[TEI.signatures]{signatures} \hyperref[TEI.source]{source} \hyperref[TEI.stamp]{stamp} \hyperref[TEI.watermark]{watermark} \hyperref[TEI.width]{width}\par 
    \item[namesdates: ]
   \hyperref[TEI.addName]{addName} \hyperref[TEI.affiliation]{affiliation} \hyperref[TEI.country]{country} \hyperref[TEI.forename]{forename} \hyperref[TEI.genName]{genName} \hyperref[TEI.geogName]{geogName} \hyperref[TEI.location]{location} \hyperref[TEI.nameLink]{nameLink} \hyperref[TEI.orgName]{orgName} \hyperref[TEI.persName]{persName} \hyperref[TEI.placeName]{placeName} \hyperref[TEI.region]{region} \hyperref[TEI.roleName]{roleName} \hyperref[TEI.settlement]{settlement} \hyperref[TEI.state]{state} \hyperref[TEI.surname]{surname}\par 
    \item[spoken: ]
   \hyperref[TEI.annotationBlock]{annotationBlock}\par 
    \item[transcr: ]
   \hyperref[TEI.addSpan]{addSpan} \hyperref[TEI.am]{am} \hyperref[TEI.damage]{damage} \hyperref[TEI.damageSpan]{damageSpan} \hyperref[TEI.delSpan]{delSpan} \hyperref[TEI.ex]{ex} \hyperref[TEI.fw]{fw} \hyperref[TEI.handShift]{handShift} \hyperref[TEI.listTranspose]{listTranspose} \hyperref[TEI.metamark]{metamark} \hyperref[TEI.mod]{mod} \hyperref[TEI.redo]{redo} \hyperref[TEI.restore]{restore} \hyperref[TEI.retrace]{retrace} \hyperref[TEI.secl]{secl} \hyperref[TEI.space]{space} \hyperref[TEI.subst]{subst} \hyperref[TEI.substJoin]{substJoin} \hyperref[TEI.supplied]{supplied} \hyperref[TEI.surplus]{surplus} \hyperref[TEI.undo]{undo}\par des données textuelles
    \item[{Note}]
  \par
L'attribut {\itshape type} peut être utilisé pour indiquer le type de proposition, avec des valeurs telles que subordonnée, infinitive, declarative, interrogative, relative etc.
    \item[{Exemple}]
  \leavevmode\bgroup\exampleFont \begin{shaded}\noindent\mbox{}{<\textbf{cl}\hspace*{6pt}{function}="{proposition\textunderscore relative\textunderscore déterminative}"\mbox{}\newline 
\hspace*{6pt}{type}="{relative}">} Il nous rejoindra dans\mbox{}\newline 
 les jours{<\textbf{cl}>}qui viennent.{</\textbf{cl}>}\mbox{}\newline 
{</\textbf{cl}>}\end{shaded}\egroup 


    \item[{Modèle de contenu}]
  \mbox{}\hfill\\[-10pt]\begin{Verbatim}[fontsize=\small]
<content>
 <macroRef key="macro.phraseSeq"/>
</content>
    
\end{Verbatim}

    \item[{Schéma Declaration}]
  \mbox{}\hfill\\[-10pt]\begin{Verbatim}[fontsize=\small]
element cl
{
   tei_att.global.attributes,
   tei_att.segLike.attributes,
   tei_att.typed.attributes,
   tei_macro.phraseSeq}
\end{Verbatim}

\end{reflist}  \index{classCode=<classCode>|oddindex}\index{scheme=@scheme!<classCode>|oddindex}
\begin{reflist}
\item[]\begin{specHead}{TEI.classCode}{<classCode> }(code de classification) contient le code de classification attribué à ce texte en référence à un système standard de classification. [\xref{http://www.tei-c.org/release/doc/tei-p5-doc/en/html/HD.html\#HD43}{2.4.3. The Text Classification}]\end{specHead} 
    \item[{Module}]
  header
    \item[{Attributs}]
  Attributs \hyperref[TEI.att.global]{att.global} (\textit{@xml:id}, \textit{@n}, \textit{@xml:lang}, \textit{@xml:base}, \textit{@xml:space})  (\hyperref[TEI.att.global.rendition]{att.global.rendition} (\textit{@rend}, \textit{@style}, \textit{@rendition})) (\hyperref[TEI.att.global.linking]{att.global.linking} (\textit{@corresp}, \textit{@synch}, \textit{@sameAs}, \textit{@copyOf}, \textit{@next}, \textit{@prev}, \textit{@exclude}, \textit{@select})) (\hyperref[TEI.att.global.analytic]{att.global.analytic} (\textit{@ana})) (\hyperref[TEI.att.global.facs]{att.global.facs} (\textit{@facs})) (\hyperref[TEI.att.global.change]{att.global.change} (\textit{@change})) (\hyperref[TEI.att.global.responsibility]{att.global.responsibility} (\textit{@cert}, \textit{@resp})) (\hyperref[TEI.att.global.source]{att.global.source} (\textit{@source})) \hfil\\[-10pt]\begin{sansreflist}
    \item[@scheme]
  identifie le système de classification ou la taxinomie utilisée.
\begin{reflist}
    \item[{Statut}]
  Requis
    \item[{Type de données}]
  \hyperref[TEI.teidata.pointer]{teidata.pointer}
\end{reflist}  
\end{sansreflist}  
    \item[{Contenu dans}]
  
    \item[core: ]
   \hyperref[TEI.imprint]{imprint}\par 
    \item[header: ]
   \hyperref[TEI.textClass]{textClass}
    \item[{Peut contenir}]
  
    \item[analysis: ]
   \hyperref[TEI.interp]{interp} \hyperref[TEI.interpGrp]{interpGrp} \hyperref[TEI.span]{span} \hyperref[TEI.spanGrp]{spanGrp}\par 
    \item[core: ]
   \hyperref[TEI.abbr]{abbr} \hyperref[TEI.address]{address} \hyperref[TEI.cb]{cb} \hyperref[TEI.choice]{choice} \hyperref[TEI.date]{date} \hyperref[TEI.distinct]{distinct} \hyperref[TEI.email]{email} \hyperref[TEI.emph]{emph} \hyperref[TEI.expan]{expan} \hyperref[TEI.foreign]{foreign} \hyperref[TEI.gap]{gap} \hyperref[TEI.gb]{gb} \hyperref[TEI.gloss]{gloss} \hyperref[TEI.hi]{hi} \hyperref[TEI.index]{index} \hyperref[TEI.lb]{lb} \hyperref[TEI.measure]{measure} \hyperref[TEI.measureGrp]{measureGrp} \hyperref[TEI.mentioned]{mentioned} \hyperref[TEI.milestone]{milestone} \hyperref[TEI.name]{name} \hyperref[TEI.note]{note} \hyperref[TEI.num]{num} \hyperref[TEI.pb]{pb} \hyperref[TEI.ptr]{ptr} \hyperref[TEI.ref]{ref} \hyperref[TEI.rs]{rs} \hyperref[TEI.soCalled]{soCalled} \hyperref[TEI.term]{term} \hyperref[TEI.time]{time} \hyperref[TEI.title]{title}\par 
    \item[figures: ]
   \hyperref[TEI.figure]{figure} \hyperref[TEI.notatedMusic]{notatedMusic}\par 
    \item[header: ]
   \hyperref[TEI.idno]{idno}\par 
    \item[iso-fs: ]
   \hyperref[TEI.fLib]{fLib} \hyperref[TEI.fs]{fs} \hyperref[TEI.fvLib]{fvLib}\par 
    \item[linking: ]
   \hyperref[TEI.alt]{alt} \hyperref[TEI.altGrp]{altGrp} \hyperref[TEI.anchor]{anchor} \hyperref[TEI.join]{join} \hyperref[TEI.joinGrp]{joinGrp} \hyperref[TEI.link]{link} \hyperref[TEI.linkGrp]{linkGrp} \hyperref[TEI.timeline]{timeline}\par 
    \item[msdescription: ]
   \hyperref[TEI.catchwords]{catchwords} \hyperref[TEI.depth]{depth} \hyperref[TEI.dim]{dim} \hyperref[TEI.dimensions]{dimensions} \hyperref[TEI.height]{height} \hyperref[TEI.heraldry]{heraldry} \hyperref[TEI.locus]{locus} \hyperref[TEI.locusGrp]{locusGrp} \hyperref[TEI.material]{material} \hyperref[TEI.objectType]{objectType} \hyperref[TEI.origDate]{origDate} \hyperref[TEI.origPlace]{origPlace} \hyperref[TEI.secFol]{secFol} \hyperref[TEI.signatures]{signatures} \hyperref[TEI.source]{source} \hyperref[TEI.stamp]{stamp} \hyperref[TEI.watermark]{watermark} \hyperref[TEI.width]{width}\par 
    \item[namesdates: ]
   \hyperref[TEI.addName]{addName} \hyperref[TEI.affiliation]{affiliation} \hyperref[TEI.country]{country} \hyperref[TEI.forename]{forename} \hyperref[TEI.genName]{genName} \hyperref[TEI.geogName]{geogName} \hyperref[TEI.location]{location} \hyperref[TEI.nameLink]{nameLink} \hyperref[TEI.orgName]{orgName} \hyperref[TEI.persName]{persName} \hyperref[TEI.placeName]{placeName} \hyperref[TEI.region]{region} \hyperref[TEI.roleName]{roleName} \hyperref[TEI.settlement]{settlement} \hyperref[TEI.state]{state} \hyperref[TEI.surname]{surname}\par 
    \item[transcr: ]
   \hyperref[TEI.addSpan]{addSpan} \hyperref[TEI.am]{am} \hyperref[TEI.damageSpan]{damageSpan} \hyperref[TEI.delSpan]{delSpan} \hyperref[TEI.ex]{ex} \hyperref[TEI.fw]{fw} \hyperref[TEI.listTranspose]{listTranspose} \hyperref[TEI.metamark]{metamark} \hyperref[TEI.space]{space} \hyperref[TEI.subst]{subst} \hyperref[TEI.substJoin]{substJoin}\par des données textuelles
    \item[{Exemple}]
  \leavevmode\bgroup\exampleFont \begin{shaded}\noindent\mbox{}{<\textbf{classCode}\hspace*{6pt}{scheme}="{http://www.udc.org}">}410{</\textbf{classCode}>}\end{shaded}\egroup 


    \item[{Exemple}]
  \leavevmode\bgroup\exampleFont \begin{shaded}\noindent\mbox{}{<\textbf{classCode}\hspace*{6pt}{scheme}="{http://www.oclc.org/}">}801{</\textbf{classCode}>}\mbox{}\newline 
{<\textbf{bibl}>}classification Dewey{</\textbf{bibl}>}\end{shaded}\egroup 


    \item[{Modèle de contenu}]
  \mbox{}\hfill\\[-10pt]\begin{Verbatim}[fontsize=\small]
<content>
 <macroRef key="macro.phraseSeq.limited"/>
</content>
    
\end{Verbatim}

    \item[{Schéma Declaration}]
  \mbox{}\hfill\\[-10pt]\begin{Verbatim}[fontsize=\small]
element classCode
{
   tei_att.global.attributes,
   attribute scheme { text },
   tei_macro.phraseSeq.limited}
\end{Verbatim}

\end{reflist}  \index{classDecl=<classDecl>|oddindex}
\begin{reflist}
\item[]\begin{specHead}{TEI.classDecl}{<classDecl> }(déclaration de classification) contient une ou plusieurs taxinomies définissant les codes de classification utilisés n’importe où dans le texte. [\xref{http://www.tei-c.org/release/doc/tei-p5-doc/en/html/HD.html\#HD55}{2.3.7. The Classification Declaration} \xref{http://www.tei-c.org/release/doc/tei-p5-doc/en/html/HD.html\#HD5}{2.3. The Encoding Description}]\end{specHead} 
    \item[{Module}]
  header
    \item[{Attributs}]
  Attributs \hyperref[TEI.att.global]{att.global} (\textit{@xml:id}, \textit{@n}, \textit{@xml:lang}, \textit{@xml:base}, \textit{@xml:space})  (\hyperref[TEI.att.global.rendition]{att.global.rendition} (\textit{@rend}, \textit{@style}, \textit{@rendition})) (\hyperref[TEI.att.global.linking]{att.global.linking} (\textit{@corresp}, \textit{@synch}, \textit{@sameAs}, \textit{@copyOf}, \textit{@next}, \textit{@prev}, \textit{@exclude}, \textit{@select})) (\hyperref[TEI.att.global.analytic]{att.global.analytic} (\textit{@ana})) (\hyperref[TEI.att.global.facs]{att.global.facs} (\textit{@facs})) (\hyperref[TEI.att.global.change]{att.global.change} (\textit{@change})) (\hyperref[TEI.att.global.responsibility]{att.global.responsibility} (\textit{@cert}, \textit{@resp})) (\hyperref[TEI.att.global.source]{att.global.source} (\textit{@source}))
    \item[{Membre du}]
  \hyperref[TEI.model.encodingDescPart]{model.encodingDescPart}
    \item[{Contenu dans}]
  
    \item[header: ]
   \hyperref[TEI.encodingDesc]{encodingDesc}
    \item[{Peut contenir}]
  
    \item[header: ]
   \hyperref[TEI.taxonomy]{taxonomy}
    \item[{Exemple}]
  \leavevmode\bgroup\exampleFont \begin{shaded}\noindent\mbox{}{<\textbf{classDecl}>}\mbox{}\newline 
\hspace*{6pt}{<\textbf{taxonomy}\hspace*{6pt}{xml:id}="{RAMEAU}">}\mbox{}\newline 
\hspace*{6pt}\hspace*{6pt}{<\textbf{bibl}>}Répertoire d'autorité-matière encyclopédique et alphabétique unifié\mbox{}\newline 
\hspace*{6pt}\hspace*{6pt}\hspace*{6pt}\hspace*{6pt} (RAMEAU) de la Bibliothèque nationale de France. {<\textbf{ptr}\hspace*{6pt}{target}="{http://rameau.bnf.fr/}"/>}\mbox{}\newline 
\hspace*{6pt}\hspace*{6pt}{</\textbf{bibl}>}\mbox{}\newline 
\hspace*{6pt}{</\textbf{taxonomy}>}\mbox{}\newline 
{</\textbf{classDecl}>}\mbox{}\newline 
\textit{<!-- ... -->}\mbox{}\newline 
{<\textbf{textClass}>}\mbox{}\newline 
\hspace*{6pt}{<\textbf{keywords}\hspace*{6pt}{scheme}="{\#RAMEAU}">}\mbox{}\newline 
\hspace*{6pt}\hspace*{6pt}{<\textbf{term}>}Bien et mal -- Enseignement coranique{</\textbf{term}>}\mbox{}\newline 
\hspace*{6pt}{</\textbf{keywords}>}\mbox{}\newline 
{</\textbf{textClass}>}\end{shaded}\egroup 


    \item[{Modèle de contenu}]
  \mbox{}\hfill\\[-10pt]\begin{Verbatim}[fontsize=\small]
<content>
 <elementRef key="taxonomy"
  maxOccurs="unbounded" minOccurs="1"/>
</content>
    
\end{Verbatim}

    \item[{Schéma Declaration}]
  \mbox{}\hfill\\[-10pt]\begin{Verbatim}[fontsize=\small]
element classDecl { tei_att.global.attributes, tei_taxonomy+ }
\end{Verbatim}

\end{reflist}  \index{collation=<collation>|oddindex}
\begin{reflist}
\item[]\begin{specHead}{TEI.collation}{<collation> }(collation) contient la description de l'organisation des feuillets ou bifeuillets d'un manuscrit [\xref{http://www.tei-c.org/release/doc/tei-p5-doc/en/html/MS.html\#msph1}{10.7.1. Object Description}]\end{specHead} 
    \item[{Module}]
  msdescription
    \item[{Attributs}]
  Attributs \hyperref[TEI.att.global]{att.global} (\textit{@xml:id}, \textit{@n}, \textit{@xml:lang}, \textit{@xml:base}, \textit{@xml:space})  (\hyperref[TEI.att.global.rendition]{att.global.rendition} (\textit{@rend}, \textit{@style}, \textit{@rendition})) (\hyperref[TEI.att.global.linking]{att.global.linking} (\textit{@corresp}, \textit{@synch}, \textit{@sameAs}, \textit{@copyOf}, \textit{@next}, \textit{@prev}, \textit{@exclude}, \textit{@select})) (\hyperref[TEI.att.global.analytic]{att.global.analytic} (\textit{@ana})) (\hyperref[TEI.att.global.facs]{att.global.facs} (\textit{@facs})) (\hyperref[TEI.att.global.change]{att.global.change} (\textit{@change})) (\hyperref[TEI.att.global.responsibility]{att.global.responsibility} (\textit{@cert}, \textit{@resp})) (\hyperref[TEI.att.global.source]{att.global.source} (\textit{@source}))
    \item[{Contenu dans}]
  
    \item[msdescription: ]
   \hyperref[TEI.supportDesc]{supportDesc}
    \item[{Peut contenir}]
  
    \item[analysis: ]
   \hyperref[TEI.c]{c} \hyperref[TEI.cl]{cl} \hyperref[TEI.interp]{interp} \hyperref[TEI.interpGrp]{interpGrp} \hyperref[TEI.m]{m} \hyperref[TEI.pc]{pc} \hyperref[TEI.phr]{phr} \hyperref[TEI.s]{s} \hyperref[TEI.span]{span} \hyperref[TEI.spanGrp]{spanGrp} \hyperref[TEI.w]{w}\par 
    \item[core: ]
   \hyperref[TEI.abbr]{abbr} \hyperref[TEI.add]{add} \hyperref[TEI.address]{address} \hyperref[TEI.bibl]{bibl} \hyperref[TEI.biblStruct]{biblStruct} \hyperref[TEI.binaryObject]{binaryObject} \hyperref[TEI.cb]{cb} \hyperref[TEI.choice]{choice} \hyperref[TEI.cit]{cit} \hyperref[TEI.corr]{corr} \hyperref[TEI.date]{date} \hyperref[TEI.del]{del} \hyperref[TEI.desc]{desc} \hyperref[TEI.distinct]{distinct} \hyperref[TEI.email]{email} \hyperref[TEI.emph]{emph} \hyperref[TEI.expan]{expan} \hyperref[TEI.foreign]{foreign} \hyperref[TEI.gap]{gap} \hyperref[TEI.gb]{gb} \hyperref[TEI.gloss]{gloss} \hyperref[TEI.graphic]{graphic} \hyperref[TEI.hi]{hi} \hyperref[TEI.index]{index} \hyperref[TEI.l]{l} \hyperref[TEI.label]{label} \hyperref[TEI.lb]{lb} \hyperref[TEI.lg]{lg} \hyperref[TEI.list]{list} \hyperref[TEI.listBibl]{listBibl} \hyperref[TEI.measure]{measure} \hyperref[TEI.measureGrp]{measureGrp} \hyperref[TEI.media]{media} \hyperref[TEI.mentioned]{mentioned} \hyperref[TEI.milestone]{milestone} \hyperref[TEI.name]{name} \hyperref[TEI.note]{note} \hyperref[TEI.num]{num} \hyperref[TEI.orig]{orig} \hyperref[TEI.p]{p} \hyperref[TEI.pb]{pb} \hyperref[TEI.ptr]{ptr} \hyperref[TEI.q]{q} \hyperref[TEI.quote]{quote} \hyperref[TEI.ref]{ref} \hyperref[TEI.reg]{reg} \hyperref[TEI.rs]{rs} \hyperref[TEI.said]{said} \hyperref[TEI.sic]{sic} \hyperref[TEI.soCalled]{soCalled} \hyperref[TEI.sp]{sp} \hyperref[TEI.stage]{stage} \hyperref[TEI.term]{term} \hyperref[TEI.time]{time} \hyperref[TEI.title]{title} \hyperref[TEI.unclear]{unclear}\par 
    \item[derived-module-tei.istex: ]
   \hyperref[TEI.math]{math} \hyperref[TEI.mrow]{mrow}\par 
    \item[figures: ]
   \hyperref[TEI.figure]{figure} \hyperref[TEI.formula]{formula} \hyperref[TEI.notatedMusic]{notatedMusic} \hyperref[TEI.table]{table}\par 
    \item[header: ]
   \hyperref[TEI.biblFull]{biblFull} \hyperref[TEI.idno]{idno}\par 
    \item[iso-fs: ]
   \hyperref[TEI.fLib]{fLib} \hyperref[TEI.fs]{fs} \hyperref[TEI.fvLib]{fvLib}\par 
    \item[linking: ]
   \hyperref[TEI.ab]{ab} \hyperref[TEI.alt]{alt} \hyperref[TEI.altGrp]{altGrp} \hyperref[TEI.anchor]{anchor} \hyperref[TEI.join]{join} \hyperref[TEI.joinGrp]{joinGrp} \hyperref[TEI.link]{link} \hyperref[TEI.linkGrp]{linkGrp} \hyperref[TEI.seg]{seg} \hyperref[TEI.timeline]{timeline}\par 
    \item[msdescription: ]
   \hyperref[TEI.catchwords]{catchwords} \hyperref[TEI.depth]{depth} \hyperref[TEI.dim]{dim} \hyperref[TEI.dimensions]{dimensions} \hyperref[TEI.height]{height} \hyperref[TEI.heraldry]{heraldry} \hyperref[TEI.locus]{locus} \hyperref[TEI.locusGrp]{locusGrp} \hyperref[TEI.material]{material} \hyperref[TEI.msDesc]{msDesc} \hyperref[TEI.objectType]{objectType} \hyperref[TEI.origDate]{origDate} \hyperref[TEI.origPlace]{origPlace} \hyperref[TEI.secFol]{secFol} \hyperref[TEI.signatures]{signatures} \hyperref[TEI.source]{source} \hyperref[TEI.stamp]{stamp} \hyperref[TEI.watermark]{watermark} \hyperref[TEI.width]{width}\par 
    \item[namesdates: ]
   \hyperref[TEI.addName]{addName} \hyperref[TEI.affiliation]{affiliation} \hyperref[TEI.country]{country} \hyperref[TEI.forename]{forename} \hyperref[TEI.genName]{genName} \hyperref[TEI.geogName]{geogName} \hyperref[TEI.listOrg]{listOrg} \hyperref[TEI.listPlace]{listPlace} \hyperref[TEI.location]{location} \hyperref[TEI.nameLink]{nameLink} \hyperref[TEI.orgName]{orgName} \hyperref[TEI.persName]{persName} \hyperref[TEI.placeName]{placeName} \hyperref[TEI.region]{region} \hyperref[TEI.roleName]{roleName} \hyperref[TEI.settlement]{settlement} \hyperref[TEI.state]{state} \hyperref[TEI.surname]{surname}\par 
    \item[spoken: ]
   \hyperref[TEI.annotationBlock]{annotationBlock}\par 
    \item[textstructure: ]
   \hyperref[TEI.floatingText]{floatingText}\par 
    \item[transcr: ]
   \hyperref[TEI.addSpan]{addSpan} \hyperref[TEI.am]{am} \hyperref[TEI.damage]{damage} \hyperref[TEI.damageSpan]{damageSpan} \hyperref[TEI.delSpan]{delSpan} \hyperref[TEI.ex]{ex} \hyperref[TEI.fw]{fw} \hyperref[TEI.handShift]{handShift} \hyperref[TEI.listTranspose]{listTranspose} \hyperref[TEI.metamark]{metamark} \hyperref[TEI.mod]{mod} \hyperref[TEI.redo]{redo} \hyperref[TEI.restore]{restore} \hyperref[TEI.retrace]{retrace} \hyperref[TEI.secl]{secl} \hyperref[TEI.space]{space} \hyperref[TEI.subst]{subst} \hyperref[TEI.substJoin]{substJoin} \hyperref[TEI.supplied]{supplied} \hyperref[TEI.surplus]{surplus} \hyperref[TEI.undo]{undo}\par des données textuelles
    \item[{Exemple}]
  \leavevmode\bgroup\exampleFont \begin{shaded}\noindent\mbox{}{<\textbf{collation}>}The written leaves preceded by an original flyleaf, conjoint with the\mbox{}\newline 
 pastedown.{</\textbf{collation}>}\end{shaded}\egroup 


    \item[{Exemple}]
  \leavevmode\bgroup\exampleFont \begin{shaded}\noindent\mbox{}{<\textbf{collation}>}\mbox{}\newline 
\hspace*{6pt}{<\textbf{p}>}\mbox{}\newline 
\hspace*{6pt}\hspace*{6pt}{<\textbf{formula}>}1-5.8 6.6 (catchword, f. 46, does not match following text) 7-8.8 9.10, 11.2\mbox{}\newline 
\hspace*{6pt}\hspace*{6pt}\hspace*{6pt}\hspace*{6pt} (through f. 82) 12-14.8 15.8(-7){</\textbf{formula}>}\mbox{}\newline 
\hspace*{6pt}\hspace*{6pt}{<\textbf{catchwords}>}Catchwords are written horizontally in center or towards the right lower\mbox{}\newline 
\hspace*{6pt}\hspace*{6pt}\hspace*{6pt}\hspace*{6pt} margin in various manners: in red ink for quires 1-6 (which are also signed in red ink\mbox{}\newline 
\hspace*{6pt}\hspace*{6pt}\hspace*{6pt}\hspace*{6pt} with letters of the alphabet and arabic numerals); quires 7-9 in ink of text within\mbox{}\newline 
\hspace*{6pt}\hspace*{6pt}\hspace*{6pt}\hspace*{6pt} yellow decorated frames; quire 10 in red decorated frame; quire 12 in ink of text;\mbox{}\newline 
\hspace*{6pt}\hspace*{6pt}\hspace*{6pt}\hspace*{6pt} quire 13 with red decorative slashes; quire 14 added in cursive hand.{</\textbf{catchwords}>}\mbox{}\newline 
\hspace*{6pt}{</\textbf{p}>}\mbox{}\newline 
{</\textbf{collation}>}\end{shaded}\egroup 


    \item[{Modèle de contenu}]
  \mbox{}\hfill\\[-10pt]\begin{Verbatim}[fontsize=\small]
<content>
 <macroRef key="macro.specialPara"/>
</content>
    
\end{Verbatim}

    \item[{Schéma Declaration}]
  \mbox{}\hfill\\[-10pt]\begin{Verbatim}[fontsize=\small]
element collation { tei_att.global.attributes, tei_macro.specialPara }
\end{Verbatim}

\end{reflist}  \index{collection=<collection>|oddindex}
\begin{reflist}
\item[]\begin{specHead}{TEI.collection}{<collection> }(collection) Contient le nom d'une collection de manuscrits, ceux-ci ne se trouvant pas nécessairement dans le même lieu de conservation. [\xref{http://www.tei-c.org/release/doc/tei-p5-doc/en/html/MS.html\#msid}{10.4. The Manuscript Identifier}]\end{specHead} 
    \item[{Module}]
  msdescription
    \item[{Attributs}]
  Attributs \hyperref[TEI.att.global]{att.global} (\textit{@xml:id}, \textit{@n}, \textit{@xml:lang}, \textit{@xml:base}, \textit{@xml:space})  (\hyperref[TEI.att.global.rendition]{att.global.rendition} (\textit{@rend}, \textit{@style}, \textit{@rendition})) (\hyperref[TEI.att.global.linking]{att.global.linking} (\textit{@corresp}, \textit{@synch}, \textit{@sameAs}, \textit{@copyOf}, \textit{@next}, \textit{@prev}, \textit{@exclude}, \textit{@select})) (\hyperref[TEI.att.global.analytic]{att.global.analytic} (\textit{@ana})) (\hyperref[TEI.att.global.facs]{att.global.facs} (\textit{@facs})) (\hyperref[TEI.att.global.change]{att.global.change} (\textit{@change})) (\hyperref[TEI.att.global.responsibility]{att.global.responsibility} (\textit{@cert}, \textit{@resp})) (\hyperref[TEI.att.global.source]{att.global.source} (\textit{@source})) \hyperref[TEI.att.naming]{att.naming} (\textit{@role}, \textit{@nymRef})  (\hyperref[TEI.att.canonical]{att.canonical} (\textit{@key}, \textit{@ref})) \hyperref[TEI.att.typed]{att.typed} (\textit{@type}, \textit{@subtype}) 
    \item[{Contenu dans}]
  
    \item[msdescription: ]
   \hyperref[TEI.altIdentifier]{altIdentifier} \hyperref[TEI.msIdentifier]{msIdentifier}
    \item[{Peut contenir}]
  Des données textuelles uniquement
    \item[{Exemple}]
  \leavevmode\bgroup\exampleFont \begin{shaded}\noindent\mbox{}{<\textbf{msIdentifier}>}\mbox{}\newline 
\hspace*{6pt}{<\textbf{country}>}USA{</\textbf{country}>}\mbox{}\newline 
\hspace*{6pt}{<\textbf{region}>}California{</\textbf{region}>}\mbox{}\newline 
\hspace*{6pt}{<\textbf{settlement}>}San Marino{</\textbf{settlement}>}\mbox{}\newline 
\hspace*{6pt}{<\textbf{repository}>}Huntington Library{</\textbf{repository}>}\mbox{}\newline 
\hspace*{6pt}{<\textbf{collection}>}Ellesmere{</\textbf{collection}>}\mbox{}\newline 
\hspace*{6pt}{<\textbf{idno}>}El 26 C 9{</\textbf{idno}>}\mbox{}\newline 
\hspace*{6pt}{<\textbf{msName}>}The Ellesmere Chaucer{</\textbf{msName}>}\mbox{}\newline 
{</\textbf{msIdentifier}>}\end{shaded}\egroup 


    \item[{Modèle de contenu}]
  \fbox{\ttfamily <content>\newline
 <macroRef key="macro.xtext"/>\newline
</content>\newline
    } 
    \item[{Schéma Declaration}]
  \mbox{}\hfill\\[-10pt]\begin{Verbatim}[fontsize=\small]
element collection
{
   tei_att.global.attributes,
   tei_att.naming.attributes,
   tei_att.typed.attributes,
   tei_macro.xtext}
\end{Verbatim}

\end{reflist}  \index{colophon=<colophon>|oddindex}
\begin{reflist}
\item[]\begin{specHead}{TEI.colophon}{<colophon> }(colophon) contient le \textit{colophon} d'une section d'un manuscrit, c'est-à-dire la transcription des informations relatives à la date, au lieu, à l'organisme commanditaire ou aux raisons de la production du manuscrit. [\xref{http://www.tei-c.org/release/doc/tei-p5-doc/en/html/MS.html\#mscoit}{10.6.1. The msItem and msItemStruct Elements}]\end{specHead} 
    \item[{Module}]
  msdescription
    \item[{Attributs}]
  Attributs \hyperref[TEI.att.global]{att.global} (\textit{@xml:id}, \textit{@n}, \textit{@xml:lang}, \textit{@xml:base}, \textit{@xml:space})  (\hyperref[TEI.att.global.rendition]{att.global.rendition} (\textit{@rend}, \textit{@style}, \textit{@rendition})) (\hyperref[TEI.att.global.linking]{att.global.linking} (\textit{@corresp}, \textit{@synch}, \textit{@sameAs}, \textit{@copyOf}, \textit{@next}, \textit{@prev}, \textit{@exclude}, \textit{@select})) (\hyperref[TEI.att.global.analytic]{att.global.analytic} (\textit{@ana})) (\hyperref[TEI.att.global.facs]{att.global.facs} (\textit{@facs})) (\hyperref[TEI.att.global.change]{att.global.change} (\textit{@change})) (\hyperref[TEI.att.global.responsibility]{att.global.responsibility} (\textit{@cert}, \textit{@resp})) (\hyperref[TEI.att.global.source]{att.global.source} (\textit{@source})) \hyperref[TEI.att.msExcerpt]{att.msExcerpt} (\textit{@defective}) 
    \item[{Membre du}]
  \hyperref[TEI.model.msQuoteLike]{model.msQuoteLike} 
    \item[{Contenu dans}]
  
    \item[msdescription: ]
   \hyperref[TEI.msItem]{msItem} \hyperref[TEI.msItemStruct]{msItemStruct}
    \item[{Peut contenir}]
  
    \item[analysis: ]
   \hyperref[TEI.c]{c} \hyperref[TEI.cl]{cl} \hyperref[TEI.interp]{interp} \hyperref[TEI.interpGrp]{interpGrp} \hyperref[TEI.m]{m} \hyperref[TEI.pc]{pc} \hyperref[TEI.phr]{phr} \hyperref[TEI.s]{s} \hyperref[TEI.span]{span} \hyperref[TEI.spanGrp]{spanGrp} \hyperref[TEI.w]{w}\par 
    \item[core: ]
   \hyperref[TEI.abbr]{abbr} \hyperref[TEI.add]{add} \hyperref[TEI.address]{address} \hyperref[TEI.binaryObject]{binaryObject} \hyperref[TEI.cb]{cb} \hyperref[TEI.choice]{choice} \hyperref[TEI.corr]{corr} \hyperref[TEI.date]{date} \hyperref[TEI.del]{del} \hyperref[TEI.distinct]{distinct} \hyperref[TEI.email]{email} \hyperref[TEI.emph]{emph} \hyperref[TEI.expan]{expan} \hyperref[TEI.foreign]{foreign} \hyperref[TEI.gap]{gap} \hyperref[TEI.gb]{gb} \hyperref[TEI.gloss]{gloss} \hyperref[TEI.graphic]{graphic} \hyperref[TEI.hi]{hi} \hyperref[TEI.index]{index} \hyperref[TEI.lb]{lb} \hyperref[TEI.measure]{measure} \hyperref[TEI.measureGrp]{measureGrp} \hyperref[TEI.media]{media} \hyperref[TEI.mentioned]{mentioned} \hyperref[TEI.milestone]{milestone} \hyperref[TEI.name]{name} \hyperref[TEI.note]{note} \hyperref[TEI.num]{num} \hyperref[TEI.orig]{orig} \hyperref[TEI.pb]{pb} \hyperref[TEI.ptr]{ptr} \hyperref[TEI.ref]{ref} \hyperref[TEI.reg]{reg} \hyperref[TEI.rs]{rs} \hyperref[TEI.sic]{sic} \hyperref[TEI.soCalled]{soCalled} \hyperref[TEI.term]{term} \hyperref[TEI.time]{time} \hyperref[TEI.title]{title} \hyperref[TEI.unclear]{unclear}\par 
    \item[derived-module-tei.istex: ]
   \hyperref[TEI.math]{math} \hyperref[TEI.mrow]{mrow}\par 
    \item[figures: ]
   \hyperref[TEI.figure]{figure} \hyperref[TEI.formula]{formula} \hyperref[TEI.notatedMusic]{notatedMusic}\par 
    \item[header: ]
   \hyperref[TEI.idno]{idno}\par 
    \item[iso-fs: ]
   \hyperref[TEI.fLib]{fLib} \hyperref[TEI.fs]{fs} \hyperref[TEI.fvLib]{fvLib}\par 
    \item[linking: ]
   \hyperref[TEI.alt]{alt} \hyperref[TEI.altGrp]{altGrp} \hyperref[TEI.anchor]{anchor} \hyperref[TEI.join]{join} \hyperref[TEI.joinGrp]{joinGrp} \hyperref[TEI.link]{link} \hyperref[TEI.linkGrp]{linkGrp} \hyperref[TEI.seg]{seg} \hyperref[TEI.timeline]{timeline}\par 
    \item[msdescription: ]
   \hyperref[TEI.catchwords]{catchwords} \hyperref[TEI.depth]{depth} \hyperref[TEI.dim]{dim} \hyperref[TEI.dimensions]{dimensions} \hyperref[TEI.height]{height} \hyperref[TEI.heraldry]{heraldry} \hyperref[TEI.locus]{locus} \hyperref[TEI.locusGrp]{locusGrp} \hyperref[TEI.material]{material} \hyperref[TEI.objectType]{objectType} \hyperref[TEI.origDate]{origDate} \hyperref[TEI.origPlace]{origPlace} \hyperref[TEI.secFol]{secFol} \hyperref[TEI.signatures]{signatures} \hyperref[TEI.source]{source} \hyperref[TEI.stamp]{stamp} \hyperref[TEI.watermark]{watermark} \hyperref[TEI.width]{width}\par 
    \item[namesdates: ]
   \hyperref[TEI.addName]{addName} \hyperref[TEI.affiliation]{affiliation} \hyperref[TEI.country]{country} \hyperref[TEI.forename]{forename} \hyperref[TEI.genName]{genName} \hyperref[TEI.geogName]{geogName} \hyperref[TEI.location]{location} \hyperref[TEI.nameLink]{nameLink} \hyperref[TEI.orgName]{orgName} \hyperref[TEI.persName]{persName} \hyperref[TEI.placeName]{placeName} \hyperref[TEI.region]{region} \hyperref[TEI.roleName]{roleName} \hyperref[TEI.settlement]{settlement} \hyperref[TEI.state]{state} \hyperref[TEI.surname]{surname}\par 
    \item[spoken: ]
   \hyperref[TEI.annotationBlock]{annotationBlock}\par 
    \item[transcr: ]
   \hyperref[TEI.addSpan]{addSpan} \hyperref[TEI.am]{am} \hyperref[TEI.damage]{damage} \hyperref[TEI.damageSpan]{damageSpan} \hyperref[TEI.delSpan]{delSpan} \hyperref[TEI.ex]{ex} \hyperref[TEI.fw]{fw} \hyperref[TEI.handShift]{handShift} \hyperref[TEI.listTranspose]{listTranspose} \hyperref[TEI.metamark]{metamark} \hyperref[TEI.mod]{mod} \hyperref[TEI.redo]{redo} \hyperref[TEI.restore]{restore} \hyperref[TEI.retrace]{retrace} \hyperref[TEI.secl]{secl} \hyperref[TEI.space]{space} \hyperref[TEI.subst]{subst} \hyperref[TEI.substJoin]{substJoin} \hyperref[TEI.supplied]{supplied} \hyperref[TEI.surplus]{surplus} \hyperref[TEI.undo]{undo}\par des données textuelles
    \item[{Exemple}]
  \leavevmode\bgroup\exampleFont \begin{shaded}\noindent\mbox{}{<\textbf{colophon}>}Ricardus Franciscus Scripsit Anno Domini\mbox{}\newline 
 1447.{</\textbf{colophon}>}\end{shaded}\egroup 


    \item[{Exemple}]
  \leavevmode\bgroup\exampleFont \begin{shaded}\noindent\mbox{}{<\textbf{colophon}>}Orate pro scriba que scripsit hunc librum : Nomen ejus Elisabeth.{</\textbf{colophon}>}\end{shaded}\egroup 


    \item[{Exemple}]
  \leavevmode\bgroup\exampleFont \begin{shaded}\noindent\mbox{}{<\textbf{colophon}>}Explicit expliceat/scriptor ludere eat.{</\textbf{colophon}>}\end{shaded}\egroup 


    \item[{Exemple}]
  \leavevmode\bgroup\exampleFont \begin{shaded}\noindent\mbox{}{<\textbf{colophon}>}Explicit venenum viciorum domini illius, qui comparavit Anno domini Millessimo\mbox{}\newline 
 Trecentesimo nonagesimo primo, Sabbato in festo sancte Marthe virginis gloriose. Laus tibi\mbox{}\newline 
 criste quia finitur libellus iste.{</\textbf{colophon}>}\end{shaded}\egroup 


    \item[{Exemple}]
  \leavevmode\bgroup\exampleFont \begin{shaded}\noindent\mbox{}{<\textbf{colophon}>}Explicit expliceat/scriptor ludere eat.{</\textbf{colophon}>}\end{shaded}\egroup 


    \item[{Exemple}]
  \leavevmode\bgroup\exampleFont \begin{shaded}\noindent\mbox{}{<\textbf{colophon}>}Explicit venenum viciorum domini illius, qui comparavit Anno\mbox{}\newline 
 domini Millessimo Trecentesimo nonagesimo primo, Sabbato in festo\mbox{}\newline 
 sancte Marthe virginis gloriose. Laus tibi criste quia finitur\mbox{}\newline 
 libellus iste.{</\textbf{colophon}>}\end{shaded}\egroup 


    \item[{Modèle de contenu}]
  \mbox{}\hfill\\[-10pt]\begin{Verbatim}[fontsize=\small]
<content>
 <macroRef key="macro.phraseSeq"/>
</content>
    
\end{Verbatim}

    \item[{Schéma Declaration}]
  \mbox{}\hfill\\[-10pt]\begin{Verbatim}[fontsize=\small]
element colophon
{
   tei_att.global.attributes,
   tei_att.msExcerpt.attributes,
   tei_macro.phraseSeq}
\end{Verbatim}

\end{reflist}  \index{cond=<cond>|oddindex}
\begin{reflist}
\item[]\begin{specHead}{TEI.cond}{<cond> }(contrainte conditionnelle de structure de traits) définit une contrainte conditionnelle de structure de traits ; la conséquence et l'antécédent sont indiqués comme structures de traits ou comme groupes de structures de traits ; la contrainte est satisfaite si à la fois l'antécédent et la conséquence englobent une structure de traits donnée, ou si l'antécédent ne l'englobe pas [\xref{http://www.tei-c.org/release/doc/tei-p5-doc/en/html/FS.html\#FD}{18.11. Feature System Declaration}]\end{specHead} 
    \item[{Module}]
  iso-fs
    \item[{Attributs}]
  Attributs \hyperref[TEI.att.global]{att.global} (\textit{@xml:id}, \textit{@n}, \textit{@xml:lang}, \textit{@xml:base}, \textit{@xml:space})  (\hyperref[TEI.att.global.rendition]{att.global.rendition} (\textit{@rend}, \textit{@style}, \textit{@rendition})) (\hyperref[TEI.att.global.linking]{att.global.linking} (\textit{@corresp}, \textit{@synch}, \textit{@sameAs}, \textit{@copyOf}, \textit{@next}, \textit{@prev}, \textit{@exclude}, \textit{@select})) (\hyperref[TEI.att.global.analytic]{att.global.analytic} (\textit{@ana})) (\hyperref[TEI.att.global.facs]{att.global.facs} (\textit{@facs})) (\hyperref[TEI.att.global.change]{att.global.change} (\textit{@change})) (\hyperref[TEI.att.global.responsibility]{att.global.responsibility} (\textit{@cert}, \textit{@resp})) (\hyperref[TEI.att.global.source]{att.global.source} (\textit{@source}))
    \item[{Contenu dans}]
  
    \item[iso-fs: ]
   \hyperref[TEI.fsConstraints]{fsConstraints}
    \item[{Peut contenir}]
  
    \item[iso-fs: ]
   \hyperref[TEI.f]{f} \hyperref[TEI.fs]{fs} \hyperref[TEI.then]{then}
    \item[{Note}]
  \par
Peut contenir une structure de traits précédente, un élément vide \hyperref[TEI.then]{<then>} et une structure de traits suivante.
    \item[{Exemple}]
  \leavevmode\bgroup\exampleFont \begin{shaded}\noindent\mbox{}{<\textbf{cond}>}\mbox{}\newline 
\hspace*{6pt}{<\textbf{fs}>}\mbox{}\newline 
\hspace*{6pt}\hspace*{6pt}{<\textbf{f}\hspace*{6pt}{name}="{BAR}">}\mbox{}\newline 
\hspace*{6pt}\hspace*{6pt}\hspace*{6pt}{<\textbf{symbol}\hspace*{6pt}{value}="{1}"/>}\mbox{}\newline 
\hspace*{6pt}\hspace*{6pt}{</\textbf{f}>}\mbox{}\newline 
\hspace*{6pt}{</\textbf{fs}>}\mbox{}\newline 
\hspace*{6pt}{<\textbf{then}/>}\mbox{}\newline 
\hspace*{6pt}{<\textbf{fs}>}\mbox{}\newline 
\hspace*{6pt}\hspace*{6pt}{<\textbf{f}\hspace*{6pt}{name}="{SUBCAT}">}\mbox{}\newline 
\hspace*{6pt}\hspace*{6pt}\hspace*{6pt}{<\textbf{binary}\hspace*{6pt}{value}="{false}"/>}\mbox{}\newline 
\hspace*{6pt}\hspace*{6pt}{</\textbf{f}>}\mbox{}\newline 
\hspace*{6pt}{</\textbf{fs}>}\mbox{}\newline 
{</\textbf{cond}>}\end{shaded}\egroup 


    \item[{Modèle de contenu}]
  \mbox{}\hfill\\[-10pt]\begin{Verbatim}[fontsize=\small]
<content>
 <sequence maxOccurs="1" minOccurs="1">
  <alternate maxOccurs="1" minOccurs="1">
   <elementRef key="fs"/>
   <elementRef key="f"/>
  </alternate>
  <elementRef key="then"/>
  <alternate maxOccurs="1" minOccurs="1">
   <elementRef key="fs"/>
   <elementRef key="f"/>
  </alternate>
 </sequence>
</content>
    
\end{Verbatim}

    \item[{Schéma Declaration}]
  \mbox{}\hfill\\[-10pt]\begin{Verbatim}[fontsize=\small]
element cond
{
   tei_att.global.attributes,
   ( ( tei_fs | tei_f ), tei_then, ( tei_fs | tei_f ) )
}
\end{Verbatim}

\end{reflist}  \index{condition=<condition>|oddindex}
\begin{reflist}
\item[]\begin{specHead}{TEI.condition}{<condition> }(état matériel) contient la description de l'état matériel du manuscrit. [\xref{http://www.tei-c.org/release/doc/tei-p5-doc/en/html/MS.html\#msphco}{10.7.1.5. Condition}]\end{specHead} 
    \item[{Module}]
  msdescription
    \item[{Attributs}]
  Attributs \hyperref[TEI.att.global]{att.global} (\textit{@xml:id}, \textit{@n}, \textit{@xml:lang}, \textit{@xml:base}, \textit{@xml:space})  (\hyperref[TEI.att.global.rendition]{att.global.rendition} (\textit{@rend}, \textit{@style}, \textit{@rendition})) (\hyperref[TEI.att.global.linking]{att.global.linking} (\textit{@corresp}, \textit{@synch}, \textit{@sameAs}, \textit{@copyOf}, \textit{@next}, \textit{@prev}, \textit{@exclude}, \textit{@select})) (\hyperref[TEI.att.global.analytic]{att.global.analytic} (\textit{@ana})) (\hyperref[TEI.att.global.facs]{att.global.facs} (\textit{@facs})) (\hyperref[TEI.att.global.change]{att.global.change} (\textit{@change})) (\hyperref[TEI.att.global.responsibility]{att.global.responsibility} (\textit{@cert}, \textit{@resp})) (\hyperref[TEI.att.global.source]{att.global.source} (\textit{@source}))
    \item[{Contenu dans}]
  
    \item[msdescription: ]
   \hyperref[TEI.binding]{binding} \hyperref[TEI.bindingDesc]{bindingDesc} \hyperref[TEI.sealDesc]{sealDesc} \hyperref[TEI.supportDesc]{supportDesc}
    \item[{Peut contenir}]
  
    \item[analysis: ]
   \hyperref[TEI.c]{c} \hyperref[TEI.cl]{cl} \hyperref[TEI.interp]{interp} \hyperref[TEI.interpGrp]{interpGrp} \hyperref[TEI.m]{m} \hyperref[TEI.pc]{pc} \hyperref[TEI.phr]{phr} \hyperref[TEI.s]{s} \hyperref[TEI.span]{span} \hyperref[TEI.spanGrp]{spanGrp} \hyperref[TEI.w]{w}\par 
    \item[core: ]
   \hyperref[TEI.abbr]{abbr} \hyperref[TEI.add]{add} \hyperref[TEI.address]{address} \hyperref[TEI.bibl]{bibl} \hyperref[TEI.biblStruct]{biblStruct} \hyperref[TEI.binaryObject]{binaryObject} \hyperref[TEI.cb]{cb} \hyperref[TEI.choice]{choice} \hyperref[TEI.cit]{cit} \hyperref[TEI.corr]{corr} \hyperref[TEI.date]{date} \hyperref[TEI.del]{del} \hyperref[TEI.desc]{desc} \hyperref[TEI.distinct]{distinct} \hyperref[TEI.email]{email} \hyperref[TEI.emph]{emph} \hyperref[TEI.expan]{expan} \hyperref[TEI.foreign]{foreign} \hyperref[TEI.gap]{gap} \hyperref[TEI.gb]{gb} \hyperref[TEI.gloss]{gloss} \hyperref[TEI.graphic]{graphic} \hyperref[TEI.hi]{hi} \hyperref[TEI.index]{index} \hyperref[TEI.l]{l} \hyperref[TEI.label]{label} \hyperref[TEI.lb]{lb} \hyperref[TEI.lg]{lg} \hyperref[TEI.list]{list} \hyperref[TEI.listBibl]{listBibl} \hyperref[TEI.measure]{measure} \hyperref[TEI.measureGrp]{measureGrp} \hyperref[TEI.media]{media} \hyperref[TEI.mentioned]{mentioned} \hyperref[TEI.milestone]{milestone} \hyperref[TEI.name]{name} \hyperref[TEI.note]{note} \hyperref[TEI.num]{num} \hyperref[TEI.orig]{orig} \hyperref[TEI.p]{p} \hyperref[TEI.pb]{pb} \hyperref[TEI.ptr]{ptr} \hyperref[TEI.q]{q} \hyperref[TEI.quote]{quote} \hyperref[TEI.ref]{ref} \hyperref[TEI.reg]{reg} \hyperref[TEI.rs]{rs} \hyperref[TEI.said]{said} \hyperref[TEI.sic]{sic} \hyperref[TEI.soCalled]{soCalled} \hyperref[TEI.sp]{sp} \hyperref[TEI.stage]{stage} \hyperref[TEI.term]{term} \hyperref[TEI.time]{time} \hyperref[TEI.title]{title} \hyperref[TEI.unclear]{unclear}\par 
    \item[derived-module-tei.istex: ]
   \hyperref[TEI.math]{math} \hyperref[TEI.mrow]{mrow}\par 
    \item[figures: ]
   \hyperref[TEI.figure]{figure} \hyperref[TEI.formula]{formula} \hyperref[TEI.notatedMusic]{notatedMusic} \hyperref[TEI.table]{table}\par 
    \item[header: ]
   \hyperref[TEI.biblFull]{biblFull} \hyperref[TEI.idno]{idno}\par 
    \item[iso-fs: ]
   \hyperref[TEI.fLib]{fLib} \hyperref[TEI.fs]{fs} \hyperref[TEI.fvLib]{fvLib}\par 
    \item[linking: ]
   \hyperref[TEI.ab]{ab} \hyperref[TEI.alt]{alt} \hyperref[TEI.altGrp]{altGrp} \hyperref[TEI.anchor]{anchor} \hyperref[TEI.join]{join} \hyperref[TEI.joinGrp]{joinGrp} \hyperref[TEI.link]{link} \hyperref[TEI.linkGrp]{linkGrp} \hyperref[TEI.seg]{seg} \hyperref[TEI.timeline]{timeline}\par 
    \item[msdescription: ]
   \hyperref[TEI.catchwords]{catchwords} \hyperref[TEI.depth]{depth} \hyperref[TEI.dim]{dim} \hyperref[TEI.dimensions]{dimensions} \hyperref[TEI.height]{height} \hyperref[TEI.heraldry]{heraldry} \hyperref[TEI.locus]{locus} \hyperref[TEI.locusGrp]{locusGrp} \hyperref[TEI.material]{material} \hyperref[TEI.msDesc]{msDesc} \hyperref[TEI.objectType]{objectType} \hyperref[TEI.origDate]{origDate} \hyperref[TEI.origPlace]{origPlace} \hyperref[TEI.secFol]{secFol} \hyperref[TEI.signatures]{signatures} \hyperref[TEI.source]{source} \hyperref[TEI.stamp]{stamp} \hyperref[TEI.watermark]{watermark} \hyperref[TEI.width]{width}\par 
    \item[namesdates: ]
   \hyperref[TEI.addName]{addName} \hyperref[TEI.affiliation]{affiliation} \hyperref[TEI.country]{country} \hyperref[TEI.forename]{forename} \hyperref[TEI.genName]{genName} \hyperref[TEI.geogName]{geogName} \hyperref[TEI.listOrg]{listOrg} \hyperref[TEI.listPlace]{listPlace} \hyperref[TEI.location]{location} \hyperref[TEI.nameLink]{nameLink} \hyperref[TEI.orgName]{orgName} \hyperref[TEI.persName]{persName} \hyperref[TEI.placeName]{placeName} \hyperref[TEI.region]{region} \hyperref[TEI.roleName]{roleName} \hyperref[TEI.settlement]{settlement} \hyperref[TEI.state]{state} \hyperref[TEI.surname]{surname}\par 
    \item[spoken: ]
   \hyperref[TEI.annotationBlock]{annotationBlock}\par 
    \item[textstructure: ]
   \hyperref[TEI.floatingText]{floatingText}\par 
    \item[transcr: ]
   \hyperref[TEI.addSpan]{addSpan} \hyperref[TEI.am]{am} \hyperref[TEI.damage]{damage} \hyperref[TEI.damageSpan]{damageSpan} \hyperref[TEI.delSpan]{delSpan} \hyperref[TEI.ex]{ex} \hyperref[TEI.fw]{fw} \hyperref[TEI.handShift]{handShift} \hyperref[TEI.listTranspose]{listTranspose} \hyperref[TEI.metamark]{metamark} \hyperref[TEI.mod]{mod} \hyperref[TEI.redo]{redo} \hyperref[TEI.restore]{restore} \hyperref[TEI.retrace]{retrace} \hyperref[TEI.secl]{secl} \hyperref[TEI.space]{space} \hyperref[TEI.subst]{subst} \hyperref[TEI.substJoin]{substJoin} \hyperref[TEI.supplied]{supplied} \hyperref[TEI.surplus]{surplus} \hyperref[TEI.undo]{undo}\par des données textuelles
    \item[{Exemple}]
  \leavevmode\bgroup\exampleFont \begin{shaded}\noindent\mbox{}{<\textbf{condition}>} Traces de mouillures anciennes plus ou moins importantes au bas des feuillets,\mbox{}\newline 
 qui n'ont pas affecté la reliure ; éraflure en tête du plat inférieur. {</\textbf{condition}>}\mbox{}\newline 
{<\textbf{condition}>}Eraflures sur les deux plats, tache d'humidité dans la partie supérieure du plat\mbox{}\newline 
 inférieur ; mors fendus en tête et en queue avec zones restaurées (minces bandes de\mbox{}\newline 
 maroquin).{</\textbf{condition}>}\end{shaded}\egroup 


    \item[{Modèle de contenu}]
  \mbox{}\hfill\\[-10pt]\begin{Verbatim}[fontsize=\small]
<content>
 <macroRef key="macro.specialPara"/>
</content>
    
\end{Verbatim}

    \item[{Schéma Declaration}]
  \mbox{}\hfill\\[-10pt]\begin{Verbatim}[fontsize=\small]
element condition { tei_att.global.attributes, tei_macro.specialPara }
\end{Verbatim}

\end{reflist}  \index{corr=<corr>|oddindex}
\begin{reflist}
\item[]\begin{specHead}{TEI.corr}{<corr> }(correction) contient la forme correcte d'un passage qui est considéré erroné dans la copie du texte. [\xref{http://www.tei-c.org/release/doc/tei-p5-doc/en/html/CO.html\#COEDCOR}{3.4.1. Apparent Errors}]\end{specHead} 
    \item[{Module}]
  core
    \item[{Attributs}]
  Attributs \hyperref[TEI.att.global]{att.global} (\textit{@xml:id}, \textit{@n}, \textit{@xml:lang}, \textit{@xml:base}, \textit{@xml:space})  (\hyperref[TEI.att.global.rendition]{att.global.rendition} (\textit{@rend}, \textit{@style}, \textit{@rendition})) (\hyperref[TEI.att.global.linking]{att.global.linking} (\textit{@corresp}, \textit{@synch}, \textit{@sameAs}, \textit{@copyOf}, \textit{@next}, \textit{@prev}, \textit{@exclude}, \textit{@select})) (\hyperref[TEI.att.global.analytic]{att.global.analytic} (\textit{@ana})) (\hyperref[TEI.att.global.facs]{att.global.facs} (\textit{@facs})) (\hyperref[TEI.att.global.change]{att.global.change} (\textit{@change})) (\hyperref[TEI.att.global.responsibility]{att.global.responsibility} (\textit{@cert}, \textit{@resp})) (\hyperref[TEI.att.global.source]{att.global.source} (\textit{@source})) \hyperref[TEI.att.editLike]{att.editLike} (\textit{@evidence}, \textit{@instant})  (\hyperref[TEI.att.dimensions]{att.dimensions} (\textit{@unit}, \textit{@quantity}, \textit{@extent}, \textit{@precision}, \textit{@scope}) (\hyperref[TEI.att.ranging]{att.ranging} (\textit{@atLeast}, \textit{@atMost}, \textit{@min}, \textit{@max}, \textit{@confidence})) ) \hyperref[TEI.att.typed]{att.typed} (\textit{@type}, \textit{@subtype}) 
    \item[{Membre du}]
  \hyperref[TEI.model.choicePart]{model.choicePart} \hyperref[TEI.model.pPart.transcriptional]{model.pPart.transcriptional}
    \item[{Contenu dans}]
  
    \item[analysis: ]
   \hyperref[TEI.cl]{cl} \hyperref[TEI.pc]{pc} \hyperref[TEI.phr]{phr} \hyperref[TEI.s]{s} \hyperref[TEI.w]{w}\par 
    \item[core: ]
   \hyperref[TEI.abbr]{abbr} \hyperref[TEI.add]{add} \hyperref[TEI.addrLine]{addrLine} \hyperref[TEI.author]{author} \hyperref[TEI.bibl]{bibl} \hyperref[TEI.biblScope]{biblScope} \hyperref[TEI.choice]{choice} \hyperref[TEI.citedRange]{citedRange} \hyperref[TEI.corr]{corr} \hyperref[TEI.date]{date} \hyperref[TEI.del]{del} \hyperref[TEI.distinct]{distinct} \hyperref[TEI.editor]{editor} \hyperref[TEI.email]{email} \hyperref[TEI.emph]{emph} \hyperref[TEI.expan]{expan} \hyperref[TEI.foreign]{foreign} \hyperref[TEI.gloss]{gloss} \hyperref[TEI.head]{head} \hyperref[TEI.headItem]{headItem} \hyperref[TEI.headLabel]{headLabel} \hyperref[TEI.hi]{hi} \hyperref[TEI.item]{item} \hyperref[TEI.l]{l} \hyperref[TEI.label]{label} \hyperref[TEI.measure]{measure} \hyperref[TEI.mentioned]{mentioned} \hyperref[TEI.name]{name} \hyperref[TEI.note]{note} \hyperref[TEI.num]{num} \hyperref[TEI.orig]{orig} \hyperref[TEI.p]{p} \hyperref[TEI.pubPlace]{pubPlace} \hyperref[TEI.publisher]{publisher} \hyperref[TEI.q]{q} \hyperref[TEI.quote]{quote} \hyperref[TEI.ref]{ref} \hyperref[TEI.reg]{reg} \hyperref[TEI.rs]{rs} \hyperref[TEI.said]{said} \hyperref[TEI.sic]{sic} \hyperref[TEI.soCalled]{soCalled} \hyperref[TEI.speaker]{speaker} \hyperref[TEI.stage]{stage} \hyperref[TEI.street]{street} \hyperref[TEI.term]{term} \hyperref[TEI.textLang]{textLang} \hyperref[TEI.time]{time} \hyperref[TEI.title]{title} \hyperref[TEI.unclear]{unclear}\par 
    \item[figures: ]
   \hyperref[TEI.cell]{cell}\par 
    \item[header: ]
   \hyperref[TEI.change]{change} \hyperref[TEI.distributor]{distributor} \hyperref[TEI.edition]{edition} \hyperref[TEI.extent]{extent} \hyperref[TEI.licence]{licence}\par 
    \item[linking: ]
   \hyperref[TEI.ab]{ab} \hyperref[TEI.seg]{seg}\par 
    \item[msdescription: ]
   \hyperref[TEI.accMat]{accMat} \hyperref[TEI.acquisition]{acquisition} \hyperref[TEI.additions]{additions} \hyperref[TEI.catchwords]{catchwords} \hyperref[TEI.collation]{collation} \hyperref[TEI.colophon]{colophon} \hyperref[TEI.condition]{condition} \hyperref[TEI.custEvent]{custEvent} \hyperref[TEI.decoNote]{decoNote} \hyperref[TEI.explicit]{explicit} \hyperref[TEI.filiation]{filiation} \hyperref[TEI.finalRubric]{finalRubric} \hyperref[TEI.foliation]{foliation} \hyperref[TEI.heraldry]{heraldry} \hyperref[TEI.incipit]{incipit} \hyperref[TEI.layout]{layout} \hyperref[TEI.material]{material} \hyperref[TEI.musicNotation]{musicNotation} \hyperref[TEI.objectType]{objectType} \hyperref[TEI.origDate]{origDate} \hyperref[TEI.origPlace]{origPlace} \hyperref[TEI.origin]{origin} \hyperref[TEI.provenance]{provenance} \hyperref[TEI.rubric]{rubric} \hyperref[TEI.secFol]{secFol} \hyperref[TEI.signatures]{signatures} \hyperref[TEI.source]{source} \hyperref[TEI.stamp]{stamp} \hyperref[TEI.summary]{summary} \hyperref[TEI.support]{support} \hyperref[TEI.surrogates]{surrogates} \hyperref[TEI.typeNote]{typeNote} \hyperref[TEI.watermark]{watermark}\par 
    \item[namesdates: ]
   \hyperref[TEI.addName]{addName} \hyperref[TEI.affiliation]{affiliation} \hyperref[TEI.country]{country} \hyperref[TEI.forename]{forename} \hyperref[TEI.genName]{genName} \hyperref[TEI.geogName]{geogName} \hyperref[TEI.nameLink]{nameLink} \hyperref[TEI.orgName]{orgName} \hyperref[TEI.persName]{persName} \hyperref[TEI.placeName]{placeName} \hyperref[TEI.region]{region} \hyperref[TEI.roleName]{roleName} \hyperref[TEI.settlement]{settlement} \hyperref[TEI.surname]{surname}\par 
    \item[textstructure: ]
   \hyperref[TEI.docAuthor]{docAuthor} \hyperref[TEI.docDate]{docDate} \hyperref[TEI.docEdition]{docEdition} \hyperref[TEI.titlePart]{titlePart}\par 
    \item[transcr: ]
   \hyperref[TEI.am]{am} \hyperref[TEI.damage]{damage} \hyperref[TEI.fw]{fw} \hyperref[TEI.metamark]{metamark} \hyperref[TEI.mod]{mod} \hyperref[TEI.restore]{restore} \hyperref[TEI.retrace]{retrace} \hyperref[TEI.secl]{secl} \hyperref[TEI.supplied]{supplied} \hyperref[TEI.surplus]{surplus}
    \item[{Peut contenir}]
  
    \item[analysis: ]
   \hyperref[TEI.c]{c} \hyperref[TEI.cl]{cl} \hyperref[TEI.interp]{interp} \hyperref[TEI.interpGrp]{interpGrp} \hyperref[TEI.m]{m} \hyperref[TEI.pc]{pc} \hyperref[TEI.phr]{phr} \hyperref[TEI.s]{s} \hyperref[TEI.span]{span} \hyperref[TEI.spanGrp]{spanGrp} \hyperref[TEI.w]{w}\par 
    \item[core: ]
   \hyperref[TEI.abbr]{abbr} \hyperref[TEI.add]{add} \hyperref[TEI.address]{address} \hyperref[TEI.bibl]{bibl} \hyperref[TEI.biblStruct]{biblStruct} \hyperref[TEI.binaryObject]{binaryObject} \hyperref[TEI.cb]{cb} \hyperref[TEI.choice]{choice} \hyperref[TEI.cit]{cit} \hyperref[TEI.corr]{corr} \hyperref[TEI.date]{date} \hyperref[TEI.del]{del} \hyperref[TEI.desc]{desc} \hyperref[TEI.distinct]{distinct} \hyperref[TEI.email]{email} \hyperref[TEI.emph]{emph} \hyperref[TEI.expan]{expan} \hyperref[TEI.foreign]{foreign} \hyperref[TEI.gap]{gap} \hyperref[TEI.gb]{gb} \hyperref[TEI.gloss]{gloss} \hyperref[TEI.graphic]{graphic} \hyperref[TEI.hi]{hi} \hyperref[TEI.index]{index} \hyperref[TEI.l]{l} \hyperref[TEI.label]{label} \hyperref[TEI.lb]{lb} \hyperref[TEI.lg]{lg} \hyperref[TEI.list]{list} \hyperref[TEI.listBibl]{listBibl} \hyperref[TEI.measure]{measure} \hyperref[TEI.measureGrp]{measureGrp} \hyperref[TEI.media]{media} \hyperref[TEI.mentioned]{mentioned} \hyperref[TEI.milestone]{milestone} \hyperref[TEI.name]{name} \hyperref[TEI.note]{note} \hyperref[TEI.num]{num} \hyperref[TEI.orig]{orig} \hyperref[TEI.pb]{pb} \hyperref[TEI.ptr]{ptr} \hyperref[TEI.q]{q} \hyperref[TEI.quote]{quote} \hyperref[TEI.ref]{ref} \hyperref[TEI.reg]{reg} \hyperref[TEI.rs]{rs} \hyperref[TEI.said]{said} \hyperref[TEI.sic]{sic} \hyperref[TEI.soCalled]{soCalled} \hyperref[TEI.stage]{stage} \hyperref[TEI.term]{term} \hyperref[TEI.time]{time} \hyperref[TEI.title]{title} \hyperref[TEI.unclear]{unclear}\par 
    \item[derived-module-tei.istex: ]
   \hyperref[TEI.math]{math} \hyperref[TEI.mrow]{mrow}\par 
    \item[figures: ]
   \hyperref[TEI.figure]{figure} \hyperref[TEI.formula]{formula} \hyperref[TEI.notatedMusic]{notatedMusic} \hyperref[TEI.table]{table}\par 
    \item[header: ]
   \hyperref[TEI.biblFull]{biblFull} \hyperref[TEI.idno]{idno}\par 
    \item[iso-fs: ]
   \hyperref[TEI.fLib]{fLib} \hyperref[TEI.fs]{fs} \hyperref[TEI.fvLib]{fvLib}\par 
    \item[linking: ]
   \hyperref[TEI.alt]{alt} \hyperref[TEI.altGrp]{altGrp} \hyperref[TEI.anchor]{anchor} \hyperref[TEI.join]{join} \hyperref[TEI.joinGrp]{joinGrp} \hyperref[TEI.link]{link} \hyperref[TEI.linkGrp]{linkGrp} \hyperref[TEI.seg]{seg} \hyperref[TEI.timeline]{timeline}\par 
    \item[msdescription: ]
   \hyperref[TEI.catchwords]{catchwords} \hyperref[TEI.depth]{depth} \hyperref[TEI.dim]{dim} \hyperref[TEI.dimensions]{dimensions} \hyperref[TEI.height]{height} \hyperref[TEI.heraldry]{heraldry} \hyperref[TEI.locus]{locus} \hyperref[TEI.locusGrp]{locusGrp} \hyperref[TEI.material]{material} \hyperref[TEI.msDesc]{msDesc} \hyperref[TEI.objectType]{objectType} \hyperref[TEI.origDate]{origDate} \hyperref[TEI.origPlace]{origPlace} \hyperref[TEI.secFol]{secFol} \hyperref[TEI.signatures]{signatures} \hyperref[TEI.source]{source} \hyperref[TEI.stamp]{stamp} \hyperref[TEI.watermark]{watermark} \hyperref[TEI.width]{width}\par 
    \item[namesdates: ]
   \hyperref[TEI.addName]{addName} \hyperref[TEI.affiliation]{affiliation} \hyperref[TEI.country]{country} \hyperref[TEI.forename]{forename} \hyperref[TEI.genName]{genName} \hyperref[TEI.geogName]{geogName} \hyperref[TEI.listOrg]{listOrg} \hyperref[TEI.listPlace]{listPlace} \hyperref[TEI.location]{location} \hyperref[TEI.nameLink]{nameLink} \hyperref[TEI.orgName]{orgName} \hyperref[TEI.persName]{persName} \hyperref[TEI.placeName]{placeName} \hyperref[TEI.region]{region} \hyperref[TEI.roleName]{roleName} \hyperref[TEI.settlement]{settlement} \hyperref[TEI.state]{state} \hyperref[TEI.surname]{surname}\par 
    \item[spoken: ]
   \hyperref[TEI.annotationBlock]{annotationBlock}\par 
    \item[textstructure: ]
   \hyperref[TEI.floatingText]{floatingText}\par 
    \item[transcr: ]
   \hyperref[TEI.addSpan]{addSpan} \hyperref[TEI.am]{am} \hyperref[TEI.damage]{damage} \hyperref[TEI.damageSpan]{damageSpan} \hyperref[TEI.delSpan]{delSpan} \hyperref[TEI.ex]{ex} \hyperref[TEI.fw]{fw} \hyperref[TEI.handShift]{handShift} \hyperref[TEI.listTranspose]{listTranspose} \hyperref[TEI.metamark]{metamark} \hyperref[TEI.mod]{mod} \hyperref[TEI.redo]{redo} \hyperref[TEI.restore]{restore} \hyperref[TEI.retrace]{retrace} \hyperref[TEI.secl]{secl} \hyperref[TEI.space]{space} \hyperref[TEI.subst]{subst} \hyperref[TEI.substJoin]{substJoin} \hyperref[TEI.supplied]{supplied} \hyperref[TEI.surplus]{surplus} \hyperref[TEI.undo]{undo}\par des données textuelles
    \item[{Exemple}]
  Si l'on veut mettre l'accent sur le fait que le texte a été corrigé, \hyperref[TEI.corr]{<corr>} seul sera employé:\leavevmode\bgroup\exampleFont \begin{shaded}\noindent\mbox{}Tel est le\mbox{}\newline 
 chat Rutterkin des sorcières Margaret et Filippa Flower, qui\mbox{}\newline 
 furent {<\textbf{corr}>}brûlées{</\textbf{corr}>} à Lincoln, le 11 mars 1619, pour avoir envoûté un parent du comte de\mbox{}\newline 
 Rutland.\end{shaded}\egroup 


    \item[{Exemple}]
  Il est aussi possible d'associer \hyperref[TEI.choice]{<choice>} et\hyperref[TEI.sic]{<sic>}, pour donner une lecture incorrecte :\leavevmode\bgroup\exampleFont \begin{shaded}\noindent\mbox{}Tel est le\mbox{}\newline 
 chat Rutterkin des sorcières Margaret et Filippa Flower, qui furent{<\textbf{choice}>}\mbox{}\newline 
\hspace*{6pt}{<\textbf{sic}>}prûlées{</\textbf{sic}>}\mbox{}\newline 
\hspace*{6pt}{<\textbf{corr}>}brûlées{</\textbf{corr}>}\mbox{}\newline 
{</\textbf{choice}>} à Lincoln, le 11 mars 1619, pour avoir envoûté un parent du comte de\mbox{}\newline 
 Rutland.\end{shaded}\egroup 


    \item[{Modèle de contenu}]
  \mbox{}\hfill\\[-10pt]\begin{Verbatim}[fontsize=\small]
<content>
 <macroRef key="macro.paraContent"/>
</content>
    
\end{Verbatim}

    \item[{Schéma Declaration}]
  \mbox{}\hfill\\[-10pt]\begin{Verbatim}[fontsize=\small]
element corr
{
   tei_att.global.attributes,
   tei_att.editLike.attributes,
   tei_att.typed.attributes,
   tei_macro.paraContent}
\end{Verbatim}

\end{reflist}  \index{correction=<correction>|oddindex}\index{status=@status!<correction>|oddindex}\index{method=@method!<correction>|oddindex}
\begin{reflist}
\item[]\begin{specHead}{TEI.correction}{<correction> }(règles de correction) établit comment et dans quelles circonstances des corrections ont été apportées au texte. [\xref{http://www.tei-c.org/release/doc/tei-p5-doc/en/html/HD.html\#HD53}{2.3.3. The Editorial Practices Declaration} \xref{http://www.tei-c.org/release/doc/tei-p5-doc/en/html/CC.html\#CCAS2}{15.3.2. Declarable Elements}]\end{specHead} 
    \item[{Module}]
  header
    \item[{Attributs}]
  Attributs \hyperref[TEI.att.global]{att.global} (\textit{@xml:id}, \textit{@n}, \textit{@xml:lang}, \textit{@xml:base}, \textit{@xml:space})  (\hyperref[TEI.att.global.rendition]{att.global.rendition} (\textit{@rend}, \textit{@style}, \textit{@rendition})) (\hyperref[TEI.att.global.linking]{att.global.linking} (\textit{@corresp}, \textit{@synch}, \textit{@sameAs}, \textit{@copyOf}, \textit{@next}, \textit{@prev}, \textit{@exclude}, \textit{@select})) (\hyperref[TEI.att.global.analytic]{att.global.analytic} (\textit{@ana})) (\hyperref[TEI.att.global.facs]{att.global.facs} (\textit{@facs})) (\hyperref[TEI.att.global.change]{att.global.change} (\textit{@change})) (\hyperref[TEI.att.global.responsibility]{att.global.responsibility} (\textit{@cert}, \textit{@resp})) (\hyperref[TEI.att.global.source]{att.global.source} (\textit{@source})) \hyperref[TEI.att.declarable]{att.declarable} (\textit{@default}) \hfil\\[-10pt]\begin{sansreflist}
    \item[@status]
  indique le degré de correction apporté au texte.
\begin{reflist}
    \item[{Statut}]
  Optionel
    \item[{Type de données}]
  \hyperref[TEI.teidata.enumerated]{teidata.enumerated}
    \item[{Les valeurs autorisées sont:}]
  \begin{description}

\item[{high}]le texte a été entièrement vérifié et corrigé.
\item[{medium}]le texte a au moins été vérifié une fois.
\item[{low}]le texte n’a pas été vérifié.
\item[{unknown}]le niveau de correction du texte est inconnu.{[Valeur par défaut] \xref{http://www.tei-c.org/Activities/Council/Working/tcw27.xml}{Deprecated}. The value will no longer be a default after 2017-09-05.}
\end{description} 
\end{reflist}  
    \item[@method]
  indique la méthode adoptée pour signaler les corrections dans le texte.
\begin{reflist}
    \item[{Statut}]
  Optionel
    \item[{Type de données}]
  \hyperref[TEI.teidata.enumerated]{teidata.enumerated}
    \item[{Les valeurs autorisées sont:}]
  \begin{description}

\item[{silent}]les corrections ont été faites sans être marquées.{[Valeur par défaut] }
\item[{markup}]les corrections ont été notées par un codage
\end{description} 
\end{reflist}  
\end{sansreflist}  
    \item[{Contenu dans}]
  —
    \item[{Peut contenir}]
  
    \item[core: ]
   \hyperref[TEI.p]{p}\par 
    \item[linking: ]
   \hyperref[TEI.ab]{ab}
    \item[{Note}]
  \par
Utilisé pour noter le résultat de la comparaison du texte et de l'original en indiquant par exemple si les différences ont été faites sans être marquées, ou si elles ont été marquées en utilisant les balises éditoriales décrites dans la section \xref{http://www.tei-c.org/release/doc/tei-p5-doc/en/html/CO.html\#COED}{3.4. Simple Editorial Changes}.
    \item[{Exemple}]
  \leavevmode\bgroup\exampleFont \begin{shaded}\noindent\mbox{}{<\textbf{correction}>}\mbox{}\newline 
\hspace*{6pt}{<\textbf{p}>}Les erreurs de transcriptions ont été détectées et corrigées à l'aide du correcteur\mbox{}\newline 
\hspace*{6pt}\hspace*{6pt} Cordial 2006 - Synapse{</\textbf{p}>}\mbox{}\newline 
{</\textbf{correction}>}\end{shaded}\egroup 


    \item[{Modèle de contenu}]
  \mbox{}\hfill\\[-10pt]\begin{Verbatim}[fontsize=\small]
<content>
 <classRef key="model.pLike"
  maxOccurs="unbounded" minOccurs="1"/>
</content>
    
\end{Verbatim}

    \item[{Schéma Declaration}]
  \mbox{}\hfill\\[-10pt]\begin{Verbatim}[fontsize=\small]
element correction
{
   tei_att.global.attributes,
   tei_att.declarable.attributes,
   attribute status { "high" | "medium" | "low" | "unknown" }?,
   attribute method { "silent" | "markup" }?,
   tei_model.pLike+
}
\end{Verbatim}

\end{reflist}  \index{country=<country>|oddindex}
\begin{reflist}
\item[]\begin{specHead}{TEI.country}{<country> }(pays) contient le nom d'une unité géo-politique, comme une nation, un pays, une colonie ou une communauté, plus grande ou administrativement supérieure à une région et plus petite qu'un bloc. [\xref{http://www.tei-c.org/release/doc/tei-p5-doc/en/html/ND.html\#NDPLAC}{13.2.3. Place Names}]\end{specHead} 
    \item[{Module}]
  namesdates
    \item[{Attributs}]
  Attributs \hyperref[TEI.att.global]{att.global} (\textit{@xml:id}, \textit{@n}, \textit{@xml:lang}, \textit{@xml:base}, \textit{@xml:space})  (\hyperref[TEI.att.global.rendition]{att.global.rendition} (\textit{@rend}, \textit{@style}, \textit{@rendition})) (\hyperref[TEI.att.global.linking]{att.global.linking} (\textit{@corresp}, \textit{@synch}, \textit{@sameAs}, \textit{@copyOf}, \textit{@next}, \textit{@prev}, \textit{@exclude}, \textit{@select})) (\hyperref[TEI.att.global.analytic]{att.global.analytic} (\textit{@ana})) (\hyperref[TEI.att.global.facs]{att.global.facs} (\textit{@facs})) (\hyperref[TEI.att.global.change]{att.global.change} (\textit{@change})) (\hyperref[TEI.att.global.responsibility]{att.global.responsibility} (\textit{@cert}, \textit{@resp})) (\hyperref[TEI.att.global.source]{att.global.source} (\textit{@source})) \hyperref[TEI.att.naming]{att.naming} (\textit{@role}, \textit{@nymRef})  (\hyperref[TEI.att.canonical]{att.canonical} (\textit{@key}, \textit{@ref})) \hyperref[TEI.att.typed]{att.typed} (\textit{@type}, \textit{@subtype}) \hyperref[TEI.att.datable]{att.datable} (\textit{@calendar}, \textit{@period})  (\hyperref[TEI.att.datable.w3c]{att.datable.w3c} (\textit{@when}, \textit{@notBefore}, \textit{@notAfter}, \textit{@from}, \textit{@to})) (\hyperref[TEI.att.datable.iso]{att.datable.iso} (\textit{@when-iso}, \textit{@notBefore-iso}, \textit{@notAfter-iso}, \textit{@from-iso}, \textit{@to-iso})) (\hyperref[TEI.att.datable.custom]{att.datable.custom} (\textit{@when-custom}, \textit{@notBefore-custom}, \textit{@notAfter-custom}, \textit{@from-custom}, \textit{@to-custom}, \textit{@datingPoint}, \textit{@datingMethod}))
    \item[{Membre du}]
  \hyperref[TEI.model.placeNamePart]{model.placeNamePart}
    \item[{Contenu dans}]
  
    \item[analysis: ]
   \hyperref[TEI.cl]{cl} \hyperref[TEI.phr]{phr} \hyperref[TEI.s]{s} \hyperref[TEI.span]{span}\par 
    \item[core: ]
   \hyperref[TEI.abbr]{abbr} \hyperref[TEI.add]{add} \hyperref[TEI.addrLine]{addrLine} \hyperref[TEI.address]{address} \hyperref[TEI.author]{author} \hyperref[TEI.bibl]{bibl} \hyperref[TEI.biblScope]{biblScope} \hyperref[TEI.citedRange]{citedRange} \hyperref[TEI.corr]{corr} \hyperref[TEI.date]{date} \hyperref[TEI.del]{del} \hyperref[TEI.desc]{desc} \hyperref[TEI.distinct]{distinct} \hyperref[TEI.editor]{editor} \hyperref[TEI.email]{email} \hyperref[TEI.emph]{emph} \hyperref[TEI.expan]{expan} \hyperref[TEI.foreign]{foreign} \hyperref[TEI.gloss]{gloss} \hyperref[TEI.head]{head} \hyperref[TEI.headItem]{headItem} \hyperref[TEI.headLabel]{headLabel} \hyperref[TEI.hi]{hi} \hyperref[TEI.item]{item} \hyperref[TEI.l]{l} \hyperref[TEI.label]{label} \hyperref[TEI.measure]{measure} \hyperref[TEI.meeting]{meeting} \hyperref[TEI.mentioned]{mentioned} \hyperref[TEI.name]{name} \hyperref[TEI.note]{note} \hyperref[TEI.num]{num} \hyperref[TEI.orig]{orig} \hyperref[TEI.p]{p} \hyperref[TEI.pubPlace]{pubPlace} \hyperref[TEI.publisher]{publisher} \hyperref[TEI.q]{q} \hyperref[TEI.quote]{quote} \hyperref[TEI.ref]{ref} \hyperref[TEI.reg]{reg} \hyperref[TEI.resp]{resp} \hyperref[TEI.rs]{rs} \hyperref[TEI.said]{said} \hyperref[TEI.sic]{sic} \hyperref[TEI.soCalled]{soCalled} \hyperref[TEI.speaker]{speaker} \hyperref[TEI.stage]{stage} \hyperref[TEI.street]{street} \hyperref[TEI.term]{term} \hyperref[TEI.textLang]{textLang} \hyperref[TEI.time]{time} \hyperref[TEI.title]{title} \hyperref[TEI.unclear]{unclear}\par 
    \item[figures: ]
   \hyperref[TEI.cell]{cell} \hyperref[TEI.figDesc]{figDesc}\par 
    \item[header: ]
   \hyperref[TEI.authority]{authority} \hyperref[TEI.change]{change} \hyperref[TEI.classCode]{classCode} \hyperref[TEI.creation]{creation} \hyperref[TEI.distributor]{distributor} \hyperref[TEI.edition]{edition} \hyperref[TEI.extent]{extent} \hyperref[TEI.funder]{funder} \hyperref[TEI.language]{language} \hyperref[TEI.licence]{licence} \hyperref[TEI.rendition]{rendition}\par 
    \item[iso-fs: ]
   \hyperref[TEI.fDescr]{fDescr} \hyperref[TEI.fsDescr]{fsDescr}\par 
    \item[linking: ]
   \hyperref[TEI.ab]{ab} \hyperref[TEI.seg]{seg}\par 
    \item[msdescription: ]
   \hyperref[TEI.accMat]{accMat} \hyperref[TEI.acquisition]{acquisition} \hyperref[TEI.additions]{additions} \hyperref[TEI.altIdentifier]{altIdentifier} \hyperref[TEI.catchwords]{catchwords} \hyperref[TEI.collation]{collation} \hyperref[TEI.colophon]{colophon} \hyperref[TEI.condition]{condition} \hyperref[TEI.custEvent]{custEvent} \hyperref[TEI.decoNote]{decoNote} \hyperref[TEI.explicit]{explicit} \hyperref[TEI.filiation]{filiation} \hyperref[TEI.finalRubric]{finalRubric} \hyperref[TEI.foliation]{foliation} \hyperref[TEI.heraldry]{heraldry} \hyperref[TEI.incipit]{incipit} \hyperref[TEI.layout]{layout} \hyperref[TEI.material]{material} \hyperref[TEI.msIdentifier]{msIdentifier} \hyperref[TEI.musicNotation]{musicNotation} \hyperref[TEI.objectType]{objectType} \hyperref[TEI.origDate]{origDate} \hyperref[TEI.origPlace]{origPlace} \hyperref[TEI.origin]{origin} \hyperref[TEI.provenance]{provenance} \hyperref[TEI.rubric]{rubric} \hyperref[TEI.secFol]{secFol} \hyperref[TEI.signatures]{signatures} \hyperref[TEI.source]{source} \hyperref[TEI.stamp]{stamp} \hyperref[TEI.summary]{summary} \hyperref[TEI.support]{support} \hyperref[TEI.surrogates]{surrogates} \hyperref[TEI.typeNote]{typeNote} \hyperref[TEI.watermark]{watermark}\par 
    \item[namesdates: ]
   \hyperref[TEI.addName]{addName} \hyperref[TEI.affiliation]{affiliation} \hyperref[TEI.country]{country} \hyperref[TEI.forename]{forename} \hyperref[TEI.genName]{genName} \hyperref[TEI.geogName]{geogName} \hyperref[TEI.location]{location} \hyperref[TEI.nameLink]{nameLink} \hyperref[TEI.org]{org} \hyperref[TEI.orgName]{orgName} \hyperref[TEI.persName]{persName} \hyperref[TEI.place]{place} \hyperref[TEI.placeName]{placeName} \hyperref[TEI.region]{region} \hyperref[TEI.roleName]{roleName} \hyperref[TEI.settlement]{settlement} \hyperref[TEI.surname]{surname}\par 
    \item[spoken: ]
   \hyperref[TEI.annotationBlock]{annotationBlock}\par 
    \item[standOff: ]
   \hyperref[TEI.listAnnotation]{listAnnotation}\par 
    \item[textstructure: ]
   \hyperref[TEI.docAuthor]{docAuthor} \hyperref[TEI.docDate]{docDate} \hyperref[TEI.docEdition]{docEdition} \hyperref[TEI.titlePart]{titlePart}\par 
    \item[transcr: ]
   \hyperref[TEI.damage]{damage} \hyperref[TEI.fw]{fw} \hyperref[TEI.metamark]{metamark} \hyperref[TEI.mod]{mod} \hyperref[TEI.restore]{restore} \hyperref[TEI.retrace]{retrace} \hyperref[TEI.secl]{secl} \hyperref[TEI.supplied]{supplied} \hyperref[TEI.surplus]{surplus}
    \item[{Peut contenir}]
  
    \item[analysis: ]
   \hyperref[TEI.c]{c} \hyperref[TEI.cl]{cl} \hyperref[TEI.interp]{interp} \hyperref[TEI.interpGrp]{interpGrp} \hyperref[TEI.m]{m} \hyperref[TEI.pc]{pc} \hyperref[TEI.phr]{phr} \hyperref[TEI.s]{s} \hyperref[TEI.span]{span} \hyperref[TEI.spanGrp]{spanGrp} \hyperref[TEI.w]{w}\par 
    \item[core: ]
   \hyperref[TEI.abbr]{abbr} \hyperref[TEI.add]{add} \hyperref[TEI.address]{address} \hyperref[TEI.binaryObject]{binaryObject} \hyperref[TEI.cb]{cb} \hyperref[TEI.choice]{choice} \hyperref[TEI.corr]{corr} \hyperref[TEI.date]{date} \hyperref[TEI.del]{del} \hyperref[TEI.distinct]{distinct} \hyperref[TEI.email]{email} \hyperref[TEI.emph]{emph} \hyperref[TEI.expan]{expan} \hyperref[TEI.foreign]{foreign} \hyperref[TEI.gap]{gap} \hyperref[TEI.gb]{gb} \hyperref[TEI.gloss]{gloss} \hyperref[TEI.graphic]{graphic} \hyperref[TEI.hi]{hi} \hyperref[TEI.index]{index} \hyperref[TEI.lb]{lb} \hyperref[TEI.measure]{measure} \hyperref[TEI.measureGrp]{measureGrp} \hyperref[TEI.media]{media} \hyperref[TEI.mentioned]{mentioned} \hyperref[TEI.milestone]{milestone} \hyperref[TEI.name]{name} \hyperref[TEI.note]{note} \hyperref[TEI.num]{num} \hyperref[TEI.orig]{orig} \hyperref[TEI.pb]{pb} \hyperref[TEI.ptr]{ptr} \hyperref[TEI.ref]{ref} \hyperref[TEI.reg]{reg} \hyperref[TEI.rs]{rs} \hyperref[TEI.sic]{sic} \hyperref[TEI.soCalled]{soCalled} \hyperref[TEI.term]{term} \hyperref[TEI.time]{time} \hyperref[TEI.title]{title} \hyperref[TEI.unclear]{unclear}\par 
    \item[derived-module-tei.istex: ]
   \hyperref[TEI.math]{math} \hyperref[TEI.mrow]{mrow}\par 
    \item[figures: ]
   \hyperref[TEI.figure]{figure} \hyperref[TEI.formula]{formula} \hyperref[TEI.notatedMusic]{notatedMusic}\par 
    \item[header: ]
   \hyperref[TEI.idno]{idno}\par 
    \item[iso-fs: ]
   \hyperref[TEI.fLib]{fLib} \hyperref[TEI.fs]{fs} \hyperref[TEI.fvLib]{fvLib}\par 
    \item[linking: ]
   \hyperref[TEI.alt]{alt} \hyperref[TEI.altGrp]{altGrp} \hyperref[TEI.anchor]{anchor} \hyperref[TEI.join]{join} \hyperref[TEI.joinGrp]{joinGrp} \hyperref[TEI.link]{link} \hyperref[TEI.linkGrp]{linkGrp} \hyperref[TEI.seg]{seg} \hyperref[TEI.timeline]{timeline}\par 
    \item[msdescription: ]
   \hyperref[TEI.catchwords]{catchwords} \hyperref[TEI.depth]{depth} \hyperref[TEI.dim]{dim} \hyperref[TEI.dimensions]{dimensions} \hyperref[TEI.height]{height} \hyperref[TEI.heraldry]{heraldry} \hyperref[TEI.locus]{locus} \hyperref[TEI.locusGrp]{locusGrp} \hyperref[TEI.material]{material} \hyperref[TEI.objectType]{objectType} \hyperref[TEI.origDate]{origDate} \hyperref[TEI.origPlace]{origPlace} \hyperref[TEI.secFol]{secFol} \hyperref[TEI.signatures]{signatures} \hyperref[TEI.source]{source} \hyperref[TEI.stamp]{stamp} \hyperref[TEI.watermark]{watermark} \hyperref[TEI.width]{width}\par 
    \item[namesdates: ]
   \hyperref[TEI.addName]{addName} \hyperref[TEI.affiliation]{affiliation} \hyperref[TEI.country]{country} \hyperref[TEI.forename]{forename} \hyperref[TEI.genName]{genName} \hyperref[TEI.geogName]{geogName} \hyperref[TEI.location]{location} \hyperref[TEI.nameLink]{nameLink} \hyperref[TEI.orgName]{orgName} \hyperref[TEI.persName]{persName} \hyperref[TEI.placeName]{placeName} \hyperref[TEI.region]{region} \hyperref[TEI.roleName]{roleName} \hyperref[TEI.settlement]{settlement} \hyperref[TEI.state]{state} \hyperref[TEI.surname]{surname}\par 
    \item[spoken: ]
   \hyperref[TEI.annotationBlock]{annotationBlock}\par 
    \item[transcr: ]
   \hyperref[TEI.addSpan]{addSpan} \hyperref[TEI.am]{am} \hyperref[TEI.damage]{damage} \hyperref[TEI.damageSpan]{damageSpan} \hyperref[TEI.delSpan]{delSpan} \hyperref[TEI.ex]{ex} \hyperref[TEI.fw]{fw} \hyperref[TEI.handShift]{handShift} \hyperref[TEI.listTranspose]{listTranspose} \hyperref[TEI.metamark]{metamark} \hyperref[TEI.mod]{mod} \hyperref[TEI.redo]{redo} \hyperref[TEI.restore]{restore} \hyperref[TEI.retrace]{retrace} \hyperref[TEI.secl]{secl} \hyperref[TEI.space]{space} \hyperref[TEI.subst]{subst} \hyperref[TEI.substJoin]{substJoin} \hyperref[TEI.supplied]{supplied} \hyperref[TEI.surplus]{surplus} \hyperref[TEI.undo]{undo}\par des données textuelles
    \item[{Note}]
  \par
La source recommandée des codes pour représenter les noms de pays est ISO 3166.
    \item[{Exemple}]
  \leavevmode\bgroup\exampleFont \begin{shaded}\noindent\mbox{}{<\textbf{country}\hspace*{6pt}{key}="{DK}">}Danemark{</\textbf{country}>}\end{shaded}\egroup 


    \item[{Modèle de contenu}]
  \mbox{}\hfill\\[-10pt]\begin{Verbatim}[fontsize=\small]
<content>
 <macroRef key="macro.phraseSeq"/>
</content>
    
\end{Verbatim}

    \item[{Schéma Declaration}]
  \mbox{}\hfill\\[-10pt]\begin{Verbatim}[fontsize=\small]
element country
{
   tei_att.global.attributes,
   tei_att.naming.attributes,
   tei_att.typed.attributes,
   tei_att.datable.attributes,
   tei_macro.phraseSeq}
\end{Verbatim}

\end{reflist}  \index{creation=<creation>|oddindex}
\begin{reflist}
\item[]\begin{specHead}{TEI.creation}{<creation> }(création) contient des informations concernant la création d’un texte. [\xref{http://www.tei-c.org/release/doc/tei-p5-doc/en/html/HD.html\#HD4C}{2.4.1. Creation} \xref{http://www.tei-c.org/release/doc/tei-p5-doc/en/html/HD.html\#HD4}{2.4. The Profile Description}]\end{specHead} 
    \item[{Module}]
  header
    \item[{Attributs}]
  Attributs \hyperref[TEI.att.global]{att.global} (\textit{@xml:id}, \textit{@n}, \textit{@xml:lang}, \textit{@xml:base}, \textit{@xml:space})  (\hyperref[TEI.att.global.rendition]{att.global.rendition} (\textit{@rend}, \textit{@style}, \textit{@rendition})) (\hyperref[TEI.att.global.linking]{att.global.linking} (\textit{@corresp}, \textit{@synch}, \textit{@sameAs}, \textit{@copyOf}, \textit{@next}, \textit{@prev}, \textit{@exclude}, \textit{@select})) (\hyperref[TEI.att.global.analytic]{att.global.analytic} (\textit{@ana})) (\hyperref[TEI.att.global.facs]{att.global.facs} (\textit{@facs})) (\hyperref[TEI.att.global.change]{att.global.change} (\textit{@change})) (\hyperref[TEI.att.global.responsibility]{att.global.responsibility} (\textit{@cert}, \textit{@resp})) (\hyperref[TEI.att.global.source]{att.global.source} (\textit{@source})) \hyperref[TEI.att.datable]{att.datable} (\textit{@calendar}, \textit{@period})  (\hyperref[TEI.att.datable.w3c]{att.datable.w3c} (\textit{@when}, \textit{@notBefore}, \textit{@notAfter}, \textit{@from}, \textit{@to})) (\hyperref[TEI.att.datable.iso]{att.datable.iso} (\textit{@when-iso}, \textit{@notBefore-iso}, \textit{@notAfter-iso}, \textit{@from-iso}, \textit{@to-iso})) (\hyperref[TEI.att.datable.custom]{att.datable.custom} (\textit{@when-custom}, \textit{@notBefore-custom}, \textit{@notAfter-custom}, \textit{@from-custom}, \textit{@to-custom}, \textit{@datingPoint}, \textit{@datingMethod}))
    \item[{Membre du}]
  \hyperref[TEI.model.profileDescPart]{model.profileDescPart}
    \item[{Contenu dans}]
  
    \item[header: ]
   \hyperref[TEI.profileDesc]{profileDesc}
    \item[{Peut contenir}]
  
    \item[core: ]
   \hyperref[TEI.abbr]{abbr} \hyperref[TEI.address]{address} \hyperref[TEI.choice]{choice} \hyperref[TEI.date]{date} \hyperref[TEI.distinct]{distinct} \hyperref[TEI.email]{email} \hyperref[TEI.emph]{emph} \hyperref[TEI.expan]{expan} \hyperref[TEI.foreign]{foreign} \hyperref[TEI.gloss]{gloss} \hyperref[TEI.hi]{hi} \hyperref[TEI.measure]{measure} \hyperref[TEI.measureGrp]{measureGrp} \hyperref[TEI.mentioned]{mentioned} \hyperref[TEI.name]{name} \hyperref[TEI.num]{num} \hyperref[TEI.ptr]{ptr} \hyperref[TEI.ref]{ref} \hyperref[TEI.rs]{rs} \hyperref[TEI.soCalled]{soCalled} \hyperref[TEI.term]{term} \hyperref[TEI.time]{time} \hyperref[TEI.title]{title}\par 
    \item[header: ]
   \hyperref[TEI.idno]{idno}\par 
    \item[msdescription: ]
   \hyperref[TEI.catchwords]{catchwords} \hyperref[TEI.depth]{depth} \hyperref[TEI.dim]{dim} \hyperref[TEI.dimensions]{dimensions} \hyperref[TEI.height]{height} \hyperref[TEI.heraldry]{heraldry} \hyperref[TEI.locus]{locus} \hyperref[TEI.locusGrp]{locusGrp} \hyperref[TEI.material]{material} \hyperref[TEI.objectType]{objectType} \hyperref[TEI.origDate]{origDate} \hyperref[TEI.origPlace]{origPlace} \hyperref[TEI.secFol]{secFol} \hyperref[TEI.signatures]{signatures} \hyperref[TEI.stamp]{stamp} \hyperref[TEI.watermark]{watermark} \hyperref[TEI.width]{width}\par 
    \item[namesdates: ]
   \hyperref[TEI.addName]{addName} \hyperref[TEI.affiliation]{affiliation} \hyperref[TEI.country]{country} \hyperref[TEI.forename]{forename} \hyperref[TEI.genName]{genName} \hyperref[TEI.geogName]{geogName} \hyperref[TEI.location]{location} \hyperref[TEI.nameLink]{nameLink} \hyperref[TEI.orgName]{orgName} \hyperref[TEI.persName]{persName} \hyperref[TEI.placeName]{placeName} \hyperref[TEI.region]{region} \hyperref[TEI.roleName]{roleName} \hyperref[TEI.settlement]{settlement} \hyperref[TEI.state]{state} \hyperref[TEI.surname]{surname}\par 
    \item[transcr: ]
   \hyperref[TEI.am]{am} \hyperref[TEI.ex]{ex} \hyperref[TEI.subst]{subst}\par des données textuelles
    \item[{Note}]
  \par
L’élément \hyperref[TEI.creation]{<creation>} peut être utilisé pour détailler des éléments concernant l’origine du texte, c’est-à-dire sa date et son lieu de composition ; on ne doit pas le confondre avec l'élément \hyperref[TEI.publicationStmt]{<publicationStmt>} qui contient la date et le lieu de publication.
    \item[{Exemple}]
  \leavevmode\bgroup\exampleFont \begin{shaded}\noindent\mbox{}{<\textbf{creation}>}\mbox{}\newline 
\hspace*{6pt}{<\textbf{date}>}Avant 1987{</\textbf{date}>}\mbox{}\newline 
{</\textbf{creation}>}\end{shaded}\egroup 


    \item[{Exemple}]
  \leavevmode\bgroup\exampleFont \begin{shaded}\noindent\mbox{}{<\textbf{creation}>}\mbox{}\newline 
\hspace*{6pt}{<\textbf{date}\hspace*{6pt}{when}="{1988-07-10}">}10 Juillet 1988{</\textbf{date}>}\mbox{}\newline 
{</\textbf{creation}>}\end{shaded}\egroup 


    \item[{Modèle de contenu}]
  \mbox{}\hfill\\[-10pt]\begin{Verbatim}[fontsize=\small]
<content>
 <alternate maxOccurs="unbounded"
  minOccurs="0">
  <textNode/>
  <classRef key="model.limitedPhrase"/>
  <elementRef key="listChange"/>
 </alternate>
</content>
    
\end{Verbatim}

    \item[{Schéma Declaration}]
  \mbox{}\hfill\\[-10pt]\begin{Verbatim}[fontsize=\small]
element creation
{
   tei_att.global.attributes,
   tei_att.datable.attributes,
   ( text | tei_model.limitedPhrase | listChange )*
}
\end{Verbatim}

\end{reflist}  \index{custEvent=<custEvent>|oddindex}
\begin{reflist}
\item[]\begin{specHead}{TEI.custEvent}{<custEvent> }(événement dans la conservation) décrit un événement dans l'histoire de la conservation du manuscrit. [\xref{http://www.tei-c.org/release/doc/tei-p5-doc/en/html/MS.html\#msadch}{10.9.1.2. Availability and Custodial History}]\end{specHead} 
    \item[{Module}]
  msdescription
    \item[{Attributs}]
  Attributs \hyperref[TEI.att.global]{att.global} (\textit{@xml:id}, \textit{@n}, \textit{@xml:lang}, \textit{@xml:base}, \textit{@xml:space})  (\hyperref[TEI.att.global.rendition]{att.global.rendition} (\textit{@rend}, \textit{@style}, \textit{@rendition})) (\hyperref[TEI.att.global.linking]{att.global.linking} (\textit{@corresp}, \textit{@synch}, \textit{@sameAs}, \textit{@copyOf}, \textit{@next}, \textit{@prev}, \textit{@exclude}, \textit{@select})) (\hyperref[TEI.att.global.analytic]{att.global.analytic} (\textit{@ana})) (\hyperref[TEI.att.global.facs]{att.global.facs} (\textit{@facs})) (\hyperref[TEI.att.global.change]{att.global.change} (\textit{@change})) (\hyperref[TEI.att.global.responsibility]{att.global.responsibility} (\textit{@cert}, \textit{@resp})) (\hyperref[TEI.att.global.source]{att.global.source} (\textit{@source})) \hyperref[TEI.att.datable]{att.datable} (\textit{@calendar}, \textit{@period})  (\hyperref[TEI.att.datable.w3c]{att.datable.w3c} (\textit{@when}, \textit{@notBefore}, \textit{@notAfter}, \textit{@from}, \textit{@to})) (\hyperref[TEI.att.datable.iso]{att.datable.iso} (\textit{@when-iso}, \textit{@notBefore-iso}, \textit{@notAfter-iso}, \textit{@from-iso}, \textit{@to-iso})) (\hyperref[TEI.att.datable.custom]{att.datable.custom} (\textit{@when-custom}, \textit{@notBefore-custom}, \textit{@notAfter-custom}, \textit{@from-custom}, \textit{@to-custom}, \textit{@datingPoint}, \textit{@datingMethod})) \hyperref[TEI.att.typed]{att.typed} (\textit{@type}, \textit{@subtype}) 
    \item[{Contenu dans}]
  
    \item[msdescription: ]
   \hyperref[TEI.custodialHist]{custodialHist}
    \item[{Peut contenir}]
  
    \item[analysis: ]
   \hyperref[TEI.c]{c} \hyperref[TEI.cl]{cl} \hyperref[TEI.interp]{interp} \hyperref[TEI.interpGrp]{interpGrp} \hyperref[TEI.m]{m} \hyperref[TEI.pc]{pc} \hyperref[TEI.phr]{phr} \hyperref[TEI.s]{s} \hyperref[TEI.span]{span} \hyperref[TEI.spanGrp]{spanGrp} \hyperref[TEI.w]{w}\par 
    \item[core: ]
   \hyperref[TEI.abbr]{abbr} \hyperref[TEI.add]{add} \hyperref[TEI.address]{address} \hyperref[TEI.bibl]{bibl} \hyperref[TEI.biblStruct]{biblStruct} \hyperref[TEI.binaryObject]{binaryObject} \hyperref[TEI.cb]{cb} \hyperref[TEI.choice]{choice} \hyperref[TEI.cit]{cit} \hyperref[TEI.corr]{corr} \hyperref[TEI.date]{date} \hyperref[TEI.del]{del} \hyperref[TEI.desc]{desc} \hyperref[TEI.distinct]{distinct} \hyperref[TEI.email]{email} \hyperref[TEI.emph]{emph} \hyperref[TEI.expan]{expan} \hyperref[TEI.foreign]{foreign} \hyperref[TEI.gap]{gap} \hyperref[TEI.gb]{gb} \hyperref[TEI.gloss]{gloss} \hyperref[TEI.graphic]{graphic} \hyperref[TEI.hi]{hi} \hyperref[TEI.index]{index} \hyperref[TEI.l]{l} \hyperref[TEI.label]{label} \hyperref[TEI.lb]{lb} \hyperref[TEI.lg]{lg} \hyperref[TEI.list]{list} \hyperref[TEI.listBibl]{listBibl} \hyperref[TEI.measure]{measure} \hyperref[TEI.measureGrp]{measureGrp} \hyperref[TEI.media]{media} \hyperref[TEI.mentioned]{mentioned} \hyperref[TEI.milestone]{milestone} \hyperref[TEI.name]{name} \hyperref[TEI.note]{note} \hyperref[TEI.num]{num} \hyperref[TEI.orig]{orig} \hyperref[TEI.p]{p} \hyperref[TEI.pb]{pb} \hyperref[TEI.ptr]{ptr} \hyperref[TEI.q]{q} \hyperref[TEI.quote]{quote} \hyperref[TEI.ref]{ref} \hyperref[TEI.reg]{reg} \hyperref[TEI.rs]{rs} \hyperref[TEI.said]{said} \hyperref[TEI.sic]{sic} \hyperref[TEI.soCalled]{soCalled} \hyperref[TEI.sp]{sp} \hyperref[TEI.stage]{stage} \hyperref[TEI.term]{term} \hyperref[TEI.time]{time} \hyperref[TEI.title]{title} \hyperref[TEI.unclear]{unclear}\par 
    \item[derived-module-tei.istex: ]
   \hyperref[TEI.math]{math} \hyperref[TEI.mrow]{mrow}\par 
    \item[figures: ]
   \hyperref[TEI.figure]{figure} \hyperref[TEI.formula]{formula} \hyperref[TEI.notatedMusic]{notatedMusic} \hyperref[TEI.table]{table}\par 
    \item[header: ]
   \hyperref[TEI.biblFull]{biblFull} \hyperref[TEI.idno]{idno}\par 
    \item[iso-fs: ]
   \hyperref[TEI.fLib]{fLib} \hyperref[TEI.fs]{fs} \hyperref[TEI.fvLib]{fvLib}\par 
    \item[linking: ]
   \hyperref[TEI.ab]{ab} \hyperref[TEI.alt]{alt} \hyperref[TEI.altGrp]{altGrp} \hyperref[TEI.anchor]{anchor} \hyperref[TEI.join]{join} \hyperref[TEI.joinGrp]{joinGrp} \hyperref[TEI.link]{link} \hyperref[TEI.linkGrp]{linkGrp} \hyperref[TEI.seg]{seg} \hyperref[TEI.timeline]{timeline}\par 
    \item[msdescription: ]
   \hyperref[TEI.catchwords]{catchwords} \hyperref[TEI.depth]{depth} \hyperref[TEI.dim]{dim} \hyperref[TEI.dimensions]{dimensions} \hyperref[TEI.height]{height} \hyperref[TEI.heraldry]{heraldry} \hyperref[TEI.locus]{locus} \hyperref[TEI.locusGrp]{locusGrp} \hyperref[TEI.material]{material} \hyperref[TEI.msDesc]{msDesc} \hyperref[TEI.objectType]{objectType} \hyperref[TEI.origDate]{origDate} \hyperref[TEI.origPlace]{origPlace} \hyperref[TEI.secFol]{secFol} \hyperref[TEI.signatures]{signatures} \hyperref[TEI.source]{source} \hyperref[TEI.stamp]{stamp} \hyperref[TEI.watermark]{watermark} \hyperref[TEI.width]{width}\par 
    \item[namesdates: ]
   \hyperref[TEI.addName]{addName} \hyperref[TEI.affiliation]{affiliation} \hyperref[TEI.country]{country} \hyperref[TEI.forename]{forename} \hyperref[TEI.genName]{genName} \hyperref[TEI.geogName]{geogName} \hyperref[TEI.listOrg]{listOrg} \hyperref[TEI.listPlace]{listPlace} \hyperref[TEI.location]{location} \hyperref[TEI.nameLink]{nameLink} \hyperref[TEI.orgName]{orgName} \hyperref[TEI.persName]{persName} \hyperref[TEI.placeName]{placeName} \hyperref[TEI.region]{region} \hyperref[TEI.roleName]{roleName} \hyperref[TEI.settlement]{settlement} \hyperref[TEI.state]{state} \hyperref[TEI.surname]{surname}\par 
    \item[spoken: ]
   \hyperref[TEI.annotationBlock]{annotationBlock}\par 
    \item[textstructure: ]
   \hyperref[TEI.floatingText]{floatingText}\par 
    \item[transcr: ]
   \hyperref[TEI.addSpan]{addSpan} \hyperref[TEI.am]{am} \hyperref[TEI.damage]{damage} \hyperref[TEI.damageSpan]{damageSpan} \hyperref[TEI.delSpan]{delSpan} \hyperref[TEI.ex]{ex} \hyperref[TEI.fw]{fw} \hyperref[TEI.handShift]{handShift} \hyperref[TEI.listTranspose]{listTranspose} \hyperref[TEI.metamark]{metamark} \hyperref[TEI.mod]{mod} \hyperref[TEI.redo]{redo} \hyperref[TEI.restore]{restore} \hyperref[TEI.retrace]{retrace} \hyperref[TEI.secl]{secl} \hyperref[TEI.space]{space} \hyperref[TEI.subst]{subst} \hyperref[TEI.substJoin]{substJoin} \hyperref[TEI.supplied]{supplied} \hyperref[TEI.surplus]{surplus} \hyperref[TEI.undo]{undo}\par des données textuelles
    \item[{Exemple}]
  \leavevmode\bgroup\exampleFont \begin{shaded}\noindent\mbox{}{<\textbf{custEvent}\hspace*{6pt}{type}="{photography}">}Photographed by David Cooper on {<\textbf{date}>}12 Dec\mbox{}\newline 
\hspace*{6pt}\hspace*{6pt} 1964{</\textbf{date}>}\mbox{}\newline 
{</\textbf{custEvent}>}\end{shaded}\egroup 


    \item[{Modèle de contenu}]
  \mbox{}\hfill\\[-10pt]\begin{Verbatim}[fontsize=\small]
<content>
 <macroRef key="macro.specialPara"/>
</content>
    
\end{Verbatim}

    \item[{Schéma Declaration}]
  \mbox{}\hfill\\[-10pt]\begin{Verbatim}[fontsize=\small]
element custEvent
{
   tei_att.global.attributes,
   tei_att.datable.attributes,
   tei_att.typed.attributes,
   tei_macro.specialPara}
\end{Verbatim}

\end{reflist}  \index{custodialHist=<custodialHist>|oddindex}
\begin{reflist}
\item[]\begin{specHead}{TEI.custodialHist}{<custodialHist> }(histoire de la conservation) contient des informations sur l'histoire de la conservation, soit en texte libre, soit sous la forme d'une série d'éléments \hyperref[TEI.custEvent]{<custEvent>}. [\xref{http://www.tei-c.org/release/doc/tei-p5-doc/en/html/MS.html\#msadch}{10.9.1.2. Availability and Custodial History}]\end{specHead} 
    \item[{Module}]
  msdescription
    \item[{Attributs}]
  Attributs \hyperref[TEI.att.global]{att.global} (\textit{@xml:id}, \textit{@n}, \textit{@xml:lang}, \textit{@xml:base}, \textit{@xml:space})  (\hyperref[TEI.att.global.rendition]{att.global.rendition} (\textit{@rend}, \textit{@style}, \textit{@rendition})) (\hyperref[TEI.att.global.linking]{att.global.linking} (\textit{@corresp}, \textit{@synch}, \textit{@sameAs}, \textit{@copyOf}, \textit{@next}, \textit{@prev}, \textit{@exclude}, \textit{@select})) (\hyperref[TEI.att.global.analytic]{att.global.analytic} (\textit{@ana})) (\hyperref[TEI.att.global.facs]{att.global.facs} (\textit{@facs})) (\hyperref[TEI.att.global.change]{att.global.change} (\textit{@change})) (\hyperref[TEI.att.global.responsibility]{att.global.responsibility} (\textit{@cert}, \textit{@resp})) (\hyperref[TEI.att.global.source]{att.global.source} (\textit{@source}))
    \item[{Contenu dans}]
  
    \item[msdescription: ]
   \hyperref[TEI.adminInfo]{adminInfo}
    \item[{Peut contenir}]
  
    \item[core: ]
   \hyperref[TEI.p]{p}\par 
    \item[linking: ]
   \hyperref[TEI.ab]{ab}\par 
    \item[msdescription: ]
   \hyperref[TEI.custEvent]{custEvent}
    \item[{Exemple}]
  \leavevmode\bgroup\exampleFont \begin{shaded}\noindent\mbox{}{<\textbf{custodialHist}>}\mbox{}\newline 
\hspace*{6pt}{<\textbf{custEvent}\hspace*{6pt}{notAfter}="{1963-02}"\mbox{}\newline 
\hspace*{6pt}\hspace*{6pt}{notBefore}="{1961-03}"\hspace*{6pt}{type}="{conservation}">} Conserved between\mbox{}\newline 
\hspace*{6pt}\hspace*{6pt} March 1961 and February 1963 at Birgitte Dalls Konserveringsværksted.{</\textbf{custEvent}>}\mbox{}\newline 
\hspace*{6pt}{<\textbf{custEvent}\hspace*{6pt}{notAfter}="{1988-05-30}"\mbox{}\newline 
\hspace*{6pt}\hspace*{6pt}{notBefore}="{1988-05-01}"\hspace*{6pt}{type}="{photography}">} Photographed\mbox{}\newline 
\hspace*{6pt}\hspace*{6pt} in May 1988 by AMI/FA.{</\textbf{custEvent}>}\mbox{}\newline 
\hspace*{6pt}{<\textbf{custEvent}\hspace*{6pt}{notAfter}="{1989-11-13}"\mbox{}\newline 
\hspace*{6pt}\hspace*{6pt}{notBefore}="{1989-11-13}"\hspace*{6pt}{type}="{transfer-dispatch}">} Dispatched to Iceland 13 November 1989.{</\textbf{custEvent}>}\mbox{}\newline 
{</\textbf{custodialHist}>}\end{shaded}\egroup 


    \item[{Modèle de contenu}]
  \mbox{}\hfill\\[-10pt]\begin{Verbatim}[fontsize=\small]
<content>
 <alternate maxOccurs="1" minOccurs="1">
  <classRef key="model.pLike"
   maxOccurs="unbounded" minOccurs="1"/>
  <elementRef key="custEvent"
   maxOccurs="unbounded" minOccurs="1"/>
 </alternate>
</content>
    
\end{Verbatim}

    \item[{Schéma Declaration}]
  \mbox{}\hfill\\[-10pt]\begin{Verbatim}[fontsize=\small]
element custodialHist
{
   tei_att.global.attributes,
   ( tei_model.pLike+ | tei_custEvent+ )
}
\end{Verbatim}

\end{reflist}  \index{damage=<damage>|oddindex}
\begin{reflist}
\item[]\begin{specHead}{TEI.damage}{<damage> }(dommage) sert à encoder une zone qui a subi des dommages dans le manuscrit témoin du texte. [\xref{http://www.tei-c.org/release/doc/tei-p5-doc/en/html/PH.html\#PHDA}{11.3.3.1. Damage, Illegibility, and Supplied Text}]\end{specHead} 
    \item[{Module}]
  transcr
    \item[{Attributs}]
  Attributs \hyperref[TEI.att.global]{att.global} (\textit{@xml:id}, \textit{@n}, \textit{@xml:lang}, \textit{@xml:base}, \textit{@xml:space})  (\hyperref[TEI.att.global.rendition]{att.global.rendition} (\textit{@rend}, \textit{@style}, \textit{@rendition})) (\hyperref[TEI.att.global.linking]{att.global.linking} (\textit{@corresp}, \textit{@synch}, \textit{@sameAs}, \textit{@copyOf}, \textit{@next}, \textit{@prev}, \textit{@exclude}, \textit{@select})) (\hyperref[TEI.att.global.analytic]{att.global.analytic} (\textit{@ana})) (\hyperref[TEI.att.global.facs]{att.global.facs} (\textit{@facs})) (\hyperref[TEI.att.global.change]{att.global.change} (\textit{@change})) (\hyperref[TEI.att.global.responsibility]{att.global.responsibility} (\textit{@cert}, \textit{@resp})) (\hyperref[TEI.att.global.source]{att.global.source} (\textit{@source})) \hyperref[TEI.att.typed]{att.typed} (\textit{@type}, \textit{@subtype}) \hyperref[TEI.att.damaged]{att.damaged} (\textit{@agent}, \textit{@degree}, \textit{@group})  (\hyperref[TEI.att.dimensions]{att.dimensions} (\textit{@unit}, \textit{@quantity}, \textit{@extent}, \textit{@precision}, \textit{@scope}) (\hyperref[TEI.att.ranging]{att.ranging} (\textit{@atLeast}, \textit{@atMost}, \textit{@min}, \textit{@max}, \textit{@confidence})) ) (\hyperref[TEI.att.written]{att.written} (\textit{@hand}))
    \item[{Membre du}]
  \hyperref[TEI.model.linePart]{model.linePart} \hyperref[TEI.model.pPart.transcriptional]{model.pPart.transcriptional}
    \item[{Contenu dans}]
  
    \item[analysis: ]
   \hyperref[TEI.cl]{cl} \hyperref[TEI.pc]{pc} \hyperref[TEI.phr]{phr} \hyperref[TEI.s]{s} \hyperref[TEI.w]{w}\par 
    \item[core: ]
   \hyperref[TEI.abbr]{abbr} \hyperref[TEI.add]{add} \hyperref[TEI.addrLine]{addrLine} \hyperref[TEI.author]{author} \hyperref[TEI.bibl]{bibl} \hyperref[TEI.biblScope]{biblScope} \hyperref[TEI.citedRange]{citedRange} \hyperref[TEI.corr]{corr} \hyperref[TEI.date]{date} \hyperref[TEI.del]{del} \hyperref[TEI.distinct]{distinct} \hyperref[TEI.editor]{editor} \hyperref[TEI.email]{email} \hyperref[TEI.emph]{emph} \hyperref[TEI.expan]{expan} \hyperref[TEI.foreign]{foreign} \hyperref[TEI.gloss]{gloss} \hyperref[TEI.head]{head} \hyperref[TEI.headItem]{headItem} \hyperref[TEI.headLabel]{headLabel} \hyperref[TEI.hi]{hi} \hyperref[TEI.item]{item} \hyperref[TEI.l]{l} \hyperref[TEI.label]{label} \hyperref[TEI.measure]{measure} \hyperref[TEI.mentioned]{mentioned} \hyperref[TEI.name]{name} \hyperref[TEI.note]{note} \hyperref[TEI.num]{num} \hyperref[TEI.orig]{orig} \hyperref[TEI.p]{p} \hyperref[TEI.pubPlace]{pubPlace} \hyperref[TEI.publisher]{publisher} \hyperref[TEI.q]{q} \hyperref[TEI.quote]{quote} \hyperref[TEI.ref]{ref} \hyperref[TEI.reg]{reg} \hyperref[TEI.rs]{rs} \hyperref[TEI.said]{said} \hyperref[TEI.sic]{sic} \hyperref[TEI.soCalled]{soCalled} \hyperref[TEI.speaker]{speaker} \hyperref[TEI.stage]{stage} \hyperref[TEI.street]{street} \hyperref[TEI.term]{term} \hyperref[TEI.textLang]{textLang} \hyperref[TEI.time]{time} \hyperref[TEI.title]{title} \hyperref[TEI.unclear]{unclear}\par 
    \item[figures: ]
   \hyperref[TEI.cell]{cell}\par 
    \item[header: ]
   \hyperref[TEI.change]{change} \hyperref[TEI.distributor]{distributor} \hyperref[TEI.edition]{edition} \hyperref[TEI.extent]{extent} \hyperref[TEI.licence]{licence}\par 
    \item[linking: ]
   \hyperref[TEI.ab]{ab} \hyperref[TEI.seg]{seg}\par 
    \item[msdescription: ]
   \hyperref[TEI.accMat]{accMat} \hyperref[TEI.acquisition]{acquisition} \hyperref[TEI.additions]{additions} \hyperref[TEI.catchwords]{catchwords} \hyperref[TEI.collation]{collation} \hyperref[TEI.colophon]{colophon} \hyperref[TEI.condition]{condition} \hyperref[TEI.custEvent]{custEvent} \hyperref[TEI.decoNote]{decoNote} \hyperref[TEI.explicit]{explicit} \hyperref[TEI.filiation]{filiation} \hyperref[TEI.finalRubric]{finalRubric} \hyperref[TEI.foliation]{foliation} \hyperref[TEI.heraldry]{heraldry} \hyperref[TEI.incipit]{incipit} \hyperref[TEI.layout]{layout} \hyperref[TEI.material]{material} \hyperref[TEI.musicNotation]{musicNotation} \hyperref[TEI.objectType]{objectType} \hyperref[TEI.origDate]{origDate} \hyperref[TEI.origPlace]{origPlace} \hyperref[TEI.origin]{origin} \hyperref[TEI.provenance]{provenance} \hyperref[TEI.rubric]{rubric} \hyperref[TEI.secFol]{secFol} \hyperref[TEI.signatures]{signatures} \hyperref[TEI.source]{source} \hyperref[TEI.stamp]{stamp} \hyperref[TEI.summary]{summary} \hyperref[TEI.support]{support} \hyperref[TEI.surrogates]{surrogates} \hyperref[TEI.typeNote]{typeNote} \hyperref[TEI.watermark]{watermark}\par 
    \item[namesdates: ]
   \hyperref[TEI.addName]{addName} \hyperref[TEI.affiliation]{affiliation} \hyperref[TEI.country]{country} \hyperref[TEI.forename]{forename} \hyperref[TEI.genName]{genName} \hyperref[TEI.geogName]{geogName} \hyperref[TEI.nameLink]{nameLink} \hyperref[TEI.orgName]{orgName} \hyperref[TEI.persName]{persName} \hyperref[TEI.placeName]{placeName} \hyperref[TEI.region]{region} \hyperref[TEI.roleName]{roleName} \hyperref[TEI.settlement]{settlement} \hyperref[TEI.surname]{surname}\par 
    \item[textstructure: ]
   \hyperref[TEI.docAuthor]{docAuthor} \hyperref[TEI.docDate]{docDate} \hyperref[TEI.docEdition]{docEdition} \hyperref[TEI.titlePart]{titlePart}\par 
    \item[transcr: ]
   \hyperref[TEI.am]{am} \hyperref[TEI.damage]{damage} \hyperref[TEI.fw]{fw} \hyperref[TEI.line]{line} \hyperref[TEI.metamark]{metamark} \hyperref[TEI.mod]{mod} \hyperref[TEI.restore]{restore} \hyperref[TEI.retrace]{retrace} \hyperref[TEI.secl]{secl} \hyperref[TEI.supplied]{supplied} \hyperref[TEI.surplus]{surplus} \hyperref[TEI.zone]{zone}
    \item[{Peut contenir}]
  
    \item[analysis: ]
   \hyperref[TEI.c]{c} \hyperref[TEI.cl]{cl} \hyperref[TEI.interp]{interp} \hyperref[TEI.interpGrp]{interpGrp} \hyperref[TEI.m]{m} \hyperref[TEI.pc]{pc} \hyperref[TEI.phr]{phr} \hyperref[TEI.s]{s} \hyperref[TEI.span]{span} \hyperref[TEI.spanGrp]{spanGrp} \hyperref[TEI.w]{w}\par 
    \item[core: ]
   \hyperref[TEI.abbr]{abbr} \hyperref[TEI.add]{add} \hyperref[TEI.address]{address} \hyperref[TEI.bibl]{bibl} \hyperref[TEI.biblStruct]{biblStruct} \hyperref[TEI.binaryObject]{binaryObject} \hyperref[TEI.cb]{cb} \hyperref[TEI.choice]{choice} \hyperref[TEI.cit]{cit} \hyperref[TEI.corr]{corr} \hyperref[TEI.date]{date} \hyperref[TEI.del]{del} \hyperref[TEI.desc]{desc} \hyperref[TEI.distinct]{distinct} \hyperref[TEI.email]{email} \hyperref[TEI.emph]{emph} \hyperref[TEI.expan]{expan} \hyperref[TEI.foreign]{foreign} \hyperref[TEI.gap]{gap} \hyperref[TEI.gb]{gb} \hyperref[TEI.gloss]{gloss} \hyperref[TEI.graphic]{graphic} \hyperref[TEI.hi]{hi} \hyperref[TEI.index]{index} \hyperref[TEI.l]{l} \hyperref[TEI.label]{label} \hyperref[TEI.lb]{lb} \hyperref[TEI.lg]{lg} \hyperref[TEI.list]{list} \hyperref[TEI.listBibl]{listBibl} \hyperref[TEI.measure]{measure} \hyperref[TEI.measureGrp]{measureGrp} \hyperref[TEI.media]{media} \hyperref[TEI.mentioned]{mentioned} \hyperref[TEI.milestone]{milestone} \hyperref[TEI.name]{name} \hyperref[TEI.note]{note} \hyperref[TEI.num]{num} \hyperref[TEI.orig]{orig} \hyperref[TEI.pb]{pb} \hyperref[TEI.ptr]{ptr} \hyperref[TEI.q]{q} \hyperref[TEI.quote]{quote} \hyperref[TEI.ref]{ref} \hyperref[TEI.reg]{reg} \hyperref[TEI.rs]{rs} \hyperref[TEI.said]{said} \hyperref[TEI.sic]{sic} \hyperref[TEI.soCalled]{soCalled} \hyperref[TEI.stage]{stage} \hyperref[TEI.term]{term} \hyperref[TEI.time]{time} \hyperref[TEI.title]{title} \hyperref[TEI.unclear]{unclear}\par 
    \item[derived-module-tei.istex: ]
   \hyperref[TEI.math]{math} \hyperref[TEI.mrow]{mrow}\par 
    \item[figures: ]
   \hyperref[TEI.figure]{figure} \hyperref[TEI.formula]{formula} \hyperref[TEI.notatedMusic]{notatedMusic} \hyperref[TEI.table]{table}\par 
    \item[header: ]
   \hyperref[TEI.biblFull]{biblFull} \hyperref[TEI.idno]{idno}\par 
    \item[iso-fs: ]
   \hyperref[TEI.fLib]{fLib} \hyperref[TEI.fs]{fs} \hyperref[TEI.fvLib]{fvLib}\par 
    \item[linking: ]
   \hyperref[TEI.alt]{alt} \hyperref[TEI.altGrp]{altGrp} \hyperref[TEI.anchor]{anchor} \hyperref[TEI.join]{join} \hyperref[TEI.joinGrp]{joinGrp} \hyperref[TEI.link]{link} \hyperref[TEI.linkGrp]{linkGrp} \hyperref[TEI.seg]{seg} \hyperref[TEI.timeline]{timeline}\par 
    \item[msdescription: ]
   \hyperref[TEI.catchwords]{catchwords} \hyperref[TEI.depth]{depth} \hyperref[TEI.dim]{dim} \hyperref[TEI.dimensions]{dimensions} \hyperref[TEI.height]{height} \hyperref[TEI.heraldry]{heraldry} \hyperref[TEI.locus]{locus} \hyperref[TEI.locusGrp]{locusGrp} \hyperref[TEI.material]{material} \hyperref[TEI.msDesc]{msDesc} \hyperref[TEI.objectType]{objectType} \hyperref[TEI.origDate]{origDate} \hyperref[TEI.origPlace]{origPlace} \hyperref[TEI.secFol]{secFol} \hyperref[TEI.signatures]{signatures} \hyperref[TEI.source]{source} \hyperref[TEI.stamp]{stamp} \hyperref[TEI.watermark]{watermark} \hyperref[TEI.width]{width}\par 
    \item[namesdates: ]
   \hyperref[TEI.addName]{addName} \hyperref[TEI.affiliation]{affiliation} \hyperref[TEI.country]{country} \hyperref[TEI.forename]{forename} \hyperref[TEI.genName]{genName} \hyperref[TEI.geogName]{geogName} \hyperref[TEI.listOrg]{listOrg} \hyperref[TEI.listPlace]{listPlace} \hyperref[TEI.location]{location} \hyperref[TEI.nameLink]{nameLink} \hyperref[TEI.orgName]{orgName} \hyperref[TEI.persName]{persName} \hyperref[TEI.placeName]{placeName} \hyperref[TEI.region]{region} \hyperref[TEI.roleName]{roleName} \hyperref[TEI.settlement]{settlement} \hyperref[TEI.state]{state} \hyperref[TEI.surname]{surname}\par 
    \item[spoken: ]
   \hyperref[TEI.annotationBlock]{annotationBlock}\par 
    \item[textstructure: ]
   \hyperref[TEI.floatingText]{floatingText}\par 
    \item[transcr: ]
   \hyperref[TEI.addSpan]{addSpan} \hyperref[TEI.am]{am} \hyperref[TEI.damage]{damage} \hyperref[TEI.damageSpan]{damageSpan} \hyperref[TEI.delSpan]{delSpan} \hyperref[TEI.ex]{ex} \hyperref[TEI.fw]{fw} \hyperref[TEI.handShift]{handShift} \hyperref[TEI.listTranspose]{listTranspose} \hyperref[TEI.metamark]{metamark} \hyperref[TEI.mod]{mod} \hyperref[TEI.redo]{redo} \hyperref[TEI.restore]{restore} \hyperref[TEI.retrace]{retrace} \hyperref[TEI.secl]{secl} \hyperref[TEI.space]{space} \hyperref[TEI.subst]{subst} \hyperref[TEI.substJoin]{substJoin} \hyperref[TEI.supplied]{supplied} \hyperref[TEI.surplus]{surplus} \hyperref[TEI.undo]{undo}\par des données textuelles
    \item[{Note}]
  \par
Puisque les dégâts causés aux témoins du texte les rendent fréquemment plus difficiles à lire, l'élément\hyperref[TEI.damage]{<damage>} contiendra souvent un élément \hyperref[TEI.unclear]{<unclear>}. Si la zone endommagée n'est pas continue (par exemple une tache affectant plusieurs morceaux de texte), on utilisera l'attribut{\itshape group} pour regrouper plusieurs éléments \hyperref[TEI.damage]{<damage>} ; alternativement, on utilisera l'élément \hyperref[TEI.join]{<join>}pour indiquer quels éléments \hyperref[TEI.damage]{<damage>} et \hyperref[TEI.unclear]{<unclear>} sont liés au même phénomène physique.\par
Les éléments \hyperref[TEI.damage]{<damage>}, \hyperref[TEI.gap]{<gap>}, \hyperref[TEI.del]{<del>}, \hyperref[TEI.unclear]{<unclear>} et\hyperref[TEI.supplied]{<supplied>} peuvent être utilisés en étroite association. Voir la section \xref{http://www.tei-c.org/release/doc/tei-p5-doc/en/html/PH.html\#PHCOMB}{11.3.3.2. Use of the gap, del, damage, unclear, and supplied Elements in Combination} pour savoir en quelle circonstance chacun de ces éléments est approprié.
    \item[{Exemple}]
  \leavevmode\bgroup\exampleFont \begin{shaded}\noindent\mbox{}{<\textbf{l}>}The Moving Finger wri{<\textbf{damage}\hspace*{6pt}{agent}="{water}"\hspace*{6pt}{group}="{1}">}es; and{</\textbf{damage}>} having writ,{</\textbf{l}>}\mbox{}\newline 
{<\textbf{l}>}Moves {<\textbf{damage}\hspace*{6pt}{agent}="{water}"\hspace*{6pt}{group}="{1}">}\mbox{}\newline 
\hspace*{6pt}\hspace*{6pt}{<\textbf{supplied}>}on: nor all your{</\textbf{supplied}>}\mbox{}\newline 
\hspace*{6pt}{</\textbf{damage}>} Piety nor Wit{</\textbf{l}>}\end{shaded}\egroup 


    \item[{Modèle de contenu}]
  \mbox{}\hfill\\[-10pt]\begin{Verbatim}[fontsize=\small]
<content>
 <macroRef key="macro.paraContent"/>
</content>
    
\end{Verbatim}

    \item[{Schéma Declaration}]
  \mbox{}\hfill\\[-10pt]\begin{Verbatim}[fontsize=\small]
element damage
{
   tei_att.global.attributes,
   tei_att.typed.attributes,
   tei_att.damaged.attributes,
   tei_macro.paraContent}
\end{Verbatim}

\end{reflist}  \index{damageSpan=<damageSpan>|oddindex}
\begin{reflist}
\item[]\begin{specHead}{TEI.damageSpan}{<damageSpan> }(partie de texte endommagée) marque le début d'une longue partie de texte, endommagée d'une manière quelconque mais toujours lisible. [\xref{http://www.tei-c.org/release/doc/tei-p5-doc/en/html/PH.html\#PHDA}{11.3.3.1. Damage, Illegibility, and Supplied Text}]\end{specHead} 
    \item[{Module}]
  transcr
    \item[{Attributs}]
  Attributs \hyperref[TEI.att.global]{att.global} (\textit{@xml:id}, \textit{@n}, \textit{@xml:lang}, \textit{@xml:base}, \textit{@xml:space})  (\hyperref[TEI.att.global.rendition]{att.global.rendition} (\textit{@rend}, \textit{@style}, \textit{@rendition})) (\hyperref[TEI.att.global.linking]{att.global.linking} (\textit{@corresp}, \textit{@synch}, \textit{@sameAs}, \textit{@copyOf}, \textit{@next}, \textit{@prev}, \textit{@exclude}, \textit{@select})) (\hyperref[TEI.att.global.analytic]{att.global.analytic} (\textit{@ana})) (\hyperref[TEI.att.global.facs]{att.global.facs} (\textit{@facs})) (\hyperref[TEI.att.global.change]{att.global.change} (\textit{@change})) (\hyperref[TEI.att.global.responsibility]{att.global.responsibility} (\textit{@cert}, \textit{@resp})) (\hyperref[TEI.att.global.source]{att.global.source} (\textit{@source})) \hyperref[TEI.att.damaged]{att.damaged} (\textit{@agent}, \textit{@degree}, \textit{@group})  (\hyperref[TEI.att.dimensions]{att.dimensions} (\textit{@unit}, \textit{@quantity}, \textit{@extent}, \textit{@precision}, \textit{@scope}) (\hyperref[TEI.att.ranging]{att.ranging} (\textit{@atLeast}, \textit{@atMost}, \textit{@min}, \textit{@max}, \textit{@confidence})) ) (\hyperref[TEI.att.written]{att.written} (\textit{@hand})) \hyperref[TEI.att.typed]{att.typed} (\textit{@type}, \textit{@subtype}) \hyperref[TEI.att.spanning]{att.spanning} (\textit{@spanTo}) 
    \item[{Membre du}]
  \hyperref[TEI.model.global.edit]{model.global.edit}
    \item[{Contenu dans}]
  
    \item[analysis: ]
   \hyperref[TEI.cl]{cl} \hyperref[TEI.m]{m} \hyperref[TEI.phr]{phr} \hyperref[TEI.s]{s} \hyperref[TEI.span]{span} \hyperref[TEI.w]{w}\par 
    \item[core: ]
   \hyperref[TEI.abbr]{abbr} \hyperref[TEI.add]{add} \hyperref[TEI.addrLine]{addrLine} \hyperref[TEI.address]{address} \hyperref[TEI.author]{author} \hyperref[TEI.bibl]{bibl} \hyperref[TEI.biblScope]{biblScope} \hyperref[TEI.cit]{cit} \hyperref[TEI.citedRange]{citedRange} \hyperref[TEI.corr]{corr} \hyperref[TEI.date]{date} \hyperref[TEI.del]{del} \hyperref[TEI.distinct]{distinct} \hyperref[TEI.editor]{editor} \hyperref[TEI.email]{email} \hyperref[TEI.emph]{emph} \hyperref[TEI.expan]{expan} \hyperref[TEI.foreign]{foreign} \hyperref[TEI.gloss]{gloss} \hyperref[TEI.head]{head} \hyperref[TEI.headItem]{headItem} \hyperref[TEI.headLabel]{headLabel} \hyperref[TEI.hi]{hi} \hyperref[TEI.imprint]{imprint} \hyperref[TEI.item]{item} \hyperref[TEI.l]{l} \hyperref[TEI.label]{label} \hyperref[TEI.lg]{lg} \hyperref[TEI.list]{list} \hyperref[TEI.measure]{measure} \hyperref[TEI.mentioned]{mentioned} \hyperref[TEI.name]{name} \hyperref[TEI.note]{note} \hyperref[TEI.num]{num} \hyperref[TEI.orig]{orig} \hyperref[TEI.p]{p} \hyperref[TEI.pubPlace]{pubPlace} \hyperref[TEI.publisher]{publisher} \hyperref[TEI.q]{q} \hyperref[TEI.quote]{quote} \hyperref[TEI.ref]{ref} \hyperref[TEI.reg]{reg} \hyperref[TEI.resp]{resp} \hyperref[TEI.rs]{rs} \hyperref[TEI.said]{said} \hyperref[TEI.series]{series} \hyperref[TEI.sic]{sic} \hyperref[TEI.soCalled]{soCalled} \hyperref[TEI.sp]{sp} \hyperref[TEI.speaker]{speaker} \hyperref[TEI.stage]{stage} \hyperref[TEI.street]{street} \hyperref[TEI.term]{term} \hyperref[TEI.textLang]{textLang} \hyperref[TEI.time]{time} \hyperref[TEI.title]{title} \hyperref[TEI.unclear]{unclear}\par 
    \item[figures: ]
   \hyperref[TEI.cell]{cell} \hyperref[TEI.figure]{figure} \hyperref[TEI.table]{table}\par 
    \item[header: ]
   \hyperref[TEI.authority]{authority} \hyperref[TEI.change]{change} \hyperref[TEI.classCode]{classCode} \hyperref[TEI.distributor]{distributor} \hyperref[TEI.edition]{edition} \hyperref[TEI.extent]{extent} \hyperref[TEI.funder]{funder} \hyperref[TEI.language]{language} \hyperref[TEI.licence]{licence}\par 
    \item[linking: ]
   \hyperref[TEI.ab]{ab} \hyperref[TEI.seg]{seg}\par 
    \item[msdescription: ]
   \hyperref[TEI.accMat]{accMat} \hyperref[TEI.acquisition]{acquisition} \hyperref[TEI.additions]{additions} \hyperref[TEI.catchwords]{catchwords} \hyperref[TEI.collation]{collation} \hyperref[TEI.colophon]{colophon} \hyperref[TEI.condition]{condition} \hyperref[TEI.custEvent]{custEvent} \hyperref[TEI.decoNote]{decoNote} \hyperref[TEI.explicit]{explicit} \hyperref[TEI.filiation]{filiation} \hyperref[TEI.finalRubric]{finalRubric} \hyperref[TEI.foliation]{foliation} \hyperref[TEI.heraldry]{heraldry} \hyperref[TEI.incipit]{incipit} \hyperref[TEI.layout]{layout} \hyperref[TEI.material]{material} \hyperref[TEI.msItem]{msItem} \hyperref[TEI.musicNotation]{musicNotation} \hyperref[TEI.objectType]{objectType} \hyperref[TEI.origDate]{origDate} \hyperref[TEI.origPlace]{origPlace} \hyperref[TEI.origin]{origin} \hyperref[TEI.provenance]{provenance} \hyperref[TEI.rubric]{rubric} \hyperref[TEI.secFol]{secFol} \hyperref[TEI.signatures]{signatures} \hyperref[TEI.source]{source} \hyperref[TEI.stamp]{stamp} \hyperref[TEI.summary]{summary} \hyperref[TEI.support]{support} \hyperref[TEI.surrogates]{surrogates} \hyperref[TEI.typeNote]{typeNote} \hyperref[TEI.watermark]{watermark}\par 
    \item[namesdates: ]
   \hyperref[TEI.addName]{addName} \hyperref[TEI.affiliation]{affiliation} \hyperref[TEI.country]{country} \hyperref[TEI.forename]{forename} \hyperref[TEI.genName]{genName} \hyperref[TEI.geogName]{geogName} \hyperref[TEI.nameLink]{nameLink} \hyperref[TEI.orgName]{orgName} \hyperref[TEI.persName]{persName} \hyperref[TEI.person]{person} \hyperref[TEI.personGrp]{personGrp} \hyperref[TEI.persona]{persona} \hyperref[TEI.placeName]{placeName} \hyperref[TEI.region]{region} \hyperref[TEI.roleName]{roleName} \hyperref[TEI.settlement]{settlement} \hyperref[TEI.surname]{surname}\par 
    \item[textstructure: ]
   \hyperref[TEI.back]{back} \hyperref[TEI.body]{body} \hyperref[TEI.div]{div} \hyperref[TEI.docAuthor]{docAuthor} \hyperref[TEI.docDate]{docDate} \hyperref[TEI.docEdition]{docEdition} \hyperref[TEI.docTitle]{docTitle} \hyperref[TEI.floatingText]{floatingText} \hyperref[TEI.front]{front} \hyperref[TEI.group]{group} \hyperref[TEI.text]{text} \hyperref[TEI.titlePage]{titlePage} \hyperref[TEI.titlePart]{titlePart}\par 
    \item[transcr: ]
   \hyperref[TEI.damage]{damage} \hyperref[TEI.fw]{fw} \hyperref[TEI.line]{line} \hyperref[TEI.metamark]{metamark} \hyperref[TEI.mod]{mod} \hyperref[TEI.restore]{restore} \hyperref[TEI.retrace]{retrace} \hyperref[TEI.secl]{secl} \hyperref[TEI.sourceDoc]{sourceDoc} \hyperref[TEI.supplied]{supplied} \hyperref[TEI.surface]{surface} \hyperref[TEI.surfaceGrp]{surfaceGrp} \hyperref[TEI.surplus]{surplus} \hyperref[TEI.zone]{zone}
    \item[{Peut contenir}]
  Elément vide
    \item[{Note}]
  \par
Le début et la fin de la partie de texte endommagée doivent être marqués : le début, par l'élément \hyperref[TEI.damageSpan]{<damageSpan>}, la fin au moyen de la cible de l'attribut {\itshape spanTo} : si aucun autre élément n'est disponible, l'élément \hyperref[TEI.anchor]{<anchor>} est utilisé à cette fin.\par
Le texte endommagé doit être au moins partiellement lisible, afin que l'encodeur soit capable de le transcrire. S'il n'est pas lisible du tout, l'élément \hyperref[TEI.damageSpan]{<damageSpan>} ne devrait pas être utilisé. L'élément \hyperref[TEI.gap]{<gap>} ou \hyperref[TEI.unclear]{<unclear>} devrait être plutôt employé, avec un attribut {\itshape reason} dont la valeur donnerait la cause de cette lecture impossible. Voir les autres sections \xref{http://www.tei-c.org/release/doc/tei-p5-doc/en/html/PH.html\#PHDA}{11.3.3.1. Damage, Illegibility, and Supplied Text} et \xref{http://www.tei-c.org/release/doc/tei-p5-doc/en/html/PH.html\#PHCOMB}{11.3.3.2. Use of the gap, del, damage, unclear, and supplied Elements in Combination}.
    \item[{Exemple}]
  \leavevmode\bgroup\exampleFont \begin{shaded}\noindent\mbox{}{<\textbf{p}>}Paragraph partially damaged. This is the undamaged portion {<\textbf{damageSpan}\hspace*{6pt}{spanTo}="{\#fr\textunderscore a34}"/>}and\mbox{}\newline 
 this the damaged portion of the paragraph.{</\textbf{p}>}\mbox{}\newline 
{<\textbf{p}>}This paragraph is entirely damaged.{</\textbf{p}>}\mbox{}\newline 
{<\textbf{p}>}Paragraph partially damaged; in the middle of this paragraph the damage ends and the\mbox{}\newline 
 anchor point marks the start of the {<\textbf{anchor}\hspace*{6pt}{xml:id}="{fr\textunderscore a34}"/>} undamaged part of the text.\mbox{}\newline 
 ...{</\textbf{p}>}\end{shaded}\egroup 


    \item[{Schematron}]
   <s:assert test="@spanTo">The @spanTo attribute of <s:name/> is required.</s:assert>
    \item[{Schematron}]
   <s:assert test="@spanTo">L'attribut spanTo est requis.</s:assert>
    \item[{Modèle de contenu}]
  \fbox{\ttfamily <content>\newline
</content>\newline
    } 
    \item[{Schéma Declaration}]
  \mbox{}\hfill\\[-10pt]\begin{Verbatim}[fontsize=\small]
element damageSpan
{
   tei_att.global.attributes,
   tei_att.damaged.attributes,
   tei_att.typed.attributes,
   tei_att.spanning.attributes,
   empty
}
\end{Verbatim}

\end{reflist}  \index{date=<date>|oddindex}\index{scheme=@scheme!<date>|oddindex}
\begin{reflist}
\item[]\begin{specHead}{TEI.date}{<date> }(date) contient une date exprimée dans n'importe quel format. [\xref{http://www.tei-c.org/release/doc/tei-p5-doc/en/html/CO.html\#CONADA}{3.5.4. Dates and Times} \xref{http://www.tei-c.org/release/doc/tei-p5-doc/en/html/HD.html\#HD24}{2.2.4. Publication, Distribution, Licensing, etc.} \xref{http://www.tei-c.org/release/doc/tei-p5-doc/en/html/HD.html\#HD6}{2.6. The Revision Description} \xref{http://www.tei-c.org/release/doc/tei-p5-doc/en/html/CO.html\#COBICOI}{3.11.2.4. Imprint, Size of a Document, and Reprint Information} \xref{http://www.tei-c.org/release/doc/tei-p5-doc/en/html/CC.html\#CCAHSE}{15.2.3. The Setting Description} \xref{http://www.tei-c.org/release/doc/tei-p5-doc/en/html/ND.html\#NDDATE}{13.3.6. Dates and Times}]\end{specHead} 
    \item[{Module}]
  core
    \item[{Attributs}]
  Attributs \hyperref[TEI.att.global]{att.global} (\textit{@xml:id}, \textit{@n}, \textit{@xml:lang}, \textit{@xml:base}, \textit{@xml:space})  (\hyperref[TEI.att.global.rendition]{att.global.rendition} (\textit{@rend}, \textit{@style}, \textit{@rendition})) (\hyperref[TEI.att.global.linking]{att.global.linking} (\textit{@corresp}, \textit{@synch}, \textit{@sameAs}, \textit{@copyOf}, \textit{@next}, \textit{@prev}, \textit{@exclude}, \textit{@select})) (\hyperref[TEI.att.global.analytic]{att.global.analytic} (\textit{@ana})) (\hyperref[TEI.att.global.facs]{att.global.facs} (\textit{@facs})) (\hyperref[TEI.att.global.change]{att.global.change} (\textit{@change})) (\hyperref[TEI.att.global.responsibility]{att.global.responsibility} (\textit{@cert}, \textit{@resp})) (\hyperref[TEI.att.global.source]{att.global.source} (\textit{@source})) \hyperref[TEI.att.datable]{att.datable} (\textit{@calendar}, \textit{@period})  (\hyperref[TEI.att.datable.w3c]{att.datable.w3c} (\textit{@when}, \textit{@notBefore}, \textit{@notAfter}, \textit{@from}, \textit{@to})) (\hyperref[TEI.att.datable.iso]{att.datable.iso} (\textit{@when-iso}, \textit{@notBefore-iso}, \textit{@notAfter-iso}, \textit{@from-iso}, \textit{@to-iso})) (\hyperref[TEI.att.datable.custom]{att.datable.custom} (\textit{@when-custom}, \textit{@notBefore-custom}, \textit{@notAfter-custom}, \textit{@from-custom}, \textit{@to-custom}, \textit{@datingPoint}, \textit{@datingMethod})) \hyperref[TEI.att.duration]{att.duration} (\hyperref[TEI.att.duration.w3c]{att.duration.w3c} (\textit{@dur})) (\hyperref[TEI.att.duration.iso]{att.duration.iso} (\textit{@dur-iso})) \hyperref[TEI.att.editLike]{att.editLike} (\textit{@evidence}, \textit{@instant})  (\hyperref[TEI.att.dimensions]{att.dimensions} (\textit{@unit}, \textit{@quantity}, \textit{@extent}, \textit{@precision}, \textit{@scope}) (\hyperref[TEI.att.ranging]{att.ranging} (\textit{@atLeast}, \textit{@atMost}, \textit{@min}, \textit{@max}, \textit{@confidence})) ) \hyperref[TEI.att.typed]{att.typed} (\textit{@type}, \textit{@subtype}) \hfil\\[-10pt]\begin{sansreflist}
    \item[@scheme]
  désigne la liste des ontologies dans lequel l'ensemble des termes concernés sont définis.
\begin{reflist}
    \item[{Statut}]
  Optionel
    \item[{Type de données}]
  \hyperref[TEI.teidata.pointer]{teidata.pointer}
\end{reflist}  
\end{sansreflist}  
    \item[{Membre du}]
  \hyperref[TEI.model.OABody]{model.OABody} \hyperref[TEI.model.dateLike]{model.dateLike} \hyperref[TEI.model.publicationStmtPart.detail]{model.publicationStmtPart.detail}
    \item[{Contenu dans}]
  
    \item[analysis: ]
   \hyperref[TEI.cl]{cl} \hyperref[TEI.phr]{phr} \hyperref[TEI.s]{s} \hyperref[TEI.span]{span}\par 
    \item[core: ]
   \hyperref[TEI.abbr]{abbr} \hyperref[TEI.add]{add} \hyperref[TEI.addrLine]{addrLine} \hyperref[TEI.analytic]{analytic} \hyperref[TEI.author]{author} \hyperref[TEI.bibl]{bibl} \hyperref[TEI.biblScope]{biblScope} \hyperref[TEI.citedRange]{citedRange} \hyperref[TEI.corr]{corr} \hyperref[TEI.date]{date} \hyperref[TEI.del]{del} \hyperref[TEI.desc]{desc} \hyperref[TEI.distinct]{distinct} \hyperref[TEI.editor]{editor} \hyperref[TEI.email]{email} \hyperref[TEI.emph]{emph} \hyperref[TEI.expan]{expan} \hyperref[TEI.foreign]{foreign} \hyperref[TEI.gloss]{gloss} \hyperref[TEI.head]{head} \hyperref[TEI.headItem]{headItem} \hyperref[TEI.headLabel]{headLabel} \hyperref[TEI.hi]{hi} \hyperref[TEI.imprint]{imprint} \hyperref[TEI.item]{item} \hyperref[TEI.l]{l} \hyperref[TEI.label]{label} \hyperref[TEI.measure]{measure} \hyperref[TEI.meeting]{meeting} \hyperref[TEI.mentioned]{mentioned} \hyperref[TEI.name]{name} \hyperref[TEI.note]{note} \hyperref[TEI.num]{num} \hyperref[TEI.orig]{orig} \hyperref[TEI.p]{p} \hyperref[TEI.pubPlace]{pubPlace} \hyperref[TEI.publisher]{publisher} \hyperref[TEI.q]{q} \hyperref[TEI.quote]{quote} \hyperref[TEI.ref]{ref} \hyperref[TEI.reg]{reg} \hyperref[TEI.resp]{resp} \hyperref[TEI.rs]{rs} \hyperref[TEI.said]{said} \hyperref[TEI.sic]{sic} \hyperref[TEI.soCalled]{soCalled} \hyperref[TEI.speaker]{speaker} \hyperref[TEI.stage]{stage} \hyperref[TEI.street]{street} \hyperref[TEI.term]{term} \hyperref[TEI.textLang]{textLang} \hyperref[TEI.time]{time} \hyperref[TEI.title]{title} \hyperref[TEI.unclear]{unclear}\par 
    \item[figures: ]
   \hyperref[TEI.cell]{cell} \hyperref[TEI.figDesc]{figDesc}\par 
    \item[header: ]
   \hyperref[TEI.authority]{authority} \hyperref[TEI.change]{change} \hyperref[TEI.classCode]{classCode} \hyperref[TEI.creation]{creation} \hyperref[TEI.distributor]{distributor} \hyperref[TEI.edition]{edition} \hyperref[TEI.extent]{extent} \hyperref[TEI.funder]{funder} \hyperref[TEI.language]{language} \hyperref[TEI.licence]{licence} \hyperref[TEI.publicationStmt]{publicationStmt} \hyperref[TEI.rendition]{rendition}\par 
    \item[iso-fs: ]
   \hyperref[TEI.fDescr]{fDescr} \hyperref[TEI.fsDescr]{fsDescr}\par 
    \item[linking: ]
   \hyperref[TEI.ab]{ab} \hyperref[TEI.seg]{seg}\par 
    \item[msdescription: ]
   \hyperref[TEI.accMat]{accMat} \hyperref[TEI.acquisition]{acquisition} \hyperref[TEI.additions]{additions} \hyperref[TEI.catchwords]{catchwords} \hyperref[TEI.collation]{collation} \hyperref[TEI.colophon]{colophon} \hyperref[TEI.condition]{condition} \hyperref[TEI.custEvent]{custEvent} \hyperref[TEI.decoNote]{decoNote} \hyperref[TEI.explicit]{explicit} \hyperref[TEI.filiation]{filiation} \hyperref[TEI.finalRubric]{finalRubric} \hyperref[TEI.foliation]{foliation} \hyperref[TEI.heraldry]{heraldry} \hyperref[TEI.incipit]{incipit} \hyperref[TEI.layout]{layout} \hyperref[TEI.material]{material} \hyperref[TEI.musicNotation]{musicNotation} \hyperref[TEI.objectType]{objectType} \hyperref[TEI.origDate]{origDate} \hyperref[TEI.origPlace]{origPlace} \hyperref[TEI.origin]{origin} \hyperref[TEI.provenance]{provenance} \hyperref[TEI.rubric]{rubric} \hyperref[TEI.secFol]{secFol} \hyperref[TEI.signatures]{signatures} \hyperref[TEI.source]{source} \hyperref[TEI.stamp]{stamp} \hyperref[TEI.summary]{summary} \hyperref[TEI.support]{support} \hyperref[TEI.surrogates]{surrogates} \hyperref[TEI.typeNote]{typeNote} \hyperref[TEI.watermark]{watermark}\par 
    \item[namesdates: ]
   \hyperref[TEI.addName]{addName} \hyperref[TEI.affiliation]{affiliation} \hyperref[TEI.country]{country} \hyperref[TEI.forename]{forename} \hyperref[TEI.genName]{genName} \hyperref[TEI.geogName]{geogName} \hyperref[TEI.nameLink]{nameLink} \hyperref[TEI.orgName]{orgName} \hyperref[TEI.persName]{persName} \hyperref[TEI.placeName]{placeName} \hyperref[TEI.region]{region} \hyperref[TEI.roleName]{roleName} \hyperref[TEI.settlement]{settlement} \hyperref[TEI.surname]{surname}\par 
    \item[spoken: ]
   \hyperref[TEI.annotationBlock]{annotationBlock}\par 
    \item[standOff: ]
   \hyperref[TEI.listAnnotation]{listAnnotation}\par 
    \item[textstructure: ]
   \hyperref[TEI.docAuthor]{docAuthor} \hyperref[TEI.docDate]{docDate} \hyperref[TEI.docEdition]{docEdition} \hyperref[TEI.titlePart]{titlePart}\par 
    \item[transcr: ]
   \hyperref[TEI.damage]{damage} \hyperref[TEI.fw]{fw} \hyperref[TEI.metamark]{metamark} \hyperref[TEI.mod]{mod} \hyperref[TEI.restore]{restore} \hyperref[TEI.retrace]{retrace} \hyperref[TEI.secl]{secl} \hyperref[TEI.supplied]{supplied} \hyperref[TEI.surplus]{surplus}
    \item[{Peut contenir}]
  
    \item[analysis: ]
   \hyperref[TEI.c]{c} \hyperref[TEI.cl]{cl} \hyperref[TEI.interp]{interp} \hyperref[TEI.interpGrp]{interpGrp} \hyperref[TEI.m]{m} \hyperref[TEI.pc]{pc} \hyperref[TEI.phr]{phr} \hyperref[TEI.s]{s} \hyperref[TEI.span]{span} \hyperref[TEI.spanGrp]{spanGrp} \hyperref[TEI.w]{w}\par 
    \item[core: ]
   \hyperref[TEI.abbr]{abbr} \hyperref[TEI.add]{add} \hyperref[TEI.address]{address} \hyperref[TEI.binaryObject]{binaryObject} \hyperref[TEI.cb]{cb} \hyperref[TEI.choice]{choice} \hyperref[TEI.corr]{corr} \hyperref[TEI.date]{date} \hyperref[TEI.del]{del} \hyperref[TEI.distinct]{distinct} \hyperref[TEI.email]{email} \hyperref[TEI.emph]{emph} \hyperref[TEI.expan]{expan} \hyperref[TEI.foreign]{foreign} \hyperref[TEI.gap]{gap} \hyperref[TEI.gb]{gb} \hyperref[TEI.gloss]{gloss} \hyperref[TEI.graphic]{graphic} \hyperref[TEI.hi]{hi} \hyperref[TEI.index]{index} \hyperref[TEI.lb]{lb} \hyperref[TEI.measure]{measure} \hyperref[TEI.measureGrp]{measureGrp} \hyperref[TEI.media]{media} \hyperref[TEI.mentioned]{mentioned} \hyperref[TEI.milestone]{milestone} \hyperref[TEI.name]{name} \hyperref[TEI.note]{note} \hyperref[TEI.num]{num} \hyperref[TEI.orig]{orig} \hyperref[TEI.pb]{pb} \hyperref[TEI.ptr]{ptr} \hyperref[TEI.ref]{ref} \hyperref[TEI.reg]{reg} \hyperref[TEI.rs]{rs} \hyperref[TEI.sic]{sic} \hyperref[TEI.soCalled]{soCalled} \hyperref[TEI.term]{term} \hyperref[TEI.time]{time} \hyperref[TEI.title]{title} \hyperref[TEI.unclear]{unclear}\par 
    \item[derived-module-tei.istex: ]
   \hyperref[TEI.math]{math} \hyperref[TEI.mrow]{mrow}\par 
    \item[figures: ]
   \hyperref[TEI.figure]{figure} \hyperref[TEI.formula]{formula} \hyperref[TEI.notatedMusic]{notatedMusic}\par 
    \item[header: ]
   \hyperref[TEI.idno]{idno}\par 
    \item[iso-fs: ]
   \hyperref[TEI.fLib]{fLib} \hyperref[TEI.fs]{fs} \hyperref[TEI.fvLib]{fvLib}\par 
    \item[linking: ]
   \hyperref[TEI.alt]{alt} \hyperref[TEI.altGrp]{altGrp} \hyperref[TEI.anchor]{anchor} \hyperref[TEI.join]{join} \hyperref[TEI.joinGrp]{joinGrp} \hyperref[TEI.link]{link} \hyperref[TEI.linkGrp]{linkGrp} \hyperref[TEI.seg]{seg} \hyperref[TEI.timeline]{timeline}\par 
    \item[msdescription: ]
   \hyperref[TEI.catchwords]{catchwords} \hyperref[TEI.depth]{depth} \hyperref[TEI.dim]{dim} \hyperref[TEI.dimensions]{dimensions} \hyperref[TEI.height]{height} \hyperref[TEI.heraldry]{heraldry} \hyperref[TEI.locus]{locus} \hyperref[TEI.locusGrp]{locusGrp} \hyperref[TEI.material]{material} \hyperref[TEI.objectType]{objectType} \hyperref[TEI.origDate]{origDate} \hyperref[TEI.origPlace]{origPlace} \hyperref[TEI.secFol]{secFol} \hyperref[TEI.signatures]{signatures} \hyperref[TEI.source]{source} \hyperref[TEI.stamp]{stamp} \hyperref[TEI.watermark]{watermark} \hyperref[TEI.width]{width}\par 
    \item[namesdates: ]
   \hyperref[TEI.addName]{addName} \hyperref[TEI.affiliation]{affiliation} \hyperref[TEI.country]{country} \hyperref[TEI.forename]{forename} \hyperref[TEI.genName]{genName} \hyperref[TEI.geogName]{geogName} \hyperref[TEI.location]{location} \hyperref[TEI.nameLink]{nameLink} \hyperref[TEI.orgName]{orgName} \hyperref[TEI.persName]{persName} \hyperref[TEI.placeName]{placeName} \hyperref[TEI.region]{region} \hyperref[TEI.roleName]{roleName} \hyperref[TEI.settlement]{settlement} \hyperref[TEI.state]{state} \hyperref[TEI.surname]{surname}\par 
    \item[spoken: ]
   \hyperref[TEI.annotationBlock]{annotationBlock}\par 
    \item[transcr: ]
   \hyperref[TEI.addSpan]{addSpan} \hyperref[TEI.am]{am} \hyperref[TEI.damage]{damage} \hyperref[TEI.damageSpan]{damageSpan} \hyperref[TEI.delSpan]{delSpan} \hyperref[TEI.ex]{ex} \hyperref[TEI.fw]{fw} \hyperref[TEI.handShift]{handShift} \hyperref[TEI.listTranspose]{listTranspose} \hyperref[TEI.metamark]{metamark} \hyperref[TEI.mod]{mod} \hyperref[TEI.redo]{redo} \hyperref[TEI.restore]{restore} \hyperref[TEI.retrace]{retrace} \hyperref[TEI.secl]{secl} \hyperref[TEI.space]{space} \hyperref[TEI.subst]{subst} \hyperref[TEI.substJoin]{substJoin} \hyperref[TEI.supplied]{supplied} \hyperref[TEI.surplus]{surplus} \hyperref[TEI.undo]{undo}\par des données textuelles
    \item[{Exemple}]
  StandOff entité nommée Date\leavevmode\bgroup\exampleFont \begin{shaded}\noindent\mbox{}{<\textbf{annotationBlock}\hspace*{6pt}{corresp}="{text}">}\mbox{}\newline 
\hspace*{6pt}{<\textbf{date}\hspace*{6pt}{change}="{\#Unitex-3.2.0-alpha}"\mbox{}\newline 
\hspace*{6pt}\hspace*{6pt}{resp}="{istex}"\mbox{}\newline 
\hspace*{6pt}\hspace*{6pt}{scheme}="{https://date-entity.data.istex.fr}">}\mbox{}\newline 
\hspace*{6pt}\hspace*{6pt}{<\textbf{term}>}1997{</\textbf{term}>}\mbox{}\newline 
\hspace*{6pt}\hspace*{6pt}{<\textbf{fs}\hspace*{6pt}{type}="{statistics}">}\mbox{}\newline 
\hspace*{6pt}\hspace*{6pt}\hspace*{6pt}{<\textbf{f}\hspace*{6pt}{name}="{frequency}">}\mbox{}\newline 
\hspace*{6pt}\hspace*{6pt}\hspace*{6pt}\hspace*{6pt}{<\textbf{numeric}\hspace*{6pt}{value}="{7}"/>}\mbox{}\newline 
\hspace*{6pt}\hspace*{6pt}\hspace*{6pt}{</\textbf{f}>}\mbox{}\newline 
\hspace*{6pt}\hspace*{6pt}{</\textbf{fs}>}\mbox{}\newline 
\hspace*{6pt}{</\textbf{date}>}\mbox{}\newline 
{</\textbf{annotationBlock}>}\end{shaded}\egroup 


    \item[{Modèle de contenu}]
  \mbox{}\hfill\\[-10pt]\begin{Verbatim}[fontsize=\small]
<content>
 <alternate maxOccurs="unbounded"
  minOccurs="0">
  <textNode/>
  <classRef key="model.gLike"/>
  <classRef key="model.phrase"/>
  <classRef key="model.global"/>
 </alternate>
</content>
    
\end{Verbatim}

    \item[{Schéma Declaration}]
  \mbox{}\hfill\\[-10pt]\begin{Verbatim}[fontsize=\small]
element date
{
   tei_att.global.attributes,
   tei_att.datable.attributes,
   tei_att.duration.attributes,
   tei_att.editLike.attributes,
   tei_att.typed.attributes,
   attribute scheme { text }?,
   ( text | tei_model.gLike | tei_model.phrase | tei_model.global )*
}
\end{Verbatim}

\end{reflist}  \index{decoDesc=<decoDesc>|oddindex}
\begin{reflist}
\item[]\begin{specHead}{TEI.decoDesc}{<decoDesc> }(description de la décoration) contient la description de la décoration du manuscrit, soit en une série de paragraphes \textit{p}, soit sous la forme d'une série d'éléments thématiques \hyperref[TEI.decoNote]{<decoNote>} [\xref{http://www.tei-c.org/release/doc/tei-p5-doc/en/html/MS.html\#msph3}{10.7.3. Bindings, Seals, and Additional Material}]\end{specHead} 
    \item[{Module}]
  msdescription
    \item[{Attributs}]
  Attributs \hyperref[TEI.att.global]{att.global} (\textit{@xml:id}, \textit{@n}, \textit{@xml:lang}, \textit{@xml:base}, \textit{@xml:space})  (\hyperref[TEI.att.global.rendition]{att.global.rendition} (\textit{@rend}, \textit{@style}, \textit{@rendition})) (\hyperref[TEI.att.global.linking]{att.global.linking} (\textit{@corresp}, \textit{@synch}, \textit{@sameAs}, \textit{@copyOf}, \textit{@next}, \textit{@prev}, \textit{@exclude}, \textit{@select})) (\hyperref[TEI.att.global.analytic]{att.global.analytic} (\textit{@ana})) (\hyperref[TEI.att.global.facs]{att.global.facs} (\textit{@facs})) (\hyperref[TEI.att.global.change]{att.global.change} (\textit{@change})) (\hyperref[TEI.att.global.responsibility]{att.global.responsibility} (\textit{@cert}, \textit{@resp})) (\hyperref[TEI.att.global.source]{att.global.source} (\textit{@source}))
    \item[{Membre du}]
  \hyperref[TEI.model.physDescPart]{model.physDescPart}
    \item[{Contenu dans}]
  
    \item[msdescription: ]
   \hyperref[TEI.physDesc]{physDesc}
    \item[{Peut contenir}]
  
    \item[core: ]
   \hyperref[TEI.p]{p}\par 
    \item[linking: ]
   \hyperref[TEI.ab]{ab}\par 
    \item[msdescription: ]
   \hyperref[TEI.decoNote]{decoNote} \hyperref[TEI.summary]{summary}
    \item[{Exemple}]
  \leavevmode\bgroup\exampleFont \begin{shaded}\noindent\mbox{}{<\textbf{decoDesc}>}\mbox{}\newline 
\hspace*{6pt}{<\textbf{p}>}Les miracles de la Vierge, par Gautier de Coinci ; un volume in-fol. de 285 feuilles,\mbox{}\newline 
\hspace*{6pt}\hspace*{6pt} orné d'initiales en or et en couleur,...{</\textbf{p}>}\mbox{}\newline 
{</\textbf{decoDesc}>}\end{shaded}\egroup 


    \item[{Modèle de contenu}]
  \mbox{}\hfill\\[-10pt]\begin{Verbatim}[fontsize=\small]
<content>
 <alternate maxOccurs="1" minOccurs="1">
  <classRef key="model.pLike"
   maxOccurs="unbounded" minOccurs="1"/>
  <sequence maxOccurs="1" minOccurs="1">
   <elementRef key="summary" minOccurs="0"/>
   <elementRef key="decoNote"
    maxOccurs="unbounded" minOccurs="1"/>
  </sequence>
 </alternate>
</content>
    
\end{Verbatim}

    \item[{Schéma Declaration}]
  \mbox{}\hfill\\[-10pt]\begin{Verbatim}[fontsize=\small]
element decoDesc
{
   tei_att.global.attributes,
   ( tei_model.pLike+ | ( tei_summary?, tei_decoNote+ ) )
}
\end{Verbatim}

\end{reflist}  \index{decoNote=<decoNote>|oddindex}
\begin{reflist}
\item[]\begin{specHead}{TEI.decoNote}{<decoNote> }(note sur un élément de décoration) contient une note décrivant soit un élément de décoration du mansucrit, soit une catégorie relativement homogène de tels éléments. [\xref{http://www.tei-c.org/release/doc/tei-p5-doc/en/html/MS.html\#msph3}{10.7.3. Bindings, Seals, and Additional Material}]\end{specHead} 
    \item[{Module}]
  msdescription
    \item[{Attributs}]
  Attributs \hyperref[TEI.att.global]{att.global} (\textit{@xml:id}, \textit{@n}, \textit{@xml:lang}, \textit{@xml:base}, \textit{@xml:space})  (\hyperref[TEI.att.global.rendition]{att.global.rendition} (\textit{@rend}, \textit{@style}, \textit{@rendition})) (\hyperref[TEI.att.global.linking]{att.global.linking} (\textit{@corresp}, \textit{@synch}, \textit{@sameAs}, \textit{@copyOf}, \textit{@next}, \textit{@prev}, \textit{@exclude}, \textit{@select})) (\hyperref[TEI.att.global.analytic]{att.global.analytic} (\textit{@ana})) (\hyperref[TEI.att.global.facs]{att.global.facs} (\textit{@facs})) (\hyperref[TEI.att.global.change]{att.global.change} (\textit{@change})) (\hyperref[TEI.att.global.responsibility]{att.global.responsibility} (\textit{@cert}, \textit{@resp})) (\hyperref[TEI.att.global.source]{att.global.source} (\textit{@source})) \hyperref[TEI.att.typed]{att.typed} (\textit{@type}, \textit{@subtype}) 
    \item[{Membre du}]
  \hyperref[TEI.model.msItemPart]{model.msItemPart} 
    \item[{Contenu dans}]
  
    \item[msdescription: ]
   \hyperref[TEI.binding]{binding} \hyperref[TEI.bindingDesc]{bindingDesc} \hyperref[TEI.decoDesc]{decoDesc} \hyperref[TEI.msItem]{msItem} \hyperref[TEI.msItemStruct]{msItemStruct} \hyperref[TEI.seal]{seal} \hyperref[TEI.sealDesc]{sealDesc}
    \item[{Peut contenir}]
  
    \item[analysis: ]
   \hyperref[TEI.c]{c} \hyperref[TEI.cl]{cl} \hyperref[TEI.interp]{interp} \hyperref[TEI.interpGrp]{interpGrp} \hyperref[TEI.m]{m} \hyperref[TEI.pc]{pc} \hyperref[TEI.phr]{phr} \hyperref[TEI.s]{s} \hyperref[TEI.span]{span} \hyperref[TEI.spanGrp]{spanGrp} \hyperref[TEI.w]{w}\par 
    \item[core: ]
   \hyperref[TEI.abbr]{abbr} \hyperref[TEI.add]{add} \hyperref[TEI.address]{address} \hyperref[TEI.bibl]{bibl} \hyperref[TEI.biblStruct]{biblStruct} \hyperref[TEI.binaryObject]{binaryObject} \hyperref[TEI.cb]{cb} \hyperref[TEI.choice]{choice} \hyperref[TEI.cit]{cit} \hyperref[TEI.corr]{corr} \hyperref[TEI.date]{date} \hyperref[TEI.del]{del} \hyperref[TEI.desc]{desc} \hyperref[TEI.distinct]{distinct} \hyperref[TEI.email]{email} \hyperref[TEI.emph]{emph} \hyperref[TEI.expan]{expan} \hyperref[TEI.foreign]{foreign} \hyperref[TEI.gap]{gap} \hyperref[TEI.gb]{gb} \hyperref[TEI.gloss]{gloss} \hyperref[TEI.graphic]{graphic} \hyperref[TEI.hi]{hi} \hyperref[TEI.index]{index} \hyperref[TEI.l]{l} \hyperref[TEI.label]{label} \hyperref[TEI.lb]{lb} \hyperref[TEI.lg]{lg} \hyperref[TEI.list]{list} \hyperref[TEI.listBibl]{listBibl} \hyperref[TEI.measure]{measure} \hyperref[TEI.measureGrp]{measureGrp} \hyperref[TEI.media]{media} \hyperref[TEI.mentioned]{mentioned} \hyperref[TEI.milestone]{milestone} \hyperref[TEI.name]{name} \hyperref[TEI.note]{note} \hyperref[TEI.num]{num} \hyperref[TEI.orig]{orig} \hyperref[TEI.p]{p} \hyperref[TEI.pb]{pb} \hyperref[TEI.ptr]{ptr} \hyperref[TEI.q]{q} \hyperref[TEI.quote]{quote} \hyperref[TEI.ref]{ref} \hyperref[TEI.reg]{reg} \hyperref[TEI.rs]{rs} \hyperref[TEI.said]{said} \hyperref[TEI.sic]{sic} \hyperref[TEI.soCalled]{soCalled} \hyperref[TEI.sp]{sp} \hyperref[TEI.stage]{stage} \hyperref[TEI.term]{term} \hyperref[TEI.time]{time} \hyperref[TEI.title]{title} \hyperref[TEI.unclear]{unclear}\par 
    \item[derived-module-tei.istex: ]
   \hyperref[TEI.math]{math} \hyperref[TEI.mrow]{mrow}\par 
    \item[figures: ]
   \hyperref[TEI.figure]{figure} \hyperref[TEI.formula]{formula} \hyperref[TEI.notatedMusic]{notatedMusic} \hyperref[TEI.table]{table}\par 
    \item[header: ]
   \hyperref[TEI.biblFull]{biblFull} \hyperref[TEI.idno]{idno}\par 
    \item[iso-fs: ]
   \hyperref[TEI.fLib]{fLib} \hyperref[TEI.fs]{fs} \hyperref[TEI.fvLib]{fvLib}\par 
    \item[linking: ]
   \hyperref[TEI.ab]{ab} \hyperref[TEI.alt]{alt} \hyperref[TEI.altGrp]{altGrp} \hyperref[TEI.anchor]{anchor} \hyperref[TEI.join]{join} \hyperref[TEI.joinGrp]{joinGrp} \hyperref[TEI.link]{link} \hyperref[TEI.linkGrp]{linkGrp} \hyperref[TEI.seg]{seg} \hyperref[TEI.timeline]{timeline}\par 
    \item[msdescription: ]
   \hyperref[TEI.catchwords]{catchwords} \hyperref[TEI.depth]{depth} \hyperref[TEI.dim]{dim} \hyperref[TEI.dimensions]{dimensions} \hyperref[TEI.height]{height} \hyperref[TEI.heraldry]{heraldry} \hyperref[TEI.locus]{locus} \hyperref[TEI.locusGrp]{locusGrp} \hyperref[TEI.material]{material} \hyperref[TEI.msDesc]{msDesc} \hyperref[TEI.objectType]{objectType} \hyperref[TEI.origDate]{origDate} \hyperref[TEI.origPlace]{origPlace} \hyperref[TEI.secFol]{secFol} \hyperref[TEI.signatures]{signatures} \hyperref[TEI.source]{source} \hyperref[TEI.stamp]{stamp} \hyperref[TEI.watermark]{watermark} \hyperref[TEI.width]{width}\par 
    \item[namesdates: ]
   \hyperref[TEI.addName]{addName} \hyperref[TEI.affiliation]{affiliation} \hyperref[TEI.country]{country} \hyperref[TEI.forename]{forename} \hyperref[TEI.genName]{genName} \hyperref[TEI.geogName]{geogName} \hyperref[TEI.listOrg]{listOrg} \hyperref[TEI.listPlace]{listPlace} \hyperref[TEI.location]{location} \hyperref[TEI.nameLink]{nameLink} \hyperref[TEI.orgName]{orgName} \hyperref[TEI.persName]{persName} \hyperref[TEI.placeName]{placeName} \hyperref[TEI.region]{region} \hyperref[TEI.roleName]{roleName} \hyperref[TEI.settlement]{settlement} \hyperref[TEI.state]{state} \hyperref[TEI.surname]{surname}\par 
    \item[spoken: ]
   \hyperref[TEI.annotationBlock]{annotationBlock}\par 
    \item[textstructure: ]
   \hyperref[TEI.floatingText]{floatingText}\par 
    \item[transcr: ]
   \hyperref[TEI.addSpan]{addSpan} \hyperref[TEI.am]{am} \hyperref[TEI.damage]{damage} \hyperref[TEI.damageSpan]{damageSpan} \hyperref[TEI.delSpan]{delSpan} \hyperref[TEI.ex]{ex} \hyperref[TEI.fw]{fw} \hyperref[TEI.handShift]{handShift} \hyperref[TEI.listTranspose]{listTranspose} \hyperref[TEI.metamark]{metamark} \hyperref[TEI.mod]{mod} \hyperref[TEI.redo]{redo} \hyperref[TEI.restore]{restore} \hyperref[TEI.retrace]{retrace} \hyperref[TEI.secl]{secl} \hyperref[TEI.space]{space} \hyperref[TEI.subst]{subst} \hyperref[TEI.substJoin]{substJoin} \hyperref[TEI.supplied]{supplied} \hyperref[TEI.surplus]{surplus} \hyperref[TEI.undo]{undo}\par des données textuelles
    \item[{Exemple}]
  \leavevmode\bgroup\exampleFont \begin{shaded}\noindent\mbox{}{<\textbf{bindingDesc}>}\mbox{}\newline 
\hspace*{6pt}{<\textbf{decoNote}\hspace*{6pt}{type}="{plats}">} à décor d’entrelacs géométriques (structure de losange et\mbox{}\newline 
\hspace*{6pt}\hspace*{6pt} rectangle) complété de fers évidés.{</\textbf{decoNote}>}\mbox{}\newline 
\hspace*{6pt}{<\textbf{decoNote}\hspace*{6pt}{type}="{plat\textunderscore sup}">}Titre {<\textbf{q}>}ivvenalis. persivs{</\textbf{q}>} et ex-libris de Jean Grolier\mbox{}\newline 
\hspace*{6pt}{<\textbf{q}>}io. grolierii et amicorvm.{</\textbf{q}>} dorés respectivement au centre et au bas du plat\mbox{}\newline 
\hspace*{6pt}\hspace*{6pt} supérieur. {</\textbf{decoNote}>}\mbox{}\newline 
\hspace*{6pt}{<\textbf{decoNote}\hspace*{6pt}{type}="{plat\textunderscore inf}">}Devise de Jean Grolier{<\textbf{q}>}portio mea sit in terra viventivm{</\textbf{q}>}\mbox{}\newline 
\hspace*{6pt}\hspace*{6pt} dorée au centre du plat inférieur.{</\textbf{decoNote}>}\mbox{}\newline 
\hspace*{6pt}{<\textbf{decoNote}\hspace*{6pt}{type}="{dos}">}Dos à cinq nerfs, sans décor ; simple filet doré sur chaque nerf et\mbox{}\newline 
\hspace*{6pt}\hspace*{6pt} en encadrement des caissons ; passages de chaînette marqués de même.{</\textbf{decoNote}>}\mbox{}\newline 
\hspace*{6pt}{<\textbf{decoNote}\hspace*{6pt}{type}="{tranchefiles}">}Tranchefiles simples unicolores, vert foncé.{</\textbf{decoNote}>}\mbox{}\newline 
\hspace*{6pt}{<\textbf{decoNote}\hspace*{6pt}{type}="{coupes}">}Filet doré sur les coupes.{</\textbf{decoNote}>}\mbox{}\newline 
\hspace*{6pt}{<\textbf{decoNote}\hspace*{6pt}{type}="{annexes}"/>}\mbox{}\newline 
\hspace*{6pt}{<\textbf{decoNote}\hspace*{6pt}{type}="{tranches}">}Tranches dorées.{</\textbf{decoNote}>}\mbox{}\newline 
\hspace*{6pt}{<\textbf{decoNote}\hspace*{6pt}{type}="{contreplats}">}Contreplats en vélin.{</\textbf{decoNote}>}\mbox{}\newline 
\hspace*{6pt}{<\textbf{decoNote}\hspace*{6pt}{type}="{chasses}">}Filet doré sur les chasses.{</\textbf{decoNote}>}\mbox{}\newline 
\textit{<!-- Description des gardes : gardes blanches ; gardes couleurs (marbrées, gaufrées, peintes, dominotées, etc.) généralement suivies de gardes blanches ; dans tous les cas, spécifier le nombre de gardes (début + fin du volume)-->}\mbox{}\newline 
\hspace*{6pt}{<\textbf{decoNote}\hspace*{6pt}{type}="{gardes}">}Gardes en papier et vélin (2+1+2 / 2+1+2) ; filigrane au\mbox{}\newline 
\hspace*{6pt}\hspace*{6pt} pot.{<\textbf{ref}>}Briquet N° XX{</\textbf{ref}>}\mbox{}\newline 
\hspace*{6pt}{</\textbf{decoNote}>}\mbox{}\newline 
\textit{<!-- Élément qui inclut aussi bien des remarques sur la couture que les charnières, claies ou modes d'attaches des plats : tous éléments de la structure dont la description est jugée utile à la description et l'identification de la reliure-->}\mbox{}\newline 
\hspace*{6pt}{<\textbf{decoNote}\hspace*{6pt}{type}="{structure}">}Defet manuscrit utilisé comme claie au contreplat inférieur\mbox{}\newline 
\hspace*{6pt}\hspace*{6pt} (visible par transparence, sous la contregarde en vélin).{</\textbf{decoNote}>}\mbox{}\newline 
{</\textbf{bindingDesc}>}\end{shaded}\egroup 


    \item[{Modèle de contenu}]
  \mbox{}\hfill\\[-10pt]\begin{Verbatim}[fontsize=\small]
<content>
 <macroRef key="macro.specialPara"/>
</content>
    
\end{Verbatim}

    \item[{Schéma Declaration}]
  \mbox{}\hfill\\[-10pt]\begin{Verbatim}[fontsize=\small]
element decoNote
{
   tei_att.global.attributes,
   tei_att.typed.attributes,
   tei_macro.specialPara}
\end{Verbatim}

\end{reflist}  \index{default=<default>|oddindex}
\begin{reflist}
\item[]\begin{specHead}{TEI.default}{<default> }(valeur de trait par défaut) représente la partie valeur d'une spécification trait-valeur contenant une valeur par défaut [\xref{http://www.tei-c.org/release/doc/tei-p5-doc/en/html/FS.html\#FSBO}{18.9. Default Values}]\end{specHead} 
    \item[{Module}]
  iso-fs
    \item[{Attributs}]
  Attributs \hyperref[TEI.att.global]{att.global} (\textit{@xml:id}, \textit{@n}, \textit{@xml:lang}, \textit{@xml:base}, \textit{@xml:space})  (\hyperref[TEI.att.global.rendition]{att.global.rendition} (\textit{@rend}, \textit{@style}, \textit{@rendition})) (\hyperref[TEI.att.global.linking]{att.global.linking} (\textit{@corresp}, \textit{@synch}, \textit{@sameAs}, \textit{@copyOf}, \textit{@next}, \textit{@prev}, \textit{@exclude}, \textit{@select})) (\hyperref[TEI.att.global.analytic]{att.global.analytic} (\textit{@ana})) (\hyperref[TEI.att.global.facs]{att.global.facs} (\textit{@facs})) (\hyperref[TEI.att.global.change]{att.global.change} (\textit{@change})) (\hyperref[TEI.att.global.responsibility]{att.global.responsibility} (\textit{@cert}, \textit{@resp})) (\hyperref[TEI.att.global.source]{att.global.source} (\textit{@source}))
    \item[{Membre du}]
  \hyperref[TEI.model.featureVal.single]{model.featureVal.single}
    \item[{Contenu dans}]
  
    \item[iso-fs: ]
   \hyperref[TEI.f]{f} \hyperref[TEI.fvLib]{fvLib} \hyperref[TEI.if]{if} \hyperref[TEI.vAlt]{vAlt} \hyperref[TEI.vColl]{vColl} \hyperref[TEI.vDefault]{vDefault} \hyperref[TEI.vLabel]{vLabel} \hyperref[TEI.vMerge]{vMerge} \hyperref[TEI.vNot]{vNot} \hyperref[TEI.vRange]{vRange}
    \item[{Peut contenir}]
  Elément vide
    \item[{Exemple}]
  \leavevmode\bgroup\exampleFont \begin{shaded}\noindent\mbox{}{<\textbf{f}\hspace*{6pt}{name}="{gender}">}\mbox{}\newline 
\hspace*{6pt}{<\textbf{default}/>}\mbox{}\newline 
{</\textbf{f}>}\end{shaded}\egroup 


    \item[{Modèle de contenu}]
  \fbox{\ttfamily <content>\newline
</content>\newline
    } 
    \item[{Schéma Declaration}]
  \fbox{\ttfamily element default ❴ tei\textunderscore att.global.attributes, empty ❵} 
\end{reflist}  \index{del=<del>|oddindex}
\begin{reflist}
\item[]\begin{specHead}{TEI.del}{<del> }(suppression) contient une lettre, un mot ou un passage supprimé, marqué comme supprimé, sinon indiqué comme superflu ou erroné dans le texte par un auteur, un copiste, un annotateur ou un correcteur. [\xref{http://www.tei-c.org/release/doc/tei-p5-doc/en/html/CO.html\#COEDADD}{3.4.3. Additions, Deletions, and Omissions}]\end{specHead} 
    \item[{Module}]
  core
    \item[{Attributs}]
  Attributs \hyperref[TEI.att.global]{att.global} (\textit{@xml:id}, \textit{@n}, \textit{@xml:lang}, \textit{@xml:base}, \textit{@xml:space})  (\hyperref[TEI.att.global.rendition]{att.global.rendition} (\textit{@rend}, \textit{@style}, \textit{@rendition})) (\hyperref[TEI.att.global.linking]{att.global.linking} (\textit{@corresp}, \textit{@synch}, \textit{@sameAs}, \textit{@copyOf}, \textit{@next}, \textit{@prev}, \textit{@exclude}, \textit{@select})) (\hyperref[TEI.att.global.analytic]{att.global.analytic} (\textit{@ana})) (\hyperref[TEI.att.global.facs]{att.global.facs} (\textit{@facs})) (\hyperref[TEI.att.global.change]{att.global.change} (\textit{@change})) (\hyperref[TEI.att.global.responsibility]{att.global.responsibility} (\textit{@cert}, \textit{@resp})) (\hyperref[TEI.att.global.source]{att.global.source} (\textit{@source})) \hyperref[TEI.att.transcriptional]{att.transcriptional} (\textit{@status}, \textit{@cause}, \textit{@seq})  (\hyperref[TEI.att.editLike]{att.editLike} (\textit{@evidence}, \textit{@instant}) (\hyperref[TEI.att.dimensions]{att.dimensions} (\textit{@unit}, \textit{@quantity}, \textit{@extent}, \textit{@precision}, \textit{@scope}) (\hyperref[TEI.att.ranging]{att.ranging} (\textit{@atLeast}, \textit{@atMost}, \textit{@min}, \textit{@max}, \textit{@confidence})) ) ) (\hyperref[TEI.att.written]{att.written} (\textit{@hand})) \hyperref[TEI.att.typed]{att.typed} (\textit{@type}, \textit{@subtype}) 
    \item[{Membre du}]
  \hyperref[TEI.model.linePart]{model.linePart} \hyperref[TEI.model.pPart.transcriptional]{model.pPart.transcriptional} 
    \item[{Contenu dans}]
  
    \item[analysis: ]
   \hyperref[TEI.cl]{cl} \hyperref[TEI.pc]{pc} \hyperref[TEI.phr]{phr} \hyperref[TEI.s]{s} \hyperref[TEI.w]{w}\par 
    \item[core: ]
   \hyperref[TEI.abbr]{abbr} \hyperref[TEI.add]{add} \hyperref[TEI.addrLine]{addrLine} \hyperref[TEI.author]{author} \hyperref[TEI.bibl]{bibl} \hyperref[TEI.biblScope]{biblScope} \hyperref[TEI.citedRange]{citedRange} \hyperref[TEI.corr]{corr} \hyperref[TEI.date]{date} \hyperref[TEI.del]{del} \hyperref[TEI.distinct]{distinct} \hyperref[TEI.editor]{editor} \hyperref[TEI.email]{email} \hyperref[TEI.emph]{emph} \hyperref[TEI.expan]{expan} \hyperref[TEI.foreign]{foreign} \hyperref[TEI.gloss]{gloss} \hyperref[TEI.head]{head} \hyperref[TEI.headItem]{headItem} \hyperref[TEI.headLabel]{headLabel} \hyperref[TEI.hi]{hi} \hyperref[TEI.item]{item} \hyperref[TEI.l]{l} \hyperref[TEI.label]{label} \hyperref[TEI.measure]{measure} \hyperref[TEI.mentioned]{mentioned} \hyperref[TEI.name]{name} \hyperref[TEI.note]{note} \hyperref[TEI.num]{num} \hyperref[TEI.orig]{orig} \hyperref[TEI.p]{p} \hyperref[TEI.pubPlace]{pubPlace} \hyperref[TEI.publisher]{publisher} \hyperref[TEI.q]{q} \hyperref[TEI.quote]{quote} \hyperref[TEI.ref]{ref} \hyperref[TEI.reg]{reg} \hyperref[TEI.rs]{rs} \hyperref[TEI.said]{said} \hyperref[TEI.sic]{sic} \hyperref[TEI.soCalled]{soCalled} \hyperref[TEI.speaker]{speaker} \hyperref[TEI.stage]{stage} \hyperref[TEI.street]{street} \hyperref[TEI.term]{term} \hyperref[TEI.textLang]{textLang} \hyperref[TEI.time]{time} \hyperref[TEI.title]{title} \hyperref[TEI.unclear]{unclear}\par 
    \item[figures: ]
   \hyperref[TEI.cell]{cell}\par 
    \item[header: ]
   \hyperref[TEI.change]{change} \hyperref[TEI.distributor]{distributor} \hyperref[TEI.edition]{edition} \hyperref[TEI.extent]{extent} \hyperref[TEI.licence]{licence}\par 
    \item[linking: ]
   \hyperref[TEI.ab]{ab} \hyperref[TEI.seg]{seg}\par 
    \item[msdescription: ]
   \hyperref[TEI.accMat]{accMat} \hyperref[TEI.acquisition]{acquisition} \hyperref[TEI.additions]{additions} \hyperref[TEI.catchwords]{catchwords} \hyperref[TEI.collation]{collation} \hyperref[TEI.colophon]{colophon} \hyperref[TEI.condition]{condition} \hyperref[TEI.custEvent]{custEvent} \hyperref[TEI.decoNote]{decoNote} \hyperref[TEI.explicit]{explicit} \hyperref[TEI.filiation]{filiation} \hyperref[TEI.finalRubric]{finalRubric} \hyperref[TEI.foliation]{foliation} \hyperref[TEI.heraldry]{heraldry} \hyperref[TEI.incipit]{incipit} \hyperref[TEI.layout]{layout} \hyperref[TEI.material]{material} \hyperref[TEI.musicNotation]{musicNotation} \hyperref[TEI.objectType]{objectType} \hyperref[TEI.origDate]{origDate} \hyperref[TEI.origPlace]{origPlace} \hyperref[TEI.origin]{origin} \hyperref[TEI.provenance]{provenance} \hyperref[TEI.rubric]{rubric} \hyperref[TEI.secFol]{secFol} \hyperref[TEI.signatures]{signatures} \hyperref[TEI.source]{source} \hyperref[TEI.stamp]{stamp} \hyperref[TEI.summary]{summary} \hyperref[TEI.support]{support} \hyperref[TEI.surrogates]{surrogates} \hyperref[TEI.typeNote]{typeNote} \hyperref[TEI.watermark]{watermark}\par 
    \item[namesdates: ]
   \hyperref[TEI.addName]{addName} \hyperref[TEI.affiliation]{affiliation} \hyperref[TEI.country]{country} \hyperref[TEI.forename]{forename} \hyperref[TEI.genName]{genName} \hyperref[TEI.geogName]{geogName} \hyperref[TEI.nameLink]{nameLink} \hyperref[TEI.orgName]{orgName} \hyperref[TEI.persName]{persName} \hyperref[TEI.placeName]{placeName} \hyperref[TEI.region]{region} \hyperref[TEI.roleName]{roleName} \hyperref[TEI.settlement]{settlement} \hyperref[TEI.surname]{surname}\par 
    \item[textstructure: ]
   \hyperref[TEI.docAuthor]{docAuthor} \hyperref[TEI.docDate]{docDate} \hyperref[TEI.docEdition]{docEdition} \hyperref[TEI.titlePart]{titlePart}\par 
    \item[transcr: ]
   \hyperref[TEI.am]{am} \hyperref[TEI.damage]{damage} \hyperref[TEI.fw]{fw} \hyperref[TEI.line]{line} \hyperref[TEI.metamark]{metamark} \hyperref[TEI.mod]{mod} \hyperref[TEI.restore]{restore} \hyperref[TEI.retrace]{retrace} \hyperref[TEI.secl]{secl} \hyperref[TEI.subst]{subst} \hyperref[TEI.supplied]{supplied} \hyperref[TEI.surplus]{surplus} \hyperref[TEI.zone]{zone}
    \item[{Peut contenir}]
  
    \item[analysis: ]
   \hyperref[TEI.c]{c} \hyperref[TEI.cl]{cl} \hyperref[TEI.interp]{interp} \hyperref[TEI.interpGrp]{interpGrp} \hyperref[TEI.m]{m} \hyperref[TEI.pc]{pc} \hyperref[TEI.phr]{phr} \hyperref[TEI.s]{s} \hyperref[TEI.span]{span} \hyperref[TEI.spanGrp]{spanGrp} \hyperref[TEI.w]{w}\par 
    \item[core: ]
   \hyperref[TEI.abbr]{abbr} \hyperref[TEI.add]{add} \hyperref[TEI.address]{address} \hyperref[TEI.bibl]{bibl} \hyperref[TEI.biblStruct]{biblStruct} \hyperref[TEI.binaryObject]{binaryObject} \hyperref[TEI.cb]{cb} \hyperref[TEI.choice]{choice} \hyperref[TEI.cit]{cit} \hyperref[TEI.corr]{corr} \hyperref[TEI.date]{date} \hyperref[TEI.del]{del} \hyperref[TEI.desc]{desc} \hyperref[TEI.distinct]{distinct} \hyperref[TEI.email]{email} \hyperref[TEI.emph]{emph} \hyperref[TEI.expan]{expan} \hyperref[TEI.foreign]{foreign} \hyperref[TEI.gap]{gap} \hyperref[TEI.gb]{gb} \hyperref[TEI.gloss]{gloss} \hyperref[TEI.graphic]{graphic} \hyperref[TEI.hi]{hi} \hyperref[TEI.index]{index} \hyperref[TEI.l]{l} \hyperref[TEI.label]{label} \hyperref[TEI.lb]{lb} \hyperref[TEI.lg]{lg} \hyperref[TEI.list]{list} \hyperref[TEI.listBibl]{listBibl} \hyperref[TEI.measure]{measure} \hyperref[TEI.measureGrp]{measureGrp} \hyperref[TEI.media]{media} \hyperref[TEI.mentioned]{mentioned} \hyperref[TEI.milestone]{milestone} \hyperref[TEI.name]{name} \hyperref[TEI.note]{note} \hyperref[TEI.num]{num} \hyperref[TEI.orig]{orig} \hyperref[TEI.pb]{pb} \hyperref[TEI.ptr]{ptr} \hyperref[TEI.q]{q} \hyperref[TEI.quote]{quote} \hyperref[TEI.ref]{ref} \hyperref[TEI.reg]{reg} \hyperref[TEI.rs]{rs} \hyperref[TEI.said]{said} \hyperref[TEI.sic]{sic} \hyperref[TEI.soCalled]{soCalled} \hyperref[TEI.stage]{stage} \hyperref[TEI.term]{term} \hyperref[TEI.time]{time} \hyperref[TEI.title]{title} \hyperref[TEI.unclear]{unclear}\par 
    \item[derived-module-tei.istex: ]
   \hyperref[TEI.math]{math} \hyperref[TEI.mrow]{mrow}\par 
    \item[figures: ]
   \hyperref[TEI.figure]{figure} \hyperref[TEI.formula]{formula} \hyperref[TEI.notatedMusic]{notatedMusic} \hyperref[TEI.table]{table}\par 
    \item[header: ]
   \hyperref[TEI.biblFull]{biblFull} \hyperref[TEI.idno]{idno}\par 
    \item[iso-fs: ]
   \hyperref[TEI.fLib]{fLib} \hyperref[TEI.fs]{fs} \hyperref[TEI.fvLib]{fvLib}\par 
    \item[linking: ]
   \hyperref[TEI.alt]{alt} \hyperref[TEI.altGrp]{altGrp} \hyperref[TEI.anchor]{anchor} \hyperref[TEI.join]{join} \hyperref[TEI.joinGrp]{joinGrp} \hyperref[TEI.link]{link} \hyperref[TEI.linkGrp]{linkGrp} \hyperref[TEI.seg]{seg} \hyperref[TEI.timeline]{timeline}\par 
    \item[msdescription: ]
   \hyperref[TEI.catchwords]{catchwords} \hyperref[TEI.depth]{depth} \hyperref[TEI.dim]{dim} \hyperref[TEI.dimensions]{dimensions} \hyperref[TEI.height]{height} \hyperref[TEI.heraldry]{heraldry} \hyperref[TEI.locus]{locus} \hyperref[TEI.locusGrp]{locusGrp} \hyperref[TEI.material]{material} \hyperref[TEI.msDesc]{msDesc} \hyperref[TEI.objectType]{objectType} \hyperref[TEI.origDate]{origDate} \hyperref[TEI.origPlace]{origPlace} \hyperref[TEI.secFol]{secFol} \hyperref[TEI.signatures]{signatures} \hyperref[TEI.source]{source} \hyperref[TEI.stamp]{stamp} \hyperref[TEI.watermark]{watermark} \hyperref[TEI.width]{width}\par 
    \item[namesdates: ]
   \hyperref[TEI.addName]{addName} \hyperref[TEI.affiliation]{affiliation} \hyperref[TEI.country]{country} \hyperref[TEI.forename]{forename} \hyperref[TEI.genName]{genName} \hyperref[TEI.geogName]{geogName} \hyperref[TEI.listOrg]{listOrg} \hyperref[TEI.listPlace]{listPlace} \hyperref[TEI.location]{location} \hyperref[TEI.nameLink]{nameLink} \hyperref[TEI.orgName]{orgName} \hyperref[TEI.persName]{persName} \hyperref[TEI.placeName]{placeName} \hyperref[TEI.region]{region} \hyperref[TEI.roleName]{roleName} \hyperref[TEI.settlement]{settlement} \hyperref[TEI.state]{state} \hyperref[TEI.surname]{surname}\par 
    \item[spoken: ]
   \hyperref[TEI.annotationBlock]{annotationBlock}\par 
    \item[textstructure: ]
   \hyperref[TEI.floatingText]{floatingText}\par 
    \item[transcr: ]
   \hyperref[TEI.addSpan]{addSpan} \hyperref[TEI.am]{am} \hyperref[TEI.damage]{damage} \hyperref[TEI.damageSpan]{damageSpan} \hyperref[TEI.delSpan]{delSpan} \hyperref[TEI.ex]{ex} \hyperref[TEI.fw]{fw} \hyperref[TEI.handShift]{handShift} \hyperref[TEI.listTranspose]{listTranspose} \hyperref[TEI.metamark]{metamark} \hyperref[TEI.mod]{mod} \hyperref[TEI.redo]{redo} \hyperref[TEI.restore]{restore} \hyperref[TEI.retrace]{retrace} \hyperref[TEI.secl]{secl} \hyperref[TEI.space]{space} \hyperref[TEI.subst]{subst} \hyperref[TEI.substJoin]{substJoin} \hyperref[TEI.supplied]{supplied} \hyperref[TEI.surplus]{surplus} \hyperref[TEI.undo]{undo}\par des données textuelles
    \item[{Note}]
  \par
Cf. \hyperref[TEI.gap]{<gap>}.\par
Les degrés d'incertitude sur ce qui est encore lisible peuvent être indiqués par l'emploi de l'élément \texttt{<certainty>} (voir \xref{http://www.tei-c.org/release/doc/tei-p5-doc/en/html/CE.html\#CE}{21. Certainty, Precision, and Responsibility}).\par
Cet élément doit être utilisé pour la suppression de courtes séquences de texte, généralement des mots ou des expressions. Il faut utiliser l'élément \hyperref[TEI.delSpan]{<delSpan>} pour les séquences de texte plus longues, celles qui contiennent des divisions structurelles, et celles qui contiennent un chevauchement d'ajouts et de suppressions.\par
Le texte supprimé doit être au moins partiellement lisible, afin que l'encodeur soit en mesure de le transcrire. La partie illisible du texte à l'intérieur d'une suppression peut être marquée au moyen de la balise \hyperref[TEI.gap]{<gap>} pour signaler la présence de texte non transcrit. La quantité de texte omise, la raison de l'omission, etc., peuvent être indiquées au moyen des attributs de l'élément \hyperref[TEI.gap]{<gap>}. Si le texte n'est pas entièrement lisible, l'élément \hyperref[TEI.unclear]{<unclear>} (disponible avec le jeu additionnel de balises pour la transcription des sources primaires) doit être utilisé pour signaler les zones de texte ne pouvant pas être lues de manière fiable. Voir les sections suivantes \xref{http://www.tei-c.org/release/doc/tei-p5-doc/en/html/PH.html\#PHOM}{11.3.1.7. Text Omitted from or Supplied in the Transcription} et, pour l'association étroite entre les balises \hyperref[TEI.del]{<del>} et \hyperref[TEI.gap]{<gap>}, \hyperref[TEI.damage]{<damage>}, \hyperref[TEI.unclear]{<unclear>} et \hyperref[TEI.supplied]{<supplied>} (ces trois dernières balises étant disponibles avec le jeu additionnel de balises pour la transcription de sources primaires), voir la section \xref{http://www.tei-c.org/release/doc/tei-p5-doc/en/html/PH.html\#PHCOMB}{11.3.3.2. Use of the gap, del, damage, unclear, and supplied Elements in Combination}.\par
La balise \hyperref[TEI.del]{<del>} ne doit pas être utilisée pour les suppressions par des éditeurs scientifiques ou des encodeurs. Dans ce cas, il faut utiliser soit la balise \hyperref[TEI.corr]{<corr>}, soit la balise \hyperref[TEI.gap]{<gap>}.
    \item[{Exemple}]
  \leavevmode\bgroup\exampleFont \begin{shaded}\noindent\mbox{}{<\textbf{l}>}\mbox{}\newline 
\hspace*{6pt}{<\textbf{del}\hspace*{6pt}{rend}="{overtyped}">}Mein{</\textbf{del}>} Frisch {<\textbf{del}\hspace*{6pt}{rend}="{overstrike}"\hspace*{6pt}{type}="{primary}">}schwebt{</\textbf{del}>}\mbox{}\newline 
 weht der Wind\mbox{}\newline 
{</\textbf{l}>}\end{shaded}\egroup 


    \item[{Exemple}]
  \leavevmode\bgroup\exampleFont \begin{shaded}\noindent\mbox{}{<\textbf{p}>}[...] mais il y reste quelque chose de mystérieux, de furtif. {<\textbf{del}\hspace*{6pt}{rend}="{overtyped}">}Je{</\textbf{del}>} On cesse un instant d'y penser; {<\textbf{del}\hspace*{6pt}{rend}="{overstrike}">}les {</\textbf{del}>}mes yeux se\mbox{}\newline 
 ferment ou {<\textbf{del}\hspace*{6pt}{rend}="{overstrike}">}s'attardent sur {</\textbf{del}>} se détournent sur un livre... On\mbox{}\newline 
 relève la tête: il est là {</\textbf{p}>}\end{shaded}\egroup 


    \item[{Modèle de contenu}]
  \mbox{}\hfill\\[-10pt]\begin{Verbatim}[fontsize=\small]
<content>
 <macroRef key="macro.paraContent"/>
</content>
    
\end{Verbatim}

    \item[{Schéma Declaration}]
  \mbox{}\hfill\\[-10pt]\begin{Verbatim}[fontsize=\small]
element del
{
   tei_att.global.attributes,
   tei_att.transcriptional.attributes,
   tei_att.typed.attributes,
   tei_macro.paraContent}
\end{Verbatim}

\end{reflist}  \index{delSpan=<delSpan>|oddindex}
\begin{reflist}
\item[]\begin{specHead}{TEI.delSpan}{<delSpan> }(partie de texte supprimée) marque le début d'une longue partie de texte supprimée, signalée comme supprimée ou bien signalée comme superflue ou fausse par un auteur, un copiste, un annotateur ou un correcteur. [\xref{http://www.tei-c.org/release/doc/tei-p5-doc/en/html/PH.html\#PHAD}{11.3.1.4. Additions and Deletions}]\end{specHead} 
    \item[{Module}]
  transcr
    \item[{Attributs}]
  Attributs \hyperref[TEI.att.global]{att.global} (\textit{@xml:id}, \textit{@n}, \textit{@xml:lang}, \textit{@xml:base}, \textit{@xml:space})  (\hyperref[TEI.att.global.rendition]{att.global.rendition} (\textit{@rend}, \textit{@style}, \textit{@rendition})) (\hyperref[TEI.att.global.linking]{att.global.linking} (\textit{@corresp}, \textit{@synch}, \textit{@sameAs}, \textit{@copyOf}, \textit{@next}, \textit{@prev}, \textit{@exclude}, \textit{@select})) (\hyperref[TEI.att.global.analytic]{att.global.analytic} (\textit{@ana})) (\hyperref[TEI.att.global.facs]{att.global.facs} (\textit{@facs})) (\hyperref[TEI.att.global.change]{att.global.change} (\textit{@change})) (\hyperref[TEI.att.global.responsibility]{att.global.responsibility} (\textit{@cert}, \textit{@resp})) (\hyperref[TEI.att.global.source]{att.global.source} (\textit{@source})) \hyperref[TEI.att.transcriptional]{att.transcriptional} (\textit{@status}, \textit{@cause}, \textit{@seq})  (\hyperref[TEI.att.editLike]{att.editLike} (\textit{@evidence}, \textit{@instant}) (\hyperref[TEI.att.dimensions]{att.dimensions} (\textit{@unit}, \textit{@quantity}, \textit{@extent}, \textit{@precision}, \textit{@scope}) (\hyperref[TEI.att.ranging]{att.ranging} (\textit{@atLeast}, \textit{@atMost}, \textit{@min}, \textit{@max}, \textit{@confidence})) ) ) (\hyperref[TEI.att.written]{att.written} (\textit{@hand})) \hyperref[TEI.att.typed]{att.typed} (\textit{@type}, \textit{@subtype}) \hyperref[TEI.att.spanning]{att.spanning} (\textit{@spanTo}) 
    \item[{Membre du}]
  \hyperref[TEI.model.global.edit]{model.global.edit}
    \item[{Contenu dans}]
  
    \item[analysis: ]
   \hyperref[TEI.cl]{cl} \hyperref[TEI.m]{m} \hyperref[TEI.phr]{phr} \hyperref[TEI.s]{s} \hyperref[TEI.span]{span} \hyperref[TEI.w]{w}\par 
    \item[core: ]
   \hyperref[TEI.abbr]{abbr} \hyperref[TEI.add]{add} \hyperref[TEI.addrLine]{addrLine} \hyperref[TEI.address]{address} \hyperref[TEI.author]{author} \hyperref[TEI.bibl]{bibl} \hyperref[TEI.biblScope]{biblScope} \hyperref[TEI.cit]{cit} \hyperref[TEI.citedRange]{citedRange} \hyperref[TEI.corr]{corr} \hyperref[TEI.date]{date} \hyperref[TEI.del]{del} \hyperref[TEI.distinct]{distinct} \hyperref[TEI.editor]{editor} \hyperref[TEI.email]{email} \hyperref[TEI.emph]{emph} \hyperref[TEI.expan]{expan} \hyperref[TEI.foreign]{foreign} \hyperref[TEI.gloss]{gloss} \hyperref[TEI.head]{head} \hyperref[TEI.headItem]{headItem} \hyperref[TEI.headLabel]{headLabel} \hyperref[TEI.hi]{hi} \hyperref[TEI.imprint]{imprint} \hyperref[TEI.item]{item} \hyperref[TEI.l]{l} \hyperref[TEI.label]{label} \hyperref[TEI.lg]{lg} \hyperref[TEI.list]{list} \hyperref[TEI.measure]{measure} \hyperref[TEI.mentioned]{mentioned} \hyperref[TEI.name]{name} \hyperref[TEI.note]{note} \hyperref[TEI.num]{num} \hyperref[TEI.orig]{orig} \hyperref[TEI.p]{p} \hyperref[TEI.pubPlace]{pubPlace} \hyperref[TEI.publisher]{publisher} \hyperref[TEI.q]{q} \hyperref[TEI.quote]{quote} \hyperref[TEI.ref]{ref} \hyperref[TEI.reg]{reg} \hyperref[TEI.resp]{resp} \hyperref[TEI.rs]{rs} \hyperref[TEI.said]{said} \hyperref[TEI.series]{series} \hyperref[TEI.sic]{sic} \hyperref[TEI.soCalled]{soCalled} \hyperref[TEI.sp]{sp} \hyperref[TEI.speaker]{speaker} \hyperref[TEI.stage]{stage} \hyperref[TEI.street]{street} \hyperref[TEI.term]{term} \hyperref[TEI.textLang]{textLang} \hyperref[TEI.time]{time} \hyperref[TEI.title]{title} \hyperref[TEI.unclear]{unclear}\par 
    \item[figures: ]
   \hyperref[TEI.cell]{cell} \hyperref[TEI.figure]{figure} \hyperref[TEI.table]{table}\par 
    \item[header: ]
   \hyperref[TEI.authority]{authority} \hyperref[TEI.change]{change} \hyperref[TEI.classCode]{classCode} \hyperref[TEI.distributor]{distributor} \hyperref[TEI.edition]{edition} \hyperref[TEI.extent]{extent} \hyperref[TEI.funder]{funder} \hyperref[TEI.language]{language} \hyperref[TEI.licence]{licence}\par 
    \item[linking: ]
   \hyperref[TEI.ab]{ab} \hyperref[TEI.seg]{seg}\par 
    \item[msdescription: ]
   \hyperref[TEI.accMat]{accMat} \hyperref[TEI.acquisition]{acquisition} \hyperref[TEI.additions]{additions} \hyperref[TEI.catchwords]{catchwords} \hyperref[TEI.collation]{collation} \hyperref[TEI.colophon]{colophon} \hyperref[TEI.condition]{condition} \hyperref[TEI.custEvent]{custEvent} \hyperref[TEI.decoNote]{decoNote} \hyperref[TEI.explicit]{explicit} \hyperref[TEI.filiation]{filiation} \hyperref[TEI.finalRubric]{finalRubric} \hyperref[TEI.foliation]{foliation} \hyperref[TEI.heraldry]{heraldry} \hyperref[TEI.incipit]{incipit} \hyperref[TEI.layout]{layout} \hyperref[TEI.material]{material} \hyperref[TEI.msItem]{msItem} \hyperref[TEI.musicNotation]{musicNotation} \hyperref[TEI.objectType]{objectType} \hyperref[TEI.origDate]{origDate} \hyperref[TEI.origPlace]{origPlace} \hyperref[TEI.origin]{origin} \hyperref[TEI.provenance]{provenance} \hyperref[TEI.rubric]{rubric} \hyperref[TEI.secFol]{secFol} \hyperref[TEI.signatures]{signatures} \hyperref[TEI.source]{source} \hyperref[TEI.stamp]{stamp} \hyperref[TEI.summary]{summary} \hyperref[TEI.support]{support} \hyperref[TEI.surrogates]{surrogates} \hyperref[TEI.typeNote]{typeNote} \hyperref[TEI.watermark]{watermark}\par 
    \item[namesdates: ]
   \hyperref[TEI.addName]{addName} \hyperref[TEI.affiliation]{affiliation} \hyperref[TEI.country]{country} \hyperref[TEI.forename]{forename} \hyperref[TEI.genName]{genName} \hyperref[TEI.geogName]{geogName} \hyperref[TEI.nameLink]{nameLink} \hyperref[TEI.orgName]{orgName} \hyperref[TEI.persName]{persName} \hyperref[TEI.person]{person} \hyperref[TEI.personGrp]{personGrp} \hyperref[TEI.persona]{persona} \hyperref[TEI.placeName]{placeName} \hyperref[TEI.region]{region} \hyperref[TEI.roleName]{roleName} \hyperref[TEI.settlement]{settlement} \hyperref[TEI.surname]{surname}\par 
    \item[textstructure: ]
   \hyperref[TEI.back]{back} \hyperref[TEI.body]{body} \hyperref[TEI.div]{div} \hyperref[TEI.docAuthor]{docAuthor} \hyperref[TEI.docDate]{docDate} \hyperref[TEI.docEdition]{docEdition} \hyperref[TEI.docTitle]{docTitle} \hyperref[TEI.floatingText]{floatingText} \hyperref[TEI.front]{front} \hyperref[TEI.group]{group} \hyperref[TEI.text]{text} \hyperref[TEI.titlePage]{titlePage} \hyperref[TEI.titlePart]{titlePart}\par 
    \item[transcr: ]
   \hyperref[TEI.damage]{damage} \hyperref[TEI.fw]{fw} \hyperref[TEI.line]{line} \hyperref[TEI.metamark]{metamark} \hyperref[TEI.mod]{mod} \hyperref[TEI.restore]{restore} \hyperref[TEI.retrace]{retrace} \hyperref[TEI.secl]{secl} \hyperref[TEI.sourceDoc]{sourceDoc} \hyperref[TEI.supplied]{supplied} \hyperref[TEI.surface]{surface} \hyperref[TEI.surfaceGrp]{surfaceGrp} \hyperref[TEI.surplus]{surplus} \hyperref[TEI.zone]{zone}
    \item[{Peut contenir}]
  Elément vide
    \item[{Note}]
  \par
Le début et la fin de la partie de texte supprimée doivent être marqués : le début, par l'élément \hyperref[TEI.delSpan]{<delSpan>}, la fin par la cible de l'attribut {\itshape spanTo}.\par
Le texte supprimé doit être au moins partiellement lisible, afin que l'encodeur soit capable de le transcrire. S'il n'est pas lisible du tout, la balise\hyperref[TEI.delSpan]{<delSpan>} ne doit pas être utilisée. Pour signaler qu'un texte ne peut être transcrit, il vaut mieux utiliser la balise \hyperref[TEI.gap]{<gap>} avec un attribut {\itshape reason} dont la valeur indique la raison pour laquelle la transcription du texte supprimé est impossible. S'il n'est pas entièrement lisible, l'élément \hyperref[TEI.unclear]{<unclear>} doit être utilisé pour signaler les parties du texte qui ne peuvent pas être lues avec certitude. Voir plus loin les sections \xref{http://www.tei-c.org/release/doc/tei-p5-doc/en/html/PH.html\#PHOM}{11.3.1.7. Text Omitted from or Supplied in the Transcription} et, à propos de l'association étroite qui existe entre \hyperref[TEI.delSpan]{<delSpan>} et les éléments \hyperref[TEI.gap]{<gap>}, \hyperref[TEI.damage]{<damage>}, \hyperref[TEI.unclear]{<unclear>} et \hyperref[TEI.supplied]{<supplied>}, la section \xref{http://www.tei-c.org/release/doc/tei-p5-doc/en/html/PH.html\#PHCOMB}{11.3.3.2. Use of the gap, del, damage, unclear, and supplied Elements in Combination}.\par
La balise \hyperref[TEI.delSpan]{<delSpan>} ne doit pas être utilisée pour des suppressions opérées par des éditeurs scientifiques ou des encodeurs. Dans ces cas, on emploiera soit la balise \hyperref[TEI.corr]{<corr>}, soit la balise \hyperref[TEI.gap]{<gap>}.
    \item[{Exemple}]
  \leavevmode\bgroup\exampleFont \begin{shaded}\noindent\mbox{}{<\textbf{p}>}Paragraph partially deleted. This is the undeleted portion {<\textbf{delSpan}\hspace*{6pt}{spanTo}="{\#fr\textunderscore a23}"/>}and\mbox{}\newline 
 this the deleted portion of the paragraph.{</\textbf{p}>}\mbox{}\newline 
{<\textbf{p}>}Paragraph deleted together with adjacent material.{</\textbf{p}>}\mbox{}\newline 
{<\textbf{p}>}Second fully deleted paragraph.{</\textbf{p}>}\mbox{}\newline 
{<\textbf{p}>}Paragraph partially deleted; in the middle of this paragraph the deletion ends and the\mbox{}\newline 
 anchor point marks the resumption {<\textbf{anchor}\hspace*{6pt}{xml:id}="{fr\textunderscore a23}"/>} of the text. ...{</\textbf{p}>}\end{shaded}\egroup 


    \item[{Schematron}]
   <s:assert test="@spanTo">The @spanTo attribute of <s:name/> is required.</s:assert>
    \item[{Schematron}]
   <s:assert test="@spanTo">L'attribut spanTo est requis.</s:assert>
    \item[{Modèle de contenu}]
  \fbox{\ttfamily <content>\newline
</content>\newline
    } 
    \item[{Schéma Declaration}]
  \mbox{}\hfill\\[-10pt]\begin{Verbatim}[fontsize=\small]
element delSpan
{
   tei_att.global.attributes,
   tei_att.transcriptional.attributes,
   tei_att.typed.attributes,
   tei_att.spanning.attributes,
   empty
}
\end{Verbatim}

\end{reflist}  \index{depth=<depth>|oddindex}
\begin{reflist}
\item[]\begin{specHead}{TEI.depth}{<depth> }(épaisseur) contient une dimension mesurée sur le dos du manuscrit. [\xref{http://www.tei-c.org/release/doc/tei-p5-doc/en/html/MS.html\#msdim}{10.3.4. Dimensions}]\end{specHead} 
    \item[{Module}]
  msdescription
    \item[{Attributs}]
  Attributs \hyperref[TEI.att.global]{att.global} (\textit{@xml:id}, \textit{@n}, \textit{@xml:lang}, \textit{@xml:base}, \textit{@xml:space})  (\hyperref[TEI.att.global.rendition]{att.global.rendition} (\textit{@rend}, \textit{@style}, \textit{@rendition})) (\hyperref[TEI.att.global.linking]{att.global.linking} (\textit{@corresp}, \textit{@synch}, \textit{@sameAs}, \textit{@copyOf}, \textit{@next}, \textit{@prev}, \textit{@exclude}, \textit{@select})) (\hyperref[TEI.att.global.analytic]{att.global.analytic} (\textit{@ana})) (\hyperref[TEI.att.global.facs]{att.global.facs} (\textit{@facs})) (\hyperref[TEI.att.global.change]{att.global.change} (\textit{@change})) (\hyperref[TEI.att.global.responsibility]{att.global.responsibility} (\textit{@cert}, \textit{@resp})) (\hyperref[TEI.att.global.source]{att.global.source} (\textit{@source})) \hyperref[TEI.att.dimensions]{att.dimensions} (\textit{@unit}, \textit{@quantity}, \textit{@extent}, \textit{@precision}, \textit{@scope})  (\hyperref[TEI.att.ranging]{att.ranging} (\textit{@atLeast}, \textit{@atMost}, \textit{@min}, \textit{@max}, \textit{@confidence}))
    \item[{Membre du}]
  \hyperref[TEI.model.dimLike]{model.dimLike} \hyperref[TEI.model.measureLike]{model.measureLike}Elément: \begin{itemize}
\item \hyperref[TEI.mpadded]{mpadded}/@depth
\end{itemize} 
    \item[{Contenu dans}]
  
    \item[analysis: ]
   \hyperref[TEI.cl]{cl} \hyperref[TEI.phr]{phr} \hyperref[TEI.s]{s} \hyperref[TEI.span]{span}\par 
    \item[core: ]
   \hyperref[TEI.abbr]{abbr} \hyperref[TEI.add]{add} \hyperref[TEI.addrLine]{addrLine} \hyperref[TEI.author]{author} \hyperref[TEI.bibl]{bibl} \hyperref[TEI.biblScope]{biblScope} \hyperref[TEI.citedRange]{citedRange} \hyperref[TEI.corr]{corr} \hyperref[TEI.date]{date} \hyperref[TEI.del]{del} \hyperref[TEI.desc]{desc} \hyperref[TEI.distinct]{distinct} \hyperref[TEI.editor]{editor} \hyperref[TEI.email]{email} \hyperref[TEI.emph]{emph} \hyperref[TEI.expan]{expan} \hyperref[TEI.foreign]{foreign} \hyperref[TEI.gloss]{gloss} \hyperref[TEI.head]{head} \hyperref[TEI.headItem]{headItem} \hyperref[TEI.headLabel]{headLabel} \hyperref[TEI.hi]{hi} \hyperref[TEI.item]{item} \hyperref[TEI.l]{l} \hyperref[TEI.label]{label} \hyperref[TEI.measure]{measure} \hyperref[TEI.measureGrp]{measureGrp} \hyperref[TEI.meeting]{meeting} \hyperref[TEI.mentioned]{mentioned} \hyperref[TEI.name]{name} \hyperref[TEI.note]{note} \hyperref[TEI.num]{num} \hyperref[TEI.orig]{orig} \hyperref[TEI.p]{p} \hyperref[TEI.pubPlace]{pubPlace} \hyperref[TEI.publisher]{publisher} \hyperref[TEI.q]{q} \hyperref[TEI.quote]{quote} \hyperref[TEI.ref]{ref} \hyperref[TEI.reg]{reg} \hyperref[TEI.resp]{resp} \hyperref[TEI.rs]{rs} \hyperref[TEI.said]{said} \hyperref[TEI.sic]{sic} \hyperref[TEI.soCalled]{soCalled} \hyperref[TEI.speaker]{speaker} \hyperref[TEI.stage]{stage} \hyperref[TEI.street]{street} \hyperref[TEI.term]{term} \hyperref[TEI.textLang]{textLang} \hyperref[TEI.time]{time} \hyperref[TEI.title]{title} \hyperref[TEI.unclear]{unclear}\par 
    \item[figures: ]
   \hyperref[TEI.cell]{cell} \hyperref[TEI.figDesc]{figDesc}\par 
    \item[header: ]
   \hyperref[TEI.authority]{authority} \hyperref[TEI.change]{change} \hyperref[TEI.classCode]{classCode} \hyperref[TEI.creation]{creation} \hyperref[TEI.distributor]{distributor} \hyperref[TEI.edition]{edition} \hyperref[TEI.extent]{extent} \hyperref[TEI.funder]{funder} \hyperref[TEI.language]{language} \hyperref[TEI.licence]{licence} \hyperref[TEI.rendition]{rendition}\par 
    \item[iso-fs: ]
   \hyperref[TEI.fDescr]{fDescr} \hyperref[TEI.fsDescr]{fsDescr}\par 
    \item[linking: ]
   \hyperref[TEI.ab]{ab} \hyperref[TEI.seg]{seg}\par 
    \item[msdescription: ]
   \hyperref[TEI.accMat]{accMat} \hyperref[TEI.acquisition]{acquisition} \hyperref[TEI.additions]{additions} \hyperref[TEI.catchwords]{catchwords} \hyperref[TEI.collation]{collation} \hyperref[TEI.colophon]{colophon} \hyperref[TEI.condition]{condition} \hyperref[TEI.custEvent]{custEvent} \hyperref[TEI.decoNote]{decoNote} \hyperref[TEI.dimensions]{dimensions} \hyperref[TEI.explicit]{explicit} \hyperref[TEI.filiation]{filiation} \hyperref[TEI.finalRubric]{finalRubric} \hyperref[TEI.foliation]{foliation} \hyperref[TEI.heraldry]{heraldry} \hyperref[TEI.incipit]{incipit} \hyperref[TEI.layout]{layout} \hyperref[TEI.material]{material} \hyperref[TEI.musicNotation]{musicNotation} \hyperref[TEI.objectType]{objectType} \hyperref[TEI.origDate]{origDate} \hyperref[TEI.origPlace]{origPlace} \hyperref[TEI.origin]{origin} \hyperref[TEI.provenance]{provenance} \hyperref[TEI.rubric]{rubric} \hyperref[TEI.secFol]{secFol} \hyperref[TEI.signatures]{signatures} \hyperref[TEI.source]{source} \hyperref[TEI.stamp]{stamp} \hyperref[TEI.summary]{summary} \hyperref[TEI.support]{support} \hyperref[TEI.surrogates]{surrogates} \hyperref[TEI.typeNote]{typeNote} \hyperref[TEI.watermark]{watermark}\par 
    \item[namesdates: ]
   \hyperref[TEI.addName]{addName} \hyperref[TEI.affiliation]{affiliation} \hyperref[TEI.country]{country} \hyperref[TEI.forename]{forename} \hyperref[TEI.genName]{genName} \hyperref[TEI.geogName]{geogName} \hyperref[TEI.location]{location} \hyperref[TEI.nameLink]{nameLink} \hyperref[TEI.orgName]{orgName} \hyperref[TEI.persName]{persName} \hyperref[TEI.placeName]{placeName} \hyperref[TEI.region]{region} \hyperref[TEI.roleName]{roleName} \hyperref[TEI.settlement]{settlement} \hyperref[TEI.surname]{surname}\par 
    \item[textstructure: ]
   \hyperref[TEI.docAuthor]{docAuthor} \hyperref[TEI.docDate]{docDate} \hyperref[TEI.docEdition]{docEdition} \hyperref[TEI.titlePart]{titlePart}\par 
    \item[transcr: ]
   \hyperref[TEI.damage]{damage} \hyperref[TEI.fw]{fw} \hyperref[TEI.metamark]{metamark} \hyperref[TEI.mod]{mod} \hyperref[TEI.restore]{restore} \hyperref[TEI.retrace]{retrace} \hyperref[TEI.secl]{secl} \hyperref[TEI.supplied]{supplied} \hyperref[TEI.surplus]{surplus}
    \item[{Peut contenir}]
  Des données textuelles uniquement
    \item[{Note}]
  \par
If used to specify the width of a non text-bearing portion of some object, for example a monument, this element conventionally refers to the axis facing the observer, and perpendicular to that indicated by the ‘width’ axis.
    \item[{Exemple}]
  \leavevmode\bgroup\exampleFont \begin{shaded}\noindent\mbox{}{<\textbf{depth}\hspace*{6pt}{quantity}="{4}"\hspace*{6pt}{unit}="{in}"/>}\end{shaded}\egroup 


    \item[{Exemple}]
  \leavevmode\bgroup\exampleFont \begin{shaded}\noindent\mbox{}{<\textbf{depth}\hspace*{6pt}{unit}="{mm}">}64{</\textbf{depth}>}\end{shaded}\egroup 


    \item[{Modèle de contenu}]
  \fbox{\ttfamily <content>\newline
 <macroRef key="macro.xtext"/>\newline
</content>\newline
    } 
    \item[{Schéma Declaration}]
  \mbox{}\hfill\\[-10pt]\begin{Verbatim}[fontsize=\small]
element depth
{
   tei_att.global.attributes,
   tei_att.dimensions.attributes,
   tei_macro.xtext}
\end{Verbatim}

\end{reflist}  \index{desc=<desc>|oddindex}
\begin{reflist}
\item[]\begin{specHead}{TEI.desc}{<desc> }(description) contient une courte description de l'objet documenté par son élément parent, qui comprend son utilisation prévue, son but, ou son application là où c'est approprié. [\xref{http://www.tei-c.org/release/doc/tei-p5-doc/en/html/TD.html\#TDcrystalsCEdc}{22.4.1. Description of Components}]\end{specHead} 
    \item[{Module}]
  core
    \item[{Attributs}]
  Attributs \hyperref[TEI.att.global]{att.global} (\textit{@xml:id}, \textit{@n}, \textit{@xml:lang}, \textit{@xml:base}, \textit{@xml:space})  (\hyperref[TEI.att.global.rendition]{att.global.rendition} (\textit{@rend}, \textit{@style}, \textit{@rendition})) (\hyperref[TEI.att.global.linking]{att.global.linking} (\textit{@corresp}, \textit{@synch}, \textit{@sameAs}, \textit{@copyOf}, \textit{@next}, \textit{@prev}, \textit{@exclude}, \textit{@select})) (\hyperref[TEI.att.global.analytic]{att.global.analytic} (\textit{@ana})) (\hyperref[TEI.att.global.facs]{att.global.facs} (\textit{@facs})) (\hyperref[TEI.att.global.change]{att.global.change} (\textit{@change})) (\hyperref[TEI.att.global.responsibility]{att.global.responsibility} (\textit{@cert}, \textit{@resp})) (\hyperref[TEI.att.global.source]{att.global.source} (\textit{@source})) \hyperref[TEI.att.translatable]{att.translatable} (\textit{@versionDate}) \hyperref[TEI.att.typed]{att.typed} (\textit{@type}, \textit{@subtype}) 
    \item[{Membre du}]
  \hyperref[TEI.model.descLike]{model.descLike} \hyperref[TEI.model.labelLike]{model.labelLike}
    \item[{Contenu dans}]
  
    \item[analysis: ]
   \hyperref[TEI.interp]{interp} \hyperref[TEI.interpGrp]{interpGrp}\par 
    \item[core: ]
   \hyperref[TEI.add]{add} \hyperref[TEI.corr]{corr} \hyperref[TEI.del]{del} \hyperref[TEI.desc]{desc} \hyperref[TEI.emph]{emph} \hyperref[TEI.gap]{gap} \hyperref[TEI.graphic]{graphic} \hyperref[TEI.head]{head} \hyperref[TEI.hi]{hi} \hyperref[TEI.item]{item} \hyperref[TEI.l]{l} \hyperref[TEI.lg]{lg} \hyperref[TEI.media]{media} \hyperref[TEI.meeting]{meeting} \hyperref[TEI.note]{note} \hyperref[TEI.orig]{orig} \hyperref[TEI.p]{p} \hyperref[TEI.q]{q} \hyperref[TEI.quote]{quote} \hyperref[TEI.ref]{ref} \hyperref[TEI.reg]{reg} \hyperref[TEI.said]{said} \hyperref[TEI.sic]{sic} \hyperref[TEI.stage]{stage} \hyperref[TEI.title]{title} \hyperref[TEI.unclear]{unclear}\par 
    \item[figures: ]
   \hyperref[TEI.cell]{cell} \hyperref[TEI.figDesc]{figDesc} \hyperref[TEI.figure]{figure} \hyperref[TEI.notatedMusic]{notatedMusic}\par 
    \item[header: ]
   \hyperref[TEI.application]{application} \hyperref[TEI.category]{category} \hyperref[TEI.change]{change} \hyperref[TEI.licence]{licence} \hyperref[TEI.rendition]{rendition} \hyperref[TEI.schemaRef]{schemaRef} \hyperref[TEI.taxonomy]{taxonomy}\par 
    \item[iso-fs: ]
   \hyperref[TEI.fDescr]{fDescr} \hyperref[TEI.fsDescr]{fsDescr}\par 
    \item[linking: ]
   \hyperref[TEI.ab]{ab} \hyperref[TEI.join]{join} \hyperref[TEI.seg]{seg}\par 
    \item[msdescription: ]
   \hyperref[TEI.accMat]{accMat} \hyperref[TEI.acquisition]{acquisition} \hyperref[TEI.additions]{additions} \hyperref[TEI.collation]{collation} \hyperref[TEI.condition]{condition} \hyperref[TEI.custEvent]{custEvent} \hyperref[TEI.decoNote]{decoNote} \hyperref[TEI.filiation]{filiation} \hyperref[TEI.foliation]{foliation} \hyperref[TEI.layout]{layout} \hyperref[TEI.musicNotation]{musicNotation} \hyperref[TEI.origin]{origin} \hyperref[TEI.provenance]{provenance} \hyperref[TEI.signatures]{signatures} \hyperref[TEI.source]{source} \hyperref[TEI.summary]{summary} \hyperref[TEI.support]{support} \hyperref[TEI.surrogates]{surrogates} \hyperref[TEI.typeNote]{typeNote}\par 
    \item[namesdates: ]
   \hyperref[TEI.event]{event} \hyperref[TEI.location]{location} \hyperref[TEI.org]{org} \hyperref[TEI.place]{place} \hyperref[TEI.state]{state}\par 
    \item[textstructure: ]
   \hyperref[TEI.body]{body} \hyperref[TEI.div]{div} \hyperref[TEI.docEdition]{docEdition} \hyperref[TEI.titlePart]{titlePart}\par 
    \item[transcr: ]
   \hyperref[TEI.damage]{damage} \hyperref[TEI.metamark]{metamark} \hyperref[TEI.mod]{mod} \hyperref[TEI.restore]{restore} \hyperref[TEI.retrace]{retrace} \hyperref[TEI.secl]{secl} \hyperref[TEI.space]{space} \hyperref[TEI.substJoin]{substJoin} \hyperref[TEI.supplied]{supplied} \hyperref[TEI.surface]{surface} \hyperref[TEI.surplus]{surplus}
    \item[{Peut contenir}]
  
    \item[core: ]
   \hyperref[TEI.abbr]{abbr} \hyperref[TEI.address]{address} \hyperref[TEI.bibl]{bibl} \hyperref[TEI.biblStruct]{biblStruct} \hyperref[TEI.choice]{choice} \hyperref[TEI.cit]{cit} \hyperref[TEI.date]{date} \hyperref[TEI.desc]{desc} \hyperref[TEI.distinct]{distinct} \hyperref[TEI.email]{email} \hyperref[TEI.emph]{emph} \hyperref[TEI.expan]{expan} \hyperref[TEI.foreign]{foreign} \hyperref[TEI.gloss]{gloss} \hyperref[TEI.hi]{hi} \hyperref[TEI.label]{label} \hyperref[TEI.list]{list} \hyperref[TEI.listBibl]{listBibl} \hyperref[TEI.measure]{measure} \hyperref[TEI.measureGrp]{measureGrp} \hyperref[TEI.mentioned]{mentioned} \hyperref[TEI.name]{name} \hyperref[TEI.num]{num} \hyperref[TEI.ptr]{ptr} \hyperref[TEI.q]{q} \hyperref[TEI.quote]{quote} \hyperref[TEI.ref]{ref} \hyperref[TEI.rs]{rs} \hyperref[TEI.said]{said} \hyperref[TEI.soCalled]{soCalled} \hyperref[TEI.stage]{stage} \hyperref[TEI.term]{term} \hyperref[TEI.time]{time} \hyperref[TEI.title]{title}\par 
    \item[figures: ]
   \hyperref[TEI.table]{table}\par 
    \item[header: ]
   \hyperref[TEI.biblFull]{biblFull} \hyperref[TEI.idno]{idno}\par 
    \item[msdescription: ]
   \hyperref[TEI.catchwords]{catchwords} \hyperref[TEI.depth]{depth} \hyperref[TEI.dim]{dim} \hyperref[TEI.dimensions]{dimensions} \hyperref[TEI.height]{height} \hyperref[TEI.heraldry]{heraldry} \hyperref[TEI.locus]{locus} \hyperref[TEI.locusGrp]{locusGrp} \hyperref[TEI.material]{material} \hyperref[TEI.msDesc]{msDesc} \hyperref[TEI.objectType]{objectType} \hyperref[TEI.origDate]{origDate} \hyperref[TEI.origPlace]{origPlace} \hyperref[TEI.secFol]{secFol} \hyperref[TEI.signatures]{signatures} \hyperref[TEI.stamp]{stamp} \hyperref[TEI.watermark]{watermark} \hyperref[TEI.width]{width}\par 
    \item[namesdates: ]
   \hyperref[TEI.addName]{addName} \hyperref[TEI.affiliation]{affiliation} \hyperref[TEI.country]{country} \hyperref[TEI.forename]{forename} \hyperref[TEI.genName]{genName} \hyperref[TEI.geogName]{geogName} \hyperref[TEI.listOrg]{listOrg} \hyperref[TEI.listPlace]{listPlace} \hyperref[TEI.location]{location} \hyperref[TEI.nameLink]{nameLink} \hyperref[TEI.orgName]{orgName} \hyperref[TEI.persName]{persName} \hyperref[TEI.placeName]{placeName} \hyperref[TEI.region]{region} \hyperref[TEI.roleName]{roleName} \hyperref[TEI.settlement]{settlement} \hyperref[TEI.state]{state} \hyperref[TEI.surname]{surname}\par 
    \item[textstructure: ]
   \hyperref[TEI.floatingText]{floatingText}\par 
    \item[transcr: ]
   \hyperref[TEI.am]{am} \hyperref[TEI.ex]{ex} \hyperref[TEI.subst]{subst}\par des données textuelles
    \item[{Note}]
  \par
La convention TEI exige que cela soit exprimé sous la forme d'une proposition finie, introduite par un verbe actif.
    \item[{Exemple}]
  \leavevmode\bgroup\exampleFont \begin{shaded}\noindent\mbox{}{<\textbf{desc}>}contient une description brève de la raison d'être et du champ d'application d'un\mbox{}\newline 
 élément, d'un attribut ou de la valeur d'un attribut, d'une classe ou une entité.{</\textbf{desc}>}\end{shaded}\egroup 


    \item[{Modèle de contenu}]
  \mbox{}\hfill\\[-10pt]\begin{Verbatim}[fontsize=\small]
<content>
 <macroRef key="macro.limitedContent"/>
</content>
    
\end{Verbatim}

    \item[{Schéma Declaration}]
  \mbox{}\hfill\\[-10pt]\begin{Verbatim}[fontsize=\small]
element desc
{
   tei_att.global.attributes,
   tei_att.translatable.attributes,
   tei_att.typed.attributes,
   tei_macro.limitedContent}
\end{Verbatim}

\end{reflist}  \index{dim=<dim>|oddindex}
\begin{reflist}
\item[]\begin{specHead}{TEI.dim}{<dim> }contains any single measurement forming part of a dimensional specification of some sort. [\xref{http://www.tei-c.org/release/doc/tei-p5-doc/en/html/MS.html\#msdim}{10.3.4. Dimensions}]\end{specHead} 
    \item[{Module}]
  msdescription
    \item[{Attributs}]
  Attributs \hyperref[TEI.att.global]{att.global} (\textit{@xml:id}, \textit{@n}, \textit{@xml:lang}, \textit{@xml:base}, \textit{@xml:space})  (\hyperref[TEI.att.global.rendition]{att.global.rendition} (\textit{@rend}, \textit{@style}, \textit{@rendition})) (\hyperref[TEI.att.global.linking]{att.global.linking} (\textit{@corresp}, \textit{@synch}, \textit{@sameAs}, \textit{@copyOf}, \textit{@next}, \textit{@prev}, \textit{@exclude}, \textit{@select})) (\hyperref[TEI.att.global.analytic]{att.global.analytic} (\textit{@ana})) (\hyperref[TEI.att.global.facs]{att.global.facs} (\textit{@facs})) (\hyperref[TEI.att.global.change]{att.global.change} (\textit{@change})) (\hyperref[TEI.att.global.responsibility]{att.global.responsibility} (\textit{@cert}, \textit{@resp})) (\hyperref[TEI.att.global.source]{att.global.source} (\textit{@source})) \hyperref[TEI.att.typed]{att.typed} (\textit{@type}, \textit{@subtype}) \hyperref[TEI.att.dimensions]{att.dimensions} (\textit{@unit}, \textit{@quantity}, \textit{@extent}, \textit{@precision}, \textit{@scope})  (\hyperref[TEI.att.ranging]{att.ranging} (\textit{@atLeast}, \textit{@atMost}, \textit{@min}, \textit{@max}, \textit{@confidence}))
    \item[{Membre du}]
  \hyperref[TEI.model.measureLike]{model.measureLike}
    \item[{Contenu dans}]
  
    \item[analysis: ]
   \hyperref[TEI.cl]{cl} \hyperref[TEI.phr]{phr} \hyperref[TEI.s]{s} \hyperref[TEI.span]{span}\par 
    \item[core: ]
   \hyperref[TEI.abbr]{abbr} \hyperref[TEI.add]{add} \hyperref[TEI.addrLine]{addrLine} \hyperref[TEI.author]{author} \hyperref[TEI.bibl]{bibl} \hyperref[TEI.biblScope]{biblScope} \hyperref[TEI.citedRange]{citedRange} \hyperref[TEI.corr]{corr} \hyperref[TEI.date]{date} \hyperref[TEI.del]{del} \hyperref[TEI.desc]{desc} \hyperref[TEI.distinct]{distinct} \hyperref[TEI.editor]{editor} \hyperref[TEI.email]{email} \hyperref[TEI.emph]{emph} \hyperref[TEI.expan]{expan} \hyperref[TEI.foreign]{foreign} \hyperref[TEI.gloss]{gloss} \hyperref[TEI.head]{head} \hyperref[TEI.headItem]{headItem} \hyperref[TEI.headLabel]{headLabel} \hyperref[TEI.hi]{hi} \hyperref[TEI.item]{item} \hyperref[TEI.l]{l} \hyperref[TEI.label]{label} \hyperref[TEI.measure]{measure} \hyperref[TEI.measureGrp]{measureGrp} \hyperref[TEI.meeting]{meeting} \hyperref[TEI.mentioned]{mentioned} \hyperref[TEI.name]{name} \hyperref[TEI.note]{note} \hyperref[TEI.num]{num} \hyperref[TEI.orig]{orig} \hyperref[TEI.p]{p} \hyperref[TEI.pubPlace]{pubPlace} \hyperref[TEI.publisher]{publisher} \hyperref[TEI.q]{q} \hyperref[TEI.quote]{quote} \hyperref[TEI.ref]{ref} \hyperref[TEI.reg]{reg} \hyperref[TEI.resp]{resp} \hyperref[TEI.rs]{rs} \hyperref[TEI.said]{said} \hyperref[TEI.sic]{sic} \hyperref[TEI.soCalled]{soCalled} \hyperref[TEI.speaker]{speaker} \hyperref[TEI.stage]{stage} \hyperref[TEI.street]{street} \hyperref[TEI.term]{term} \hyperref[TEI.textLang]{textLang} \hyperref[TEI.time]{time} \hyperref[TEI.title]{title} \hyperref[TEI.unclear]{unclear}\par 
    \item[figures: ]
   \hyperref[TEI.cell]{cell} \hyperref[TEI.figDesc]{figDesc}\par 
    \item[header: ]
   \hyperref[TEI.authority]{authority} \hyperref[TEI.change]{change} \hyperref[TEI.classCode]{classCode} \hyperref[TEI.creation]{creation} \hyperref[TEI.distributor]{distributor} \hyperref[TEI.edition]{edition} \hyperref[TEI.extent]{extent} \hyperref[TEI.funder]{funder} \hyperref[TEI.language]{language} \hyperref[TEI.licence]{licence} \hyperref[TEI.rendition]{rendition}\par 
    \item[iso-fs: ]
   \hyperref[TEI.fDescr]{fDescr} \hyperref[TEI.fsDescr]{fsDescr}\par 
    \item[linking: ]
   \hyperref[TEI.ab]{ab} \hyperref[TEI.seg]{seg}\par 
    \item[msdescription: ]
   \hyperref[TEI.accMat]{accMat} \hyperref[TEI.acquisition]{acquisition} \hyperref[TEI.additions]{additions} \hyperref[TEI.catchwords]{catchwords} \hyperref[TEI.collation]{collation} \hyperref[TEI.colophon]{colophon} \hyperref[TEI.condition]{condition} \hyperref[TEI.custEvent]{custEvent} \hyperref[TEI.decoNote]{decoNote} \hyperref[TEI.dimensions]{dimensions} \hyperref[TEI.explicit]{explicit} \hyperref[TEI.filiation]{filiation} \hyperref[TEI.finalRubric]{finalRubric} \hyperref[TEI.foliation]{foliation} \hyperref[TEI.heraldry]{heraldry} \hyperref[TEI.incipit]{incipit} \hyperref[TEI.layout]{layout} \hyperref[TEI.material]{material} \hyperref[TEI.musicNotation]{musicNotation} \hyperref[TEI.objectType]{objectType} \hyperref[TEI.origDate]{origDate} \hyperref[TEI.origPlace]{origPlace} \hyperref[TEI.origin]{origin} \hyperref[TEI.provenance]{provenance} \hyperref[TEI.rubric]{rubric} \hyperref[TEI.secFol]{secFol} \hyperref[TEI.signatures]{signatures} \hyperref[TEI.source]{source} \hyperref[TEI.stamp]{stamp} \hyperref[TEI.summary]{summary} \hyperref[TEI.support]{support} \hyperref[TEI.surrogates]{surrogates} \hyperref[TEI.typeNote]{typeNote} \hyperref[TEI.watermark]{watermark}\par 
    \item[namesdates: ]
   \hyperref[TEI.addName]{addName} \hyperref[TEI.affiliation]{affiliation} \hyperref[TEI.country]{country} \hyperref[TEI.forename]{forename} \hyperref[TEI.genName]{genName} \hyperref[TEI.geogName]{geogName} \hyperref[TEI.location]{location} \hyperref[TEI.nameLink]{nameLink} \hyperref[TEI.orgName]{orgName} \hyperref[TEI.persName]{persName} \hyperref[TEI.placeName]{placeName} \hyperref[TEI.region]{region} \hyperref[TEI.roleName]{roleName} \hyperref[TEI.settlement]{settlement} \hyperref[TEI.surname]{surname}\par 
    \item[textstructure: ]
   \hyperref[TEI.docAuthor]{docAuthor} \hyperref[TEI.docDate]{docDate} \hyperref[TEI.docEdition]{docEdition} \hyperref[TEI.titlePart]{titlePart}\par 
    \item[transcr: ]
   \hyperref[TEI.damage]{damage} \hyperref[TEI.fw]{fw} \hyperref[TEI.metamark]{metamark} \hyperref[TEI.mod]{mod} \hyperref[TEI.restore]{restore} \hyperref[TEI.retrace]{retrace} \hyperref[TEI.secl]{secl} \hyperref[TEI.supplied]{supplied} \hyperref[TEI.surplus]{surplus}
    \item[{Peut contenir}]
  Des données textuelles uniquement
    \item[{Note}]
  \par
The specific elements \hyperref[TEI.width]{<width>}, \hyperref[TEI.height]{<height>}, and \hyperref[TEI.depth]{<depth>} should be used in preference to this generic element wherever appropriate.
    \item[{Exemple}]
  \leavevmode\bgroup\exampleFont \begin{shaded}\noindent\mbox{}{<\textbf{dim}\hspace*{6pt}{extent}="{4.67 in}"\hspace*{6pt}{type}="{circumference}"/>}\end{shaded}\egroup 


    \item[{Modèle de contenu}]
  \fbox{\ttfamily <content>\newline
 <macroRef key="macro.xtext"/>\newline
</content>\newline
    } 
    \item[{Schéma Declaration}]
  \mbox{}\hfill\\[-10pt]\begin{Verbatim}[fontsize=\small]
element dim
{
   tei_att.global.attributes,
   tei_att.typed.attributes,
   tei_att.dimensions.attributes,
   tei_macro.xtext}
\end{Verbatim}

\end{reflist}  \index{dimensions=<dimensions>|oddindex}\index{type=@type!<dimensions>|oddindex}
\begin{reflist}
\item[]\begin{specHead}{TEI.dimensions}{<dimensions> }(dimensions) contient une spécification des dimensions. [\xref{http://www.tei-c.org/release/doc/tei-p5-doc/en/html/MS.html\#msdim}{10.3.4. Dimensions}]\end{specHead} 
    \item[{Module}]
  msdescription
    \item[{Attributs}]
  Attributs \hyperref[TEI.att.global]{att.global} (\textit{@xml:id}, \textit{@n}, \textit{@xml:lang}, \textit{@xml:base}, \textit{@xml:space})  (\hyperref[TEI.att.global.rendition]{att.global.rendition} (\textit{@rend}, \textit{@style}, \textit{@rendition})) (\hyperref[TEI.att.global.linking]{att.global.linking} (\textit{@corresp}, \textit{@synch}, \textit{@sameAs}, \textit{@copyOf}, \textit{@next}, \textit{@prev}, \textit{@exclude}, \textit{@select})) (\hyperref[TEI.att.global.analytic]{att.global.analytic} (\textit{@ana})) (\hyperref[TEI.att.global.facs]{att.global.facs} (\textit{@facs})) (\hyperref[TEI.att.global.change]{att.global.change} (\textit{@change})) (\hyperref[TEI.att.global.responsibility]{att.global.responsibility} (\textit{@cert}, \textit{@resp})) (\hyperref[TEI.att.global.source]{att.global.source} (\textit{@source})) \hyperref[TEI.att.dimensions]{att.dimensions} (\textit{@unit}, \textit{@quantity}, \textit{@extent}, \textit{@precision}, \textit{@scope})  (\hyperref[TEI.att.ranging]{att.ranging} (\textit{@atLeast}, \textit{@atMost}, \textit{@min}, \textit{@max}, \textit{@confidence})) \hfil\\[-10pt]\begin{sansreflist}
    \item[@type]
  indique quel aspect de l'objet est mesuré.
\begin{reflist}
    \item[{Statut}]
  Optionel
    \item[{Type de données}]
  \hyperref[TEI.teidata.enumerated]{teidata.enumerated}
    \item[{Exemple de valeurs possibles:}]
  \begin{description}

\item[{leaves}]Les dimensions concernent une ou plusieurs feuilles (par exemple une feuille unique, un ensemble de feuilles ou une partie reliée séparément).
\item[{ruled}]Les dimensions concernent la zone de la réglure d'une feuille.
\item[{pricked}]Les dimensions concernent la zone d'une feuille qui a été piquée pour préparer la réglure (à utiliser lorsqu'elle diffère significativement de la zone réglée ou lorsque la réglure n'est pas mesurable).
\item[{written}]Les dimensions concernent la zone écrite de la feuille, dont la hauteur est mesurée depuis le haut des blancs sur la ligne d'écriture supérieure jusqu'au dernier des blancs sur la dernière ligne écrite.
\item[{miniatures}]Les dimensions concernent les miniatures contenues dans le manuscrit.
\item[{binding}]Les dimensions concernent la reliure qui contient le codex ou le manuscrit.
\item[{box}]Les dimensions concernent la boîte ou autre conteneur dans lequel le manuscrit est conservé.
\end{description} 
\end{reflist}  
\end{sansreflist}  
    \item[{Membre du}]
  \hyperref[TEI.model.pPart.msdesc]{model.pPart.msdesc}
    \item[{Contenu dans}]
  
    \item[analysis: ]
   \hyperref[TEI.cl]{cl} \hyperref[TEI.phr]{phr} \hyperref[TEI.s]{s} \hyperref[TEI.span]{span}\par 
    \item[core: ]
   \hyperref[TEI.abbr]{abbr} \hyperref[TEI.add]{add} \hyperref[TEI.addrLine]{addrLine} \hyperref[TEI.author]{author} \hyperref[TEI.biblScope]{biblScope} \hyperref[TEI.citedRange]{citedRange} \hyperref[TEI.corr]{corr} \hyperref[TEI.date]{date} \hyperref[TEI.del]{del} \hyperref[TEI.desc]{desc} \hyperref[TEI.distinct]{distinct} \hyperref[TEI.editor]{editor} \hyperref[TEI.email]{email} \hyperref[TEI.emph]{emph} \hyperref[TEI.expan]{expan} \hyperref[TEI.foreign]{foreign} \hyperref[TEI.gloss]{gloss} \hyperref[TEI.head]{head} \hyperref[TEI.headItem]{headItem} \hyperref[TEI.headLabel]{headLabel} \hyperref[TEI.hi]{hi} \hyperref[TEI.item]{item} \hyperref[TEI.l]{l} \hyperref[TEI.label]{label} \hyperref[TEI.measure]{measure} \hyperref[TEI.meeting]{meeting} \hyperref[TEI.mentioned]{mentioned} \hyperref[TEI.name]{name} \hyperref[TEI.note]{note} \hyperref[TEI.num]{num} \hyperref[TEI.orig]{orig} \hyperref[TEI.p]{p} \hyperref[TEI.pubPlace]{pubPlace} \hyperref[TEI.publisher]{publisher} \hyperref[TEI.q]{q} \hyperref[TEI.quote]{quote} \hyperref[TEI.ref]{ref} \hyperref[TEI.reg]{reg} \hyperref[TEI.resp]{resp} \hyperref[TEI.rs]{rs} \hyperref[TEI.said]{said} \hyperref[TEI.sic]{sic} \hyperref[TEI.soCalled]{soCalled} \hyperref[TEI.speaker]{speaker} \hyperref[TEI.stage]{stage} \hyperref[TEI.street]{street} \hyperref[TEI.term]{term} \hyperref[TEI.textLang]{textLang} \hyperref[TEI.time]{time} \hyperref[TEI.title]{title} \hyperref[TEI.unclear]{unclear}\par 
    \item[figures: ]
   \hyperref[TEI.cell]{cell} \hyperref[TEI.figDesc]{figDesc}\par 
    \item[header: ]
   \hyperref[TEI.authority]{authority} \hyperref[TEI.change]{change} \hyperref[TEI.classCode]{classCode} \hyperref[TEI.creation]{creation} \hyperref[TEI.distributor]{distributor} \hyperref[TEI.edition]{edition} \hyperref[TEI.extent]{extent} \hyperref[TEI.funder]{funder} \hyperref[TEI.language]{language} \hyperref[TEI.licence]{licence} \hyperref[TEI.rendition]{rendition}\par 
    \item[iso-fs: ]
   \hyperref[TEI.fDescr]{fDescr} \hyperref[TEI.fsDescr]{fsDescr}\par 
    \item[linking: ]
   \hyperref[TEI.ab]{ab} \hyperref[TEI.seg]{seg}\par 
    \item[msdescription: ]
   \hyperref[TEI.accMat]{accMat} \hyperref[TEI.acquisition]{acquisition} \hyperref[TEI.additions]{additions} \hyperref[TEI.catchwords]{catchwords} \hyperref[TEI.collation]{collation} \hyperref[TEI.colophon]{colophon} \hyperref[TEI.condition]{condition} \hyperref[TEI.custEvent]{custEvent} \hyperref[TEI.decoNote]{decoNote} \hyperref[TEI.explicit]{explicit} \hyperref[TEI.filiation]{filiation} \hyperref[TEI.finalRubric]{finalRubric} \hyperref[TEI.foliation]{foliation} \hyperref[TEI.heraldry]{heraldry} \hyperref[TEI.incipit]{incipit} \hyperref[TEI.layout]{layout} \hyperref[TEI.material]{material} \hyperref[TEI.musicNotation]{musicNotation} \hyperref[TEI.objectType]{objectType} \hyperref[TEI.origDate]{origDate} \hyperref[TEI.origPlace]{origPlace} \hyperref[TEI.origin]{origin} \hyperref[TEI.provenance]{provenance} \hyperref[TEI.rubric]{rubric} \hyperref[TEI.secFol]{secFol} \hyperref[TEI.signatures]{signatures} \hyperref[TEI.source]{source} \hyperref[TEI.stamp]{stamp} \hyperref[TEI.summary]{summary} \hyperref[TEI.support]{support} \hyperref[TEI.surrogates]{surrogates} \hyperref[TEI.typeNote]{typeNote} \hyperref[TEI.watermark]{watermark}\par 
    \item[namesdates: ]
   \hyperref[TEI.addName]{addName} \hyperref[TEI.affiliation]{affiliation} \hyperref[TEI.country]{country} \hyperref[TEI.forename]{forename} \hyperref[TEI.genName]{genName} \hyperref[TEI.geogName]{geogName} \hyperref[TEI.nameLink]{nameLink} \hyperref[TEI.orgName]{orgName} \hyperref[TEI.persName]{persName} \hyperref[TEI.placeName]{placeName} \hyperref[TEI.region]{region} \hyperref[TEI.roleName]{roleName} \hyperref[TEI.settlement]{settlement} \hyperref[TEI.surname]{surname}\par 
    \item[textstructure: ]
   \hyperref[TEI.docAuthor]{docAuthor} \hyperref[TEI.docDate]{docDate} \hyperref[TEI.docEdition]{docEdition} \hyperref[TEI.titlePart]{titlePart}\par 
    \item[transcr: ]
   \hyperref[TEI.damage]{damage} \hyperref[TEI.fw]{fw} \hyperref[TEI.metamark]{metamark} \hyperref[TEI.mod]{mod} \hyperref[TEI.restore]{restore} \hyperref[TEI.retrace]{retrace} \hyperref[TEI.secl]{secl} \hyperref[TEI.supplied]{supplied} \hyperref[TEI.surplus]{surplus}
    \item[{Peut contenir}]
  
    \item[msdescription: ]
   \hyperref[TEI.depth]{depth} \hyperref[TEI.dim]{dim} \hyperref[TEI.height]{height} \hyperref[TEI.width]{width}
    \item[{Note}]
  \par
Contient la mesure de la hauteur, de la largeur et de la profondeur d'un objet à 1, 2 ou 3 dimensions.
    \item[{Exemple}]
  \leavevmode\bgroup\exampleFont \begin{shaded}\noindent\mbox{}{<\textbf{dimensions}\hspace*{6pt}{type}="{leaves}">}\mbox{}\newline 
\hspace*{6pt}{<\textbf{height}\hspace*{6pt}{scope}="{range}">}157-160{</\textbf{height}>}\mbox{}\newline 
\hspace*{6pt}{<\textbf{width}>}105{</\textbf{width}>}\mbox{}\newline 
{</\textbf{dimensions}>}\mbox{}\newline 
{<\textbf{dimensions}\hspace*{6pt}{type}="{ruled}">}\mbox{}\newline 
\hspace*{6pt}{<\textbf{height}\hspace*{6pt}{scope}="{most}">}90{</\textbf{height}>}\mbox{}\newline 
\hspace*{6pt}{<\textbf{width}\hspace*{6pt}{scope}="{most}">}48{</\textbf{width}>}\mbox{}\newline 
{</\textbf{dimensions}>}\mbox{}\newline 
{<\textbf{dimensions}\hspace*{6pt}{unit}="{in}">}\mbox{}\newline 
\hspace*{6pt}{<\textbf{height}>}12{</\textbf{height}>}\mbox{}\newline 
\hspace*{6pt}{<\textbf{width}>}10{</\textbf{width}>}\mbox{}\newline 
{</\textbf{dimensions}>}\end{shaded}\egroup 


    \item[{Exemple}]
  \leavevmode\bgroup\exampleFont \begin{shaded}\noindent\mbox{}{<\textbf{dimensions}\hspace*{6pt}{type}="{binding}">}\mbox{}\newline 
\hspace*{6pt}{<\textbf{height}\hspace*{6pt}{unit}="{mm}">}328 (336){</\textbf{height}>}\mbox{}\newline 
\hspace*{6pt}{<\textbf{width}\hspace*{6pt}{unit}="{mm}">}203{</\textbf{width}>}\mbox{}\newline 
\hspace*{6pt}{<\textbf{depth}\hspace*{6pt}{unit}="{mm}">}74{</\textbf{depth}>}\mbox{}\newline 
{</\textbf{dimensions}>}\end{shaded}\egroup 


    \item[{Exemple}]
  Quand de simples quantités numériques sont impliquées, elles peuvent être exprimées par l'attribut {\itshape quantity} sur chaque ou sur tous les éléments enfants, comme dans l'exemple suivant :\leavevmode\bgroup\exampleFont \begin{shaded}\noindent\mbox{}{<\textbf{dimensions}\hspace*{6pt}{type}="{binding}">}\mbox{}\newline 
\hspace*{6pt}{<\textbf{height}\hspace*{6pt}{unit}="{mm}">}170{</\textbf{height}>}\mbox{}\newline 
\hspace*{6pt}{<\textbf{width}\hspace*{6pt}{unit}="{mm}">}98{</\textbf{width}>}\mbox{}\newline 
\hspace*{6pt}{<\textbf{depth}\hspace*{6pt}{unit}="{mm}">}15{</\textbf{depth}>}\mbox{}\newline 
{</\textbf{dimensions}>}\mbox{}\newline 
{<\textbf{dimensions}\hspace*{6pt}{type}="{binding}">}\mbox{}\newline 
\hspace*{6pt}{<\textbf{height}\hspace*{6pt}{unit}="{mm}">}168{</\textbf{height}>}\mbox{}\newline 
\hspace*{6pt}{<\textbf{width}\hspace*{6pt}{unit}="{mm}">}106{</\textbf{width}>}\mbox{}\newline 
\hspace*{6pt}{<\textbf{depth}\hspace*{6pt}{unit}="{mm}">}22{</\textbf{depth}>}\mbox{}\newline 
{</\textbf{dimensions}>}\end{shaded}\egroup 


    \item[{Schematron}]
   <s:report test="count(tei:width)> 1">The element <s:name/> may appear once only </s:report> <s:report test="count(tei:height)> 1">The element <s:name/> may appear once only </s:report> <s:report test="count(tei:depth)> 1">The element <s:name/> may appear once only </s:report>
    \item[{Modèle de contenu}]
  \mbox{}\hfill\\[-10pt]\begin{Verbatim}[fontsize=\small]
<content>
 <alternate maxOccurs="unbounded"
  minOccurs="0">
  <elementRef key="dim"/>
  <classRef key="model.dimLike"/>
 </alternate>
</content>
    
\end{Verbatim}

    \item[{Schéma Declaration}]
  \mbox{}\hfill\\[-10pt]\begin{Verbatim}[fontsize=\small]
element dimensions
{
   tei_att.global.attributes,
   tei_att.dimensions.attributes,
   attribute type { text }?,
   ( tei_dim | tei_model.dimLike )*
}
\end{Verbatim}

\end{reflist}  \index{distinct=<distinct>|oddindex}\index{type=@type!<distinct>|oddindex}\index{time=@time!<distinct>|oddindex}\index{space=@space!<distinct>|oddindex}\index{social=@social!<distinct>|oddindex}
\begin{reflist}
\item[]\begin{specHead}{TEI.distinct}{<distinct> }identifie tout mot ou toute expression en la considérérant comme linguistiquement spécifique, par exemple comme étant archaïque, technique, dialectale, inusitée, ou comme appartenant à une langue spécifique. [\xref{http://www.tei-c.org/release/doc/tei-p5-doc/en/html/CO.html\#COHQHD}{3.3.2.3. Other Linguistically Distinct Material}]\end{specHead} 
    \item[{Module}]
  core
    \item[{Attributs}]
  Attributs \hyperref[TEI.att.global]{att.global} (\textit{@xml:id}, \textit{@n}, \textit{@xml:lang}, \textit{@xml:base}, \textit{@xml:space})  (\hyperref[TEI.att.global.rendition]{att.global.rendition} (\textit{@rend}, \textit{@style}, \textit{@rendition})) (\hyperref[TEI.att.global.linking]{att.global.linking} (\textit{@corresp}, \textit{@synch}, \textit{@sameAs}, \textit{@copyOf}, \textit{@next}, \textit{@prev}, \textit{@exclude}, \textit{@select})) (\hyperref[TEI.att.global.analytic]{att.global.analytic} (\textit{@ana})) (\hyperref[TEI.att.global.facs]{att.global.facs} (\textit{@facs})) (\hyperref[TEI.att.global.change]{att.global.change} (\textit{@change})) (\hyperref[TEI.att.global.responsibility]{att.global.responsibility} (\textit{@cert}, \textit{@resp})) (\hyperref[TEI.att.global.source]{att.global.source} (\textit{@source})) \hfil\\[-10pt]\begin{sansreflist}
    \item[@type]
  précise la variété de langue ou le registre de langue auquels appartiennent le mot ou l'expression.
\begin{reflist}
    \item[{Statut}]
  Optionel
    \item[{Type de données}]
  \hyperref[TEI.teidata.enumerated]{teidata.enumerated}
\end{reflist}  
    \item[@time]
  précise comment l'expression est diachroniquement distincte.
\begin{reflist}
    \item[{Statut}]
  Optionel
    \item[{Type de données}]
  \hyperref[TEI.teidata.text]{teidata.text}
\end{reflist}  
    \item[@space]
  précise comment l'expression se caractérise de façon diatopique.
\begin{reflist}
    \item[{Statut}]
  Optionel
    \item[{Type de données}]
  \hyperref[TEI.teidata.text]{teidata.text}
\end{reflist}  
    \item[@social]
  précise comment l'expression se caractérise de façon diastatique.
\begin{reflist}
    \item[{Statut}]
  Optionel
    \item[{Type de données}]
  \hyperref[TEI.teidata.text]{teidata.text}
\end{reflist}  
\end{sansreflist}  
    \item[{Membre du}]
  \hyperref[TEI.model.emphLike]{model.emphLike}
    \item[{Contenu dans}]
  
    \item[analysis: ]
   \hyperref[TEI.cl]{cl} \hyperref[TEI.phr]{phr} \hyperref[TEI.s]{s} \hyperref[TEI.span]{span}\par 
    \item[core: ]
   \hyperref[TEI.abbr]{abbr} \hyperref[TEI.add]{add} \hyperref[TEI.addrLine]{addrLine} \hyperref[TEI.author]{author} \hyperref[TEI.bibl]{bibl} \hyperref[TEI.biblScope]{biblScope} \hyperref[TEI.citedRange]{citedRange} \hyperref[TEI.corr]{corr} \hyperref[TEI.date]{date} \hyperref[TEI.del]{del} \hyperref[TEI.desc]{desc} \hyperref[TEI.distinct]{distinct} \hyperref[TEI.editor]{editor} \hyperref[TEI.email]{email} \hyperref[TEI.emph]{emph} \hyperref[TEI.expan]{expan} \hyperref[TEI.foreign]{foreign} \hyperref[TEI.gloss]{gloss} \hyperref[TEI.head]{head} \hyperref[TEI.headItem]{headItem} \hyperref[TEI.headLabel]{headLabel} \hyperref[TEI.hi]{hi} \hyperref[TEI.item]{item} \hyperref[TEI.l]{l} \hyperref[TEI.label]{label} \hyperref[TEI.measure]{measure} \hyperref[TEI.meeting]{meeting} \hyperref[TEI.mentioned]{mentioned} \hyperref[TEI.name]{name} \hyperref[TEI.note]{note} \hyperref[TEI.num]{num} \hyperref[TEI.orig]{orig} \hyperref[TEI.p]{p} \hyperref[TEI.pubPlace]{pubPlace} \hyperref[TEI.publisher]{publisher} \hyperref[TEI.q]{q} \hyperref[TEI.quote]{quote} \hyperref[TEI.ref]{ref} \hyperref[TEI.reg]{reg} \hyperref[TEI.resp]{resp} \hyperref[TEI.rs]{rs} \hyperref[TEI.said]{said} \hyperref[TEI.sic]{sic} \hyperref[TEI.soCalled]{soCalled} \hyperref[TEI.speaker]{speaker} \hyperref[TEI.stage]{stage} \hyperref[TEI.street]{street} \hyperref[TEI.term]{term} \hyperref[TEI.textLang]{textLang} \hyperref[TEI.time]{time} \hyperref[TEI.title]{title} \hyperref[TEI.unclear]{unclear}\par 
    \item[figures: ]
   \hyperref[TEI.cell]{cell} \hyperref[TEI.figDesc]{figDesc}\par 
    \item[header: ]
   \hyperref[TEI.authority]{authority} \hyperref[TEI.change]{change} \hyperref[TEI.classCode]{classCode} \hyperref[TEI.creation]{creation} \hyperref[TEI.distributor]{distributor} \hyperref[TEI.edition]{edition} \hyperref[TEI.extent]{extent} \hyperref[TEI.funder]{funder} \hyperref[TEI.language]{language} \hyperref[TEI.licence]{licence} \hyperref[TEI.rendition]{rendition}\par 
    \item[iso-fs: ]
   \hyperref[TEI.fDescr]{fDescr} \hyperref[TEI.fsDescr]{fsDescr}\par 
    \item[linking: ]
   \hyperref[TEI.ab]{ab} \hyperref[TEI.seg]{seg}\par 
    \item[msdescription: ]
   \hyperref[TEI.accMat]{accMat} \hyperref[TEI.acquisition]{acquisition} \hyperref[TEI.additions]{additions} \hyperref[TEI.catchwords]{catchwords} \hyperref[TEI.collation]{collation} \hyperref[TEI.colophon]{colophon} \hyperref[TEI.condition]{condition} \hyperref[TEI.custEvent]{custEvent} \hyperref[TEI.decoNote]{decoNote} \hyperref[TEI.explicit]{explicit} \hyperref[TEI.filiation]{filiation} \hyperref[TEI.finalRubric]{finalRubric} \hyperref[TEI.foliation]{foliation} \hyperref[TEI.heraldry]{heraldry} \hyperref[TEI.incipit]{incipit} \hyperref[TEI.layout]{layout} \hyperref[TEI.material]{material} \hyperref[TEI.musicNotation]{musicNotation} \hyperref[TEI.objectType]{objectType} \hyperref[TEI.origDate]{origDate} \hyperref[TEI.origPlace]{origPlace} \hyperref[TEI.origin]{origin} \hyperref[TEI.provenance]{provenance} \hyperref[TEI.rubric]{rubric} \hyperref[TEI.secFol]{secFol} \hyperref[TEI.signatures]{signatures} \hyperref[TEI.source]{source} \hyperref[TEI.stamp]{stamp} \hyperref[TEI.summary]{summary} \hyperref[TEI.support]{support} \hyperref[TEI.surrogates]{surrogates} \hyperref[TEI.typeNote]{typeNote} \hyperref[TEI.watermark]{watermark}\par 
    \item[namesdates: ]
   \hyperref[TEI.addName]{addName} \hyperref[TEI.affiliation]{affiliation} \hyperref[TEI.country]{country} \hyperref[TEI.forename]{forename} \hyperref[TEI.genName]{genName} \hyperref[TEI.geogName]{geogName} \hyperref[TEI.nameLink]{nameLink} \hyperref[TEI.orgName]{orgName} \hyperref[TEI.persName]{persName} \hyperref[TEI.placeName]{placeName} \hyperref[TEI.region]{region} \hyperref[TEI.roleName]{roleName} \hyperref[TEI.settlement]{settlement} \hyperref[TEI.surname]{surname}\par 
    \item[textstructure: ]
   \hyperref[TEI.docAuthor]{docAuthor} \hyperref[TEI.docDate]{docDate} \hyperref[TEI.docEdition]{docEdition} \hyperref[TEI.titlePart]{titlePart}\par 
    \item[transcr: ]
   \hyperref[TEI.damage]{damage} \hyperref[TEI.fw]{fw} \hyperref[TEI.metamark]{metamark} \hyperref[TEI.mod]{mod} \hyperref[TEI.restore]{restore} \hyperref[TEI.retrace]{retrace} \hyperref[TEI.secl]{secl} \hyperref[TEI.supplied]{supplied} \hyperref[TEI.surplus]{surplus}
    \item[{Peut contenir}]
  
    \item[analysis: ]
   \hyperref[TEI.c]{c} \hyperref[TEI.cl]{cl} \hyperref[TEI.interp]{interp} \hyperref[TEI.interpGrp]{interpGrp} \hyperref[TEI.m]{m} \hyperref[TEI.pc]{pc} \hyperref[TEI.phr]{phr} \hyperref[TEI.s]{s} \hyperref[TEI.span]{span} \hyperref[TEI.spanGrp]{spanGrp} \hyperref[TEI.w]{w}\par 
    \item[core: ]
   \hyperref[TEI.abbr]{abbr} \hyperref[TEI.add]{add} \hyperref[TEI.address]{address} \hyperref[TEI.binaryObject]{binaryObject} \hyperref[TEI.cb]{cb} \hyperref[TEI.choice]{choice} \hyperref[TEI.corr]{corr} \hyperref[TEI.date]{date} \hyperref[TEI.del]{del} \hyperref[TEI.distinct]{distinct} \hyperref[TEI.email]{email} \hyperref[TEI.emph]{emph} \hyperref[TEI.expan]{expan} \hyperref[TEI.foreign]{foreign} \hyperref[TEI.gap]{gap} \hyperref[TEI.gb]{gb} \hyperref[TEI.gloss]{gloss} \hyperref[TEI.graphic]{graphic} \hyperref[TEI.hi]{hi} \hyperref[TEI.index]{index} \hyperref[TEI.lb]{lb} \hyperref[TEI.measure]{measure} \hyperref[TEI.measureGrp]{measureGrp} \hyperref[TEI.media]{media} \hyperref[TEI.mentioned]{mentioned} \hyperref[TEI.milestone]{milestone} \hyperref[TEI.name]{name} \hyperref[TEI.note]{note} \hyperref[TEI.num]{num} \hyperref[TEI.orig]{orig} \hyperref[TEI.pb]{pb} \hyperref[TEI.ptr]{ptr} \hyperref[TEI.ref]{ref} \hyperref[TEI.reg]{reg} \hyperref[TEI.rs]{rs} \hyperref[TEI.sic]{sic} \hyperref[TEI.soCalled]{soCalled} \hyperref[TEI.term]{term} \hyperref[TEI.time]{time} \hyperref[TEI.title]{title} \hyperref[TEI.unclear]{unclear}\par 
    \item[derived-module-tei.istex: ]
   \hyperref[TEI.math]{math} \hyperref[TEI.mrow]{mrow}\par 
    \item[figures: ]
   \hyperref[TEI.figure]{figure} \hyperref[TEI.formula]{formula} \hyperref[TEI.notatedMusic]{notatedMusic}\par 
    \item[header: ]
   \hyperref[TEI.idno]{idno}\par 
    \item[iso-fs: ]
   \hyperref[TEI.fLib]{fLib} \hyperref[TEI.fs]{fs} \hyperref[TEI.fvLib]{fvLib}\par 
    \item[linking: ]
   \hyperref[TEI.alt]{alt} \hyperref[TEI.altGrp]{altGrp} \hyperref[TEI.anchor]{anchor} \hyperref[TEI.join]{join} \hyperref[TEI.joinGrp]{joinGrp} \hyperref[TEI.link]{link} \hyperref[TEI.linkGrp]{linkGrp} \hyperref[TEI.seg]{seg} \hyperref[TEI.timeline]{timeline}\par 
    \item[msdescription: ]
   \hyperref[TEI.catchwords]{catchwords} \hyperref[TEI.depth]{depth} \hyperref[TEI.dim]{dim} \hyperref[TEI.dimensions]{dimensions} \hyperref[TEI.height]{height} \hyperref[TEI.heraldry]{heraldry} \hyperref[TEI.locus]{locus} \hyperref[TEI.locusGrp]{locusGrp} \hyperref[TEI.material]{material} \hyperref[TEI.objectType]{objectType} \hyperref[TEI.origDate]{origDate} \hyperref[TEI.origPlace]{origPlace} \hyperref[TEI.secFol]{secFol} \hyperref[TEI.signatures]{signatures} \hyperref[TEI.source]{source} \hyperref[TEI.stamp]{stamp} \hyperref[TEI.watermark]{watermark} \hyperref[TEI.width]{width}\par 
    \item[namesdates: ]
   \hyperref[TEI.addName]{addName} \hyperref[TEI.affiliation]{affiliation} \hyperref[TEI.country]{country} \hyperref[TEI.forename]{forename} \hyperref[TEI.genName]{genName} \hyperref[TEI.geogName]{geogName} \hyperref[TEI.location]{location} \hyperref[TEI.nameLink]{nameLink} \hyperref[TEI.orgName]{orgName} \hyperref[TEI.persName]{persName} \hyperref[TEI.placeName]{placeName} \hyperref[TEI.region]{region} \hyperref[TEI.roleName]{roleName} \hyperref[TEI.settlement]{settlement} \hyperref[TEI.state]{state} \hyperref[TEI.surname]{surname}\par 
    \item[spoken: ]
   \hyperref[TEI.annotationBlock]{annotationBlock}\par 
    \item[transcr: ]
   \hyperref[TEI.addSpan]{addSpan} \hyperref[TEI.am]{am} \hyperref[TEI.damage]{damage} \hyperref[TEI.damageSpan]{damageSpan} \hyperref[TEI.delSpan]{delSpan} \hyperref[TEI.ex]{ex} \hyperref[TEI.fw]{fw} \hyperref[TEI.handShift]{handShift} \hyperref[TEI.listTranspose]{listTranspose} \hyperref[TEI.metamark]{metamark} \hyperref[TEI.mod]{mod} \hyperref[TEI.redo]{redo} \hyperref[TEI.restore]{restore} \hyperref[TEI.retrace]{retrace} \hyperref[TEI.secl]{secl} \hyperref[TEI.space]{space} \hyperref[TEI.subst]{subst} \hyperref[TEI.substJoin]{substJoin} \hyperref[TEI.supplied]{supplied} \hyperref[TEI.surplus]{surplus} \hyperref[TEI.undo]{undo}\par des données textuelles
    \item[{Exemple}]
  \leavevmode\bgroup\exampleFont \begin{shaded}\noindent\mbox{}{<\textbf{p}>}- Elle fait chier, cette {<\textbf{distinct}\hspace*{6pt}{type}="{verlan}">}meuf{</\textbf{distinct}>}. Tu confonds amour et\mbox{}\newline 
{<\textbf{distinct}\hspace*{6pt}{type}="{argot}">}piquouse{</\textbf{distinct}>}. Tu l'auras, ton {<\textbf{distinct}\hspace*{6pt}{type}="{argot}">}shoot{</\textbf{distinct}>}, merde ! {</\textbf{p}>}\end{shaded}\egroup 


    \item[{Modèle de contenu}]
  \mbox{}\hfill\\[-10pt]\begin{Verbatim}[fontsize=\small]
<content>
 <macroRef key="macro.phraseSeq"/>
</content>
    
\end{Verbatim}

    \item[{Schéma Declaration}]
  \mbox{}\hfill\\[-10pt]\begin{Verbatim}[fontsize=\small]
element distinct
{
   tei_att.global.attributes,
   attribute type { text }?,
   attribute time { text }?,
   attribute space { text }?,
   attribute social { text }?,
   tei_macro.phraseSeq}
\end{Verbatim}

\end{reflist}  \index{distributor=<distributor>|oddindex}
\begin{reflist}
\item[]\begin{specHead}{TEI.distributor}{<distributor> }(diffuseur) donne le nom d’une personne ou d’un organisme responsable de la diffusion d’un texte. [\xref{http://www.tei-c.org/release/doc/tei-p5-doc/en/html/HD.html\#HD24}{2.2.4. Publication, Distribution, Licensing, etc.}]\end{specHead} 
    \item[{Module}]
  header
    \item[{Attributs}]
  Attributs \hyperref[TEI.att.global]{att.global} (\textit{@xml:id}, \textit{@n}, \textit{@xml:lang}, \textit{@xml:base}, \textit{@xml:space})  (\hyperref[TEI.att.global.rendition]{att.global.rendition} (\textit{@rend}, \textit{@style}, \textit{@rendition})) (\hyperref[TEI.att.global.linking]{att.global.linking} (\textit{@corresp}, \textit{@synch}, \textit{@sameAs}, \textit{@copyOf}, \textit{@next}, \textit{@prev}, \textit{@exclude}, \textit{@select})) (\hyperref[TEI.att.global.analytic]{att.global.analytic} (\textit{@ana})) (\hyperref[TEI.att.global.facs]{att.global.facs} (\textit{@facs})) (\hyperref[TEI.att.global.change]{att.global.change} (\textit{@change})) (\hyperref[TEI.att.global.responsibility]{att.global.responsibility} (\textit{@cert}, \textit{@resp})) (\hyperref[TEI.att.global.source]{att.global.source} (\textit{@source}))
    \item[{Membre du}]
  \hyperref[TEI.model.imprintPart]{model.imprintPart} \hyperref[TEI.model.publicationStmtPart.agency]{model.publicationStmtPart.agency}
    \item[{Contenu dans}]
  
    \item[core: ]
   \hyperref[TEI.bibl]{bibl} \hyperref[TEI.imprint]{imprint}\par 
    \item[header: ]
   \hyperref[TEI.publicationStmt]{publicationStmt}
    \item[{Peut contenir}]
  
    \item[analysis: ]
   \hyperref[TEI.c]{c} \hyperref[TEI.cl]{cl} \hyperref[TEI.interp]{interp} \hyperref[TEI.interpGrp]{interpGrp} \hyperref[TEI.m]{m} \hyperref[TEI.pc]{pc} \hyperref[TEI.phr]{phr} \hyperref[TEI.s]{s} \hyperref[TEI.span]{span} \hyperref[TEI.spanGrp]{spanGrp} \hyperref[TEI.w]{w}\par 
    \item[core: ]
   \hyperref[TEI.abbr]{abbr} \hyperref[TEI.add]{add} \hyperref[TEI.address]{address} \hyperref[TEI.binaryObject]{binaryObject} \hyperref[TEI.cb]{cb} \hyperref[TEI.choice]{choice} \hyperref[TEI.corr]{corr} \hyperref[TEI.date]{date} \hyperref[TEI.del]{del} \hyperref[TEI.distinct]{distinct} \hyperref[TEI.email]{email} \hyperref[TEI.emph]{emph} \hyperref[TEI.expan]{expan} \hyperref[TEI.foreign]{foreign} \hyperref[TEI.gap]{gap} \hyperref[TEI.gb]{gb} \hyperref[TEI.gloss]{gloss} \hyperref[TEI.graphic]{graphic} \hyperref[TEI.hi]{hi} \hyperref[TEI.index]{index} \hyperref[TEI.lb]{lb} \hyperref[TEI.measure]{measure} \hyperref[TEI.measureGrp]{measureGrp} \hyperref[TEI.media]{media} \hyperref[TEI.mentioned]{mentioned} \hyperref[TEI.milestone]{milestone} \hyperref[TEI.name]{name} \hyperref[TEI.note]{note} \hyperref[TEI.num]{num} \hyperref[TEI.orig]{orig} \hyperref[TEI.pb]{pb} \hyperref[TEI.ptr]{ptr} \hyperref[TEI.ref]{ref} \hyperref[TEI.reg]{reg} \hyperref[TEI.rs]{rs} \hyperref[TEI.sic]{sic} \hyperref[TEI.soCalled]{soCalled} \hyperref[TEI.term]{term} \hyperref[TEI.time]{time} \hyperref[TEI.title]{title} \hyperref[TEI.unclear]{unclear}\par 
    \item[derived-module-tei.istex: ]
   \hyperref[TEI.math]{math} \hyperref[TEI.mrow]{mrow}\par 
    \item[figures: ]
   \hyperref[TEI.figure]{figure} \hyperref[TEI.formula]{formula} \hyperref[TEI.notatedMusic]{notatedMusic}\par 
    \item[header: ]
   \hyperref[TEI.idno]{idno}\par 
    \item[iso-fs: ]
   \hyperref[TEI.fLib]{fLib} \hyperref[TEI.fs]{fs} \hyperref[TEI.fvLib]{fvLib}\par 
    \item[linking: ]
   \hyperref[TEI.alt]{alt} \hyperref[TEI.altGrp]{altGrp} \hyperref[TEI.anchor]{anchor} \hyperref[TEI.join]{join} \hyperref[TEI.joinGrp]{joinGrp} \hyperref[TEI.link]{link} \hyperref[TEI.linkGrp]{linkGrp} \hyperref[TEI.seg]{seg} \hyperref[TEI.timeline]{timeline}\par 
    \item[msdescription: ]
   \hyperref[TEI.catchwords]{catchwords} \hyperref[TEI.depth]{depth} \hyperref[TEI.dim]{dim} \hyperref[TEI.dimensions]{dimensions} \hyperref[TEI.height]{height} \hyperref[TEI.heraldry]{heraldry} \hyperref[TEI.locus]{locus} \hyperref[TEI.locusGrp]{locusGrp} \hyperref[TEI.material]{material} \hyperref[TEI.objectType]{objectType} \hyperref[TEI.origDate]{origDate} \hyperref[TEI.origPlace]{origPlace} \hyperref[TEI.secFol]{secFol} \hyperref[TEI.signatures]{signatures} \hyperref[TEI.source]{source} \hyperref[TEI.stamp]{stamp} \hyperref[TEI.watermark]{watermark} \hyperref[TEI.width]{width}\par 
    \item[namesdates: ]
   \hyperref[TEI.addName]{addName} \hyperref[TEI.affiliation]{affiliation} \hyperref[TEI.country]{country} \hyperref[TEI.forename]{forename} \hyperref[TEI.genName]{genName} \hyperref[TEI.geogName]{geogName} \hyperref[TEI.location]{location} \hyperref[TEI.nameLink]{nameLink} \hyperref[TEI.orgName]{orgName} \hyperref[TEI.persName]{persName} \hyperref[TEI.placeName]{placeName} \hyperref[TEI.region]{region} \hyperref[TEI.roleName]{roleName} \hyperref[TEI.settlement]{settlement} \hyperref[TEI.state]{state} \hyperref[TEI.surname]{surname}\par 
    \item[spoken: ]
   \hyperref[TEI.annotationBlock]{annotationBlock}\par 
    \item[transcr: ]
   \hyperref[TEI.addSpan]{addSpan} \hyperref[TEI.am]{am} \hyperref[TEI.damage]{damage} \hyperref[TEI.damageSpan]{damageSpan} \hyperref[TEI.delSpan]{delSpan} \hyperref[TEI.ex]{ex} \hyperref[TEI.fw]{fw} \hyperref[TEI.handShift]{handShift} \hyperref[TEI.listTranspose]{listTranspose} \hyperref[TEI.metamark]{metamark} \hyperref[TEI.mod]{mod} \hyperref[TEI.redo]{redo} \hyperref[TEI.restore]{restore} \hyperref[TEI.retrace]{retrace} \hyperref[TEI.secl]{secl} \hyperref[TEI.space]{space} \hyperref[TEI.subst]{subst} \hyperref[TEI.substJoin]{substJoin} \hyperref[TEI.supplied]{supplied} \hyperref[TEI.surplus]{surplus} \hyperref[TEI.undo]{undo}\par des données textuelles
    \item[{Exemple}]
  \leavevmode\bgroup\exampleFont \begin{shaded}\noindent\mbox{}{<\textbf{distributor}>}Laboratoire : Analyse et Traitement Informatique de la Langue Française){</\textbf{distributor}>}\mbox{}\newline 
{<\textbf{distributor}>}Centre National de la Recherche Scientifique{</\textbf{distributor}>}\end{shaded}\egroup 


    \item[{Modèle de contenu}]
  \mbox{}\hfill\\[-10pt]\begin{Verbatim}[fontsize=\small]
<content>
 <macroRef key="macro.phraseSeq"/>
</content>
    
\end{Verbatim}

    \item[{Schéma Declaration}]
  \mbox{}\hfill\\[-10pt]\begin{Verbatim}[fontsize=\small]
element distributor { tei_att.global.attributes, tei_macro.phraseSeq }
\end{Verbatim}

\end{reflist}  \index{div=<div>|oddindex}
\begin{reflist}
\item[]\begin{specHead}{TEI.div}{<div> }(division du texte) contient une subdivision dans le texte préliminaire, dans le corps d’un texte ou dans le texte postliminaire. [\xref{http://www.tei-c.org/release/doc/tei-p5-doc/en/html/DS.html\#DSDIV}{4.1. Divisions of the Body}]\end{specHead} 
    \item[{Module}]
  textstructure
    \item[{Attributs}]
  Attributs \hyperref[TEI.att.global]{att.global} (\textit{@xml:id}, \textit{@n}, \textit{@xml:lang}, \textit{@xml:base}, \textit{@xml:space})  (\hyperref[TEI.att.global.rendition]{att.global.rendition} (\textit{@rend}, \textit{@style}, \textit{@rendition})) (\hyperref[TEI.att.global.linking]{att.global.linking} (\textit{@corresp}, \textit{@synch}, \textit{@sameAs}, \textit{@copyOf}, \textit{@next}, \textit{@prev}, \textit{@exclude}, \textit{@select})) (\hyperref[TEI.att.global.analytic]{att.global.analytic} (\textit{@ana})) (\hyperref[TEI.att.global.facs]{att.global.facs} (\textit{@facs})) (\hyperref[TEI.att.global.change]{att.global.change} (\textit{@change})) (\hyperref[TEI.att.global.responsibility]{att.global.responsibility} (\textit{@cert}, \textit{@resp})) (\hyperref[TEI.att.global.source]{att.global.source} (\textit{@source})) \hyperref[TEI.att.divLike]{att.divLike} (\textit{@org}, \textit{@sample})  (\hyperref[TEI.att.fragmentable]{att.fragmentable} (\textit{@part})) \hyperref[TEI.att.typed]{att.typed} (\textit{@type}, \textit{@subtype}) \hyperref[TEI.att.declaring]{att.declaring} (\textit{@decls}) \hyperref[TEI.att.written]{att.written} (\textit{@hand}) 
    \item[{Membre du}]
  \hyperref[TEI.model.divLike]{model.divLike}
    \item[{Contenu dans}]
  
    \item[textstructure: ]
   \hyperref[TEI.back]{back} \hyperref[TEI.body]{body} \hyperref[TEI.div]{div} \hyperref[TEI.front]{front}
    \item[{Peut contenir}]
  
    \item[analysis: ]
   \hyperref[TEI.interp]{interp} \hyperref[TEI.interpGrp]{interpGrp} \hyperref[TEI.span]{span} \hyperref[TEI.spanGrp]{spanGrp}\par 
    \item[core: ]
   \hyperref[TEI.bibl]{bibl} \hyperref[TEI.biblStruct]{biblStruct} \hyperref[TEI.cb]{cb} \hyperref[TEI.cit]{cit} \hyperref[TEI.desc]{desc} \hyperref[TEI.divGen]{divGen} \hyperref[TEI.gap]{gap} \hyperref[TEI.gb]{gb} \hyperref[TEI.head]{head} \hyperref[TEI.index]{index} \hyperref[TEI.l]{l} \hyperref[TEI.label]{label} \hyperref[TEI.lb]{lb} \hyperref[TEI.lg]{lg} \hyperref[TEI.list]{list} \hyperref[TEI.listBibl]{listBibl} \hyperref[TEI.meeting]{meeting} \hyperref[TEI.milestone]{milestone} \hyperref[TEI.note]{note} \hyperref[TEI.p]{p} \hyperref[TEI.pb]{pb} \hyperref[TEI.q]{q} \hyperref[TEI.quote]{quote} \hyperref[TEI.said]{said} \hyperref[TEI.sp]{sp} \hyperref[TEI.stage]{stage}\par 
    \item[figures: ]
   \hyperref[TEI.figure]{figure} \hyperref[TEI.notatedMusic]{notatedMusic} \hyperref[TEI.table]{table}\par 
    \item[header: ]
   \hyperref[TEI.biblFull]{biblFull}\par 
    \item[iso-fs: ]
   \hyperref[TEI.fLib]{fLib} \hyperref[TEI.fs]{fs} \hyperref[TEI.fvLib]{fvLib}\par 
    \item[linking: ]
   \hyperref[TEI.ab]{ab} \hyperref[TEI.alt]{alt} \hyperref[TEI.altGrp]{altGrp} \hyperref[TEI.anchor]{anchor} \hyperref[TEI.join]{join} \hyperref[TEI.joinGrp]{joinGrp} \hyperref[TEI.link]{link} \hyperref[TEI.linkGrp]{linkGrp} \hyperref[TEI.timeline]{timeline}\par 
    \item[msdescription: ]
   \hyperref[TEI.msDesc]{msDesc} \hyperref[TEI.source]{source}\par 
    \item[namesdates: ]
   \hyperref[TEI.listOrg]{listOrg} \hyperref[TEI.listPlace]{listPlace}\par 
    \item[spoken: ]
   \hyperref[TEI.annotationBlock]{annotationBlock}\par 
    \item[textstructure: ]
   \hyperref[TEI.div]{div} \hyperref[TEI.docAuthor]{docAuthor} \hyperref[TEI.docDate]{docDate} \hyperref[TEI.floatingText]{floatingText}\par 
    \item[transcr: ]
   \hyperref[TEI.addSpan]{addSpan} \hyperref[TEI.damageSpan]{damageSpan} \hyperref[TEI.delSpan]{delSpan} \hyperref[TEI.fw]{fw} \hyperref[TEI.listTranspose]{listTranspose} \hyperref[TEI.metamark]{metamark} \hyperref[TEI.space]{space} \hyperref[TEI.substJoin]{substJoin}
    \item[{Exemple}]
  \leavevmode\bgroup\exampleFont \begin{shaded}\noindent\mbox{}{<\textbf{body}>}\mbox{}\newline 
\hspace*{6pt}{<\textbf{div}\hspace*{6pt}{type}="{oeuvre}">}\mbox{}\newline 
\hspace*{6pt}\hspace*{6pt}{<\textbf{head}>}Les Chouans {</\textbf{head}>}\mbox{}\newline 
\hspace*{6pt}\hspace*{6pt}{<\textbf{div}\hspace*{6pt}{n}="{1}"\hspace*{6pt}{type}="{partie}">}\mbox{}\newline 
\hspace*{6pt}\hspace*{6pt}\hspace*{6pt}{<\textbf{head}>} Première partie{</\textbf{head}>}\mbox{}\newline 
\hspace*{6pt}\hspace*{6pt}\hspace*{6pt}{<\textbf{head}>} L'embuscade{</\textbf{head}>}\mbox{}\newline 
\hspace*{6pt}\hspace*{6pt}\hspace*{6pt}{<\textbf{div}\hspace*{6pt}{n}="{1}"\hspace*{6pt}{type}="{chapitre}">}\mbox{}\newline 
\hspace*{6pt}\hspace*{6pt}\hspace*{6pt}\hspace*{6pt}{<\textbf{head}>}Chapitre premier {</\textbf{head}>}\mbox{}\newline 
\hspace*{6pt}\hspace*{6pt}\hspace*{6pt}\hspace*{6pt}{<\textbf{div}\hspace*{6pt}{n}="{1}">}\mbox{}\newline 
\hspace*{6pt}\hspace*{6pt}\hspace*{6pt}\hspace*{6pt}\hspace*{6pt}{<\textbf{head}>}I{</\textbf{head}>}\mbox{}\newline 
\hspace*{6pt}\hspace*{6pt}\hspace*{6pt}\hspace*{6pt}\hspace*{6pt}{<\textbf{p}>}Dans les premiers jours de l'an VIII, au commencement de vendémiaire, ou, pour\mbox{}\newline 
\hspace*{6pt}\hspace*{6pt}\hspace*{6pt}\hspace*{6pt}\hspace*{6pt}\hspace*{6pt}\hspace*{6pt}\hspace*{6pt}\hspace*{6pt}\hspace*{6pt} se conformer au calendrier actuel, vers la fin du mois de septembre 1799, une\mbox{}\newline 
\hspace*{6pt}\hspace*{6pt}\hspace*{6pt}\hspace*{6pt}\hspace*{6pt}\hspace*{6pt}\hspace*{6pt}\hspace*{6pt}\hspace*{6pt}\hspace*{6pt} centaine de paysans et un assez grand nombre de bourgeois, partis le matin de\mbox{}\newline 
\hspace*{6pt}\hspace*{6pt}\hspace*{6pt}\hspace*{6pt}\hspace*{6pt}\hspace*{6pt}\hspace*{6pt}\hspace*{6pt}\hspace*{6pt}\hspace*{6pt} Fougères pour se rendre à Mayenne, gravissaient la montagne de la Pèlerine,\mbox{}\newline 
\hspace*{6pt}\hspace*{6pt}\hspace*{6pt}\hspace*{6pt}\hspace*{6pt}\hspace*{6pt}\hspace*{6pt}\hspace*{6pt}\hspace*{6pt}\hspace*{6pt} située à mi-chemin environ de Fougères à Ernée, petite ville où les voyageurs\mbox{}\newline 
\hspace*{6pt}\hspace*{6pt}\hspace*{6pt}\hspace*{6pt}\hspace*{6pt}\hspace*{6pt}\hspace*{6pt}\hspace*{6pt}\hspace*{6pt}\hspace*{6pt} ont coutume de se reposer. {</\textbf{p}>}\mbox{}\newline 
\hspace*{6pt}\hspace*{6pt}\hspace*{6pt}\hspace*{6pt}{</\textbf{div}>}\mbox{}\newline 
\hspace*{6pt}\hspace*{6pt}\hspace*{6pt}{</\textbf{div}>}\mbox{}\newline 
\hspace*{6pt}\hspace*{6pt}{</\textbf{div}>}\mbox{}\newline 
\hspace*{6pt}{</\textbf{div}>}\mbox{}\newline 
{</\textbf{body}>}\end{shaded}\egroup 


    \item[{Schematron}]
   <s:report test="ancestor::tei:l"> Abstract model violation: Lines may not contain higher-level structural elements such as div. </s:report>
    \item[{Schematron}]
   <s:report test="ancestor::tei:p or ancestor::tei:ab and not(ancestor::tei:floatingText)"> Abstract model violation: p and ab may not contain higher-level structural elements such as div. </s:report>
    \item[{Modèle de contenu}]
  \mbox{}\hfill\\[-10pt]\begin{Verbatim}[fontsize=\small]
<content>
 <sequence maxOccurs="1" minOccurs="1">
  <alternate maxOccurs="unbounded"
   minOccurs="0">
   <classRef key="model.divTop"/>
   <classRef key="model.global"/>
  </alternate>
  <sequence maxOccurs="1" minOccurs="0">
   <alternate maxOccurs="1" minOccurs="1">
    <sequence maxOccurs="unbounded"
     minOccurs="1">
     <alternate maxOccurs="1" minOccurs="1">
      <classRef key="model.divLike"/>
      <classRef key="model.divGenLike"/>
     </alternate>
     <classRef key="model.global"
      maxOccurs="unbounded" minOccurs="0"/>
    </sequence>
    <sequence maxOccurs="1" minOccurs="1">
     <sequence maxOccurs="unbounded"
      minOccurs="1">
      <classRef key="model.common"/>
      <classRef key="model.global"
       maxOccurs="unbounded" minOccurs="0"/>
     </sequence>
     <sequence maxOccurs="unbounded"
      minOccurs="0">
      <alternate maxOccurs="1"
       minOccurs="1">
       <classRef key="model.divLike"/>
       <classRef key="model.divGenLike"/>
      </alternate>
      <classRef key="model.global"
       maxOccurs="unbounded" minOccurs="0"/>
     </sequence>
    </sequence>
   </alternate>
   <sequence maxOccurs="unbounded"
    minOccurs="0">
    <classRef key="model.divBottom"/>
    <classRef key="model.global"
     maxOccurs="unbounded" minOccurs="0"/>
   </sequence>
  </sequence>
 </sequence>
</content>
    
\end{Verbatim}

    \item[{Schéma Declaration}]
  \mbox{}\hfill\\[-10pt]\begin{Verbatim}[fontsize=\small]
element div
{
   tei_att.global.attributes,
   tei_att.divLike.attributes,
   tei_att.typed.attributes,
   tei_att.declaring.attributes,
   tei_att.written.attributes,
   (
      ( tei_model.divTop | tei_model.global )*,
      (
         (
            ( ( tei_model.divLike | tei_model.divGenLike ), tei_model.global* )+
          | (
               ( tei_model.common, tei_model.global* )+,
               (
                  ( tei_model.divLike | tei_model.divGenLike ),
                  tei_model.global*
               )*
            )
         ),
         ( tei_model.divBottom, tei_model.global* )*
      )?
   )
}
\end{Verbatim}

\end{reflist}  \index{divGen=<divGen>|oddindex}\index{type=@type!<divGen>|oddindex}
\begin{reflist}
\item[]\begin{specHead}{TEI.divGen}{<divGen> }(division de texte générée automatiquement) indique l'emplacement où doit apparaître une division du texte générée automatiquement par une application de traitement de texte. [\xref{http://www.tei-c.org/release/doc/tei-p5-doc/en/html/CO.html\#CONOIX}{3.8.2. Index Entries}]\end{specHead} 
    \item[{Module}]
  core
    \item[{Attributs}]
  Attributs \hyperref[TEI.att.global]{att.global} (\textit{@xml:id}, \textit{@n}, \textit{@xml:lang}, \textit{@xml:base}, \textit{@xml:space})  (\hyperref[TEI.att.global.rendition]{att.global.rendition} (\textit{@rend}, \textit{@style}, \textit{@rendition})) (\hyperref[TEI.att.global.linking]{att.global.linking} (\textit{@corresp}, \textit{@synch}, \textit{@sameAs}, \textit{@copyOf}, \textit{@next}, \textit{@prev}, \textit{@exclude}, \textit{@select})) (\hyperref[TEI.att.global.analytic]{att.global.analytic} (\textit{@ana})) (\hyperref[TEI.att.global.facs]{att.global.facs} (\textit{@facs})) (\hyperref[TEI.att.global.change]{att.global.change} (\textit{@change})) (\hyperref[TEI.att.global.responsibility]{att.global.responsibility} (\textit{@cert}, \textit{@resp})) (\hyperref[TEI.att.global.source]{att.global.source} (\textit{@source})) \hfil\\[-10pt]\begin{sansreflist}
    \item[@type]
  précise le type de section de texte qui apparaîtra par génération automatique (par exemple : index, table des matières, etc.)
\begin{reflist}
    \item[{Statut}]
  Optionel
    \item[{Type de données}]
  \hyperref[TEI.teidata.enumerated]{teidata.enumerated}
    \item[{Exemple de valeurs possibles:}]
  \begin{description}

\item[{index}]un index doit être généré et inséré à cet endroit.
\item[{toc}]une table des matières
\item[{figlist}]une liste des figures
\item[{tablist}]une liste des tableaux
\end{description} 
    \item[{Note}]
  \par
Les valeurs de cet attribut dépendent de l'application utilisée ; celles qui sont données ci-dessus sont utiles dans le processus de production du document XML, mais leur liste n'est en aucun cas exhaustive.
\end{reflist}  
\end{sansreflist}  
    \item[{Membre du}]
  \hyperref[TEI.model.divGenLike]{model.divGenLike} \hyperref[TEI.model.frontPart]{model.frontPart}
    \item[{Contenu dans}]
  
    \item[textstructure: ]
   \hyperref[TEI.back]{back} \hyperref[TEI.body]{body} \hyperref[TEI.div]{div} \hyperref[TEI.front]{front}
    \item[{Peut contenir}]
  
    \item[core: ]
   \hyperref[TEI.head]{head}
    \item[{Note}]
  \par
Cet élément est plutôt utilisé pendant la production ou la manipulation du document TEI, que dans le processus de transcription de documents préexistants ; il permet de spécifier à quel endroit du document les index, tables des matières, etc., devront être générés par programme.
    \item[{Exemple}]
  Une utilisation de cet élément est de permettre au logiciel de traiter des documents afin de générer en sortie un index et de l' insérer à l'endroit approprié. L'exemple ci-dessous suppose que l'attribut {\itshape indexName} sur les éléments \hyperref[TEI.index]{<index>} dans le texte a été employé pour spécifier des entrées d'index pour deux index produits, nommés NAMES and THINGS:\leavevmode\bgroup\exampleFont \begin{shaded}\noindent\mbox{}{<\textbf{back}>}\mbox{}\newline 
\hspace*{6pt}{<\textbf{div1}\hspace*{6pt}{type}="{backmat}">}\mbox{}\newline 
\hspace*{6pt}\hspace*{6pt}{<\textbf{head}>}Bibliographie{</\textbf{head}>}\mbox{}\newline 
\textit{<!-- ... -->}\mbox{}\newline 
\hspace*{6pt}{</\textbf{div1}>}\mbox{}\newline 
\hspace*{6pt}{<\textbf{div1}\hspace*{6pt}{type}="{backmat}">}\mbox{}\newline 
\hspace*{6pt}\hspace*{6pt}{<\textbf{head}>}Indices{</\textbf{head}>}\mbox{}\newline 
\hspace*{6pt}\hspace*{6pt}{<\textbf{divGen}\hspace*{6pt}{n}="{Index Nominum}"\hspace*{6pt}{type}="{NAMES}"/>}\mbox{}\newline 
\hspace*{6pt}\hspace*{6pt}{<\textbf{divGen}\hspace*{6pt}{n}="{Index Rerum}"\hspace*{6pt}{type}="{THINGS}"/>}\mbox{}\newline 
\hspace*{6pt}{</\textbf{div1}>}\mbox{}\newline 
{</\textbf{back}>}\end{shaded}\egroup 


    \item[{Exemple}]
  Un autre usage de \hyperref[TEI.divGen]{<divGen>} est de spécifier l'emplacement d'une table des matières automatiquement produite.\leavevmode\bgroup\exampleFont \begin{shaded}\noindent\mbox{}{<\textbf{front}>}\mbox{}\newline 
\hspace*{6pt}{<\textbf{divGen}\hspace*{6pt}{type}="{toc}"/>}\mbox{}\newline 
\hspace*{6pt}{<\textbf{div}>}\mbox{}\newline 
\hspace*{6pt}\hspace*{6pt}{<\textbf{head}>}Préface{</\textbf{head}>}\mbox{}\newline 
\hspace*{6pt}\hspace*{6pt}{<\textbf{p}>} ... {</\textbf{p}>}\mbox{}\newline 
\hspace*{6pt}{</\textbf{div}>}\mbox{}\newline 
{</\textbf{front}>}\end{shaded}\egroup 


    \item[{Modèle de contenu}]
  \mbox{}\hfill\\[-10pt]\begin{Verbatim}[fontsize=\small]
<content>
 <classRef key="model.headLike"
  maxOccurs="unbounded" minOccurs="0"/>
</content>
    
\end{Verbatim}

    \item[{Schéma Declaration}]
  \mbox{}\hfill\\[-10pt]\begin{Verbatim}[fontsize=\small]
element divGen
{
   tei_att.global.attributes,
   attribute type { text }?,
   tei_model.headLike*
}
\end{Verbatim}

\end{reflist}  \index{docAuthor=<docAuthor>|oddindex}
\begin{reflist}
\item[]\begin{specHead}{TEI.docAuthor}{<docAuthor> }(auteur du document) contient le nom de l’auteur du document tel qu’il est donné sur la page de titre (ce nom est le plus souvent contenu dans une mention de responsabilité) . [\xref{http://www.tei-c.org/release/doc/tei-p5-doc/en/html/DS.html\#DSTITL}{4.6. Title Pages}]\end{specHead} 
    \item[{Module}]
  textstructure
    \item[{Attributs}]
  Attributs \hyperref[TEI.att.global]{att.global} (\textit{@xml:id}, \textit{@n}, \textit{@xml:lang}, \textit{@xml:base}, \textit{@xml:space})  (\hyperref[TEI.att.global.rendition]{att.global.rendition} (\textit{@rend}, \textit{@style}, \textit{@rendition})) (\hyperref[TEI.att.global.linking]{att.global.linking} (\textit{@corresp}, \textit{@synch}, \textit{@sameAs}, \textit{@copyOf}, \textit{@next}, \textit{@prev}, \textit{@exclude}, \textit{@select})) (\hyperref[TEI.att.global.analytic]{att.global.analytic} (\textit{@ana})) (\hyperref[TEI.att.global.facs]{att.global.facs} (\textit{@facs})) (\hyperref[TEI.att.global.change]{att.global.change} (\textit{@change})) (\hyperref[TEI.att.global.responsibility]{att.global.responsibility} (\textit{@cert}, \textit{@resp})) (\hyperref[TEI.att.global.source]{att.global.source} (\textit{@source})) \hyperref[TEI.att.canonical]{att.canonical} (\textit{@key}, \textit{@ref}) 
    \item[{Membre du}]
  \hyperref[TEI.model.divWrapper]{model.divWrapper} \hyperref[TEI.model.pLike.front]{model.pLike.front} \hyperref[TEI.model.titlepagePart]{model.titlepagePart}
    \item[{Contenu dans}]
  
    \item[core: ]
   \hyperref[TEI.lg]{lg} \hyperref[TEI.list]{list}\par 
    \item[figures: ]
   \hyperref[TEI.figure]{figure} \hyperref[TEI.table]{table}\par 
    \item[msdescription: ]
   \hyperref[TEI.msItem]{msItem}\par 
    \item[textstructure: ]
   \hyperref[TEI.back]{back} \hyperref[TEI.body]{body} \hyperref[TEI.div]{div} \hyperref[TEI.front]{front} \hyperref[TEI.group]{group} \hyperref[TEI.titlePage]{titlePage}
    \item[{Peut contenir}]
  
    \item[analysis: ]
   \hyperref[TEI.c]{c} \hyperref[TEI.cl]{cl} \hyperref[TEI.interp]{interp} \hyperref[TEI.interpGrp]{interpGrp} \hyperref[TEI.m]{m} \hyperref[TEI.pc]{pc} \hyperref[TEI.phr]{phr} \hyperref[TEI.s]{s} \hyperref[TEI.span]{span} \hyperref[TEI.spanGrp]{spanGrp} \hyperref[TEI.w]{w}\par 
    \item[core: ]
   \hyperref[TEI.abbr]{abbr} \hyperref[TEI.add]{add} \hyperref[TEI.address]{address} \hyperref[TEI.binaryObject]{binaryObject} \hyperref[TEI.cb]{cb} \hyperref[TEI.choice]{choice} \hyperref[TEI.corr]{corr} \hyperref[TEI.date]{date} \hyperref[TEI.del]{del} \hyperref[TEI.distinct]{distinct} \hyperref[TEI.email]{email} \hyperref[TEI.emph]{emph} \hyperref[TEI.expan]{expan} \hyperref[TEI.foreign]{foreign} \hyperref[TEI.gap]{gap} \hyperref[TEI.gb]{gb} \hyperref[TEI.gloss]{gloss} \hyperref[TEI.graphic]{graphic} \hyperref[TEI.hi]{hi} \hyperref[TEI.index]{index} \hyperref[TEI.lb]{lb} \hyperref[TEI.measure]{measure} \hyperref[TEI.measureGrp]{measureGrp} \hyperref[TEI.media]{media} \hyperref[TEI.mentioned]{mentioned} \hyperref[TEI.milestone]{milestone} \hyperref[TEI.name]{name} \hyperref[TEI.note]{note} \hyperref[TEI.num]{num} \hyperref[TEI.orig]{orig} \hyperref[TEI.pb]{pb} \hyperref[TEI.ptr]{ptr} \hyperref[TEI.ref]{ref} \hyperref[TEI.reg]{reg} \hyperref[TEI.rs]{rs} \hyperref[TEI.sic]{sic} \hyperref[TEI.soCalled]{soCalled} \hyperref[TEI.term]{term} \hyperref[TEI.time]{time} \hyperref[TEI.title]{title} \hyperref[TEI.unclear]{unclear}\par 
    \item[derived-module-tei.istex: ]
   \hyperref[TEI.math]{math} \hyperref[TEI.mrow]{mrow}\par 
    \item[figures: ]
   \hyperref[TEI.figure]{figure} \hyperref[TEI.formula]{formula} \hyperref[TEI.notatedMusic]{notatedMusic}\par 
    \item[header: ]
   \hyperref[TEI.idno]{idno}\par 
    \item[iso-fs: ]
   \hyperref[TEI.fLib]{fLib} \hyperref[TEI.fs]{fs} \hyperref[TEI.fvLib]{fvLib}\par 
    \item[linking: ]
   \hyperref[TEI.alt]{alt} \hyperref[TEI.altGrp]{altGrp} \hyperref[TEI.anchor]{anchor} \hyperref[TEI.join]{join} \hyperref[TEI.joinGrp]{joinGrp} \hyperref[TEI.link]{link} \hyperref[TEI.linkGrp]{linkGrp} \hyperref[TEI.seg]{seg} \hyperref[TEI.timeline]{timeline}\par 
    \item[msdescription: ]
   \hyperref[TEI.catchwords]{catchwords} \hyperref[TEI.depth]{depth} \hyperref[TEI.dim]{dim} \hyperref[TEI.dimensions]{dimensions} \hyperref[TEI.height]{height} \hyperref[TEI.heraldry]{heraldry} \hyperref[TEI.locus]{locus} \hyperref[TEI.locusGrp]{locusGrp} \hyperref[TEI.material]{material} \hyperref[TEI.objectType]{objectType} \hyperref[TEI.origDate]{origDate} \hyperref[TEI.origPlace]{origPlace} \hyperref[TEI.secFol]{secFol} \hyperref[TEI.signatures]{signatures} \hyperref[TEI.source]{source} \hyperref[TEI.stamp]{stamp} \hyperref[TEI.watermark]{watermark} \hyperref[TEI.width]{width}\par 
    \item[namesdates: ]
   \hyperref[TEI.addName]{addName} \hyperref[TEI.affiliation]{affiliation} \hyperref[TEI.country]{country} \hyperref[TEI.forename]{forename} \hyperref[TEI.genName]{genName} \hyperref[TEI.geogName]{geogName} \hyperref[TEI.location]{location} \hyperref[TEI.nameLink]{nameLink} \hyperref[TEI.orgName]{orgName} \hyperref[TEI.persName]{persName} \hyperref[TEI.placeName]{placeName} \hyperref[TEI.region]{region} \hyperref[TEI.roleName]{roleName} \hyperref[TEI.settlement]{settlement} \hyperref[TEI.state]{state} \hyperref[TEI.surname]{surname}\par 
    \item[spoken: ]
   \hyperref[TEI.annotationBlock]{annotationBlock}\par 
    \item[transcr: ]
   \hyperref[TEI.addSpan]{addSpan} \hyperref[TEI.am]{am} \hyperref[TEI.damage]{damage} \hyperref[TEI.damageSpan]{damageSpan} \hyperref[TEI.delSpan]{delSpan} \hyperref[TEI.ex]{ex} \hyperref[TEI.fw]{fw} \hyperref[TEI.handShift]{handShift} \hyperref[TEI.listTranspose]{listTranspose} \hyperref[TEI.metamark]{metamark} \hyperref[TEI.mod]{mod} \hyperref[TEI.redo]{redo} \hyperref[TEI.restore]{restore} \hyperref[TEI.retrace]{retrace} \hyperref[TEI.secl]{secl} \hyperref[TEI.space]{space} \hyperref[TEI.subst]{subst} \hyperref[TEI.substJoin]{substJoin} \hyperref[TEI.supplied]{supplied} \hyperref[TEI.surplus]{surplus} \hyperref[TEI.undo]{undo}\par des données textuelles
    \item[{Note}]
  \par
Le nom de l'auteur d'un document apparaît souvent au sein d'un élément \texttt{<byline>}, mais l'élément \hyperref[TEI.docAuthor]{<docAuthor>} peut être utilisé même si l'élément \texttt{<byline>} n'est pas présent.
    \item[{Exemple}]
  \leavevmode\bgroup\exampleFont \begin{shaded}\noindent\mbox{}{<\textbf{titlePage}>}\mbox{}\newline 
\hspace*{6pt}{<\textbf{docTitle}>}\mbox{}\newline 
\hspace*{6pt}\hspace*{6pt}{<\textbf{titlePart}>}Le quart livre de faicts et dict Heroiques du bon\mbox{}\newline 
\hspace*{6pt}\hspace*{6pt}\hspace*{6pt}\hspace*{6pt} Pantagruel{</\textbf{titlePart}>}\mbox{}\newline 
\hspace*{6pt}{</\textbf{docTitle}>}\mbox{}\newline 
\hspace*{6pt}{<\textbf{byline}>}Composé par {<\textbf{docAuthor}>}M. François Rabelais{</\textbf{docAuthor}>} docteur en Medicine.{</\textbf{byline}>}\mbox{}\newline 
{</\textbf{titlePage}>}\end{shaded}\egroup 


    \item[{Modèle de contenu}]
  \mbox{}\hfill\\[-10pt]\begin{Verbatim}[fontsize=\small]
<content>
 <macroRef key="macro.phraseSeq"/>
</content>
    
\end{Verbatim}

    \item[{Schéma Declaration}]
  \mbox{}\hfill\\[-10pt]\begin{Verbatim}[fontsize=\small]
element docAuthor
{
   tei_att.global.attributes,
   tei_att.canonical.attributes,
   tei_macro.phraseSeq}
\end{Verbatim}

\end{reflist}  \index{docDate=<docDate>|oddindex}\index{when=@when!<docDate>|oddindex}
\begin{reflist}
\item[]\begin{specHead}{TEI.docDate}{<docDate> }(date du document) contient la date d’un document telle qu’elle est (généralement ) donnée sur une page de titre. [\xref{http://www.tei-c.org/release/doc/tei-p5-doc/en/html/DS.html\#DSTITL}{4.6. Title Pages}]\end{specHead} 
    \item[{Module}]
  textstructure
    \item[{Attributs}]
  Attributs \hyperref[TEI.att.global]{att.global} (\textit{@xml:id}, \textit{@n}, \textit{@xml:lang}, \textit{@xml:base}, \textit{@xml:space})  (\hyperref[TEI.att.global.rendition]{att.global.rendition} (\textit{@rend}, \textit{@style}, \textit{@rendition})) (\hyperref[TEI.att.global.linking]{att.global.linking} (\textit{@corresp}, \textit{@synch}, \textit{@sameAs}, \textit{@copyOf}, \textit{@next}, \textit{@prev}, \textit{@exclude}, \textit{@select})) (\hyperref[TEI.att.global.analytic]{att.global.analytic} (\textit{@ana})) (\hyperref[TEI.att.global.facs]{att.global.facs} (\textit{@facs})) (\hyperref[TEI.att.global.change]{att.global.change} (\textit{@change})) (\hyperref[TEI.att.global.responsibility]{att.global.responsibility} (\textit{@cert}, \textit{@resp})) (\hyperref[TEI.att.global.source]{att.global.source} (\textit{@source})) \hfil\\[-10pt]\begin{sansreflist}
    \item[@when]
  donne la date dans une forme standard, c'est-à-dire. YYYY-MM-DD.
\begin{reflist}
    \item[{Statut}]
  Optionel
    \item[{Type de données}]
  \hyperref[TEI.teidata.temporal.w3c]{teidata.temporal.w3c}
    \item[{Note}]
  \par
Pour les dates dont la forme est simple, il est recommandé que l'attribut {\itshape when} donne la date dans le calendrier grégorien ou grégorien proleptique au format AAAA-MM-JJ spécifié par le standard \textit{XML Schema Part 2}.
\end{reflist}  
\end{sansreflist}  
    \item[{Membre du}]
  \hyperref[TEI.model.divWrapper]{model.divWrapper} \hyperref[TEI.model.pLike.front]{model.pLike.front} \hyperref[TEI.model.titlepagePart]{model.titlepagePart}
    \item[{Contenu dans}]
  
    \item[core: ]
   \hyperref[TEI.lg]{lg} \hyperref[TEI.list]{list}\par 
    \item[figures: ]
   \hyperref[TEI.figure]{figure} \hyperref[TEI.table]{table}\par 
    \item[msdescription: ]
   \hyperref[TEI.msItem]{msItem}\par 
    \item[textstructure: ]
   \hyperref[TEI.back]{back} \hyperref[TEI.body]{body} \hyperref[TEI.div]{div} \hyperref[TEI.front]{front} \hyperref[TEI.group]{group} \hyperref[TEI.titlePage]{titlePage}
    \item[{Peut contenir}]
  
    \item[analysis: ]
   \hyperref[TEI.c]{c} \hyperref[TEI.cl]{cl} \hyperref[TEI.interp]{interp} \hyperref[TEI.interpGrp]{interpGrp} \hyperref[TEI.m]{m} \hyperref[TEI.pc]{pc} \hyperref[TEI.phr]{phr} \hyperref[TEI.s]{s} \hyperref[TEI.span]{span} \hyperref[TEI.spanGrp]{spanGrp} \hyperref[TEI.w]{w}\par 
    \item[core: ]
   \hyperref[TEI.abbr]{abbr} \hyperref[TEI.add]{add} \hyperref[TEI.address]{address} \hyperref[TEI.binaryObject]{binaryObject} \hyperref[TEI.cb]{cb} \hyperref[TEI.choice]{choice} \hyperref[TEI.corr]{corr} \hyperref[TEI.date]{date} \hyperref[TEI.del]{del} \hyperref[TEI.distinct]{distinct} \hyperref[TEI.email]{email} \hyperref[TEI.emph]{emph} \hyperref[TEI.expan]{expan} \hyperref[TEI.foreign]{foreign} \hyperref[TEI.gap]{gap} \hyperref[TEI.gb]{gb} \hyperref[TEI.gloss]{gloss} \hyperref[TEI.graphic]{graphic} \hyperref[TEI.hi]{hi} \hyperref[TEI.index]{index} \hyperref[TEI.lb]{lb} \hyperref[TEI.measure]{measure} \hyperref[TEI.measureGrp]{measureGrp} \hyperref[TEI.media]{media} \hyperref[TEI.mentioned]{mentioned} \hyperref[TEI.milestone]{milestone} \hyperref[TEI.name]{name} \hyperref[TEI.note]{note} \hyperref[TEI.num]{num} \hyperref[TEI.orig]{orig} \hyperref[TEI.pb]{pb} \hyperref[TEI.ptr]{ptr} \hyperref[TEI.ref]{ref} \hyperref[TEI.reg]{reg} \hyperref[TEI.rs]{rs} \hyperref[TEI.sic]{sic} \hyperref[TEI.soCalled]{soCalled} \hyperref[TEI.term]{term} \hyperref[TEI.time]{time} \hyperref[TEI.title]{title} \hyperref[TEI.unclear]{unclear}\par 
    \item[derived-module-tei.istex: ]
   \hyperref[TEI.math]{math} \hyperref[TEI.mrow]{mrow}\par 
    \item[figures: ]
   \hyperref[TEI.figure]{figure} \hyperref[TEI.formula]{formula} \hyperref[TEI.notatedMusic]{notatedMusic}\par 
    \item[header: ]
   \hyperref[TEI.idno]{idno}\par 
    \item[iso-fs: ]
   \hyperref[TEI.fLib]{fLib} \hyperref[TEI.fs]{fs} \hyperref[TEI.fvLib]{fvLib}\par 
    \item[linking: ]
   \hyperref[TEI.alt]{alt} \hyperref[TEI.altGrp]{altGrp} \hyperref[TEI.anchor]{anchor} \hyperref[TEI.join]{join} \hyperref[TEI.joinGrp]{joinGrp} \hyperref[TEI.link]{link} \hyperref[TEI.linkGrp]{linkGrp} \hyperref[TEI.seg]{seg} \hyperref[TEI.timeline]{timeline}\par 
    \item[msdescription: ]
   \hyperref[TEI.catchwords]{catchwords} \hyperref[TEI.depth]{depth} \hyperref[TEI.dim]{dim} \hyperref[TEI.dimensions]{dimensions} \hyperref[TEI.height]{height} \hyperref[TEI.heraldry]{heraldry} \hyperref[TEI.locus]{locus} \hyperref[TEI.locusGrp]{locusGrp} \hyperref[TEI.material]{material} \hyperref[TEI.objectType]{objectType} \hyperref[TEI.origDate]{origDate} \hyperref[TEI.origPlace]{origPlace} \hyperref[TEI.secFol]{secFol} \hyperref[TEI.signatures]{signatures} \hyperref[TEI.source]{source} \hyperref[TEI.stamp]{stamp} \hyperref[TEI.watermark]{watermark} \hyperref[TEI.width]{width}\par 
    \item[namesdates: ]
   \hyperref[TEI.addName]{addName} \hyperref[TEI.affiliation]{affiliation} \hyperref[TEI.country]{country} \hyperref[TEI.forename]{forename} \hyperref[TEI.genName]{genName} \hyperref[TEI.geogName]{geogName} \hyperref[TEI.location]{location} \hyperref[TEI.nameLink]{nameLink} \hyperref[TEI.orgName]{orgName} \hyperref[TEI.persName]{persName} \hyperref[TEI.placeName]{placeName} \hyperref[TEI.region]{region} \hyperref[TEI.roleName]{roleName} \hyperref[TEI.settlement]{settlement} \hyperref[TEI.state]{state} \hyperref[TEI.surname]{surname}\par 
    \item[spoken: ]
   \hyperref[TEI.annotationBlock]{annotationBlock}\par 
    \item[transcr: ]
   \hyperref[TEI.addSpan]{addSpan} \hyperref[TEI.am]{am} \hyperref[TEI.damage]{damage} \hyperref[TEI.damageSpan]{damageSpan} \hyperref[TEI.delSpan]{delSpan} \hyperref[TEI.ex]{ex} \hyperref[TEI.fw]{fw} \hyperref[TEI.handShift]{handShift} \hyperref[TEI.listTranspose]{listTranspose} \hyperref[TEI.metamark]{metamark} \hyperref[TEI.mod]{mod} \hyperref[TEI.redo]{redo} \hyperref[TEI.restore]{restore} \hyperref[TEI.retrace]{retrace} \hyperref[TEI.secl]{secl} \hyperref[TEI.space]{space} \hyperref[TEI.subst]{subst} \hyperref[TEI.substJoin]{substJoin} \hyperref[TEI.supplied]{supplied} \hyperref[TEI.surplus]{surplus} \hyperref[TEI.undo]{undo}\par des données textuelles
    \item[{Note}]
  \par
Voir l'élément générique \hyperref[TEI.date]{<date>} dans le module \textit{core}. L'élément spécifique \hyperref[TEI.docDate]{<docDate>} est fourni à toutes fins utiles pour encoder et traiter la date des documents, puisque celle-ci requiert une gestion particulière pour de nombreux besoins.
    \item[{Exemple}]
  \leavevmode\bgroup\exampleFont \begin{shaded}\noindent\mbox{}{<\textbf{docImprint}>}Lettres Modernes Minard, {<\textbf{pubPlace}>}PARIS-CAEN{</\textbf{pubPlace}>}\mbox{}\newline 
\hspace*{6pt}{<\textbf{docDate}>}2003{</\textbf{docDate}>}\mbox{}\newline 
{</\textbf{docImprint}>}\end{shaded}\egroup 


    \item[{Modèle de contenu}]
  \mbox{}\hfill\\[-10pt]\begin{Verbatim}[fontsize=\small]
<content>
 <macroRef key="macro.phraseSeq"/>
</content>
    
\end{Verbatim}

    \item[{Schéma Declaration}]
  \mbox{}\hfill\\[-10pt]\begin{Verbatim}[fontsize=\small]
element docDate
{
   tei_att.global.attributes,
   attribute when { text }?,
   tei_macro.phraseSeq}
\end{Verbatim}

\end{reflist}  \index{docEdition=<docEdition>|oddindex}
\begin{reflist}
\item[]\begin{specHead}{TEI.docEdition}{<docEdition> }(édition du document) contient une mention d’édition telle qu’elle figure sur la page de titre d’un document. [\xref{http://www.tei-c.org/release/doc/tei-p5-doc/en/html/DS.html\#DSTITL}{4.6. Title Pages}]\end{specHead} 
    \item[{Module}]
  textstructure
    \item[{Attributs}]
  Attributs \hyperref[TEI.att.global]{att.global} (\textit{@xml:id}, \textit{@n}, \textit{@xml:lang}, \textit{@xml:base}, \textit{@xml:space})  (\hyperref[TEI.att.global.rendition]{att.global.rendition} (\textit{@rend}, \textit{@style}, \textit{@rendition})) (\hyperref[TEI.att.global.linking]{att.global.linking} (\textit{@corresp}, \textit{@synch}, \textit{@sameAs}, \textit{@copyOf}, \textit{@next}, \textit{@prev}, \textit{@exclude}, \textit{@select})) (\hyperref[TEI.att.global.analytic]{att.global.analytic} (\textit{@ana})) (\hyperref[TEI.att.global.facs]{att.global.facs} (\textit{@facs})) (\hyperref[TEI.att.global.change]{att.global.change} (\textit{@change})) (\hyperref[TEI.att.global.responsibility]{att.global.responsibility} (\textit{@cert}, \textit{@resp})) (\hyperref[TEI.att.global.source]{att.global.source} (\textit{@source}))
    \item[{Membre du}]
  \hyperref[TEI.model.pLike.front]{model.pLike.front} \hyperref[TEI.model.titlepagePart]{model.titlepagePart}
    \item[{Contenu dans}]
  
    \item[msdescription: ]
   \hyperref[TEI.msItem]{msItem}\par 
    \item[textstructure: ]
   \hyperref[TEI.back]{back} \hyperref[TEI.front]{front} \hyperref[TEI.titlePage]{titlePage}
    \item[{Peut contenir}]
  
    \item[analysis: ]
   \hyperref[TEI.c]{c} \hyperref[TEI.cl]{cl} \hyperref[TEI.interp]{interp} \hyperref[TEI.interpGrp]{interpGrp} \hyperref[TEI.m]{m} \hyperref[TEI.pc]{pc} \hyperref[TEI.phr]{phr} \hyperref[TEI.s]{s} \hyperref[TEI.span]{span} \hyperref[TEI.spanGrp]{spanGrp} \hyperref[TEI.w]{w}\par 
    \item[core: ]
   \hyperref[TEI.abbr]{abbr} \hyperref[TEI.add]{add} \hyperref[TEI.address]{address} \hyperref[TEI.bibl]{bibl} \hyperref[TEI.biblStruct]{biblStruct} \hyperref[TEI.binaryObject]{binaryObject} \hyperref[TEI.cb]{cb} \hyperref[TEI.choice]{choice} \hyperref[TEI.cit]{cit} \hyperref[TEI.corr]{corr} \hyperref[TEI.date]{date} \hyperref[TEI.del]{del} \hyperref[TEI.desc]{desc} \hyperref[TEI.distinct]{distinct} \hyperref[TEI.email]{email} \hyperref[TEI.emph]{emph} \hyperref[TEI.expan]{expan} \hyperref[TEI.foreign]{foreign} \hyperref[TEI.gap]{gap} \hyperref[TEI.gb]{gb} \hyperref[TEI.gloss]{gloss} \hyperref[TEI.graphic]{graphic} \hyperref[TEI.hi]{hi} \hyperref[TEI.index]{index} \hyperref[TEI.l]{l} \hyperref[TEI.label]{label} \hyperref[TEI.lb]{lb} \hyperref[TEI.lg]{lg} \hyperref[TEI.list]{list} \hyperref[TEI.listBibl]{listBibl} \hyperref[TEI.measure]{measure} \hyperref[TEI.measureGrp]{measureGrp} \hyperref[TEI.media]{media} \hyperref[TEI.mentioned]{mentioned} \hyperref[TEI.milestone]{milestone} \hyperref[TEI.name]{name} \hyperref[TEI.note]{note} \hyperref[TEI.num]{num} \hyperref[TEI.orig]{orig} \hyperref[TEI.pb]{pb} \hyperref[TEI.ptr]{ptr} \hyperref[TEI.q]{q} \hyperref[TEI.quote]{quote} \hyperref[TEI.ref]{ref} \hyperref[TEI.reg]{reg} \hyperref[TEI.rs]{rs} \hyperref[TEI.said]{said} \hyperref[TEI.sic]{sic} \hyperref[TEI.soCalled]{soCalled} \hyperref[TEI.stage]{stage} \hyperref[TEI.term]{term} \hyperref[TEI.time]{time} \hyperref[TEI.title]{title} \hyperref[TEI.unclear]{unclear}\par 
    \item[derived-module-tei.istex: ]
   \hyperref[TEI.math]{math} \hyperref[TEI.mrow]{mrow}\par 
    \item[figures: ]
   \hyperref[TEI.figure]{figure} \hyperref[TEI.formula]{formula} \hyperref[TEI.notatedMusic]{notatedMusic} \hyperref[TEI.table]{table}\par 
    \item[header: ]
   \hyperref[TEI.biblFull]{biblFull} \hyperref[TEI.idno]{idno}\par 
    \item[iso-fs: ]
   \hyperref[TEI.fLib]{fLib} \hyperref[TEI.fs]{fs} \hyperref[TEI.fvLib]{fvLib}\par 
    \item[linking: ]
   \hyperref[TEI.alt]{alt} \hyperref[TEI.altGrp]{altGrp} \hyperref[TEI.anchor]{anchor} \hyperref[TEI.join]{join} \hyperref[TEI.joinGrp]{joinGrp} \hyperref[TEI.link]{link} \hyperref[TEI.linkGrp]{linkGrp} \hyperref[TEI.seg]{seg} \hyperref[TEI.timeline]{timeline}\par 
    \item[msdescription: ]
   \hyperref[TEI.catchwords]{catchwords} \hyperref[TEI.depth]{depth} \hyperref[TEI.dim]{dim} \hyperref[TEI.dimensions]{dimensions} \hyperref[TEI.height]{height} \hyperref[TEI.heraldry]{heraldry} \hyperref[TEI.locus]{locus} \hyperref[TEI.locusGrp]{locusGrp} \hyperref[TEI.material]{material} \hyperref[TEI.msDesc]{msDesc} \hyperref[TEI.objectType]{objectType} \hyperref[TEI.origDate]{origDate} \hyperref[TEI.origPlace]{origPlace} \hyperref[TEI.secFol]{secFol} \hyperref[TEI.signatures]{signatures} \hyperref[TEI.source]{source} \hyperref[TEI.stamp]{stamp} \hyperref[TEI.watermark]{watermark} \hyperref[TEI.width]{width}\par 
    \item[namesdates: ]
   \hyperref[TEI.addName]{addName} \hyperref[TEI.affiliation]{affiliation} \hyperref[TEI.country]{country} \hyperref[TEI.forename]{forename} \hyperref[TEI.genName]{genName} \hyperref[TEI.geogName]{geogName} \hyperref[TEI.listOrg]{listOrg} \hyperref[TEI.listPlace]{listPlace} \hyperref[TEI.location]{location} \hyperref[TEI.nameLink]{nameLink} \hyperref[TEI.orgName]{orgName} \hyperref[TEI.persName]{persName} \hyperref[TEI.placeName]{placeName} \hyperref[TEI.region]{region} \hyperref[TEI.roleName]{roleName} \hyperref[TEI.settlement]{settlement} \hyperref[TEI.state]{state} \hyperref[TEI.surname]{surname}\par 
    \item[spoken: ]
   \hyperref[TEI.annotationBlock]{annotationBlock}\par 
    \item[textstructure: ]
   \hyperref[TEI.floatingText]{floatingText}\par 
    \item[transcr: ]
   \hyperref[TEI.addSpan]{addSpan} \hyperref[TEI.am]{am} \hyperref[TEI.damage]{damage} \hyperref[TEI.damageSpan]{damageSpan} \hyperref[TEI.delSpan]{delSpan} \hyperref[TEI.ex]{ex} \hyperref[TEI.fw]{fw} \hyperref[TEI.handShift]{handShift} \hyperref[TEI.listTranspose]{listTranspose} \hyperref[TEI.metamark]{metamark} \hyperref[TEI.mod]{mod} \hyperref[TEI.redo]{redo} \hyperref[TEI.restore]{restore} \hyperref[TEI.retrace]{retrace} \hyperref[TEI.secl]{secl} \hyperref[TEI.space]{space} \hyperref[TEI.subst]{subst} \hyperref[TEI.substJoin]{substJoin} \hyperref[TEI.supplied]{supplied} \hyperref[TEI.surplus]{surplus} \hyperref[TEI.undo]{undo}\par des données textuelles
    \item[{Note}]
  \par
Voir l'élément \hyperref[TEI.edition]{<edition>} dans une citation bibliographique. Comme d'habitude, un nom abrégé a été donné à l'élément le plus fréquent.
    \item[{Exemple}]
  \leavevmode\bgroup\exampleFont \begin{shaded}\noindent\mbox{}{<\textbf{docEdition}>}3e Edition Augmentée{</\textbf{docEdition}>}\end{shaded}\egroup 


    \item[{Modèle de contenu}]
  \mbox{}\hfill\\[-10pt]\begin{Verbatim}[fontsize=\small]
<content>
 <macroRef key="macro.paraContent"/>
</content>
    
\end{Verbatim}

    \item[{Schéma Declaration}]
  \mbox{}\hfill\\[-10pt]\begin{Verbatim}[fontsize=\small]
element docEdition { tei_att.global.attributes, tei_macro.paraContent }
\end{Verbatim}

\end{reflist}  \index{docTitle=<docTitle>|oddindex}
\begin{reflist}
\item[]\begin{specHead}{TEI.docTitle}{<docTitle> }(titre du document) contient le titre d’un document, incluant la totalité de ses composants tels qu’ils sont donnés sur la page de titre. [\xref{http://www.tei-c.org/release/doc/tei-p5-doc/en/html/DS.html\#DSTITL}{4.6. Title Pages}]\end{specHead} 
    \item[{Module}]
  textstructure
    \item[{Attributs}]
  Attributs \hyperref[TEI.att.global]{att.global} (\textit{@xml:id}, \textit{@n}, \textit{@xml:lang}, \textit{@xml:base}, \textit{@xml:space})  (\hyperref[TEI.att.global.rendition]{att.global.rendition} (\textit{@rend}, \textit{@style}, \textit{@rendition})) (\hyperref[TEI.att.global.linking]{att.global.linking} (\textit{@corresp}, \textit{@synch}, \textit{@sameAs}, \textit{@copyOf}, \textit{@next}, \textit{@prev}, \textit{@exclude}, \textit{@select})) (\hyperref[TEI.att.global.analytic]{att.global.analytic} (\textit{@ana})) (\hyperref[TEI.att.global.facs]{att.global.facs} (\textit{@facs})) (\hyperref[TEI.att.global.change]{att.global.change} (\textit{@change})) (\hyperref[TEI.att.global.responsibility]{att.global.responsibility} (\textit{@cert}, \textit{@resp})) (\hyperref[TEI.att.global.source]{att.global.source} (\textit{@source})) \hyperref[TEI.att.canonical]{att.canonical} (\textit{@key}, \textit{@ref}) 
    \item[{Membre du}]
  \hyperref[TEI.model.pLike.front]{model.pLike.front} \hyperref[TEI.model.titlepagePart]{model.titlepagePart}
    \item[{Contenu dans}]
  
    \item[msdescription: ]
   \hyperref[TEI.msItem]{msItem}\par 
    \item[textstructure: ]
   \hyperref[TEI.back]{back} \hyperref[TEI.front]{front} \hyperref[TEI.titlePage]{titlePage}
    \item[{Peut contenir}]
  
    \item[analysis: ]
   \hyperref[TEI.interp]{interp} \hyperref[TEI.interpGrp]{interpGrp} \hyperref[TEI.span]{span} \hyperref[TEI.spanGrp]{spanGrp}\par 
    \item[core: ]
   \hyperref[TEI.cb]{cb} \hyperref[TEI.gap]{gap} \hyperref[TEI.gb]{gb} \hyperref[TEI.index]{index} \hyperref[TEI.lb]{lb} \hyperref[TEI.milestone]{milestone} \hyperref[TEI.note]{note} \hyperref[TEI.pb]{pb}\par 
    \item[figures: ]
   \hyperref[TEI.figure]{figure} \hyperref[TEI.notatedMusic]{notatedMusic}\par 
    \item[iso-fs: ]
   \hyperref[TEI.fLib]{fLib} \hyperref[TEI.fs]{fs} \hyperref[TEI.fvLib]{fvLib}\par 
    \item[linking: ]
   \hyperref[TEI.alt]{alt} \hyperref[TEI.altGrp]{altGrp} \hyperref[TEI.anchor]{anchor} \hyperref[TEI.join]{join} \hyperref[TEI.joinGrp]{joinGrp} \hyperref[TEI.link]{link} \hyperref[TEI.linkGrp]{linkGrp} \hyperref[TEI.timeline]{timeline}\par 
    \item[msdescription: ]
   \hyperref[TEI.source]{source}\par 
    \item[textstructure: ]
   \hyperref[TEI.titlePart]{titlePart}\par 
    \item[transcr: ]
   \hyperref[TEI.addSpan]{addSpan} \hyperref[TEI.damageSpan]{damageSpan} \hyperref[TEI.delSpan]{delSpan} \hyperref[TEI.fw]{fw} \hyperref[TEI.listTranspose]{listTranspose} \hyperref[TEI.metamark]{metamark} \hyperref[TEI.space]{space} \hyperref[TEI.substJoin]{substJoin}
    \item[{Exemple}]
  \leavevmode\bgroup\exampleFont \begin{shaded}\noindent\mbox{}{<\textbf{docTitle}>}\mbox{}\newline 
\hspace*{6pt}{<\textbf{titlePart}\hspace*{6pt}{type}="{main}">}LES CHOSES{</\textbf{titlePart}>}\mbox{}\newline 
\hspace*{6pt}{<\textbf{titlePart}\hspace*{6pt}{type}="{sub}">}Une histoire des années soixante.{</\textbf{titlePart}>}\mbox{}\newline 
{</\textbf{docTitle}>}\end{shaded}\egroup 


    \item[{Modèle de contenu}]
  \mbox{}\hfill\\[-10pt]\begin{Verbatim}[fontsize=\small]
<content>
 <sequence maxOccurs="1" minOccurs="1">
  <classRef key="model.global"
   maxOccurs="unbounded" minOccurs="0"/>
  <sequence maxOccurs="unbounded"
   minOccurs="1">
   <elementRef key="titlePart"/>
   <classRef key="model.global"
    maxOccurs="unbounded" minOccurs="0"/>
  </sequence>
 </sequence>
</content>
    
\end{Verbatim}

    \item[{Schéma Declaration}]
  \mbox{}\hfill\\[-10pt]\begin{Verbatim}[fontsize=\small]
element docTitle
{
   tei_att.global.attributes,
   tei_att.canonical.attributes,
   ( tei_model.global*, ( tei_titlePart, tei_model.global* )+ )
}
\end{Verbatim}

\end{reflist}  \index{edition=<edition>|oddindex}
\begin{reflist}
\item[]\begin{specHead}{TEI.edition}{<edition> }(édition) décrit les particularités de l’édition d’un texte. [\xref{http://www.tei-c.org/release/doc/tei-p5-doc/en/html/HD.html\#HD22}{2.2.2. The Edition Statement}]\end{specHead} 
    \item[{Module}]
  header
    \item[{Attributs}]
  Attributs \hyperref[TEI.att.global]{att.global} (\textit{@xml:id}, \textit{@n}, \textit{@xml:lang}, \textit{@xml:base}, \textit{@xml:space})  (\hyperref[TEI.att.global.rendition]{att.global.rendition} (\textit{@rend}, \textit{@style}, \textit{@rendition})) (\hyperref[TEI.att.global.linking]{att.global.linking} (\textit{@corresp}, \textit{@synch}, \textit{@sameAs}, \textit{@copyOf}, \textit{@next}, \textit{@prev}, \textit{@exclude}, \textit{@select})) (\hyperref[TEI.att.global.analytic]{att.global.analytic} (\textit{@ana})) (\hyperref[TEI.att.global.facs]{att.global.facs} (\textit{@facs})) (\hyperref[TEI.att.global.change]{att.global.change} (\textit{@change})) (\hyperref[TEI.att.global.responsibility]{att.global.responsibility} (\textit{@cert}, \textit{@resp})) (\hyperref[TEI.att.global.source]{att.global.source} (\textit{@source}))
    \item[{Membre du}]
  \hyperref[TEI.model.biblPart]{model.biblPart} 
    \item[{Contenu dans}]
  
    \item[core: ]
   \hyperref[TEI.bibl]{bibl} \hyperref[TEI.monogr]{monogr}\par 
    \item[header: ]
   \hyperref[TEI.editionStmt]{editionStmt}
    \item[{Peut contenir}]
  
    \item[analysis: ]
   \hyperref[TEI.c]{c} \hyperref[TEI.cl]{cl} \hyperref[TEI.interp]{interp} \hyperref[TEI.interpGrp]{interpGrp} \hyperref[TEI.m]{m} \hyperref[TEI.pc]{pc} \hyperref[TEI.phr]{phr} \hyperref[TEI.s]{s} \hyperref[TEI.span]{span} \hyperref[TEI.spanGrp]{spanGrp} \hyperref[TEI.w]{w}\par 
    \item[core: ]
   \hyperref[TEI.abbr]{abbr} \hyperref[TEI.add]{add} \hyperref[TEI.address]{address} \hyperref[TEI.binaryObject]{binaryObject} \hyperref[TEI.cb]{cb} \hyperref[TEI.choice]{choice} \hyperref[TEI.corr]{corr} \hyperref[TEI.date]{date} \hyperref[TEI.del]{del} \hyperref[TEI.distinct]{distinct} \hyperref[TEI.email]{email} \hyperref[TEI.emph]{emph} \hyperref[TEI.expan]{expan} \hyperref[TEI.foreign]{foreign} \hyperref[TEI.gap]{gap} \hyperref[TEI.gb]{gb} \hyperref[TEI.gloss]{gloss} \hyperref[TEI.graphic]{graphic} \hyperref[TEI.hi]{hi} \hyperref[TEI.index]{index} \hyperref[TEI.lb]{lb} \hyperref[TEI.measure]{measure} \hyperref[TEI.measureGrp]{measureGrp} \hyperref[TEI.media]{media} \hyperref[TEI.mentioned]{mentioned} \hyperref[TEI.milestone]{milestone} \hyperref[TEI.name]{name} \hyperref[TEI.note]{note} \hyperref[TEI.num]{num} \hyperref[TEI.orig]{orig} \hyperref[TEI.pb]{pb} \hyperref[TEI.ptr]{ptr} \hyperref[TEI.ref]{ref} \hyperref[TEI.reg]{reg} \hyperref[TEI.rs]{rs} \hyperref[TEI.sic]{sic} \hyperref[TEI.soCalled]{soCalled} \hyperref[TEI.term]{term} \hyperref[TEI.time]{time} \hyperref[TEI.title]{title} \hyperref[TEI.unclear]{unclear}\par 
    \item[derived-module-tei.istex: ]
   \hyperref[TEI.math]{math} \hyperref[TEI.mrow]{mrow}\par 
    \item[figures: ]
   \hyperref[TEI.figure]{figure} \hyperref[TEI.formula]{formula} \hyperref[TEI.notatedMusic]{notatedMusic}\par 
    \item[header: ]
   \hyperref[TEI.idno]{idno}\par 
    \item[iso-fs: ]
   \hyperref[TEI.fLib]{fLib} \hyperref[TEI.fs]{fs} \hyperref[TEI.fvLib]{fvLib}\par 
    \item[linking: ]
   \hyperref[TEI.alt]{alt} \hyperref[TEI.altGrp]{altGrp} \hyperref[TEI.anchor]{anchor} \hyperref[TEI.join]{join} \hyperref[TEI.joinGrp]{joinGrp} \hyperref[TEI.link]{link} \hyperref[TEI.linkGrp]{linkGrp} \hyperref[TEI.seg]{seg} \hyperref[TEI.timeline]{timeline}\par 
    \item[msdescription: ]
   \hyperref[TEI.catchwords]{catchwords} \hyperref[TEI.depth]{depth} \hyperref[TEI.dim]{dim} \hyperref[TEI.dimensions]{dimensions} \hyperref[TEI.height]{height} \hyperref[TEI.heraldry]{heraldry} \hyperref[TEI.locus]{locus} \hyperref[TEI.locusGrp]{locusGrp} \hyperref[TEI.material]{material} \hyperref[TEI.objectType]{objectType} \hyperref[TEI.origDate]{origDate} \hyperref[TEI.origPlace]{origPlace} \hyperref[TEI.secFol]{secFol} \hyperref[TEI.signatures]{signatures} \hyperref[TEI.source]{source} \hyperref[TEI.stamp]{stamp} \hyperref[TEI.watermark]{watermark} \hyperref[TEI.width]{width}\par 
    \item[namesdates: ]
   \hyperref[TEI.addName]{addName} \hyperref[TEI.affiliation]{affiliation} \hyperref[TEI.country]{country} \hyperref[TEI.forename]{forename} \hyperref[TEI.genName]{genName} \hyperref[TEI.geogName]{geogName} \hyperref[TEI.location]{location} \hyperref[TEI.nameLink]{nameLink} \hyperref[TEI.orgName]{orgName} \hyperref[TEI.persName]{persName} \hyperref[TEI.placeName]{placeName} \hyperref[TEI.region]{region} \hyperref[TEI.roleName]{roleName} \hyperref[TEI.settlement]{settlement} \hyperref[TEI.state]{state} \hyperref[TEI.surname]{surname}\par 
    \item[spoken: ]
   \hyperref[TEI.annotationBlock]{annotationBlock}\par 
    \item[transcr: ]
   \hyperref[TEI.addSpan]{addSpan} \hyperref[TEI.am]{am} \hyperref[TEI.damage]{damage} \hyperref[TEI.damageSpan]{damageSpan} \hyperref[TEI.delSpan]{delSpan} \hyperref[TEI.ex]{ex} \hyperref[TEI.fw]{fw} \hyperref[TEI.handShift]{handShift} \hyperref[TEI.listTranspose]{listTranspose} \hyperref[TEI.metamark]{metamark} \hyperref[TEI.mod]{mod} \hyperref[TEI.redo]{redo} \hyperref[TEI.restore]{restore} \hyperref[TEI.retrace]{retrace} \hyperref[TEI.secl]{secl} \hyperref[TEI.space]{space} \hyperref[TEI.subst]{subst} \hyperref[TEI.substJoin]{substJoin} \hyperref[TEI.supplied]{supplied} \hyperref[TEI.surplus]{surplus} \hyperref[TEI.undo]{undo}\par des données textuelles
    \item[{Exemple}]
  \leavevmode\bgroup\exampleFont \begin{shaded}\noindent\mbox{}{<\textbf{edition}>}Première édition électronique, Nancy {<\textbf{date}>}2002{</\textbf{date}>}\mbox{}\newline 
{</\textbf{edition}>}\end{shaded}\egroup 


    \item[{Modèle de contenu}]
  \mbox{}\hfill\\[-10pt]\begin{Verbatim}[fontsize=\small]
<content>
 <macroRef key="macro.phraseSeq"/>
</content>
    
\end{Verbatim}

    \item[{Schéma Declaration}]
  \mbox{}\hfill\\[-10pt]\begin{Verbatim}[fontsize=\small]
element edition { tei_att.global.attributes, tei_macro.phraseSeq }
\end{Verbatim}

\end{reflist}  \index{editionStmt=<editionStmt>|oddindex}
\begin{reflist}
\item[]\begin{specHead}{TEI.editionStmt}{<editionStmt> }(mention d'édition) regroupe les informations relatives à l’édition d’un texte. [\xref{http://www.tei-c.org/release/doc/tei-p5-doc/en/html/HD.html\#HD22}{2.2.2. The Edition Statement} \xref{http://www.tei-c.org/release/doc/tei-p5-doc/en/html/HD.html\#HD2}{2.2. The File Description}]\end{specHead} 
    \item[{Module}]
  header
    \item[{Attributs}]
  Attributs \hyperref[TEI.att.global]{att.global} (\textit{@xml:id}, \textit{@n}, \textit{@xml:lang}, \textit{@xml:base}, \textit{@xml:space})  (\hyperref[TEI.att.global.rendition]{att.global.rendition} (\textit{@rend}, \textit{@style}, \textit{@rendition})) (\hyperref[TEI.att.global.linking]{att.global.linking} (\textit{@corresp}, \textit{@synch}, \textit{@sameAs}, \textit{@copyOf}, \textit{@next}, \textit{@prev}, \textit{@exclude}, \textit{@select})) (\hyperref[TEI.att.global.analytic]{att.global.analytic} (\textit{@ana})) (\hyperref[TEI.att.global.facs]{att.global.facs} (\textit{@facs})) (\hyperref[TEI.att.global.change]{att.global.change} (\textit{@change})) (\hyperref[TEI.att.global.responsibility]{att.global.responsibility} (\textit{@cert}, \textit{@resp})) (\hyperref[TEI.att.global.source]{att.global.source} (\textit{@source}))
    \item[{Contenu dans}]
  
    \item[header: ]
   \hyperref[TEI.biblFull]{biblFull} \hyperref[TEI.fileDesc]{fileDesc}
    \item[{Peut contenir}]
  
    \item[core: ]
   \hyperref[TEI.author]{author} \hyperref[TEI.editor]{editor} \hyperref[TEI.meeting]{meeting} \hyperref[TEI.p]{p} \hyperref[TEI.respStmt]{respStmt}\par 
    \item[header: ]
   \hyperref[TEI.edition]{edition} \hyperref[TEI.funder]{funder}\par 
    \item[linking: ]
   \hyperref[TEI.ab]{ab}
    \item[{Exemple}]
  \leavevmode\bgroup\exampleFont \begin{shaded}\noindent\mbox{}{<\textbf{editionStmt}>}\mbox{}\newline 
\hspace*{6pt}{<\textbf{edition}>}Deuxième édition{</\textbf{edition}>}\mbox{}\newline 
\hspace*{6pt}{<\textbf{respStmt}>}\mbox{}\newline 
\hspace*{6pt}\hspace*{6pt}{<\textbf{resp}>}réalisée par{</\textbf{resp}>}\mbox{}\newline 
\hspace*{6pt}\hspace*{6pt}{<\textbf{name}>}L. F.{</\textbf{name}>}\mbox{}\newline 
\hspace*{6pt}{</\textbf{respStmt}>}\mbox{}\newline 
{</\textbf{editionStmt}>}\end{shaded}\egroup 


    \item[{Exemple}]
  \leavevmode\bgroup\exampleFont \begin{shaded}\noindent\mbox{}{<\textbf{editionStmt}>}\mbox{}\newline 
\hspace*{6pt}{<\textbf{p}>}Première édition électronique, Nancy, {<\textbf{date}>} 2002{</\textbf{date}>}, réalisée dans le cadre de la\mbox{}\newline 
\hspace*{6pt}\hspace*{6pt} base {<\textbf{ref}\hspace*{6pt}{target}="{http://www.frantext.fr/}">}FRANTEXT{</\textbf{ref}>} .{</\textbf{p}>}\mbox{}\newline 
{</\textbf{editionStmt}>}\end{shaded}\egroup 


    \item[{Modèle de contenu}]
  \mbox{}\hfill\\[-10pt]\begin{Verbatim}[fontsize=\small]
<content>
 <alternate maxOccurs="1" minOccurs="1">
  <classRef key="model.pLike"
   maxOccurs="unbounded" minOccurs="1"/>
  <sequence maxOccurs="1" minOccurs="1">
   <elementRef key="edition"/>
   <classRef key="model.respLike"
    maxOccurs="unbounded" minOccurs="0"/>
  </sequence>
 </alternate>
</content>
    
\end{Verbatim}

    \item[{Schéma Declaration}]
  \mbox{}\hfill\\[-10pt]\begin{Verbatim}[fontsize=\small]
element editionStmt
{
   tei_att.global.attributes,
   ( tei_model.pLike+ | ( tei_edition, tei_model.respLike* ) )
}
\end{Verbatim}

\end{reflist}  \index{editor=<editor>|oddindex}
\begin{reflist}
\item[]\begin{specHead}{TEI.editor}{<editor> }mention de responsabilité secondaire pour un item bibliographique, par exemple le nom d'une personne, d'une institution ou d'un organisme (ou de plusieurs d'entre eux) comme éditeur scientifique, compilateur, traducteur, etc. [\xref{http://www.tei-c.org/release/doc/tei-p5-doc/en/html/CO.html\#COBICOR}{3.11.2.2. Titles, Authors, and Editors}]\end{specHead} 
    \item[{Module}]
  core
    \item[{Attributs}]
  Attributs \hyperref[TEI.att.global]{att.global} (\textit{@xml:id}, \textit{@n}, \textit{@xml:lang}, \textit{@xml:base}, \textit{@xml:space})  (\hyperref[TEI.att.global.rendition]{att.global.rendition} (\textit{@rend}, \textit{@style}, \textit{@rendition})) (\hyperref[TEI.att.global.linking]{att.global.linking} (\textit{@corresp}, \textit{@synch}, \textit{@sameAs}, \textit{@copyOf}, \textit{@next}, \textit{@prev}, \textit{@exclude}, \textit{@select})) (\hyperref[TEI.att.global.analytic]{att.global.analytic} (\textit{@ana})) (\hyperref[TEI.att.global.facs]{att.global.facs} (\textit{@facs})) (\hyperref[TEI.att.global.change]{att.global.change} (\textit{@change})) (\hyperref[TEI.att.global.responsibility]{att.global.responsibility} (\textit{@cert}, \textit{@resp})) (\hyperref[TEI.att.global.source]{att.global.source} (\textit{@source})) \hyperref[TEI.att.naming]{att.naming} (\textit{@role}, \textit{@nymRef})  (\hyperref[TEI.att.canonical]{att.canonical} (\textit{@key}, \textit{@ref}))
    \item[{Membre du}]
  \hyperref[TEI.model.respLike]{model.respLike} 
    \item[{Contenu dans}]
  
    \item[core: ]
   \hyperref[TEI.analytic]{analytic} \hyperref[TEI.bibl]{bibl} \hyperref[TEI.monogr]{monogr} \hyperref[TEI.series]{series}\par 
    \item[header: ]
   \hyperref[TEI.editionStmt]{editionStmt} \hyperref[TEI.seriesStmt]{seriesStmt} \hyperref[TEI.titleStmt]{titleStmt}\par 
    \item[msdescription: ]
   \hyperref[TEI.msItem]{msItem}
    \item[{Peut contenir}]
  
    \item[analysis: ]
   \hyperref[TEI.c]{c} \hyperref[TEI.cl]{cl} \hyperref[TEI.interp]{interp} \hyperref[TEI.interpGrp]{interpGrp} \hyperref[TEI.m]{m} \hyperref[TEI.pc]{pc} \hyperref[TEI.phr]{phr} \hyperref[TEI.s]{s} \hyperref[TEI.span]{span} \hyperref[TEI.spanGrp]{spanGrp} \hyperref[TEI.w]{w}\par 
    \item[core: ]
   \hyperref[TEI.abbr]{abbr} \hyperref[TEI.add]{add} \hyperref[TEI.address]{address} \hyperref[TEI.binaryObject]{binaryObject} \hyperref[TEI.cb]{cb} \hyperref[TEI.choice]{choice} \hyperref[TEI.corr]{corr} \hyperref[TEI.date]{date} \hyperref[TEI.del]{del} \hyperref[TEI.distinct]{distinct} \hyperref[TEI.email]{email} \hyperref[TEI.emph]{emph} \hyperref[TEI.expan]{expan} \hyperref[TEI.foreign]{foreign} \hyperref[TEI.gap]{gap} \hyperref[TEI.gb]{gb} \hyperref[TEI.gloss]{gloss} \hyperref[TEI.graphic]{graphic} \hyperref[TEI.hi]{hi} \hyperref[TEI.index]{index} \hyperref[TEI.lb]{lb} \hyperref[TEI.measure]{measure} \hyperref[TEI.measureGrp]{measureGrp} \hyperref[TEI.media]{media} \hyperref[TEI.mentioned]{mentioned} \hyperref[TEI.milestone]{milestone} \hyperref[TEI.name]{name} \hyperref[TEI.note]{note} \hyperref[TEI.num]{num} \hyperref[TEI.orig]{orig} \hyperref[TEI.pb]{pb} \hyperref[TEI.ptr]{ptr} \hyperref[TEI.ref]{ref} \hyperref[TEI.reg]{reg} \hyperref[TEI.rs]{rs} \hyperref[TEI.sic]{sic} \hyperref[TEI.soCalled]{soCalled} \hyperref[TEI.term]{term} \hyperref[TEI.time]{time} \hyperref[TEI.title]{title} \hyperref[TEI.unclear]{unclear}\par 
    \item[derived-module-tei.istex: ]
   \hyperref[TEI.math]{math} \hyperref[TEI.mrow]{mrow}\par 
    \item[figures: ]
   \hyperref[TEI.figure]{figure} \hyperref[TEI.formula]{formula} \hyperref[TEI.notatedMusic]{notatedMusic}\par 
    \item[header: ]
   \hyperref[TEI.idno]{idno}\par 
    \item[iso-fs: ]
   \hyperref[TEI.fLib]{fLib} \hyperref[TEI.fs]{fs} \hyperref[TEI.fvLib]{fvLib}\par 
    \item[linking: ]
   \hyperref[TEI.alt]{alt} \hyperref[TEI.altGrp]{altGrp} \hyperref[TEI.anchor]{anchor} \hyperref[TEI.join]{join} \hyperref[TEI.joinGrp]{joinGrp} \hyperref[TEI.link]{link} \hyperref[TEI.linkGrp]{linkGrp} \hyperref[TEI.seg]{seg} \hyperref[TEI.timeline]{timeline}\par 
    \item[msdescription: ]
   \hyperref[TEI.catchwords]{catchwords} \hyperref[TEI.depth]{depth} \hyperref[TEI.dim]{dim} \hyperref[TEI.dimensions]{dimensions} \hyperref[TEI.height]{height} \hyperref[TEI.heraldry]{heraldry} \hyperref[TEI.locus]{locus} \hyperref[TEI.locusGrp]{locusGrp} \hyperref[TEI.material]{material} \hyperref[TEI.objectType]{objectType} \hyperref[TEI.origDate]{origDate} \hyperref[TEI.origPlace]{origPlace} \hyperref[TEI.secFol]{secFol} \hyperref[TEI.signatures]{signatures} \hyperref[TEI.source]{source} \hyperref[TEI.stamp]{stamp} \hyperref[TEI.watermark]{watermark} \hyperref[TEI.width]{width}\par 
    \item[namesdates: ]
   \hyperref[TEI.addName]{addName} \hyperref[TEI.affiliation]{affiliation} \hyperref[TEI.country]{country} \hyperref[TEI.forename]{forename} \hyperref[TEI.genName]{genName} \hyperref[TEI.geogName]{geogName} \hyperref[TEI.location]{location} \hyperref[TEI.nameLink]{nameLink} \hyperref[TEI.orgName]{orgName} \hyperref[TEI.persName]{persName} \hyperref[TEI.placeName]{placeName} \hyperref[TEI.region]{region} \hyperref[TEI.roleName]{roleName} \hyperref[TEI.settlement]{settlement} \hyperref[TEI.state]{state} \hyperref[TEI.surname]{surname}\par 
    \item[spoken: ]
   \hyperref[TEI.annotationBlock]{annotationBlock}\par 
    \item[transcr: ]
   \hyperref[TEI.addSpan]{addSpan} \hyperref[TEI.am]{am} \hyperref[TEI.damage]{damage} \hyperref[TEI.damageSpan]{damageSpan} \hyperref[TEI.delSpan]{delSpan} \hyperref[TEI.ex]{ex} \hyperref[TEI.fw]{fw} \hyperref[TEI.handShift]{handShift} \hyperref[TEI.listTranspose]{listTranspose} \hyperref[TEI.metamark]{metamark} \hyperref[TEI.mod]{mod} \hyperref[TEI.redo]{redo} \hyperref[TEI.restore]{restore} \hyperref[TEI.retrace]{retrace} \hyperref[TEI.secl]{secl} \hyperref[TEI.space]{space} \hyperref[TEI.subst]{subst} \hyperref[TEI.substJoin]{substJoin} \hyperref[TEI.supplied]{supplied} \hyperref[TEI.surplus]{surplus} \hyperref[TEI.undo]{undo}\par des données textuelles
    \item[{Note}]
  \par
Il est conseillé d'adopter un format cohérent.\par
Particulièrement lorsque le catalogage repose sur le contenu de l'en-tête TEI, il est conseillé d'utiliser des listes d'autorité reconnues pour trouver la forme exacte des noms de personnes.
    \item[{Exemple}]
  \leavevmode\bgroup\exampleFont \begin{shaded}\noindent\mbox{}{<\textbf{editor}>} Pierre-Jules Hetzel{</\textbf{editor}>}\mbox{}\newline 
{<\textbf{editor}\hspace*{6pt}{role}="{illustrator}">}George Roux{</\textbf{editor}>}\end{shaded}\egroup 


    \item[{Modèle de contenu}]
  \mbox{}\hfill\\[-10pt]\begin{Verbatim}[fontsize=\small]
<content>
 <macroRef key="macro.phraseSeq"/>
</content>
    
\end{Verbatim}

    \item[{Schéma Declaration}]
  \mbox{}\hfill\\[-10pt]\begin{Verbatim}[fontsize=\small]
element editor
{
   tei_att.global.attributes,
   tei_att.naming.attributes,
   tei_macro.phraseSeq}
\end{Verbatim}

\end{reflist}  \index{email=<email>|oddindex}
\begin{reflist}
\item[]\begin{specHead}{TEI.email}{<email> }(adresse de courrier électronique) contient l'adresse de courriel identifiant un emplacement où un courriel peut être envoyé. [\xref{http://www.tei-c.org/release/doc/tei-p5-doc/en/html/CO.html\#CONAAD}{3.5.2. Addresses}]\end{specHead} 
    \item[{Module}]
  core
    \item[{Attributs}]
  Attributs \hyperref[TEI.att.global]{att.global} (\textit{@xml:id}, \textit{@n}, \textit{@xml:lang}, \textit{@xml:base}, \textit{@xml:space})  (\hyperref[TEI.att.global.rendition]{att.global.rendition} (\textit{@rend}, \textit{@style}, \textit{@rendition})) (\hyperref[TEI.att.global.linking]{att.global.linking} (\textit{@corresp}, \textit{@synch}, \textit{@sameAs}, \textit{@copyOf}, \textit{@next}, \textit{@prev}, \textit{@exclude}, \textit{@select})) (\hyperref[TEI.att.global.analytic]{att.global.analytic} (\textit{@ana})) (\hyperref[TEI.att.global.facs]{att.global.facs} (\textit{@facs})) (\hyperref[TEI.att.global.change]{att.global.change} (\textit{@change})) (\hyperref[TEI.att.global.responsibility]{att.global.responsibility} (\textit{@cert}, \textit{@resp})) (\hyperref[TEI.att.global.source]{att.global.source} (\textit{@source}))
    \item[{Membre du}]
  \hyperref[TEI.model.addressLike]{model.addressLike}
    \item[{Contenu dans}]
  
    \item[analysis: ]
   \hyperref[TEI.cl]{cl} \hyperref[TEI.phr]{phr} \hyperref[TEI.s]{s} \hyperref[TEI.span]{span}\par 
    \item[core: ]
   \hyperref[TEI.abbr]{abbr} \hyperref[TEI.add]{add} \hyperref[TEI.addrLine]{addrLine} \hyperref[TEI.author]{author} \hyperref[TEI.bibl]{bibl} \hyperref[TEI.biblScope]{biblScope} \hyperref[TEI.citedRange]{citedRange} \hyperref[TEI.corr]{corr} \hyperref[TEI.date]{date} \hyperref[TEI.del]{del} \hyperref[TEI.desc]{desc} \hyperref[TEI.distinct]{distinct} \hyperref[TEI.editor]{editor} \hyperref[TEI.email]{email} \hyperref[TEI.emph]{emph} \hyperref[TEI.expan]{expan} \hyperref[TEI.foreign]{foreign} \hyperref[TEI.gloss]{gloss} \hyperref[TEI.head]{head} \hyperref[TEI.headItem]{headItem} \hyperref[TEI.headLabel]{headLabel} \hyperref[TEI.hi]{hi} \hyperref[TEI.item]{item} \hyperref[TEI.l]{l} \hyperref[TEI.label]{label} \hyperref[TEI.measure]{measure} \hyperref[TEI.meeting]{meeting} \hyperref[TEI.mentioned]{mentioned} \hyperref[TEI.name]{name} \hyperref[TEI.note]{note} \hyperref[TEI.num]{num} \hyperref[TEI.orig]{orig} \hyperref[TEI.p]{p} \hyperref[TEI.pubPlace]{pubPlace} \hyperref[TEI.publisher]{publisher} \hyperref[TEI.q]{q} \hyperref[TEI.quote]{quote} \hyperref[TEI.ref]{ref} \hyperref[TEI.reg]{reg} \hyperref[TEI.resp]{resp} \hyperref[TEI.rs]{rs} \hyperref[TEI.said]{said} \hyperref[TEI.sic]{sic} \hyperref[TEI.soCalled]{soCalled} \hyperref[TEI.speaker]{speaker} \hyperref[TEI.stage]{stage} \hyperref[TEI.street]{street} \hyperref[TEI.term]{term} \hyperref[TEI.textLang]{textLang} \hyperref[TEI.time]{time} \hyperref[TEI.title]{title} \hyperref[TEI.unclear]{unclear}\par 
    \item[figures: ]
   \hyperref[TEI.cell]{cell} \hyperref[TEI.figDesc]{figDesc}\par 
    \item[header: ]
   \hyperref[TEI.authority]{authority} \hyperref[TEI.change]{change} \hyperref[TEI.classCode]{classCode} \hyperref[TEI.creation]{creation} \hyperref[TEI.distributor]{distributor} \hyperref[TEI.edition]{edition} \hyperref[TEI.extent]{extent} \hyperref[TEI.funder]{funder} \hyperref[TEI.language]{language} \hyperref[TEI.licence]{licence} \hyperref[TEI.rendition]{rendition}\par 
    \item[iso-fs: ]
   \hyperref[TEI.fDescr]{fDescr} \hyperref[TEI.fsDescr]{fsDescr}\par 
    \item[linking: ]
   \hyperref[TEI.ab]{ab} \hyperref[TEI.seg]{seg}\par 
    \item[msdescription: ]
   \hyperref[TEI.accMat]{accMat} \hyperref[TEI.acquisition]{acquisition} \hyperref[TEI.additions]{additions} \hyperref[TEI.catchwords]{catchwords} \hyperref[TEI.collation]{collation} \hyperref[TEI.colophon]{colophon} \hyperref[TEI.condition]{condition} \hyperref[TEI.custEvent]{custEvent} \hyperref[TEI.decoNote]{decoNote} \hyperref[TEI.explicit]{explicit} \hyperref[TEI.filiation]{filiation} \hyperref[TEI.finalRubric]{finalRubric} \hyperref[TEI.foliation]{foliation} \hyperref[TEI.heraldry]{heraldry} \hyperref[TEI.incipit]{incipit} \hyperref[TEI.layout]{layout} \hyperref[TEI.material]{material} \hyperref[TEI.musicNotation]{musicNotation} \hyperref[TEI.objectType]{objectType} \hyperref[TEI.origDate]{origDate} \hyperref[TEI.origPlace]{origPlace} \hyperref[TEI.origin]{origin} \hyperref[TEI.provenance]{provenance} \hyperref[TEI.rubric]{rubric} \hyperref[TEI.secFol]{secFol} \hyperref[TEI.signatures]{signatures} \hyperref[TEI.source]{source} \hyperref[TEI.stamp]{stamp} \hyperref[TEI.summary]{summary} \hyperref[TEI.support]{support} \hyperref[TEI.surrogates]{surrogates} \hyperref[TEI.typeNote]{typeNote} \hyperref[TEI.watermark]{watermark}\par 
    \item[namesdates: ]
   \hyperref[TEI.addName]{addName} \hyperref[TEI.affiliation]{affiliation} \hyperref[TEI.country]{country} \hyperref[TEI.forename]{forename} \hyperref[TEI.genName]{genName} \hyperref[TEI.geogName]{geogName} \hyperref[TEI.location]{location} \hyperref[TEI.nameLink]{nameLink} \hyperref[TEI.orgName]{orgName} \hyperref[TEI.persName]{persName} \hyperref[TEI.placeName]{placeName} \hyperref[TEI.region]{region} \hyperref[TEI.roleName]{roleName} \hyperref[TEI.settlement]{settlement} \hyperref[TEI.surname]{surname}\par 
    \item[textstructure: ]
   \hyperref[TEI.docAuthor]{docAuthor} \hyperref[TEI.docDate]{docDate} \hyperref[TEI.docEdition]{docEdition} \hyperref[TEI.titlePart]{titlePart}\par 
    \item[transcr: ]
   \hyperref[TEI.damage]{damage} \hyperref[TEI.fw]{fw} \hyperref[TEI.metamark]{metamark} \hyperref[TEI.mod]{mod} \hyperref[TEI.restore]{restore} \hyperref[TEI.retrace]{retrace} \hyperref[TEI.secl]{secl} \hyperref[TEI.supplied]{supplied} \hyperref[TEI.surplus]{surplus}
    \item[{Peut contenir}]
  
    \item[analysis: ]
   \hyperref[TEI.c]{c} \hyperref[TEI.cl]{cl} \hyperref[TEI.interp]{interp} \hyperref[TEI.interpGrp]{interpGrp} \hyperref[TEI.m]{m} \hyperref[TEI.pc]{pc} \hyperref[TEI.phr]{phr} \hyperref[TEI.s]{s} \hyperref[TEI.span]{span} \hyperref[TEI.spanGrp]{spanGrp} \hyperref[TEI.w]{w}\par 
    \item[core: ]
   \hyperref[TEI.abbr]{abbr} \hyperref[TEI.add]{add} \hyperref[TEI.address]{address} \hyperref[TEI.binaryObject]{binaryObject} \hyperref[TEI.cb]{cb} \hyperref[TEI.choice]{choice} \hyperref[TEI.corr]{corr} \hyperref[TEI.date]{date} \hyperref[TEI.del]{del} \hyperref[TEI.distinct]{distinct} \hyperref[TEI.email]{email} \hyperref[TEI.emph]{emph} \hyperref[TEI.expan]{expan} \hyperref[TEI.foreign]{foreign} \hyperref[TEI.gap]{gap} \hyperref[TEI.gb]{gb} \hyperref[TEI.gloss]{gloss} \hyperref[TEI.graphic]{graphic} \hyperref[TEI.hi]{hi} \hyperref[TEI.index]{index} \hyperref[TEI.lb]{lb} \hyperref[TEI.measure]{measure} \hyperref[TEI.measureGrp]{measureGrp} \hyperref[TEI.media]{media} \hyperref[TEI.mentioned]{mentioned} \hyperref[TEI.milestone]{milestone} \hyperref[TEI.name]{name} \hyperref[TEI.note]{note} \hyperref[TEI.num]{num} \hyperref[TEI.orig]{orig} \hyperref[TEI.pb]{pb} \hyperref[TEI.ptr]{ptr} \hyperref[TEI.ref]{ref} \hyperref[TEI.reg]{reg} \hyperref[TEI.rs]{rs} \hyperref[TEI.sic]{sic} \hyperref[TEI.soCalled]{soCalled} \hyperref[TEI.term]{term} \hyperref[TEI.time]{time} \hyperref[TEI.title]{title} \hyperref[TEI.unclear]{unclear}\par 
    \item[derived-module-tei.istex: ]
   \hyperref[TEI.math]{math} \hyperref[TEI.mrow]{mrow}\par 
    \item[figures: ]
   \hyperref[TEI.figure]{figure} \hyperref[TEI.formula]{formula} \hyperref[TEI.notatedMusic]{notatedMusic}\par 
    \item[header: ]
   \hyperref[TEI.idno]{idno}\par 
    \item[iso-fs: ]
   \hyperref[TEI.fLib]{fLib} \hyperref[TEI.fs]{fs} \hyperref[TEI.fvLib]{fvLib}\par 
    \item[linking: ]
   \hyperref[TEI.alt]{alt} \hyperref[TEI.altGrp]{altGrp} \hyperref[TEI.anchor]{anchor} \hyperref[TEI.join]{join} \hyperref[TEI.joinGrp]{joinGrp} \hyperref[TEI.link]{link} \hyperref[TEI.linkGrp]{linkGrp} \hyperref[TEI.seg]{seg} \hyperref[TEI.timeline]{timeline}\par 
    \item[msdescription: ]
   \hyperref[TEI.catchwords]{catchwords} \hyperref[TEI.depth]{depth} \hyperref[TEI.dim]{dim} \hyperref[TEI.dimensions]{dimensions} \hyperref[TEI.height]{height} \hyperref[TEI.heraldry]{heraldry} \hyperref[TEI.locus]{locus} \hyperref[TEI.locusGrp]{locusGrp} \hyperref[TEI.material]{material} \hyperref[TEI.objectType]{objectType} \hyperref[TEI.origDate]{origDate} \hyperref[TEI.origPlace]{origPlace} \hyperref[TEI.secFol]{secFol} \hyperref[TEI.signatures]{signatures} \hyperref[TEI.source]{source} \hyperref[TEI.stamp]{stamp} \hyperref[TEI.watermark]{watermark} \hyperref[TEI.width]{width}\par 
    \item[namesdates: ]
   \hyperref[TEI.addName]{addName} \hyperref[TEI.affiliation]{affiliation} \hyperref[TEI.country]{country} \hyperref[TEI.forename]{forename} \hyperref[TEI.genName]{genName} \hyperref[TEI.geogName]{geogName} \hyperref[TEI.location]{location} \hyperref[TEI.nameLink]{nameLink} \hyperref[TEI.orgName]{orgName} \hyperref[TEI.persName]{persName} \hyperref[TEI.placeName]{placeName} \hyperref[TEI.region]{region} \hyperref[TEI.roleName]{roleName} \hyperref[TEI.settlement]{settlement} \hyperref[TEI.state]{state} \hyperref[TEI.surname]{surname}\par 
    \item[spoken: ]
   \hyperref[TEI.annotationBlock]{annotationBlock}\par 
    \item[transcr: ]
   \hyperref[TEI.addSpan]{addSpan} \hyperref[TEI.am]{am} \hyperref[TEI.damage]{damage} \hyperref[TEI.damageSpan]{damageSpan} \hyperref[TEI.delSpan]{delSpan} \hyperref[TEI.ex]{ex} \hyperref[TEI.fw]{fw} \hyperref[TEI.handShift]{handShift} \hyperref[TEI.listTranspose]{listTranspose} \hyperref[TEI.metamark]{metamark} \hyperref[TEI.mod]{mod} \hyperref[TEI.redo]{redo} \hyperref[TEI.restore]{restore} \hyperref[TEI.retrace]{retrace} \hyperref[TEI.secl]{secl} \hyperref[TEI.space]{space} \hyperref[TEI.subst]{subst} \hyperref[TEI.substJoin]{substJoin} \hyperref[TEI.supplied]{supplied} \hyperref[TEI.surplus]{surplus} \hyperref[TEI.undo]{undo}\par des données textuelles
    \item[{Note}]
  \par
Le format d'une adresse de courrier électronique internet moderne est défini dans la \xref{https://tools.ietf.org/html/rfc2822}{RFC 2822}
    \item[{Exemple}]
  \leavevmode\bgroup\exampleFont \begin{shaded}\noindent\mbox{}{<\textbf{email}>}membership@tei-c.org{</\textbf{email}>}\end{shaded}\egroup 


    \item[{Exemple}]
  \leavevmode\bgroup\exampleFont \begin{shaded}\noindent\mbox{}{<\textbf{email}>}membership@tei-c.org{</\textbf{email}>}\end{shaded}\egroup 


    \item[{Modèle de contenu}]
  \mbox{}\hfill\\[-10pt]\begin{Verbatim}[fontsize=\small]
<content>
 <macroRef key="macro.phraseSeq"/>
</content>
    
\end{Verbatim}

    \item[{Schéma Declaration}]
  \mbox{}\hfill\\[-10pt]\begin{Verbatim}[fontsize=\small]
element email { tei_att.global.attributes, tei_macro.phraseSeq }
\end{Verbatim}

\end{reflist}  \index{emph=<emph>|oddindex}
\begin{reflist}
\item[]\begin{specHead}{TEI.emph}{<emph> }(mis en valeur) marque des mots ou des expressions qui sont accentués ou mis en valeur pour un motif linguistique ou rhétorique. [\xref{http://www.tei-c.org/release/doc/tei-p5-doc/en/html/CO.html\#COHQHE}{3.3.2.2. Emphatic Words and Phrases} \xref{http://www.tei-c.org/release/doc/tei-p5-doc/en/html/CO.html\#COHQH}{3.3.2. Emphasis, Foreign Words, and Unusual Language}]\end{specHead} 
    \item[{Module}]
  core
    \item[{Attributs}]
  Attributs \hyperref[TEI.att.global]{att.global} (\textit{@xml:id}, \textit{@n}, \textit{@xml:lang}, \textit{@xml:base}, \textit{@xml:space})  (\hyperref[TEI.att.global.rendition]{att.global.rendition} (\textit{@rend}, \textit{@style}, \textit{@rendition})) (\hyperref[TEI.att.global.linking]{att.global.linking} (\textit{@corresp}, \textit{@synch}, \textit{@sameAs}, \textit{@copyOf}, \textit{@next}, \textit{@prev}, \textit{@exclude}, \textit{@select})) (\hyperref[TEI.att.global.analytic]{att.global.analytic} (\textit{@ana})) (\hyperref[TEI.att.global.facs]{att.global.facs} (\textit{@facs})) (\hyperref[TEI.att.global.change]{att.global.change} (\textit{@change})) (\hyperref[TEI.att.global.responsibility]{att.global.responsibility} (\textit{@cert}, \textit{@resp})) (\hyperref[TEI.att.global.source]{att.global.source} (\textit{@source}))
    \item[{Membre du}]
  \hyperref[TEI.model.emphLike]{model.emphLike}
    \item[{Contenu dans}]
  
    \item[analysis: ]
   \hyperref[TEI.cl]{cl} \hyperref[TEI.phr]{phr} \hyperref[TEI.s]{s} \hyperref[TEI.span]{span}\par 
    \item[core: ]
   \hyperref[TEI.abbr]{abbr} \hyperref[TEI.add]{add} \hyperref[TEI.addrLine]{addrLine} \hyperref[TEI.author]{author} \hyperref[TEI.bibl]{bibl} \hyperref[TEI.biblScope]{biblScope} \hyperref[TEI.citedRange]{citedRange} \hyperref[TEI.corr]{corr} \hyperref[TEI.date]{date} \hyperref[TEI.del]{del} \hyperref[TEI.desc]{desc} \hyperref[TEI.distinct]{distinct} \hyperref[TEI.editor]{editor} \hyperref[TEI.email]{email} \hyperref[TEI.emph]{emph} \hyperref[TEI.expan]{expan} \hyperref[TEI.foreign]{foreign} \hyperref[TEI.gloss]{gloss} \hyperref[TEI.head]{head} \hyperref[TEI.headItem]{headItem} \hyperref[TEI.headLabel]{headLabel} \hyperref[TEI.hi]{hi} \hyperref[TEI.item]{item} \hyperref[TEI.l]{l} \hyperref[TEI.label]{label} \hyperref[TEI.measure]{measure} \hyperref[TEI.meeting]{meeting} \hyperref[TEI.mentioned]{mentioned} \hyperref[TEI.name]{name} \hyperref[TEI.note]{note} \hyperref[TEI.num]{num} \hyperref[TEI.orig]{orig} \hyperref[TEI.p]{p} \hyperref[TEI.pubPlace]{pubPlace} \hyperref[TEI.publisher]{publisher} \hyperref[TEI.q]{q} \hyperref[TEI.quote]{quote} \hyperref[TEI.ref]{ref} \hyperref[TEI.reg]{reg} \hyperref[TEI.resp]{resp} \hyperref[TEI.rs]{rs} \hyperref[TEI.said]{said} \hyperref[TEI.sic]{sic} \hyperref[TEI.soCalled]{soCalled} \hyperref[TEI.speaker]{speaker} \hyperref[TEI.stage]{stage} \hyperref[TEI.street]{street} \hyperref[TEI.term]{term} \hyperref[TEI.textLang]{textLang} \hyperref[TEI.time]{time} \hyperref[TEI.title]{title} \hyperref[TEI.unclear]{unclear}\par 
    \item[figures: ]
   \hyperref[TEI.cell]{cell} \hyperref[TEI.figDesc]{figDesc}\par 
    \item[header: ]
   \hyperref[TEI.authority]{authority} \hyperref[TEI.change]{change} \hyperref[TEI.classCode]{classCode} \hyperref[TEI.creation]{creation} \hyperref[TEI.distributor]{distributor} \hyperref[TEI.edition]{edition} \hyperref[TEI.extent]{extent} \hyperref[TEI.funder]{funder} \hyperref[TEI.language]{language} \hyperref[TEI.licence]{licence} \hyperref[TEI.rendition]{rendition}\par 
    \item[iso-fs: ]
   \hyperref[TEI.fDescr]{fDescr} \hyperref[TEI.fsDescr]{fsDescr}\par 
    \item[linking: ]
   \hyperref[TEI.ab]{ab} \hyperref[TEI.seg]{seg}\par 
    \item[msdescription: ]
   \hyperref[TEI.accMat]{accMat} \hyperref[TEI.acquisition]{acquisition} \hyperref[TEI.additions]{additions} \hyperref[TEI.catchwords]{catchwords} \hyperref[TEI.collation]{collation} \hyperref[TEI.colophon]{colophon} \hyperref[TEI.condition]{condition} \hyperref[TEI.custEvent]{custEvent} \hyperref[TEI.decoNote]{decoNote} \hyperref[TEI.explicit]{explicit} \hyperref[TEI.filiation]{filiation} \hyperref[TEI.finalRubric]{finalRubric} \hyperref[TEI.foliation]{foliation} \hyperref[TEI.heraldry]{heraldry} \hyperref[TEI.incipit]{incipit} \hyperref[TEI.layout]{layout} \hyperref[TEI.material]{material} \hyperref[TEI.musicNotation]{musicNotation} \hyperref[TEI.objectType]{objectType} \hyperref[TEI.origDate]{origDate} \hyperref[TEI.origPlace]{origPlace} \hyperref[TEI.origin]{origin} \hyperref[TEI.provenance]{provenance} \hyperref[TEI.rubric]{rubric} \hyperref[TEI.secFol]{secFol} \hyperref[TEI.signatures]{signatures} \hyperref[TEI.source]{source} \hyperref[TEI.stamp]{stamp} \hyperref[TEI.summary]{summary} \hyperref[TEI.support]{support} \hyperref[TEI.surrogates]{surrogates} \hyperref[TEI.typeNote]{typeNote} \hyperref[TEI.watermark]{watermark}\par 
    \item[namesdates: ]
   \hyperref[TEI.addName]{addName} \hyperref[TEI.affiliation]{affiliation} \hyperref[TEI.country]{country} \hyperref[TEI.forename]{forename} \hyperref[TEI.genName]{genName} \hyperref[TEI.geogName]{geogName} \hyperref[TEI.nameLink]{nameLink} \hyperref[TEI.orgName]{orgName} \hyperref[TEI.persName]{persName} \hyperref[TEI.placeName]{placeName} \hyperref[TEI.region]{region} \hyperref[TEI.roleName]{roleName} \hyperref[TEI.settlement]{settlement} \hyperref[TEI.surname]{surname}\par 
    \item[textstructure: ]
   \hyperref[TEI.docAuthor]{docAuthor} \hyperref[TEI.docDate]{docDate} \hyperref[TEI.docEdition]{docEdition} \hyperref[TEI.titlePart]{titlePart}\par 
    \item[transcr: ]
   \hyperref[TEI.damage]{damage} \hyperref[TEI.fw]{fw} \hyperref[TEI.metamark]{metamark} \hyperref[TEI.mod]{mod} \hyperref[TEI.restore]{restore} \hyperref[TEI.retrace]{retrace} \hyperref[TEI.secl]{secl} \hyperref[TEI.supplied]{supplied} \hyperref[TEI.surplus]{surplus}
    \item[{Peut contenir}]
  
    \item[analysis: ]
   \hyperref[TEI.c]{c} \hyperref[TEI.cl]{cl} \hyperref[TEI.interp]{interp} \hyperref[TEI.interpGrp]{interpGrp} \hyperref[TEI.m]{m} \hyperref[TEI.pc]{pc} \hyperref[TEI.phr]{phr} \hyperref[TEI.s]{s} \hyperref[TEI.span]{span} \hyperref[TEI.spanGrp]{spanGrp} \hyperref[TEI.w]{w}\par 
    \item[core: ]
   \hyperref[TEI.abbr]{abbr} \hyperref[TEI.add]{add} \hyperref[TEI.address]{address} \hyperref[TEI.bibl]{bibl} \hyperref[TEI.biblStruct]{biblStruct} \hyperref[TEI.binaryObject]{binaryObject} \hyperref[TEI.cb]{cb} \hyperref[TEI.choice]{choice} \hyperref[TEI.cit]{cit} \hyperref[TEI.corr]{corr} \hyperref[TEI.date]{date} \hyperref[TEI.del]{del} \hyperref[TEI.desc]{desc} \hyperref[TEI.distinct]{distinct} \hyperref[TEI.email]{email} \hyperref[TEI.emph]{emph} \hyperref[TEI.expan]{expan} \hyperref[TEI.foreign]{foreign} \hyperref[TEI.gap]{gap} \hyperref[TEI.gb]{gb} \hyperref[TEI.gloss]{gloss} \hyperref[TEI.graphic]{graphic} \hyperref[TEI.hi]{hi} \hyperref[TEI.index]{index} \hyperref[TEI.l]{l} \hyperref[TEI.label]{label} \hyperref[TEI.lb]{lb} \hyperref[TEI.lg]{lg} \hyperref[TEI.list]{list} \hyperref[TEI.listBibl]{listBibl} \hyperref[TEI.measure]{measure} \hyperref[TEI.measureGrp]{measureGrp} \hyperref[TEI.media]{media} \hyperref[TEI.mentioned]{mentioned} \hyperref[TEI.milestone]{milestone} \hyperref[TEI.name]{name} \hyperref[TEI.note]{note} \hyperref[TEI.num]{num} \hyperref[TEI.orig]{orig} \hyperref[TEI.pb]{pb} \hyperref[TEI.ptr]{ptr} \hyperref[TEI.q]{q} \hyperref[TEI.quote]{quote} \hyperref[TEI.ref]{ref} \hyperref[TEI.reg]{reg} \hyperref[TEI.rs]{rs} \hyperref[TEI.said]{said} \hyperref[TEI.sic]{sic} \hyperref[TEI.soCalled]{soCalled} \hyperref[TEI.stage]{stage} \hyperref[TEI.term]{term} \hyperref[TEI.time]{time} \hyperref[TEI.title]{title} \hyperref[TEI.unclear]{unclear}\par 
    \item[derived-module-tei.istex: ]
   \hyperref[TEI.math]{math} \hyperref[TEI.mrow]{mrow}\par 
    \item[figures: ]
   \hyperref[TEI.figure]{figure} \hyperref[TEI.formula]{formula} \hyperref[TEI.notatedMusic]{notatedMusic} \hyperref[TEI.table]{table}\par 
    \item[header: ]
   \hyperref[TEI.biblFull]{biblFull} \hyperref[TEI.idno]{idno}\par 
    \item[iso-fs: ]
   \hyperref[TEI.fLib]{fLib} \hyperref[TEI.fs]{fs} \hyperref[TEI.fvLib]{fvLib}\par 
    \item[linking: ]
   \hyperref[TEI.alt]{alt} \hyperref[TEI.altGrp]{altGrp} \hyperref[TEI.anchor]{anchor} \hyperref[TEI.join]{join} \hyperref[TEI.joinGrp]{joinGrp} \hyperref[TEI.link]{link} \hyperref[TEI.linkGrp]{linkGrp} \hyperref[TEI.seg]{seg} \hyperref[TEI.timeline]{timeline}\par 
    \item[msdescription: ]
   \hyperref[TEI.catchwords]{catchwords} \hyperref[TEI.depth]{depth} \hyperref[TEI.dim]{dim} \hyperref[TEI.dimensions]{dimensions} \hyperref[TEI.height]{height} \hyperref[TEI.heraldry]{heraldry} \hyperref[TEI.locus]{locus} \hyperref[TEI.locusGrp]{locusGrp} \hyperref[TEI.material]{material} \hyperref[TEI.msDesc]{msDesc} \hyperref[TEI.objectType]{objectType} \hyperref[TEI.origDate]{origDate} \hyperref[TEI.origPlace]{origPlace} \hyperref[TEI.secFol]{secFol} \hyperref[TEI.signatures]{signatures} \hyperref[TEI.source]{source} \hyperref[TEI.stamp]{stamp} \hyperref[TEI.watermark]{watermark} \hyperref[TEI.width]{width}\par 
    \item[namesdates: ]
   \hyperref[TEI.addName]{addName} \hyperref[TEI.affiliation]{affiliation} \hyperref[TEI.country]{country} \hyperref[TEI.forename]{forename} \hyperref[TEI.genName]{genName} \hyperref[TEI.geogName]{geogName} \hyperref[TEI.listOrg]{listOrg} \hyperref[TEI.listPlace]{listPlace} \hyperref[TEI.location]{location} \hyperref[TEI.nameLink]{nameLink} \hyperref[TEI.orgName]{orgName} \hyperref[TEI.persName]{persName} \hyperref[TEI.placeName]{placeName} \hyperref[TEI.region]{region} \hyperref[TEI.roleName]{roleName} \hyperref[TEI.settlement]{settlement} \hyperref[TEI.state]{state} \hyperref[TEI.surname]{surname}\par 
    \item[spoken: ]
   \hyperref[TEI.annotationBlock]{annotationBlock}\par 
    \item[textstructure: ]
   \hyperref[TEI.floatingText]{floatingText}\par 
    \item[transcr: ]
   \hyperref[TEI.addSpan]{addSpan} \hyperref[TEI.am]{am} \hyperref[TEI.damage]{damage} \hyperref[TEI.damageSpan]{damageSpan} \hyperref[TEI.delSpan]{delSpan} \hyperref[TEI.ex]{ex} \hyperref[TEI.fw]{fw} \hyperref[TEI.handShift]{handShift} \hyperref[TEI.listTranspose]{listTranspose} \hyperref[TEI.metamark]{metamark} \hyperref[TEI.mod]{mod} \hyperref[TEI.redo]{redo} \hyperref[TEI.restore]{restore} \hyperref[TEI.retrace]{retrace} \hyperref[TEI.secl]{secl} \hyperref[TEI.space]{space} \hyperref[TEI.subst]{subst} \hyperref[TEI.substJoin]{substJoin} \hyperref[TEI.supplied]{supplied} \hyperref[TEI.surplus]{surplus} \hyperref[TEI.undo]{undo}\par des données textuelles
    \item[{Exemple}]
  \leavevmode\bgroup\exampleFont \begin{shaded}\noindent\mbox{}{<\textbf{div}>}\mbox{}\newline 
\hspace*{6pt}{<\textbf{p}>}«Mes amis, dit-il, mes amis, je... je... »{</\textbf{p}>}\mbox{}\newline 
\hspace*{6pt}{<\textbf{p}>}Mais quelque chose l'étouffait. Il ne pouvait pas achever sa phrase.{</\textbf{p}>}\mbox{}\newline 
\hspace*{6pt}{<\textbf{p}>} Alors il se tourna vers le tableau, prit un morceau de craie, et, en appuyant de\mbox{}\newline 
\hspace*{6pt}\hspace*{6pt} toutes ses forces, il écrivit aussi gros qu'il put : {</\textbf{p}>}\mbox{}\newline 
\hspace*{6pt}{<\textbf{p}>}\mbox{}\newline 
\hspace*{6pt}\hspace*{6pt}{<\textbf{emph}>}«vive la France !"»{</\textbf{emph}>}\mbox{}\newline 
\hspace*{6pt}{</\textbf{p}>}\mbox{}\newline 
\hspace*{6pt}{<\textbf{p}>} Puis il resta là, la tête appuyée au mur, et, sans parler, avec sa main il nous\mbox{}\newline 
\hspace*{6pt}\hspace*{6pt} faisait signe:{</\textbf{p}>}\mbox{}\newline 
\hspace*{6pt}{<\textbf{p}>}«C'est fini...allez-vous-en.»{</\textbf{p}>}\mbox{}\newline 
{</\textbf{div}>}\end{shaded}\egroup 


    \item[{Exemple}]
  \leavevmode\bgroup\exampleFont \begin{shaded}\noindent\mbox{}{<\textbf{p}>} Tu sais quoi ? On l'aurait proposé pour{<\textbf{emph}>}la médaille{</\textbf{emph}>} ! {</\textbf{p}>}\end{shaded}\egroup 


    \item[{Modèle de contenu}]
  \mbox{}\hfill\\[-10pt]\begin{Verbatim}[fontsize=\small]
<content>
 <macroRef key="macro.paraContent"/>
</content>
    
\end{Verbatim}

    \item[{Schéma Declaration}]
  \mbox{}\hfill\\[-10pt]\begin{Verbatim}[fontsize=\small]
element emph { tei_att.global.attributes, tei_macro.paraContent }
\end{Verbatim}

\end{reflist}  \index{encodingDesc=<encodingDesc>|oddindex}
\begin{reflist}
\item[]\begin{specHead}{TEI.encodingDesc}{<encodingDesc> }(description de l'encodage) documente la relation d'un texte électronique avec sa ou ses sources. [\xref{http://www.tei-c.org/release/doc/tei-p5-doc/en/html/HD.html\#HD5}{2.3. The Encoding Description} \xref{http://www.tei-c.org/release/doc/tei-p5-doc/en/html/HD.html\#HD11}{2.1.1. The TEI Header and Its Components}]\end{specHead} 
    \item[{Module}]
  header
    \item[{Attributs}]
  Attributs \hyperref[TEI.att.global]{att.global} (\textit{@xml:id}, \textit{@n}, \textit{@xml:lang}, \textit{@xml:base}, \textit{@xml:space})  (\hyperref[TEI.att.global.rendition]{att.global.rendition} (\textit{@rend}, \textit{@style}, \textit{@rendition})) (\hyperref[TEI.att.global.linking]{att.global.linking} (\textit{@corresp}, \textit{@synch}, \textit{@sameAs}, \textit{@copyOf}, \textit{@next}, \textit{@prev}, \textit{@exclude}, \textit{@select})) (\hyperref[TEI.att.global.analytic]{att.global.analytic} (\textit{@ana})) (\hyperref[TEI.att.global.facs]{att.global.facs} (\textit{@facs})) (\hyperref[TEI.att.global.change]{att.global.change} (\textit{@change})) (\hyperref[TEI.att.global.responsibility]{att.global.responsibility} (\textit{@cert}, \textit{@resp})) (\hyperref[TEI.att.global.source]{att.global.source} (\textit{@source}))
    \item[{Membre du}]
  \hyperref[TEI.model.teiHeaderPart]{model.teiHeaderPart}
    \item[{Contenu dans}]
  
    \item[header: ]
   \hyperref[TEI.teiHeader]{teiHeader}
    \item[{Peut contenir}]
  
    \item[core: ]
   \hyperref[TEI.p]{p}\par 
    \item[header: ]
   \hyperref[TEI.appInfo]{appInfo} \hyperref[TEI.classDecl]{classDecl} \hyperref[TEI.schemaRef]{schemaRef}\par 
    \item[iso-fs: ]
   \hyperref[TEI.fsdDecl]{fsdDecl}\par 
    \item[linking: ]
   \hyperref[TEI.ab]{ab}
    \item[{Exemple}]
  \leavevmode\bgroup\exampleFont \begin{shaded}\noindent\mbox{}{<\textbf{encodingDesc}>}\mbox{}\newline 
\hspace*{6pt}{<\textbf{projectDesc}>}\mbox{}\newline 
\hspace*{6pt}\hspace*{6pt}{<\textbf{p}>}Corpus de\mbox{}\newline 
\hspace*{6pt}\hspace*{6pt}\hspace*{6pt}\hspace*{6pt} textes sélectionnés pour la formation MISAT, Frejus,\mbox{}\newline 
\hspace*{6pt}\hspace*{6pt}\hspace*{6pt}\hspace*{6pt} juillet 2010.\mbox{}\newline 
\hspace*{6pt}\hspace*{6pt}{</\textbf{p}>}\mbox{}\newline 
\hspace*{6pt}{</\textbf{projectDesc}>}\mbox{}\newline 
\hspace*{6pt}{<\textbf{samplingDecl}>}\mbox{}\newline 
\hspace*{6pt}\hspace*{6pt}{<\textbf{p}>}Corpus\mbox{}\newline 
\hspace*{6pt}\hspace*{6pt}\hspace*{6pt}\hspace*{6pt} d'échantillons de 2000 mots pris au début de chaque texte. {</\textbf{p}>}\mbox{}\newline 
\hspace*{6pt}{</\textbf{samplingDecl}>}\mbox{}\newline 
{</\textbf{encodingDesc}>}\end{shaded}\egroup 


    \item[{Modèle de contenu}]
  \mbox{}\hfill\\[-10pt]\begin{Verbatim}[fontsize=\small]
<content>
 <alternate maxOccurs="unbounded"
  minOccurs="1">
  <classRef key="model.encodingDescPart"/>
  <classRef key="model.pLike"/>
 </alternate>
</content>
    
\end{Verbatim}

    \item[{Schéma Declaration}]
  \mbox{}\hfill\\[-10pt]\begin{Verbatim}[fontsize=\small]
element encodingDesc
{
   tei_att.global.attributes,
   ( tei_model.encodingDescPart | tei_model.pLike )+
}
\end{Verbatim}

\end{reflist}  \index{event=<event>|oddindex}\index{where=@where!<event>|oddindex}
\begin{reflist}
\item[]\begin{specHead}{TEI.event}{<event> }(évènement) contient des données liées à tout type d'évènement significatif dans l'existence d'une personne, d'un lieu ou d'une organisation. [\xref{http://www.tei-c.org/release/doc/tei-p5-doc/en/html/ND.html\#NDPERSbp}{13.3.1. Basic Principles}]\end{specHead} 
    \item[{Module}]
  namesdates
    \item[{Attributs}]
  Attributs \hyperref[TEI.att.global]{att.global} (\textit{@xml:id}, \textit{@n}, \textit{@xml:lang}, \textit{@xml:base}, \textit{@xml:space})  (\hyperref[TEI.att.global.rendition]{att.global.rendition} (\textit{@rend}, \textit{@style}, \textit{@rendition})) (\hyperref[TEI.att.global.linking]{att.global.linking} (\textit{@corresp}, \textit{@synch}, \textit{@sameAs}, \textit{@copyOf}, \textit{@next}, \textit{@prev}, \textit{@exclude}, \textit{@select})) (\hyperref[TEI.att.global.analytic]{att.global.analytic} (\textit{@ana})) (\hyperref[TEI.att.global.facs]{att.global.facs} (\textit{@facs})) (\hyperref[TEI.att.global.change]{att.global.change} (\textit{@change})) (\hyperref[TEI.att.global.responsibility]{att.global.responsibility} (\textit{@cert}, \textit{@resp})) (\hyperref[TEI.att.global.source]{att.global.source} (\textit{@source})) \hyperref[TEI.att.datable]{att.datable} (\textit{@calendar}, \textit{@period})  (\hyperref[TEI.att.datable.w3c]{att.datable.w3c} (\textit{@when}, \textit{@notBefore}, \textit{@notAfter}, \textit{@from}, \textit{@to})) (\hyperref[TEI.att.datable.iso]{att.datable.iso} (\textit{@when-iso}, \textit{@notBefore-iso}, \textit{@notAfter-iso}, \textit{@from-iso}, \textit{@to-iso})) (\hyperref[TEI.att.datable.custom]{att.datable.custom} (\textit{@when-custom}, \textit{@notBefore-custom}, \textit{@notAfter-custom}, \textit{@from-custom}, \textit{@to-custom}, \textit{@datingPoint}, \textit{@datingMethod})) \hyperref[TEI.att.editLike]{att.editLike} (\textit{@evidence}, \textit{@instant})  (\hyperref[TEI.att.dimensions]{att.dimensions} (\textit{@unit}, \textit{@quantity}, \textit{@extent}, \textit{@precision}, \textit{@scope}) (\hyperref[TEI.att.ranging]{att.ranging} (\textit{@atLeast}, \textit{@atMost}, \textit{@min}, \textit{@max}, \textit{@confidence})) ) \hyperref[TEI.att.typed]{att.typed} (\textit{@type}, \textit{@subtype}) \hyperref[TEI.att.naming]{att.naming} (\textit{@role}, \textit{@nymRef})  (\hyperref[TEI.att.canonical]{att.canonical} (\textit{@key}, \textit{@ref})) \hyperref[TEI.att.sortable]{att.sortable} (\textit{@sortKey}) \hfil\\[-10pt]\begin{sansreflist}
    \item[@where]
  indique la localisation d'un évènement en pointant vers un élément \hyperref[TEI.place]{<place>}
\begin{reflist}
    \item[{Statut}]
  Optionel
    \item[{Type de données}]
  \hyperref[TEI.teidata.pointer]{teidata.pointer}
\end{reflist}  
\end{sansreflist}  
    \item[{Membre du}]
  \hyperref[TEI.model.eventLike]{model.eventLike}
    \item[{Contenu dans}]
  
    \item[namesdates: ]
   \hyperref[TEI.event]{event} \hyperref[TEI.org]{org} \hyperref[TEI.person]{person} \hyperref[TEI.personGrp]{personGrp} \hyperref[TEI.persona]{persona} \hyperref[TEI.place]{place}
    \item[{Peut contenir}]
  
    \item[core: ]
   \hyperref[TEI.bibl]{bibl} \hyperref[TEI.biblStruct]{biblStruct} \hyperref[TEI.desc]{desc} \hyperref[TEI.head]{head} \hyperref[TEI.label]{label} \hyperref[TEI.listBibl]{listBibl} \hyperref[TEI.note]{note} \hyperref[TEI.p]{p}\par 
    \item[header: ]
   \hyperref[TEI.biblFull]{biblFull}\par 
    \item[linking: ]
   \hyperref[TEI.ab]{ab} \hyperref[TEI.link]{link} \hyperref[TEI.linkGrp]{linkGrp}\par 
    \item[msdescription: ]
   \hyperref[TEI.msDesc]{msDesc}\par 
    \item[namesdates: ]
   \hyperref[TEI.event]{event}
    \item[{Exemple}]
  \leavevmode\bgroup\exampleFont \begin{shaded}\noindent\mbox{}{<\textbf{person}>}\mbox{}\newline 
\hspace*{6pt}{<\textbf{event}\hspace*{6pt}{type}="{mat}"\hspace*{6pt}{when}="{1972-10-12}">}\mbox{}\newline 
\hspace*{6pt}\hspace*{6pt}{<\textbf{label}>}inscription{</\textbf{label}>}\mbox{}\newline 
\hspace*{6pt}{</\textbf{event}>}\mbox{}\newline 
\hspace*{6pt}{<\textbf{event}\hspace*{6pt}{type}="{grad}"\hspace*{6pt}{when}="{1975-06-23}">}\mbox{}\newline 
\hspace*{6pt}\hspace*{6pt}{<\textbf{label}>}diplômé{</\textbf{label}>}\mbox{}\newline 
\hspace*{6pt}{</\textbf{event}>}\mbox{}\newline 
{</\textbf{person}>}\end{shaded}\egroup 


    \item[{Modèle de contenu}]
  \mbox{}\hfill\\[-10pt]\begin{Verbatim}[fontsize=\small]
<content>
 <sequence maxOccurs="1" minOccurs="1">
  <classRef key="model.headLike"
   maxOccurs="unbounded" minOccurs="0"/>
  <alternate maxOccurs="1" minOccurs="1">
   <classRef key="model.pLike"
    maxOccurs="unbounded" minOccurs="1"/>
   <classRef key="model.labelLike"
    maxOccurs="unbounded" minOccurs="1"/>
  </alternate>
  <alternate maxOccurs="unbounded"
   minOccurs="0">
   <classRef key="model.noteLike"/>
   <classRef key="model.biblLike"/>
   <elementRef key="linkGrp"/>
   <elementRef key="link"/>
  </alternate>
  <elementRef key="event"
   maxOccurs="unbounded" minOccurs="0"/>
 </sequence>
</content>
    
\end{Verbatim}

    \item[{Schéma Declaration}]
  \mbox{}\hfill\\[-10pt]\begin{Verbatim}[fontsize=\small]
element event
{
   tei_att.global.attributes,
   tei_att.datable.attributes,
   tei_att.editLike.attributes,
   tei_att.typed.attributes,
   tei_att.naming.attributes,
   tei_att.sortable.attributes,
   attribute where { text }?,
   (
      tei_model.headLike*,
      ( tei_model.pLike+ | tei_model.labelLike+ ),
      ( tei_model.noteLike | tei_model.biblLike | tei_linkGrp | tei_link )*,
      tei_event*
   )
}
\end{Verbatim}

\end{reflist}  \index{ex=<ex>|oddindex}
\begin{reflist}
\item[]\begin{specHead}{TEI.ex}{<ex> }(développement éditorial) contient une succession de lettres ajoutées par un éditeur ou un transcripteur pour développer une abréviation. [\xref{http://www.tei-c.org/release/doc/tei-p5-doc/en/html/PH.html\#PHAB}{11.3.1.2. Abbreviation and Expansion}]\end{specHead} 
    \item[{Module}]
  transcr
    \item[{Attributs}]
  Attributs \hyperref[TEI.att.global]{att.global} (\textit{@xml:id}, \textit{@n}, \textit{@xml:lang}, \textit{@xml:base}, \textit{@xml:space})  (\hyperref[TEI.att.global.rendition]{att.global.rendition} (\textit{@rend}, \textit{@style}, \textit{@rendition})) (\hyperref[TEI.att.global.linking]{att.global.linking} (\textit{@corresp}, \textit{@synch}, \textit{@sameAs}, \textit{@copyOf}, \textit{@next}, \textit{@prev}, \textit{@exclude}, \textit{@select})) (\hyperref[TEI.att.global.analytic]{att.global.analytic} (\textit{@ana})) (\hyperref[TEI.att.global.facs]{att.global.facs} (\textit{@facs})) (\hyperref[TEI.att.global.change]{att.global.change} (\textit{@change})) (\hyperref[TEI.att.global.responsibility]{att.global.responsibility} (\textit{@cert}, \textit{@resp})) (\hyperref[TEI.att.global.source]{att.global.source} (\textit{@source})) \hyperref[TEI.att.editLike]{att.editLike} (\textit{@evidence}, \textit{@instant})  (\hyperref[TEI.att.dimensions]{att.dimensions} (\textit{@unit}, \textit{@quantity}, \textit{@extent}, \textit{@precision}, \textit{@scope}) (\hyperref[TEI.att.ranging]{att.ranging} (\textit{@atLeast}, \textit{@atMost}, \textit{@min}, \textit{@max}, \textit{@confidence})) )
    \item[{Membre du}]
  \hyperref[TEI.model.choicePart]{model.choicePart} \hyperref[TEI.model.pPart.editorial]{model.pPart.editorial}
    \item[{Contenu dans}]
  
    \item[analysis: ]
   \hyperref[TEI.cl]{cl} \hyperref[TEI.pc]{pc} \hyperref[TEI.phr]{phr} \hyperref[TEI.s]{s} \hyperref[TEI.span]{span} \hyperref[TEI.w]{w}\par 
    \item[core: ]
   \hyperref[TEI.abbr]{abbr} \hyperref[TEI.add]{add} \hyperref[TEI.addrLine]{addrLine} \hyperref[TEI.author]{author} \hyperref[TEI.bibl]{bibl} \hyperref[TEI.biblScope]{biblScope} \hyperref[TEI.choice]{choice} \hyperref[TEI.citedRange]{citedRange} \hyperref[TEI.corr]{corr} \hyperref[TEI.date]{date} \hyperref[TEI.del]{del} \hyperref[TEI.desc]{desc} \hyperref[TEI.distinct]{distinct} \hyperref[TEI.editor]{editor} \hyperref[TEI.email]{email} \hyperref[TEI.emph]{emph} \hyperref[TEI.expan]{expan} \hyperref[TEI.foreign]{foreign} \hyperref[TEI.gloss]{gloss} \hyperref[TEI.head]{head} \hyperref[TEI.headItem]{headItem} \hyperref[TEI.headLabel]{headLabel} \hyperref[TEI.hi]{hi} \hyperref[TEI.item]{item} \hyperref[TEI.l]{l} \hyperref[TEI.label]{label} \hyperref[TEI.measure]{measure} \hyperref[TEI.meeting]{meeting} \hyperref[TEI.mentioned]{mentioned} \hyperref[TEI.name]{name} \hyperref[TEI.note]{note} \hyperref[TEI.num]{num} \hyperref[TEI.orig]{orig} \hyperref[TEI.p]{p} \hyperref[TEI.pubPlace]{pubPlace} \hyperref[TEI.publisher]{publisher} \hyperref[TEI.q]{q} \hyperref[TEI.quote]{quote} \hyperref[TEI.ref]{ref} \hyperref[TEI.reg]{reg} \hyperref[TEI.resp]{resp} \hyperref[TEI.rs]{rs} \hyperref[TEI.said]{said} \hyperref[TEI.sic]{sic} \hyperref[TEI.soCalled]{soCalled} \hyperref[TEI.speaker]{speaker} \hyperref[TEI.stage]{stage} \hyperref[TEI.street]{street} \hyperref[TEI.term]{term} \hyperref[TEI.textLang]{textLang} \hyperref[TEI.time]{time} \hyperref[TEI.title]{title} \hyperref[TEI.unclear]{unclear}\par 
    \item[figures: ]
   \hyperref[TEI.cell]{cell} \hyperref[TEI.figDesc]{figDesc}\par 
    \item[header: ]
   \hyperref[TEI.authority]{authority} \hyperref[TEI.change]{change} \hyperref[TEI.classCode]{classCode} \hyperref[TEI.creation]{creation} \hyperref[TEI.distributor]{distributor} \hyperref[TEI.edition]{edition} \hyperref[TEI.extent]{extent} \hyperref[TEI.funder]{funder} \hyperref[TEI.language]{language} \hyperref[TEI.licence]{licence} \hyperref[TEI.rendition]{rendition}\par 
    \item[iso-fs: ]
   \hyperref[TEI.fDescr]{fDescr} \hyperref[TEI.fsDescr]{fsDescr}\par 
    \item[linking: ]
   \hyperref[TEI.ab]{ab} \hyperref[TEI.seg]{seg}\par 
    \item[msdescription: ]
   \hyperref[TEI.accMat]{accMat} \hyperref[TEI.acquisition]{acquisition} \hyperref[TEI.additions]{additions} \hyperref[TEI.catchwords]{catchwords} \hyperref[TEI.collation]{collation} \hyperref[TEI.colophon]{colophon} \hyperref[TEI.condition]{condition} \hyperref[TEI.custEvent]{custEvent} \hyperref[TEI.decoNote]{decoNote} \hyperref[TEI.explicit]{explicit} \hyperref[TEI.filiation]{filiation} \hyperref[TEI.finalRubric]{finalRubric} \hyperref[TEI.foliation]{foliation} \hyperref[TEI.heraldry]{heraldry} \hyperref[TEI.incipit]{incipit} \hyperref[TEI.layout]{layout} \hyperref[TEI.material]{material} \hyperref[TEI.musicNotation]{musicNotation} \hyperref[TEI.objectType]{objectType} \hyperref[TEI.origDate]{origDate} \hyperref[TEI.origPlace]{origPlace} \hyperref[TEI.origin]{origin} \hyperref[TEI.provenance]{provenance} \hyperref[TEI.rubric]{rubric} \hyperref[TEI.secFol]{secFol} \hyperref[TEI.signatures]{signatures} \hyperref[TEI.source]{source} \hyperref[TEI.stamp]{stamp} \hyperref[TEI.summary]{summary} \hyperref[TEI.support]{support} \hyperref[TEI.surrogates]{surrogates} \hyperref[TEI.typeNote]{typeNote} \hyperref[TEI.watermark]{watermark}\par 
    \item[namesdates: ]
   \hyperref[TEI.addName]{addName} \hyperref[TEI.affiliation]{affiliation} \hyperref[TEI.country]{country} \hyperref[TEI.forename]{forename} \hyperref[TEI.genName]{genName} \hyperref[TEI.geogName]{geogName} \hyperref[TEI.nameLink]{nameLink} \hyperref[TEI.orgName]{orgName} \hyperref[TEI.persName]{persName} \hyperref[TEI.placeName]{placeName} \hyperref[TEI.region]{region} \hyperref[TEI.roleName]{roleName} \hyperref[TEI.settlement]{settlement} \hyperref[TEI.surname]{surname}\par 
    \item[textstructure: ]
   \hyperref[TEI.docAuthor]{docAuthor} \hyperref[TEI.docDate]{docDate} \hyperref[TEI.docEdition]{docEdition} \hyperref[TEI.titlePart]{titlePart}\par 
    \item[transcr: ]
   \hyperref[TEI.damage]{damage} \hyperref[TEI.fw]{fw} \hyperref[TEI.metamark]{metamark} \hyperref[TEI.mod]{mod} \hyperref[TEI.restore]{restore} \hyperref[TEI.retrace]{retrace} \hyperref[TEI.secl]{secl} \hyperref[TEI.supplied]{supplied} \hyperref[TEI.surplus]{surplus}
    \item[{Peut contenir}]
  Des données textuelles uniquement
    \item[{Exemple}]
  \leavevmode\bgroup\exampleFont \begin{shaded}\noindent\mbox{}Il habite au 15 {<\textbf{choice}>}\mbox{}\newline 
\hspace*{6pt}{<\textbf{expan}>}b{<\textbf{ex}>}oulevard{</\textbf{ex}>}d{</\textbf{expan}>}\mbox{}\newline 
\hspace*{6pt}{<\textbf{abbr}>}bd{</\textbf{abbr}>}\mbox{}\newline 
{</\textbf{choice}>}Clemenceau. \end{shaded}\egroup 


    \item[{Modèle de contenu}]
  \fbox{\ttfamily <content>\newline
 <macroRef key="macro.xtext"/>\newline
</content>\newline
    } 
    \item[{Schéma Declaration}]
  \mbox{}\hfill\\[-10pt]\begin{Verbatim}[fontsize=\small]
element ex
{
   tei_att.global.attributes,
   tei_att.editLike.attributes,
   tei_macro.xtext}
\end{Verbatim}

\end{reflist}  \index{expan=<expan>|oddindex}
\begin{reflist}
\item[]\begin{specHead}{TEI.expan}{<expan> }(expansion) contient l'expansion d'une abréviation. [\xref{http://www.tei-c.org/release/doc/tei-p5-doc/en/html/CO.html\#CONAAB}{3.5.5. Abbreviations and Their Expansions}]\end{specHead} 
    \item[{Module}]
  core
    \item[{Attributs}]
  Attributs \hyperref[TEI.att.global]{att.global} (\textit{@xml:id}, \textit{@n}, \textit{@xml:lang}, \textit{@xml:base}, \textit{@xml:space})  (\hyperref[TEI.att.global.rendition]{att.global.rendition} (\textit{@rend}, \textit{@style}, \textit{@rendition})) (\hyperref[TEI.att.global.linking]{att.global.linking} (\textit{@corresp}, \textit{@synch}, \textit{@sameAs}, \textit{@copyOf}, \textit{@next}, \textit{@prev}, \textit{@exclude}, \textit{@select})) (\hyperref[TEI.att.global.analytic]{att.global.analytic} (\textit{@ana})) (\hyperref[TEI.att.global.facs]{att.global.facs} (\textit{@facs})) (\hyperref[TEI.att.global.change]{att.global.change} (\textit{@change})) (\hyperref[TEI.att.global.responsibility]{att.global.responsibility} (\textit{@cert}, \textit{@resp})) (\hyperref[TEI.att.global.source]{att.global.source} (\textit{@source})) \hyperref[TEI.att.editLike]{att.editLike} (\textit{@evidence}, \textit{@instant})  (\hyperref[TEI.att.dimensions]{att.dimensions} (\textit{@unit}, \textit{@quantity}, \textit{@extent}, \textit{@precision}, \textit{@scope}) (\hyperref[TEI.att.ranging]{att.ranging} (\textit{@atLeast}, \textit{@atMost}, \textit{@min}, \textit{@max}, \textit{@confidence})) )
    \item[{Membre du}]
  \hyperref[TEI.model.choicePart]{model.choicePart} \hyperref[TEI.model.pPart.editorial]{model.pPart.editorial}
    \item[{Contenu dans}]
  
    \item[analysis: ]
   \hyperref[TEI.cl]{cl} \hyperref[TEI.pc]{pc} \hyperref[TEI.phr]{phr} \hyperref[TEI.s]{s} \hyperref[TEI.span]{span} \hyperref[TEI.w]{w}\par 
    \item[core: ]
   \hyperref[TEI.abbr]{abbr} \hyperref[TEI.add]{add} \hyperref[TEI.addrLine]{addrLine} \hyperref[TEI.author]{author} \hyperref[TEI.bibl]{bibl} \hyperref[TEI.biblScope]{biblScope} \hyperref[TEI.choice]{choice} \hyperref[TEI.citedRange]{citedRange} \hyperref[TEI.corr]{corr} \hyperref[TEI.date]{date} \hyperref[TEI.del]{del} \hyperref[TEI.desc]{desc} \hyperref[TEI.distinct]{distinct} \hyperref[TEI.editor]{editor} \hyperref[TEI.email]{email} \hyperref[TEI.emph]{emph} \hyperref[TEI.expan]{expan} \hyperref[TEI.foreign]{foreign} \hyperref[TEI.gloss]{gloss} \hyperref[TEI.head]{head} \hyperref[TEI.headItem]{headItem} \hyperref[TEI.headLabel]{headLabel} \hyperref[TEI.hi]{hi} \hyperref[TEI.item]{item} \hyperref[TEI.l]{l} \hyperref[TEI.label]{label} \hyperref[TEI.measure]{measure} \hyperref[TEI.meeting]{meeting} \hyperref[TEI.mentioned]{mentioned} \hyperref[TEI.name]{name} \hyperref[TEI.note]{note} \hyperref[TEI.num]{num} \hyperref[TEI.orig]{orig} \hyperref[TEI.p]{p} \hyperref[TEI.pubPlace]{pubPlace} \hyperref[TEI.publisher]{publisher} \hyperref[TEI.q]{q} \hyperref[TEI.quote]{quote} \hyperref[TEI.ref]{ref} \hyperref[TEI.reg]{reg} \hyperref[TEI.resp]{resp} \hyperref[TEI.rs]{rs} \hyperref[TEI.said]{said} \hyperref[TEI.sic]{sic} \hyperref[TEI.soCalled]{soCalled} \hyperref[TEI.speaker]{speaker} \hyperref[TEI.stage]{stage} \hyperref[TEI.street]{street} \hyperref[TEI.term]{term} \hyperref[TEI.textLang]{textLang} \hyperref[TEI.time]{time} \hyperref[TEI.title]{title} \hyperref[TEI.unclear]{unclear}\par 
    \item[figures: ]
   \hyperref[TEI.cell]{cell} \hyperref[TEI.figDesc]{figDesc}\par 
    \item[header: ]
   \hyperref[TEI.authority]{authority} \hyperref[TEI.change]{change} \hyperref[TEI.classCode]{classCode} \hyperref[TEI.creation]{creation} \hyperref[TEI.distributor]{distributor} \hyperref[TEI.edition]{edition} \hyperref[TEI.extent]{extent} \hyperref[TEI.funder]{funder} \hyperref[TEI.language]{language} \hyperref[TEI.licence]{licence} \hyperref[TEI.rendition]{rendition}\par 
    \item[iso-fs: ]
   \hyperref[TEI.fDescr]{fDescr} \hyperref[TEI.fsDescr]{fsDescr}\par 
    \item[linking: ]
   \hyperref[TEI.ab]{ab} \hyperref[TEI.seg]{seg}\par 
    \item[msdescription: ]
   \hyperref[TEI.accMat]{accMat} \hyperref[TEI.acquisition]{acquisition} \hyperref[TEI.additions]{additions} \hyperref[TEI.catchwords]{catchwords} \hyperref[TEI.collation]{collation} \hyperref[TEI.colophon]{colophon} \hyperref[TEI.condition]{condition} \hyperref[TEI.custEvent]{custEvent} \hyperref[TEI.decoNote]{decoNote} \hyperref[TEI.explicit]{explicit} \hyperref[TEI.filiation]{filiation} \hyperref[TEI.finalRubric]{finalRubric} \hyperref[TEI.foliation]{foliation} \hyperref[TEI.heraldry]{heraldry} \hyperref[TEI.incipit]{incipit} \hyperref[TEI.layout]{layout} \hyperref[TEI.material]{material} \hyperref[TEI.musicNotation]{musicNotation} \hyperref[TEI.objectType]{objectType} \hyperref[TEI.origDate]{origDate} \hyperref[TEI.origPlace]{origPlace} \hyperref[TEI.origin]{origin} \hyperref[TEI.provenance]{provenance} \hyperref[TEI.rubric]{rubric} \hyperref[TEI.secFol]{secFol} \hyperref[TEI.signatures]{signatures} \hyperref[TEI.source]{source} \hyperref[TEI.stamp]{stamp} \hyperref[TEI.summary]{summary} \hyperref[TEI.support]{support} \hyperref[TEI.surrogates]{surrogates} \hyperref[TEI.typeNote]{typeNote} \hyperref[TEI.watermark]{watermark}\par 
    \item[namesdates: ]
   \hyperref[TEI.addName]{addName} \hyperref[TEI.affiliation]{affiliation} \hyperref[TEI.country]{country} \hyperref[TEI.forename]{forename} \hyperref[TEI.genName]{genName} \hyperref[TEI.geogName]{geogName} \hyperref[TEI.nameLink]{nameLink} \hyperref[TEI.orgName]{orgName} \hyperref[TEI.persName]{persName} \hyperref[TEI.placeName]{placeName} \hyperref[TEI.region]{region} \hyperref[TEI.roleName]{roleName} \hyperref[TEI.settlement]{settlement} \hyperref[TEI.surname]{surname}\par 
    \item[textstructure: ]
   \hyperref[TEI.docAuthor]{docAuthor} \hyperref[TEI.docDate]{docDate} \hyperref[TEI.docEdition]{docEdition} \hyperref[TEI.titlePart]{titlePart}\par 
    \item[transcr: ]
   \hyperref[TEI.damage]{damage} \hyperref[TEI.fw]{fw} \hyperref[TEI.metamark]{metamark} \hyperref[TEI.mod]{mod} \hyperref[TEI.restore]{restore} \hyperref[TEI.retrace]{retrace} \hyperref[TEI.secl]{secl} \hyperref[TEI.supplied]{supplied} \hyperref[TEI.surplus]{surplus}
    \item[{Peut contenir}]
  
    \item[analysis: ]
   \hyperref[TEI.c]{c} \hyperref[TEI.cl]{cl} \hyperref[TEI.interp]{interp} \hyperref[TEI.interpGrp]{interpGrp} \hyperref[TEI.m]{m} \hyperref[TEI.pc]{pc} \hyperref[TEI.phr]{phr} \hyperref[TEI.s]{s} \hyperref[TEI.span]{span} \hyperref[TEI.spanGrp]{spanGrp} \hyperref[TEI.w]{w}\par 
    \item[core: ]
   \hyperref[TEI.abbr]{abbr} \hyperref[TEI.add]{add} \hyperref[TEI.address]{address} \hyperref[TEI.binaryObject]{binaryObject} \hyperref[TEI.cb]{cb} \hyperref[TEI.choice]{choice} \hyperref[TEI.corr]{corr} \hyperref[TEI.date]{date} \hyperref[TEI.del]{del} \hyperref[TEI.distinct]{distinct} \hyperref[TEI.email]{email} \hyperref[TEI.emph]{emph} \hyperref[TEI.expan]{expan} \hyperref[TEI.foreign]{foreign} \hyperref[TEI.gap]{gap} \hyperref[TEI.gb]{gb} \hyperref[TEI.gloss]{gloss} \hyperref[TEI.graphic]{graphic} \hyperref[TEI.hi]{hi} \hyperref[TEI.index]{index} \hyperref[TEI.lb]{lb} \hyperref[TEI.measure]{measure} \hyperref[TEI.measureGrp]{measureGrp} \hyperref[TEI.media]{media} \hyperref[TEI.mentioned]{mentioned} \hyperref[TEI.milestone]{milestone} \hyperref[TEI.name]{name} \hyperref[TEI.note]{note} \hyperref[TEI.num]{num} \hyperref[TEI.orig]{orig} \hyperref[TEI.pb]{pb} \hyperref[TEI.ptr]{ptr} \hyperref[TEI.ref]{ref} \hyperref[TEI.reg]{reg} \hyperref[TEI.rs]{rs} \hyperref[TEI.sic]{sic} \hyperref[TEI.soCalled]{soCalled} \hyperref[TEI.term]{term} \hyperref[TEI.time]{time} \hyperref[TEI.title]{title} \hyperref[TEI.unclear]{unclear}\par 
    \item[derived-module-tei.istex: ]
   \hyperref[TEI.math]{math} \hyperref[TEI.mrow]{mrow}\par 
    \item[figures: ]
   \hyperref[TEI.figure]{figure} \hyperref[TEI.formula]{formula} \hyperref[TEI.notatedMusic]{notatedMusic}\par 
    \item[header: ]
   \hyperref[TEI.idno]{idno}\par 
    \item[iso-fs: ]
   \hyperref[TEI.fLib]{fLib} \hyperref[TEI.fs]{fs} \hyperref[TEI.fvLib]{fvLib}\par 
    \item[linking: ]
   \hyperref[TEI.alt]{alt} \hyperref[TEI.altGrp]{altGrp} \hyperref[TEI.anchor]{anchor} \hyperref[TEI.join]{join} \hyperref[TEI.joinGrp]{joinGrp} \hyperref[TEI.link]{link} \hyperref[TEI.linkGrp]{linkGrp} \hyperref[TEI.seg]{seg} \hyperref[TEI.timeline]{timeline}\par 
    \item[msdescription: ]
   \hyperref[TEI.catchwords]{catchwords} \hyperref[TEI.depth]{depth} \hyperref[TEI.dim]{dim} \hyperref[TEI.dimensions]{dimensions} \hyperref[TEI.height]{height} \hyperref[TEI.heraldry]{heraldry} \hyperref[TEI.locus]{locus} \hyperref[TEI.locusGrp]{locusGrp} \hyperref[TEI.material]{material} \hyperref[TEI.objectType]{objectType} \hyperref[TEI.origDate]{origDate} \hyperref[TEI.origPlace]{origPlace} \hyperref[TEI.secFol]{secFol} \hyperref[TEI.signatures]{signatures} \hyperref[TEI.source]{source} \hyperref[TEI.stamp]{stamp} \hyperref[TEI.watermark]{watermark} \hyperref[TEI.width]{width}\par 
    \item[namesdates: ]
   \hyperref[TEI.addName]{addName} \hyperref[TEI.affiliation]{affiliation} \hyperref[TEI.country]{country} \hyperref[TEI.forename]{forename} \hyperref[TEI.genName]{genName} \hyperref[TEI.geogName]{geogName} \hyperref[TEI.location]{location} \hyperref[TEI.nameLink]{nameLink} \hyperref[TEI.orgName]{orgName} \hyperref[TEI.persName]{persName} \hyperref[TEI.placeName]{placeName} \hyperref[TEI.region]{region} \hyperref[TEI.roleName]{roleName} \hyperref[TEI.settlement]{settlement} \hyperref[TEI.state]{state} \hyperref[TEI.surname]{surname}\par 
    \item[spoken: ]
   \hyperref[TEI.annotationBlock]{annotationBlock}\par 
    \item[transcr: ]
   \hyperref[TEI.addSpan]{addSpan} \hyperref[TEI.am]{am} \hyperref[TEI.damage]{damage} \hyperref[TEI.damageSpan]{damageSpan} \hyperref[TEI.delSpan]{delSpan} \hyperref[TEI.ex]{ex} \hyperref[TEI.fw]{fw} \hyperref[TEI.handShift]{handShift} \hyperref[TEI.listTranspose]{listTranspose} \hyperref[TEI.metamark]{metamark} \hyperref[TEI.mod]{mod} \hyperref[TEI.redo]{redo} \hyperref[TEI.restore]{restore} \hyperref[TEI.retrace]{retrace} \hyperref[TEI.secl]{secl} \hyperref[TEI.space]{space} \hyperref[TEI.subst]{subst} \hyperref[TEI.substJoin]{substJoin} \hyperref[TEI.supplied]{supplied} \hyperref[TEI.surplus]{surplus} \hyperref[TEI.undo]{undo}\par des données textuelles
    \item[{Note}]
  \par
En général, le contenu de cet élément doit être une expression ou un mot complet. L'élément \hyperref[TEI.ex]{<ex>} fourni par le module \textsf{transcr} peut être utilisé pour baliser des suites de lettres données dans une expansion de ce type.
    \item[{Exemple}]
  \leavevmode\bgroup\exampleFont \begin{shaded}\noindent\mbox{}Il habite\mbox{}\newline 
{<\textbf{choice}>}\mbox{}\newline 
\hspace*{6pt}{<\textbf{expan}>}Avenue{</\textbf{expan}>}\mbox{}\newline 
\hspace*{6pt}{<\textbf{abbr}>}Av.{</\textbf{abbr}>}\mbox{}\newline 
{</\textbf{choice}>}de la Paix\mbox{}\newline 
\end{shaded}\egroup 


    \item[{Modèle de contenu}]
  \mbox{}\hfill\\[-10pt]\begin{Verbatim}[fontsize=\small]
<content>
 <macroRef key="macro.phraseSeq"/>
</content>
    
\end{Verbatim}

    \item[{Schéma Declaration}]
  \mbox{}\hfill\\[-10pt]\begin{Verbatim}[fontsize=\small]
element expan
{
   tei_att.global.attributes,
   tei_att.editLike.attributes,
   tei_macro.phraseSeq}
\end{Verbatim}

\end{reflist}  \index{explicit=<explicit>|oddindex}
\begin{reflist}
\item[]\begin{specHead}{TEI.explicit}{<explicit> }(explicit) contient l'\textit{explicit} d'une section d'un manuscrit, c'est-à-dire les mots terminant le texte proprement dit, à l'exclusion de toute rubrique ou colophon qui pourraient le suivre. [\xref{http://www.tei-c.org/release/doc/tei-p5-doc/en/html/MS.html\#mscoit}{10.6.1. The msItem and msItemStruct Elements}]\end{specHead} 
    \item[{Module}]
  msdescription
    \item[{Attributs}]
  Attributs \hyperref[TEI.att.global]{att.global} (\textit{@xml:id}, \textit{@n}, \textit{@xml:lang}, \textit{@xml:base}, \textit{@xml:space})  (\hyperref[TEI.att.global.rendition]{att.global.rendition} (\textit{@rend}, \textit{@style}, \textit{@rendition})) (\hyperref[TEI.att.global.linking]{att.global.linking} (\textit{@corresp}, \textit{@synch}, \textit{@sameAs}, \textit{@copyOf}, \textit{@next}, \textit{@prev}, \textit{@exclude}, \textit{@select})) (\hyperref[TEI.att.global.analytic]{att.global.analytic} (\textit{@ana})) (\hyperref[TEI.att.global.facs]{att.global.facs} (\textit{@facs})) (\hyperref[TEI.att.global.change]{att.global.change} (\textit{@change})) (\hyperref[TEI.att.global.responsibility]{att.global.responsibility} (\textit{@cert}, \textit{@resp})) (\hyperref[TEI.att.global.source]{att.global.source} (\textit{@source})) \hyperref[TEI.att.typed]{att.typed} (\textit{@type}, \textit{@subtype}) \hyperref[TEI.att.msExcerpt]{att.msExcerpt} (\textit{@defective}) 
    \item[{Membre du}]
  \hyperref[TEI.model.msQuoteLike]{model.msQuoteLike} 
    \item[{Contenu dans}]
  
    \item[msdescription: ]
   \hyperref[TEI.msItem]{msItem} \hyperref[TEI.msItemStruct]{msItemStruct}
    \item[{Peut contenir}]
  
    \item[analysis: ]
   \hyperref[TEI.c]{c} \hyperref[TEI.cl]{cl} \hyperref[TEI.interp]{interp} \hyperref[TEI.interpGrp]{interpGrp} \hyperref[TEI.m]{m} \hyperref[TEI.pc]{pc} \hyperref[TEI.phr]{phr} \hyperref[TEI.s]{s} \hyperref[TEI.span]{span} \hyperref[TEI.spanGrp]{spanGrp} \hyperref[TEI.w]{w}\par 
    \item[core: ]
   \hyperref[TEI.abbr]{abbr} \hyperref[TEI.add]{add} \hyperref[TEI.address]{address} \hyperref[TEI.binaryObject]{binaryObject} \hyperref[TEI.cb]{cb} \hyperref[TEI.choice]{choice} \hyperref[TEI.corr]{corr} \hyperref[TEI.date]{date} \hyperref[TEI.del]{del} \hyperref[TEI.distinct]{distinct} \hyperref[TEI.email]{email} \hyperref[TEI.emph]{emph} \hyperref[TEI.expan]{expan} \hyperref[TEI.foreign]{foreign} \hyperref[TEI.gap]{gap} \hyperref[TEI.gb]{gb} \hyperref[TEI.gloss]{gloss} \hyperref[TEI.graphic]{graphic} \hyperref[TEI.hi]{hi} \hyperref[TEI.index]{index} \hyperref[TEI.lb]{lb} \hyperref[TEI.measure]{measure} \hyperref[TEI.measureGrp]{measureGrp} \hyperref[TEI.media]{media} \hyperref[TEI.mentioned]{mentioned} \hyperref[TEI.milestone]{milestone} \hyperref[TEI.name]{name} \hyperref[TEI.note]{note} \hyperref[TEI.num]{num} \hyperref[TEI.orig]{orig} \hyperref[TEI.pb]{pb} \hyperref[TEI.ptr]{ptr} \hyperref[TEI.ref]{ref} \hyperref[TEI.reg]{reg} \hyperref[TEI.rs]{rs} \hyperref[TEI.sic]{sic} \hyperref[TEI.soCalled]{soCalled} \hyperref[TEI.term]{term} \hyperref[TEI.time]{time} \hyperref[TEI.title]{title} \hyperref[TEI.unclear]{unclear}\par 
    \item[derived-module-tei.istex: ]
   \hyperref[TEI.math]{math} \hyperref[TEI.mrow]{mrow}\par 
    \item[figures: ]
   \hyperref[TEI.figure]{figure} \hyperref[TEI.formula]{formula} \hyperref[TEI.notatedMusic]{notatedMusic}\par 
    \item[header: ]
   \hyperref[TEI.idno]{idno}\par 
    \item[iso-fs: ]
   \hyperref[TEI.fLib]{fLib} \hyperref[TEI.fs]{fs} \hyperref[TEI.fvLib]{fvLib}\par 
    \item[linking: ]
   \hyperref[TEI.alt]{alt} \hyperref[TEI.altGrp]{altGrp} \hyperref[TEI.anchor]{anchor} \hyperref[TEI.join]{join} \hyperref[TEI.joinGrp]{joinGrp} \hyperref[TEI.link]{link} \hyperref[TEI.linkGrp]{linkGrp} \hyperref[TEI.seg]{seg} \hyperref[TEI.timeline]{timeline}\par 
    \item[msdescription: ]
   \hyperref[TEI.catchwords]{catchwords} \hyperref[TEI.depth]{depth} \hyperref[TEI.dim]{dim} \hyperref[TEI.dimensions]{dimensions} \hyperref[TEI.height]{height} \hyperref[TEI.heraldry]{heraldry} \hyperref[TEI.locus]{locus} \hyperref[TEI.locusGrp]{locusGrp} \hyperref[TEI.material]{material} \hyperref[TEI.objectType]{objectType} \hyperref[TEI.origDate]{origDate} \hyperref[TEI.origPlace]{origPlace} \hyperref[TEI.secFol]{secFol} \hyperref[TEI.signatures]{signatures} \hyperref[TEI.source]{source} \hyperref[TEI.stamp]{stamp} \hyperref[TEI.watermark]{watermark} \hyperref[TEI.width]{width}\par 
    \item[namesdates: ]
   \hyperref[TEI.addName]{addName} \hyperref[TEI.affiliation]{affiliation} \hyperref[TEI.country]{country} \hyperref[TEI.forename]{forename} \hyperref[TEI.genName]{genName} \hyperref[TEI.geogName]{geogName} \hyperref[TEI.location]{location} \hyperref[TEI.nameLink]{nameLink} \hyperref[TEI.orgName]{orgName} \hyperref[TEI.persName]{persName} \hyperref[TEI.placeName]{placeName} \hyperref[TEI.region]{region} \hyperref[TEI.roleName]{roleName} \hyperref[TEI.settlement]{settlement} \hyperref[TEI.state]{state} \hyperref[TEI.surname]{surname}\par 
    \item[spoken: ]
   \hyperref[TEI.annotationBlock]{annotationBlock}\par 
    \item[transcr: ]
   \hyperref[TEI.addSpan]{addSpan} \hyperref[TEI.am]{am} \hyperref[TEI.damage]{damage} \hyperref[TEI.damageSpan]{damageSpan} \hyperref[TEI.delSpan]{delSpan} \hyperref[TEI.ex]{ex} \hyperref[TEI.fw]{fw} \hyperref[TEI.handShift]{handShift} \hyperref[TEI.listTranspose]{listTranspose} \hyperref[TEI.metamark]{metamark} \hyperref[TEI.mod]{mod} \hyperref[TEI.redo]{redo} \hyperref[TEI.restore]{restore} \hyperref[TEI.retrace]{retrace} \hyperref[TEI.secl]{secl} \hyperref[TEI.space]{space} \hyperref[TEI.subst]{subst} \hyperref[TEI.substJoin]{substJoin} \hyperref[TEI.supplied]{supplied} \hyperref[TEI.surplus]{surplus} \hyperref[TEI.undo]{undo}\par des données textuelles
    \item[{Exemple}]
  \leavevmode\bgroup\exampleFont \begin{shaded}\noindent\mbox{}{<\textbf{explicit}>}sed libera nos a malo.{</\textbf{explicit}>}\mbox{}\newline 
{<\textbf{rubric}>}Hic explicit oratio qui dicitur dominica.{</\textbf{rubric}>}\mbox{}\newline 
{<\textbf{explicit}\hspace*{6pt}{type}="{defective}">}ex materia quasi et forma sibi\mbox{}\newline 
 proporti{<\textbf{gap}/>}\mbox{}\newline 
{</\textbf{explicit}>}\mbox{}\newline 
{<\textbf{explicit}\hspace*{6pt}{type}="{reverse}">}saued be shulle that doome of day the at\mbox{}\newline 
{</\textbf{explicit}>}\end{shaded}\egroup 


    \item[{Exemple}]
  \leavevmode\bgroup\exampleFont \begin{shaded}\noindent\mbox{}{<\textbf{explicit}>}sed libera nos a malo.{</\textbf{explicit}>}\mbox{}\newline 
{<\textbf{rubric}>}Hic explicit oratio qui dicitur dominica.{</\textbf{rubric}>}\mbox{}\newline 
{<\textbf{explicit}\hspace*{6pt}{type}="{defective}">}ex materia quasi et forma sibi proporti{<\textbf{gap}/>}\mbox{}\newline 
{</\textbf{explicit}>}\mbox{}\newline 
{<\textbf{explicit}\hspace*{6pt}{type}="{reverse}">}saued be shulle that doome of day the at{</\textbf{explicit}>}\end{shaded}\egroup 


    \item[{Modèle de contenu}]
  \mbox{}\hfill\\[-10pt]\begin{Verbatim}[fontsize=\small]
<content>
 <macroRef key="macro.phraseSeq"/>
</content>
    
\end{Verbatim}

    \item[{Schéma Declaration}]
  \mbox{}\hfill\\[-10pt]\begin{Verbatim}[fontsize=\small]
element explicit
{
   tei_att.global.attributes,
   tei_att.typed.attributes,
   tei_att.msExcerpt.attributes,
   tei_macro.phraseSeq}
\end{Verbatim}

\end{reflist}  \index{extent=<extent>|oddindex}
\begin{reflist}
\item[]\begin{specHead}{TEI.extent}{<extent> }(étendue) décrit la taille approximative d’un texte stocké sur son support, numérique ou non numérique, exprimé dans une unité quelconque appropriée. [\xref{http://www.tei-c.org/release/doc/tei-p5-doc/en/html/HD.html\#HD23}{2.2.3. Type and Extent of File} \xref{http://www.tei-c.org/release/doc/tei-p5-doc/en/html/HD.html\#HD2}{2.2. The File Description} \xref{http://www.tei-c.org/release/doc/tei-p5-doc/en/html/CO.html\#COBICOI}{3.11.2.4. Imprint, Size of a Document, and Reprint Information} \xref{http://www.tei-c.org/release/doc/tei-p5-doc/en/html/MS.html\#msph1}{10.7.1. Object Description}]\end{specHead} 
    \item[{Module}]
  header
    \item[{Attributs}]
  Attributs \hyperref[TEI.att.global]{att.global} (\textit{@xml:id}, \textit{@n}, \textit{@xml:lang}, \textit{@xml:base}, \textit{@xml:space})  (\hyperref[TEI.att.global.rendition]{att.global.rendition} (\textit{@rend}, \textit{@style}, \textit{@rendition})) (\hyperref[TEI.att.global.linking]{att.global.linking} (\textit{@corresp}, \textit{@synch}, \textit{@sameAs}, \textit{@copyOf}, \textit{@next}, \textit{@prev}, \textit{@exclude}, \textit{@select})) (\hyperref[TEI.att.global.analytic]{att.global.analytic} (\textit{@ana})) (\hyperref[TEI.att.global.facs]{att.global.facs} (\textit{@facs})) (\hyperref[TEI.att.global.change]{att.global.change} (\textit{@change})) (\hyperref[TEI.att.global.responsibility]{att.global.responsibility} (\textit{@cert}, \textit{@resp})) (\hyperref[TEI.att.global.source]{att.global.source} (\textit{@source}))
    \item[{Membre du}]
  \hyperref[TEI.model.biblPart]{model.biblPart} 
    \item[{Contenu dans}]
  
    \item[core: ]
   \hyperref[TEI.bibl]{bibl} \hyperref[TEI.monogr]{monogr}\par 
    \item[header: ]
   \hyperref[TEI.biblFull]{biblFull} \hyperref[TEI.fileDesc]{fileDesc}\par 
    \item[msdescription: ]
   \hyperref[TEI.supportDesc]{supportDesc}
    \item[{Peut contenir}]
  
    \item[analysis: ]
   \hyperref[TEI.c]{c} \hyperref[TEI.cl]{cl} \hyperref[TEI.interp]{interp} \hyperref[TEI.interpGrp]{interpGrp} \hyperref[TEI.m]{m} \hyperref[TEI.pc]{pc} \hyperref[TEI.phr]{phr} \hyperref[TEI.s]{s} \hyperref[TEI.span]{span} \hyperref[TEI.spanGrp]{spanGrp} \hyperref[TEI.w]{w}\par 
    \item[core: ]
   \hyperref[TEI.abbr]{abbr} \hyperref[TEI.add]{add} \hyperref[TEI.address]{address} \hyperref[TEI.binaryObject]{binaryObject} \hyperref[TEI.cb]{cb} \hyperref[TEI.choice]{choice} \hyperref[TEI.corr]{corr} \hyperref[TEI.date]{date} \hyperref[TEI.del]{del} \hyperref[TEI.distinct]{distinct} \hyperref[TEI.email]{email} \hyperref[TEI.emph]{emph} \hyperref[TEI.expan]{expan} \hyperref[TEI.foreign]{foreign} \hyperref[TEI.gap]{gap} \hyperref[TEI.gb]{gb} \hyperref[TEI.gloss]{gloss} \hyperref[TEI.graphic]{graphic} \hyperref[TEI.hi]{hi} \hyperref[TEI.index]{index} \hyperref[TEI.lb]{lb} \hyperref[TEI.measure]{measure} \hyperref[TEI.measureGrp]{measureGrp} \hyperref[TEI.media]{media} \hyperref[TEI.mentioned]{mentioned} \hyperref[TEI.milestone]{milestone} \hyperref[TEI.name]{name} \hyperref[TEI.note]{note} \hyperref[TEI.num]{num} \hyperref[TEI.orig]{orig} \hyperref[TEI.pb]{pb} \hyperref[TEI.ptr]{ptr} \hyperref[TEI.ref]{ref} \hyperref[TEI.reg]{reg} \hyperref[TEI.rs]{rs} \hyperref[TEI.sic]{sic} \hyperref[TEI.soCalled]{soCalled} \hyperref[TEI.term]{term} \hyperref[TEI.time]{time} \hyperref[TEI.title]{title} \hyperref[TEI.unclear]{unclear}\par 
    \item[derived-module-tei.istex: ]
   \hyperref[TEI.math]{math} \hyperref[TEI.mrow]{mrow}\par 
    \item[figures: ]
   \hyperref[TEI.figure]{figure} \hyperref[TEI.formula]{formula} \hyperref[TEI.notatedMusic]{notatedMusic}\par 
    \item[header: ]
   \hyperref[TEI.idno]{idno}\par 
    \item[iso-fs: ]
   \hyperref[TEI.fLib]{fLib} \hyperref[TEI.fs]{fs} \hyperref[TEI.fvLib]{fvLib}\par 
    \item[linking: ]
   \hyperref[TEI.alt]{alt} \hyperref[TEI.altGrp]{altGrp} \hyperref[TEI.anchor]{anchor} \hyperref[TEI.join]{join} \hyperref[TEI.joinGrp]{joinGrp} \hyperref[TEI.link]{link} \hyperref[TEI.linkGrp]{linkGrp} \hyperref[TEI.seg]{seg} \hyperref[TEI.timeline]{timeline}\par 
    \item[msdescription: ]
   \hyperref[TEI.catchwords]{catchwords} \hyperref[TEI.depth]{depth} \hyperref[TEI.dim]{dim} \hyperref[TEI.dimensions]{dimensions} \hyperref[TEI.height]{height} \hyperref[TEI.heraldry]{heraldry} \hyperref[TEI.locus]{locus} \hyperref[TEI.locusGrp]{locusGrp} \hyperref[TEI.material]{material} \hyperref[TEI.objectType]{objectType} \hyperref[TEI.origDate]{origDate} \hyperref[TEI.origPlace]{origPlace} \hyperref[TEI.secFol]{secFol} \hyperref[TEI.signatures]{signatures} \hyperref[TEI.source]{source} \hyperref[TEI.stamp]{stamp} \hyperref[TEI.watermark]{watermark} \hyperref[TEI.width]{width}\par 
    \item[namesdates: ]
   \hyperref[TEI.addName]{addName} \hyperref[TEI.affiliation]{affiliation} \hyperref[TEI.country]{country} \hyperref[TEI.forename]{forename} \hyperref[TEI.genName]{genName} \hyperref[TEI.geogName]{geogName} \hyperref[TEI.location]{location} \hyperref[TEI.nameLink]{nameLink} \hyperref[TEI.orgName]{orgName} \hyperref[TEI.persName]{persName} \hyperref[TEI.placeName]{placeName} \hyperref[TEI.region]{region} \hyperref[TEI.roleName]{roleName} \hyperref[TEI.settlement]{settlement} \hyperref[TEI.state]{state} \hyperref[TEI.surname]{surname}\par 
    \item[spoken: ]
   \hyperref[TEI.annotationBlock]{annotationBlock}\par 
    \item[transcr: ]
   \hyperref[TEI.addSpan]{addSpan} \hyperref[TEI.am]{am} \hyperref[TEI.damage]{damage} \hyperref[TEI.damageSpan]{damageSpan} \hyperref[TEI.delSpan]{delSpan} \hyperref[TEI.ex]{ex} \hyperref[TEI.fw]{fw} \hyperref[TEI.handShift]{handShift} \hyperref[TEI.listTranspose]{listTranspose} \hyperref[TEI.metamark]{metamark} \hyperref[TEI.mod]{mod} \hyperref[TEI.redo]{redo} \hyperref[TEI.restore]{restore} \hyperref[TEI.retrace]{retrace} \hyperref[TEI.secl]{secl} \hyperref[TEI.space]{space} \hyperref[TEI.subst]{subst} \hyperref[TEI.substJoin]{substJoin} \hyperref[TEI.supplied]{supplied} \hyperref[TEI.surplus]{surplus} \hyperref[TEI.undo]{undo}\par des données textuelles
    \item[{Exemple}]
  \leavevmode\bgroup\exampleFont \begin{shaded}\noindent\mbox{}{<\textbf{extent}>}198 pages{</\textbf{extent}>}\mbox{}\newline 
{<\textbf{extent}>}90 195 mots{</\textbf{extent}>}\mbox{}\newline 
{<\textbf{extent}>}1 Mo{</\textbf{extent}>}\end{shaded}\egroup 


    \item[{Modèle de contenu}]
  \mbox{}\hfill\\[-10pt]\begin{Verbatim}[fontsize=\small]
<content>
 <macroRef key="macro.phraseSeq"/>
</content>
    
\end{Verbatim}

    \item[{Schéma Declaration}]
  \mbox{}\hfill\\[-10pt]\begin{Verbatim}[fontsize=\small]
element extent { tei_att.global.attributes, tei_macro.phraseSeq }
\end{Verbatim}

\end{reflist}  \index{f=<f>|oddindex}\index{name=@name!<f>|oddindex}\index{fVal=@fVal!<f>|oddindex}
\begin{reflist}
\item[]\begin{specHead}{TEI.f}{<f> }(trait) représente une \textit{spécification trait-valeur}, c'est-à-dire l'association d'un nom avec une valeur d’un type quelconque parmi plusieurs. [\xref{http://www.tei-c.org/release/doc/tei-p5-doc/en/html/FS.html\#FSBI}{18.2. Elementary Feature Structures and the Binary Feature Value}]\end{specHead} 
    \item[{Module}]
  iso-fs
    \item[{Attributs}]
  Attributs \hyperref[TEI.att.global]{att.global} (\textit{@xml:id}, \textit{@n}, \textit{@xml:lang}, \textit{@xml:base}, \textit{@xml:space})  (\hyperref[TEI.att.global.rendition]{att.global.rendition} (\textit{@rend}, \textit{@style}, \textit{@rendition})) (\hyperref[TEI.att.global.linking]{att.global.linking} (\textit{@corresp}, \textit{@synch}, \textit{@sameAs}, \textit{@copyOf}, \textit{@next}, \textit{@prev}, \textit{@exclude}, \textit{@select})) (\hyperref[TEI.att.global.analytic]{att.global.analytic} (\textit{@ana})) (\hyperref[TEI.att.global.facs]{att.global.facs} (\textit{@facs})) (\hyperref[TEI.att.global.change]{att.global.change} (\textit{@change})) (\hyperref[TEI.att.global.responsibility]{att.global.responsibility} (\textit{@cert}, \textit{@resp})) (\hyperref[TEI.att.global.source]{att.global.source} (\textit{@source})) \hyperref[TEI.att.datcat]{att.datcat} (\textit{@datcat}, \textit{@valueDatcat}) \hfil\\[-10pt]\begin{sansreflist}
    \item[@name]
  donne un nom pour le trait
\begin{reflist}
    \item[{Statut}]
  Requis
    \item[{Type de données}]
  \hyperref[TEI.teidata.name]{teidata.name}
\end{reflist}  
    \item[@fVal]
  (valeur de traits) référence n'importe quel élément pouvant être utilisé pour représenter la valeur d'un trait.
\begin{reflist}
    \item[{Statut}]
  Optionel
    \item[{Type de données}]
  \hyperref[TEI.teidata.pointer]{teidata.pointer}
    \item[{Note}]
  \par
Si cet attribut est fourni en plus d'un contenu, la valeur référencée doit être unifiée avec ce contenu.
\end{reflist}  
\end{sansreflist}  
    \item[{Contenu dans}]
  
    \item[iso-fs: ]
   \hyperref[TEI.bicond]{bicond} \hyperref[TEI.cond]{cond} \hyperref[TEI.fLib]{fLib} \hyperref[TEI.fs]{fs} \hyperref[TEI.if]{if}
    \item[{Peut contenir}]
  
    \item[iso-fs: ]
   \hyperref[TEI.binary]{binary} \hyperref[TEI.default]{default} \hyperref[TEI.fs]{fs} \hyperref[TEI.numeric]{numeric} \hyperref[TEI.string]{string} \hyperref[TEI.symbol]{symbol} \hyperref[TEI.vAlt]{vAlt} \hyperref[TEI.vColl]{vColl} \hyperref[TEI.vLabel]{vLabel} \hyperref[TEI.vMerge]{vMerge} \hyperref[TEI.vNot]{vNot}\par des données textuelles
    \item[{Note}]
  \par
Si l'élément est vide, une valeur doit être fournie pour l'attribut {\itshape fVal}.
    \item[{Exemple}]
  l'élément \hyperref[TEI.f]{<f>}contient\leavevmode\bgroup\exampleFont \begin{shaded}\noindent\mbox{}{<\textbf{f}\hspace*{6pt}{name}="{frequency}">}\mbox{}\newline 
\hspace*{6pt}{<\textbf{numeric}\hspace*{6pt}{value}="{2}"/>}\mbox{}\newline 
{</\textbf{f}>}\end{shaded}\egroup 


    \item[{Modèle de contenu}]
  \mbox{}\hfill\\[-10pt]\begin{Verbatim}[fontsize=\small]
<content>
 <alternate maxOccurs="1" minOccurs="1">
  <textNode/>
  <classRef key="model.featureVal"/>
 </alternate>
</content>
    
\end{Verbatim}

    \item[{Schéma Declaration}]
  \mbox{}\hfill\\[-10pt]\begin{Verbatim}[fontsize=\small]
element f
{
   tei_att.global.attributes,
   tei_att.datcat.attributes,
   attribute name { text },
   attribute fVal { text }?,
   ( text | tei_model.featureVal )
}
\end{Verbatim}

\end{reflist}  \index{fDecl=<fDecl>|oddindex}\index{name=@name!<fDecl>|oddindex}\index{optional=@optional!<fDecl>|oddindex}
\begin{reflist}
\item[]\begin{specHead}{TEI.fDecl}{<fDecl> }(déclaration de trait) déclare un trait unique, en en précisant le nom, l'organisation, la liste de valeurs autorisées et, éventuellement, la valeur par défaut. [\xref{http://www.tei-c.org/release/doc/tei-p5-doc/en/html/FS.html\#FD}{18.11. Feature System Declaration}]\end{specHead} 
    \item[{Module}]
  iso-fs
    \item[{Attributs}]
  Attributs \hyperref[TEI.att.global]{att.global} (\textit{@xml:id}, \textit{@n}, \textit{@xml:lang}, \textit{@xml:base}, \textit{@xml:space})  (\hyperref[TEI.att.global.rendition]{att.global.rendition} (\textit{@rend}, \textit{@style}, \textit{@rendition})) (\hyperref[TEI.att.global.linking]{att.global.linking} (\textit{@corresp}, \textit{@synch}, \textit{@sameAs}, \textit{@copyOf}, \textit{@next}, \textit{@prev}, \textit{@exclude}, \textit{@select})) (\hyperref[TEI.att.global.analytic]{att.global.analytic} (\textit{@ana})) (\hyperref[TEI.att.global.facs]{att.global.facs} (\textit{@facs})) (\hyperref[TEI.att.global.change]{att.global.change} (\textit{@change})) (\hyperref[TEI.att.global.responsibility]{att.global.responsibility} (\textit{@cert}, \textit{@resp})) (\hyperref[TEI.att.global.source]{att.global.source} (\textit{@source})) \hfil\\[-10pt]\begin{sansreflist}
    \item[@name]
  indique le nom du trait déclaré ; correspond à l'attribut {\itshape name} des éléments \hyperref[TEI.f]{<f>} du texte.
\begin{reflist}
    \item[{Statut}]
  Requis
    \item[{Type de données}]
  \hyperref[TEI.teidata.name]{teidata.name}
\end{reflist}  
    \item[@optional]
  indique si la valeur de ce trait peut ou non exister
\begin{reflist}
    \item[{Statut}]
  Optionel
    \item[{Type de données}]
  \hyperref[TEI.teidata.truthValue]{teidata.truthValue}
    \item[{Valeur par défaut}]
  true
    \item[{Note}]
  \par
Si un trait est indiqué comme facultatif, il est possible de l'omettre d'une structure de traits. Si un trait obligatoire est omis, il est alors réputé avoir une valeur par défaut, déclarée explicitement, ou, si aucune valeur par défaut n'est fournie, la valeur spéciale any. Si un trait facultatif est omis, il est réputé manquant et aucune valeur possible n'est prise en compte (y compris celle par défaut).
\end{reflist}  
\end{sansreflist}  
    \item[{Contenu dans}]
  
    \item[iso-fs: ]
   \hyperref[TEI.fsDecl]{fsDecl}
    \item[{Peut contenir}]
  
    \item[iso-fs: ]
   \hyperref[TEI.fDescr]{fDescr} \hyperref[TEI.vDefault]{vDefault} \hyperref[TEI.vRange]{vRange}
    \item[{Exemple}]
  \leavevmode\bgroup\exampleFont \begin{shaded}\noindent\mbox{}{<\textbf{fDecl}\hspace*{6pt}{name}="{INV}">}\mbox{}\newline 
\hspace*{6pt}{<\textbf{fDescr}>}inverted sentence{</\textbf{fDescr}>}\mbox{}\newline 
\hspace*{6pt}{<\textbf{vRange}>}\mbox{}\newline 
\hspace*{6pt}\hspace*{6pt}{<\textbf{vAlt}>}\mbox{}\newline 
\hspace*{6pt}\hspace*{6pt}\hspace*{6pt}{<\textbf{binary}\hspace*{6pt}{value}="{true}"/>}\mbox{}\newline 
\hspace*{6pt}\hspace*{6pt}\hspace*{6pt}{<\textbf{binary}\hspace*{6pt}{value}="{false}"/>}\mbox{}\newline 
\hspace*{6pt}\hspace*{6pt}{</\textbf{vAlt}>}\mbox{}\newline 
\hspace*{6pt}{</\textbf{vRange}>}\mbox{}\newline 
\hspace*{6pt}{<\textbf{vDefault}>}\mbox{}\newline 
\hspace*{6pt}\hspace*{6pt}{<\textbf{binary}\hspace*{6pt}{value}="{false}"/>}\mbox{}\newline 
\hspace*{6pt}{</\textbf{vDefault}>}\mbox{}\newline 
{</\textbf{fDecl}>}\end{shaded}\egroup 


    \item[{Modèle de contenu}]
  \mbox{}\hfill\\[-10pt]\begin{Verbatim}[fontsize=\small]
<content>
 <sequence maxOccurs="1" minOccurs="1">
  <elementRef key="fDescr" minOccurs="0"/>
  <elementRef key="vRange"/>
  <elementRef key="vDefault" minOccurs="0"/>
 </sequence>
</content>
    
\end{Verbatim}

    \item[{Schéma Declaration}]
  \mbox{}\hfill\\[-10pt]\begin{Verbatim}[fontsize=\small]
element fDecl
{
   tei_att.global.attributes,
   attribute name { text },
   attribute optional { text }?,
   ( tei_fDescr?, tei_vRange, tei_vDefault? )
}
\end{Verbatim}

\end{reflist}  \index{fDescr=<fDescr>|oddindex}
\begin{reflist}
\item[]\begin{specHead}{TEI.fDescr}{<fDescr> }(description de trait (dans FSD)) décrit en texte libre le trait déclaré et ses valeurs [\xref{http://www.tei-c.org/release/doc/tei-p5-doc/en/html/FS.html\#FD}{18.11. Feature System Declaration}]\end{specHead} 
    \item[{Module}]
  iso-fs
    \item[{Attributs}]
  Attributs \hyperref[TEI.att.global]{att.global} (\textit{@xml:id}, \textit{@n}, \textit{@xml:lang}, \textit{@xml:base}, \textit{@xml:space})  (\hyperref[TEI.att.global.rendition]{att.global.rendition} (\textit{@rend}, \textit{@style}, \textit{@rendition})) (\hyperref[TEI.att.global.linking]{att.global.linking} (\textit{@corresp}, \textit{@synch}, \textit{@sameAs}, \textit{@copyOf}, \textit{@next}, \textit{@prev}, \textit{@exclude}, \textit{@select})) (\hyperref[TEI.att.global.analytic]{att.global.analytic} (\textit{@ana})) (\hyperref[TEI.att.global.facs]{att.global.facs} (\textit{@facs})) (\hyperref[TEI.att.global.change]{att.global.change} (\textit{@change})) (\hyperref[TEI.att.global.responsibility]{att.global.responsibility} (\textit{@cert}, \textit{@resp})) (\hyperref[TEI.att.global.source]{att.global.source} (\textit{@source}))
    \item[{Contenu dans}]
  
    \item[iso-fs: ]
   \hyperref[TEI.fDecl]{fDecl}
    \item[{Peut contenir}]
  
    \item[core: ]
   \hyperref[TEI.abbr]{abbr} \hyperref[TEI.address]{address} \hyperref[TEI.bibl]{bibl} \hyperref[TEI.biblStruct]{biblStruct} \hyperref[TEI.choice]{choice} \hyperref[TEI.cit]{cit} \hyperref[TEI.date]{date} \hyperref[TEI.desc]{desc} \hyperref[TEI.distinct]{distinct} \hyperref[TEI.email]{email} \hyperref[TEI.emph]{emph} \hyperref[TEI.expan]{expan} \hyperref[TEI.foreign]{foreign} \hyperref[TEI.gloss]{gloss} \hyperref[TEI.hi]{hi} \hyperref[TEI.label]{label} \hyperref[TEI.list]{list} \hyperref[TEI.listBibl]{listBibl} \hyperref[TEI.measure]{measure} \hyperref[TEI.measureGrp]{measureGrp} \hyperref[TEI.mentioned]{mentioned} \hyperref[TEI.name]{name} \hyperref[TEI.num]{num} \hyperref[TEI.ptr]{ptr} \hyperref[TEI.q]{q} \hyperref[TEI.quote]{quote} \hyperref[TEI.ref]{ref} \hyperref[TEI.rs]{rs} \hyperref[TEI.said]{said} \hyperref[TEI.soCalled]{soCalled} \hyperref[TEI.stage]{stage} \hyperref[TEI.term]{term} \hyperref[TEI.time]{time} \hyperref[TEI.title]{title}\par 
    \item[figures: ]
   \hyperref[TEI.table]{table}\par 
    \item[header: ]
   \hyperref[TEI.biblFull]{biblFull} \hyperref[TEI.idno]{idno}\par 
    \item[msdescription: ]
   \hyperref[TEI.catchwords]{catchwords} \hyperref[TEI.depth]{depth} \hyperref[TEI.dim]{dim} \hyperref[TEI.dimensions]{dimensions} \hyperref[TEI.height]{height} \hyperref[TEI.heraldry]{heraldry} \hyperref[TEI.locus]{locus} \hyperref[TEI.locusGrp]{locusGrp} \hyperref[TEI.material]{material} \hyperref[TEI.msDesc]{msDesc} \hyperref[TEI.objectType]{objectType} \hyperref[TEI.origDate]{origDate} \hyperref[TEI.origPlace]{origPlace} \hyperref[TEI.secFol]{secFol} \hyperref[TEI.signatures]{signatures} \hyperref[TEI.stamp]{stamp} \hyperref[TEI.watermark]{watermark} \hyperref[TEI.width]{width}\par 
    \item[namesdates: ]
   \hyperref[TEI.addName]{addName} \hyperref[TEI.affiliation]{affiliation} \hyperref[TEI.country]{country} \hyperref[TEI.forename]{forename} \hyperref[TEI.genName]{genName} \hyperref[TEI.geogName]{geogName} \hyperref[TEI.listOrg]{listOrg} \hyperref[TEI.listPlace]{listPlace} \hyperref[TEI.location]{location} \hyperref[TEI.nameLink]{nameLink} \hyperref[TEI.orgName]{orgName} \hyperref[TEI.persName]{persName} \hyperref[TEI.placeName]{placeName} \hyperref[TEI.region]{region} \hyperref[TEI.roleName]{roleName} \hyperref[TEI.settlement]{settlement} \hyperref[TEI.state]{state} \hyperref[TEI.surname]{surname}\par 
    \item[textstructure: ]
   \hyperref[TEI.floatingText]{floatingText}\par 
    \item[transcr: ]
   \hyperref[TEI.am]{am} \hyperref[TEI.ex]{ex} \hyperref[TEI.subst]{subst}\par des données textuelles
    \item[{Note}]
  \par
Peut contenir des caractères, des éléments de niveau expression ou de niveau intermédiaire.
    \item[{Exemple}]
  \leavevmode\bgroup\exampleFont \begin{shaded}\noindent\mbox{}{<\textbf{fDecl}\hspace*{6pt}{name}="{INV}">}\mbox{}\newline 
\hspace*{6pt}{<\textbf{fDescr}>}inverted sentence{</\textbf{fDescr}>}\mbox{}\newline 
\hspace*{6pt}{<\textbf{vRange}>}\mbox{}\newline 
\hspace*{6pt}\hspace*{6pt}{<\textbf{vAlt}>}\mbox{}\newline 
\hspace*{6pt}\hspace*{6pt}\hspace*{6pt}{<\textbf{binary}\hspace*{6pt}{value}="{true}"/>}\mbox{}\newline 
\hspace*{6pt}\hspace*{6pt}\hspace*{6pt}{<\textbf{binary}\hspace*{6pt}{value}="{false}"/>}\mbox{}\newline 
\hspace*{6pt}\hspace*{6pt}{</\textbf{vAlt}>}\mbox{}\newline 
\hspace*{6pt}{</\textbf{vRange}>}\mbox{}\newline 
\hspace*{6pt}{<\textbf{vDefault}>}\mbox{}\newline 
\hspace*{6pt}\hspace*{6pt}{<\textbf{binary}\hspace*{6pt}{value}="{false}"/>}\mbox{}\newline 
\hspace*{6pt}{</\textbf{vDefault}>}\mbox{}\newline 
{</\textbf{fDecl}>}\end{shaded}\egroup 


    \item[{Modèle de contenu}]
  \mbox{}\hfill\\[-10pt]\begin{Verbatim}[fontsize=\small]
<content>
 <macroRef key="macro.limitedContent"/>
</content>
    
\end{Verbatim}

    \item[{Schéma Declaration}]
  \mbox{}\hfill\\[-10pt]\begin{Verbatim}[fontsize=\small]
element fDescr { tei_att.global.attributes, tei_macro.limitedContent }
\end{Verbatim}

\end{reflist}  \index{fLib=<fLib>|oddindex}
\begin{reflist}
\item[]\begin{specHead}{TEI.fLib}{<fLib> }(bibliothèque de traits) rassemble une bibliothèque de traits [\xref{http://www.tei-c.org/release/doc/tei-p5-doc/en/html/FS.html\#FSFL}{18.4. Feature Libraries and Feature-Value Libraries}]\end{specHead} 
    \item[{Module}]
  iso-fs
    \item[{Attributs}]
  Attributs \hyperref[TEI.att.global]{att.global} (\textit{@xml:id}, \textit{@n}, \textit{@xml:lang}, \textit{@xml:base}, \textit{@xml:space})  (\hyperref[TEI.att.global.rendition]{att.global.rendition} (\textit{@rend}, \textit{@style}, \textit{@rendition})) (\hyperref[TEI.att.global.linking]{att.global.linking} (\textit{@corresp}, \textit{@synch}, \textit{@sameAs}, \textit{@copyOf}, \textit{@next}, \textit{@prev}, \textit{@exclude}, \textit{@select})) (\hyperref[TEI.att.global.analytic]{att.global.analytic} (\textit{@ana})) (\hyperref[TEI.att.global.facs]{att.global.facs} (\textit{@facs})) (\hyperref[TEI.att.global.change]{att.global.change} (\textit{@change})) (\hyperref[TEI.att.global.responsibility]{att.global.responsibility} (\textit{@cert}, \textit{@resp})) (\hyperref[TEI.att.global.source]{att.global.source} (\textit{@source}))
    \item[{Membre du}]
  \hyperref[TEI.model.global.meta]{model.global.meta}
    \item[{Contenu dans}]
  
    \item[analysis: ]
   \hyperref[TEI.cl]{cl} \hyperref[TEI.m]{m} \hyperref[TEI.phr]{phr} \hyperref[TEI.s]{s} \hyperref[TEI.span]{span} \hyperref[TEI.w]{w}\par 
    \item[core: ]
   \hyperref[TEI.abbr]{abbr} \hyperref[TEI.add]{add} \hyperref[TEI.addrLine]{addrLine} \hyperref[TEI.address]{address} \hyperref[TEI.author]{author} \hyperref[TEI.bibl]{bibl} \hyperref[TEI.biblScope]{biblScope} \hyperref[TEI.cit]{cit} \hyperref[TEI.citedRange]{citedRange} \hyperref[TEI.corr]{corr} \hyperref[TEI.date]{date} \hyperref[TEI.del]{del} \hyperref[TEI.distinct]{distinct} \hyperref[TEI.editor]{editor} \hyperref[TEI.email]{email} \hyperref[TEI.emph]{emph} \hyperref[TEI.expan]{expan} \hyperref[TEI.foreign]{foreign} \hyperref[TEI.gloss]{gloss} \hyperref[TEI.head]{head} \hyperref[TEI.headItem]{headItem} \hyperref[TEI.headLabel]{headLabel} \hyperref[TEI.hi]{hi} \hyperref[TEI.imprint]{imprint} \hyperref[TEI.item]{item} \hyperref[TEI.l]{l} \hyperref[TEI.label]{label} \hyperref[TEI.lg]{lg} \hyperref[TEI.list]{list} \hyperref[TEI.measure]{measure} \hyperref[TEI.mentioned]{mentioned} \hyperref[TEI.name]{name} \hyperref[TEI.note]{note} \hyperref[TEI.num]{num} \hyperref[TEI.orig]{orig} \hyperref[TEI.p]{p} \hyperref[TEI.pubPlace]{pubPlace} \hyperref[TEI.publisher]{publisher} \hyperref[TEI.q]{q} \hyperref[TEI.quote]{quote} \hyperref[TEI.ref]{ref} \hyperref[TEI.reg]{reg} \hyperref[TEI.resp]{resp} \hyperref[TEI.rs]{rs} \hyperref[TEI.said]{said} \hyperref[TEI.series]{series} \hyperref[TEI.sic]{sic} \hyperref[TEI.soCalled]{soCalled} \hyperref[TEI.sp]{sp} \hyperref[TEI.speaker]{speaker} \hyperref[TEI.stage]{stage} \hyperref[TEI.street]{street} \hyperref[TEI.term]{term} \hyperref[TEI.textLang]{textLang} \hyperref[TEI.time]{time} \hyperref[TEI.title]{title} \hyperref[TEI.unclear]{unclear}\par 
    \item[figures: ]
   \hyperref[TEI.cell]{cell} \hyperref[TEI.figure]{figure} \hyperref[TEI.table]{table}\par 
    \item[header: ]
   \hyperref[TEI.authority]{authority} \hyperref[TEI.change]{change} \hyperref[TEI.classCode]{classCode} \hyperref[TEI.distributor]{distributor} \hyperref[TEI.edition]{edition} \hyperref[TEI.extent]{extent} \hyperref[TEI.funder]{funder} \hyperref[TEI.language]{language} \hyperref[TEI.licence]{licence}\par 
    \item[linking: ]
   \hyperref[TEI.ab]{ab} \hyperref[TEI.seg]{seg}\par 
    \item[msdescription: ]
   \hyperref[TEI.accMat]{accMat} \hyperref[TEI.acquisition]{acquisition} \hyperref[TEI.additions]{additions} \hyperref[TEI.catchwords]{catchwords} \hyperref[TEI.collation]{collation} \hyperref[TEI.colophon]{colophon} \hyperref[TEI.condition]{condition} \hyperref[TEI.custEvent]{custEvent} \hyperref[TEI.decoNote]{decoNote} \hyperref[TEI.explicit]{explicit} \hyperref[TEI.filiation]{filiation} \hyperref[TEI.finalRubric]{finalRubric} \hyperref[TEI.foliation]{foliation} \hyperref[TEI.heraldry]{heraldry} \hyperref[TEI.incipit]{incipit} \hyperref[TEI.layout]{layout} \hyperref[TEI.material]{material} \hyperref[TEI.msItem]{msItem} \hyperref[TEI.musicNotation]{musicNotation} \hyperref[TEI.objectType]{objectType} \hyperref[TEI.origDate]{origDate} \hyperref[TEI.origPlace]{origPlace} \hyperref[TEI.origin]{origin} \hyperref[TEI.provenance]{provenance} \hyperref[TEI.rubric]{rubric} \hyperref[TEI.secFol]{secFol} \hyperref[TEI.signatures]{signatures} \hyperref[TEI.source]{source} \hyperref[TEI.stamp]{stamp} \hyperref[TEI.summary]{summary} \hyperref[TEI.support]{support} \hyperref[TEI.surrogates]{surrogates} \hyperref[TEI.typeNote]{typeNote} \hyperref[TEI.watermark]{watermark}\par 
    \item[namesdates: ]
   \hyperref[TEI.addName]{addName} \hyperref[TEI.affiliation]{affiliation} \hyperref[TEI.country]{country} \hyperref[TEI.forename]{forename} \hyperref[TEI.genName]{genName} \hyperref[TEI.geogName]{geogName} \hyperref[TEI.nameLink]{nameLink} \hyperref[TEI.orgName]{orgName} \hyperref[TEI.persName]{persName} \hyperref[TEI.person]{person} \hyperref[TEI.personGrp]{personGrp} \hyperref[TEI.persona]{persona} \hyperref[TEI.placeName]{placeName} \hyperref[TEI.region]{region} \hyperref[TEI.roleName]{roleName} \hyperref[TEI.settlement]{settlement} \hyperref[TEI.surname]{surname}\par 
    \item[spoken: ]
   \hyperref[TEI.annotationBlock]{annotationBlock}\par 
    \item[standOff: ]
   \hyperref[TEI.listAnnotation]{listAnnotation}\par 
    \item[textstructure: ]
   \hyperref[TEI.back]{back} \hyperref[TEI.body]{body} \hyperref[TEI.div]{div} \hyperref[TEI.docAuthor]{docAuthor} \hyperref[TEI.docDate]{docDate} \hyperref[TEI.docEdition]{docEdition} \hyperref[TEI.docTitle]{docTitle} \hyperref[TEI.floatingText]{floatingText} \hyperref[TEI.front]{front} \hyperref[TEI.group]{group} \hyperref[TEI.text]{text} \hyperref[TEI.titlePage]{titlePage} \hyperref[TEI.titlePart]{titlePart}\par 
    \item[transcr: ]
   \hyperref[TEI.damage]{damage} \hyperref[TEI.fw]{fw} \hyperref[TEI.line]{line} \hyperref[TEI.metamark]{metamark} \hyperref[TEI.mod]{mod} \hyperref[TEI.restore]{restore} \hyperref[TEI.retrace]{retrace} \hyperref[TEI.secl]{secl} \hyperref[TEI.sourceDoc]{sourceDoc} \hyperref[TEI.supplied]{supplied} \hyperref[TEI.surface]{surface} \hyperref[TEI.surfaceGrp]{surfaceGrp} \hyperref[TEI.surplus]{surplus} \hyperref[TEI.zone]{zone}
    \item[{Peut contenir}]
  
    \item[iso-fs: ]
   \hyperref[TEI.f]{f}
    \item[{Note}]
  \par
L'attribut global {\itshape n} peut être utilisé pour fournir un nom informel afin de catégoriser les contenus de la bibliothèque.
    \item[{Exemple}]
  \leavevmode\bgroup\exampleFont \begin{shaded}\noindent\mbox{}{<\textbf{fLib}\hspace*{6pt}{n}="{agreement features}">}\mbox{}\newline 
\hspace*{6pt}{<\textbf{f}\hspace*{6pt}{name}="{person}"\hspace*{6pt}{xml:id}="{fr\textunderscore pers1}">}\mbox{}\newline 
\hspace*{6pt}\hspace*{6pt}{<\textbf{symbol}\hspace*{6pt}{value}="{first}"/>}\mbox{}\newline 
\hspace*{6pt}{</\textbf{f}>}\mbox{}\newline 
\hspace*{6pt}{<\textbf{f}\hspace*{6pt}{name}="{person}"\hspace*{6pt}{xml:id}="{fr\textunderscore pers2}">}\mbox{}\newline 
\hspace*{6pt}\hspace*{6pt}{<\textbf{symbol}\hspace*{6pt}{value}="{second}"/>}\mbox{}\newline 
\hspace*{6pt}{</\textbf{f}>}\mbox{}\newline 
\hspace*{6pt}{<\textbf{f}\hspace*{6pt}{name}="{number}"\hspace*{6pt}{xml:id}="{fr\textunderscore nums}">}\mbox{}\newline 
\hspace*{6pt}\hspace*{6pt}{<\textbf{symbol}\hspace*{6pt}{value}="{singular}"/>}\mbox{}\newline 
\hspace*{6pt}{</\textbf{f}>}\mbox{}\newline 
\hspace*{6pt}{<\textbf{f}\hspace*{6pt}{name}="{number}"\hspace*{6pt}{xml:id}="{fr\textunderscore nump}">}\mbox{}\newline 
\hspace*{6pt}\hspace*{6pt}{<\textbf{symbol}\hspace*{6pt}{value}="{plural}"/>}\mbox{}\newline 
\hspace*{6pt}{</\textbf{f}>}\mbox{}\newline 
{</\textbf{fLib}>}\end{shaded}\egroup 


    \item[{Modèle de contenu}]
  \mbox{}\hfill\\[-10pt]\begin{Verbatim}[fontsize=\small]
<content>
 <elementRef key="f" maxOccurs="unbounded"
  minOccurs="1"/>
</content>
    
\end{Verbatim}

    \item[{Schéma Declaration}]
  \mbox{}\hfill\\[-10pt]\begin{Verbatim}[fontsize=\small]
element fLib { tei_att.global.attributes, tei_f+ }
\end{Verbatim}

\end{reflist}  \index{facsimile=<facsimile>|oddindex}
\begin{reflist}
\item[]\begin{specHead}{TEI.facsimile}{<facsimile> }contient une représentation d'une source écrite quelconque sous la forme d'un ensemble d'images plutôt que sous la forme d'un texte transcrit ou encodé. [\xref{http://www.tei-c.org/release/doc/tei-p5-doc/en/html/PH.html\#PHFAX}{11.1. Digital Facsimiles}]\end{specHead} 
    \item[{Module}]
  transcr
    \item[{Attributs}]
  Attributs \hyperref[TEI.att.global]{att.global} (\textit{@xml:id}, \textit{@n}, \textit{@xml:lang}, \textit{@xml:base}, \textit{@xml:space})  (\hyperref[TEI.att.global.rendition]{att.global.rendition} (\textit{@rend}, \textit{@style}, \textit{@rendition})) (\hyperref[TEI.att.global.linking]{att.global.linking} (\textit{@corresp}, \textit{@synch}, \textit{@sameAs}, \textit{@copyOf}, \textit{@next}, \textit{@prev}, \textit{@exclude}, \textit{@select})) (\hyperref[TEI.att.global.analytic]{att.global.analytic} (\textit{@ana})) (\hyperref[TEI.att.global.facs]{att.global.facs} (\textit{@facs})) (\hyperref[TEI.att.global.change]{att.global.change} (\textit{@change})) (\hyperref[TEI.att.global.responsibility]{att.global.responsibility} (\textit{@cert}, \textit{@resp})) (\hyperref[TEI.att.global.source]{att.global.source} (\textit{@source})) \hyperref[TEI.att.declaring]{att.declaring} (\textit{@decls}) 
    \item[{Membre du}]
  \hyperref[TEI.model.resourceLike]{model.resourceLike}
    \item[{Contenu dans}]
  
    \item[core: ]
   \hyperref[TEI.teiCorpus]{teiCorpus}\par 
    \item[standOff: ]
   \hyperref[TEI.standOff]{standOff}\par 
    \item[textstructure: ]
   \hyperref[TEI.TEI]{TEI}
    \item[{Peut contenir}]
  
    \item[core: ]
   \hyperref[TEI.binaryObject]{binaryObject} \hyperref[TEI.graphic]{graphic} \hyperref[TEI.media]{media}\par 
    \item[derived-module-tei.istex: ]
   \hyperref[TEI.math]{math} \hyperref[TEI.mrow]{mrow}\par 
    \item[figures: ]
   \hyperref[TEI.formula]{formula}\par 
    \item[textstructure: ]
   \hyperref[TEI.back]{back} \hyperref[TEI.front]{front}\par 
    \item[transcr: ]
   \hyperref[TEI.surface]{surface} \hyperref[TEI.surfaceGrp]{surfaceGrp}
    \item[{Exemple}]
  \leavevmode\bgroup\exampleFont \begin{shaded}\noindent\mbox{}{<\textbf{facsimile}>}\mbox{}\newline 
\hspace*{6pt}{<\textbf{graphic}\hspace*{6pt}{url}="{page1.png}"/>}\mbox{}\newline 
\hspace*{6pt}{<\textbf{surface}>}\mbox{}\newline 
\hspace*{6pt}\hspace*{6pt}{<\textbf{graphic}\hspace*{6pt}{url}="{page2-highRes.png}"/>}\mbox{}\newline 
\hspace*{6pt}\hspace*{6pt}{<\textbf{graphic}\hspace*{6pt}{url}="{page2-lowRes.png}"/>}\mbox{}\newline 
\hspace*{6pt}{</\textbf{surface}>}\mbox{}\newline 
\hspace*{6pt}{<\textbf{graphic}\hspace*{6pt}{url}="{page3.png}"/>}\mbox{}\newline 
\hspace*{6pt}{<\textbf{graphic}\hspace*{6pt}{url}="{page4.png}"/>}\mbox{}\newline 
{</\textbf{facsimile}>}\end{shaded}\egroup 


    \item[{Exemple}]
  \leavevmode\bgroup\exampleFont \begin{shaded}\noindent\mbox{}{<\textbf{facsimile}>}\mbox{}\newline 
\hspace*{6pt}{<\textbf{surface}\hspace*{6pt}{lrx}="{200}"\hspace*{6pt}{lry}="{300}"\hspace*{6pt}{ulx}="{0}"\hspace*{6pt}{uly}="{0}">}\mbox{}\newline 
\hspace*{6pt}\hspace*{6pt}{<\textbf{graphic}\hspace*{6pt}{url}="{Bovelles-49r.png}"/>}\mbox{}\newline 
\hspace*{6pt}{</\textbf{surface}>}\mbox{}\newline 
{</\textbf{facsimile}>}\end{shaded}\egroup 


    \item[{Modèle de contenu}]
  \mbox{}\hfill\\[-10pt]\begin{Verbatim}[fontsize=\small]
<content>
 <sequence maxOccurs="1" minOccurs="1">
  <elementRef key="front" minOccurs="0"/>
  <alternate maxOccurs="unbounded"
   minOccurs="1">
   <classRef key="model.graphicLike"/>
   <elementRef key="surface"/>
   <elementRef key="surfaceGrp"/>
  </alternate>
  <elementRef key="back" minOccurs="0"/>
 </sequence>
</content>
    
\end{Verbatim}

    \item[{Schéma Declaration}]
  \mbox{}\hfill\\[-10pt]\begin{Verbatim}[fontsize=\small]
element facsimile
{
   tei_att.global.attributes,
   tei_att.declaring.attributes,
   (
      tei_front?,
      ( tei_model.graphicLike | tei_surface | tei_surfaceGrp )+,
      tei_back?
   )
}
\end{Verbatim}

\end{reflist}  \index{figDesc=<figDesc>|oddindex}
\begin{reflist}
\item[]\begin{specHead}{TEI.figDesc}{<figDesc> }(description d'une figure) contient une brève description de l'apparence ou du contenu d'une représentation graphique, pour documenter une image sans avoir à l'afficher [\xref{http://www.tei-c.org/release/doc/tei-p5-doc/en/html/FT.html\#FTGRA}{14.4. Specific Elements for Graphic Images}]\end{specHead} 
    \item[{Module}]
  figures
    \item[{Attributs}]
  Attributs \hyperref[TEI.att.global]{att.global} (\textit{@xml:id}, \textit{@n}, \textit{@xml:lang}, \textit{@xml:base}, \textit{@xml:space})  (\hyperref[TEI.att.global.rendition]{att.global.rendition} (\textit{@rend}, \textit{@style}, \textit{@rendition})) (\hyperref[TEI.att.global.linking]{att.global.linking} (\textit{@corresp}, \textit{@synch}, \textit{@sameAs}, \textit{@copyOf}, \textit{@next}, \textit{@prev}, \textit{@exclude}, \textit{@select})) (\hyperref[TEI.att.global.analytic]{att.global.analytic} (\textit{@ana})) (\hyperref[TEI.att.global.facs]{att.global.facs} (\textit{@facs})) (\hyperref[TEI.att.global.change]{att.global.change} (\textit{@change})) (\hyperref[TEI.att.global.responsibility]{att.global.responsibility} (\textit{@cert}, \textit{@resp})) (\hyperref[TEI.att.global.source]{att.global.source} (\textit{@source}))
    \item[{Contenu dans}]
  
    \item[figures: ]
   \hyperref[TEI.figure]{figure}
    \item[{Peut contenir}]
  
    \item[core: ]
   \hyperref[TEI.abbr]{abbr} \hyperref[TEI.address]{address} \hyperref[TEI.bibl]{bibl} \hyperref[TEI.biblStruct]{biblStruct} \hyperref[TEI.binaryObject]{binaryObject} \hyperref[TEI.choice]{choice} \hyperref[TEI.cit]{cit} \hyperref[TEI.date]{date} \hyperref[TEI.desc]{desc} \hyperref[TEI.distinct]{distinct} \hyperref[TEI.email]{email} \hyperref[TEI.emph]{emph} \hyperref[TEI.expan]{expan} \hyperref[TEI.foreign]{foreign} \hyperref[TEI.gloss]{gloss} \hyperref[TEI.graphic]{graphic} \hyperref[TEI.hi]{hi} \hyperref[TEI.label]{label} \hyperref[TEI.list]{list} \hyperref[TEI.listBibl]{listBibl} \hyperref[TEI.measure]{measure} \hyperref[TEI.measureGrp]{measureGrp} \hyperref[TEI.media]{media} \hyperref[TEI.mentioned]{mentioned} \hyperref[TEI.name]{name} \hyperref[TEI.num]{num} \hyperref[TEI.ptr]{ptr} \hyperref[TEI.q]{q} \hyperref[TEI.quote]{quote} \hyperref[TEI.ref]{ref} \hyperref[TEI.rs]{rs} \hyperref[TEI.said]{said} \hyperref[TEI.soCalled]{soCalled} \hyperref[TEI.stage]{stage} \hyperref[TEI.term]{term} \hyperref[TEI.time]{time} \hyperref[TEI.title]{title}\par 
    \item[derived-module-tei.istex: ]
   \hyperref[TEI.math]{math} \hyperref[TEI.mrow]{mrow}\par 
    \item[figures: ]
   \hyperref[TEI.formula]{formula} \hyperref[TEI.table]{table}\par 
    \item[header: ]
   \hyperref[TEI.biblFull]{biblFull} \hyperref[TEI.idno]{idno}\par 
    \item[msdescription: ]
   \hyperref[TEI.catchwords]{catchwords} \hyperref[TEI.depth]{depth} \hyperref[TEI.dim]{dim} \hyperref[TEI.dimensions]{dimensions} \hyperref[TEI.height]{height} \hyperref[TEI.heraldry]{heraldry} \hyperref[TEI.locus]{locus} \hyperref[TEI.locusGrp]{locusGrp} \hyperref[TEI.material]{material} \hyperref[TEI.msDesc]{msDesc} \hyperref[TEI.objectType]{objectType} \hyperref[TEI.origDate]{origDate} \hyperref[TEI.origPlace]{origPlace} \hyperref[TEI.secFol]{secFol} \hyperref[TEI.signatures]{signatures} \hyperref[TEI.stamp]{stamp} \hyperref[TEI.watermark]{watermark} \hyperref[TEI.width]{width}\par 
    \item[namesdates: ]
   \hyperref[TEI.addName]{addName} \hyperref[TEI.affiliation]{affiliation} \hyperref[TEI.country]{country} \hyperref[TEI.forename]{forename} \hyperref[TEI.genName]{genName} \hyperref[TEI.geogName]{geogName} \hyperref[TEI.listOrg]{listOrg} \hyperref[TEI.listPlace]{listPlace} \hyperref[TEI.location]{location} \hyperref[TEI.nameLink]{nameLink} \hyperref[TEI.orgName]{orgName} \hyperref[TEI.persName]{persName} \hyperref[TEI.placeName]{placeName} \hyperref[TEI.region]{region} \hyperref[TEI.roleName]{roleName} \hyperref[TEI.settlement]{settlement} \hyperref[TEI.state]{state} \hyperref[TEI.surname]{surname}\par 
    \item[textstructure: ]
   \hyperref[TEI.floatingText]{floatingText}\par 
    \item[transcr: ]
   \hyperref[TEI.am]{am} \hyperref[TEI.ex]{ex} \hyperref[TEI.subst]{subst}\par des données textuelles
    \item[{Exemple}]
  Cas où une description de figure contient du mathML\leavevmode\bgroup\exampleFont \begin{shaded}\noindent\mbox{}{<\textbf{figDesc}>}Parameter estimates for {<\textbf{m:math}>}\mbox{}\newline 
\hspace*{6pt}\hspace*{6pt}{<\textbf{m:mrow}>}\mbox{}\newline 
\hspace*{6pt}\hspace*{6pt}\hspace*{6pt}{<\textbf{m:msub}>}\mbox{}\newline 
\hspace*{6pt}\hspace*{6pt}\hspace*{6pt}\hspace*{6pt}{<\textbf{m:mrow}>}\mbox{}\newline 
\hspace*{6pt}\hspace*{6pt}\hspace*{6pt}\hspace*{6pt}\hspace*{6pt}{<\textbf{m:mi}\hspace*{6pt}{mathvariant}="{normal}">}F{</\textbf{m:mi}>}\mbox{}\newline 
\hspace*{6pt}\hspace*{6pt}\hspace*{6pt}\hspace*{6pt}{</\textbf{m:mrow}>}\mbox{}\newline 
\hspace*{6pt}\hspace*{6pt}\hspace*{6pt}\hspace*{6pt}{<\textbf{m:mrow}>}\mbox{}\newline 
\hspace*{6pt}\hspace*{6pt}\hspace*{6pt}\hspace*{6pt}\hspace*{6pt}{<\textbf{m:mi}\hspace*{6pt}{mathvariant}="{normal}">}ST{</\textbf{m:mi}>}\mbox{}\newline 
\hspace*{6pt}\hspace*{6pt}\hspace*{6pt}\hspace*{6pt}{</\textbf{m:mrow}>}\mbox{}\newline 
\hspace*{6pt}\hspace*{6pt}\hspace*{6pt}{</\textbf{m:msub}>}\mbox{}\newline 
\hspace*{6pt}\hspace*{6pt}{</\textbf{m:mrow}>}\mbox{}\newline 
\hspace*{6pt}{</\textbf{m:math}>}\mbox{}\newline 
 (black, top panel), {<\textbf{hi}\hspace*{6pt}{rend}="{italic}">}α{</\textbf{hi}>} (blue, bottom panel), {<\textbf{hi}\hspace*{6pt}{rend}="{italic}">}β{</\textbf{hi}>} (red, bottom\mbox{}\newline 
 panel) by their physical position across one of the largest scaffolds in the\mbox{}\newline 
{<\textbf{hi}\hspace*{6pt}{rend}="{italic}">}M. vitellinus{</\textbf{hi}>} draft genome assembly.{</\textbf{figDesc}>}\end{shaded}\egroup 


    \item[{Modèle de contenu}]
  \mbox{}\hfill\\[-10pt]\begin{Verbatim}[fontsize=\small]
<content>
 <alternate maxOccurs="unbounded"
  minOccurs="1">
  <classRef key="model.graphicLike"/>
  <classRef key="macro.limitedContent"/>
 </alternate>
</content>
    
\end{Verbatim}

    \item[{Schéma Declaration}]
  \mbox{}\hfill\\[-10pt]\begin{Verbatim}[fontsize=\small]
element figDesc
{
   tei_att.global.attributes,
   ( tei_model.graphicLike | tei_macro.limitedContent )+
}
\end{Verbatim}

\end{reflist}  \index{figure=<figure>|oddindex}
\begin{reflist}
\item[]\begin{specHead}{TEI.figure}{<figure> }(figure) regroupe des éléments représentant ou contenant une information graphique comme une illustration ou une figure. [\xref{http://www.tei-c.org/release/doc/tei-p5-doc/en/html/FT.html\#FTGRA}{14.4. Specific Elements for Graphic Images}]\end{specHead} 
    \item[{Module}]
  figures
    \item[{Attributs}]
  Attributs \hyperref[TEI.att.global]{att.global} (\textit{@xml:id}, \textit{@n}, \textit{@xml:lang}, \textit{@xml:base}, \textit{@xml:space})  (\hyperref[TEI.att.global.rendition]{att.global.rendition} (\textit{@rend}, \textit{@style}, \textit{@rendition})) (\hyperref[TEI.att.global.linking]{att.global.linking} (\textit{@corresp}, \textit{@synch}, \textit{@sameAs}, \textit{@copyOf}, \textit{@next}, \textit{@prev}, \textit{@exclude}, \textit{@select})) (\hyperref[TEI.att.global.analytic]{att.global.analytic} (\textit{@ana})) (\hyperref[TEI.att.global.facs]{att.global.facs} (\textit{@facs})) (\hyperref[TEI.att.global.change]{att.global.change} (\textit{@change})) (\hyperref[TEI.att.global.responsibility]{att.global.responsibility} (\textit{@cert}, \textit{@resp})) (\hyperref[TEI.att.global.source]{att.global.source} (\textit{@source})) \hyperref[TEI.att.placement]{att.placement} (\textit{@place}) \hyperref[TEI.att.typed]{att.typed} (\textit{@type}, \textit{@subtype}) 
    \item[{Membre du}]
  \hyperref[TEI.model.global]{model.global}
    \item[{Contenu dans}]
  
    \item[analysis: ]
   \hyperref[TEI.cl]{cl} \hyperref[TEI.m]{m} \hyperref[TEI.phr]{phr} \hyperref[TEI.s]{s} \hyperref[TEI.span]{span} \hyperref[TEI.w]{w}\par 
    \item[core: ]
   \hyperref[TEI.abbr]{abbr} \hyperref[TEI.add]{add} \hyperref[TEI.addrLine]{addrLine} \hyperref[TEI.address]{address} \hyperref[TEI.author]{author} \hyperref[TEI.bibl]{bibl} \hyperref[TEI.biblScope]{biblScope} \hyperref[TEI.cit]{cit} \hyperref[TEI.citedRange]{citedRange} \hyperref[TEI.corr]{corr} \hyperref[TEI.date]{date} \hyperref[TEI.del]{del} \hyperref[TEI.distinct]{distinct} \hyperref[TEI.editor]{editor} \hyperref[TEI.email]{email} \hyperref[TEI.emph]{emph} \hyperref[TEI.expan]{expan} \hyperref[TEI.foreign]{foreign} \hyperref[TEI.gloss]{gloss} \hyperref[TEI.head]{head} \hyperref[TEI.headItem]{headItem} \hyperref[TEI.headLabel]{headLabel} \hyperref[TEI.hi]{hi} \hyperref[TEI.imprint]{imprint} \hyperref[TEI.item]{item} \hyperref[TEI.l]{l} \hyperref[TEI.label]{label} \hyperref[TEI.lg]{lg} \hyperref[TEI.list]{list} \hyperref[TEI.measure]{measure} \hyperref[TEI.mentioned]{mentioned} \hyperref[TEI.name]{name} \hyperref[TEI.note]{note} \hyperref[TEI.num]{num} \hyperref[TEI.orig]{orig} \hyperref[TEI.p]{p} \hyperref[TEI.pubPlace]{pubPlace} \hyperref[TEI.publisher]{publisher} \hyperref[TEI.q]{q} \hyperref[TEI.quote]{quote} \hyperref[TEI.ref]{ref} \hyperref[TEI.reg]{reg} \hyperref[TEI.resp]{resp} \hyperref[TEI.rs]{rs} \hyperref[TEI.said]{said} \hyperref[TEI.series]{series} \hyperref[TEI.sic]{sic} \hyperref[TEI.soCalled]{soCalled} \hyperref[TEI.sp]{sp} \hyperref[TEI.speaker]{speaker} \hyperref[TEI.stage]{stage} \hyperref[TEI.street]{street} \hyperref[TEI.term]{term} \hyperref[TEI.textLang]{textLang} \hyperref[TEI.time]{time} \hyperref[TEI.title]{title} \hyperref[TEI.unclear]{unclear}\par 
    \item[figures: ]
   \hyperref[TEI.cell]{cell} \hyperref[TEI.figure]{figure} \hyperref[TEI.table]{table}\par 
    \item[header: ]
   \hyperref[TEI.authority]{authority} \hyperref[TEI.change]{change} \hyperref[TEI.classCode]{classCode} \hyperref[TEI.distributor]{distributor} \hyperref[TEI.edition]{edition} \hyperref[TEI.extent]{extent} \hyperref[TEI.funder]{funder} \hyperref[TEI.language]{language} \hyperref[TEI.licence]{licence}\par 
    \item[linking: ]
   \hyperref[TEI.ab]{ab} \hyperref[TEI.seg]{seg}\par 
    \item[msdescription: ]
   \hyperref[TEI.accMat]{accMat} \hyperref[TEI.acquisition]{acquisition} \hyperref[TEI.additions]{additions} \hyperref[TEI.catchwords]{catchwords} \hyperref[TEI.collation]{collation} \hyperref[TEI.colophon]{colophon} \hyperref[TEI.condition]{condition} \hyperref[TEI.custEvent]{custEvent} \hyperref[TEI.decoNote]{decoNote} \hyperref[TEI.explicit]{explicit} \hyperref[TEI.filiation]{filiation} \hyperref[TEI.finalRubric]{finalRubric} \hyperref[TEI.foliation]{foliation} \hyperref[TEI.heraldry]{heraldry} \hyperref[TEI.incipit]{incipit} \hyperref[TEI.layout]{layout} \hyperref[TEI.material]{material} \hyperref[TEI.msItem]{msItem} \hyperref[TEI.musicNotation]{musicNotation} \hyperref[TEI.objectType]{objectType} \hyperref[TEI.origDate]{origDate} \hyperref[TEI.origPlace]{origPlace} \hyperref[TEI.origin]{origin} \hyperref[TEI.provenance]{provenance} \hyperref[TEI.rubric]{rubric} \hyperref[TEI.secFol]{secFol} \hyperref[TEI.signatures]{signatures} \hyperref[TEI.source]{source} \hyperref[TEI.stamp]{stamp} \hyperref[TEI.summary]{summary} \hyperref[TEI.support]{support} \hyperref[TEI.surrogates]{surrogates} \hyperref[TEI.typeNote]{typeNote} \hyperref[TEI.watermark]{watermark}\par 
    \item[namesdates: ]
   \hyperref[TEI.addName]{addName} \hyperref[TEI.affiliation]{affiliation} \hyperref[TEI.country]{country} \hyperref[TEI.forename]{forename} \hyperref[TEI.genName]{genName} \hyperref[TEI.geogName]{geogName} \hyperref[TEI.nameLink]{nameLink} \hyperref[TEI.orgName]{orgName} \hyperref[TEI.persName]{persName} \hyperref[TEI.person]{person} \hyperref[TEI.personGrp]{personGrp} \hyperref[TEI.persona]{persona} \hyperref[TEI.placeName]{placeName} \hyperref[TEI.region]{region} \hyperref[TEI.roleName]{roleName} \hyperref[TEI.settlement]{settlement} \hyperref[TEI.surname]{surname}\par 
    \item[textstructure: ]
   \hyperref[TEI.back]{back} \hyperref[TEI.body]{body} \hyperref[TEI.div]{div} \hyperref[TEI.docAuthor]{docAuthor} \hyperref[TEI.docDate]{docDate} \hyperref[TEI.docEdition]{docEdition} \hyperref[TEI.docTitle]{docTitle} \hyperref[TEI.floatingText]{floatingText} \hyperref[TEI.front]{front} \hyperref[TEI.group]{group} \hyperref[TEI.text]{text} \hyperref[TEI.titlePage]{titlePage} \hyperref[TEI.titlePart]{titlePart}\par 
    \item[transcr: ]
   \hyperref[TEI.damage]{damage} \hyperref[TEI.fw]{fw} \hyperref[TEI.line]{line} \hyperref[TEI.metamark]{metamark} \hyperref[TEI.mod]{mod} \hyperref[TEI.restore]{restore} \hyperref[TEI.retrace]{retrace} \hyperref[TEI.secl]{secl} \hyperref[TEI.sourceDoc]{sourceDoc} \hyperref[TEI.supplied]{supplied} \hyperref[TEI.surface]{surface} \hyperref[TEI.surfaceGrp]{surfaceGrp} \hyperref[TEI.surplus]{surplus} \hyperref[TEI.zone]{zone}
    \item[{Peut contenir}]
  
    \item[analysis: ]
   \hyperref[TEI.interp]{interp} \hyperref[TEI.interpGrp]{interpGrp} \hyperref[TEI.span]{span} \hyperref[TEI.spanGrp]{spanGrp}\par 
    \item[core: ]
   \hyperref[TEI.bibl]{bibl} \hyperref[TEI.biblStruct]{biblStruct} \hyperref[TEI.binaryObject]{binaryObject} \hyperref[TEI.cb]{cb} \hyperref[TEI.cit]{cit} \hyperref[TEI.desc]{desc} \hyperref[TEI.gap]{gap} \hyperref[TEI.gb]{gb} \hyperref[TEI.graphic]{graphic} \hyperref[TEI.head]{head} \hyperref[TEI.index]{index} \hyperref[TEI.l]{l} \hyperref[TEI.label]{label} \hyperref[TEI.lb]{lb} \hyperref[TEI.lg]{lg} \hyperref[TEI.list]{list} \hyperref[TEI.listBibl]{listBibl} \hyperref[TEI.media]{media} \hyperref[TEI.meeting]{meeting} \hyperref[TEI.milestone]{milestone} \hyperref[TEI.note]{note} \hyperref[TEI.p]{p} \hyperref[TEI.pb]{pb} \hyperref[TEI.q]{q} \hyperref[TEI.quote]{quote} \hyperref[TEI.said]{said} \hyperref[TEI.sp]{sp} \hyperref[TEI.stage]{stage}\par 
    \item[derived-module-tei.istex: ]
   \hyperref[TEI.math]{math} \hyperref[TEI.mrow]{mrow}\par 
    \item[figures: ]
   \hyperref[TEI.figDesc]{figDesc} \hyperref[TEI.figure]{figure} \hyperref[TEI.formula]{formula} \hyperref[TEI.notatedMusic]{notatedMusic} \hyperref[TEI.table]{table}\par 
    \item[header: ]
   \hyperref[TEI.biblFull]{biblFull}\par 
    \item[iso-fs: ]
   \hyperref[TEI.fLib]{fLib} \hyperref[TEI.fs]{fs} \hyperref[TEI.fvLib]{fvLib}\par 
    \item[linking: ]
   \hyperref[TEI.ab]{ab} \hyperref[TEI.alt]{alt} \hyperref[TEI.altGrp]{altGrp} \hyperref[TEI.anchor]{anchor} \hyperref[TEI.join]{join} \hyperref[TEI.joinGrp]{joinGrp} \hyperref[TEI.link]{link} \hyperref[TEI.linkGrp]{linkGrp} \hyperref[TEI.timeline]{timeline}\par 
    \item[msdescription: ]
   \hyperref[TEI.msDesc]{msDesc} \hyperref[TEI.source]{source}\par 
    \item[namesdates: ]
   \hyperref[TEI.listOrg]{listOrg} \hyperref[TEI.listPlace]{listPlace}\par 
    \item[spoken: ]
   \hyperref[TEI.annotationBlock]{annotationBlock}\par 
    \item[textstructure: ]
   \hyperref[TEI.docAuthor]{docAuthor} \hyperref[TEI.docDate]{docDate} \hyperref[TEI.floatingText]{floatingText}\par 
    \item[transcr: ]
   \hyperref[TEI.addSpan]{addSpan} \hyperref[TEI.damageSpan]{damageSpan} \hyperref[TEI.delSpan]{delSpan} \hyperref[TEI.fw]{fw} \hyperref[TEI.listTranspose]{listTranspose} \hyperref[TEI.metamark]{metamark} \hyperref[TEI.space]{space} \hyperref[TEI.substJoin]{substJoin}
    \item[{Exemple}]
  \leavevmode\bgroup\exampleFont \begin{shaded}\noindent\mbox{}{<\textbf{figure}>}\mbox{}\newline 
\hspace*{6pt}{<\textbf{head}>}La tour rouge, de Giorgio De Chirico{</\textbf{head}>}\mbox{}\newline 
\hspace*{6pt}{<\textbf{figDesc}>}Le tableau représente un donjon au pied duquel s'étend un espace quasiment vide,\mbox{}\newline 
\hspace*{6pt}\hspace*{6pt} hormis quelques détails{</\textbf{figDesc}>}\mbox{}\newline 
\hspace*{6pt}{<\textbf{graphic}\hspace*{6pt}{scale}="{0.5}"\mbox{}\newline 
\hspace*{6pt}\hspace*{6pt}{url}="{http://www.cineclubdecaen.com/cinepho/peint/dechericho/tourrouge.jpg}"/>}\mbox{}\newline 
{</\textbf{figure}>}\end{shaded}\egroup 


    \item[{Modèle de contenu}]
  \mbox{}\hfill\\[-10pt]\begin{Verbatim}[fontsize=\small]
<content>
 <alternate maxOccurs="unbounded"
  minOccurs="0">
  <classRef key="model.headLike"/>
  <classRef key="model.common"/>
  <elementRef key="figDesc"/>
  <classRef key="model.graphicLike"/>
  <classRef key="model.global"/>
  <classRef key="model.divBottom"/>
 </alternate>
</content>
    
\end{Verbatim}

    \item[{Schéma Declaration}]
  \mbox{}\hfill\\[-10pt]\begin{Verbatim}[fontsize=\small]
element figure
{
   tei_att.global.attributes,
   tei_att.placement.attributes,
   tei_att.typed.attributes,
   (
      tei_model.headLike    | tei_model.common    | tei_figDesc    | tei_model.graphicLike    | tei_model.global    | tei_model.divBottom   )*
}
\end{Verbatim}

\end{reflist}  \index{fileDesc=<fileDesc>|oddindex}
\begin{reflist}
\item[]\begin{specHead}{TEI.fileDesc}{<fileDesc> }(description bibliographique du fichier) contient une description bibliographique complète du fichier électronique. [\xref{http://www.tei-c.org/release/doc/tei-p5-doc/en/html/HD.html\#HD2}{2.2. The File Description} \xref{http://www.tei-c.org/release/doc/tei-p5-doc/en/html/HD.html\#HD11}{2.1.1. The TEI Header and Its Components}]\end{specHead} 
    \item[{Module}]
  header
    \item[{Attributs}]
  Attributs \hyperref[TEI.att.global]{att.global} (\textit{@xml:id}, \textit{@n}, \textit{@xml:lang}, \textit{@xml:base}, \textit{@xml:space})  (\hyperref[TEI.att.global.rendition]{att.global.rendition} (\textit{@rend}, \textit{@style}, \textit{@rendition})) (\hyperref[TEI.att.global.linking]{att.global.linking} (\textit{@corresp}, \textit{@synch}, \textit{@sameAs}, \textit{@copyOf}, \textit{@next}, \textit{@prev}, \textit{@exclude}, \textit{@select})) (\hyperref[TEI.att.global.analytic]{att.global.analytic} (\textit{@ana})) (\hyperref[TEI.att.global.facs]{att.global.facs} (\textit{@facs})) (\hyperref[TEI.att.global.change]{att.global.change} (\textit{@change})) (\hyperref[TEI.att.global.responsibility]{att.global.responsibility} (\textit{@cert}, \textit{@resp})) (\hyperref[TEI.att.global.source]{att.global.source} (\textit{@source}))
    \item[{Contenu dans}]
  
    \item[header: ]
   \hyperref[TEI.biblFull]{biblFull} \hyperref[TEI.teiHeader]{teiHeader}
    \item[{Peut contenir}]
  
    \item[header: ]
   \hyperref[TEI.editionStmt]{editionStmt} \hyperref[TEI.extent]{extent} \hyperref[TEI.notesStmt]{notesStmt} \hyperref[TEI.publicationStmt]{publicationStmt} \hyperref[TEI.seriesStmt]{seriesStmt} \hyperref[TEI.sourceDesc]{sourceDesc} \hyperref[TEI.titleStmt]{titleStmt}
    \item[{Note}]
  \par
Cet élément est la source d'information principale pour créer une notice de catalogage ou une référence bibliographique destinée à un fichier électronique. Il fournit le titre et les mentions de responsabilité, ainsi que des informations sur la publication ou la distribution du fichier, sur la collection à laquelle il appartient le cas échéant, ainsi que des notes détaillées sur des informations qui n'apparaissent pas ailleurs dans l'en-tête. Il contient également une description bibliographique complète de la ou des sources du texte produit.
    \item[{Exemple}]
  \leavevmode\bgroup\exampleFont \begin{shaded}\noindent\mbox{}{<\textbf{teiHeader}>}\mbox{}\newline 
\hspace*{6pt}{<\textbf{fileDesc}>}\mbox{}\newline 
\hspace*{6pt}\hspace*{6pt}{<\textbf{titleStmt}>}\mbox{}\newline 
\hspace*{6pt}\hspace*{6pt}\hspace*{6pt}{<\textbf{title}>}Le document TEI minimal{</\textbf{title}>}\mbox{}\newline 
\hspace*{6pt}\hspace*{6pt}{</\textbf{titleStmt}>}\mbox{}\newline 
\hspace*{6pt}\hspace*{6pt}{<\textbf{publicationStmt}>}\mbox{}\newline 
\hspace*{6pt}\hspace*{6pt}\hspace*{6pt}{<\textbf{p}>}Distribué comme partie de TEI P5{</\textbf{p}>}\mbox{}\newline 
\hspace*{6pt}\hspace*{6pt}{</\textbf{publicationStmt}>}\mbox{}\newline 
\hspace*{6pt}\hspace*{6pt}{<\textbf{sourceDesc}>}\mbox{}\newline 
\hspace*{6pt}\hspace*{6pt}\hspace*{6pt}{<\textbf{p}>}Aucune source : ce document est né numérique{</\textbf{p}>}\mbox{}\newline 
\hspace*{6pt}\hspace*{6pt}{</\textbf{sourceDesc}>}\mbox{}\newline 
\hspace*{6pt}{</\textbf{fileDesc}>}\mbox{}\newline 
{</\textbf{teiHeader}>}\end{shaded}\egroup 


    \item[{Modèle de contenu}]
  \mbox{}\hfill\\[-10pt]\begin{Verbatim}[fontsize=\small]
<content>
 <sequence maxOccurs="1" minOccurs="1">
  <sequence maxOccurs="1" minOccurs="1">
   <elementRef key="titleStmt"/>
   <elementRef key="editionStmt"
    minOccurs="0"/>
   <elementRef key="extent" minOccurs="0"/>
   <elementRef key="publicationStmt"/>
   <elementRef key="seriesStmt"
    minOccurs="0"/>
   <elementRef key="notesStmt"
    minOccurs="0"/>
  </sequence>
  <elementRef key="sourceDesc"
   maxOccurs="unbounded" minOccurs="1"/>
 </sequence>
</content>
    
\end{Verbatim}

    \item[{Schéma Declaration}]
  \mbox{}\hfill\\[-10pt]\begin{Verbatim}[fontsize=\small]
element fileDesc
{
   tei_att.global.attributes,
   (
      (
         tei_titleStmt,
         tei_editionStmt?,
         tei_extent?,
         tei_publicationStmt,
         tei_seriesStmt?,
         tei_notesStmt?
      ),
      tei_sourceDesc+
   )
}
\end{Verbatim}

\end{reflist}  \index{filiation=<filiation>|oddindex}
\begin{reflist}
\item[]\begin{specHead}{TEI.filiation}{<filiation> }(filiation) contient les informations sur la \textit{filiation} du manuscrit, c'est-à-dire sur ses relations avec d'autres manuscrits connus du même texte, ses \textit{protographes}, \textit{antigraphes} et \textit{apographes}. [\xref{http://www.tei-c.org/release/doc/tei-p5-doc/en/html/MS.html\#mscoit}{10.6.1. The msItem and msItemStruct Elements}]\end{specHead} 
    \item[{Module}]
  msdescription
    \item[{Attributs}]
  Attributs \hyperref[TEI.att.global]{att.global} (\textit{@xml:id}, \textit{@n}, \textit{@xml:lang}, \textit{@xml:base}, \textit{@xml:space})  (\hyperref[TEI.att.global.rendition]{att.global.rendition} (\textit{@rend}, \textit{@style}, \textit{@rendition})) (\hyperref[TEI.att.global.linking]{att.global.linking} (\textit{@corresp}, \textit{@synch}, \textit{@sameAs}, \textit{@copyOf}, \textit{@next}, \textit{@prev}, \textit{@exclude}, \textit{@select})) (\hyperref[TEI.att.global.analytic]{att.global.analytic} (\textit{@ana})) (\hyperref[TEI.att.global.facs]{att.global.facs} (\textit{@facs})) (\hyperref[TEI.att.global.change]{att.global.change} (\textit{@change})) (\hyperref[TEI.att.global.responsibility]{att.global.responsibility} (\textit{@cert}, \textit{@resp})) (\hyperref[TEI.att.global.source]{att.global.source} (\textit{@source})) \hyperref[TEI.att.typed]{att.typed} (\textit{@type}, \textit{@subtype}) 
    \item[{Membre du}]
  \hyperref[TEI.model.msItemPart]{model.msItemPart} 
    \item[{Contenu dans}]
  
    \item[msdescription: ]
   \hyperref[TEI.msItem]{msItem} \hyperref[TEI.msItemStruct]{msItemStruct}
    \item[{Peut contenir}]
  
    \item[analysis: ]
   \hyperref[TEI.c]{c} \hyperref[TEI.cl]{cl} \hyperref[TEI.interp]{interp} \hyperref[TEI.interpGrp]{interpGrp} \hyperref[TEI.m]{m} \hyperref[TEI.pc]{pc} \hyperref[TEI.phr]{phr} \hyperref[TEI.s]{s} \hyperref[TEI.span]{span} \hyperref[TEI.spanGrp]{spanGrp} \hyperref[TEI.w]{w}\par 
    \item[core: ]
   \hyperref[TEI.abbr]{abbr} \hyperref[TEI.add]{add} \hyperref[TEI.address]{address} \hyperref[TEI.bibl]{bibl} \hyperref[TEI.biblStruct]{biblStruct} \hyperref[TEI.binaryObject]{binaryObject} \hyperref[TEI.cb]{cb} \hyperref[TEI.choice]{choice} \hyperref[TEI.cit]{cit} \hyperref[TEI.corr]{corr} \hyperref[TEI.date]{date} \hyperref[TEI.del]{del} \hyperref[TEI.desc]{desc} \hyperref[TEI.distinct]{distinct} \hyperref[TEI.email]{email} \hyperref[TEI.emph]{emph} \hyperref[TEI.expan]{expan} \hyperref[TEI.foreign]{foreign} \hyperref[TEI.gap]{gap} \hyperref[TEI.gb]{gb} \hyperref[TEI.gloss]{gloss} \hyperref[TEI.graphic]{graphic} \hyperref[TEI.hi]{hi} \hyperref[TEI.index]{index} \hyperref[TEI.l]{l} \hyperref[TEI.label]{label} \hyperref[TEI.lb]{lb} \hyperref[TEI.lg]{lg} \hyperref[TEI.list]{list} \hyperref[TEI.listBibl]{listBibl} \hyperref[TEI.measure]{measure} \hyperref[TEI.measureGrp]{measureGrp} \hyperref[TEI.media]{media} \hyperref[TEI.mentioned]{mentioned} \hyperref[TEI.milestone]{milestone} \hyperref[TEI.name]{name} \hyperref[TEI.note]{note} \hyperref[TEI.num]{num} \hyperref[TEI.orig]{orig} \hyperref[TEI.p]{p} \hyperref[TEI.pb]{pb} \hyperref[TEI.ptr]{ptr} \hyperref[TEI.q]{q} \hyperref[TEI.quote]{quote} \hyperref[TEI.ref]{ref} \hyperref[TEI.reg]{reg} \hyperref[TEI.rs]{rs} \hyperref[TEI.said]{said} \hyperref[TEI.sic]{sic} \hyperref[TEI.soCalled]{soCalled} \hyperref[TEI.sp]{sp} \hyperref[TEI.stage]{stage} \hyperref[TEI.term]{term} \hyperref[TEI.time]{time} \hyperref[TEI.title]{title} \hyperref[TEI.unclear]{unclear}\par 
    \item[derived-module-tei.istex: ]
   \hyperref[TEI.math]{math} \hyperref[TEI.mrow]{mrow}\par 
    \item[figures: ]
   \hyperref[TEI.figure]{figure} \hyperref[TEI.formula]{formula} \hyperref[TEI.notatedMusic]{notatedMusic} \hyperref[TEI.table]{table}\par 
    \item[header: ]
   \hyperref[TEI.biblFull]{biblFull} \hyperref[TEI.idno]{idno}\par 
    \item[iso-fs: ]
   \hyperref[TEI.fLib]{fLib} \hyperref[TEI.fs]{fs} \hyperref[TEI.fvLib]{fvLib}\par 
    \item[linking: ]
   \hyperref[TEI.ab]{ab} \hyperref[TEI.alt]{alt} \hyperref[TEI.altGrp]{altGrp} \hyperref[TEI.anchor]{anchor} \hyperref[TEI.join]{join} \hyperref[TEI.joinGrp]{joinGrp} \hyperref[TEI.link]{link} \hyperref[TEI.linkGrp]{linkGrp} \hyperref[TEI.seg]{seg} \hyperref[TEI.timeline]{timeline}\par 
    \item[msdescription: ]
   \hyperref[TEI.catchwords]{catchwords} \hyperref[TEI.depth]{depth} \hyperref[TEI.dim]{dim} \hyperref[TEI.dimensions]{dimensions} \hyperref[TEI.height]{height} \hyperref[TEI.heraldry]{heraldry} \hyperref[TEI.locus]{locus} \hyperref[TEI.locusGrp]{locusGrp} \hyperref[TEI.material]{material} \hyperref[TEI.msDesc]{msDesc} \hyperref[TEI.objectType]{objectType} \hyperref[TEI.origDate]{origDate} \hyperref[TEI.origPlace]{origPlace} \hyperref[TEI.secFol]{secFol} \hyperref[TEI.signatures]{signatures} \hyperref[TEI.source]{source} \hyperref[TEI.stamp]{stamp} \hyperref[TEI.watermark]{watermark} \hyperref[TEI.width]{width}\par 
    \item[namesdates: ]
   \hyperref[TEI.addName]{addName} \hyperref[TEI.affiliation]{affiliation} \hyperref[TEI.country]{country} \hyperref[TEI.forename]{forename} \hyperref[TEI.genName]{genName} \hyperref[TEI.geogName]{geogName} \hyperref[TEI.listOrg]{listOrg} \hyperref[TEI.listPlace]{listPlace} \hyperref[TEI.location]{location} \hyperref[TEI.nameLink]{nameLink} \hyperref[TEI.orgName]{orgName} \hyperref[TEI.persName]{persName} \hyperref[TEI.placeName]{placeName} \hyperref[TEI.region]{region} \hyperref[TEI.roleName]{roleName} \hyperref[TEI.settlement]{settlement} \hyperref[TEI.state]{state} \hyperref[TEI.surname]{surname}\par 
    \item[spoken: ]
   \hyperref[TEI.annotationBlock]{annotationBlock}\par 
    \item[textstructure: ]
   \hyperref[TEI.floatingText]{floatingText}\par 
    \item[transcr: ]
   \hyperref[TEI.addSpan]{addSpan} \hyperref[TEI.am]{am} \hyperref[TEI.damage]{damage} \hyperref[TEI.damageSpan]{damageSpan} \hyperref[TEI.delSpan]{delSpan} \hyperref[TEI.ex]{ex} \hyperref[TEI.fw]{fw} \hyperref[TEI.handShift]{handShift} \hyperref[TEI.listTranspose]{listTranspose} \hyperref[TEI.metamark]{metamark} \hyperref[TEI.mod]{mod} \hyperref[TEI.redo]{redo} \hyperref[TEI.restore]{restore} \hyperref[TEI.retrace]{retrace} \hyperref[TEI.secl]{secl} \hyperref[TEI.space]{space} \hyperref[TEI.subst]{subst} \hyperref[TEI.substJoin]{substJoin} \hyperref[TEI.supplied]{supplied} \hyperref[TEI.surplus]{surplus} \hyperref[TEI.undo]{undo}\par des données textuelles
    \item[{Exemple}]
  \leavevmode\bgroup\exampleFont \begin{shaded}\noindent\mbox{}{<\textbf{msContents}>}\mbox{}\newline 
\hspace*{6pt}{<\textbf{msItem}>}\mbox{}\newline 
\hspace*{6pt}\hspace*{6pt}{<\textbf{title}>}Beljakovski sbornik{</\textbf{title}>}\mbox{}\newline 
\hspace*{6pt}\hspace*{6pt}{<\textbf{filiation}\hspace*{6pt}{type}="{protograph}">}Bulgarian{</\textbf{filiation}>}\mbox{}\newline 
\hspace*{6pt}\hspace*{6pt}{<\textbf{filiation}\hspace*{6pt}{type}="{antigraph}">}Middle Bulgarian{</\textbf{filiation}>}\mbox{}\newline 
\hspace*{6pt}\hspace*{6pt}{<\textbf{filiation}\hspace*{6pt}{type}="{apograph}">}\mbox{}\newline 
\hspace*{6pt}\hspace*{6pt}\hspace*{6pt}{<\textbf{ref}\hspace*{6pt}{target}="{\#DN17}">}Dujchev N 17{</\textbf{ref}>}\mbox{}\newline 
\hspace*{6pt}\hspace*{6pt}{</\textbf{filiation}>}\mbox{}\newline 
\hspace*{6pt}{</\textbf{msItem}>}\mbox{}\newline 
{</\textbf{msContents}>}\mbox{}\newline 
\textit{<!-- ... -->}\mbox{}\newline 
{<\textbf{msDesc}\hspace*{6pt}{xml:id}="{DN17}">}\mbox{}\newline 
\hspace*{6pt}{<\textbf{msIdentifier}>}\mbox{}\newline 
\hspace*{6pt}\hspace*{6pt}{<\textbf{settlement}>}Faraway{</\textbf{settlement}>}\mbox{}\newline 
\hspace*{6pt}{</\textbf{msIdentifier}>}\mbox{}\newline 
\textit{<!-- ... -->}\mbox{}\newline 
{</\textbf{msDesc}>}\end{shaded}\egroup 

In this example, the reference to ‘Dujchev N17’ includes a link to some other manuscript description which has the identifier \texttt{DN17}.
    \item[{Exemple}]
  \leavevmode\bgroup\exampleFont \begin{shaded}\noindent\mbox{}{<\textbf{msItem}>}\mbox{}\newline 
\hspace*{6pt}{<\textbf{title}>}Guan-ben{</\textbf{title}>}\mbox{}\newline 
\hspace*{6pt}{<\textbf{filiation}>}\mbox{}\newline 
\hspace*{6pt}\hspace*{6pt}{<\textbf{p}>}The "Guan-ben" was widely current among mathematicians in the\mbox{}\newline 
\hspace*{6pt}\hspace*{6pt}\hspace*{6pt}\hspace*{6pt} Qing dynasty, and "Zhao Qimei version" was also read. It is\mbox{}\newline 
\hspace*{6pt}\hspace*{6pt}\hspace*{6pt}\hspace*{6pt} therefore difficult to know the correct filiation path to follow.\mbox{}\newline 
\hspace*{6pt}\hspace*{6pt}\hspace*{6pt}\hspace*{6pt} The study of this era is much indebted to Li Di. We explain the\mbox{}\newline 
\hspace*{6pt}\hspace*{6pt}\hspace*{6pt}\hspace*{6pt} outline of his conclusion here. Kong Guangsen\mbox{}\newline 
\hspace*{6pt}\hspace*{6pt}\hspace*{6pt}\hspace*{6pt} (1752-1786)(17) was from the same town as Dai Zhen, so he obtained\mbox{}\newline 
\hspace*{6pt}\hspace*{6pt}\hspace*{6pt}\hspace*{6pt} "Guan-ben" from him and studied it(18). Li Huang (d. 1811)\mbox{}\newline 
\hspace*{6pt}\hspace*{6pt}\hspace*{6pt}\hspace*{6pt} (19) took part in editing Si Ku Quan Shu, so he must have had\mbox{}\newline 
\hspace*{6pt}\hspace*{6pt}\hspace*{6pt}\hspace*{6pt} "Guan-ben". Then Zhang Dunren (1754-1834) obtained this version,\mbox{}\newline 
\hspace*{6pt}\hspace*{6pt}\hspace*{6pt}\hspace*{6pt} and studied "Da Yan Zong Shu Shu" (The General Dayan\mbox{}\newline 
\hspace*{6pt}\hspace*{6pt}\hspace*{6pt}\hspace*{6pt} Computation). He wrote Jiu Yi Suan Shu (Mathematics\mbox{}\newline 
\hspace*{6pt}\hspace*{6pt}\hspace*{6pt}\hspace*{6pt} Searching for One, 1803) based on this version of Shu Xue Jiu\mbox{}\newline 
\hspace*{6pt}\hspace*{6pt}\hspace*{6pt}\hspace*{6pt} Zhang (20).{</\textbf{p}>}\mbox{}\newline 
\hspace*{6pt}\hspace*{6pt}{<\textbf{p}>}One of the most important persons in restoring our knowledge\mbox{}\newline 
\hspace*{6pt}\hspace*{6pt}\hspace*{6pt}\hspace*{6pt} concerning the filiation of these books was Li Rui (1768(21)\mbox{}\newline 
\hspace*{6pt}\hspace*{6pt}\hspace*{6pt}\hspace*{6pt} -1817)(see his biography). ... only two volumes remain of this\mbox{}\newline 
\hspace*{6pt}\hspace*{6pt}\hspace*{6pt}\hspace*{6pt} manuscript, as far as chapter 6 (chapter 3 part 2) p.13, that is,\mbox{}\newline 
\hspace*{6pt}\hspace*{6pt}\hspace*{6pt}\hspace*{6pt} question 2 of "Huan Tian San Ji" (square of three loops),\mbox{}\newline 
\hspace*{6pt}\hspace*{6pt}\hspace*{6pt}\hspace*{6pt} which later has been lost.{</\textbf{p}>}\mbox{}\newline 
\hspace*{6pt}{</\textbf{filiation}>}\mbox{}\newline 
{</\textbf{msItem}>}\mbox{}\newline 
\textit{<!--http://www2.nkfust.edu.tw/\textasciitilde jochi/ed1.htm-->}\end{shaded}\egroup 


    \item[{Modèle de contenu}]
  \mbox{}\hfill\\[-10pt]\begin{Verbatim}[fontsize=\small]
<content>
 <macroRef key="macro.specialPara"/>
</content>
    
\end{Verbatim}

    \item[{Schéma Declaration}]
  \mbox{}\hfill\\[-10pt]\begin{Verbatim}[fontsize=\small]
element filiation
{
   tei_att.global.attributes,
   tei_att.typed.attributes,
   tei_macro.specialPara}
\end{Verbatim}

\end{reflist}  \index{finalRubric=<finalRubric>|oddindex}
\begin{reflist}
\item[]\begin{specHead}{TEI.finalRubric}{<finalRubric> }(rubrique de fin) Contient les derniers mots d'une section de texte, qui incluent souvent la mention de son auteur et de son titre, et sont généralement différenciés du texte lui-même par l'utilisation d'une encre rouge, par une taille ou un style d'écriture particuliers, ou par tout autre moyen visuel. [\xref{http://www.tei-c.org/release/doc/tei-p5-doc/en/html/MS.html\#mscoit}{10.6.1. The msItem and msItemStruct Elements}]\end{specHead} 
    \item[{Module}]
  msdescription
    \item[{Attributs}]
  Attributs \hyperref[TEI.att.global]{att.global} (\textit{@xml:id}, \textit{@n}, \textit{@xml:lang}, \textit{@xml:base}, \textit{@xml:space})  (\hyperref[TEI.att.global.rendition]{att.global.rendition} (\textit{@rend}, \textit{@style}, \textit{@rendition})) (\hyperref[TEI.att.global.linking]{att.global.linking} (\textit{@corresp}, \textit{@synch}, \textit{@sameAs}, \textit{@copyOf}, \textit{@next}, \textit{@prev}, \textit{@exclude}, \textit{@select})) (\hyperref[TEI.att.global.analytic]{att.global.analytic} (\textit{@ana})) (\hyperref[TEI.att.global.facs]{att.global.facs} (\textit{@facs})) (\hyperref[TEI.att.global.change]{att.global.change} (\textit{@change})) (\hyperref[TEI.att.global.responsibility]{att.global.responsibility} (\textit{@cert}, \textit{@resp})) (\hyperref[TEI.att.global.source]{att.global.source} (\textit{@source})) \hyperref[TEI.att.typed]{att.typed} (\textit{@type}, \textit{@subtype}) \hyperref[TEI.att.msExcerpt]{att.msExcerpt} (\textit{@defective}) 
    \item[{Membre du}]
  \hyperref[TEI.model.msQuoteLike]{model.msQuoteLike} 
    \item[{Contenu dans}]
  
    \item[msdescription: ]
   \hyperref[TEI.msItem]{msItem} \hyperref[TEI.msItemStruct]{msItemStruct}
    \item[{Peut contenir}]
  
    \item[analysis: ]
   \hyperref[TEI.c]{c} \hyperref[TEI.cl]{cl} \hyperref[TEI.interp]{interp} \hyperref[TEI.interpGrp]{interpGrp} \hyperref[TEI.m]{m} \hyperref[TEI.pc]{pc} \hyperref[TEI.phr]{phr} \hyperref[TEI.s]{s} \hyperref[TEI.span]{span} \hyperref[TEI.spanGrp]{spanGrp} \hyperref[TEI.w]{w}\par 
    \item[core: ]
   \hyperref[TEI.abbr]{abbr} \hyperref[TEI.add]{add} \hyperref[TEI.address]{address} \hyperref[TEI.binaryObject]{binaryObject} \hyperref[TEI.cb]{cb} \hyperref[TEI.choice]{choice} \hyperref[TEI.corr]{corr} \hyperref[TEI.date]{date} \hyperref[TEI.del]{del} \hyperref[TEI.distinct]{distinct} \hyperref[TEI.email]{email} \hyperref[TEI.emph]{emph} \hyperref[TEI.expan]{expan} \hyperref[TEI.foreign]{foreign} \hyperref[TEI.gap]{gap} \hyperref[TEI.gb]{gb} \hyperref[TEI.gloss]{gloss} \hyperref[TEI.graphic]{graphic} \hyperref[TEI.hi]{hi} \hyperref[TEI.index]{index} \hyperref[TEI.lb]{lb} \hyperref[TEI.measure]{measure} \hyperref[TEI.measureGrp]{measureGrp} \hyperref[TEI.media]{media} \hyperref[TEI.mentioned]{mentioned} \hyperref[TEI.milestone]{milestone} \hyperref[TEI.name]{name} \hyperref[TEI.note]{note} \hyperref[TEI.num]{num} \hyperref[TEI.orig]{orig} \hyperref[TEI.pb]{pb} \hyperref[TEI.ptr]{ptr} \hyperref[TEI.ref]{ref} \hyperref[TEI.reg]{reg} \hyperref[TEI.rs]{rs} \hyperref[TEI.sic]{sic} \hyperref[TEI.soCalled]{soCalled} \hyperref[TEI.term]{term} \hyperref[TEI.time]{time} \hyperref[TEI.title]{title} \hyperref[TEI.unclear]{unclear}\par 
    \item[derived-module-tei.istex: ]
   \hyperref[TEI.math]{math} \hyperref[TEI.mrow]{mrow}\par 
    \item[figures: ]
   \hyperref[TEI.figure]{figure} \hyperref[TEI.formula]{formula} \hyperref[TEI.notatedMusic]{notatedMusic}\par 
    \item[header: ]
   \hyperref[TEI.idno]{idno}\par 
    \item[iso-fs: ]
   \hyperref[TEI.fLib]{fLib} \hyperref[TEI.fs]{fs} \hyperref[TEI.fvLib]{fvLib}\par 
    \item[linking: ]
   \hyperref[TEI.alt]{alt} \hyperref[TEI.altGrp]{altGrp} \hyperref[TEI.anchor]{anchor} \hyperref[TEI.join]{join} \hyperref[TEI.joinGrp]{joinGrp} \hyperref[TEI.link]{link} \hyperref[TEI.linkGrp]{linkGrp} \hyperref[TEI.seg]{seg} \hyperref[TEI.timeline]{timeline}\par 
    \item[msdescription: ]
   \hyperref[TEI.catchwords]{catchwords} \hyperref[TEI.depth]{depth} \hyperref[TEI.dim]{dim} \hyperref[TEI.dimensions]{dimensions} \hyperref[TEI.height]{height} \hyperref[TEI.heraldry]{heraldry} \hyperref[TEI.locus]{locus} \hyperref[TEI.locusGrp]{locusGrp} \hyperref[TEI.material]{material} \hyperref[TEI.objectType]{objectType} \hyperref[TEI.origDate]{origDate} \hyperref[TEI.origPlace]{origPlace} \hyperref[TEI.secFol]{secFol} \hyperref[TEI.signatures]{signatures} \hyperref[TEI.source]{source} \hyperref[TEI.stamp]{stamp} \hyperref[TEI.watermark]{watermark} \hyperref[TEI.width]{width}\par 
    \item[namesdates: ]
   \hyperref[TEI.addName]{addName} \hyperref[TEI.affiliation]{affiliation} \hyperref[TEI.country]{country} \hyperref[TEI.forename]{forename} \hyperref[TEI.genName]{genName} \hyperref[TEI.geogName]{geogName} \hyperref[TEI.location]{location} \hyperref[TEI.nameLink]{nameLink} \hyperref[TEI.orgName]{orgName} \hyperref[TEI.persName]{persName} \hyperref[TEI.placeName]{placeName} \hyperref[TEI.region]{region} \hyperref[TEI.roleName]{roleName} \hyperref[TEI.settlement]{settlement} \hyperref[TEI.state]{state} \hyperref[TEI.surname]{surname}\par 
    \item[spoken: ]
   \hyperref[TEI.annotationBlock]{annotationBlock}\par 
    \item[transcr: ]
   \hyperref[TEI.addSpan]{addSpan} \hyperref[TEI.am]{am} \hyperref[TEI.damage]{damage} \hyperref[TEI.damageSpan]{damageSpan} \hyperref[TEI.delSpan]{delSpan} \hyperref[TEI.ex]{ex} \hyperref[TEI.fw]{fw} \hyperref[TEI.handShift]{handShift} \hyperref[TEI.listTranspose]{listTranspose} \hyperref[TEI.metamark]{metamark} \hyperref[TEI.mod]{mod} \hyperref[TEI.redo]{redo} \hyperref[TEI.restore]{restore} \hyperref[TEI.retrace]{retrace} \hyperref[TEI.secl]{secl} \hyperref[TEI.space]{space} \hyperref[TEI.subst]{subst} \hyperref[TEI.substJoin]{substJoin} \hyperref[TEI.supplied]{supplied} \hyperref[TEI.surplus]{surplus} \hyperref[TEI.undo]{undo}\par des données textuelles
    \item[{Exemple}]
  \leavevmode\bgroup\exampleFont \begin{shaded}\noindent\mbox{}{<\textbf{finalRubric}>}Explicit le romans de la Rose ou l'art\mbox{}\newline 
 d'amours est toute enclose.{</\textbf{finalRubric}>}\mbox{}\newline 
{<\textbf{finalRubric}>}ok lúkv ver þar Brennu-Nials savgv{</\textbf{finalRubric}>}\end{shaded}\egroup 


    \item[{Exemple}]
  \leavevmode\bgroup\exampleFont \begin{shaded}\noindent\mbox{}{<\textbf{finalRubric}>}Explicit le romans de la Rose ou l'art d'amours est toute enclose.{</\textbf{finalRubric}>}\mbox{}\newline 
{<\textbf{finalRubric}>}Ci falt la geste que Turoldus declinet. {</\textbf{finalRubric}>}\end{shaded}\egroup 


    \item[{Modèle de contenu}]
  \mbox{}\hfill\\[-10pt]\begin{Verbatim}[fontsize=\small]
<content>
 <macroRef key="macro.phraseSeq"/>
</content>
    
\end{Verbatim}

    \item[{Schéma Declaration}]
  \mbox{}\hfill\\[-10pt]\begin{Verbatim}[fontsize=\small]
element finalRubric
{
   tei_att.global.attributes,
   tei_att.typed.attributes,
   tei_att.msExcerpt.attributes,
   tei_macro.phraseSeq}
\end{Verbatim}

\end{reflist}  \index{floatingText=<floatingText>|oddindex}
\begin{reflist}
\item[]\begin{specHead}{TEI.floatingText}{<floatingText> }(texte mobile) contient un texte quelconque, homogène ou composite, qui interrompt le texte le contenant à n’importe quel endroit et après lequel le texte environnant reprend. [\xref{http://www.tei-c.org/release/doc/tei-p5-doc/en/html/DS.html\#DSFLT}{4.3.2. Floating Texts}]\end{specHead} 
    \item[{Module}]
  textstructure
    \item[{Attributs}]
  Attributs \hyperref[TEI.att.global]{att.global} (\textit{@xml:id}, \textit{@n}, \textit{@xml:lang}, \textit{@xml:base}, \textit{@xml:space})  (\hyperref[TEI.att.global.rendition]{att.global.rendition} (\textit{@rend}, \textit{@style}, \textit{@rendition})) (\hyperref[TEI.att.global.linking]{att.global.linking} (\textit{@corresp}, \textit{@synch}, \textit{@sameAs}, \textit{@copyOf}, \textit{@next}, \textit{@prev}, \textit{@exclude}, \textit{@select})) (\hyperref[TEI.att.global.analytic]{att.global.analytic} (\textit{@ana})) (\hyperref[TEI.att.global.facs]{att.global.facs} (\textit{@facs})) (\hyperref[TEI.att.global.change]{att.global.change} (\textit{@change})) (\hyperref[TEI.att.global.responsibility]{att.global.responsibility} (\textit{@cert}, \textit{@resp})) (\hyperref[TEI.att.global.source]{att.global.source} (\textit{@source})) \hyperref[TEI.att.declaring]{att.declaring} (\textit{@decls}) \hyperref[TEI.att.typed]{att.typed} (\textit{@type}, \textit{@subtype}) 
    \item[{Membre du}]
  \hyperref[TEI.model.qLike]{model.qLike}
    \item[{Contenu dans}]
  
    \item[core: ]
   \hyperref[TEI.add]{add} \hyperref[TEI.cit]{cit} \hyperref[TEI.corr]{corr} \hyperref[TEI.del]{del} \hyperref[TEI.desc]{desc} \hyperref[TEI.emph]{emph} \hyperref[TEI.head]{head} \hyperref[TEI.hi]{hi} \hyperref[TEI.item]{item} \hyperref[TEI.l]{l} \hyperref[TEI.meeting]{meeting} \hyperref[TEI.note]{note} \hyperref[TEI.orig]{orig} \hyperref[TEI.p]{p} \hyperref[TEI.q]{q} \hyperref[TEI.quote]{quote} \hyperref[TEI.ref]{ref} \hyperref[TEI.reg]{reg} \hyperref[TEI.said]{said} \hyperref[TEI.sic]{sic} \hyperref[TEI.sp]{sp} \hyperref[TEI.stage]{stage} \hyperref[TEI.title]{title} \hyperref[TEI.unclear]{unclear}\par 
    \item[figures: ]
   \hyperref[TEI.cell]{cell} \hyperref[TEI.figDesc]{figDesc} \hyperref[TEI.figure]{figure}\par 
    \item[header: ]
   \hyperref[TEI.change]{change} \hyperref[TEI.licence]{licence} \hyperref[TEI.rendition]{rendition}\par 
    \item[iso-fs: ]
   \hyperref[TEI.fDescr]{fDescr} \hyperref[TEI.fsDescr]{fsDescr}\par 
    \item[linking: ]
   \hyperref[TEI.ab]{ab} \hyperref[TEI.seg]{seg}\par 
    \item[msdescription: ]
   \hyperref[TEI.accMat]{accMat} \hyperref[TEI.acquisition]{acquisition} \hyperref[TEI.additions]{additions} \hyperref[TEI.collation]{collation} \hyperref[TEI.condition]{condition} \hyperref[TEI.custEvent]{custEvent} \hyperref[TEI.decoNote]{decoNote} \hyperref[TEI.filiation]{filiation} \hyperref[TEI.foliation]{foliation} \hyperref[TEI.layout]{layout} \hyperref[TEI.musicNotation]{musicNotation} \hyperref[TEI.origin]{origin} \hyperref[TEI.provenance]{provenance} \hyperref[TEI.signatures]{signatures} \hyperref[TEI.source]{source} \hyperref[TEI.summary]{summary} \hyperref[TEI.support]{support} \hyperref[TEI.surrogates]{surrogates} \hyperref[TEI.typeNote]{typeNote}\par 
    \item[textstructure: ]
   \hyperref[TEI.body]{body} \hyperref[TEI.div]{div} \hyperref[TEI.docEdition]{docEdition} \hyperref[TEI.titlePart]{titlePart}\par 
    \item[transcr: ]
   \hyperref[TEI.damage]{damage} \hyperref[TEI.metamark]{metamark} \hyperref[TEI.mod]{mod} \hyperref[TEI.restore]{restore} \hyperref[TEI.retrace]{retrace} \hyperref[TEI.secl]{secl} \hyperref[TEI.supplied]{supplied} \hyperref[TEI.surplus]{surplus}
    \item[{Peut contenir}]
  
    \item[analysis: ]
   \hyperref[TEI.interp]{interp} \hyperref[TEI.interpGrp]{interpGrp} \hyperref[TEI.span]{span} \hyperref[TEI.spanGrp]{spanGrp}\par 
    \item[core: ]
   \hyperref[TEI.cb]{cb} \hyperref[TEI.gap]{gap} \hyperref[TEI.gb]{gb} \hyperref[TEI.index]{index} \hyperref[TEI.lb]{lb} \hyperref[TEI.milestone]{milestone} \hyperref[TEI.note]{note} \hyperref[TEI.pb]{pb}\par 
    \item[figures: ]
   \hyperref[TEI.figure]{figure} \hyperref[TEI.notatedMusic]{notatedMusic}\par 
    \item[iso-fs: ]
   \hyperref[TEI.fLib]{fLib} \hyperref[TEI.fs]{fs} \hyperref[TEI.fvLib]{fvLib}\par 
    \item[linking: ]
   \hyperref[TEI.alt]{alt} \hyperref[TEI.altGrp]{altGrp} \hyperref[TEI.anchor]{anchor} \hyperref[TEI.join]{join} \hyperref[TEI.joinGrp]{joinGrp} \hyperref[TEI.link]{link} \hyperref[TEI.linkGrp]{linkGrp} \hyperref[TEI.timeline]{timeline}\par 
    \item[msdescription: ]
   \hyperref[TEI.source]{source}\par 
    \item[textstructure: ]
   \hyperref[TEI.back]{back} \hyperref[TEI.body]{body} \hyperref[TEI.front]{front} \hyperref[TEI.group]{group}\par 
    \item[transcr: ]
   \hyperref[TEI.addSpan]{addSpan} \hyperref[TEI.damageSpan]{damageSpan} \hyperref[TEI.delSpan]{delSpan} \hyperref[TEI.fw]{fw} \hyperref[TEI.listTranspose]{listTranspose} \hyperref[TEI.metamark]{metamark} \hyperref[TEI.space]{space} \hyperref[TEI.substJoin]{substJoin}
    \item[{Note}]
  \par
Un texte "flottant" a le même contenu que tout autre texte : il peut donc être interrompu par un autre texte "flottant" ou contenir un groupe de textes composites.
    \item[{Exemple}]
  \leavevmode\bgroup\exampleFont \begin{shaded}\noindent\mbox{}{<\textbf{body}>}\mbox{}\newline 
\hspace*{6pt}{<\textbf{div}\hspace*{6pt}{type}="{scene}">}\mbox{}\newline 
\hspace*{6pt}\hspace*{6pt}{<\textbf{sp}>}\mbox{}\newline 
\hspace*{6pt}\hspace*{6pt}\hspace*{6pt}{<\textbf{p}>}Chut ! Les acteurs commencent...{</\textbf{p}>}\mbox{}\newline 
\hspace*{6pt}\hspace*{6pt}{</\textbf{sp}>}\mbox{}\newline 
\hspace*{6pt}\hspace*{6pt}{<\textbf{floatingText}\hspace*{6pt}{type}="{pwp}">}\mbox{}\newline 
\hspace*{6pt}\hspace*{6pt}\hspace*{6pt}{<\textbf{body}>}\mbox{}\newline 
\hspace*{6pt}\hspace*{6pt}\hspace*{6pt}\hspace*{6pt}{<\textbf{div}\hspace*{6pt}{type}="{act}">}\mbox{}\newline 
\hspace*{6pt}\hspace*{6pt}\hspace*{6pt}\hspace*{6pt}\hspace*{6pt}{<\textbf{sp}>}\mbox{}\newline 
\hspace*{6pt}\hspace*{6pt}\hspace*{6pt}\hspace*{6pt}\hspace*{6pt}\hspace*{6pt}{<\textbf{l}>}Notre histoire se passe à Athènes [...]{</\textbf{l}>}\mbox{}\newline 
\textit{<!-- ... rest of nested act here -->}\mbox{}\newline 
\hspace*{6pt}\hspace*{6pt}\hspace*{6pt}\hspace*{6pt}\hspace*{6pt}{</\textbf{sp}>}\mbox{}\newline 
\hspace*{6pt}\hspace*{6pt}\hspace*{6pt}\hspace*{6pt}{</\textbf{div}>}\mbox{}\newline 
\hspace*{6pt}\hspace*{6pt}\hspace*{6pt}{</\textbf{body}>}\mbox{}\newline 
\hspace*{6pt}\hspace*{6pt}{</\textbf{floatingText}>}\mbox{}\newline 
\hspace*{6pt}\hspace*{6pt}{<\textbf{sp}>}\mbox{}\newline 
\hspace*{6pt}\hspace*{6pt}\hspace*{6pt}{<\textbf{p}>}La pièce est maintenant finie ...{</\textbf{p}>}\mbox{}\newline 
\hspace*{6pt}\hspace*{6pt}{</\textbf{sp}>}\mbox{}\newline 
\hspace*{6pt}{</\textbf{div}>}\mbox{}\newline 
{</\textbf{body}>}\end{shaded}\egroup 


    \item[{Modèle de contenu}]
  \mbox{}\hfill\\[-10pt]\begin{Verbatim}[fontsize=\small]
<content>
 <sequence maxOccurs="1" minOccurs="1">
  <classRef key="model.global"
   maxOccurs="unbounded" minOccurs="0"/>
  <sequence maxOccurs="1" minOccurs="0">
   <elementRef key="front"/>
   <classRef key="model.global"
    maxOccurs="unbounded" minOccurs="0"/>
  </sequence>
  <alternate maxOccurs="1" minOccurs="1">
   <elementRef key="body"/>
   <elementRef key="group"/>
  </alternate>
  <classRef key="model.global"
   maxOccurs="unbounded" minOccurs="0"/>
  <sequence maxOccurs="1" minOccurs="0">
   <elementRef key="back"/>
   <classRef key="model.global"
    maxOccurs="unbounded" minOccurs="0"/>
  </sequence>
 </sequence>
</content>
    
\end{Verbatim}

    \item[{Schéma Declaration}]
  \mbox{}\hfill\\[-10pt]\begin{Verbatim}[fontsize=\small]
element floatingText
{
   tei_att.global.attributes,
   tei_att.declaring.attributes,
   tei_att.typed.attributes,
   (
      tei_model.global*,
      ( tei_front, tei_model.global* )?,
      ( tei_body | tei_group ),
      tei_model.global*,
      ( tei_back, tei_model.global* )?
   )
}
\end{Verbatim}

\end{reflist}  \index{foliation=<foliation>|oddindex}
\begin{reflist}
\item[]\begin{specHead}{TEI.foliation}{<foliation> }(foliotation) décrit le ou les systèmes de numérotation des feuillets ou pages d'un codex [\xref{http://www.tei-c.org/release/doc/tei-p5-doc/en/html/MS.html\#msphfo}{10.7.1.4. Foliation}]\end{specHead} 
    \item[{Module}]
  msdescription
    \item[{Attributs}]
  Attributs \hyperref[TEI.att.global]{att.global} (\textit{@xml:id}, \textit{@n}, \textit{@xml:lang}, \textit{@xml:base}, \textit{@xml:space})  (\hyperref[TEI.att.global.rendition]{att.global.rendition} (\textit{@rend}, \textit{@style}, \textit{@rendition})) (\hyperref[TEI.att.global.linking]{att.global.linking} (\textit{@corresp}, \textit{@synch}, \textit{@sameAs}, \textit{@copyOf}, \textit{@next}, \textit{@prev}, \textit{@exclude}, \textit{@select})) (\hyperref[TEI.att.global.analytic]{att.global.analytic} (\textit{@ana})) (\hyperref[TEI.att.global.facs]{att.global.facs} (\textit{@facs})) (\hyperref[TEI.att.global.change]{att.global.change} (\textit{@change})) (\hyperref[TEI.att.global.responsibility]{att.global.responsibility} (\textit{@cert}, \textit{@resp})) (\hyperref[TEI.att.global.source]{att.global.source} (\textit{@source}))
    \item[{Contenu dans}]
  
    \item[msdescription: ]
   \hyperref[TEI.supportDesc]{supportDesc}
    \item[{Peut contenir}]
  
    \item[analysis: ]
   \hyperref[TEI.c]{c} \hyperref[TEI.cl]{cl} \hyperref[TEI.interp]{interp} \hyperref[TEI.interpGrp]{interpGrp} \hyperref[TEI.m]{m} \hyperref[TEI.pc]{pc} \hyperref[TEI.phr]{phr} \hyperref[TEI.s]{s} \hyperref[TEI.span]{span} \hyperref[TEI.spanGrp]{spanGrp} \hyperref[TEI.w]{w}\par 
    \item[core: ]
   \hyperref[TEI.abbr]{abbr} \hyperref[TEI.add]{add} \hyperref[TEI.address]{address} \hyperref[TEI.bibl]{bibl} \hyperref[TEI.biblStruct]{biblStruct} \hyperref[TEI.binaryObject]{binaryObject} \hyperref[TEI.cb]{cb} \hyperref[TEI.choice]{choice} \hyperref[TEI.cit]{cit} \hyperref[TEI.corr]{corr} \hyperref[TEI.date]{date} \hyperref[TEI.del]{del} \hyperref[TEI.desc]{desc} \hyperref[TEI.distinct]{distinct} \hyperref[TEI.email]{email} \hyperref[TEI.emph]{emph} \hyperref[TEI.expan]{expan} \hyperref[TEI.foreign]{foreign} \hyperref[TEI.gap]{gap} \hyperref[TEI.gb]{gb} \hyperref[TEI.gloss]{gloss} \hyperref[TEI.graphic]{graphic} \hyperref[TEI.hi]{hi} \hyperref[TEI.index]{index} \hyperref[TEI.l]{l} \hyperref[TEI.label]{label} \hyperref[TEI.lb]{lb} \hyperref[TEI.lg]{lg} \hyperref[TEI.list]{list} \hyperref[TEI.listBibl]{listBibl} \hyperref[TEI.measure]{measure} \hyperref[TEI.measureGrp]{measureGrp} \hyperref[TEI.media]{media} \hyperref[TEI.mentioned]{mentioned} \hyperref[TEI.milestone]{milestone} \hyperref[TEI.name]{name} \hyperref[TEI.note]{note} \hyperref[TEI.num]{num} \hyperref[TEI.orig]{orig} \hyperref[TEI.p]{p} \hyperref[TEI.pb]{pb} \hyperref[TEI.ptr]{ptr} \hyperref[TEI.q]{q} \hyperref[TEI.quote]{quote} \hyperref[TEI.ref]{ref} \hyperref[TEI.reg]{reg} \hyperref[TEI.rs]{rs} \hyperref[TEI.said]{said} \hyperref[TEI.sic]{sic} \hyperref[TEI.soCalled]{soCalled} \hyperref[TEI.sp]{sp} \hyperref[TEI.stage]{stage} \hyperref[TEI.term]{term} \hyperref[TEI.time]{time} \hyperref[TEI.title]{title} \hyperref[TEI.unclear]{unclear}\par 
    \item[derived-module-tei.istex: ]
   \hyperref[TEI.math]{math} \hyperref[TEI.mrow]{mrow}\par 
    \item[figures: ]
   \hyperref[TEI.figure]{figure} \hyperref[TEI.formula]{formula} \hyperref[TEI.notatedMusic]{notatedMusic} \hyperref[TEI.table]{table}\par 
    \item[header: ]
   \hyperref[TEI.biblFull]{biblFull} \hyperref[TEI.idno]{idno}\par 
    \item[iso-fs: ]
   \hyperref[TEI.fLib]{fLib} \hyperref[TEI.fs]{fs} \hyperref[TEI.fvLib]{fvLib}\par 
    \item[linking: ]
   \hyperref[TEI.ab]{ab} \hyperref[TEI.alt]{alt} \hyperref[TEI.altGrp]{altGrp} \hyperref[TEI.anchor]{anchor} \hyperref[TEI.join]{join} \hyperref[TEI.joinGrp]{joinGrp} \hyperref[TEI.link]{link} \hyperref[TEI.linkGrp]{linkGrp} \hyperref[TEI.seg]{seg} \hyperref[TEI.timeline]{timeline}\par 
    \item[msdescription: ]
   \hyperref[TEI.catchwords]{catchwords} \hyperref[TEI.depth]{depth} \hyperref[TEI.dim]{dim} \hyperref[TEI.dimensions]{dimensions} \hyperref[TEI.height]{height} \hyperref[TEI.heraldry]{heraldry} \hyperref[TEI.locus]{locus} \hyperref[TEI.locusGrp]{locusGrp} \hyperref[TEI.material]{material} \hyperref[TEI.msDesc]{msDesc} \hyperref[TEI.objectType]{objectType} \hyperref[TEI.origDate]{origDate} \hyperref[TEI.origPlace]{origPlace} \hyperref[TEI.secFol]{secFol} \hyperref[TEI.signatures]{signatures} \hyperref[TEI.source]{source} \hyperref[TEI.stamp]{stamp} \hyperref[TEI.watermark]{watermark} \hyperref[TEI.width]{width}\par 
    \item[namesdates: ]
   \hyperref[TEI.addName]{addName} \hyperref[TEI.affiliation]{affiliation} \hyperref[TEI.country]{country} \hyperref[TEI.forename]{forename} \hyperref[TEI.genName]{genName} \hyperref[TEI.geogName]{geogName} \hyperref[TEI.listOrg]{listOrg} \hyperref[TEI.listPlace]{listPlace} \hyperref[TEI.location]{location} \hyperref[TEI.nameLink]{nameLink} \hyperref[TEI.orgName]{orgName} \hyperref[TEI.persName]{persName} \hyperref[TEI.placeName]{placeName} \hyperref[TEI.region]{region} \hyperref[TEI.roleName]{roleName} \hyperref[TEI.settlement]{settlement} \hyperref[TEI.state]{state} \hyperref[TEI.surname]{surname}\par 
    \item[spoken: ]
   \hyperref[TEI.annotationBlock]{annotationBlock}\par 
    \item[textstructure: ]
   \hyperref[TEI.floatingText]{floatingText}\par 
    \item[transcr: ]
   \hyperref[TEI.addSpan]{addSpan} \hyperref[TEI.am]{am} \hyperref[TEI.damage]{damage} \hyperref[TEI.damageSpan]{damageSpan} \hyperref[TEI.delSpan]{delSpan} \hyperref[TEI.ex]{ex} \hyperref[TEI.fw]{fw} \hyperref[TEI.handShift]{handShift} \hyperref[TEI.listTranspose]{listTranspose} \hyperref[TEI.metamark]{metamark} \hyperref[TEI.mod]{mod} \hyperref[TEI.redo]{redo} \hyperref[TEI.restore]{restore} \hyperref[TEI.retrace]{retrace} \hyperref[TEI.secl]{secl} \hyperref[TEI.space]{space} \hyperref[TEI.subst]{subst} \hyperref[TEI.substJoin]{substJoin} \hyperref[TEI.supplied]{supplied} \hyperref[TEI.surplus]{surplus} \hyperref[TEI.undo]{undo}\par des données textuelles
    \item[{Exemple}]
  \leavevmode\bgroup\exampleFont \begin{shaded}\noindent\mbox{}{<\textbf{foliation}>}Contemporary foliation in red roman numerals in the centre of the outer\mbox{}\newline 
 margin.{</\textbf{foliation}>}\end{shaded}\egroup 


    \item[{Modèle de contenu}]
  \mbox{}\hfill\\[-10pt]\begin{Verbatim}[fontsize=\small]
<content>
 <macroRef key="macro.specialPara"/>
</content>
    
\end{Verbatim}

    \item[{Schéma Declaration}]
  \mbox{}\hfill\\[-10pt]\begin{Verbatim}[fontsize=\small]
element foliation { tei_att.global.attributes, tei_macro.specialPara }
\end{Verbatim}

\end{reflist}  \index{foreign=<foreign>|oddindex}
\begin{reflist}
\item[]\begin{specHead}{TEI.foreign}{<foreign> }(étranger) reconnaît un mot ou une expression comme appartenant à une langue différente de celle du contexte. [\xref{http://www.tei-c.org/release/doc/tei-p5-doc/en/html/CO.html\#COHQHF}{3.3.2.1. Foreign Words or Expressions}]\end{specHead} 
    \item[{Module}]
  core
    \item[{Attributs}]
  Attributs \hyperref[TEI.att.global]{att.global} (\textit{@xml:id}, \textit{@n}, \textit{@xml:lang}, \textit{@xml:base}, \textit{@xml:space})  (\hyperref[TEI.att.global.rendition]{att.global.rendition} (\textit{@rend}, \textit{@style}, \textit{@rendition})) (\hyperref[TEI.att.global.linking]{att.global.linking} (\textit{@corresp}, \textit{@synch}, \textit{@sameAs}, \textit{@copyOf}, \textit{@next}, \textit{@prev}, \textit{@exclude}, \textit{@select})) (\hyperref[TEI.att.global.analytic]{att.global.analytic} (\textit{@ana})) (\hyperref[TEI.att.global.facs]{att.global.facs} (\textit{@facs})) (\hyperref[TEI.att.global.change]{att.global.change} (\textit{@change})) (\hyperref[TEI.att.global.responsibility]{att.global.responsibility} (\textit{@cert}, \textit{@resp})) (\hyperref[TEI.att.global.source]{att.global.source} (\textit{@source}))
    \item[{Membre du}]
  \hyperref[TEI.model.emphLike]{model.emphLike}
    \item[{Contenu dans}]
  
    \item[analysis: ]
   \hyperref[TEI.cl]{cl} \hyperref[TEI.phr]{phr} \hyperref[TEI.s]{s} \hyperref[TEI.span]{span}\par 
    \item[core: ]
   \hyperref[TEI.abbr]{abbr} \hyperref[TEI.add]{add} \hyperref[TEI.addrLine]{addrLine} \hyperref[TEI.author]{author} \hyperref[TEI.bibl]{bibl} \hyperref[TEI.biblScope]{biblScope} \hyperref[TEI.citedRange]{citedRange} \hyperref[TEI.corr]{corr} \hyperref[TEI.date]{date} \hyperref[TEI.del]{del} \hyperref[TEI.desc]{desc} \hyperref[TEI.distinct]{distinct} \hyperref[TEI.editor]{editor} \hyperref[TEI.email]{email} \hyperref[TEI.emph]{emph} \hyperref[TEI.expan]{expan} \hyperref[TEI.foreign]{foreign} \hyperref[TEI.gloss]{gloss} \hyperref[TEI.head]{head} \hyperref[TEI.headItem]{headItem} \hyperref[TEI.headLabel]{headLabel} \hyperref[TEI.hi]{hi} \hyperref[TEI.item]{item} \hyperref[TEI.l]{l} \hyperref[TEI.label]{label} \hyperref[TEI.measure]{measure} \hyperref[TEI.meeting]{meeting} \hyperref[TEI.mentioned]{mentioned} \hyperref[TEI.name]{name} \hyperref[TEI.note]{note} \hyperref[TEI.num]{num} \hyperref[TEI.orig]{orig} \hyperref[TEI.p]{p} \hyperref[TEI.pubPlace]{pubPlace} \hyperref[TEI.publisher]{publisher} \hyperref[TEI.q]{q} \hyperref[TEI.quote]{quote} \hyperref[TEI.ref]{ref} \hyperref[TEI.reg]{reg} \hyperref[TEI.resp]{resp} \hyperref[TEI.rs]{rs} \hyperref[TEI.said]{said} \hyperref[TEI.sic]{sic} \hyperref[TEI.soCalled]{soCalled} \hyperref[TEI.speaker]{speaker} \hyperref[TEI.stage]{stage} \hyperref[TEI.street]{street} \hyperref[TEI.term]{term} \hyperref[TEI.textLang]{textLang} \hyperref[TEI.time]{time} \hyperref[TEI.title]{title} \hyperref[TEI.unclear]{unclear}\par 
    \item[figures: ]
   \hyperref[TEI.cell]{cell} \hyperref[TEI.figDesc]{figDesc}\par 
    \item[header: ]
   \hyperref[TEI.authority]{authority} \hyperref[TEI.change]{change} \hyperref[TEI.classCode]{classCode} \hyperref[TEI.creation]{creation} \hyperref[TEI.distributor]{distributor} \hyperref[TEI.edition]{edition} \hyperref[TEI.extent]{extent} \hyperref[TEI.funder]{funder} \hyperref[TEI.language]{language} \hyperref[TEI.licence]{licence} \hyperref[TEI.rendition]{rendition}\par 
    \item[iso-fs: ]
   \hyperref[TEI.fDescr]{fDescr} \hyperref[TEI.fsDescr]{fsDescr}\par 
    \item[linking: ]
   \hyperref[TEI.ab]{ab} \hyperref[TEI.seg]{seg}\par 
    \item[msdescription: ]
   \hyperref[TEI.accMat]{accMat} \hyperref[TEI.acquisition]{acquisition} \hyperref[TEI.additions]{additions} \hyperref[TEI.catchwords]{catchwords} \hyperref[TEI.collation]{collation} \hyperref[TEI.colophon]{colophon} \hyperref[TEI.condition]{condition} \hyperref[TEI.custEvent]{custEvent} \hyperref[TEI.decoNote]{decoNote} \hyperref[TEI.explicit]{explicit} \hyperref[TEI.filiation]{filiation} \hyperref[TEI.finalRubric]{finalRubric} \hyperref[TEI.foliation]{foliation} \hyperref[TEI.heraldry]{heraldry} \hyperref[TEI.incipit]{incipit} \hyperref[TEI.layout]{layout} \hyperref[TEI.material]{material} \hyperref[TEI.musicNotation]{musicNotation} \hyperref[TEI.objectType]{objectType} \hyperref[TEI.origDate]{origDate} \hyperref[TEI.origPlace]{origPlace} \hyperref[TEI.origin]{origin} \hyperref[TEI.provenance]{provenance} \hyperref[TEI.rubric]{rubric} \hyperref[TEI.secFol]{secFol} \hyperref[TEI.signatures]{signatures} \hyperref[TEI.source]{source} \hyperref[TEI.stamp]{stamp} \hyperref[TEI.summary]{summary} \hyperref[TEI.support]{support} \hyperref[TEI.surrogates]{surrogates} \hyperref[TEI.typeNote]{typeNote} \hyperref[TEI.watermark]{watermark}\par 
    \item[namesdates: ]
   \hyperref[TEI.addName]{addName} \hyperref[TEI.affiliation]{affiliation} \hyperref[TEI.country]{country} \hyperref[TEI.forename]{forename} \hyperref[TEI.genName]{genName} \hyperref[TEI.geogName]{geogName} \hyperref[TEI.nameLink]{nameLink} \hyperref[TEI.orgName]{orgName} \hyperref[TEI.persName]{persName} \hyperref[TEI.placeName]{placeName} \hyperref[TEI.region]{region} \hyperref[TEI.roleName]{roleName} \hyperref[TEI.settlement]{settlement} \hyperref[TEI.surname]{surname}\par 
    \item[textstructure: ]
   \hyperref[TEI.docAuthor]{docAuthor} \hyperref[TEI.docDate]{docDate} \hyperref[TEI.docEdition]{docEdition} \hyperref[TEI.titlePart]{titlePart}\par 
    \item[transcr: ]
   \hyperref[TEI.damage]{damage} \hyperref[TEI.fw]{fw} \hyperref[TEI.metamark]{metamark} \hyperref[TEI.mod]{mod} \hyperref[TEI.restore]{restore} \hyperref[TEI.retrace]{retrace} \hyperref[TEI.secl]{secl} \hyperref[TEI.supplied]{supplied} \hyperref[TEI.surplus]{surplus}
    \item[{Peut contenir}]
  
    \item[analysis: ]
   \hyperref[TEI.c]{c} \hyperref[TEI.cl]{cl} \hyperref[TEI.interp]{interp} \hyperref[TEI.interpGrp]{interpGrp} \hyperref[TEI.m]{m} \hyperref[TEI.pc]{pc} \hyperref[TEI.phr]{phr} \hyperref[TEI.s]{s} \hyperref[TEI.span]{span} \hyperref[TEI.spanGrp]{spanGrp} \hyperref[TEI.w]{w}\par 
    \item[core: ]
   \hyperref[TEI.abbr]{abbr} \hyperref[TEI.add]{add} \hyperref[TEI.address]{address} \hyperref[TEI.binaryObject]{binaryObject} \hyperref[TEI.cb]{cb} \hyperref[TEI.choice]{choice} \hyperref[TEI.corr]{corr} \hyperref[TEI.date]{date} \hyperref[TEI.del]{del} \hyperref[TEI.distinct]{distinct} \hyperref[TEI.email]{email} \hyperref[TEI.emph]{emph} \hyperref[TEI.expan]{expan} \hyperref[TEI.foreign]{foreign} \hyperref[TEI.gap]{gap} \hyperref[TEI.gb]{gb} \hyperref[TEI.gloss]{gloss} \hyperref[TEI.graphic]{graphic} \hyperref[TEI.hi]{hi} \hyperref[TEI.index]{index} \hyperref[TEI.lb]{lb} \hyperref[TEI.measure]{measure} \hyperref[TEI.measureGrp]{measureGrp} \hyperref[TEI.media]{media} \hyperref[TEI.mentioned]{mentioned} \hyperref[TEI.milestone]{milestone} \hyperref[TEI.name]{name} \hyperref[TEI.note]{note} \hyperref[TEI.num]{num} \hyperref[TEI.orig]{orig} \hyperref[TEI.pb]{pb} \hyperref[TEI.ptr]{ptr} \hyperref[TEI.ref]{ref} \hyperref[TEI.reg]{reg} \hyperref[TEI.rs]{rs} \hyperref[TEI.sic]{sic} \hyperref[TEI.soCalled]{soCalled} \hyperref[TEI.term]{term} \hyperref[TEI.time]{time} \hyperref[TEI.title]{title} \hyperref[TEI.unclear]{unclear}\par 
    \item[derived-module-tei.istex: ]
   \hyperref[TEI.math]{math} \hyperref[TEI.mrow]{mrow}\par 
    \item[figures: ]
   \hyperref[TEI.figure]{figure} \hyperref[TEI.formula]{formula} \hyperref[TEI.notatedMusic]{notatedMusic}\par 
    \item[header: ]
   \hyperref[TEI.idno]{idno}\par 
    \item[iso-fs: ]
   \hyperref[TEI.fLib]{fLib} \hyperref[TEI.fs]{fs} \hyperref[TEI.fvLib]{fvLib}\par 
    \item[linking: ]
   \hyperref[TEI.alt]{alt} \hyperref[TEI.altGrp]{altGrp} \hyperref[TEI.anchor]{anchor} \hyperref[TEI.join]{join} \hyperref[TEI.joinGrp]{joinGrp} \hyperref[TEI.link]{link} \hyperref[TEI.linkGrp]{linkGrp} \hyperref[TEI.seg]{seg} \hyperref[TEI.timeline]{timeline}\par 
    \item[msdescription: ]
   \hyperref[TEI.catchwords]{catchwords} \hyperref[TEI.depth]{depth} \hyperref[TEI.dim]{dim} \hyperref[TEI.dimensions]{dimensions} \hyperref[TEI.height]{height} \hyperref[TEI.heraldry]{heraldry} \hyperref[TEI.locus]{locus} \hyperref[TEI.locusGrp]{locusGrp} \hyperref[TEI.material]{material} \hyperref[TEI.objectType]{objectType} \hyperref[TEI.origDate]{origDate} \hyperref[TEI.origPlace]{origPlace} \hyperref[TEI.secFol]{secFol} \hyperref[TEI.signatures]{signatures} \hyperref[TEI.source]{source} \hyperref[TEI.stamp]{stamp} \hyperref[TEI.watermark]{watermark} \hyperref[TEI.width]{width}\par 
    \item[namesdates: ]
   \hyperref[TEI.addName]{addName} \hyperref[TEI.affiliation]{affiliation} \hyperref[TEI.country]{country} \hyperref[TEI.forename]{forename} \hyperref[TEI.genName]{genName} \hyperref[TEI.geogName]{geogName} \hyperref[TEI.location]{location} \hyperref[TEI.nameLink]{nameLink} \hyperref[TEI.orgName]{orgName} \hyperref[TEI.persName]{persName} \hyperref[TEI.placeName]{placeName} \hyperref[TEI.region]{region} \hyperref[TEI.roleName]{roleName} \hyperref[TEI.settlement]{settlement} \hyperref[TEI.state]{state} \hyperref[TEI.surname]{surname}\par 
    \item[spoken: ]
   \hyperref[TEI.annotationBlock]{annotationBlock}\par 
    \item[transcr: ]
   \hyperref[TEI.addSpan]{addSpan} \hyperref[TEI.am]{am} \hyperref[TEI.damage]{damage} \hyperref[TEI.damageSpan]{damageSpan} \hyperref[TEI.delSpan]{delSpan} \hyperref[TEI.ex]{ex} \hyperref[TEI.fw]{fw} \hyperref[TEI.handShift]{handShift} \hyperref[TEI.listTranspose]{listTranspose} \hyperref[TEI.metamark]{metamark} \hyperref[TEI.mod]{mod} \hyperref[TEI.redo]{redo} \hyperref[TEI.restore]{restore} \hyperref[TEI.retrace]{retrace} \hyperref[TEI.secl]{secl} \hyperref[TEI.space]{space} \hyperref[TEI.subst]{subst} \hyperref[TEI.substJoin]{substJoin} \hyperref[TEI.supplied]{supplied} \hyperref[TEI.surplus]{surplus} \hyperref[TEI.undo]{undo}\par des données textuelles
    \item[{Note}]
  \par
L'attribut global {\itshape xml:lang} doit être fourni dans cet élément pour identifier la langue à laquelle appartiennent le mot ou l'expression balisée. Comme ailleurs, sa valeur devrait être une balise de langue définie dans \xref{http://www.tei-c.org/release/doc/tei-p5-doc/en/html/CH.html\#CHSH}{6.1. Language Identification}.\par
Cet élément n'est utilisé que lorsqu'il n'y a pas d'autre élément disponible pour baliser l'expression ou les mots concernés. L'attribut global {\itshape xml:lang} doit être préféré à cet élément lorsqu'il s'agit de caractériser la langue de l'ensemble d'un élément textuel.\par
L'élément \hyperref[TEI.distinct]{<distinct>} peut être utilisé pour identifier des expressions appartenant à des variétés de langue ou à des registres qui ne sont pas généralement considérés comme de vraies langues.
    \item[{Exemple}]
  \leavevmode\bgroup\exampleFont \begin{shaded}\noindent\mbox{}{<\textbf{foreign}\hspace*{6pt}{xml:lang}="{la}">}Et vobis{</\textbf{foreign}>} messieurs,\mbox{}\newline 
 Ce ne seroyt que bon que nous rendissiez noz cloches...\end{shaded}\egroup 


    \item[{Exemple}]
  \leavevmode\bgroup\exampleFont \begin{shaded}\noindent\mbox{}{<\textbf{p}>}Pendant ce temps-là, dans le bureau du rez- de-chaussée, les secrétaires faisaient du\mbox{}\newline 
{<\textbf{foreign}\hspace*{6pt}{xml:lang}="{en}">}hulla-hoop{</\textbf{foreign}>}.{</\textbf{p}>}\end{shaded}\egroup 


    \item[{Modèle de contenu}]
  \mbox{}\hfill\\[-10pt]\begin{Verbatim}[fontsize=\small]
<content>
 <macroRef key="macro.phraseSeq"/>
</content>
    
\end{Verbatim}

    \item[{Schéma Declaration}]
  \mbox{}\hfill\\[-10pt]\begin{Verbatim}[fontsize=\small]
element foreign { tei_att.global.attributes, tei_macro.phraseSeq }
\end{Verbatim}

\end{reflist}  \index{forename=<forename>|oddindex}
\begin{reflist}
\item[]\begin{specHead}{TEI.forename}{<forename> }(prénom) contient un prénom, qu'il soit donné ou un nom de baptême. [\xref{http://www.tei-c.org/release/doc/tei-p5-doc/en/html/ND.html\#NDPER}{13.2.1. Personal Names}]\end{specHead} 
    \item[{Module}]
  namesdates
    \item[{Attributs}]
  Attributs \hyperref[TEI.att.global]{att.global} (\textit{@xml:id}, \textit{@n}, \textit{@xml:lang}, \textit{@xml:base}, \textit{@xml:space})  (\hyperref[TEI.att.global.rendition]{att.global.rendition} (\textit{@rend}, \textit{@style}, \textit{@rendition})) (\hyperref[TEI.att.global.linking]{att.global.linking} (\textit{@corresp}, \textit{@synch}, \textit{@sameAs}, \textit{@copyOf}, \textit{@next}, \textit{@prev}, \textit{@exclude}, \textit{@select})) (\hyperref[TEI.att.global.analytic]{att.global.analytic} (\textit{@ana})) (\hyperref[TEI.att.global.facs]{att.global.facs} (\textit{@facs})) (\hyperref[TEI.att.global.change]{att.global.change} (\textit{@change})) (\hyperref[TEI.att.global.responsibility]{att.global.responsibility} (\textit{@cert}, \textit{@resp})) (\hyperref[TEI.att.global.source]{att.global.source} (\textit{@source})) \hyperref[TEI.att.personal]{att.personal} (\textit{@full}, \textit{@sort})  (\hyperref[TEI.att.naming]{att.naming} (\textit{@role}, \textit{@nymRef}) (\hyperref[TEI.att.canonical]{att.canonical} (\textit{@key}, \textit{@ref})) ) \hyperref[TEI.att.typed]{att.typed} (\textit{@type}, \textit{@subtype}) 
    \item[{Membre du}]
  \hyperref[TEI.model.persNamePart]{model.persNamePart}
    \item[{Contenu dans}]
  
    \item[analysis: ]
   \hyperref[TEI.cl]{cl} \hyperref[TEI.phr]{phr} \hyperref[TEI.s]{s} \hyperref[TEI.span]{span}\par 
    \item[core: ]
   \hyperref[TEI.abbr]{abbr} \hyperref[TEI.add]{add} \hyperref[TEI.addrLine]{addrLine} \hyperref[TEI.address]{address} \hyperref[TEI.author]{author} \hyperref[TEI.bibl]{bibl} \hyperref[TEI.biblScope]{biblScope} \hyperref[TEI.citedRange]{citedRange} \hyperref[TEI.corr]{corr} \hyperref[TEI.date]{date} \hyperref[TEI.del]{del} \hyperref[TEI.desc]{desc} \hyperref[TEI.distinct]{distinct} \hyperref[TEI.editor]{editor} \hyperref[TEI.email]{email} \hyperref[TEI.emph]{emph} \hyperref[TEI.expan]{expan} \hyperref[TEI.foreign]{foreign} \hyperref[TEI.gloss]{gloss} \hyperref[TEI.head]{head} \hyperref[TEI.headItem]{headItem} \hyperref[TEI.headLabel]{headLabel} \hyperref[TEI.hi]{hi} \hyperref[TEI.item]{item} \hyperref[TEI.l]{l} \hyperref[TEI.label]{label} \hyperref[TEI.measure]{measure} \hyperref[TEI.meeting]{meeting} \hyperref[TEI.mentioned]{mentioned} \hyperref[TEI.name]{name} \hyperref[TEI.note]{note} \hyperref[TEI.num]{num} \hyperref[TEI.orig]{orig} \hyperref[TEI.p]{p} \hyperref[TEI.pubPlace]{pubPlace} \hyperref[TEI.publisher]{publisher} \hyperref[TEI.q]{q} \hyperref[TEI.quote]{quote} \hyperref[TEI.ref]{ref} \hyperref[TEI.reg]{reg} \hyperref[TEI.resp]{resp} \hyperref[TEI.rs]{rs} \hyperref[TEI.said]{said} \hyperref[TEI.sic]{sic} \hyperref[TEI.soCalled]{soCalled} \hyperref[TEI.speaker]{speaker} \hyperref[TEI.stage]{stage} \hyperref[TEI.street]{street} \hyperref[TEI.term]{term} \hyperref[TEI.textLang]{textLang} \hyperref[TEI.time]{time} \hyperref[TEI.title]{title} \hyperref[TEI.unclear]{unclear}\par 
    \item[figures: ]
   \hyperref[TEI.cell]{cell} \hyperref[TEI.figDesc]{figDesc}\par 
    \item[header: ]
   \hyperref[TEI.authority]{authority} \hyperref[TEI.change]{change} \hyperref[TEI.classCode]{classCode} \hyperref[TEI.creation]{creation} \hyperref[TEI.distributor]{distributor} \hyperref[TEI.edition]{edition} \hyperref[TEI.extent]{extent} \hyperref[TEI.funder]{funder} \hyperref[TEI.language]{language} \hyperref[TEI.licence]{licence} \hyperref[TEI.rendition]{rendition}\par 
    \item[iso-fs: ]
   \hyperref[TEI.fDescr]{fDescr} \hyperref[TEI.fsDescr]{fsDescr}\par 
    \item[linking: ]
   \hyperref[TEI.ab]{ab} \hyperref[TEI.seg]{seg}\par 
    \item[msdescription: ]
   \hyperref[TEI.accMat]{accMat} \hyperref[TEI.acquisition]{acquisition} \hyperref[TEI.additions]{additions} \hyperref[TEI.catchwords]{catchwords} \hyperref[TEI.collation]{collation} \hyperref[TEI.colophon]{colophon} \hyperref[TEI.condition]{condition} \hyperref[TEI.custEvent]{custEvent} \hyperref[TEI.decoNote]{decoNote} \hyperref[TEI.explicit]{explicit} \hyperref[TEI.filiation]{filiation} \hyperref[TEI.finalRubric]{finalRubric} \hyperref[TEI.foliation]{foliation} \hyperref[TEI.heraldry]{heraldry} \hyperref[TEI.incipit]{incipit} \hyperref[TEI.layout]{layout} \hyperref[TEI.material]{material} \hyperref[TEI.musicNotation]{musicNotation} \hyperref[TEI.objectType]{objectType} \hyperref[TEI.origDate]{origDate} \hyperref[TEI.origPlace]{origPlace} \hyperref[TEI.origin]{origin} \hyperref[TEI.provenance]{provenance} \hyperref[TEI.rubric]{rubric} \hyperref[TEI.secFol]{secFol} \hyperref[TEI.signatures]{signatures} \hyperref[TEI.source]{source} \hyperref[TEI.stamp]{stamp} \hyperref[TEI.summary]{summary} \hyperref[TEI.support]{support} \hyperref[TEI.surrogates]{surrogates} \hyperref[TEI.typeNote]{typeNote} \hyperref[TEI.watermark]{watermark}\par 
    \item[namesdates: ]
   \hyperref[TEI.addName]{addName} \hyperref[TEI.affiliation]{affiliation} \hyperref[TEI.country]{country} \hyperref[TEI.forename]{forename} \hyperref[TEI.genName]{genName} \hyperref[TEI.geogName]{geogName} \hyperref[TEI.nameLink]{nameLink} \hyperref[TEI.org]{org} \hyperref[TEI.orgName]{orgName} \hyperref[TEI.persName]{persName} \hyperref[TEI.placeName]{placeName} \hyperref[TEI.region]{region} \hyperref[TEI.roleName]{roleName} \hyperref[TEI.settlement]{settlement} \hyperref[TEI.surname]{surname}\par 
    \item[spoken: ]
   \hyperref[TEI.annotationBlock]{annotationBlock}\par 
    \item[standOff: ]
   \hyperref[TEI.listAnnotation]{listAnnotation}\par 
    \item[textstructure: ]
   \hyperref[TEI.docAuthor]{docAuthor} \hyperref[TEI.docDate]{docDate} \hyperref[TEI.docEdition]{docEdition} \hyperref[TEI.titlePart]{titlePart}\par 
    \item[transcr: ]
   \hyperref[TEI.damage]{damage} \hyperref[TEI.fw]{fw} \hyperref[TEI.metamark]{metamark} \hyperref[TEI.mod]{mod} \hyperref[TEI.restore]{restore} \hyperref[TEI.retrace]{retrace} \hyperref[TEI.secl]{secl} \hyperref[TEI.supplied]{supplied} \hyperref[TEI.surplus]{surplus}
    \item[{Peut contenir}]
  
    \item[analysis: ]
   \hyperref[TEI.c]{c} \hyperref[TEI.cl]{cl} \hyperref[TEI.interp]{interp} \hyperref[TEI.interpGrp]{interpGrp} \hyperref[TEI.m]{m} \hyperref[TEI.pc]{pc} \hyperref[TEI.phr]{phr} \hyperref[TEI.s]{s} \hyperref[TEI.span]{span} \hyperref[TEI.spanGrp]{spanGrp} \hyperref[TEI.w]{w}\par 
    \item[core: ]
   \hyperref[TEI.abbr]{abbr} \hyperref[TEI.add]{add} \hyperref[TEI.address]{address} \hyperref[TEI.binaryObject]{binaryObject} \hyperref[TEI.cb]{cb} \hyperref[TEI.choice]{choice} \hyperref[TEI.corr]{corr} \hyperref[TEI.date]{date} \hyperref[TEI.del]{del} \hyperref[TEI.distinct]{distinct} \hyperref[TEI.email]{email} \hyperref[TEI.emph]{emph} \hyperref[TEI.expan]{expan} \hyperref[TEI.foreign]{foreign} \hyperref[TEI.gap]{gap} \hyperref[TEI.gb]{gb} \hyperref[TEI.gloss]{gloss} \hyperref[TEI.graphic]{graphic} \hyperref[TEI.hi]{hi} \hyperref[TEI.index]{index} \hyperref[TEI.lb]{lb} \hyperref[TEI.measure]{measure} \hyperref[TEI.measureGrp]{measureGrp} \hyperref[TEI.media]{media} \hyperref[TEI.mentioned]{mentioned} \hyperref[TEI.milestone]{milestone} \hyperref[TEI.name]{name} \hyperref[TEI.note]{note} \hyperref[TEI.num]{num} \hyperref[TEI.orig]{orig} \hyperref[TEI.pb]{pb} \hyperref[TEI.ptr]{ptr} \hyperref[TEI.ref]{ref} \hyperref[TEI.reg]{reg} \hyperref[TEI.rs]{rs} \hyperref[TEI.sic]{sic} \hyperref[TEI.soCalled]{soCalled} \hyperref[TEI.term]{term} \hyperref[TEI.time]{time} \hyperref[TEI.title]{title} \hyperref[TEI.unclear]{unclear}\par 
    \item[derived-module-tei.istex: ]
   \hyperref[TEI.math]{math} \hyperref[TEI.mrow]{mrow}\par 
    \item[figures: ]
   \hyperref[TEI.figure]{figure} \hyperref[TEI.formula]{formula} \hyperref[TEI.notatedMusic]{notatedMusic}\par 
    \item[header: ]
   \hyperref[TEI.idno]{idno}\par 
    \item[iso-fs: ]
   \hyperref[TEI.fLib]{fLib} \hyperref[TEI.fs]{fs} \hyperref[TEI.fvLib]{fvLib}\par 
    \item[linking: ]
   \hyperref[TEI.alt]{alt} \hyperref[TEI.altGrp]{altGrp} \hyperref[TEI.anchor]{anchor} \hyperref[TEI.join]{join} \hyperref[TEI.joinGrp]{joinGrp} \hyperref[TEI.link]{link} \hyperref[TEI.linkGrp]{linkGrp} \hyperref[TEI.seg]{seg} \hyperref[TEI.timeline]{timeline}\par 
    \item[msdescription: ]
   \hyperref[TEI.catchwords]{catchwords} \hyperref[TEI.depth]{depth} \hyperref[TEI.dim]{dim} \hyperref[TEI.dimensions]{dimensions} \hyperref[TEI.height]{height} \hyperref[TEI.heraldry]{heraldry} \hyperref[TEI.locus]{locus} \hyperref[TEI.locusGrp]{locusGrp} \hyperref[TEI.material]{material} \hyperref[TEI.objectType]{objectType} \hyperref[TEI.origDate]{origDate} \hyperref[TEI.origPlace]{origPlace} \hyperref[TEI.secFol]{secFol} \hyperref[TEI.signatures]{signatures} \hyperref[TEI.source]{source} \hyperref[TEI.stamp]{stamp} \hyperref[TEI.watermark]{watermark} \hyperref[TEI.width]{width}\par 
    \item[namesdates: ]
   \hyperref[TEI.addName]{addName} \hyperref[TEI.affiliation]{affiliation} \hyperref[TEI.country]{country} \hyperref[TEI.forename]{forename} \hyperref[TEI.genName]{genName} \hyperref[TEI.geogName]{geogName} \hyperref[TEI.location]{location} \hyperref[TEI.nameLink]{nameLink} \hyperref[TEI.orgName]{orgName} \hyperref[TEI.persName]{persName} \hyperref[TEI.placeName]{placeName} \hyperref[TEI.region]{region} \hyperref[TEI.roleName]{roleName} \hyperref[TEI.settlement]{settlement} \hyperref[TEI.state]{state} \hyperref[TEI.surname]{surname}\par 
    \item[spoken: ]
   \hyperref[TEI.annotationBlock]{annotationBlock}\par 
    \item[transcr: ]
   \hyperref[TEI.addSpan]{addSpan} \hyperref[TEI.am]{am} \hyperref[TEI.damage]{damage} \hyperref[TEI.damageSpan]{damageSpan} \hyperref[TEI.delSpan]{delSpan} \hyperref[TEI.ex]{ex} \hyperref[TEI.fw]{fw} \hyperref[TEI.handShift]{handShift} \hyperref[TEI.listTranspose]{listTranspose} \hyperref[TEI.metamark]{metamark} \hyperref[TEI.mod]{mod} \hyperref[TEI.redo]{redo} \hyperref[TEI.restore]{restore} \hyperref[TEI.retrace]{retrace} \hyperref[TEI.secl]{secl} \hyperref[TEI.space]{space} \hyperref[TEI.subst]{subst} \hyperref[TEI.substJoin]{substJoin} \hyperref[TEI.supplied]{supplied} \hyperref[TEI.surplus]{surplus} \hyperref[TEI.undo]{undo}\par des données textuelles
    \item[{Exemple}]
  \leavevmode\bgroup\exampleFont \begin{shaded}\noindent\mbox{}{<\textbf{persName}>}\mbox{}\newline 
\hspace*{6pt}{<\textbf{roleName}>}Ex-Président{</\textbf{roleName}>}\mbox{}\newline 
\hspace*{6pt}{<\textbf{forename}>}George{</\textbf{forename}>}\mbox{}\newline 
\hspace*{6pt}{<\textbf{surname}>}Bush{</\textbf{surname}>}\mbox{}\newline 
{</\textbf{persName}>}\end{shaded}\egroup 


    \item[{Modèle de contenu}]
  \mbox{}\hfill\\[-10pt]\begin{Verbatim}[fontsize=\small]
<content>
 <macroRef key="macro.phraseSeq"/>
</content>
    
\end{Verbatim}

    \item[{Schéma Declaration}]
  \mbox{}\hfill\\[-10pt]\begin{Verbatim}[fontsize=\small]
element forename
{
   tei_att.global.attributes,
   tei_att.personal.attributes,
   tei_att.typed.attributes,
   tei_macro.phraseSeq}
\end{Verbatim}

\end{reflist}  \index{formula=<formula>|oddindex}
\begin{reflist}
\item[]\begin{specHead}{TEI.formula}{<formula> }(formule) contient une formule mathématique ou tout autre type de formule [\xref{http://www.tei-c.org/release/doc/tei-p5-doc/en/html/FT.html\#FTFOR}{14.2. Formulæ and Mathematical Expressions}]\end{specHead} 
    \item[{Module}]
  figures
    \item[{Attributs}]
  Attributs \hyperref[TEI.att.global]{att.global} (\textit{@xml:id}, \textit{@n}, \textit{@xml:lang}, \textit{@xml:base}, \textit{@xml:space})  (\hyperref[TEI.att.global.rendition]{att.global.rendition} (\textit{@rend}, \textit{@style}, \textit{@rendition})) (\hyperref[TEI.att.global.linking]{att.global.linking} (\textit{@corresp}, \textit{@synch}, \textit{@sameAs}, \textit{@copyOf}, \textit{@next}, \textit{@prev}, \textit{@exclude}, \textit{@select})) (\hyperref[TEI.att.global.analytic]{att.global.analytic} (\textit{@ana})) (\hyperref[TEI.att.global.facs]{att.global.facs} (\textit{@facs})) (\hyperref[TEI.att.global.change]{att.global.change} (\textit{@change})) (\hyperref[TEI.att.global.responsibility]{att.global.responsibility} (\textit{@cert}, \textit{@resp})) (\hyperref[TEI.att.global.source]{att.global.source} (\textit{@source})) \hyperref[TEI.att.notated]{att.notated} (\textit{@notation}) 
    \item[{Membre du}]
  \hyperref[TEI.model.graphicLike]{model.graphicLike}
    \item[{Contenu dans}]
  
    \item[analysis: ]
   \hyperref[TEI.cl]{cl} \hyperref[TEI.phr]{phr} \hyperref[TEI.s]{s}\par 
    \item[core: ]
   \hyperref[TEI.abbr]{abbr} \hyperref[TEI.add]{add} \hyperref[TEI.addrLine]{addrLine} \hyperref[TEI.author]{author} \hyperref[TEI.biblScope]{biblScope} \hyperref[TEI.citedRange]{citedRange} \hyperref[TEI.corr]{corr} \hyperref[TEI.date]{date} \hyperref[TEI.del]{del} \hyperref[TEI.distinct]{distinct} \hyperref[TEI.editor]{editor} \hyperref[TEI.email]{email} \hyperref[TEI.emph]{emph} \hyperref[TEI.expan]{expan} \hyperref[TEI.foreign]{foreign} \hyperref[TEI.gloss]{gloss} \hyperref[TEI.head]{head} \hyperref[TEI.headItem]{headItem} \hyperref[TEI.headLabel]{headLabel} \hyperref[TEI.hi]{hi} \hyperref[TEI.item]{item} \hyperref[TEI.l]{l} \hyperref[TEI.label]{label} \hyperref[TEI.measure]{measure} \hyperref[TEI.mentioned]{mentioned} \hyperref[TEI.name]{name} \hyperref[TEI.note]{note} \hyperref[TEI.num]{num} \hyperref[TEI.orig]{orig} \hyperref[TEI.p]{p} \hyperref[TEI.pubPlace]{pubPlace} \hyperref[TEI.publisher]{publisher} \hyperref[TEI.q]{q} \hyperref[TEI.quote]{quote} \hyperref[TEI.ref]{ref} \hyperref[TEI.reg]{reg} \hyperref[TEI.rs]{rs} \hyperref[TEI.said]{said} \hyperref[TEI.sic]{sic} \hyperref[TEI.soCalled]{soCalled} \hyperref[TEI.speaker]{speaker} \hyperref[TEI.stage]{stage} \hyperref[TEI.street]{street} \hyperref[TEI.term]{term} \hyperref[TEI.textLang]{textLang} \hyperref[TEI.time]{time} \hyperref[TEI.title]{title} \hyperref[TEI.unclear]{unclear}\par 
    \item[figures: ]
   \hyperref[TEI.cell]{cell} \hyperref[TEI.figDesc]{figDesc} \hyperref[TEI.figure]{figure} \hyperref[TEI.formula]{formula} \hyperref[TEI.table]{table}\par 
    \item[header: ]
   \hyperref[TEI.change]{change} \hyperref[TEI.distributor]{distributor} \hyperref[TEI.edition]{edition} \hyperref[TEI.extent]{extent} \hyperref[TEI.licence]{licence}\par 
    \item[linking: ]
   \hyperref[TEI.ab]{ab} \hyperref[TEI.seg]{seg}\par 
    \item[msdescription: ]
   \hyperref[TEI.accMat]{accMat} \hyperref[TEI.acquisition]{acquisition} \hyperref[TEI.additions]{additions} \hyperref[TEI.catchwords]{catchwords} \hyperref[TEI.collation]{collation} \hyperref[TEI.colophon]{colophon} \hyperref[TEI.condition]{condition} \hyperref[TEI.custEvent]{custEvent} \hyperref[TEI.decoNote]{decoNote} \hyperref[TEI.explicit]{explicit} \hyperref[TEI.filiation]{filiation} \hyperref[TEI.finalRubric]{finalRubric} \hyperref[TEI.foliation]{foliation} \hyperref[TEI.heraldry]{heraldry} \hyperref[TEI.incipit]{incipit} \hyperref[TEI.layout]{layout} \hyperref[TEI.material]{material} \hyperref[TEI.musicNotation]{musicNotation} \hyperref[TEI.objectType]{objectType} \hyperref[TEI.origDate]{origDate} \hyperref[TEI.origPlace]{origPlace} \hyperref[TEI.origin]{origin} \hyperref[TEI.provenance]{provenance} \hyperref[TEI.rubric]{rubric} \hyperref[TEI.secFol]{secFol} \hyperref[TEI.signatures]{signatures} \hyperref[TEI.source]{source} \hyperref[TEI.stamp]{stamp} \hyperref[TEI.summary]{summary} \hyperref[TEI.support]{support} \hyperref[TEI.surrogates]{surrogates} \hyperref[TEI.typeNote]{typeNote} \hyperref[TEI.watermark]{watermark}\par 
    \item[namesdates: ]
   \hyperref[TEI.addName]{addName} \hyperref[TEI.affiliation]{affiliation} \hyperref[TEI.country]{country} \hyperref[TEI.forename]{forename} \hyperref[TEI.genName]{genName} \hyperref[TEI.geogName]{geogName} \hyperref[TEI.nameLink]{nameLink} \hyperref[TEI.orgName]{orgName} \hyperref[TEI.persName]{persName} \hyperref[TEI.placeName]{placeName} \hyperref[TEI.region]{region} \hyperref[TEI.roleName]{roleName} \hyperref[TEI.settlement]{settlement} \hyperref[TEI.surname]{surname}\par 
    \item[textstructure: ]
   \hyperref[TEI.docAuthor]{docAuthor} \hyperref[TEI.docDate]{docDate} \hyperref[TEI.docEdition]{docEdition} \hyperref[TEI.titlePart]{titlePart}\par 
    \item[transcr: ]
   \hyperref[TEI.damage]{damage} \hyperref[TEI.facsimile]{facsimile} \hyperref[TEI.fw]{fw} \hyperref[TEI.metamark]{metamark} \hyperref[TEI.mod]{mod} \hyperref[TEI.restore]{restore} \hyperref[TEI.retrace]{retrace} \hyperref[TEI.secl]{secl} \hyperref[TEI.sourceDoc]{sourceDoc} \hyperref[TEI.supplied]{supplied} \hyperref[TEI.surface]{surface} \hyperref[TEI.surplus]{surplus} \hyperref[TEI.zone]{zone}
    \item[{Peut contenir}]
  
    \item[core: ]
   \hyperref[TEI.binaryObject]{binaryObject} \hyperref[TEI.graphic]{graphic} \hyperref[TEI.hi]{hi} \hyperref[TEI.media]{media}\par 
    \item[derived-module-tei.istex: ]
   \hyperref[TEI.math]{math} \hyperref[TEI.mrow]{mrow}\par 
    \item[figures: ]
   \hyperref[TEI.formula]{formula}\par des données textuelles
    \item[{Exemple}]
  \leavevmode\bgroup\exampleFont \begin{shaded}\noindent\mbox{}{<\textbf{formula}\hspace*{6pt}{notation}="{tex}">}\$E=mc\textasciicircum 2\${</\textbf{formula}>}\end{shaded}\egroup 


    \item[{Exemple}]
  \leavevmode\bgroup\exampleFont \begin{shaded}\noindent\mbox{}{<\textbf{formula}\hspace*{6pt}{notation}="{none}">}E=mc{<\textbf{hi}\hspace*{6pt}{rend}="{sup}">}2{</\textbf{hi}>}\mbox{}\newline 
{</\textbf{formula}>}\end{shaded}\egroup 


    \item[{Exemple}]
  \leavevmode\bgroup\exampleFont \begin{shaded}\noindent\mbox{}{<\textbf{formula}\hspace*{6pt}{notation}="{mathml}">}\mbox{}\newline 
\hspace*{6pt}{<\textbf{m:math}>}\mbox{}\newline 
\hspace*{6pt}\hspace*{6pt}{<\textbf{m:mi}>}E{</\textbf{m:mi}>}\mbox{}\newline 
\hspace*{6pt}\hspace*{6pt}{<\textbf{m:mo}>}={</\textbf{m:mo}>}\mbox{}\newline 
\hspace*{6pt}\hspace*{6pt}{<\textbf{m:mi}>}m{</\textbf{m:mi}>}\mbox{}\newline 
\hspace*{6pt}\hspace*{6pt}{<\textbf{m:msup}>}\mbox{}\newline 
\hspace*{6pt}\hspace*{6pt}\hspace*{6pt}{<\textbf{m:mrow}>}\mbox{}\newline 
\hspace*{6pt}\hspace*{6pt}\hspace*{6pt}\hspace*{6pt}{<\textbf{m:mi}>}c{</\textbf{m:mi}>}\mbox{}\newline 
\hspace*{6pt}\hspace*{6pt}\hspace*{6pt}{</\textbf{m:mrow}>}\mbox{}\newline 
\hspace*{6pt}\hspace*{6pt}\hspace*{6pt}{<\textbf{m:mrow}>}\mbox{}\newline 
\hspace*{6pt}\hspace*{6pt}\hspace*{6pt}\hspace*{6pt}{<\textbf{m:mn}>}2{</\textbf{m:mn}>}\mbox{}\newline 
\hspace*{6pt}\hspace*{6pt}\hspace*{6pt}{</\textbf{m:mrow}>}\mbox{}\newline 
\hspace*{6pt}\hspace*{6pt}{</\textbf{m:msup}>}\mbox{}\newline 
\hspace*{6pt}{</\textbf{m:math}>}\mbox{}\newline 
{</\textbf{formula}>}\end{shaded}\egroup 


    \item[{Modèle de contenu}]
  \mbox{}\hfill\\[-10pt]\begin{Verbatim}[fontsize=\small]
<content>
 <alternate maxOccurs="unbounded"
  minOccurs="0">
  <textNode/>
  <classRef key="model.graphicLike"/>
  <classRef key="model.hiLike"/>
 </alternate>
</content>
    
\end{Verbatim}

    \item[{Schéma Declaration}]
  \mbox{}\hfill\\[-10pt]\begin{Verbatim}[fontsize=\small]
element formula
{
   tei_att.global.attributes,
   tei_att.notated.attributes,
   ( text | tei_model.graphicLike | tei_model.hiLike )*
}
\end{Verbatim}

\end{reflist}  \index{front=<front>|oddindex}
\begin{reflist}
\item[]\begin{specHead}{TEI.front}{<front> }(texte préliminaire) contient tout ce qui est au début du document, avant le corps du texte : page de titre, dédicaces, préfaces, etc. [\xref{http://www.tei-c.org/release/doc/tei-p5-doc/en/html/DS.html\#DSTITL}{4.6. Title Pages} \xref{http://www.tei-c.org/release/doc/tei-p5-doc/en/html/DS.html\#DS}{4. Default Text Structure}]\end{specHead} 
    \item[{Module}]
  textstructure
    \item[{Attributs}]
  Attributs \hyperref[TEI.att.global]{att.global} (\textit{@xml:id}, \textit{@n}, \textit{@xml:lang}, \textit{@xml:base}, \textit{@xml:space})  (\hyperref[TEI.att.global.rendition]{att.global.rendition} (\textit{@rend}, \textit{@style}, \textit{@rendition})) (\hyperref[TEI.att.global.linking]{att.global.linking} (\textit{@corresp}, \textit{@synch}, \textit{@sameAs}, \textit{@copyOf}, \textit{@next}, \textit{@prev}, \textit{@exclude}, \textit{@select})) (\hyperref[TEI.att.global.analytic]{att.global.analytic} (\textit{@ana})) (\hyperref[TEI.att.global.facs]{att.global.facs} (\textit{@facs})) (\hyperref[TEI.att.global.change]{att.global.change} (\textit{@change})) (\hyperref[TEI.att.global.responsibility]{att.global.responsibility} (\textit{@cert}, \textit{@resp})) (\hyperref[TEI.att.global.source]{att.global.source} (\textit{@source})) \hyperref[TEI.att.declaring]{att.declaring} (\textit{@decls}) 
    \item[{Contenu dans}]
  
    \item[textstructure: ]
   \hyperref[TEI.floatingText]{floatingText} \hyperref[TEI.text]{text}\par 
    \item[transcr: ]
   \hyperref[TEI.facsimile]{facsimile}
    \item[{Peut contenir}]
  
    \item[analysis: ]
   \hyperref[TEI.interp]{interp} \hyperref[TEI.interpGrp]{interpGrp} \hyperref[TEI.span]{span} \hyperref[TEI.spanGrp]{spanGrp}\par 
    \item[core: ]
   \hyperref[TEI.cb]{cb} \hyperref[TEI.divGen]{divGen} \hyperref[TEI.gap]{gap} \hyperref[TEI.gb]{gb} \hyperref[TEI.head]{head} \hyperref[TEI.index]{index} \hyperref[TEI.lb]{lb} \hyperref[TEI.listBibl]{listBibl} \hyperref[TEI.meeting]{meeting} \hyperref[TEI.milestone]{milestone} \hyperref[TEI.note]{note} \hyperref[TEI.p]{p} \hyperref[TEI.pb]{pb}\par 
    \item[figures: ]
   \hyperref[TEI.figure]{figure} \hyperref[TEI.notatedMusic]{notatedMusic}\par 
    \item[iso-fs: ]
   \hyperref[TEI.fLib]{fLib} \hyperref[TEI.fs]{fs} \hyperref[TEI.fvLib]{fvLib}\par 
    \item[linking: ]
   \hyperref[TEI.ab]{ab} \hyperref[TEI.alt]{alt} \hyperref[TEI.altGrp]{altGrp} \hyperref[TEI.anchor]{anchor} \hyperref[TEI.join]{join} \hyperref[TEI.joinGrp]{joinGrp} \hyperref[TEI.link]{link} \hyperref[TEI.linkGrp]{linkGrp} \hyperref[TEI.timeline]{timeline}\par 
    \item[msdescription: ]
   \hyperref[TEI.source]{source}\par 
    \item[textstructure: ]
   \hyperref[TEI.div]{div} \hyperref[TEI.docAuthor]{docAuthor} \hyperref[TEI.docDate]{docDate} \hyperref[TEI.docEdition]{docEdition} \hyperref[TEI.docTitle]{docTitle} \hyperref[TEI.titlePage]{titlePage} \hyperref[TEI.titlePart]{titlePart}\par 
    \item[transcr: ]
   \hyperref[TEI.addSpan]{addSpan} \hyperref[TEI.damageSpan]{damageSpan} \hyperref[TEI.delSpan]{delSpan} \hyperref[TEI.fw]{fw} \hyperref[TEI.listTranspose]{listTranspose} \hyperref[TEI.metamark]{metamark} \hyperref[TEI.space]{space} \hyperref[TEI.substJoin]{substJoin}
    \item[{Note}]
  \par
Because cultural conventions differ as to which elements are grouped as front matter and which as back matter, the content models for the \hyperref[TEI.front]{<front>} and \hyperref[TEI.back]{<back>} elements are identical.
    \item[{Exemple}]
  \leavevmode\bgroup\exampleFont \begin{shaded}\noindent\mbox{}{<\textbf{front}>}\mbox{}\newline 
\hspace*{6pt}{<\textbf{epigraph}>}\mbox{}\newline 
\hspace*{6pt}\hspace*{6pt}{<\textbf{quote}>}Nam Sibyllam quidem Cumis ego ipse oculis meis vidi in ampulla\mbox{}\newline 
\hspace*{6pt}\hspace*{6pt}\hspace*{6pt}\hspace*{6pt} pendere, et cum illi pueri dicerent: {<\textbf{q}\hspace*{6pt}{xml:lang}="{gr}">}Σίβυλλα τί\mbox{}\newline 
\hspace*{6pt}\hspace*{6pt}\hspace*{6pt}\hspace*{6pt}\hspace*{6pt}\hspace*{6pt} θέλεις{</\textbf{q}>}; respondebat illa: {<\textbf{q}\hspace*{6pt}{xml:lang}="{gr}">}ὰποθανεῖν θέλω.{</\textbf{q}>}\mbox{}\newline 
\hspace*{6pt}\hspace*{6pt}{</\textbf{quote}>}\mbox{}\newline 
\hspace*{6pt}{</\textbf{epigraph}>}\mbox{}\newline 
\hspace*{6pt}{<\textbf{div}\hspace*{6pt}{type}="{dedication}">}\mbox{}\newline 
\hspace*{6pt}\hspace*{6pt}{<\textbf{p}>}For Ezra Pound {<\textbf{q}\hspace*{6pt}{xml:lang}="{it}">}il miglior fabbro.{</\textbf{q}>}\mbox{}\newline 
\hspace*{6pt}\hspace*{6pt}{</\textbf{p}>}\mbox{}\newline 
\hspace*{6pt}{</\textbf{div}>}\mbox{}\newline 
{</\textbf{front}>}\end{shaded}\egroup 


    \item[{Exemple}]
  \leavevmode\bgroup\exampleFont \begin{shaded}\noindent\mbox{}{<\textbf{front}>}\mbox{}\newline 
\hspace*{6pt}{<\textbf{div}\hspace*{6pt}{type}="{dedication}">}\mbox{}\newline 
\hspace*{6pt}\hspace*{6pt}{<\textbf{p}>}à la mémoire de Raymond Queneau{</\textbf{p}>}\mbox{}\newline 
\hspace*{6pt}{</\textbf{div}>}\mbox{}\newline 
\hspace*{6pt}{<\textbf{div}\hspace*{6pt}{type}="{avertissement}">}\mbox{}\newline 
\hspace*{6pt}\hspace*{6pt}{<\textbf{p}>}L'amitié, l'histoire et la littérature m'ont fourni quelques-uns\mbox{}\newline 
\hspace*{6pt}\hspace*{6pt}\hspace*{6pt}\hspace*{6pt} des.personnages de ce livre. Toute autre ressemblance avec des\mbox{}\newline 
\hspace*{6pt}\hspace*{6pt}\hspace*{6pt}\hspace*{6pt} individus vivants ou ayant réellement ou fictivement existé ne\mbox{}\newline 
\hspace*{6pt}\hspace*{6pt}\hspace*{6pt}\hspace*{6pt} saurait être que coïncidence.{</\textbf{p}>}\mbox{}\newline 
\hspace*{6pt}\hspace*{6pt}{<\textbf{epigraph}>}\mbox{}\newline 
\hspace*{6pt}\hspace*{6pt}\hspace*{6pt}{<\textbf{quote}>}Regarde de tous tes yeux, regarde {<\textbf{bibl}>}(Jules Verne, Michel\mbox{}\newline 
\hspace*{6pt}\hspace*{6pt}\hspace*{6pt}\hspace*{6pt}\hspace*{6pt}\hspace*{6pt}\hspace*{6pt}\hspace*{6pt} Strogoff ){</\textbf{bibl}>}\mbox{}\newline 
\hspace*{6pt}\hspace*{6pt}\hspace*{6pt}{</\textbf{quote}>}\mbox{}\newline 
\hspace*{6pt}\hspace*{6pt}{</\textbf{epigraph}>}\mbox{}\newline 
\hspace*{6pt}{</\textbf{div}>}\mbox{}\newline 
\hspace*{6pt}{<\textbf{div}\hspace*{6pt}{type}="{preambule}">}\mbox{}\newline 
\hspace*{6pt}\hspace*{6pt}{<\textbf{head}>}PRÉAMBULE{</\textbf{head}>}\mbox{}\newline 
\hspace*{6pt}\hspace*{6pt}{<\textbf{epigraph}>}\mbox{}\newline 
\hspace*{6pt}\hspace*{6pt}\hspace*{6pt}{<\textbf{quote}>}\mbox{}\newline 
\hspace*{6pt}\hspace*{6pt}\hspace*{6pt}\hspace*{6pt}{<\textbf{q}>}L'œil suit les chemins qui lui ont été ménagés dans l'oeuvre\mbox{}\newline 
\hspace*{6pt}\hspace*{6pt}\hspace*{6pt}\hspace*{6pt}{<\textbf{bibl}>}(Paul Klee, Pädagosisches Skizzenbuch){</\textbf{bibl}>}\mbox{}\newline 
\hspace*{6pt}\hspace*{6pt}\hspace*{6pt}\hspace*{6pt}{</\textbf{q}>}\mbox{}\newline 
\hspace*{6pt}\hspace*{6pt}\hspace*{6pt}{</\textbf{quote}>}\mbox{}\newline 
\hspace*{6pt}\hspace*{6pt}{</\textbf{epigraph}>}\mbox{}\newline 
\hspace*{6pt}\hspace*{6pt}{<\textbf{p}>} Au départ, l'art du puzzle semble un art bref, un art mince, tout\mbox{}\newline 
\hspace*{6pt}\hspace*{6pt}\hspace*{6pt}\hspace*{6pt} entier contenu dans un maigre enseignement de la Gestalttheorie :\mbox{}\newline 
\hspace*{6pt}\hspace*{6pt}\hspace*{6pt}\hspace*{6pt} ...{</\textbf{p}>}\mbox{}\newline 
\hspace*{6pt}{</\textbf{div}>}\mbox{}\newline 
{</\textbf{front}>}\end{shaded}\egroup 


    \item[{Exemple}]
  \leavevmode\bgroup\exampleFont \begin{shaded}\noindent\mbox{}{<\textbf{front}>}\mbox{}\newline 
\hspace*{6pt}{<\textbf{div}\hspace*{6pt}{type}="{preface}">}\mbox{}\newline 
\hspace*{6pt}\hspace*{6pt}{<\textbf{head}>}Préface{</\textbf{head}>}\mbox{}\newline 
\hspace*{6pt}\hspace*{6pt}{<\textbf{p}>}Tant qu'il existera, par le fait des lois et des moeurs, une\mbox{}\newline 
\hspace*{6pt}\hspace*{6pt}\hspace*{6pt}\hspace*{6pt} damnation sociale créant artificiellement, en pleine civilisation,\mbox{}\newline 
\hspace*{6pt}\hspace*{6pt}\hspace*{6pt}\hspace*{6pt} des enfers, et compliquant d'une fatalité humaine la destinée qui\mbox{}\newline 
\hspace*{6pt}\hspace*{6pt}\hspace*{6pt}\hspace*{6pt} est divine ; tant que les trois problèmes du siècle, la dégradation\mbox{}\newline 
\hspace*{6pt}\hspace*{6pt}\hspace*{6pt}\hspace*{6pt} de l'homme par le prolétariat, la déchéance de la femme par la faim,\mbox{}\newline 
\hspace*{6pt}\hspace*{6pt}\hspace*{6pt}\hspace*{6pt} l'atrophie de l'enfant par la nuit, ne seront pas résolus; tant que,\mbox{}\newline 
\hspace*{6pt}\hspace*{6pt}\hspace*{6pt}\hspace*{6pt} dans certaines régions, l'asphyxie sociale sera possible; en\mbox{}\newline 
\hspace*{6pt}\hspace*{6pt}\hspace*{6pt}\hspace*{6pt} d'autres termes, et à un point de vue plus étendu encore, tant qu'il\mbox{}\newline 
\hspace*{6pt}\hspace*{6pt}\hspace*{6pt}\hspace*{6pt} aura sur la terre ignorance et misère, des livres de la nature de\mbox{}\newline 
\hspace*{6pt}\hspace*{6pt}\hspace*{6pt}\hspace*{6pt} celui-ci pourront ne pas être inutiles.{</\textbf{p}>}\mbox{}\newline 
\hspace*{6pt}\hspace*{6pt}{<\textbf{closer}>}\mbox{}\newline 
\hspace*{6pt}\hspace*{6pt}\hspace*{6pt}{<\textbf{dateline}>}\mbox{}\newline 
\hspace*{6pt}\hspace*{6pt}\hspace*{6pt}\hspace*{6pt}{<\textbf{name}\hspace*{6pt}{type}="{place}">}Hauteville-House{</\textbf{name}>}\mbox{}\newline 
\hspace*{6pt}\hspace*{6pt}\hspace*{6pt}\hspace*{6pt}{<\textbf{date}>}1er janvier 1862{</\textbf{date}>}\mbox{}\newline 
\hspace*{6pt}\hspace*{6pt}\hspace*{6pt}{</\textbf{dateline}>}\mbox{}\newline 
\hspace*{6pt}\hspace*{6pt}{</\textbf{closer}>}\mbox{}\newline 
\hspace*{6pt}{</\textbf{div}>}\mbox{}\newline 
{</\textbf{front}>}\end{shaded}\egroup 


    \item[{Modèle de contenu}]
  \mbox{}\hfill\\[-10pt]\begin{Verbatim}[fontsize=\small]
<content>
 <sequence maxOccurs="1" minOccurs="1">
  <alternate maxOccurs="unbounded"
   minOccurs="0">
   <classRef key="model.frontPart"/>
   <classRef key="model.pLike"/>
   <classRef key="model.pLike.front"/>
   <classRef key="model.global"/>
  </alternate>
  <sequence maxOccurs="1" minOccurs="0">
   <alternate maxOccurs="1" minOccurs="1">
    <sequence maxOccurs="1" minOccurs="1">
     <classRef key="model.div1Like"/>
     <alternate maxOccurs="unbounded"
      minOccurs="0">
      <classRef key="model.div1Like"/>
      <classRef key="model.frontPart"/>
      <classRef key="model.global"/>
     </alternate>
    </sequence>
    <sequence maxOccurs="1" minOccurs="1">
     <classRef key="model.divLike"/>
     <alternate maxOccurs="unbounded"
      minOccurs="0">
      <classRef key="model.divLike"/>
      <classRef key="model.frontPart"/>
      <classRef key="model.global"/>
     </alternate>
    </sequence>
   </alternate>
   <sequence maxOccurs="1" minOccurs="0">
    <classRef key="model.divBottom"/>
    <alternate maxOccurs="unbounded"
     minOccurs="0">
     <classRef key="model.divBottom"/>
     <classRef key="model.global"/>
    </alternate>
   </sequence>
  </sequence>
 </sequence>
</content>
    
\end{Verbatim}

    \item[{Schéma Declaration}]
  \mbox{}\hfill\\[-10pt]\begin{Verbatim}[fontsize=\small]
element front
{
   tei_att.global.attributes,
   tei_att.declaring.attributes,
   (
      (
         tei_model.frontPart       | tei_model.pLike       | tei_model.pLike.front       | tei_model.global      )*,
      (
         (
            (
               tei_model.div1Like,
               ( tei_model.div1Like | tei_model.frontPart | tei_model.global )*
            )
          | (
               tei_model.divLike,
               ( tei_model.divLike | tei_model.frontPart | tei_model.global )*
            )
         ),
         ( tei_model.divBottom, ( tei_model.divBottom | tei_model.global )* )?
      )?
   )
}
\end{Verbatim}

\end{reflist}  \index{fs=<fs>|oddindex}\index{type=@type!<fs>|oddindex}\index{feats=@feats!<fs>|oddindex}
\begin{reflist}
\item[]\begin{specHead}{TEI.fs}{<fs> }(structure de traits) représente une \textit{structure de traits}, c'est-à-dire un ensemble de paires trait-valeur organisé comme une unité structurelle. [\xref{http://www.tei-c.org/release/doc/tei-p5-doc/en/html/FS.html\#FSBI}{18.2. Elementary Feature Structures and the Binary Feature Value}]\end{specHead} 
    \item[{Module}]
  iso-fs
    \item[{Attributs}]
  Attributs \hyperref[TEI.att.global]{att.global} (\textit{@xml:id}, \textit{@n}, \textit{@xml:lang}, \textit{@xml:base}, \textit{@xml:space})  (\hyperref[TEI.att.global.rendition]{att.global.rendition} (\textit{@rend}, \textit{@style}, \textit{@rendition})) (\hyperref[TEI.att.global.linking]{att.global.linking} (\textit{@corresp}, \textit{@synch}, \textit{@sameAs}, \textit{@copyOf}, \textit{@next}, \textit{@prev}, \textit{@exclude}, \textit{@select})) (\hyperref[TEI.att.global.analytic]{att.global.analytic} (\textit{@ana})) (\hyperref[TEI.att.global.facs]{att.global.facs} (\textit{@facs})) (\hyperref[TEI.att.global.change]{att.global.change} (\textit{@change})) (\hyperref[TEI.att.global.responsibility]{att.global.responsibility} (\textit{@cert}, \textit{@resp})) (\hyperref[TEI.att.global.source]{att.global.source} (\textit{@source})) \hyperref[TEI.att.datcat]{att.datcat} (\textit{@datcat}, \textit{@valueDatcat}) \hfil\\[-10pt]\begin{sansreflist}
    \item[@type]
  spécifie le type de la structure de traits.
\begin{reflist}
    \item[{Statut}]
  Optionel
    \item[{Type de données}]
  \hyperref[TEI.teidata.enumerated]{teidata.enumerated}
\end{reflist}  
    \item[@feats]
  (traits) référence les spécifications trait-valeur qui caractérisent cette structure de traits.
\begin{reflist}
    \item[{Statut}]
  Optionel
    \item[{Type de données}]
  1–∞ occurrences de \hyperref[TEI.teidata.pointer]{teidata.pointer} séparé par un espace
    \item[{Note}]
  \par
Peut être utilisé soit à la place de traits pris comme contenu, soit en plus. Dans ce dernier cas, les traits référencés et contenus sont unifiés. 
\end{reflist}  
\end{sansreflist}  
    \item[{Membre du}]
  \hyperref[TEI.model.featureVal.complex]{model.featureVal.complex} \hyperref[TEI.model.global.meta]{model.global.meta} 
    \item[{Contenu dans}]
  
    \item[analysis: ]
   \hyperref[TEI.cl]{cl} \hyperref[TEI.m]{m} \hyperref[TEI.phr]{phr} \hyperref[TEI.s]{s} \hyperref[TEI.span]{span} \hyperref[TEI.w]{w}\par 
    \item[core: ]
   \hyperref[TEI.abbr]{abbr} \hyperref[TEI.add]{add} \hyperref[TEI.addrLine]{addrLine} \hyperref[TEI.address]{address} \hyperref[TEI.author]{author} \hyperref[TEI.bibl]{bibl} \hyperref[TEI.biblScope]{biblScope} \hyperref[TEI.cit]{cit} \hyperref[TEI.citedRange]{citedRange} \hyperref[TEI.corr]{corr} \hyperref[TEI.date]{date} \hyperref[TEI.del]{del} \hyperref[TEI.distinct]{distinct} \hyperref[TEI.editor]{editor} \hyperref[TEI.email]{email} \hyperref[TEI.emph]{emph} \hyperref[TEI.expan]{expan} \hyperref[TEI.foreign]{foreign} \hyperref[TEI.gloss]{gloss} \hyperref[TEI.head]{head} \hyperref[TEI.headItem]{headItem} \hyperref[TEI.headLabel]{headLabel} \hyperref[TEI.hi]{hi} \hyperref[TEI.imprint]{imprint} \hyperref[TEI.item]{item} \hyperref[TEI.l]{l} \hyperref[TEI.label]{label} \hyperref[TEI.lg]{lg} \hyperref[TEI.list]{list} \hyperref[TEI.measure]{measure} \hyperref[TEI.mentioned]{mentioned} \hyperref[TEI.name]{name} \hyperref[TEI.note]{note} \hyperref[TEI.num]{num} \hyperref[TEI.orig]{orig} \hyperref[TEI.p]{p} \hyperref[TEI.pubPlace]{pubPlace} \hyperref[TEI.publisher]{publisher} \hyperref[TEI.q]{q} \hyperref[TEI.quote]{quote} \hyperref[TEI.ref]{ref} \hyperref[TEI.reg]{reg} \hyperref[TEI.resp]{resp} \hyperref[TEI.rs]{rs} \hyperref[TEI.said]{said} \hyperref[TEI.series]{series} \hyperref[TEI.sic]{sic} \hyperref[TEI.soCalled]{soCalled} \hyperref[TEI.sp]{sp} \hyperref[TEI.speaker]{speaker} \hyperref[TEI.stage]{stage} \hyperref[TEI.street]{street} \hyperref[TEI.term]{term} \hyperref[TEI.textLang]{textLang} \hyperref[TEI.time]{time} \hyperref[TEI.title]{title} \hyperref[TEI.unclear]{unclear}\par 
    \item[figures: ]
   \hyperref[TEI.cell]{cell} \hyperref[TEI.figure]{figure} \hyperref[TEI.table]{table}\par 
    \item[header: ]
   \hyperref[TEI.authority]{authority} \hyperref[TEI.change]{change} \hyperref[TEI.classCode]{classCode} \hyperref[TEI.distributor]{distributor} \hyperref[TEI.edition]{edition} \hyperref[TEI.extent]{extent} \hyperref[TEI.funder]{funder} \hyperref[TEI.language]{language} \hyperref[TEI.licence]{licence}\par 
    \item[iso-fs: ]
   \hyperref[TEI.bicond]{bicond} \hyperref[TEI.cond]{cond} \hyperref[TEI.f]{f} \hyperref[TEI.fvLib]{fvLib} \hyperref[TEI.if]{if} \hyperref[TEI.vAlt]{vAlt} \hyperref[TEI.vColl]{vColl} \hyperref[TEI.vDefault]{vDefault} \hyperref[TEI.vLabel]{vLabel} \hyperref[TEI.vMerge]{vMerge} \hyperref[TEI.vNot]{vNot} \hyperref[TEI.vRange]{vRange}\par 
    \item[linking: ]
   \hyperref[TEI.ab]{ab} \hyperref[TEI.seg]{seg}\par 
    \item[msdescription: ]
   \hyperref[TEI.accMat]{accMat} \hyperref[TEI.acquisition]{acquisition} \hyperref[TEI.additions]{additions} \hyperref[TEI.catchwords]{catchwords} \hyperref[TEI.collation]{collation} \hyperref[TEI.colophon]{colophon} \hyperref[TEI.condition]{condition} \hyperref[TEI.custEvent]{custEvent} \hyperref[TEI.decoNote]{decoNote} \hyperref[TEI.explicit]{explicit} \hyperref[TEI.filiation]{filiation} \hyperref[TEI.finalRubric]{finalRubric} \hyperref[TEI.foliation]{foliation} \hyperref[TEI.heraldry]{heraldry} \hyperref[TEI.incipit]{incipit} \hyperref[TEI.layout]{layout} \hyperref[TEI.material]{material} \hyperref[TEI.msItem]{msItem} \hyperref[TEI.musicNotation]{musicNotation} \hyperref[TEI.objectType]{objectType} \hyperref[TEI.origDate]{origDate} \hyperref[TEI.origPlace]{origPlace} \hyperref[TEI.origin]{origin} \hyperref[TEI.provenance]{provenance} \hyperref[TEI.rubric]{rubric} \hyperref[TEI.secFol]{secFol} \hyperref[TEI.signatures]{signatures} \hyperref[TEI.source]{source} \hyperref[TEI.stamp]{stamp} \hyperref[TEI.summary]{summary} \hyperref[TEI.support]{support} \hyperref[TEI.surrogates]{surrogates} \hyperref[TEI.typeNote]{typeNote} \hyperref[TEI.watermark]{watermark}\par 
    \item[namesdates: ]
   \hyperref[TEI.addName]{addName} \hyperref[TEI.affiliation]{affiliation} \hyperref[TEI.country]{country} \hyperref[TEI.forename]{forename} \hyperref[TEI.genName]{genName} \hyperref[TEI.geogName]{geogName} \hyperref[TEI.nameLink]{nameLink} \hyperref[TEI.orgName]{orgName} \hyperref[TEI.persName]{persName} \hyperref[TEI.person]{person} \hyperref[TEI.personGrp]{personGrp} \hyperref[TEI.persona]{persona} \hyperref[TEI.placeName]{placeName} \hyperref[TEI.region]{region} \hyperref[TEI.roleName]{roleName} \hyperref[TEI.settlement]{settlement} \hyperref[TEI.surname]{surname}\par 
    \item[spoken: ]
   \hyperref[TEI.annotationBlock]{annotationBlock}\par 
    \item[standOff: ]
   \hyperref[TEI.listAnnotation]{listAnnotation}\par 
    \item[textstructure: ]
   \hyperref[TEI.back]{back} \hyperref[TEI.body]{body} \hyperref[TEI.div]{div} \hyperref[TEI.docAuthor]{docAuthor} \hyperref[TEI.docDate]{docDate} \hyperref[TEI.docEdition]{docEdition} \hyperref[TEI.docTitle]{docTitle} \hyperref[TEI.floatingText]{floatingText} \hyperref[TEI.front]{front} \hyperref[TEI.group]{group} \hyperref[TEI.text]{text} \hyperref[TEI.titlePage]{titlePage} \hyperref[TEI.titlePart]{titlePart}\par 
    \item[transcr: ]
   \hyperref[TEI.damage]{damage} \hyperref[TEI.fw]{fw} \hyperref[TEI.line]{line} \hyperref[TEI.metamark]{metamark} \hyperref[TEI.mod]{mod} \hyperref[TEI.restore]{restore} \hyperref[TEI.retrace]{retrace} \hyperref[TEI.secl]{secl} \hyperref[TEI.sourceDoc]{sourceDoc} \hyperref[TEI.supplied]{supplied} \hyperref[TEI.surface]{surface} \hyperref[TEI.surfaceGrp]{surfaceGrp} \hyperref[TEI.surplus]{surplus} \hyperref[TEI.zone]{zone}
    \item[{Peut contenir}]
  
    \item[iso-fs: ]
   \hyperref[TEI.f]{f}
    \item[{Exemple}]
  l'élément \hyperref[TEI.fs]{<fs>}contient\leavevmode\bgroup\exampleFont \begin{shaded}\noindent\mbox{}{<\textbf{fs}\hspace*{6pt}{type}="{statistics}">}\mbox{}\newline 
\hspace*{6pt}{<\textbf{f}\hspace*{6pt}{name}="{frequency}">}\mbox{}\newline 
\hspace*{6pt}\hspace*{6pt}{<\textbf{numeric}\hspace*{6pt}{value}="{7}"/>}\mbox{}\newline 
\hspace*{6pt}{</\textbf{f}>}\mbox{}\newline 
{</\textbf{fs}>}\end{shaded}\egroup 


    \item[{Modèle de contenu}]
  \mbox{}\hfill\\[-10pt]\begin{Verbatim}[fontsize=\small]
<content>
 <elementRef key="f" maxOccurs="unbounded"
  minOccurs="0"/>
</content>
    
\end{Verbatim}

    \item[{Schéma Declaration}]
  \mbox{}\hfill\\[-10pt]\begin{Verbatim}[fontsize=\small]
element fs
{
   tei_att.global.attributes,
   tei_att.datcat.attributes,
   attribute type { text }?,
   attribute feats { list { + } }?,
   tei_f*
}
\end{Verbatim}

\end{reflist}  \index{fsConstraints=<fsConstraints>|oddindex}
\begin{reflist}
\item[]\begin{specHead}{TEI.fsConstraints}{<fsConstraints> }(contraintes de structure de traits) définit les contraintes sur le contenu de structures de traits bien formées [\xref{http://www.tei-c.org/release/doc/tei-p5-doc/en/html/FS.html\#FD}{18.11. Feature System Declaration}]\end{specHead} 
    \item[{Module}]
  iso-fs
    \item[{Attributs}]
  Attributs \hyperref[TEI.att.global]{att.global} (\textit{@xml:id}, \textit{@n}, \textit{@xml:lang}, \textit{@xml:base}, \textit{@xml:space})  (\hyperref[TEI.att.global.rendition]{att.global.rendition} (\textit{@rend}, \textit{@style}, \textit{@rendition})) (\hyperref[TEI.att.global.linking]{att.global.linking} (\textit{@corresp}, \textit{@synch}, \textit{@sameAs}, \textit{@copyOf}, \textit{@next}, \textit{@prev}, \textit{@exclude}, \textit{@select})) (\hyperref[TEI.att.global.analytic]{att.global.analytic} (\textit{@ana})) (\hyperref[TEI.att.global.facs]{att.global.facs} (\textit{@facs})) (\hyperref[TEI.att.global.change]{att.global.change} (\textit{@change})) (\hyperref[TEI.att.global.responsibility]{att.global.responsibility} (\textit{@cert}, \textit{@resp})) (\hyperref[TEI.att.global.source]{att.global.source} (\textit{@source}))
    \item[{Contenu dans}]
  
    \item[iso-fs: ]
   \hyperref[TEI.fsDecl]{fsDecl}
    \item[{Peut contenir}]
  
    \item[iso-fs: ]
   \hyperref[TEI.bicond]{bicond} \hyperref[TEI.cond]{cond}
    \item[{Note}]
  \par
Peut contenir une série d'éléments conditionnels ou biconditionnels.
    \item[{Exemple}]
  \leavevmode\bgroup\exampleFont \begin{shaded}\noindent\mbox{}{<\textbf{fsConstraints}>}\mbox{}\newline 
\hspace*{6pt}{<\textbf{cond}>}\mbox{}\newline 
\hspace*{6pt}\hspace*{6pt}{<\textbf{fs}>}\mbox{}\newline 
\textit{<!-- ...-->}\mbox{}\newline 
\hspace*{6pt}\hspace*{6pt}{</\textbf{fs}>}\mbox{}\newline 
\hspace*{6pt}\hspace*{6pt}{<\textbf{then}/>}\mbox{}\newline 
\hspace*{6pt}\hspace*{6pt}{<\textbf{fs}>}\mbox{}\newline 
\textit{<!-- ... -->}\mbox{}\newline 
\hspace*{6pt}\hspace*{6pt}{</\textbf{fs}>}\mbox{}\newline 
\hspace*{6pt}{</\textbf{cond}>}\mbox{}\newline 
{</\textbf{fsConstraints}>}\end{shaded}\egroup 


    \item[{Modèle de contenu}]
  \mbox{}\hfill\\[-10pt]\begin{Verbatim}[fontsize=\small]
<content>
 <alternate maxOccurs="unbounded"
  minOccurs="0">
  <elementRef key="cond"/>
  <elementRef key="bicond"/>
 </alternate>
</content>
    
\end{Verbatim}

    \item[{Schéma Declaration}]
  \mbox{}\hfill\\[-10pt]\begin{Verbatim}[fontsize=\small]
element fsConstraints { tei_att.global.attributes, ( tei_cond | tei_bicond )* }
\end{Verbatim}

\end{reflist}  \index{fsDecl=<fsDecl>|oddindex}\index{type=@type!<fsDecl>|oddindex}\index{baseTypes=@baseTypes!<fsDecl>|oddindex}
\begin{reflist}
\item[]\begin{specHead}{TEI.fsDecl}{<fsDecl> }(déclaration de structure de traits) déclare un type de structure de traits [\xref{http://www.tei-c.org/release/doc/tei-p5-doc/en/html/FS.html\#FD}{18.11. Feature System Declaration}]\end{specHead} 
    \item[{Module}]
  iso-fs
    \item[{Attributs}]
  Attributs \hyperref[TEI.att.global]{att.global} (\textit{@xml:id}, \textit{@n}, \textit{@xml:lang}, \textit{@xml:base}, \textit{@xml:space})  (\hyperref[TEI.att.global.rendition]{att.global.rendition} (\textit{@rend}, \textit{@style}, \textit{@rendition})) (\hyperref[TEI.att.global.linking]{att.global.linking} (\textit{@corresp}, \textit{@synch}, \textit{@sameAs}, \textit{@copyOf}, \textit{@next}, \textit{@prev}, \textit{@exclude}, \textit{@select})) (\hyperref[TEI.att.global.analytic]{att.global.analytic} (\textit{@ana})) (\hyperref[TEI.att.global.facs]{att.global.facs} (\textit{@facs})) (\hyperref[TEI.att.global.change]{att.global.change} (\textit{@change})) (\hyperref[TEI.att.global.responsibility]{att.global.responsibility} (\textit{@cert}, \textit{@resp})) (\hyperref[TEI.att.global.source]{att.global.source} (\textit{@source})) \hfil\\[-10pt]\begin{sansreflist}
    \item[@type]
  attribue un nom au type de structure de traits déclaré.
\begin{reflist}
    \item[{Statut}]
  Requis
    \item[{Type de données}]
  \hyperref[TEI.teidata.enumerated]{teidata.enumerated}
\end{reflist}  
    \item[@baseTypes]
  donne le nom d'une ou plusieurs structures de traits "type", de laquelle ou desquelles il hérite des spécifications et des contraintes de traits. Si ce type inclut une spécification de traits du même nom que l'une de celles spécifiées par cet attribut, ou si plus d'une spécification du même nom est transmise par héritage, l'ensemble des valeurs possibles est définie par unification. De même, l'ensemble des contraintes applicables résulte de la combinaison de celles qui sont spécifiées explicitement à l'intérieur de cet élément et de celles qui découlent de l'attribut {\itshape baseTypes}. Quand aucun attribut {\itshape baseTypes} n'est précisé, aucune spécification de traits ni contrainte n'est transmise par héritage.
\begin{reflist}
    \item[{Statut}]
  Optionel
    \item[{Type de données}]
  1–∞ occurrences de \hyperref[TEI.teidata.name]{teidata.name} séparé par un espace
    \item[{Note}]
  \par
L'héritage est défini ici comme une relation monotone.\par
La combinaison de contraintes peut générer une contradiction, par exemple si deux spécifications données pour le même trait présentent des plages disjointes de valeurs et qu'au moins une de ces spécifications est obligatoire. Dans ce cas, il n'y a pas de représentant valide du type défini.
\end{reflist}  
\end{sansreflist}  
    \item[{Contenu dans}]
  
    \item[iso-fs: ]
   \hyperref[TEI.fsdDecl]{fsdDecl}
    \item[{Peut contenir}]
  
    \item[iso-fs: ]
   \hyperref[TEI.fDecl]{fDecl} \hyperref[TEI.fsConstraints]{fsConstraints} \hyperref[TEI.fsDescr]{fsDescr}
    \item[{Exemple}]
  \leavevmode\bgroup\exampleFont \begin{shaded}\noindent\mbox{}{<\textbf{fsDecl}\hspace*{6pt}{type}="{SomeName}">}\mbox{}\newline 
\hspace*{6pt}{<\textbf{fsDescr}>}Describes what this type of fs represents{</\textbf{fsDescr}>}\mbox{}\newline 
\hspace*{6pt}{<\textbf{fDecl}\hspace*{6pt}{name}="{featureOne}">}\mbox{}\newline 
\textit{<!-- The declaration for featureOne -->}\mbox{}\newline 
\hspace*{6pt}\hspace*{6pt}{<\textbf{vRange}>}\mbox{}\newline 
\textit{<!-- the range of possible values for this feature -->}\mbox{}\newline 
\hspace*{6pt}\hspace*{6pt}{</\textbf{vRange}>}\mbox{}\newline 
\hspace*{6pt}{</\textbf{fDecl}>}\mbox{}\newline 
\hspace*{6pt}{<\textbf{fDecl}\hspace*{6pt}{name}="{featureTwo}">}\mbox{}\newline 
\textit{<!-- The declaration for featureTwo -->}\mbox{}\newline 
\hspace*{6pt}\hspace*{6pt}{<\textbf{vRange}>}\mbox{}\newline 
\textit{<!-- the range of possible values for this feature -->}\mbox{}\newline 
\hspace*{6pt}\hspace*{6pt}{</\textbf{vRange}>}\mbox{}\newline 
\hspace*{6pt}{</\textbf{fDecl}>}\mbox{}\newline 
\hspace*{6pt}{<\textbf{fsConstraints}>}\mbox{}\newline 
\textit{<!-- Any additional constraints for the feature structure -->}\mbox{}\newline 
\hspace*{6pt}{</\textbf{fsConstraints}>}\mbox{}\newline 
{</\textbf{fsDecl}>}\end{shaded}\egroup 


    \item[{Modèle de contenu}]
  \mbox{}\hfill\\[-10pt]\begin{Verbatim}[fontsize=\small]
<content>
 <sequence maxOccurs="1" minOccurs="1">
  <elementRef key="fsDescr" minOccurs="0"/>
  <elementRef key="fDecl"
   maxOccurs="unbounded" minOccurs="1"/>
  <elementRef key="fsConstraints"
   minOccurs="0"/>
 </sequence>
</content>
    
\end{Verbatim}

    \item[{Schéma Declaration}]
  \mbox{}\hfill\\[-10pt]\begin{Verbatim}[fontsize=\small]
element fsDecl
{
   tei_att.global.attributes,
   attribute type { text },
   attribute baseTypes { list { + } }?,
   ( tei_fsDescr?, tei_fDecl+, tei_fsConstraints? )
}
\end{Verbatim}

\end{reflist}  \index{fsDescr=<fsDescr>|oddindex}
\begin{reflist}
\item[]\begin{specHead}{TEI.fsDescr}{<fsDescr> }(description de système de traits (dans FSD)) décrit en texte libre ce que représente le type de structure de traits déclaré dans le fsDecl englobant [\xref{http://www.tei-c.org/release/doc/tei-p5-doc/en/html/FS.html\#FD}{18.11. Feature System Declaration}]\end{specHead} 
    \item[{Module}]
  iso-fs
    \item[{Attributs}]
  Attributs \hyperref[TEI.att.global]{att.global} (\textit{@xml:id}, \textit{@n}, \textit{@xml:lang}, \textit{@xml:base}, \textit{@xml:space})  (\hyperref[TEI.att.global.rendition]{att.global.rendition} (\textit{@rend}, \textit{@style}, \textit{@rendition})) (\hyperref[TEI.att.global.linking]{att.global.linking} (\textit{@corresp}, \textit{@synch}, \textit{@sameAs}, \textit{@copyOf}, \textit{@next}, \textit{@prev}, \textit{@exclude}, \textit{@select})) (\hyperref[TEI.att.global.analytic]{att.global.analytic} (\textit{@ana})) (\hyperref[TEI.att.global.facs]{att.global.facs} (\textit{@facs})) (\hyperref[TEI.att.global.change]{att.global.change} (\textit{@change})) (\hyperref[TEI.att.global.responsibility]{att.global.responsibility} (\textit{@cert}, \textit{@resp})) (\hyperref[TEI.att.global.source]{att.global.source} (\textit{@source}))
    \item[{Contenu dans}]
  
    \item[iso-fs: ]
   \hyperref[TEI.fsDecl]{fsDecl}
    \item[{Peut contenir}]
  
    \item[core: ]
   \hyperref[TEI.abbr]{abbr} \hyperref[TEI.address]{address} \hyperref[TEI.bibl]{bibl} \hyperref[TEI.biblStruct]{biblStruct} \hyperref[TEI.choice]{choice} \hyperref[TEI.cit]{cit} \hyperref[TEI.date]{date} \hyperref[TEI.desc]{desc} \hyperref[TEI.distinct]{distinct} \hyperref[TEI.email]{email} \hyperref[TEI.emph]{emph} \hyperref[TEI.expan]{expan} \hyperref[TEI.foreign]{foreign} \hyperref[TEI.gloss]{gloss} \hyperref[TEI.hi]{hi} \hyperref[TEI.label]{label} \hyperref[TEI.list]{list} \hyperref[TEI.listBibl]{listBibl} \hyperref[TEI.measure]{measure} \hyperref[TEI.measureGrp]{measureGrp} \hyperref[TEI.mentioned]{mentioned} \hyperref[TEI.name]{name} \hyperref[TEI.num]{num} \hyperref[TEI.ptr]{ptr} \hyperref[TEI.q]{q} \hyperref[TEI.quote]{quote} \hyperref[TEI.ref]{ref} \hyperref[TEI.rs]{rs} \hyperref[TEI.said]{said} \hyperref[TEI.soCalled]{soCalled} \hyperref[TEI.stage]{stage} \hyperref[TEI.term]{term} \hyperref[TEI.time]{time} \hyperref[TEI.title]{title}\par 
    \item[figures: ]
   \hyperref[TEI.table]{table}\par 
    \item[header: ]
   \hyperref[TEI.biblFull]{biblFull} \hyperref[TEI.idno]{idno}\par 
    \item[msdescription: ]
   \hyperref[TEI.catchwords]{catchwords} \hyperref[TEI.depth]{depth} \hyperref[TEI.dim]{dim} \hyperref[TEI.dimensions]{dimensions} \hyperref[TEI.height]{height} \hyperref[TEI.heraldry]{heraldry} \hyperref[TEI.locus]{locus} \hyperref[TEI.locusGrp]{locusGrp} \hyperref[TEI.material]{material} \hyperref[TEI.msDesc]{msDesc} \hyperref[TEI.objectType]{objectType} \hyperref[TEI.origDate]{origDate} \hyperref[TEI.origPlace]{origPlace} \hyperref[TEI.secFol]{secFol} \hyperref[TEI.signatures]{signatures} \hyperref[TEI.stamp]{stamp} \hyperref[TEI.watermark]{watermark} \hyperref[TEI.width]{width}\par 
    \item[namesdates: ]
   \hyperref[TEI.addName]{addName} \hyperref[TEI.affiliation]{affiliation} \hyperref[TEI.country]{country} \hyperref[TEI.forename]{forename} \hyperref[TEI.genName]{genName} \hyperref[TEI.geogName]{geogName} \hyperref[TEI.listOrg]{listOrg} \hyperref[TEI.listPlace]{listPlace} \hyperref[TEI.location]{location} \hyperref[TEI.nameLink]{nameLink} \hyperref[TEI.orgName]{orgName} \hyperref[TEI.persName]{persName} \hyperref[TEI.placeName]{placeName} \hyperref[TEI.region]{region} \hyperref[TEI.roleName]{roleName} \hyperref[TEI.settlement]{settlement} \hyperref[TEI.state]{state} \hyperref[TEI.surname]{surname}\par 
    \item[textstructure: ]
   \hyperref[TEI.floatingText]{floatingText}\par 
    \item[transcr: ]
   \hyperref[TEI.am]{am} \hyperref[TEI.ex]{ex} \hyperref[TEI.subst]{subst}\par des données textuelles
    \item[{Note}]
  \par
Peut contenir des caractères, des éléments de niveau expression ou de niveau intermédiaire.
    \item[{Exemple}]
  \leavevmode\bgroup\exampleFont \begin{shaded}\noindent\mbox{}{<\textbf{fsDecl}\hspace*{6pt}{type}="{Agreement}">}\mbox{}\newline 
\hspace*{6pt}{<\textbf{fsDescr}>}This type of feature structure encodes the features\mbox{}\newline 
\hspace*{6pt}\hspace*{6pt} for subject-verb agreement in English{</\textbf{fsDescr}>}\mbox{}\newline 
\hspace*{6pt}{<\textbf{fDecl}\hspace*{6pt}{name}="{PERS}">}\mbox{}\newline 
\hspace*{6pt}\hspace*{6pt}{<\textbf{fDescr}>}person (first, second, or third){</\textbf{fDescr}>}\mbox{}\newline 
\hspace*{6pt}\hspace*{6pt}{<\textbf{vRange}>}\mbox{}\newline 
\hspace*{6pt}\hspace*{6pt}\hspace*{6pt}{<\textbf{vAlt}>}\mbox{}\newline 
\hspace*{6pt}\hspace*{6pt}\hspace*{6pt}\hspace*{6pt}{<\textbf{symbol}\hspace*{6pt}{value}="{first}"/>}\mbox{}\newline 
\hspace*{6pt}\hspace*{6pt}\hspace*{6pt}\hspace*{6pt}{<\textbf{symbol}\hspace*{6pt}{value}="{second}"/>}\mbox{}\newline 
\hspace*{6pt}\hspace*{6pt}\hspace*{6pt}\hspace*{6pt}{<\textbf{symbol}\hspace*{6pt}{value}="{third}"/>}\mbox{}\newline 
\hspace*{6pt}\hspace*{6pt}\hspace*{6pt}{</\textbf{vAlt}>}\mbox{}\newline 
\hspace*{6pt}\hspace*{6pt}{</\textbf{vRange}>}\mbox{}\newline 
\hspace*{6pt}{</\textbf{fDecl}>}\mbox{}\newline 
\hspace*{6pt}{<\textbf{fDecl}\hspace*{6pt}{name}="{NUM}">}\mbox{}\newline 
\hspace*{6pt}\hspace*{6pt}{<\textbf{fDescr}>}number (singular or plural){</\textbf{fDescr}>}\mbox{}\newline 
\hspace*{6pt}\hspace*{6pt}{<\textbf{vRange}>}\mbox{}\newline 
\hspace*{6pt}\hspace*{6pt}\hspace*{6pt}{<\textbf{vAlt}>}\mbox{}\newline 
\hspace*{6pt}\hspace*{6pt}\hspace*{6pt}\hspace*{6pt}{<\textbf{symbol}\hspace*{6pt}{value}="{singular}"/>}\mbox{}\newline 
\hspace*{6pt}\hspace*{6pt}\hspace*{6pt}\hspace*{6pt}{<\textbf{symbol}\hspace*{6pt}{value}="{plural}"/>}\mbox{}\newline 
\hspace*{6pt}\hspace*{6pt}\hspace*{6pt}{</\textbf{vAlt}>}\mbox{}\newline 
\hspace*{6pt}\hspace*{6pt}{</\textbf{vRange}>}\mbox{}\newline 
\hspace*{6pt}{</\textbf{fDecl}>}\mbox{}\newline 
{</\textbf{fsDecl}>}\end{shaded}\egroup 


    \item[{Modèle de contenu}]
  \mbox{}\hfill\\[-10pt]\begin{Verbatim}[fontsize=\small]
<content>
 <macroRef key="macro.limitedContent"/>
</content>
    
\end{Verbatim}

    \item[{Schéma Declaration}]
  \mbox{}\hfill\\[-10pt]\begin{Verbatim}[fontsize=\small]
element fsDescr { tei_att.global.attributes, tei_macro.limitedContent }
\end{Verbatim}

\end{reflist}  \index{fsdDecl=<fsdDecl>|oddindex}
\begin{reflist}
\item[]\begin{specHead}{TEI.fsdDecl}{<fsdDecl> }(Déclaration de système de traits (FSD)) fournit une déclaration du système de traits consistant en une ou plusieurs déclarations de structure de traits ou des liens vers une déclaration de structure de traits. [\xref{http://www.tei-c.org/release/doc/tei-p5-doc/en/html/FS.html\#FD}{18.11. Feature System Declaration}]\end{specHead} 
    \item[{Module}]
  iso-fs
    \item[{Attributs}]
  Attributs \hyperref[TEI.att.global]{att.global} (\textit{@xml:id}, \textit{@n}, \textit{@xml:lang}, \textit{@xml:base}, \textit{@xml:space})  (\hyperref[TEI.att.global.rendition]{att.global.rendition} (\textit{@rend}, \textit{@style}, \textit{@rendition})) (\hyperref[TEI.att.global.linking]{att.global.linking} (\textit{@corresp}, \textit{@synch}, \textit{@sameAs}, \textit{@copyOf}, \textit{@next}, \textit{@prev}, \textit{@exclude}, \textit{@select})) (\hyperref[TEI.att.global.analytic]{att.global.analytic} (\textit{@ana})) (\hyperref[TEI.att.global.facs]{att.global.facs} (\textit{@facs})) (\hyperref[TEI.att.global.change]{att.global.change} (\textit{@change})) (\hyperref[TEI.att.global.responsibility]{att.global.responsibility} (\textit{@cert}, \textit{@resp})) (\hyperref[TEI.att.global.source]{att.global.source} (\textit{@source}))
    \item[{Membre du}]
  \hyperref[TEI.model.encodingDescPart]{model.encodingDescPart} \hyperref[TEI.model.resourceLike]{model.resourceLike}
    \item[{Contenu dans}]
  
    \item[core: ]
   \hyperref[TEI.teiCorpus]{teiCorpus}\par 
    \item[header: ]
   \hyperref[TEI.encodingDesc]{encodingDesc}\par 
    \item[standOff: ]
   \hyperref[TEI.standOff]{standOff}\par 
    \item[textstructure: ]
   \hyperref[TEI.TEI]{TEI}
    \item[{Peut contenir}]
  
    \item[iso-fs: ]
   \hyperref[TEI.fsDecl]{fsDecl} \hyperref[TEI.fsdLink]{fsdLink}
    \item[{Exemple}]
  \leavevmode\bgroup\exampleFont \begin{shaded}\noindent\mbox{}{<\textbf{fsdDecl}>}\mbox{}\newline 
\hspace*{6pt}{<\textbf{fsDecl}\hspace*{6pt}{type}="{GPSG}">}\mbox{}\newline 
\hspace*{6pt}\hspace*{6pt}{<\textbf{fDecl}\hspace*{6pt}{name}="{GPSG\textunderscore feat1}"\mbox{}\newline 
\hspace*{6pt}\hspace*{6pt}\hspace*{6pt}{xml:id}="{GPSG-1-fr}">}\mbox{}\newline 
\hspace*{6pt}\hspace*{6pt}\hspace*{6pt}{<\textbf{vRange}>}\mbox{}\newline 
\hspace*{6pt}\hspace*{6pt}\hspace*{6pt}\hspace*{6pt}{<\textbf{vAlt}>}\mbox{}\newline 
\hspace*{6pt}\hspace*{6pt}\hspace*{6pt}\hspace*{6pt}\hspace*{6pt}{<\textbf{symbol}\hspace*{6pt}{value}="{red}"/>}\mbox{}\newline 
\hspace*{6pt}\hspace*{6pt}\hspace*{6pt}\hspace*{6pt}\hspace*{6pt}{<\textbf{symbol}\hspace*{6pt}{value}="{blue}"/>}\mbox{}\newline 
\hspace*{6pt}\hspace*{6pt}\hspace*{6pt}\hspace*{6pt}\hspace*{6pt}{<\textbf{symbol}\hspace*{6pt}{value}="{green}"/>}\mbox{}\newline 
\hspace*{6pt}\hspace*{6pt}\hspace*{6pt}\hspace*{6pt}{</\textbf{vAlt}>}\mbox{}\newline 
\hspace*{6pt}\hspace*{6pt}\hspace*{6pt}{</\textbf{vRange}>}\mbox{}\newline 
\hspace*{6pt}\hspace*{6pt}{</\textbf{fDecl}>}\mbox{}\newline 
\textit{<!--other feature declarations for GPSG here ... -->}\mbox{}\newline 
\hspace*{6pt}{</\textbf{fsDecl}>}\mbox{}\newline 
\hspace*{6pt}{<\textbf{fsdLink}\hspace*{6pt}{target}="{http://www.example.com/fsdLib.xml\#LX123}"\mbox{}\newline 
\hspace*{6pt}\hspace*{6pt}{type}="{subentry}"/>}\mbox{}\newline 
{</\textbf{fsdDecl}>}\end{shaded}\egroup 


    \item[{Modèle de contenu}]
  \mbox{}\hfill\\[-10pt]\begin{Verbatim}[fontsize=\small]
<content>
 <alternate maxOccurs="unbounded"
  minOccurs="1">
  <elementRef key="fsDecl"/>
  <elementRef key="fsdLink"/>
 </alternate>
</content>
    
\end{Verbatim}

    \item[{Schéma Declaration}]
  \mbox{}\hfill\\[-10pt]\begin{Verbatim}[fontsize=\small]
element fsdDecl { tei_att.global.attributes, ( tei_fsDecl | tei_fsdLink )+ }
\end{Verbatim}

\end{reflist}  \index{fsdLink=<fsdLink>|oddindex}\index{type=@type!<fsdLink>|oddindex}\index{target=@target!<fsdLink>|oddindex}
\begin{reflist}
\item[]\begin{specHead}{TEI.fsdLink}{<fsdLink> }(lien vers la déclaration d'une structure de traits) associe le nom d'une structure de traits "type" à sa déclaration de structure de traits. [\xref{http://www.tei-c.org/release/doc/tei-p5-doc/en/html/FS.html\#FD}{18.11. Feature System Declaration}]\end{specHead} 
    \item[{Module}]
  iso-fs
    \item[{Attributs}]
  Attributs \hyperref[TEI.att.global]{att.global} (\textit{@xml:id}, \textit{@n}, \textit{@xml:lang}, \textit{@xml:base}, \textit{@xml:space})  (\hyperref[TEI.att.global.rendition]{att.global.rendition} (\textit{@rend}, \textit{@style}, \textit{@rendition})) (\hyperref[TEI.att.global.linking]{att.global.linking} (\textit{@corresp}, \textit{@synch}, \textit{@sameAs}, \textit{@copyOf}, \textit{@next}, \textit{@prev}, \textit{@exclude}, \textit{@select})) (\hyperref[TEI.att.global.analytic]{att.global.analytic} (\textit{@ana})) (\hyperref[TEI.att.global.facs]{att.global.facs} (\textit{@facs})) (\hyperref[TEI.att.global.change]{att.global.change} (\textit{@change})) (\hyperref[TEI.att.global.responsibility]{att.global.responsibility} (\textit{@cert}, \textit{@resp})) (\hyperref[TEI.att.global.source]{att.global.source} (\textit{@source})) \hfil\\[-10pt]\begin{sansreflist}
    \item[@type]
  identifie le type de structure de traits à documenter ; ce sera la valeur de l’attribut {\itshape type} d’au moins une structure de traits.
\begin{reflist}
    \item[{Statut}]
  Requis
    \item[{Type de données}]
  \hyperref[TEI.teidata.enumerated]{teidata.enumerated}
\end{reflist}  
    \item[@target]
  fournit un pointeur vers un élément de déclaration de structure de traits (\hyperref[TEI.fsDecl]{<fsDecl>}) dans le document courant ou ailleurs.
\begin{reflist}
    \item[{Statut}]
  Requis
    \item[{Type de données}]
  \hyperref[TEI.teidata.pointer]{teidata.pointer}
\end{reflist}  
\end{sansreflist}  
    \item[{Contenu dans}]
  
    \item[iso-fs: ]
   \hyperref[TEI.fsdDecl]{fsdDecl}
    \item[{Peut contenir}]
  Elément vide
    \item[{Exemple}]
  \leavevmode\bgroup\exampleFont \begin{shaded}\noindent\mbox{}{<\textbf{fsdLink}\hspace*{6pt}{target}="{http://www.example.com/fsdLib.xml\#L1234}"\mbox{}\newline 
\hspace*{6pt}{type}="{subentry}"/>}\end{shaded}\egroup 


    \item[{Exemple}]
  \leavevmode\bgroup\exampleFont \begin{shaded}\noindent\mbox{}{<\textbf{fsdLink}\hspace*{6pt}{target}="{http://www.example.com/fsdLib.xml\#L1234}"\mbox{}\newline 
\hspace*{6pt}{type}="{subentry}"/>}\end{shaded}\egroup 


    \item[{Modèle de contenu}]
  \fbox{\ttfamily <content>\newline
</content>\newline
    } 
    \item[{Schéma Declaration}]
  \mbox{}\hfill\\[-10pt]\begin{Verbatim}[fontsize=\small]
element fsdLink
{
   tei_att.global.attributes,
   attribute type { text },
   attribute target { text },
   empty
}
\end{Verbatim}

\end{reflist}  \index{funder=<funder>|oddindex}
\begin{reflist}
\item[]\begin{specHead}{TEI.funder}{<funder> }(financeur) désigne le nom d’une personne ou d’un organisme responsable du financement d’un projet ou d’un texte. [\xref{http://www.tei-c.org/release/doc/tei-p5-doc/en/html/HD.html\#HD21}{2.2.1. The Title Statement}]\end{specHead} 
    \item[{Module}]
  header
    \item[{Attributs}]
  Attributs \hyperref[TEI.att.global]{att.global} (\textit{@xml:id}, \textit{@n}, \textit{@xml:lang}, \textit{@xml:base}, \textit{@xml:space})  (\hyperref[TEI.att.global.rendition]{att.global.rendition} (\textit{@rend}, \textit{@style}, \textit{@rendition})) (\hyperref[TEI.att.global.linking]{att.global.linking} (\textit{@corresp}, \textit{@synch}, \textit{@sameAs}, \textit{@copyOf}, \textit{@next}, \textit{@prev}, \textit{@exclude}, \textit{@select})) (\hyperref[TEI.att.global.analytic]{att.global.analytic} (\textit{@ana})) (\hyperref[TEI.att.global.facs]{att.global.facs} (\textit{@facs})) (\hyperref[TEI.att.global.change]{att.global.change} (\textit{@change})) (\hyperref[TEI.att.global.responsibility]{att.global.responsibility} (\textit{@cert}, \textit{@resp})) (\hyperref[TEI.att.global.source]{att.global.source} (\textit{@source})) \hyperref[TEI.att.canonical]{att.canonical} (\textit{@key}, \textit{@ref}) 
    \item[{Membre du}]
  \hyperref[TEI.model.respLike]{model.respLike} 
    \item[{Contenu dans}]
  
    \item[core: ]
   \hyperref[TEI.bibl]{bibl} \hyperref[TEI.monogr]{monogr}\par 
    \item[header: ]
   \hyperref[TEI.editionStmt]{editionStmt} \hyperref[TEI.titleStmt]{titleStmt}\par 
    \item[msdescription: ]
   \hyperref[TEI.msItem]{msItem}
    \item[{Peut contenir}]
  
    \item[analysis: ]
   \hyperref[TEI.interp]{interp} \hyperref[TEI.interpGrp]{interpGrp} \hyperref[TEI.span]{span} \hyperref[TEI.spanGrp]{spanGrp}\par 
    \item[core: ]
   \hyperref[TEI.abbr]{abbr} \hyperref[TEI.address]{address} \hyperref[TEI.cb]{cb} \hyperref[TEI.choice]{choice} \hyperref[TEI.date]{date} \hyperref[TEI.distinct]{distinct} \hyperref[TEI.email]{email} \hyperref[TEI.emph]{emph} \hyperref[TEI.expan]{expan} \hyperref[TEI.foreign]{foreign} \hyperref[TEI.gap]{gap} \hyperref[TEI.gb]{gb} \hyperref[TEI.gloss]{gloss} \hyperref[TEI.hi]{hi} \hyperref[TEI.index]{index} \hyperref[TEI.lb]{lb} \hyperref[TEI.measure]{measure} \hyperref[TEI.measureGrp]{measureGrp} \hyperref[TEI.mentioned]{mentioned} \hyperref[TEI.milestone]{milestone} \hyperref[TEI.name]{name} \hyperref[TEI.note]{note} \hyperref[TEI.num]{num} \hyperref[TEI.pb]{pb} \hyperref[TEI.ptr]{ptr} \hyperref[TEI.ref]{ref} \hyperref[TEI.rs]{rs} \hyperref[TEI.soCalled]{soCalled} \hyperref[TEI.term]{term} \hyperref[TEI.time]{time} \hyperref[TEI.title]{title}\par 
    \item[figures: ]
   \hyperref[TEI.figure]{figure} \hyperref[TEI.notatedMusic]{notatedMusic}\par 
    \item[header: ]
   \hyperref[TEI.idno]{idno}\par 
    \item[iso-fs: ]
   \hyperref[TEI.fLib]{fLib} \hyperref[TEI.fs]{fs} \hyperref[TEI.fvLib]{fvLib}\par 
    \item[linking: ]
   \hyperref[TEI.alt]{alt} \hyperref[TEI.altGrp]{altGrp} \hyperref[TEI.anchor]{anchor} \hyperref[TEI.join]{join} \hyperref[TEI.joinGrp]{joinGrp} \hyperref[TEI.link]{link} \hyperref[TEI.linkGrp]{linkGrp} \hyperref[TEI.timeline]{timeline}\par 
    \item[msdescription: ]
   \hyperref[TEI.catchwords]{catchwords} \hyperref[TEI.depth]{depth} \hyperref[TEI.dim]{dim} \hyperref[TEI.dimensions]{dimensions} \hyperref[TEI.height]{height} \hyperref[TEI.heraldry]{heraldry} \hyperref[TEI.locus]{locus} \hyperref[TEI.locusGrp]{locusGrp} \hyperref[TEI.material]{material} \hyperref[TEI.objectType]{objectType} \hyperref[TEI.origDate]{origDate} \hyperref[TEI.origPlace]{origPlace} \hyperref[TEI.secFol]{secFol} \hyperref[TEI.signatures]{signatures} \hyperref[TEI.source]{source} \hyperref[TEI.stamp]{stamp} \hyperref[TEI.watermark]{watermark} \hyperref[TEI.width]{width}\par 
    \item[namesdates: ]
   \hyperref[TEI.addName]{addName} \hyperref[TEI.affiliation]{affiliation} \hyperref[TEI.country]{country} \hyperref[TEI.forename]{forename} \hyperref[TEI.genName]{genName} \hyperref[TEI.geogName]{geogName} \hyperref[TEI.location]{location} \hyperref[TEI.nameLink]{nameLink} \hyperref[TEI.orgName]{orgName} \hyperref[TEI.persName]{persName} \hyperref[TEI.placeName]{placeName} \hyperref[TEI.region]{region} \hyperref[TEI.roleName]{roleName} \hyperref[TEI.settlement]{settlement} \hyperref[TEI.state]{state} \hyperref[TEI.surname]{surname}\par 
    \item[transcr: ]
   \hyperref[TEI.addSpan]{addSpan} \hyperref[TEI.am]{am} \hyperref[TEI.damageSpan]{damageSpan} \hyperref[TEI.delSpan]{delSpan} \hyperref[TEI.ex]{ex} \hyperref[TEI.fw]{fw} \hyperref[TEI.listTranspose]{listTranspose} \hyperref[TEI.metamark]{metamark} \hyperref[TEI.space]{space} \hyperref[TEI.subst]{subst} \hyperref[TEI.substJoin]{substJoin}\par des données textuelles
    \item[{Note}]
  \par
Les financeurs apportent un soutien financier au projet ; ils se distinguent des \textit{commanditaires}, qui apportent une caution , une autorité intellectuelle.
    \item[{Exemple}]
  \leavevmode\bgroup\exampleFont \begin{shaded}\noindent\mbox{}{<\textbf{funder}>}Ministère de l'Enseignement supérieur et de la Recherche{</\textbf{funder}>}\mbox{}\newline 
{<\textbf{funder}>}Conseil général de Meurthe-et-Moselle {</\textbf{funder}>}\end{shaded}\egroup 


    \item[{Modèle de contenu}]
  \mbox{}\hfill\\[-10pt]\begin{Verbatim}[fontsize=\small]
<content>
 <macroRef key="macro.phraseSeq.limited"/>
</content>
    
\end{Verbatim}

    \item[{Schéma Declaration}]
  \mbox{}\hfill\\[-10pt]\begin{Verbatim}[fontsize=\small]
element funder
{
   tei_att.global.attributes,
   tei_att.canonical.attributes,
   tei_macro.phraseSeq.limited}
\end{Verbatim}

\end{reflist}  \index{fvLib=<fvLib>|oddindex}
\begin{reflist}
\item[]\begin{specHead}{TEI.fvLib}{<fvLib> }(bibliothèque trait-valeur) rassemble une bibliothèque d'éléments trait-valeur réutilisables (y compris des structures de traits complètes). [\xref{http://www.tei-c.org/release/doc/tei-p5-doc/en/html/FS.html\#FSFL}{18.4. Feature Libraries and Feature-Value Libraries}]\end{specHead} 
    \item[{Module}]
  iso-fs
    \item[{Attributs}]
  Attributs \hyperref[TEI.att.global]{att.global} (\textit{@xml:id}, \textit{@n}, \textit{@xml:lang}, \textit{@xml:base}, \textit{@xml:space})  (\hyperref[TEI.att.global.rendition]{att.global.rendition} (\textit{@rend}, \textit{@style}, \textit{@rendition})) (\hyperref[TEI.att.global.linking]{att.global.linking} (\textit{@corresp}, \textit{@synch}, \textit{@sameAs}, \textit{@copyOf}, \textit{@next}, \textit{@prev}, \textit{@exclude}, \textit{@select})) (\hyperref[TEI.att.global.analytic]{att.global.analytic} (\textit{@ana})) (\hyperref[TEI.att.global.facs]{att.global.facs} (\textit{@facs})) (\hyperref[TEI.att.global.change]{att.global.change} (\textit{@change})) (\hyperref[TEI.att.global.responsibility]{att.global.responsibility} (\textit{@cert}, \textit{@resp})) (\hyperref[TEI.att.global.source]{att.global.source} (\textit{@source}))
    \item[{Membre du}]
  \hyperref[TEI.model.global.meta]{model.global.meta}
    \item[{Contenu dans}]
  
    \item[analysis: ]
   \hyperref[TEI.cl]{cl} \hyperref[TEI.m]{m} \hyperref[TEI.phr]{phr} \hyperref[TEI.s]{s} \hyperref[TEI.span]{span} \hyperref[TEI.w]{w}\par 
    \item[core: ]
   \hyperref[TEI.abbr]{abbr} \hyperref[TEI.add]{add} \hyperref[TEI.addrLine]{addrLine} \hyperref[TEI.address]{address} \hyperref[TEI.author]{author} \hyperref[TEI.bibl]{bibl} \hyperref[TEI.biblScope]{biblScope} \hyperref[TEI.cit]{cit} \hyperref[TEI.citedRange]{citedRange} \hyperref[TEI.corr]{corr} \hyperref[TEI.date]{date} \hyperref[TEI.del]{del} \hyperref[TEI.distinct]{distinct} \hyperref[TEI.editor]{editor} \hyperref[TEI.email]{email} \hyperref[TEI.emph]{emph} \hyperref[TEI.expan]{expan} \hyperref[TEI.foreign]{foreign} \hyperref[TEI.gloss]{gloss} \hyperref[TEI.head]{head} \hyperref[TEI.headItem]{headItem} \hyperref[TEI.headLabel]{headLabel} \hyperref[TEI.hi]{hi} \hyperref[TEI.imprint]{imprint} \hyperref[TEI.item]{item} \hyperref[TEI.l]{l} \hyperref[TEI.label]{label} \hyperref[TEI.lg]{lg} \hyperref[TEI.list]{list} \hyperref[TEI.measure]{measure} \hyperref[TEI.mentioned]{mentioned} \hyperref[TEI.name]{name} \hyperref[TEI.note]{note} \hyperref[TEI.num]{num} \hyperref[TEI.orig]{orig} \hyperref[TEI.p]{p} \hyperref[TEI.pubPlace]{pubPlace} \hyperref[TEI.publisher]{publisher} \hyperref[TEI.q]{q} \hyperref[TEI.quote]{quote} \hyperref[TEI.ref]{ref} \hyperref[TEI.reg]{reg} \hyperref[TEI.resp]{resp} \hyperref[TEI.rs]{rs} \hyperref[TEI.said]{said} \hyperref[TEI.series]{series} \hyperref[TEI.sic]{sic} \hyperref[TEI.soCalled]{soCalled} \hyperref[TEI.sp]{sp} \hyperref[TEI.speaker]{speaker} \hyperref[TEI.stage]{stage} \hyperref[TEI.street]{street} \hyperref[TEI.term]{term} \hyperref[TEI.textLang]{textLang} \hyperref[TEI.time]{time} \hyperref[TEI.title]{title} \hyperref[TEI.unclear]{unclear}\par 
    \item[figures: ]
   \hyperref[TEI.cell]{cell} \hyperref[TEI.figure]{figure} \hyperref[TEI.table]{table}\par 
    \item[header: ]
   \hyperref[TEI.authority]{authority} \hyperref[TEI.change]{change} \hyperref[TEI.classCode]{classCode} \hyperref[TEI.distributor]{distributor} \hyperref[TEI.edition]{edition} \hyperref[TEI.extent]{extent} \hyperref[TEI.funder]{funder} \hyperref[TEI.language]{language} \hyperref[TEI.licence]{licence}\par 
    \item[linking: ]
   \hyperref[TEI.ab]{ab} \hyperref[TEI.seg]{seg}\par 
    \item[msdescription: ]
   \hyperref[TEI.accMat]{accMat} \hyperref[TEI.acquisition]{acquisition} \hyperref[TEI.additions]{additions} \hyperref[TEI.catchwords]{catchwords} \hyperref[TEI.collation]{collation} \hyperref[TEI.colophon]{colophon} \hyperref[TEI.condition]{condition} \hyperref[TEI.custEvent]{custEvent} \hyperref[TEI.decoNote]{decoNote} \hyperref[TEI.explicit]{explicit} \hyperref[TEI.filiation]{filiation} \hyperref[TEI.finalRubric]{finalRubric} \hyperref[TEI.foliation]{foliation} \hyperref[TEI.heraldry]{heraldry} \hyperref[TEI.incipit]{incipit} \hyperref[TEI.layout]{layout} \hyperref[TEI.material]{material} \hyperref[TEI.msItem]{msItem} \hyperref[TEI.musicNotation]{musicNotation} \hyperref[TEI.objectType]{objectType} \hyperref[TEI.origDate]{origDate} \hyperref[TEI.origPlace]{origPlace} \hyperref[TEI.origin]{origin} \hyperref[TEI.provenance]{provenance} \hyperref[TEI.rubric]{rubric} \hyperref[TEI.secFol]{secFol} \hyperref[TEI.signatures]{signatures} \hyperref[TEI.source]{source} \hyperref[TEI.stamp]{stamp} \hyperref[TEI.summary]{summary} \hyperref[TEI.support]{support} \hyperref[TEI.surrogates]{surrogates} \hyperref[TEI.typeNote]{typeNote} \hyperref[TEI.watermark]{watermark}\par 
    \item[namesdates: ]
   \hyperref[TEI.addName]{addName} \hyperref[TEI.affiliation]{affiliation} \hyperref[TEI.country]{country} \hyperref[TEI.forename]{forename} \hyperref[TEI.genName]{genName} \hyperref[TEI.geogName]{geogName} \hyperref[TEI.nameLink]{nameLink} \hyperref[TEI.orgName]{orgName} \hyperref[TEI.persName]{persName} \hyperref[TEI.person]{person} \hyperref[TEI.personGrp]{personGrp} \hyperref[TEI.persona]{persona} \hyperref[TEI.placeName]{placeName} \hyperref[TEI.region]{region} \hyperref[TEI.roleName]{roleName} \hyperref[TEI.settlement]{settlement} \hyperref[TEI.surname]{surname}\par 
    \item[spoken: ]
   \hyperref[TEI.annotationBlock]{annotationBlock}\par 
    \item[standOff: ]
   \hyperref[TEI.listAnnotation]{listAnnotation}\par 
    \item[textstructure: ]
   \hyperref[TEI.back]{back} \hyperref[TEI.body]{body} \hyperref[TEI.div]{div} \hyperref[TEI.docAuthor]{docAuthor} \hyperref[TEI.docDate]{docDate} \hyperref[TEI.docEdition]{docEdition} \hyperref[TEI.docTitle]{docTitle} \hyperref[TEI.floatingText]{floatingText} \hyperref[TEI.front]{front} \hyperref[TEI.group]{group} \hyperref[TEI.text]{text} \hyperref[TEI.titlePage]{titlePage} \hyperref[TEI.titlePart]{titlePart}\par 
    \item[transcr: ]
   \hyperref[TEI.damage]{damage} \hyperref[TEI.fw]{fw} \hyperref[TEI.line]{line} \hyperref[TEI.metamark]{metamark} \hyperref[TEI.mod]{mod} \hyperref[TEI.restore]{restore} \hyperref[TEI.retrace]{retrace} \hyperref[TEI.secl]{secl} \hyperref[TEI.sourceDoc]{sourceDoc} \hyperref[TEI.supplied]{supplied} \hyperref[TEI.surface]{surface} \hyperref[TEI.surfaceGrp]{surfaceGrp} \hyperref[TEI.surplus]{surplus} \hyperref[TEI.zone]{zone}
    \item[{Peut contenir}]
  
    \item[iso-fs: ]
   \hyperref[TEI.binary]{binary} \hyperref[TEI.default]{default} \hyperref[TEI.fs]{fs} \hyperref[TEI.numeric]{numeric} \hyperref[TEI.string]{string} \hyperref[TEI.symbol]{symbol} \hyperref[TEI.vAlt]{vAlt} \hyperref[TEI.vColl]{vColl} \hyperref[TEI.vLabel]{vLabel} \hyperref[TEI.vMerge]{vMerge} \hyperref[TEI.vNot]{vNot}
    \item[{Note}]
  \par
Une bibliothèque de valeurs de trait peut inclure n'importe quel nombre de valeurs quelconques, y compris des occurences multiples de valeurs identiques telles que \texttt{<binary value="true"/>} ou \texttt{default}. La seule chose absolument unique dans une bibliothèque de valeurs de trait est l'ensemble des étiquettes utilisées pour identifier les valeurs.
    \item[{Exemple}]
  \leavevmode\bgroup\exampleFont \begin{shaded}\noindent\mbox{}{<\textbf{fvLib}\hspace*{6pt}{n}="{symbolic values}">}\mbox{}\newline 
\hspace*{6pt}{<\textbf{symbol}\hspace*{6pt}{value}="{first}"\hspace*{6pt}{xml:id}="{fr\textunderscore sfirst}"/>}\mbox{}\newline 
\hspace*{6pt}{<\textbf{symbol}\hspace*{6pt}{value}="{second}"\hspace*{6pt}{xml:id}="{fr\textunderscore ssecond}"/>}\mbox{}\newline 
\hspace*{6pt}{<\textbf{symbol}\hspace*{6pt}{value}="{singular}"\hspace*{6pt}{xml:id}="{fr\textunderscore ssing}"/>}\mbox{}\newline 
\hspace*{6pt}{<\textbf{symbol}\hspace*{6pt}{value}="{plural}"\hspace*{6pt}{xml:id}="{fr\textunderscore splur}"/>}\mbox{}\newline 
{</\textbf{fvLib}>}\end{shaded}\egroup 


    \item[{Modèle de contenu}]
  \mbox{}\hfill\\[-10pt]\begin{Verbatim}[fontsize=\small]
<content>
 <classRef key="model.featureVal"
  maxOccurs="unbounded" minOccurs="0"/>
</content>
    
\end{Verbatim}

    \item[{Schéma Declaration}]
  \mbox{}\hfill\\[-10pt]\begin{Verbatim}[fontsize=\small]
element fvLib { tei_att.global.attributes, tei_model.featureVal* }
\end{Verbatim}

\end{reflist}  \index{fw=<fw>|oddindex}\index{type=@type!<fw>|oddindex}
\begin{reflist}
\item[]\begin{specHead}{TEI.fw}{<fw> }(élément de mise en page) permet d'encoder un titre courant (en haut ou en bas de la page), une réclame ou une autre information comparable, qui apparaît sur la page courante. [\xref{http://www.tei-c.org/release/doc/tei-p5-doc/en/html/PH.html\#PHSK}{11.6. Headers, Footers, and Similar Matter}]\end{specHead} 
    \item[{Module}]
  transcr
    \item[{Attributs}]
  Attributs \hyperref[TEI.att.global]{att.global} (\textit{@xml:id}, \textit{@n}, \textit{@xml:lang}, \textit{@xml:base}, \textit{@xml:space})  (\hyperref[TEI.att.global.rendition]{att.global.rendition} (\textit{@rend}, \textit{@style}, \textit{@rendition})) (\hyperref[TEI.att.global.linking]{att.global.linking} (\textit{@corresp}, \textit{@synch}, \textit{@sameAs}, \textit{@copyOf}, \textit{@next}, \textit{@prev}, \textit{@exclude}, \textit{@select})) (\hyperref[TEI.att.global.analytic]{att.global.analytic} (\textit{@ana})) (\hyperref[TEI.att.global.facs]{att.global.facs} (\textit{@facs})) (\hyperref[TEI.att.global.change]{att.global.change} (\textit{@change})) (\hyperref[TEI.att.global.responsibility]{att.global.responsibility} (\textit{@cert}, \textit{@resp})) (\hyperref[TEI.att.global.source]{att.global.source} (\textit{@source})) \hyperref[TEI.att.placement]{att.placement} (\textit{@place}) \hyperref[TEI.att.written]{att.written} (\textit{@hand}) \hfil\\[-10pt]\begin{sansreflist}
    \item[@type]
  caractérise l'information encodée conformément à une typologie appropriée.
\begin{reflist}
    \item[{Statut}]
  Recommendé
    \item[{Type de données}]
  \hyperref[TEI.teidata.enumerated]{teidata.enumerated}
    \item[{Exemple de valeurs possibles:}]
  \begin{description}

\item[{header}]un titre courant en haut de la page
\item[{footer}]un titre courant en bas de la page
\item[{pageNum}](numéro de page) un numéro de page ou un symbole de foliotation
\item[{lineNum}](numéro de ligne) numéro d'une ligne en prose ou en vers
\item[{sig}](signature) signature ou marque de cahier
\item[{catch}](réclame) une réclame
\end{description} 
\end{reflist}  
\end{sansreflist}  
    \item[{Membre du}]
  \hyperref[TEI.model.milestoneLike]{model.milestoneLike}
    \item[{Contenu dans}]
  
    \item[analysis: ]
   \hyperref[TEI.cl]{cl} \hyperref[TEI.m]{m} \hyperref[TEI.phr]{phr} \hyperref[TEI.s]{s} \hyperref[TEI.span]{span} \hyperref[TEI.w]{w}\par 
    \item[core: ]
   \hyperref[TEI.abbr]{abbr} \hyperref[TEI.add]{add} \hyperref[TEI.addrLine]{addrLine} \hyperref[TEI.address]{address} \hyperref[TEI.author]{author} \hyperref[TEI.bibl]{bibl} \hyperref[TEI.biblScope]{biblScope} \hyperref[TEI.cit]{cit} \hyperref[TEI.citedRange]{citedRange} \hyperref[TEI.corr]{corr} \hyperref[TEI.date]{date} \hyperref[TEI.del]{del} \hyperref[TEI.distinct]{distinct} \hyperref[TEI.editor]{editor} \hyperref[TEI.email]{email} \hyperref[TEI.emph]{emph} \hyperref[TEI.expan]{expan} \hyperref[TEI.foreign]{foreign} \hyperref[TEI.gloss]{gloss} \hyperref[TEI.head]{head} \hyperref[TEI.headItem]{headItem} \hyperref[TEI.headLabel]{headLabel} \hyperref[TEI.hi]{hi} \hyperref[TEI.imprint]{imprint} \hyperref[TEI.item]{item} \hyperref[TEI.l]{l} \hyperref[TEI.label]{label} \hyperref[TEI.lg]{lg} \hyperref[TEI.list]{list} \hyperref[TEI.listBibl]{listBibl} \hyperref[TEI.measure]{measure} \hyperref[TEI.mentioned]{mentioned} \hyperref[TEI.name]{name} \hyperref[TEI.note]{note} \hyperref[TEI.num]{num} \hyperref[TEI.orig]{orig} \hyperref[TEI.p]{p} \hyperref[TEI.pubPlace]{pubPlace} \hyperref[TEI.publisher]{publisher} \hyperref[TEI.q]{q} \hyperref[TEI.quote]{quote} \hyperref[TEI.ref]{ref} \hyperref[TEI.reg]{reg} \hyperref[TEI.resp]{resp} \hyperref[TEI.rs]{rs} \hyperref[TEI.said]{said} \hyperref[TEI.series]{series} \hyperref[TEI.sic]{sic} \hyperref[TEI.soCalled]{soCalled} \hyperref[TEI.sp]{sp} \hyperref[TEI.speaker]{speaker} \hyperref[TEI.stage]{stage} \hyperref[TEI.street]{street} \hyperref[TEI.term]{term} \hyperref[TEI.textLang]{textLang} \hyperref[TEI.time]{time} \hyperref[TEI.title]{title} \hyperref[TEI.unclear]{unclear}\par 
    \item[figures: ]
   \hyperref[TEI.cell]{cell} \hyperref[TEI.figure]{figure} \hyperref[TEI.table]{table}\par 
    \item[header: ]
   \hyperref[TEI.authority]{authority} \hyperref[TEI.change]{change} \hyperref[TEI.classCode]{classCode} \hyperref[TEI.distributor]{distributor} \hyperref[TEI.edition]{edition} \hyperref[TEI.extent]{extent} \hyperref[TEI.funder]{funder} \hyperref[TEI.language]{language} \hyperref[TEI.licence]{licence}\par 
    \item[linking: ]
   \hyperref[TEI.ab]{ab} \hyperref[TEI.seg]{seg}\par 
    \item[msdescription: ]
   \hyperref[TEI.accMat]{accMat} \hyperref[TEI.acquisition]{acquisition} \hyperref[TEI.additions]{additions} \hyperref[TEI.catchwords]{catchwords} \hyperref[TEI.collation]{collation} \hyperref[TEI.colophon]{colophon} \hyperref[TEI.condition]{condition} \hyperref[TEI.custEvent]{custEvent} \hyperref[TEI.decoNote]{decoNote} \hyperref[TEI.explicit]{explicit} \hyperref[TEI.filiation]{filiation} \hyperref[TEI.finalRubric]{finalRubric} \hyperref[TEI.foliation]{foliation} \hyperref[TEI.heraldry]{heraldry} \hyperref[TEI.incipit]{incipit} \hyperref[TEI.layout]{layout} \hyperref[TEI.material]{material} \hyperref[TEI.msItem]{msItem} \hyperref[TEI.musicNotation]{musicNotation} \hyperref[TEI.objectType]{objectType} \hyperref[TEI.origDate]{origDate} \hyperref[TEI.origPlace]{origPlace} \hyperref[TEI.origin]{origin} \hyperref[TEI.provenance]{provenance} \hyperref[TEI.rubric]{rubric} \hyperref[TEI.secFol]{secFol} \hyperref[TEI.signatures]{signatures} \hyperref[TEI.source]{source} \hyperref[TEI.stamp]{stamp} \hyperref[TEI.summary]{summary} \hyperref[TEI.support]{support} \hyperref[TEI.surrogates]{surrogates} \hyperref[TEI.typeNote]{typeNote} \hyperref[TEI.watermark]{watermark}\par 
    \item[namesdates: ]
   \hyperref[TEI.addName]{addName} \hyperref[TEI.affiliation]{affiliation} \hyperref[TEI.country]{country} \hyperref[TEI.forename]{forename} \hyperref[TEI.genName]{genName} \hyperref[TEI.geogName]{geogName} \hyperref[TEI.nameLink]{nameLink} \hyperref[TEI.org]{org} \hyperref[TEI.orgName]{orgName} \hyperref[TEI.persName]{persName} \hyperref[TEI.person]{person} \hyperref[TEI.personGrp]{personGrp} \hyperref[TEI.persona]{persona} \hyperref[TEI.placeName]{placeName} \hyperref[TEI.region]{region} \hyperref[TEI.roleName]{roleName} \hyperref[TEI.settlement]{settlement} \hyperref[TEI.surname]{surname}\par 
    \item[textstructure: ]
   \hyperref[TEI.back]{back} \hyperref[TEI.body]{body} \hyperref[TEI.div]{div} \hyperref[TEI.docAuthor]{docAuthor} \hyperref[TEI.docDate]{docDate} \hyperref[TEI.docEdition]{docEdition} \hyperref[TEI.docTitle]{docTitle} \hyperref[TEI.floatingText]{floatingText} \hyperref[TEI.front]{front} \hyperref[TEI.group]{group} \hyperref[TEI.text]{text} \hyperref[TEI.titlePage]{titlePage} \hyperref[TEI.titlePart]{titlePart}\par 
    \item[transcr: ]
   \hyperref[TEI.damage]{damage} \hyperref[TEI.fw]{fw} \hyperref[TEI.line]{line} \hyperref[TEI.metamark]{metamark} \hyperref[TEI.mod]{mod} \hyperref[TEI.restore]{restore} \hyperref[TEI.retrace]{retrace} \hyperref[TEI.secl]{secl} \hyperref[TEI.sourceDoc]{sourceDoc} \hyperref[TEI.subst]{subst} \hyperref[TEI.supplied]{supplied} \hyperref[TEI.surface]{surface} \hyperref[TEI.surfaceGrp]{surfaceGrp} \hyperref[TEI.surplus]{surplus} \hyperref[TEI.zone]{zone}
    \item[{Peut contenir}]
  
    \item[analysis: ]
   \hyperref[TEI.c]{c} \hyperref[TEI.cl]{cl} \hyperref[TEI.interp]{interp} \hyperref[TEI.interpGrp]{interpGrp} \hyperref[TEI.m]{m} \hyperref[TEI.pc]{pc} \hyperref[TEI.phr]{phr} \hyperref[TEI.s]{s} \hyperref[TEI.span]{span} \hyperref[TEI.spanGrp]{spanGrp} \hyperref[TEI.w]{w}\par 
    \item[core: ]
   \hyperref[TEI.abbr]{abbr} \hyperref[TEI.add]{add} \hyperref[TEI.address]{address} \hyperref[TEI.binaryObject]{binaryObject} \hyperref[TEI.cb]{cb} \hyperref[TEI.choice]{choice} \hyperref[TEI.corr]{corr} \hyperref[TEI.date]{date} \hyperref[TEI.del]{del} \hyperref[TEI.distinct]{distinct} \hyperref[TEI.email]{email} \hyperref[TEI.emph]{emph} \hyperref[TEI.expan]{expan} \hyperref[TEI.foreign]{foreign} \hyperref[TEI.gap]{gap} \hyperref[TEI.gb]{gb} \hyperref[TEI.gloss]{gloss} \hyperref[TEI.graphic]{graphic} \hyperref[TEI.hi]{hi} \hyperref[TEI.index]{index} \hyperref[TEI.lb]{lb} \hyperref[TEI.measure]{measure} \hyperref[TEI.measureGrp]{measureGrp} \hyperref[TEI.media]{media} \hyperref[TEI.mentioned]{mentioned} \hyperref[TEI.milestone]{milestone} \hyperref[TEI.name]{name} \hyperref[TEI.note]{note} \hyperref[TEI.num]{num} \hyperref[TEI.orig]{orig} \hyperref[TEI.pb]{pb} \hyperref[TEI.ptr]{ptr} \hyperref[TEI.ref]{ref} \hyperref[TEI.reg]{reg} \hyperref[TEI.rs]{rs} \hyperref[TEI.sic]{sic} \hyperref[TEI.soCalled]{soCalled} \hyperref[TEI.term]{term} \hyperref[TEI.time]{time} \hyperref[TEI.title]{title} \hyperref[TEI.unclear]{unclear}\par 
    \item[derived-module-tei.istex: ]
   \hyperref[TEI.math]{math} \hyperref[TEI.mrow]{mrow}\par 
    \item[figures: ]
   \hyperref[TEI.figure]{figure} \hyperref[TEI.formula]{formula} \hyperref[TEI.notatedMusic]{notatedMusic}\par 
    \item[header: ]
   \hyperref[TEI.idno]{idno}\par 
    \item[iso-fs: ]
   \hyperref[TEI.fLib]{fLib} \hyperref[TEI.fs]{fs} \hyperref[TEI.fvLib]{fvLib}\par 
    \item[linking: ]
   \hyperref[TEI.alt]{alt} \hyperref[TEI.altGrp]{altGrp} \hyperref[TEI.anchor]{anchor} \hyperref[TEI.join]{join} \hyperref[TEI.joinGrp]{joinGrp} \hyperref[TEI.link]{link} \hyperref[TEI.linkGrp]{linkGrp} \hyperref[TEI.seg]{seg} \hyperref[TEI.timeline]{timeline}\par 
    \item[msdescription: ]
   \hyperref[TEI.catchwords]{catchwords} \hyperref[TEI.depth]{depth} \hyperref[TEI.dim]{dim} \hyperref[TEI.dimensions]{dimensions} \hyperref[TEI.height]{height} \hyperref[TEI.heraldry]{heraldry} \hyperref[TEI.locus]{locus} \hyperref[TEI.locusGrp]{locusGrp} \hyperref[TEI.material]{material} \hyperref[TEI.objectType]{objectType} \hyperref[TEI.origDate]{origDate} \hyperref[TEI.origPlace]{origPlace} \hyperref[TEI.secFol]{secFol} \hyperref[TEI.signatures]{signatures} \hyperref[TEI.source]{source} \hyperref[TEI.stamp]{stamp} \hyperref[TEI.watermark]{watermark} \hyperref[TEI.width]{width}\par 
    \item[namesdates: ]
   \hyperref[TEI.addName]{addName} \hyperref[TEI.affiliation]{affiliation} \hyperref[TEI.country]{country} \hyperref[TEI.forename]{forename} \hyperref[TEI.genName]{genName} \hyperref[TEI.geogName]{geogName} \hyperref[TEI.location]{location} \hyperref[TEI.nameLink]{nameLink} \hyperref[TEI.orgName]{orgName} \hyperref[TEI.persName]{persName} \hyperref[TEI.placeName]{placeName} \hyperref[TEI.region]{region} \hyperref[TEI.roleName]{roleName} \hyperref[TEI.settlement]{settlement} \hyperref[TEI.state]{state} \hyperref[TEI.surname]{surname}\par 
    \item[spoken: ]
   \hyperref[TEI.annotationBlock]{annotationBlock}\par 
    \item[transcr: ]
   \hyperref[TEI.addSpan]{addSpan} \hyperref[TEI.am]{am} \hyperref[TEI.damage]{damage} \hyperref[TEI.damageSpan]{damageSpan} \hyperref[TEI.delSpan]{delSpan} \hyperref[TEI.ex]{ex} \hyperref[TEI.fw]{fw} \hyperref[TEI.handShift]{handShift} \hyperref[TEI.listTranspose]{listTranspose} \hyperref[TEI.metamark]{metamark} \hyperref[TEI.mod]{mod} \hyperref[TEI.redo]{redo} \hyperref[TEI.restore]{restore} \hyperref[TEI.retrace]{retrace} \hyperref[TEI.secl]{secl} \hyperref[TEI.space]{space} \hyperref[TEI.subst]{subst} \hyperref[TEI.substJoin]{substJoin} \hyperref[TEI.supplied]{supplied} \hyperref[TEI.surplus]{surplus} \hyperref[TEI.undo]{undo}\par des données textuelles
    \item[{Note}]
  \par
Quand le titre courant s'applique à tout un chapitre ou une section, il est habituellement plus commode de le relier au chapitre ou à la section, par exemple en utilisant l'attribut {\itshape rend}. L'élément \hyperref[TEI.fw]{<fw>} est utilisé pour des cas où le titre courant change de page en page ou quand il est primordial de relever des détails de mise en page ou la structure interne des titres courants.
    \item[{Exemple}]
  \leavevmode\bgroup\exampleFont \begin{shaded}\noindent\mbox{}{<\textbf{fw}\hspace*{6pt}{place}="{bot}"\hspace*{6pt}{type}="{sig}">}C3{</\textbf{fw}>}\end{shaded}\egroup 


    \item[{Modèle de contenu}]
  \mbox{}\hfill\\[-10pt]\begin{Verbatim}[fontsize=\small]
<content>
 <macroRef key="macro.phraseSeq"/>
</content>
    
\end{Verbatim}

    \item[{Schéma Declaration}]
  \mbox{}\hfill\\[-10pt]\begin{Verbatim}[fontsize=\small]
element fw
{
   tei_att.global.attributes,
   tei_att.placement.attributes,
   tei_att.written.attributes,
   attribute type { text }?,
   tei_macro.phraseSeq}
\end{Verbatim}

\end{reflist}  \index{gap=<gap>|oddindex}\index{reason=@reason!<gap>|oddindex}\index{hand=@hand!<gap>|oddindex}\index{agent=@agent!<gap>|oddindex}
\begin{reflist}
\item[]\begin{specHead}{TEI.gap}{<gap> }(omission) indique une omission dans une transcription, soit pour des raisons éditoriales décrites dans l'en-tête TEI au cours d’un échantillonnage, soit parce que le matériau est illisible ou inaudible. [\xref{http://www.tei-c.org/release/doc/tei-p5-doc/en/html/CO.html\#COEDADD}{3.4.3. Additions, Deletions, and Omissions}]\end{specHead} 
    \item[{Module}]
  core
    \item[{Attributs}]
  Attributs \hyperref[TEI.att.global]{att.global} (\textit{@xml:id}, \textit{@n}, \textit{@xml:lang}, \textit{@xml:base}, \textit{@xml:space})  (\hyperref[TEI.att.global.rendition]{att.global.rendition} (\textit{@rend}, \textit{@style}, \textit{@rendition})) (\hyperref[TEI.att.global.linking]{att.global.linking} (\textit{@corresp}, \textit{@synch}, \textit{@sameAs}, \textit{@copyOf}, \textit{@next}, \textit{@prev}, \textit{@exclude}, \textit{@select})) (\hyperref[TEI.att.global.analytic]{att.global.analytic} (\textit{@ana})) (\hyperref[TEI.att.global.facs]{att.global.facs} (\textit{@facs})) (\hyperref[TEI.att.global.change]{att.global.change} (\textit{@change})) (\hyperref[TEI.att.global.responsibility]{att.global.responsibility} (\textit{@cert}, \textit{@resp})) (\hyperref[TEI.att.global.source]{att.global.source} (\textit{@source})) \hyperref[TEI.att.timed]{att.timed} (\textit{@start}, \textit{@end})  (\hyperref[TEI.att.duration]{att.duration} (\hyperref[TEI.att.duration.w3c]{att.duration.w3c} (\textit{@dur})) (\hyperref[TEI.att.duration.iso]{att.duration.iso} (\textit{@dur-iso})) ) \hyperref[TEI.att.editLike]{att.editLike} (\textit{@evidence}, \textit{@instant})  (\hyperref[TEI.att.dimensions]{att.dimensions} (\textit{@unit}, \textit{@quantity}, \textit{@extent}, \textit{@precision}, \textit{@scope}) (\hyperref[TEI.att.ranging]{att.ranging} (\textit{@atLeast}, \textit{@atMost}, \textit{@min}, \textit{@max}, \textit{@confidence})) ) \hfil\\[-10pt]\begin{sansreflist}
    \item[@reason]
  donne la raison de l'omission
\begin{reflist}
    \item[{Statut}]
  Optionel
    \item[{Type de données}]
  1–∞ occurrences de \hyperref[TEI.teidata.enumerated]{teidata.enumerated} séparé par un espace
    \item[{Les valeurs suggérées comprennent:}]
  \begin{description}

\item[{cancelled}](biffé)
\item[{deleted}]
\item[{editorial}]for features omitted from transcription due to editorial policy
\item[{illegible}](illisible)
\item[{inaudible}](inaudible)
\item[{irrelevant}](non pertinent)
\item[{sampling}](échantillonnage)
\end{description} 
\end{reflist}  
    \item[@hand]
  lorsque du texte est omis de la transcription en raison d'une suppression volontaire par une main identifiable, indique quelle est cette main.
\begin{reflist}
    \item[\xref{http://www.tei-c.org/Activities/Council/Working/tcw27.xml}{Deprecated}]
  will be removed on 2017-08-01
    \item[{Statut}]
  Optionel
    \item[{Type de données}]
  \hyperref[TEI.teidata.pointer]{teidata.pointer}
\end{reflist}  
    \item[@agent]
  lorsque du texte est omis de la transcription en raison d'un dommage, catégorise la cause du dommage, si celle-ci peut être identifiée.
\begin{reflist}
    \item[{Statut}]
  Optionel
    \item[{Type de données}]
  \hyperref[TEI.teidata.enumerated]{teidata.enumerated}
    \item[{Exemple de valeurs possibles:}]
  \begin{description}

\item[{rubbing}]dégâts provoqués par le frottement des bords de la feuille.
\item[{mildew}]dégâts provoqués par de la moisissure sur la surface de feuille.
\item[{smoke}]dégâts provoqués par de la fumée.
\end{description} 
\end{reflist}  
\end{sansreflist}  
    \item[{Membre du}]
  \hyperref[TEI.model.global.edit]{model.global.edit}
    \item[{Contenu dans}]
  
    \item[analysis: ]
   \hyperref[TEI.cl]{cl} \hyperref[TEI.m]{m} \hyperref[TEI.phr]{phr} \hyperref[TEI.s]{s} \hyperref[TEI.span]{span} \hyperref[TEI.w]{w}\par 
    \item[core: ]
   \hyperref[TEI.abbr]{abbr} \hyperref[TEI.add]{add} \hyperref[TEI.addrLine]{addrLine} \hyperref[TEI.address]{address} \hyperref[TEI.author]{author} \hyperref[TEI.bibl]{bibl} \hyperref[TEI.biblScope]{biblScope} \hyperref[TEI.cit]{cit} \hyperref[TEI.citedRange]{citedRange} \hyperref[TEI.corr]{corr} \hyperref[TEI.date]{date} \hyperref[TEI.del]{del} \hyperref[TEI.distinct]{distinct} \hyperref[TEI.editor]{editor} \hyperref[TEI.email]{email} \hyperref[TEI.emph]{emph} \hyperref[TEI.expan]{expan} \hyperref[TEI.foreign]{foreign} \hyperref[TEI.gloss]{gloss} \hyperref[TEI.head]{head} \hyperref[TEI.headItem]{headItem} \hyperref[TEI.headLabel]{headLabel} \hyperref[TEI.hi]{hi} \hyperref[TEI.imprint]{imprint} \hyperref[TEI.item]{item} \hyperref[TEI.l]{l} \hyperref[TEI.label]{label} \hyperref[TEI.lg]{lg} \hyperref[TEI.list]{list} \hyperref[TEI.measure]{measure} \hyperref[TEI.mentioned]{mentioned} \hyperref[TEI.name]{name} \hyperref[TEI.note]{note} \hyperref[TEI.num]{num} \hyperref[TEI.orig]{orig} \hyperref[TEI.p]{p} \hyperref[TEI.pubPlace]{pubPlace} \hyperref[TEI.publisher]{publisher} \hyperref[TEI.q]{q} \hyperref[TEI.quote]{quote} \hyperref[TEI.ref]{ref} \hyperref[TEI.reg]{reg} \hyperref[TEI.resp]{resp} \hyperref[TEI.rs]{rs} \hyperref[TEI.said]{said} \hyperref[TEI.series]{series} \hyperref[TEI.sic]{sic} \hyperref[TEI.soCalled]{soCalled} \hyperref[TEI.sp]{sp} \hyperref[TEI.speaker]{speaker} \hyperref[TEI.stage]{stage} \hyperref[TEI.street]{street} \hyperref[TEI.term]{term} \hyperref[TEI.textLang]{textLang} \hyperref[TEI.time]{time} \hyperref[TEI.title]{title} \hyperref[TEI.unclear]{unclear}\par 
    \item[figures: ]
   \hyperref[TEI.cell]{cell} \hyperref[TEI.figure]{figure} \hyperref[TEI.table]{table}\par 
    \item[header: ]
   \hyperref[TEI.authority]{authority} \hyperref[TEI.change]{change} \hyperref[TEI.classCode]{classCode} \hyperref[TEI.distributor]{distributor} \hyperref[TEI.edition]{edition} \hyperref[TEI.extent]{extent} \hyperref[TEI.funder]{funder} \hyperref[TEI.language]{language} \hyperref[TEI.licence]{licence}\par 
    \item[linking: ]
   \hyperref[TEI.ab]{ab} \hyperref[TEI.seg]{seg}\par 
    \item[msdescription: ]
   \hyperref[TEI.accMat]{accMat} \hyperref[TEI.acquisition]{acquisition} \hyperref[TEI.additions]{additions} \hyperref[TEI.catchwords]{catchwords} \hyperref[TEI.collation]{collation} \hyperref[TEI.colophon]{colophon} \hyperref[TEI.condition]{condition} \hyperref[TEI.custEvent]{custEvent} \hyperref[TEI.decoNote]{decoNote} \hyperref[TEI.explicit]{explicit} \hyperref[TEI.filiation]{filiation} \hyperref[TEI.finalRubric]{finalRubric} \hyperref[TEI.foliation]{foliation} \hyperref[TEI.heraldry]{heraldry} \hyperref[TEI.incipit]{incipit} \hyperref[TEI.layout]{layout} \hyperref[TEI.material]{material} \hyperref[TEI.msItem]{msItem} \hyperref[TEI.musicNotation]{musicNotation} \hyperref[TEI.objectType]{objectType} \hyperref[TEI.origDate]{origDate} \hyperref[TEI.origPlace]{origPlace} \hyperref[TEI.origin]{origin} \hyperref[TEI.provenance]{provenance} \hyperref[TEI.rubric]{rubric} \hyperref[TEI.secFol]{secFol} \hyperref[TEI.signatures]{signatures} \hyperref[TEI.source]{source} \hyperref[TEI.stamp]{stamp} \hyperref[TEI.summary]{summary} \hyperref[TEI.support]{support} \hyperref[TEI.surrogates]{surrogates} \hyperref[TEI.typeNote]{typeNote} \hyperref[TEI.watermark]{watermark}\par 
    \item[namesdates: ]
   \hyperref[TEI.addName]{addName} \hyperref[TEI.affiliation]{affiliation} \hyperref[TEI.country]{country} \hyperref[TEI.forename]{forename} \hyperref[TEI.genName]{genName} \hyperref[TEI.geogName]{geogName} \hyperref[TEI.nameLink]{nameLink} \hyperref[TEI.orgName]{orgName} \hyperref[TEI.persName]{persName} \hyperref[TEI.person]{person} \hyperref[TEI.personGrp]{personGrp} \hyperref[TEI.persona]{persona} \hyperref[TEI.placeName]{placeName} \hyperref[TEI.region]{region} \hyperref[TEI.roleName]{roleName} \hyperref[TEI.settlement]{settlement} \hyperref[TEI.surname]{surname}\par 
    \item[textstructure: ]
   \hyperref[TEI.back]{back} \hyperref[TEI.body]{body} \hyperref[TEI.div]{div} \hyperref[TEI.docAuthor]{docAuthor} \hyperref[TEI.docDate]{docDate} \hyperref[TEI.docEdition]{docEdition} \hyperref[TEI.docTitle]{docTitle} \hyperref[TEI.floatingText]{floatingText} \hyperref[TEI.front]{front} \hyperref[TEI.group]{group} \hyperref[TEI.text]{text} \hyperref[TEI.titlePage]{titlePage} \hyperref[TEI.titlePart]{titlePart}\par 
    \item[transcr: ]
   \hyperref[TEI.damage]{damage} \hyperref[TEI.fw]{fw} \hyperref[TEI.line]{line} \hyperref[TEI.metamark]{metamark} \hyperref[TEI.mod]{mod} \hyperref[TEI.restore]{restore} \hyperref[TEI.retrace]{retrace} \hyperref[TEI.secl]{secl} \hyperref[TEI.sourceDoc]{sourceDoc} \hyperref[TEI.supplied]{supplied} \hyperref[TEI.surface]{surface} \hyperref[TEI.surfaceGrp]{surfaceGrp} \hyperref[TEI.surplus]{surplus} \hyperref[TEI.zone]{zone}
    \item[{Peut contenir}]
  
    \item[core: ]
   \hyperref[TEI.desc]{desc}
    \item[{Note}]
  \par
Les éléments du jeu de balises de base \hyperref[TEI.gap]{<gap>}, \hyperref[TEI.unclear]{<unclear>}, et \hyperref[TEI.del]{<del>} peuvent être étroitement associés avec l'utilisation des éléments \hyperref[TEI.damage]{<damage>} et \hyperref[TEI.supplied]{<supplied>} qui sont disponibles si l'on utilise le jeu de balises additionnel pour la transcription des sources primaires. Voir la section \xref{http://www.tei-c.org/release/doc/tei-p5-doc/en/html/PH.html\#PHCOMB}{11.3.3.2. Use of the gap, del, damage, unclear, and supplied Elements in Combination} pour plus de détails sur l'élément le plus pertinent suivant les circonstances.
    \item[{Exemple}]
  \leavevmode\bgroup\exampleFont \begin{shaded}\noindent\mbox{}{<\textbf{gap}\hspace*{6pt}{quantity}="{4}"\hspace*{6pt}{reason}="{illegible}"\mbox{}\newline 
\hspace*{6pt}{unit}="{chars}"/>}\end{shaded}\egroup 


    \item[{Exemple}]
  \leavevmode\bgroup\exampleFont \begin{shaded}\noindent\mbox{}{<\textbf{gap}\hspace*{6pt}{quantity}="{1}"\hspace*{6pt}{reason}="{sampling}"\mbox{}\newline 
\hspace*{6pt}{unit}="{essay}"/>}\end{shaded}\egroup 


    \item[{Modèle de contenu}]
  \mbox{}\hfill\\[-10pt]\begin{Verbatim}[fontsize=\small]
<content>
 <alternate maxOccurs="unbounded"
  minOccurs="0">
  <classRef key="model.descLike"/>
  <classRef key="model.certLike"/>
 </alternate>
</content>
    
\end{Verbatim}

    \item[{Schéma Declaration}]
  \mbox{}\hfill\\[-10pt]\begin{Verbatim}[fontsize=\small]
element gap
{
   tei_att.global.attributes,
   tei_att.timed.attributes,
   tei_att.editLike.attributes,
   attribute reason
   {
      list
      {
         (
            "cancelled"
          | "deleted"
          | "editorial"
          | "illegible"
          | "inaudible"
          | "irrelevant"
          | "sampling"
         )+
      }
   }?,
   attribute hand { text }?,
   attribute agent { text }?,
   ( tei_model.descLike | tei_model.certLike )*
}
\end{Verbatim}

\end{reflist}  \index{gb=<gb>|oddindex}
\begin{reflist}
\item[]\begin{specHead}{TEI.gb}{<gb> }(gathering beginning) marks the beginning of a new gathering or quire in a transcribed codex. [\xref{http://www.tei-c.org/release/doc/tei-p5-doc/en/html/CO.html\#CORS5}{3.10.3. Milestone Elements}]\end{specHead} 
    \item[{Module}]
  core
    \item[{Attributs}]
  Attributs \hyperref[TEI.att.global]{att.global} (\textit{@xml:id}, \textit{@n}, \textit{@xml:lang}, \textit{@xml:base}, \textit{@xml:space})  (\hyperref[TEI.att.global.rendition]{att.global.rendition} (\textit{@rend}, \textit{@style}, \textit{@rendition})) (\hyperref[TEI.att.global.linking]{att.global.linking} (\textit{@corresp}, \textit{@synch}, \textit{@sameAs}, \textit{@copyOf}, \textit{@next}, \textit{@prev}, \textit{@exclude}, \textit{@select})) (\hyperref[TEI.att.global.analytic]{att.global.analytic} (\textit{@ana})) (\hyperref[TEI.att.global.facs]{att.global.facs} (\textit{@facs})) (\hyperref[TEI.att.global.change]{att.global.change} (\textit{@change})) (\hyperref[TEI.att.global.responsibility]{att.global.responsibility} (\textit{@cert}, \textit{@resp})) (\hyperref[TEI.att.global.source]{att.global.source} (\textit{@source})) \hyperref[TEI.att.typed]{att.typed} (\textit{@type}, \textit{@subtype}) \hyperref[TEI.att.spanning]{att.spanning} (\textit{@spanTo}) \hyperref[TEI.att.breaking]{att.breaking} (\textit{@break}) 
    \item[{Membre du}]
  \hyperref[TEI.model.milestoneLike]{model.milestoneLike}
    \item[{Contenu dans}]
  
    \item[analysis: ]
   \hyperref[TEI.cl]{cl} \hyperref[TEI.m]{m} \hyperref[TEI.phr]{phr} \hyperref[TEI.s]{s} \hyperref[TEI.span]{span} \hyperref[TEI.w]{w}\par 
    \item[core: ]
   \hyperref[TEI.abbr]{abbr} \hyperref[TEI.add]{add} \hyperref[TEI.addrLine]{addrLine} \hyperref[TEI.address]{address} \hyperref[TEI.author]{author} \hyperref[TEI.bibl]{bibl} \hyperref[TEI.biblScope]{biblScope} \hyperref[TEI.cit]{cit} \hyperref[TEI.citedRange]{citedRange} \hyperref[TEI.corr]{corr} \hyperref[TEI.date]{date} \hyperref[TEI.del]{del} \hyperref[TEI.distinct]{distinct} \hyperref[TEI.editor]{editor} \hyperref[TEI.email]{email} \hyperref[TEI.emph]{emph} \hyperref[TEI.expan]{expan} \hyperref[TEI.foreign]{foreign} \hyperref[TEI.gloss]{gloss} \hyperref[TEI.head]{head} \hyperref[TEI.headItem]{headItem} \hyperref[TEI.headLabel]{headLabel} \hyperref[TEI.hi]{hi} \hyperref[TEI.imprint]{imprint} \hyperref[TEI.item]{item} \hyperref[TEI.l]{l} \hyperref[TEI.label]{label} \hyperref[TEI.lg]{lg} \hyperref[TEI.list]{list} \hyperref[TEI.listBibl]{listBibl} \hyperref[TEI.measure]{measure} \hyperref[TEI.mentioned]{mentioned} \hyperref[TEI.name]{name} \hyperref[TEI.note]{note} \hyperref[TEI.num]{num} \hyperref[TEI.orig]{orig} \hyperref[TEI.p]{p} \hyperref[TEI.pubPlace]{pubPlace} \hyperref[TEI.publisher]{publisher} \hyperref[TEI.q]{q} \hyperref[TEI.quote]{quote} \hyperref[TEI.ref]{ref} \hyperref[TEI.reg]{reg} \hyperref[TEI.resp]{resp} \hyperref[TEI.rs]{rs} \hyperref[TEI.said]{said} \hyperref[TEI.series]{series} \hyperref[TEI.sic]{sic} \hyperref[TEI.soCalled]{soCalled} \hyperref[TEI.sp]{sp} \hyperref[TEI.speaker]{speaker} \hyperref[TEI.stage]{stage} \hyperref[TEI.street]{street} \hyperref[TEI.term]{term} \hyperref[TEI.textLang]{textLang} \hyperref[TEI.time]{time} \hyperref[TEI.title]{title} \hyperref[TEI.unclear]{unclear}\par 
    \item[figures: ]
   \hyperref[TEI.cell]{cell} \hyperref[TEI.figure]{figure} \hyperref[TEI.table]{table}\par 
    \item[header: ]
   \hyperref[TEI.authority]{authority} \hyperref[TEI.change]{change} \hyperref[TEI.classCode]{classCode} \hyperref[TEI.distributor]{distributor} \hyperref[TEI.edition]{edition} \hyperref[TEI.extent]{extent} \hyperref[TEI.funder]{funder} \hyperref[TEI.language]{language} \hyperref[TEI.licence]{licence}\par 
    \item[linking: ]
   \hyperref[TEI.ab]{ab} \hyperref[TEI.seg]{seg}\par 
    \item[msdescription: ]
   \hyperref[TEI.accMat]{accMat} \hyperref[TEI.acquisition]{acquisition} \hyperref[TEI.additions]{additions} \hyperref[TEI.catchwords]{catchwords} \hyperref[TEI.collation]{collation} \hyperref[TEI.colophon]{colophon} \hyperref[TEI.condition]{condition} \hyperref[TEI.custEvent]{custEvent} \hyperref[TEI.decoNote]{decoNote} \hyperref[TEI.explicit]{explicit} \hyperref[TEI.filiation]{filiation} \hyperref[TEI.finalRubric]{finalRubric} \hyperref[TEI.foliation]{foliation} \hyperref[TEI.heraldry]{heraldry} \hyperref[TEI.incipit]{incipit} \hyperref[TEI.layout]{layout} \hyperref[TEI.material]{material} \hyperref[TEI.msItem]{msItem} \hyperref[TEI.musicNotation]{musicNotation} \hyperref[TEI.objectType]{objectType} \hyperref[TEI.origDate]{origDate} \hyperref[TEI.origPlace]{origPlace} \hyperref[TEI.origin]{origin} \hyperref[TEI.provenance]{provenance} \hyperref[TEI.rubric]{rubric} \hyperref[TEI.secFol]{secFol} \hyperref[TEI.signatures]{signatures} \hyperref[TEI.source]{source} \hyperref[TEI.stamp]{stamp} \hyperref[TEI.summary]{summary} \hyperref[TEI.support]{support} \hyperref[TEI.surrogates]{surrogates} \hyperref[TEI.typeNote]{typeNote} \hyperref[TEI.watermark]{watermark}\par 
    \item[namesdates: ]
   \hyperref[TEI.addName]{addName} \hyperref[TEI.affiliation]{affiliation} \hyperref[TEI.country]{country} \hyperref[TEI.forename]{forename} \hyperref[TEI.genName]{genName} \hyperref[TEI.geogName]{geogName} \hyperref[TEI.nameLink]{nameLink} \hyperref[TEI.org]{org} \hyperref[TEI.orgName]{orgName} \hyperref[TEI.persName]{persName} \hyperref[TEI.person]{person} \hyperref[TEI.personGrp]{personGrp} \hyperref[TEI.persona]{persona} \hyperref[TEI.placeName]{placeName} \hyperref[TEI.region]{region} \hyperref[TEI.roleName]{roleName} \hyperref[TEI.settlement]{settlement} \hyperref[TEI.surname]{surname}\par 
    \item[textstructure: ]
   \hyperref[TEI.back]{back} \hyperref[TEI.body]{body} \hyperref[TEI.div]{div} \hyperref[TEI.docAuthor]{docAuthor} \hyperref[TEI.docDate]{docDate} \hyperref[TEI.docEdition]{docEdition} \hyperref[TEI.docTitle]{docTitle} \hyperref[TEI.floatingText]{floatingText} \hyperref[TEI.front]{front} \hyperref[TEI.group]{group} \hyperref[TEI.text]{text} \hyperref[TEI.titlePage]{titlePage} \hyperref[TEI.titlePart]{titlePart}\par 
    \item[transcr: ]
   \hyperref[TEI.damage]{damage} \hyperref[TEI.fw]{fw} \hyperref[TEI.line]{line} \hyperref[TEI.metamark]{metamark} \hyperref[TEI.mod]{mod} \hyperref[TEI.restore]{restore} \hyperref[TEI.retrace]{retrace} \hyperref[TEI.secl]{secl} \hyperref[TEI.sourceDoc]{sourceDoc} \hyperref[TEI.subst]{subst} \hyperref[TEI.supplied]{supplied} \hyperref[TEI.surface]{surface} \hyperref[TEI.surfaceGrp]{surfaceGrp} \hyperref[TEI.surplus]{surplus} \hyperref[TEI.zone]{zone}
    \item[{Peut contenir}]
  Elément vide
    \item[{Note}]
  \par
By convention, \hyperref[TEI.gb]{<gb>} elements should appear at the start of the first page in the gathering. The global {\itshape n} attribute indicates the number or other value used to identify this gathering in a collation.\par
The {\itshape type} attribute may be used to further characterize the gathering in any respect.
    \item[{Exemple}]
  \leavevmode\bgroup\exampleFont \begin{shaded}\noindent\mbox{}{<\textbf{gb}\hspace*{6pt}{n}="{iii}"/>}\mbox{}\newline 
{<\textbf{pb}\hspace*{6pt}{n}="{2r}"/>}\mbox{}\newline 
\textit{<!-- material from page 2 recto of gathering iii here -->}\mbox{}\newline 
{<\textbf{pb}\hspace*{6pt}{n}="{2v}"/>}\mbox{}\newline 
\textit{<!-- material from page 2 verso of gathering iii here -->}\end{shaded}\egroup 


    \item[{Modèle de contenu}]
  \fbox{\ttfamily <content>\newline
</content>\newline
    } 
    \item[{Schéma Declaration}]
  \mbox{}\hfill\\[-10pt]\begin{Verbatim}[fontsize=\small]
element gb
{
   tei_att.global.attributes,
   tei_att.typed.attributes,
   tei_att.spanning.attributes,
   tei_att.breaking.attributes,
   empty
}
\end{Verbatim}

\end{reflist}  \index{genName=<genName>|oddindex}
\begin{reflist}
\item[]\begin{specHead}{TEI.genName}{<genName> }(qualificatif générationnel de nom) contient une composante de nom utilisée pour distinguer des noms, par ailleurs similaires, sur la base de l'âge ou de la génération des personnes concernées. [\xref{http://www.tei-c.org/release/doc/tei-p5-doc/en/html/ND.html\#NDPER}{13.2.1. Personal Names}]\end{specHead} 
    \item[{Module}]
  namesdates
    \item[{Attributs}]
  Attributs \hyperref[TEI.att.global]{att.global} (\textit{@xml:id}, \textit{@n}, \textit{@xml:lang}, \textit{@xml:base}, \textit{@xml:space})  (\hyperref[TEI.att.global.rendition]{att.global.rendition} (\textit{@rend}, \textit{@style}, \textit{@rendition})) (\hyperref[TEI.att.global.linking]{att.global.linking} (\textit{@corresp}, \textit{@synch}, \textit{@sameAs}, \textit{@copyOf}, \textit{@next}, \textit{@prev}, \textit{@exclude}, \textit{@select})) (\hyperref[TEI.att.global.analytic]{att.global.analytic} (\textit{@ana})) (\hyperref[TEI.att.global.facs]{att.global.facs} (\textit{@facs})) (\hyperref[TEI.att.global.change]{att.global.change} (\textit{@change})) (\hyperref[TEI.att.global.responsibility]{att.global.responsibility} (\textit{@cert}, \textit{@resp})) (\hyperref[TEI.att.global.source]{att.global.source} (\textit{@source})) \hyperref[TEI.att.personal]{att.personal} (\textit{@full}, \textit{@sort})  (\hyperref[TEI.att.naming]{att.naming} (\textit{@role}, \textit{@nymRef}) (\hyperref[TEI.att.canonical]{att.canonical} (\textit{@key}, \textit{@ref})) ) \hyperref[TEI.att.typed]{att.typed} (\textit{@type}, \textit{@subtype}) 
    \item[{Membre du}]
  \hyperref[TEI.model.persNamePart]{model.persNamePart}
    \item[{Contenu dans}]
  
    \item[analysis: ]
   \hyperref[TEI.cl]{cl} \hyperref[TEI.phr]{phr} \hyperref[TEI.s]{s} \hyperref[TEI.span]{span}\par 
    \item[core: ]
   \hyperref[TEI.abbr]{abbr} \hyperref[TEI.add]{add} \hyperref[TEI.addrLine]{addrLine} \hyperref[TEI.address]{address} \hyperref[TEI.author]{author} \hyperref[TEI.bibl]{bibl} \hyperref[TEI.biblScope]{biblScope} \hyperref[TEI.citedRange]{citedRange} \hyperref[TEI.corr]{corr} \hyperref[TEI.date]{date} \hyperref[TEI.del]{del} \hyperref[TEI.desc]{desc} \hyperref[TEI.distinct]{distinct} \hyperref[TEI.editor]{editor} \hyperref[TEI.email]{email} \hyperref[TEI.emph]{emph} \hyperref[TEI.expan]{expan} \hyperref[TEI.foreign]{foreign} \hyperref[TEI.gloss]{gloss} \hyperref[TEI.head]{head} \hyperref[TEI.headItem]{headItem} \hyperref[TEI.headLabel]{headLabel} \hyperref[TEI.hi]{hi} \hyperref[TEI.item]{item} \hyperref[TEI.l]{l} \hyperref[TEI.label]{label} \hyperref[TEI.measure]{measure} \hyperref[TEI.meeting]{meeting} \hyperref[TEI.mentioned]{mentioned} \hyperref[TEI.name]{name} \hyperref[TEI.note]{note} \hyperref[TEI.num]{num} \hyperref[TEI.orig]{orig} \hyperref[TEI.p]{p} \hyperref[TEI.pubPlace]{pubPlace} \hyperref[TEI.publisher]{publisher} \hyperref[TEI.q]{q} \hyperref[TEI.quote]{quote} \hyperref[TEI.ref]{ref} \hyperref[TEI.reg]{reg} \hyperref[TEI.resp]{resp} \hyperref[TEI.rs]{rs} \hyperref[TEI.said]{said} \hyperref[TEI.sic]{sic} \hyperref[TEI.soCalled]{soCalled} \hyperref[TEI.speaker]{speaker} \hyperref[TEI.stage]{stage} \hyperref[TEI.street]{street} \hyperref[TEI.term]{term} \hyperref[TEI.textLang]{textLang} \hyperref[TEI.time]{time} \hyperref[TEI.title]{title} \hyperref[TEI.unclear]{unclear}\par 
    \item[figures: ]
   \hyperref[TEI.cell]{cell} \hyperref[TEI.figDesc]{figDesc}\par 
    \item[header: ]
   \hyperref[TEI.authority]{authority} \hyperref[TEI.change]{change} \hyperref[TEI.classCode]{classCode} \hyperref[TEI.creation]{creation} \hyperref[TEI.distributor]{distributor} \hyperref[TEI.edition]{edition} \hyperref[TEI.extent]{extent} \hyperref[TEI.funder]{funder} \hyperref[TEI.language]{language} \hyperref[TEI.licence]{licence} \hyperref[TEI.rendition]{rendition}\par 
    \item[iso-fs: ]
   \hyperref[TEI.fDescr]{fDescr} \hyperref[TEI.fsDescr]{fsDescr}\par 
    \item[linking: ]
   \hyperref[TEI.ab]{ab} \hyperref[TEI.seg]{seg}\par 
    \item[msdescription: ]
   \hyperref[TEI.accMat]{accMat} \hyperref[TEI.acquisition]{acquisition} \hyperref[TEI.additions]{additions} \hyperref[TEI.catchwords]{catchwords} \hyperref[TEI.collation]{collation} \hyperref[TEI.colophon]{colophon} \hyperref[TEI.condition]{condition} \hyperref[TEI.custEvent]{custEvent} \hyperref[TEI.decoNote]{decoNote} \hyperref[TEI.explicit]{explicit} \hyperref[TEI.filiation]{filiation} \hyperref[TEI.finalRubric]{finalRubric} \hyperref[TEI.foliation]{foliation} \hyperref[TEI.heraldry]{heraldry} \hyperref[TEI.incipit]{incipit} \hyperref[TEI.layout]{layout} \hyperref[TEI.material]{material} \hyperref[TEI.musicNotation]{musicNotation} \hyperref[TEI.objectType]{objectType} \hyperref[TEI.origDate]{origDate} \hyperref[TEI.origPlace]{origPlace} \hyperref[TEI.origin]{origin} \hyperref[TEI.provenance]{provenance} \hyperref[TEI.rubric]{rubric} \hyperref[TEI.secFol]{secFol} \hyperref[TEI.signatures]{signatures} \hyperref[TEI.source]{source} \hyperref[TEI.stamp]{stamp} \hyperref[TEI.summary]{summary} \hyperref[TEI.support]{support} \hyperref[TEI.surrogates]{surrogates} \hyperref[TEI.typeNote]{typeNote} \hyperref[TEI.watermark]{watermark}\par 
    \item[namesdates: ]
   \hyperref[TEI.addName]{addName} \hyperref[TEI.affiliation]{affiliation} \hyperref[TEI.country]{country} \hyperref[TEI.forename]{forename} \hyperref[TEI.genName]{genName} \hyperref[TEI.geogName]{geogName} \hyperref[TEI.nameLink]{nameLink} \hyperref[TEI.org]{org} \hyperref[TEI.orgName]{orgName} \hyperref[TEI.persName]{persName} \hyperref[TEI.placeName]{placeName} \hyperref[TEI.region]{region} \hyperref[TEI.roleName]{roleName} \hyperref[TEI.settlement]{settlement} \hyperref[TEI.surname]{surname}\par 
    \item[spoken: ]
   \hyperref[TEI.annotationBlock]{annotationBlock}\par 
    \item[standOff: ]
   \hyperref[TEI.listAnnotation]{listAnnotation}\par 
    \item[textstructure: ]
   \hyperref[TEI.docAuthor]{docAuthor} \hyperref[TEI.docDate]{docDate} \hyperref[TEI.docEdition]{docEdition} \hyperref[TEI.titlePart]{titlePart}\par 
    \item[transcr: ]
   \hyperref[TEI.damage]{damage} \hyperref[TEI.fw]{fw} \hyperref[TEI.metamark]{metamark} \hyperref[TEI.mod]{mod} \hyperref[TEI.restore]{restore} \hyperref[TEI.retrace]{retrace} \hyperref[TEI.secl]{secl} \hyperref[TEI.supplied]{supplied} \hyperref[TEI.surplus]{surplus}
    \item[{Peut contenir}]
  
    \item[analysis: ]
   \hyperref[TEI.c]{c} \hyperref[TEI.cl]{cl} \hyperref[TEI.interp]{interp} \hyperref[TEI.interpGrp]{interpGrp} \hyperref[TEI.m]{m} \hyperref[TEI.pc]{pc} \hyperref[TEI.phr]{phr} \hyperref[TEI.s]{s} \hyperref[TEI.span]{span} \hyperref[TEI.spanGrp]{spanGrp} \hyperref[TEI.w]{w}\par 
    \item[core: ]
   \hyperref[TEI.abbr]{abbr} \hyperref[TEI.add]{add} \hyperref[TEI.address]{address} \hyperref[TEI.binaryObject]{binaryObject} \hyperref[TEI.cb]{cb} \hyperref[TEI.choice]{choice} \hyperref[TEI.corr]{corr} \hyperref[TEI.date]{date} \hyperref[TEI.del]{del} \hyperref[TEI.distinct]{distinct} \hyperref[TEI.email]{email} \hyperref[TEI.emph]{emph} \hyperref[TEI.expan]{expan} \hyperref[TEI.foreign]{foreign} \hyperref[TEI.gap]{gap} \hyperref[TEI.gb]{gb} \hyperref[TEI.gloss]{gloss} \hyperref[TEI.graphic]{graphic} \hyperref[TEI.hi]{hi} \hyperref[TEI.index]{index} \hyperref[TEI.lb]{lb} \hyperref[TEI.measure]{measure} \hyperref[TEI.measureGrp]{measureGrp} \hyperref[TEI.media]{media} \hyperref[TEI.mentioned]{mentioned} \hyperref[TEI.milestone]{milestone} \hyperref[TEI.name]{name} \hyperref[TEI.note]{note} \hyperref[TEI.num]{num} \hyperref[TEI.orig]{orig} \hyperref[TEI.pb]{pb} \hyperref[TEI.ptr]{ptr} \hyperref[TEI.ref]{ref} \hyperref[TEI.reg]{reg} \hyperref[TEI.rs]{rs} \hyperref[TEI.sic]{sic} \hyperref[TEI.soCalled]{soCalled} \hyperref[TEI.term]{term} \hyperref[TEI.time]{time} \hyperref[TEI.title]{title} \hyperref[TEI.unclear]{unclear}\par 
    \item[derived-module-tei.istex: ]
   \hyperref[TEI.math]{math} \hyperref[TEI.mrow]{mrow}\par 
    \item[figures: ]
   \hyperref[TEI.figure]{figure} \hyperref[TEI.formula]{formula} \hyperref[TEI.notatedMusic]{notatedMusic}\par 
    \item[header: ]
   \hyperref[TEI.idno]{idno}\par 
    \item[iso-fs: ]
   \hyperref[TEI.fLib]{fLib} \hyperref[TEI.fs]{fs} \hyperref[TEI.fvLib]{fvLib}\par 
    \item[linking: ]
   \hyperref[TEI.alt]{alt} \hyperref[TEI.altGrp]{altGrp} \hyperref[TEI.anchor]{anchor} \hyperref[TEI.join]{join} \hyperref[TEI.joinGrp]{joinGrp} \hyperref[TEI.link]{link} \hyperref[TEI.linkGrp]{linkGrp} \hyperref[TEI.seg]{seg} \hyperref[TEI.timeline]{timeline}\par 
    \item[msdescription: ]
   \hyperref[TEI.catchwords]{catchwords} \hyperref[TEI.depth]{depth} \hyperref[TEI.dim]{dim} \hyperref[TEI.dimensions]{dimensions} \hyperref[TEI.height]{height} \hyperref[TEI.heraldry]{heraldry} \hyperref[TEI.locus]{locus} \hyperref[TEI.locusGrp]{locusGrp} \hyperref[TEI.material]{material} \hyperref[TEI.objectType]{objectType} \hyperref[TEI.origDate]{origDate} \hyperref[TEI.origPlace]{origPlace} \hyperref[TEI.secFol]{secFol} \hyperref[TEI.signatures]{signatures} \hyperref[TEI.source]{source} \hyperref[TEI.stamp]{stamp} \hyperref[TEI.watermark]{watermark} \hyperref[TEI.width]{width}\par 
    \item[namesdates: ]
   \hyperref[TEI.addName]{addName} \hyperref[TEI.affiliation]{affiliation} \hyperref[TEI.country]{country} \hyperref[TEI.forename]{forename} \hyperref[TEI.genName]{genName} \hyperref[TEI.geogName]{geogName} \hyperref[TEI.location]{location} \hyperref[TEI.nameLink]{nameLink} \hyperref[TEI.orgName]{orgName} \hyperref[TEI.persName]{persName} \hyperref[TEI.placeName]{placeName} \hyperref[TEI.region]{region} \hyperref[TEI.roleName]{roleName} \hyperref[TEI.settlement]{settlement} \hyperref[TEI.state]{state} \hyperref[TEI.surname]{surname}\par 
    \item[spoken: ]
   \hyperref[TEI.annotationBlock]{annotationBlock}\par 
    \item[transcr: ]
   \hyperref[TEI.addSpan]{addSpan} \hyperref[TEI.am]{am} \hyperref[TEI.damage]{damage} \hyperref[TEI.damageSpan]{damageSpan} \hyperref[TEI.delSpan]{delSpan} \hyperref[TEI.ex]{ex} \hyperref[TEI.fw]{fw} \hyperref[TEI.handShift]{handShift} \hyperref[TEI.listTranspose]{listTranspose} \hyperref[TEI.metamark]{metamark} \hyperref[TEI.mod]{mod} \hyperref[TEI.redo]{redo} \hyperref[TEI.restore]{restore} \hyperref[TEI.retrace]{retrace} \hyperref[TEI.secl]{secl} \hyperref[TEI.space]{space} \hyperref[TEI.subst]{subst} \hyperref[TEI.substJoin]{substJoin} \hyperref[TEI.supplied]{supplied} \hyperref[TEI.surplus]{surplus} \hyperref[TEI.undo]{undo}\par des données textuelles
    \item[{Exemple}]
  \leavevmode\bgroup\exampleFont \begin{shaded}\noindent\mbox{}{<\textbf{persName}>}\mbox{}\newline 
\hspace*{6pt}{<\textbf{forename}>}Louis{</\textbf{forename}>}\mbox{}\newline 
\hspace*{6pt}{<\textbf{genName}>}XIV{</\textbf{genName}>}\mbox{}\newline 
{</\textbf{persName}>}\end{shaded}\egroup 


    \item[{Exemple}]
  \leavevmode\bgroup\exampleFont \begin{shaded}\noindent\mbox{}{<\textbf{persName}>}\mbox{}\newline 
\hspace*{6pt}{<\textbf{surname}>}Louis X{</\textbf{surname}>}\mbox{}\newline 
\hspace*{6pt}{<\textbf{genName}\hspace*{6pt}{type}="{epithet}">}Le Hutin{</\textbf{genName}>}\mbox{}\newline 
{</\textbf{persName}>}\end{shaded}\egroup 


    \item[{Modèle de contenu}]
  \mbox{}\hfill\\[-10pt]\begin{Verbatim}[fontsize=\small]
<content>
 <macroRef key="macro.phraseSeq"/>
</content>
    
\end{Verbatim}

    \item[{Schéma Declaration}]
  \mbox{}\hfill\\[-10pt]\begin{Verbatim}[fontsize=\small]
element genName
{
   tei_att.global.attributes,
   tei_att.personal.attributes,
   tei_att.typed.attributes,
   tei_macro.phraseSeq}
\end{Verbatim}

\end{reflist}  \index{geogName=<geogName>|oddindex}\index{scheme=@scheme!<geogName>|oddindex}
\begin{reflist}
\item[]\begin{specHead}{TEI.geogName}{<geogName> }(nom de lieu géographique) un nom associé à une caractéristique géographique comme Windrush Valley ou le Mont Sinaï. [\xref{http://www.tei-c.org/release/doc/tei-p5-doc/en/html/ND.html\#NDPLAC}{13.2.3. Place Names}]\end{specHead} 
    \item[{Module}]
  namesdates
    \item[{Attributs}]
  Attributs \hyperref[TEI.att.datable]{att.datable} (\textit{@calendar}, \textit{@period})  (\hyperref[TEI.att.datable.w3c]{att.datable.w3c} (\textit{@when}, \textit{@notBefore}, \textit{@notAfter}, \textit{@from}, \textit{@to})) (\hyperref[TEI.att.datable.iso]{att.datable.iso} (\textit{@when-iso}, \textit{@notBefore-iso}, \textit{@notAfter-iso}, \textit{@from-iso}, \textit{@to-iso})) (\hyperref[TEI.att.datable.custom]{att.datable.custom} (\textit{@when-custom}, \textit{@notBefore-custom}, \textit{@notAfter-custom}, \textit{@from-custom}, \textit{@to-custom}, \textit{@datingPoint}, \textit{@datingMethod})) \hyperref[TEI.att.editLike]{att.editLike} (\textit{@evidence}, \textit{@instant})  (\hyperref[TEI.att.dimensions]{att.dimensions} (\textit{@unit}, \textit{@quantity}, \textit{@extent}, \textit{@precision}, \textit{@scope}) (\hyperref[TEI.att.ranging]{att.ranging} (\textit{@atLeast}, \textit{@atMost}, \textit{@min}, \textit{@max}, \textit{@confidence})) ) \hyperref[TEI.att.global]{att.global} (\textit{@xml:id}, \textit{@n}, \textit{@xml:lang}, \textit{@xml:base}, \textit{@xml:space})  (\hyperref[TEI.att.global.rendition]{att.global.rendition} (\textit{@rend}, \textit{@style}, \textit{@rendition})) (\hyperref[TEI.att.global.linking]{att.global.linking} (\textit{@corresp}, \textit{@synch}, \textit{@sameAs}, \textit{@copyOf}, \textit{@next}, \textit{@prev}, \textit{@exclude}, \textit{@select})) (\hyperref[TEI.att.global.analytic]{att.global.analytic} (\textit{@ana})) (\hyperref[TEI.att.global.facs]{att.global.facs} (\textit{@facs})) (\hyperref[TEI.att.global.change]{att.global.change} (\textit{@change})) (\hyperref[TEI.att.global.responsibility]{att.global.responsibility} (\textit{@cert}, \textit{@resp})) (\hyperref[TEI.att.global.source]{att.global.source} (\textit{@source})) \hyperref[TEI.att.naming]{att.naming} (\textit{@role}, \textit{@nymRef})  (\hyperref[TEI.att.canonical]{att.canonical} (\textit{@key}, \textit{@ref})) \hyperref[TEI.att.typed]{att.typed} (\textit{@type}, \textit{@subtype}) \hfil\\[-10pt]\begin{sansreflist}
    \item[@scheme]
  désigne la liste des ontologies dans lequel l'ensemble des termes concernés sont définis.
\begin{reflist}
    \item[{Statut}]
  Optionel
    \item[{Type de données}]
  \hyperref[TEI.teidata.pointer]{teidata.pointer}
\end{reflist}  
\end{sansreflist}  
    \item[{Membre du}]
  \hyperref[TEI.model.placeNamePart]{model.placeNamePart}
    \item[{Contenu dans}]
  
    \item[analysis: ]
   \hyperref[TEI.cl]{cl} \hyperref[TEI.phr]{phr} \hyperref[TEI.s]{s} \hyperref[TEI.span]{span}\par 
    \item[core: ]
   \hyperref[TEI.abbr]{abbr} \hyperref[TEI.add]{add} \hyperref[TEI.addrLine]{addrLine} \hyperref[TEI.address]{address} \hyperref[TEI.author]{author} \hyperref[TEI.bibl]{bibl} \hyperref[TEI.biblScope]{biblScope} \hyperref[TEI.citedRange]{citedRange} \hyperref[TEI.corr]{corr} \hyperref[TEI.date]{date} \hyperref[TEI.del]{del} \hyperref[TEI.desc]{desc} \hyperref[TEI.distinct]{distinct} \hyperref[TEI.editor]{editor} \hyperref[TEI.email]{email} \hyperref[TEI.emph]{emph} \hyperref[TEI.expan]{expan} \hyperref[TEI.foreign]{foreign} \hyperref[TEI.gloss]{gloss} \hyperref[TEI.head]{head} \hyperref[TEI.headItem]{headItem} \hyperref[TEI.headLabel]{headLabel} \hyperref[TEI.hi]{hi} \hyperref[TEI.item]{item} \hyperref[TEI.l]{l} \hyperref[TEI.label]{label} \hyperref[TEI.measure]{measure} \hyperref[TEI.meeting]{meeting} \hyperref[TEI.mentioned]{mentioned} \hyperref[TEI.name]{name} \hyperref[TEI.note]{note} \hyperref[TEI.num]{num} \hyperref[TEI.orig]{orig} \hyperref[TEI.p]{p} \hyperref[TEI.pubPlace]{pubPlace} \hyperref[TEI.publisher]{publisher} \hyperref[TEI.q]{q} \hyperref[TEI.quote]{quote} \hyperref[TEI.ref]{ref} \hyperref[TEI.reg]{reg} \hyperref[TEI.resp]{resp} \hyperref[TEI.rs]{rs} \hyperref[TEI.said]{said} \hyperref[TEI.sic]{sic} \hyperref[TEI.soCalled]{soCalled} \hyperref[TEI.speaker]{speaker} \hyperref[TEI.stage]{stage} \hyperref[TEI.street]{street} \hyperref[TEI.term]{term} \hyperref[TEI.textLang]{textLang} \hyperref[TEI.time]{time} \hyperref[TEI.title]{title} \hyperref[TEI.unclear]{unclear}\par 
    \item[figures: ]
   \hyperref[TEI.cell]{cell} \hyperref[TEI.figDesc]{figDesc}\par 
    \item[header: ]
   \hyperref[TEI.authority]{authority} \hyperref[TEI.change]{change} \hyperref[TEI.classCode]{classCode} \hyperref[TEI.creation]{creation} \hyperref[TEI.distributor]{distributor} \hyperref[TEI.edition]{edition} \hyperref[TEI.extent]{extent} \hyperref[TEI.funder]{funder} \hyperref[TEI.language]{language} \hyperref[TEI.licence]{licence} \hyperref[TEI.rendition]{rendition}\par 
    \item[iso-fs: ]
   \hyperref[TEI.fDescr]{fDescr} \hyperref[TEI.fsDescr]{fsDescr}\par 
    \item[linking: ]
   \hyperref[TEI.ab]{ab} \hyperref[TEI.seg]{seg}\par 
    \item[msdescription: ]
   \hyperref[TEI.accMat]{accMat} \hyperref[TEI.acquisition]{acquisition} \hyperref[TEI.additions]{additions} \hyperref[TEI.altIdentifier]{altIdentifier} \hyperref[TEI.catchwords]{catchwords} \hyperref[TEI.collation]{collation} \hyperref[TEI.colophon]{colophon} \hyperref[TEI.condition]{condition} \hyperref[TEI.custEvent]{custEvent} \hyperref[TEI.decoNote]{decoNote} \hyperref[TEI.explicit]{explicit} \hyperref[TEI.filiation]{filiation} \hyperref[TEI.finalRubric]{finalRubric} \hyperref[TEI.foliation]{foliation} \hyperref[TEI.heraldry]{heraldry} \hyperref[TEI.incipit]{incipit} \hyperref[TEI.layout]{layout} \hyperref[TEI.material]{material} \hyperref[TEI.msIdentifier]{msIdentifier} \hyperref[TEI.musicNotation]{musicNotation} \hyperref[TEI.objectType]{objectType} \hyperref[TEI.origDate]{origDate} \hyperref[TEI.origPlace]{origPlace} \hyperref[TEI.origin]{origin} \hyperref[TEI.provenance]{provenance} \hyperref[TEI.rubric]{rubric} \hyperref[TEI.secFol]{secFol} \hyperref[TEI.signatures]{signatures} \hyperref[TEI.source]{source} \hyperref[TEI.stamp]{stamp} \hyperref[TEI.summary]{summary} \hyperref[TEI.support]{support} \hyperref[TEI.surrogates]{surrogates} \hyperref[TEI.typeNote]{typeNote} \hyperref[TEI.watermark]{watermark}\par 
    \item[namesdates: ]
   \hyperref[TEI.addName]{addName} \hyperref[TEI.affiliation]{affiliation} \hyperref[TEI.country]{country} \hyperref[TEI.forename]{forename} \hyperref[TEI.genName]{genName} \hyperref[TEI.geogName]{geogName} \hyperref[TEI.location]{location} \hyperref[TEI.nameLink]{nameLink} \hyperref[TEI.org]{org} \hyperref[TEI.orgName]{orgName} \hyperref[TEI.persName]{persName} \hyperref[TEI.place]{place} \hyperref[TEI.placeName]{placeName} \hyperref[TEI.region]{region} \hyperref[TEI.roleName]{roleName} \hyperref[TEI.settlement]{settlement} \hyperref[TEI.surname]{surname}\par 
    \item[spoken: ]
   \hyperref[TEI.annotationBlock]{annotationBlock}\par 
    \item[standOff: ]
   \hyperref[TEI.listAnnotation]{listAnnotation}\par 
    \item[textstructure: ]
   \hyperref[TEI.docAuthor]{docAuthor} \hyperref[TEI.docDate]{docDate} \hyperref[TEI.docEdition]{docEdition} \hyperref[TEI.titlePart]{titlePart}\par 
    \item[transcr: ]
   \hyperref[TEI.damage]{damage} \hyperref[TEI.fw]{fw} \hyperref[TEI.metamark]{metamark} \hyperref[TEI.mod]{mod} \hyperref[TEI.restore]{restore} \hyperref[TEI.retrace]{retrace} \hyperref[TEI.secl]{secl} \hyperref[TEI.supplied]{supplied} \hyperref[TEI.surplus]{surplus}
    \item[{Peut contenir}]
  
    \item[analysis: ]
   \hyperref[TEI.c]{c} \hyperref[TEI.cl]{cl} \hyperref[TEI.interp]{interp} \hyperref[TEI.interpGrp]{interpGrp} \hyperref[TEI.m]{m} \hyperref[TEI.pc]{pc} \hyperref[TEI.phr]{phr} \hyperref[TEI.s]{s} \hyperref[TEI.span]{span} \hyperref[TEI.spanGrp]{spanGrp} \hyperref[TEI.w]{w}\par 
    \item[core: ]
   \hyperref[TEI.abbr]{abbr} \hyperref[TEI.add]{add} \hyperref[TEI.address]{address} \hyperref[TEI.binaryObject]{binaryObject} \hyperref[TEI.cb]{cb} \hyperref[TEI.choice]{choice} \hyperref[TEI.corr]{corr} \hyperref[TEI.date]{date} \hyperref[TEI.del]{del} \hyperref[TEI.distinct]{distinct} \hyperref[TEI.email]{email} \hyperref[TEI.emph]{emph} \hyperref[TEI.expan]{expan} \hyperref[TEI.foreign]{foreign} \hyperref[TEI.gap]{gap} \hyperref[TEI.gb]{gb} \hyperref[TEI.gloss]{gloss} \hyperref[TEI.graphic]{graphic} \hyperref[TEI.hi]{hi} \hyperref[TEI.index]{index} \hyperref[TEI.lb]{lb} \hyperref[TEI.measure]{measure} \hyperref[TEI.measureGrp]{measureGrp} \hyperref[TEI.media]{media} \hyperref[TEI.mentioned]{mentioned} \hyperref[TEI.milestone]{milestone} \hyperref[TEI.name]{name} \hyperref[TEI.note]{note} \hyperref[TEI.num]{num} \hyperref[TEI.orig]{orig} \hyperref[TEI.pb]{pb} \hyperref[TEI.ptr]{ptr} \hyperref[TEI.ref]{ref} \hyperref[TEI.reg]{reg} \hyperref[TEI.rs]{rs} \hyperref[TEI.sic]{sic} \hyperref[TEI.soCalled]{soCalled} \hyperref[TEI.term]{term} \hyperref[TEI.time]{time} \hyperref[TEI.title]{title} \hyperref[TEI.unclear]{unclear}\par 
    \item[derived-module-tei.istex: ]
   \hyperref[TEI.math]{math} \hyperref[TEI.mrow]{mrow}\par 
    \item[figures: ]
   \hyperref[TEI.figure]{figure} \hyperref[TEI.formula]{formula} \hyperref[TEI.notatedMusic]{notatedMusic}\par 
    \item[header: ]
   \hyperref[TEI.idno]{idno}\par 
    \item[iso-fs: ]
   \hyperref[TEI.fLib]{fLib} \hyperref[TEI.fs]{fs} \hyperref[TEI.fvLib]{fvLib}\par 
    \item[linking: ]
   \hyperref[TEI.alt]{alt} \hyperref[TEI.altGrp]{altGrp} \hyperref[TEI.anchor]{anchor} \hyperref[TEI.join]{join} \hyperref[TEI.joinGrp]{joinGrp} \hyperref[TEI.link]{link} \hyperref[TEI.linkGrp]{linkGrp} \hyperref[TEI.seg]{seg} \hyperref[TEI.timeline]{timeline}\par 
    \item[msdescription: ]
   \hyperref[TEI.catchwords]{catchwords} \hyperref[TEI.depth]{depth} \hyperref[TEI.dim]{dim} \hyperref[TEI.dimensions]{dimensions} \hyperref[TEI.height]{height} \hyperref[TEI.heraldry]{heraldry} \hyperref[TEI.locus]{locus} \hyperref[TEI.locusGrp]{locusGrp} \hyperref[TEI.material]{material} \hyperref[TEI.objectType]{objectType} \hyperref[TEI.origDate]{origDate} \hyperref[TEI.origPlace]{origPlace} \hyperref[TEI.secFol]{secFol} \hyperref[TEI.signatures]{signatures} \hyperref[TEI.source]{source} \hyperref[TEI.stamp]{stamp} \hyperref[TEI.watermark]{watermark} \hyperref[TEI.width]{width}\par 
    \item[namesdates: ]
   \hyperref[TEI.addName]{addName} \hyperref[TEI.affiliation]{affiliation} \hyperref[TEI.country]{country} \hyperref[TEI.forename]{forename} \hyperref[TEI.genName]{genName} \hyperref[TEI.geogName]{geogName} \hyperref[TEI.location]{location} \hyperref[TEI.nameLink]{nameLink} \hyperref[TEI.orgName]{orgName} \hyperref[TEI.persName]{persName} \hyperref[TEI.placeName]{placeName} \hyperref[TEI.region]{region} \hyperref[TEI.roleName]{roleName} \hyperref[TEI.settlement]{settlement} \hyperref[TEI.state]{state} \hyperref[TEI.surname]{surname}\par 
    \item[spoken: ]
   \hyperref[TEI.annotationBlock]{annotationBlock}\par 
    \item[transcr: ]
   \hyperref[TEI.addSpan]{addSpan} \hyperref[TEI.am]{am} \hyperref[TEI.damage]{damage} \hyperref[TEI.damageSpan]{damageSpan} \hyperref[TEI.delSpan]{delSpan} \hyperref[TEI.ex]{ex} \hyperref[TEI.fw]{fw} \hyperref[TEI.handShift]{handShift} \hyperref[TEI.listTranspose]{listTranspose} \hyperref[TEI.metamark]{metamark} \hyperref[TEI.mod]{mod} \hyperref[TEI.redo]{redo} \hyperref[TEI.restore]{restore} \hyperref[TEI.retrace]{retrace} \hyperref[TEI.secl]{secl} \hyperref[TEI.space]{space} \hyperref[TEI.subst]{subst} \hyperref[TEI.substJoin]{substJoin} \hyperref[TEI.supplied]{supplied} \hyperref[TEI.surplus]{surplus} \hyperref[TEI.undo]{undo}\par des données textuelles
    \item[{Exemple}]
  StandOff enrichissement entité nommée geogName\leavevmode\bgroup\exampleFont \begin{shaded}\noindent\mbox{}{<\textbf{annotationBlock}\hspace*{6pt}{corresp}="{text}">}\mbox{}\newline 
\hspace*{6pt}{<\textbf{geogName}\hspace*{6pt}{change}="{\#Unitex-3.2.0-alpha}"\mbox{}\newline 
\hspace*{6pt}\hspace*{6pt}{resp}="{istex}"\mbox{}\newline 
\hspace*{6pt}\hspace*{6pt}{scheme}="{https://geogname-entity.data.istex.fr}">}\mbox{}\newline 
\hspace*{6pt}\hspace*{6pt}{<\textbf{term}>}Mississippi River{</\textbf{term}>}\mbox{}\newline 
\hspace*{6pt}\hspace*{6pt}{<\textbf{fs}\hspace*{6pt}{type}="{statistics}">}\mbox{}\newline 
\hspace*{6pt}\hspace*{6pt}\hspace*{6pt}{<\textbf{f}\hspace*{6pt}{name}="{frequency}">}\mbox{}\newline 
\hspace*{6pt}\hspace*{6pt}\hspace*{6pt}\hspace*{6pt}{<\textbf{numeric}\hspace*{6pt}{value}="{2}"/>}\mbox{}\newline 
\hspace*{6pt}\hspace*{6pt}\hspace*{6pt}{</\textbf{f}>}\mbox{}\newline 
\hspace*{6pt}\hspace*{6pt}{</\textbf{fs}>}\mbox{}\newline 
\hspace*{6pt}{</\textbf{geogName}>}\mbox{}\newline 
{</\textbf{annotationBlock}>}\end{shaded}\egroup 


    \item[{Modèle de contenu}]
  \mbox{}\hfill\\[-10pt]\begin{Verbatim}[fontsize=\small]
<content>
 <classRef key="model.placeNamePart"/>
 <macroRef key="macro.phraseSeq"/>
</content>
    
\end{Verbatim}

    \item[{Schéma Declaration}]
  \mbox{}\hfill\\[-10pt]\begin{Verbatim}[fontsize=\small]
element geogName
{
   tei_att.datable.attributes,
   tei_att.editLike.attributes,
   tei_att.global.attributes,
   tei_att.naming.attributes,
   tei_att.typed.attributes,
   attribute scheme { text }?,
   tei_model.placeNamePart,
   tei_macro.phraseSeq}
\end{Verbatim}

\end{reflist}  \index{gloss=<gloss>|oddindex}
\begin{reflist}
\item[]\begin{specHead}{TEI.gloss}{<gloss> }(glose) identifie une expression ou un mot utilisé pour fournir une glose ou une définition à quelque autre mot ou expression. [\xref{http://www.tei-c.org/release/doc/tei-p5-doc/en/html/CO.html\#COHQU}{3.3.4. Terms, Glosses, Equivalents, and Descriptions} \xref{http://www.tei-c.org/release/doc/tei-p5-doc/en/html/TD.html\#TDcrystalsCEdc}{22.4.1. Description of Components}]\end{specHead} 
    \item[{Module}]
  core
    \item[{Attributs}]
  Attributs \hyperref[TEI.att.global]{att.global} (\textit{@xml:id}, \textit{@n}, \textit{@xml:lang}, \textit{@xml:base}, \textit{@xml:space})  (\hyperref[TEI.att.global.rendition]{att.global.rendition} (\textit{@rend}, \textit{@style}, \textit{@rendition})) (\hyperref[TEI.att.global.linking]{att.global.linking} (\textit{@corresp}, \textit{@synch}, \textit{@sameAs}, \textit{@copyOf}, \textit{@next}, \textit{@prev}, \textit{@exclude}, \textit{@select})) (\hyperref[TEI.att.global.analytic]{att.global.analytic} (\textit{@ana})) (\hyperref[TEI.att.global.facs]{att.global.facs} (\textit{@facs})) (\hyperref[TEI.att.global.change]{att.global.change} (\textit{@change})) (\hyperref[TEI.att.global.responsibility]{att.global.responsibility} (\textit{@cert}, \textit{@resp})) (\hyperref[TEI.att.global.source]{att.global.source} (\textit{@source})) \hyperref[TEI.att.declaring]{att.declaring} (\textit{@decls}) \hyperref[TEI.att.translatable]{att.translatable} (\textit{@versionDate}) \hyperref[TEI.att.typed]{att.typed} (\textit{@type}, \textit{@subtype}) \hyperref[TEI.att.pointing]{att.pointing} (\textit{@targetLang}, \textit{@target}, \textit{@evaluate}) \hyperref[TEI.att.cReferencing]{att.cReferencing} (\textit{@cRef}) 
    \item[{Membre du}]
  \hyperref[TEI.model.emphLike]{model.emphLike} \hyperref[TEI.model.glossLike]{model.glossLike}
    \item[{Contenu dans}]
  
    \item[analysis: ]
   \hyperref[TEI.cl]{cl} \hyperref[TEI.phr]{phr} \hyperref[TEI.s]{s} \hyperref[TEI.span]{span}\par 
    \item[core: ]
   \hyperref[TEI.abbr]{abbr} \hyperref[TEI.add]{add} \hyperref[TEI.addrLine]{addrLine} \hyperref[TEI.author]{author} \hyperref[TEI.bibl]{bibl} \hyperref[TEI.biblScope]{biblScope} \hyperref[TEI.citedRange]{citedRange} \hyperref[TEI.corr]{corr} \hyperref[TEI.date]{date} \hyperref[TEI.del]{del} \hyperref[TEI.desc]{desc} \hyperref[TEI.distinct]{distinct} \hyperref[TEI.editor]{editor} \hyperref[TEI.email]{email} \hyperref[TEI.emph]{emph} \hyperref[TEI.expan]{expan} \hyperref[TEI.foreign]{foreign} \hyperref[TEI.gloss]{gloss} \hyperref[TEI.head]{head} \hyperref[TEI.headItem]{headItem} \hyperref[TEI.headLabel]{headLabel} \hyperref[TEI.hi]{hi} \hyperref[TEI.item]{item} \hyperref[TEI.l]{l} \hyperref[TEI.label]{label} \hyperref[TEI.measure]{measure} \hyperref[TEI.meeting]{meeting} \hyperref[TEI.mentioned]{mentioned} \hyperref[TEI.name]{name} \hyperref[TEI.note]{note} \hyperref[TEI.num]{num} \hyperref[TEI.orig]{orig} \hyperref[TEI.p]{p} \hyperref[TEI.pubPlace]{pubPlace} \hyperref[TEI.publisher]{publisher} \hyperref[TEI.q]{q} \hyperref[TEI.quote]{quote} \hyperref[TEI.ref]{ref} \hyperref[TEI.reg]{reg} \hyperref[TEI.resp]{resp} \hyperref[TEI.rs]{rs} \hyperref[TEI.said]{said} \hyperref[TEI.sic]{sic} \hyperref[TEI.soCalled]{soCalled} \hyperref[TEI.speaker]{speaker} \hyperref[TEI.stage]{stage} \hyperref[TEI.street]{street} \hyperref[TEI.term]{term} \hyperref[TEI.textLang]{textLang} \hyperref[TEI.time]{time} \hyperref[TEI.title]{title} \hyperref[TEI.unclear]{unclear}\par 
    \item[figures: ]
   \hyperref[TEI.cell]{cell} \hyperref[TEI.figDesc]{figDesc}\par 
    \item[header: ]
   \hyperref[TEI.authority]{authority} \hyperref[TEI.category]{category} \hyperref[TEI.change]{change} \hyperref[TEI.classCode]{classCode} \hyperref[TEI.creation]{creation} \hyperref[TEI.distributor]{distributor} \hyperref[TEI.edition]{edition} \hyperref[TEI.extent]{extent} \hyperref[TEI.funder]{funder} \hyperref[TEI.language]{language} \hyperref[TEI.licence]{licence} \hyperref[TEI.rendition]{rendition} \hyperref[TEI.taxonomy]{taxonomy}\par 
    \item[iso-fs: ]
   \hyperref[TEI.fDescr]{fDescr} \hyperref[TEI.fsDescr]{fsDescr}\par 
    \item[linking: ]
   \hyperref[TEI.ab]{ab} \hyperref[TEI.joinGrp]{joinGrp} \hyperref[TEI.seg]{seg}\par 
    \item[msdescription: ]
   \hyperref[TEI.accMat]{accMat} \hyperref[TEI.acquisition]{acquisition} \hyperref[TEI.additions]{additions} \hyperref[TEI.catchwords]{catchwords} \hyperref[TEI.collation]{collation} \hyperref[TEI.colophon]{colophon} \hyperref[TEI.condition]{condition} \hyperref[TEI.custEvent]{custEvent} \hyperref[TEI.decoNote]{decoNote} \hyperref[TEI.explicit]{explicit} \hyperref[TEI.filiation]{filiation} \hyperref[TEI.finalRubric]{finalRubric} \hyperref[TEI.foliation]{foliation} \hyperref[TEI.heraldry]{heraldry} \hyperref[TEI.incipit]{incipit} \hyperref[TEI.layout]{layout} \hyperref[TEI.material]{material} \hyperref[TEI.musicNotation]{musicNotation} \hyperref[TEI.objectType]{objectType} \hyperref[TEI.origDate]{origDate} \hyperref[TEI.origPlace]{origPlace} \hyperref[TEI.origin]{origin} \hyperref[TEI.provenance]{provenance} \hyperref[TEI.rubric]{rubric} \hyperref[TEI.secFol]{secFol} \hyperref[TEI.signatures]{signatures} \hyperref[TEI.source]{source} \hyperref[TEI.stamp]{stamp} \hyperref[TEI.summary]{summary} \hyperref[TEI.support]{support} \hyperref[TEI.surrogates]{surrogates} \hyperref[TEI.typeNote]{typeNote} \hyperref[TEI.watermark]{watermark}\par 
    \item[namesdates: ]
   \hyperref[TEI.addName]{addName} \hyperref[TEI.affiliation]{affiliation} \hyperref[TEI.country]{country} \hyperref[TEI.forename]{forename} \hyperref[TEI.genName]{genName} \hyperref[TEI.geogName]{geogName} \hyperref[TEI.nameLink]{nameLink} \hyperref[TEI.orgName]{orgName} \hyperref[TEI.persName]{persName} \hyperref[TEI.placeName]{placeName} \hyperref[TEI.region]{region} \hyperref[TEI.roleName]{roleName} \hyperref[TEI.settlement]{settlement} \hyperref[TEI.surname]{surname}\par 
    \item[textstructure: ]
   \hyperref[TEI.docAuthor]{docAuthor} \hyperref[TEI.docDate]{docDate} \hyperref[TEI.docEdition]{docEdition} \hyperref[TEI.titlePart]{titlePart}\par 
    \item[transcr: ]
   \hyperref[TEI.damage]{damage} \hyperref[TEI.fw]{fw} \hyperref[TEI.metamark]{metamark} \hyperref[TEI.mod]{mod} \hyperref[TEI.restore]{restore} \hyperref[TEI.retrace]{retrace} \hyperref[TEI.secl]{secl} \hyperref[TEI.supplied]{supplied} \hyperref[TEI.surplus]{surplus}
    \item[{Peut contenir}]
  
    \item[analysis: ]
   \hyperref[TEI.c]{c} \hyperref[TEI.cl]{cl} \hyperref[TEI.interp]{interp} \hyperref[TEI.interpGrp]{interpGrp} \hyperref[TEI.m]{m} \hyperref[TEI.pc]{pc} \hyperref[TEI.phr]{phr} \hyperref[TEI.s]{s} \hyperref[TEI.span]{span} \hyperref[TEI.spanGrp]{spanGrp} \hyperref[TEI.w]{w}\par 
    \item[core: ]
   \hyperref[TEI.abbr]{abbr} \hyperref[TEI.add]{add} \hyperref[TEI.address]{address} \hyperref[TEI.binaryObject]{binaryObject} \hyperref[TEI.cb]{cb} \hyperref[TEI.choice]{choice} \hyperref[TEI.corr]{corr} \hyperref[TEI.date]{date} \hyperref[TEI.del]{del} \hyperref[TEI.distinct]{distinct} \hyperref[TEI.email]{email} \hyperref[TEI.emph]{emph} \hyperref[TEI.expan]{expan} \hyperref[TEI.foreign]{foreign} \hyperref[TEI.gap]{gap} \hyperref[TEI.gb]{gb} \hyperref[TEI.gloss]{gloss} \hyperref[TEI.graphic]{graphic} \hyperref[TEI.hi]{hi} \hyperref[TEI.index]{index} \hyperref[TEI.lb]{lb} \hyperref[TEI.measure]{measure} \hyperref[TEI.measureGrp]{measureGrp} \hyperref[TEI.media]{media} \hyperref[TEI.mentioned]{mentioned} \hyperref[TEI.milestone]{milestone} \hyperref[TEI.name]{name} \hyperref[TEI.note]{note} \hyperref[TEI.num]{num} \hyperref[TEI.orig]{orig} \hyperref[TEI.pb]{pb} \hyperref[TEI.ptr]{ptr} \hyperref[TEI.ref]{ref} \hyperref[TEI.reg]{reg} \hyperref[TEI.rs]{rs} \hyperref[TEI.sic]{sic} \hyperref[TEI.soCalled]{soCalled} \hyperref[TEI.term]{term} \hyperref[TEI.time]{time} \hyperref[TEI.title]{title} \hyperref[TEI.unclear]{unclear}\par 
    \item[derived-module-tei.istex: ]
   \hyperref[TEI.math]{math} \hyperref[TEI.mrow]{mrow}\par 
    \item[figures: ]
   \hyperref[TEI.figure]{figure} \hyperref[TEI.formula]{formula} \hyperref[TEI.notatedMusic]{notatedMusic}\par 
    \item[header: ]
   \hyperref[TEI.idno]{idno}\par 
    \item[iso-fs: ]
   \hyperref[TEI.fLib]{fLib} \hyperref[TEI.fs]{fs} \hyperref[TEI.fvLib]{fvLib}\par 
    \item[linking: ]
   \hyperref[TEI.alt]{alt} \hyperref[TEI.altGrp]{altGrp} \hyperref[TEI.anchor]{anchor} \hyperref[TEI.join]{join} \hyperref[TEI.joinGrp]{joinGrp} \hyperref[TEI.link]{link} \hyperref[TEI.linkGrp]{linkGrp} \hyperref[TEI.seg]{seg} \hyperref[TEI.timeline]{timeline}\par 
    \item[msdescription: ]
   \hyperref[TEI.catchwords]{catchwords} \hyperref[TEI.depth]{depth} \hyperref[TEI.dim]{dim} \hyperref[TEI.dimensions]{dimensions} \hyperref[TEI.height]{height} \hyperref[TEI.heraldry]{heraldry} \hyperref[TEI.locus]{locus} \hyperref[TEI.locusGrp]{locusGrp} \hyperref[TEI.material]{material} \hyperref[TEI.objectType]{objectType} \hyperref[TEI.origDate]{origDate} \hyperref[TEI.origPlace]{origPlace} \hyperref[TEI.secFol]{secFol} \hyperref[TEI.signatures]{signatures} \hyperref[TEI.source]{source} \hyperref[TEI.stamp]{stamp} \hyperref[TEI.watermark]{watermark} \hyperref[TEI.width]{width}\par 
    \item[namesdates: ]
   \hyperref[TEI.addName]{addName} \hyperref[TEI.affiliation]{affiliation} \hyperref[TEI.country]{country} \hyperref[TEI.forename]{forename} \hyperref[TEI.genName]{genName} \hyperref[TEI.geogName]{geogName} \hyperref[TEI.location]{location} \hyperref[TEI.nameLink]{nameLink} \hyperref[TEI.orgName]{orgName} \hyperref[TEI.persName]{persName} \hyperref[TEI.placeName]{placeName} \hyperref[TEI.region]{region} \hyperref[TEI.roleName]{roleName} \hyperref[TEI.settlement]{settlement} \hyperref[TEI.state]{state} \hyperref[TEI.surname]{surname}\par 
    \item[spoken: ]
   \hyperref[TEI.annotationBlock]{annotationBlock}\par 
    \item[transcr: ]
   \hyperref[TEI.addSpan]{addSpan} \hyperref[TEI.am]{am} \hyperref[TEI.damage]{damage} \hyperref[TEI.damageSpan]{damageSpan} \hyperref[TEI.delSpan]{delSpan} \hyperref[TEI.ex]{ex} \hyperref[TEI.fw]{fw} \hyperref[TEI.handShift]{handShift} \hyperref[TEI.listTranspose]{listTranspose} \hyperref[TEI.metamark]{metamark} \hyperref[TEI.mod]{mod} \hyperref[TEI.redo]{redo} \hyperref[TEI.restore]{restore} \hyperref[TEI.retrace]{retrace} \hyperref[TEI.secl]{secl} \hyperref[TEI.space]{space} \hyperref[TEI.subst]{subst} \hyperref[TEI.substJoin]{substJoin} \hyperref[TEI.supplied]{supplied} \hyperref[TEI.surplus]{surplus} \hyperref[TEI.undo]{undo}\par des données textuelles
    \item[{Note}]
  \par
Les attributs {\itshape target} et {\itshape cRef} sont exclusifs l'un de l'autre.
    \item[{Exemple}]
  \leavevmode\bgroup\exampleFont \begin{shaded}\noindent\mbox{} Les{<\textbf{term}>} embrayeurs{</\textbf{term}>} sont \mbox{}\newline 
{<\textbf{gloss}>}une classe de mots dont le sens varie avec la situation; ces\mbox{}\newline 
 mots, n'ayant pas de référence propre dans la langue, ne reçoivent un référent que\mbox{}\newline 
 lorsqu'ils sont inclus dans un message.{</\textbf{gloss}>}\end{shaded}\egroup 


    \item[{Modèle de contenu}]
  \mbox{}\hfill\\[-10pt]\begin{Verbatim}[fontsize=\small]
<content>
 <macroRef key="macro.phraseSeq"/>
</content>
    
\end{Verbatim}

    \item[{Schéma Declaration}]
  \mbox{}\hfill\\[-10pt]\begin{Verbatim}[fontsize=\small]
element gloss
{
   tei_att.global.attributes,
   tei_att.declaring.attributes,
   tei_att.translatable.attributes,
   tei_att.typed.attributes,
   tei_att.pointing.attributes,
   tei_att.cReferencing.attributes,
   tei_macro.phraseSeq}
\end{Verbatim}

\end{reflist}  \index{graphic=<graphic>|oddindex}
\begin{reflist}
\item[]\begin{specHead}{TEI.graphic}{<graphic> }indique l'emplacement d'une image, d'une illustration ou d'un schéma intégrés. [\xref{http://www.tei-c.org/release/doc/tei-p5-doc/en/html/CO.html\#COGR}{3.9. Graphics and Other Non-textual Components} \xref{http://www.tei-c.org/release/doc/tei-p5-doc/en/html/PH.html\#PHFAX}{11.1. Digital Facsimiles}]\end{specHead} 
    \item[{Module}]
  core
    \item[{Attributs}]
  Attributs \hyperref[TEI.att.global]{att.global} (\textit{@xml:id}, \textit{@n}, \textit{@xml:lang}, \textit{@xml:base}, \textit{@xml:space})  (\hyperref[TEI.att.global.rendition]{att.global.rendition} (\textit{@rend}, \textit{@style}, \textit{@rendition})) (\hyperref[TEI.att.global.linking]{att.global.linking} (\textit{@corresp}, \textit{@synch}, \textit{@sameAs}, \textit{@copyOf}, \textit{@next}, \textit{@prev}, \textit{@exclude}, \textit{@select})) (\hyperref[TEI.att.global.analytic]{att.global.analytic} (\textit{@ana})) (\hyperref[TEI.att.global.facs]{att.global.facs} (\textit{@facs})) (\hyperref[TEI.att.global.change]{att.global.change} (\textit{@change})) (\hyperref[TEI.att.global.responsibility]{att.global.responsibility} (\textit{@cert}, \textit{@resp})) (\hyperref[TEI.att.global.source]{att.global.source} (\textit{@source})) \hyperref[TEI.att.media]{att.media} (\textit{@width}, \textit{@height}, \textit{@scale})  (\hyperref[TEI.att.internetMedia]{att.internetMedia} (\textit{@mimeType})) \hyperref[TEI.att.resourced]{att.resourced} (\textit{@url}) \hyperref[TEI.att.declaring]{att.declaring} (\textit{@decls}) 
    \item[{Membre du}]
  \hyperref[TEI.model.graphicLike]{model.graphicLike} \hyperref[TEI.model.titlepagePart]{model.titlepagePart} 
    \item[{Contenu dans}]
  
    \item[analysis: ]
   \hyperref[TEI.cl]{cl} \hyperref[TEI.phr]{phr} \hyperref[TEI.s]{s}\par 
    \item[core: ]
   \hyperref[TEI.abbr]{abbr} \hyperref[TEI.add]{add} \hyperref[TEI.addrLine]{addrLine} \hyperref[TEI.author]{author} \hyperref[TEI.biblScope]{biblScope} \hyperref[TEI.citedRange]{citedRange} \hyperref[TEI.corr]{corr} \hyperref[TEI.date]{date} \hyperref[TEI.del]{del} \hyperref[TEI.distinct]{distinct} \hyperref[TEI.editor]{editor} \hyperref[TEI.email]{email} \hyperref[TEI.emph]{emph} \hyperref[TEI.expan]{expan} \hyperref[TEI.foreign]{foreign} \hyperref[TEI.gloss]{gloss} \hyperref[TEI.head]{head} \hyperref[TEI.headItem]{headItem} \hyperref[TEI.headLabel]{headLabel} \hyperref[TEI.hi]{hi} \hyperref[TEI.item]{item} \hyperref[TEI.l]{l} \hyperref[TEI.label]{label} \hyperref[TEI.measure]{measure} \hyperref[TEI.mentioned]{mentioned} \hyperref[TEI.name]{name} \hyperref[TEI.note]{note} \hyperref[TEI.num]{num} \hyperref[TEI.orig]{orig} \hyperref[TEI.p]{p} \hyperref[TEI.pubPlace]{pubPlace} \hyperref[TEI.publisher]{publisher} \hyperref[TEI.q]{q} \hyperref[TEI.quote]{quote} \hyperref[TEI.ref]{ref} \hyperref[TEI.reg]{reg} \hyperref[TEI.rs]{rs} \hyperref[TEI.said]{said} \hyperref[TEI.sic]{sic} \hyperref[TEI.soCalled]{soCalled} \hyperref[TEI.speaker]{speaker} \hyperref[TEI.stage]{stage} \hyperref[TEI.street]{street} \hyperref[TEI.term]{term} \hyperref[TEI.textLang]{textLang} \hyperref[TEI.time]{time} \hyperref[TEI.title]{title} \hyperref[TEI.unclear]{unclear}\par 
    \item[figures: ]
   \hyperref[TEI.cell]{cell} \hyperref[TEI.figDesc]{figDesc} \hyperref[TEI.figure]{figure} \hyperref[TEI.formula]{formula} \hyperref[TEI.notatedMusic]{notatedMusic} \hyperref[TEI.table]{table}\par 
    \item[header: ]
   \hyperref[TEI.change]{change} \hyperref[TEI.distributor]{distributor} \hyperref[TEI.edition]{edition} \hyperref[TEI.extent]{extent} \hyperref[TEI.licence]{licence}\par 
    \item[linking: ]
   \hyperref[TEI.ab]{ab} \hyperref[TEI.seg]{seg}\par 
    \item[msdescription: ]
   \hyperref[TEI.accMat]{accMat} \hyperref[TEI.acquisition]{acquisition} \hyperref[TEI.additions]{additions} \hyperref[TEI.catchwords]{catchwords} \hyperref[TEI.collation]{collation} \hyperref[TEI.colophon]{colophon} \hyperref[TEI.condition]{condition} \hyperref[TEI.custEvent]{custEvent} \hyperref[TEI.decoNote]{decoNote} \hyperref[TEI.explicit]{explicit} \hyperref[TEI.filiation]{filiation} \hyperref[TEI.finalRubric]{finalRubric} \hyperref[TEI.foliation]{foliation} \hyperref[TEI.heraldry]{heraldry} \hyperref[TEI.incipit]{incipit} \hyperref[TEI.layout]{layout} \hyperref[TEI.material]{material} \hyperref[TEI.msItem]{msItem} \hyperref[TEI.musicNotation]{musicNotation} \hyperref[TEI.objectType]{objectType} \hyperref[TEI.origDate]{origDate} \hyperref[TEI.origPlace]{origPlace} \hyperref[TEI.origin]{origin} \hyperref[TEI.provenance]{provenance} \hyperref[TEI.rubric]{rubric} \hyperref[TEI.secFol]{secFol} \hyperref[TEI.signatures]{signatures} \hyperref[TEI.source]{source} \hyperref[TEI.stamp]{stamp} \hyperref[TEI.summary]{summary} \hyperref[TEI.support]{support} \hyperref[TEI.surrogates]{surrogates} \hyperref[TEI.typeNote]{typeNote} \hyperref[TEI.watermark]{watermark}\par 
    \item[namesdates: ]
   \hyperref[TEI.addName]{addName} \hyperref[TEI.affiliation]{affiliation} \hyperref[TEI.country]{country} \hyperref[TEI.forename]{forename} \hyperref[TEI.genName]{genName} \hyperref[TEI.geogName]{geogName} \hyperref[TEI.nameLink]{nameLink} \hyperref[TEI.orgName]{orgName} \hyperref[TEI.persName]{persName} \hyperref[TEI.placeName]{placeName} \hyperref[TEI.region]{region} \hyperref[TEI.roleName]{roleName} \hyperref[TEI.settlement]{settlement} \hyperref[TEI.surname]{surname}\par 
    \item[textstructure: ]
   \hyperref[TEI.docAuthor]{docAuthor} \hyperref[TEI.docDate]{docDate} \hyperref[TEI.docEdition]{docEdition} \hyperref[TEI.titlePage]{titlePage} \hyperref[TEI.titlePart]{titlePart}\par 
    \item[transcr: ]
   \hyperref[TEI.damage]{damage} \hyperref[TEI.facsimile]{facsimile} \hyperref[TEI.fw]{fw} \hyperref[TEI.metamark]{metamark} \hyperref[TEI.mod]{mod} \hyperref[TEI.restore]{restore} \hyperref[TEI.retrace]{retrace} \hyperref[TEI.secl]{secl} \hyperref[TEI.sourceDoc]{sourceDoc} \hyperref[TEI.supplied]{supplied} \hyperref[TEI.surface]{surface} \hyperref[TEI.surplus]{surplus} \hyperref[TEI.zone]{zone}
    \item[{Peut contenir}]
  
    \item[core: ]
   \hyperref[TEI.desc]{desc}
    \item[{Note}]
  \par
L'attribut {\itshape mimeType} doit être utilisé pour spécifier le type MIME de l'image référencée par l'attribut {\itshape url}.
    \item[{Exemple}]
  \leavevmode\bgroup\exampleFont \begin{shaded}\noindent\mbox{}{<\textbf{figure}>}\mbox{}\newline 
\hspace*{6pt}{<\textbf{graphic}\hspace*{6pt}{url}="{fig1.png}"/>}\mbox{}\newline 
\hspace*{6pt}{<\textbf{head}>}Figure Une : Jan van Eyck, La Vierge du chancelier Rolin{</\textbf{head}>}\mbox{}\newline 
\hspace*{6pt}{<\textbf{p}>}Si, attiré par la curiosité, on a l'imprudence de l'approcher d'un peu trop prés, c'est fini, on est pris pour tout le temps que peut durer l'effort d'une attention soutenue ; on s'extasie devant la finesse du détail ... il va toujours plus loin, franchit une à une les croupes des collines verdoyantes ; se repose un moment sur une ligne lointaine de montagnes neigeuses; pour se perdre ensuite dans l'infini d'un ciel à peine bleu, où s'estompent de flottantes nuées. {</\textbf{p}>}\mbox{}\newline 
{</\textbf{figure}>}\end{shaded}\egroup 


    \item[{Modèle de contenu}]
  \mbox{}\hfill\\[-10pt]\begin{Verbatim}[fontsize=\small]
<content>
 <classRef key="model.descLike"
  maxOccurs="unbounded" minOccurs="0"/>
</content>
    
\end{Verbatim}

    \item[{Schéma Declaration}]
  \mbox{}\hfill\\[-10pt]\begin{Verbatim}[fontsize=\small]
element graphic
{
   tei_att.global.attributes,
   tei_att.media.attributes,
   tei_att.resourced.attributes,
   tei_att.declaring.attributes,
   tei_model.descLike*
}
\end{Verbatim}

\end{reflist}  \index{group=<group>|oddindex}
\begin{reflist}
\item[]\begin{specHead}{TEI.group}{<group> }(groupe) contient un ensemble de textes distincts (ou des groupes de textes de ce type), considérés comme formant une unité, par exemple pour présenter les œuvres complètes d’un auteur, une suite d’essais en prose, etc. [\xref{http://www.tei-c.org/release/doc/tei-p5-doc/en/html/DS.html\#DS}{4. Default Text Structure} \xref{http://www.tei-c.org/release/doc/tei-p5-doc/en/html/DS.html\#DSGRP}{4.3.1. Grouped Texts} \xref{http://www.tei-c.org/release/doc/tei-p5-doc/en/html/CC.html\#CCDEF}{15.1. Varieties of Composite Text}]\end{specHead} 
    \item[{Module}]
  textstructure
    \item[{Attributs}]
  Attributs \hyperref[TEI.att.global]{att.global} (\textit{@xml:id}, \textit{@n}, \textit{@xml:lang}, \textit{@xml:base}, \textit{@xml:space})  (\hyperref[TEI.att.global.rendition]{att.global.rendition} (\textit{@rend}, \textit{@style}, \textit{@rendition})) (\hyperref[TEI.att.global.linking]{att.global.linking} (\textit{@corresp}, \textit{@synch}, \textit{@sameAs}, \textit{@copyOf}, \textit{@next}, \textit{@prev}, \textit{@exclude}, \textit{@select})) (\hyperref[TEI.att.global.analytic]{att.global.analytic} (\textit{@ana})) (\hyperref[TEI.att.global.facs]{att.global.facs} (\textit{@facs})) (\hyperref[TEI.att.global.change]{att.global.change} (\textit{@change})) (\hyperref[TEI.att.global.responsibility]{att.global.responsibility} (\textit{@cert}, \textit{@resp})) (\hyperref[TEI.att.global.source]{att.global.source} (\textit{@source})) \hyperref[TEI.att.declaring]{att.declaring} (\textit{@decls}) \hyperref[TEI.att.typed]{att.typed} (\textit{@type}, \textit{@subtype}) 
    \item[{Contenu dans}]
  
    \item[textstructure: ]
   \hyperref[TEI.floatingText]{floatingText} \hyperref[TEI.group]{group} \hyperref[TEI.text]{text}
    \item[{Peut contenir}]
  
    \item[analysis: ]
   \hyperref[TEI.interp]{interp} \hyperref[TEI.interpGrp]{interpGrp} \hyperref[TEI.span]{span} \hyperref[TEI.spanGrp]{spanGrp}\par 
    \item[core: ]
   \hyperref[TEI.cb]{cb} \hyperref[TEI.gap]{gap} \hyperref[TEI.gb]{gb} \hyperref[TEI.head]{head} \hyperref[TEI.index]{index} \hyperref[TEI.lb]{lb} \hyperref[TEI.meeting]{meeting} \hyperref[TEI.milestone]{milestone} \hyperref[TEI.note]{note} \hyperref[TEI.pb]{pb}\par 
    \item[figures: ]
   \hyperref[TEI.figure]{figure} \hyperref[TEI.notatedMusic]{notatedMusic}\par 
    \item[iso-fs: ]
   \hyperref[TEI.fLib]{fLib} \hyperref[TEI.fs]{fs} \hyperref[TEI.fvLib]{fvLib}\par 
    \item[linking: ]
   \hyperref[TEI.alt]{alt} \hyperref[TEI.altGrp]{altGrp} \hyperref[TEI.anchor]{anchor} \hyperref[TEI.join]{join} \hyperref[TEI.joinGrp]{joinGrp} \hyperref[TEI.link]{link} \hyperref[TEI.linkGrp]{linkGrp} \hyperref[TEI.timeline]{timeline}\par 
    \item[msdescription: ]
   \hyperref[TEI.source]{source}\par 
    \item[textstructure: ]
   \hyperref[TEI.docAuthor]{docAuthor} \hyperref[TEI.docDate]{docDate} \hyperref[TEI.group]{group} \hyperref[TEI.text]{text}\par 
    \item[transcr: ]
   \hyperref[TEI.addSpan]{addSpan} \hyperref[TEI.damageSpan]{damageSpan} \hyperref[TEI.delSpan]{delSpan} \hyperref[TEI.fw]{fw} \hyperref[TEI.listTranspose]{listTranspose} \hyperref[TEI.metamark]{metamark} \hyperref[TEI.space]{space} \hyperref[TEI.substJoin]{substJoin}
    \item[{Exemple}]
  \leavevmode\bgroup\exampleFont \begin{shaded}\noindent\mbox{}{<\textbf{TEI} xmlns="http://www.tei-c.org/ns/1.0">}\mbox{}\newline 
\hspace*{6pt}{<\textbf{teiHeader}>}\mbox{}\newline 
\textit{<!--[ en-tête du texte composite ]-->}\mbox{}\newline 
\hspace*{6pt}{</\textbf{teiHeader}>}\mbox{}\newline 
\hspace*{6pt}{<\textbf{text}>}\mbox{}\newline 
\hspace*{6pt}\hspace*{6pt}{<\textbf{front}>}\mbox{}\newline 
\textit{<!--[ partie préliminaire du texte composite  ]-->}\mbox{}\newline 
\hspace*{6pt}\hspace*{6pt}{</\textbf{front}>}\mbox{}\newline 
\hspace*{6pt}\hspace*{6pt}{<\textbf{group}>}\mbox{}\newline 
\hspace*{6pt}\hspace*{6pt}\hspace*{6pt}{<\textbf{text}>}\mbox{}\newline 
\hspace*{6pt}\hspace*{6pt}\hspace*{6pt}\hspace*{6pt}{<\textbf{front}>}\mbox{}\newline 
\textit{<!--[ partie préliminaire du premier texte ]-->}\mbox{}\newline 
\hspace*{6pt}\hspace*{6pt}\hspace*{6pt}\hspace*{6pt}{</\textbf{front}>}\mbox{}\newline 
\hspace*{6pt}\hspace*{6pt}\hspace*{6pt}\hspace*{6pt}{<\textbf{body}>}\mbox{}\newline 
\textit{<!--[ corps  du premier texte ]-->}\mbox{}\newline 
\hspace*{6pt}\hspace*{6pt}\hspace*{6pt}\hspace*{6pt}{</\textbf{body}>}\mbox{}\newline 
\hspace*{6pt}\hspace*{6pt}\hspace*{6pt}\hspace*{6pt}{<\textbf{back}>}\mbox{}\newline 
\textit{<!--[ annexe  du premier texte ]-->}\mbox{}\newline 
\hspace*{6pt}\hspace*{6pt}\hspace*{6pt}\hspace*{6pt}{</\textbf{back}>}\mbox{}\newline 
\hspace*{6pt}\hspace*{6pt}\hspace*{6pt}{</\textbf{text}>}\mbox{}\newline 
\hspace*{6pt}\hspace*{6pt}\hspace*{6pt}{<\textbf{text}>}\mbox{}\newline 
\hspace*{6pt}\hspace*{6pt}\hspace*{6pt}\hspace*{6pt}{<\textbf{front}>}\mbox{}\newline 
\textit{<!--[ partie préliminaire du deuxième texte ]-->}\mbox{}\newline 
\hspace*{6pt}\hspace*{6pt}\hspace*{6pt}\hspace*{6pt}{</\textbf{front}>}\mbox{}\newline 
\hspace*{6pt}\hspace*{6pt}\hspace*{6pt}\hspace*{6pt}{<\textbf{body}>}\mbox{}\newline 
\textit{<!--[ corps du deuxième texte ]-->}\mbox{}\newline 
\hspace*{6pt}\hspace*{6pt}\hspace*{6pt}\hspace*{6pt}{</\textbf{body}>}\mbox{}\newline 
\hspace*{6pt}\hspace*{6pt}\hspace*{6pt}\hspace*{6pt}{<\textbf{back}>}\mbox{}\newline 
\textit{<!--[ annexe du deuxième texte ]-->}\mbox{}\newline 
\hspace*{6pt}\hspace*{6pt}\hspace*{6pt}\hspace*{6pt}{</\textbf{back}>}\mbox{}\newline 
\hspace*{6pt}\hspace*{6pt}\hspace*{6pt}{</\textbf{text}>}\mbox{}\newline 
\textit{<!--[ encore de textes, simples ou composites  ]-->}\mbox{}\newline 
\hspace*{6pt}\hspace*{6pt}{</\textbf{group}>}\mbox{}\newline 
\hspace*{6pt}\hspace*{6pt}{<\textbf{back}>}\mbox{}\newline 
\textit{<!--[ annexe du texte composite  ]-->}\mbox{}\newline 
\hspace*{6pt}\hspace*{6pt}{</\textbf{back}>}\mbox{}\newline 
\hspace*{6pt}{</\textbf{text}>}\mbox{}\newline 
{</\textbf{TEI}>}\end{shaded}\egroup 


    \item[{Modèle de contenu}]
  \mbox{}\hfill\\[-10pt]\begin{Verbatim}[fontsize=\small]
<content>
 <sequence maxOccurs="1" minOccurs="1">
  <alternate maxOccurs="unbounded"
   minOccurs="0">
   <classRef key="model.divTop"/>
   <classRef key="model.global"/>
  </alternate>
  <sequence maxOccurs="1" minOccurs="1">
   <alternate maxOccurs="1" minOccurs="1">
    <elementRef key="text"/>
    <elementRef key="group"/>
   </alternate>
   <alternate maxOccurs="unbounded"
    minOccurs="0">
    <elementRef key="text"/>
    <elementRef key="group"/>
    <classRef key="model.global"/>
   </alternate>
  </sequence>
  <classRef key="model.divBottom"
   maxOccurs="unbounded" minOccurs="0"/>
 </sequence>
</content>
    
\end{Verbatim}

    \item[{Schéma Declaration}]
  \mbox{}\hfill\\[-10pt]\begin{Verbatim}[fontsize=\small]
element group
{
   tei_att.global.attributes,
   tei_att.declaring.attributes,
   tei_att.typed.attributes,
   (
      ( tei_model.divTop | tei_model.global )*,
      (
         ( tei_text | tei_group ),
         ( tei_text | tei_group | tei_model.global )*
      ),
      tei_model.divBottom*
   )
}
\end{Verbatim}

\end{reflist}  \index{handDesc=<handDesc>|oddindex}\index{hands=@hands!<handDesc>|oddindex}
\begin{reflist}
\item[]\begin{specHead}{TEI.handDesc}{<handDesc> }(description des écritures) contient la description des différents types d'écriture utilisés dans un manuscrit. [\xref{http://www.tei-c.org/release/doc/tei-p5-doc/en/html/MS.html\#msph2}{10.7.2. Writing, Decoration, and Other Notations}]\end{specHead} 
    \item[{Module}]
  msdescription
    \item[{Attributs}]
  Attributs \hyperref[TEI.att.global]{att.global} (\textit{@xml:id}, \textit{@n}, \textit{@xml:lang}, \textit{@xml:base}, \textit{@xml:space})  (\hyperref[TEI.att.global.rendition]{att.global.rendition} (\textit{@rend}, \textit{@style}, \textit{@rendition})) (\hyperref[TEI.att.global.linking]{att.global.linking} (\textit{@corresp}, \textit{@synch}, \textit{@sameAs}, \textit{@copyOf}, \textit{@next}, \textit{@prev}, \textit{@exclude}, \textit{@select})) (\hyperref[TEI.att.global.analytic]{att.global.analytic} (\textit{@ana})) (\hyperref[TEI.att.global.facs]{att.global.facs} (\textit{@facs})) (\hyperref[TEI.att.global.change]{att.global.change} (\textit{@change})) (\hyperref[TEI.att.global.responsibility]{att.global.responsibility} (\textit{@cert}, \textit{@resp})) (\hyperref[TEI.att.global.source]{att.global.source} (\textit{@source})) \hfil\\[-10pt]\begin{sansreflist}
    \item[@hands]
  (mains) spécifie le nombre de mains différentes qui ont pu être identifiées dans le manuscrit
\begin{reflist}
    \item[{Statut}]
  Optionel
    \item[{Type de données}]
  \hyperref[TEI.teidata.count]{teidata.count}
\end{reflist}  
\end{sansreflist}  
    \item[{Membre du}]
  \hyperref[TEI.model.physDescPart]{model.physDescPart}
    \item[{Contenu dans}]
  
    \item[msdescription: ]
   \hyperref[TEI.physDesc]{physDesc}
    \item[{Peut contenir}]
  
    \item[core: ]
   \hyperref[TEI.p]{p}\par 
    \item[linking: ]
   \hyperref[TEI.ab]{ab}\par 
    \item[msdescription: ]
   \hyperref[TEI.summary]{summary}
    \item[{Exemple}]
  \leavevmode\bgroup\exampleFont \begin{shaded}\noindent\mbox{}{<\textbf{handDesc}>}\mbox{}\newline 
\hspace*{6pt}{<\textbf{handNote}\hspace*{6pt}{scope}="{major}">}Written throughout in {<\textbf{term}>}angelicana formata{</\textbf{term}>}.{</\textbf{handNote}>}\mbox{}\newline 
{</\textbf{handDesc}>}\end{shaded}\egroup 


    \item[{Exemple}]
  \leavevmode\bgroup\exampleFont \begin{shaded}\noindent\mbox{}{<\textbf{handDesc}\hspace*{6pt}{hands}="{2}">}\mbox{}\newline 
\hspace*{6pt}{<\textbf{p}>}The manuscript is written in two contemporary hands, otherwise unknown, but clearly\mbox{}\newline 
\hspace*{6pt}\hspace*{6pt} those of practised scribes. Hand I writes ff. 1r-22v and hand II ff. 23 and 24. Some\mbox{}\newline 
\hspace*{6pt}\hspace*{6pt} scholars, notably Verner Dahlerup and Hreinn Benediktsson, have argued for a third hand\mbox{}\newline 
\hspace*{6pt}\hspace*{6pt} on f. 24, but the evidence for this is insubstantial.{</\textbf{p}>}\mbox{}\newline 
{</\textbf{handDesc}>}\end{shaded}\egroup 


    \item[{Exemple}]
  \leavevmode\bgroup\exampleFont \begin{shaded}\noindent\mbox{}{<\textbf{handDesc}\hspace*{6pt}{hands}="{2}">}\mbox{}\newline 
\hspace*{6pt}{<\textbf{handNote}\hspace*{6pt}{medium}="{typescript}"\mbox{}\newline 
\hspace*{6pt}\hspace*{6pt}{xml:id}="{fr\textunderscore TSE}">}Authorial typescript {</\textbf{handNote}>}\mbox{}\newline 
\hspace*{6pt}{<\textbf{handNote}\hspace*{6pt}{medium}="{red-ink}"\hspace*{6pt}{xml:id}="{fr\textunderscore EP}">}Ezra Pound's annotations{</\textbf{handNote}>}\mbox{}\newline 
{</\textbf{handDesc}>}\end{shaded}\egroup 


    \item[{Modèle de contenu}]
  \mbox{}\hfill\\[-10pt]\begin{Verbatim}[fontsize=\small]
<content>
 <alternate maxOccurs="1" minOccurs="1">
  <classRef key="model.pLike"
   maxOccurs="unbounded" minOccurs="1"/>
  <sequence maxOccurs="1" minOccurs="1">
   <elementRef key="summary" minOccurs="0"/>
   <elementRef key="handNote"
    maxOccurs="unbounded" minOccurs="1"/>
  </sequence>
 </alternate>
</content>
    
\end{Verbatim}

    \item[{Schéma Declaration}]
  \mbox{}\hfill\\[-10pt]\begin{Verbatim}[fontsize=\small]
element handDesc
{
   tei_att.global.attributes,
   attribute hands { text }?,
   ( tei_model.pLike+ | ( tei_summary?, handNote+ ) )
}
\end{Verbatim}

\end{reflist}  \index{handNotes=<handNotes>|oddindex}
\begin{reflist}
\item[]\begin{specHead}{TEI.handNotes}{<handNotes> }contient un ou plusieurs éléments \texttt{<handNote>} qui documentent les différentes mains identifiées dans les textes source. [\xref{http://www.tei-c.org/release/doc/tei-p5-doc/en/html/PH.html\#PHDH}{11.3.2.1. Document Hands}]\end{specHead} 
    \item[{Module}]
  transcr
    \item[{Attributs}]
  Attributs \hyperref[TEI.att.global]{att.global} (\textit{@xml:id}, \textit{@n}, \textit{@xml:lang}, \textit{@xml:base}, \textit{@xml:space})  (\hyperref[TEI.att.global.rendition]{att.global.rendition} (\textit{@rend}, \textit{@style}, \textit{@rendition})) (\hyperref[TEI.att.global.linking]{att.global.linking} (\textit{@corresp}, \textit{@synch}, \textit{@sameAs}, \textit{@copyOf}, \textit{@next}, \textit{@prev}, \textit{@exclude}, \textit{@select})) (\hyperref[TEI.att.global.analytic]{att.global.analytic} (\textit{@ana})) (\hyperref[TEI.att.global.facs]{att.global.facs} (\textit{@facs})) (\hyperref[TEI.att.global.change]{att.global.change} (\textit{@change})) (\hyperref[TEI.att.global.responsibility]{att.global.responsibility} (\textit{@cert}, \textit{@resp})) (\hyperref[TEI.att.global.source]{att.global.source} (\textit{@source}))
    \item[{Membre du}]
  \hyperref[TEI.model.profileDescPart]{model.profileDescPart}
    \item[{Contenu dans}]
  
    \item[header: ]
   \hyperref[TEI.profileDesc]{profileDesc}
    \item[{Peut contenir}]
  Elément vide
    \item[{Exemple}]
  \leavevmode\bgroup\exampleFont \begin{shaded}\noindent\mbox{}{<\textbf{handNotes}>}\mbox{}\newline 
\hspace*{6pt}{<\textbf{handNote}\hspace*{6pt}{medium}="{brown-ink}"\mbox{}\newline 
\hspace*{6pt}\hspace*{6pt}{script}="{copperplate}"\hspace*{6pt}{xml:id}="{H1}">}Carefully written with regular descenders{</\textbf{handNote}>}\mbox{}\newline 
\hspace*{6pt}{<\textbf{handNote}\hspace*{6pt}{medium}="{pencil}"\hspace*{6pt}{script}="{print}"\mbox{}\newline 
\hspace*{6pt}\hspace*{6pt}{xml:id}="{H2}">}Unschooled scrawl{</\textbf{handNote}>}\mbox{}\newline 
{</\textbf{handNotes}>}\end{shaded}\egroup 


    \item[{Modèle de contenu}]
  \mbox{}\hfill\\[-10pt]\begin{Verbatim}[fontsize=\small]
<content>
 <elementRef key="handNote"
  maxOccurs="unbounded" minOccurs="1"/>
</content>
    
\end{Verbatim}

    \item[{Schéma Declaration}]
  \mbox{}\hfill\\[-10pt]\begin{Verbatim}[fontsize=\small]
element handNotes { tei_att.global.attributes, handNote+ }
\end{Verbatim}

\end{reflist}  \index{handShift=<handShift>|oddindex}\index{new=@new!<handShift>|oddindex}
\begin{reflist}
\item[]\begin{specHead}{TEI.handShift}{<handShift> }(reprise de main) marque le début d'une section du texte écrite par une nouvelle main ou le début d'une nouvelle séance d'écriture. [\xref{http://www.tei-c.org/release/doc/tei-p5-doc/en/html/PH.html\#PHDH}{11.3.2.1. Document Hands}]\end{specHead} 
    \item[{Module}]
  transcr
    \item[{Attributs}]
  Attributs \hyperref[TEI.att.global]{att.global} (\textit{@xml:id}, \textit{@n}, \textit{@xml:lang}, \textit{@xml:base}, \textit{@xml:space})  (\hyperref[TEI.att.global.rendition]{att.global.rendition} (\textit{@rend}, \textit{@style}, \textit{@rendition})) (\hyperref[TEI.att.global.linking]{att.global.linking} (\textit{@corresp}, \textit{@synch}, \textit{@sameAs}, \textit{@copyOf}, \textit{@next}, \textit{@prev}, \textit{@exclude}, \textit{@select})) (\hyperref[TEI.att.global.analytic]{att.global.analytic} (\textit{@ana})) (\hyperref[TEI.att.global.facs]{att.global.facs} (\textit{@facs})) (\hyperref[TEI.att.global.change]{att.global.change} (\textit{@change})) (\hyperref[TEI.att.global.responsibility]{att.global.responsibility} (\textit{@cert}, \textit{@resp})) (\hyperref[TEI.att.global.source]{att.global.source} (\textit{@source})) \hyperref[TEI.att.handFeatures]{att.handFeatures} (\textit{@scribe}, \textit{@scribeRef}, \textit{@script}, \textit{@scriptRef}, \textit{@medium}, \textit{@scope}) \hfil\\[-10pt]\begin{sansreflist}
    \item[@new]
  donne l'identifiant de la nouvelle main.
\begin{reflist}
    \item[{Statut}]
  Recommendé
    \item[{Type de données}]
  \hyperref[TEI.teidata.pointer]{teidata.pointer}
\end{reflist}  
\end{sansreflist}  
    \item[{Membre du}]
  \hyperref[TEI.model.linePart]{model.linePart} \hyperref[TEI.model.pPart.transcriptional]{model.pPart.transcriptional}
    \item[{Contenu dans}]
  
    \item[analysis: ]
   \hyperref[TEI.cl]{cl} \hyperref[TEI.pc]{pc} \hyperref[TEI.phr]{phr} \hyperref[TEI.s]{s} \hyperref[TEI.w]{w}\par 
    \item[core: ]
   \hyperref[TEI.abbr]{abbr} \hyperref[TEI.add]{add} \hyperref[TEI.addrLine]{addrLine} \hyperref[TEI.author]{author} \hyperref[TEI.bibl]{bibl} \hyperref[TEI.biblScope]{biblScope} \hyperref[TEI.citedRange]{citedRange} \hyperref[TEI.corr]{corr} \hyperref[TEI.date]{date} \hyperref[TEI.del]{del} \hyperref[TEI.distinct]{distinct} \hyperref[TEI.editor]{editor} \hyperref[TEI.email]{email} \hyperref[TEI.emph]{emph} \hyperref[TEI.expan]{expan} \hyperref[TEI.foreign]{foreign} \hyperref[TEI.gloss]{gloss} \hyperref[TEI.head]{head} \hyperref[TEI.headItem]{headItem} \hyperref[TEI.headLabel]{headLabel} \hyperref[TEI.hi]{hi} \hyperref[TEI.item]{item} \hyperref[TEI.l]{l} \hyperref[TEI.label]{label} \hyperref[TEI.measure]{measure} \hyperref[TEI.mentioned]{mentioned} \hyperref[TEI.name]{name} \hyperref[TEI.note]{note} \hyperref[TEI.num]{num} \hyperref[TEI.orig]{orig} \hyperref[TEI.p]{p} \hyperref[TEI.pubPlace]{pubPlace} \hyperref[TEI.publisher]{publisher} \hyperref[TEI.q]{q} \hyperref[TEI.quote]{quote} \hyperref[TEI.ref]{ref} \hyperref[TEI.reg]{reg} \hyperref[TEI.rs]{rs} \hyperref[TEI.said]{said} \hyperref[TEI.sic]{sic} \hyperref[TEI.soCalled]{soCalled} \hyperref[TEI.speaker]{speaker} \hyperref[TEI.stage]{stage} \hyperref[TEI.street]{street} \hyperref[TEI.term]{term} \hyperref[TEI.textLang]{textLang} \hyperref[TEI.time]{time} \hyperref[TEI.title]{title} \hyperref[TEI.unclear]{unclear}\par 
    \item[figures: ]
   \hyperref[TEI.cell]{cell}\par 
    \item[header: ]
   \hyperref[TEI.change]{change} \hyperref[TEI.distributor]{distributor} \hyperref[TEI.edition]{edition} \hyperref[TEI.extent]{extent} \hyperref[TEI.licence]{licence}\par 
    \item[linking: ]
   \hyperref[TEI.ab]{ab} \hyperref[TEI.seg]{seg}\par 
    \item[msdescription: ]
   \hyperref[TEI.accMat]{accMat} \hyperref[TEI.acquisition]{acquisition} \hyperref[TEI.additions]{additions} \hyperref[TEI.catchwords]{catchwords} \hyperref[TEI.collation]{collation} \hyperref[TEI.colophon]{colophon} \hyperref[TEI.condition]{condition} \hyperref[TEI.custEvent]{custEvent} \hyperref[TEI.decoNote]{decoNote} \hyperref[TEI.explicit]{explicit} \hyperref[TEI.filiation]{filiation} \hyperref[TEI.finalRubric]{finalRubric} \hyperref[TEI.foliation]{foliation} \hyperref[TEI.heraldry]{heraldry} \hyperref[TEI.incipit]{incipit} \hyperref[TEI.layout]{layout} \hyperref[TEI.material]{material} \hyperref[TEI.musicNotation]{musicNotation} \hyperref[TEI.objectType]{objectType} \hyperref[TEI.origDate]{origDate} \hyperref[TEI.origPlace]{origPlace} \hyperref[TEI.origin]{origin} \hyperref[TEI.provenance]{provenance} \hyperref[TEI.rubric]{rubric} \hyperref[TEI.secFol]{secFol} \hyperref[TEI.signatures]{signatures} \hyperref[TEI.source]{source} \hyperref[TEI.stamp]{stamp} \hyperref[TEI.summary]{summary} \hyperref[TEI.support]{support} \hyperref[TEI.surrogates]{surrogates} \hyperref[TEI.typeNote]{typeNote} \hyperref[TEI.watermark]{watermark}\par 
    \item[namesdates: ]
   \hyperref[TEI.addName]{addName} \hyperref[TEI.affiliation]{affiliation} \hyperref[TEI.country]{country} \hyperref[TEI.forename]{forename} \hyperref[TEI.genName]{genName} \hyperref[TEI.geogName]{geogName} \hyperref[TEI.nameLink]{nameLink} \hyperref[TEI.orgName]{orgName} \hyperref[TEI.persName]{persName} \hyperref[TEI.placeName]{placeName} \hyperref[TEI.region]{region} \hyperref[TEI.roleName]{roleName} \hyperref[TEI.settlement]{settlement} \hyperref[TEI.surname]{surname}\par 
    \item[textstructure: ]
   \hyperref[TEI.docAuthor]{docAuthor} \hyperref[TEI.docDate]{docDate} \hyperref[TEI.docEdition]{docEdition} \hyperref[TEI.titlePart]{titlePart}\par 
    \item[transcr: ]
   \hyperref[TEI.am]{am} \hyperref[TEI.damage]{damage} \hyperref[TEI.fw]{fw} \hyperref[TEI.line]{line} \hyperref[TEI.metamark]{metamark} \hyperref[TEI.mod]{mod} \hyperref[TEI.restore]{restore} \hyperref[TEI.retrace]{retrace} \hyperref[TEI.secl]{secl} \hyperref[TEI.supplied]{supplied} \hyperref[TEI.surplus]{surplus} \hyperref[TEI.zone]{zone}
    \item[{Peut contenir}]
  Elément vide
    \item[{Note}]
  \par
L'élément \hyperref[TEI.handShift]{<handShift>}peut être utilisé soit pour noter un changement de main dans le document (comme le passage d'un scribe à un autre, d'un style d'écriture à un autre), soit pour indiquer un changement dans la main, comme un changement d'écriture, de caractère ou d'encre.
    \item[{Exemple}]
  \leavevmode\bgroup\exampleFont \begin{shaded}\noindent\mbox{}{<\textbf{l}>}When wolde the cat dwelle in his ynne{</\textbf{l}>}\mbox{}\newline 
{<\textbf{handShift}\hspace*{6pt}{medium}="{greenish-ink}"/>}\mbox{}\newline 
{<\textbf{l}>}And if the cattes skynne be slyk {<\textbf{handShift}\hspace*{6pt}{medium}="{black-ink}"/>} and gaye{</\textbf{l}>}\end{shaded}\egroup 


    \item[{Modèle de contenu}]
  \fbox{\ttfamily <content>\newline
</content>\newline
    } 
    \item[{Schéma Declaration}]
  \mbox{}\hfill\\[-10pt]\begin{Verbatim}[fontsize=\small]
element handShift
{
   tei_att.global.attributes,
   tei_att.handFeatures.attributes,
   attribute new { text }?,
   empty
}
\end{Verbatim}

\end{reflist}  \index{head=<head>|oddindex}
\begin{reflist}
\item[]\begin{specHead}{TEI.head}{<head> }(en-tête) contient tout type d'en-tête, par exemple le titre d'une section, ou l'intitulé d'une liste, d'un glossaire, d'une description de manuscrit, etc. [\xref{http://www.tei-c.org/release/doc/tei-p5-doc/en/html/DS.html\#DSHD}{4.2.1. Headings and Trailers}]\end{specHead} 
    \item[{Module}]
  core
    \item[{Attributs}]
  Attributs \hyperref[TEI.att.global]{att.global} (\textit{@xml:id}, \textit{@n}, \textit{@xml:lang}, \textit{@xml:base}, \textit{@xml:space})  (\hyperref[TEI.att.global.rendition]{att.global.rendition} (\textit{@rend}, \textit{@style}, \textit{@rendition})) (\hyperref[TEI.att.global.linking]{att.global.linking} (\textit{@corresp}, \textit{@synch}, \textit{@sameAs}, \textit{@copyOf}, \textit{@next}, \textit{@prev}, \textit{@exclude}, \textit{@select})) (\hyperref[TEI.att.global.analytic]{att.global.analytic} (\textit{@ana})) (\hyperref[TEI.att.global.facs]{att.global.facs} (\textit{@facs})) (\hyperref[TEI.att.global.change]{att.global.change} (\textit{@change})) (\hyperref[TEI.att.global.responsibility]{att.global.responsibility} (\textit{@cert}, \textit{@resp})) (\hyperref[TEI.att.global.source]{att.global.source} (\textit{@source})) \hyperref[TEI.att.typed]{att.typed} (\textit{@type}, \textit{@subtype}) \hyperref[TEI.att.placement]{att.placement} (\textit{@place}) \hyperref[TEI.att.written]{att.written} (\textit{@hand}) 
    \item[{Membre du}]
  \hyperref[TEI.model.headLike]{model.headLike} \hyperref[TEI.model.pLike.front]{model.pLike.front}
    \item[{Contenu dans}]
  
    \item[core: ]
   \hyperref[TEI.divGen]{divGen} \hyperref[TEI.lg]{lg} \hyperref[TEI.list]{list} \hyperref[TEI.listBibl]{listBibl}\par 
    \item[figures: ]
   \hyperref[TEI.figure]{figure} \hyperref[TEI.table]{table}\par 
    \item[header: ]
   \hyperref[TEI.abstract]{abstract}\par 
    \item[msdescription: ]
   \hyperref[TEI.msDesc]{msDesc} \hyperref[TEI.msFrag]{msFrag} \hyperref[TEI.msPart]{msPart}\par 
    \item[namesdates: ]
   \hyperref[TEI.event]{event} \hyperref[TEI.listOrg]{listOrg} \hyperref[TEI.listPlace]{listPlace} \hyperref[TEI.org]{org} \hyperref[TEI.place]{place} \hyperref[TEI.state]{state}\par 
    \item[standOff: ]
   \hyperref[TEI.listAnnotation]{listAnnotation}\par 
    \item[textstructure: ]
   \hyperref[TEI.back]{back} \hyperref[TEI.body]{body} \hyperref[TEI.div]{div} \hyperref[TEI.front]{front} \hyperref[TEI.group]{group}
    \item[{Peut contenir}]
  
    \item[analysis: ]
   \hyperref[TEI.c]{c} \hyperref[TEI.cl]{cl} \hyperref[TEI.interp]{interp} \hyperref[TEI.interpGrp]{interpGrp} \hyperref[TEI.m]{m} \hyperref[TEI.pc]{pc} \hyperref[TEI.phr]{phr} \hyperref[TEI.s]{s} \hyperref[TEI.span]{span} \hyperref[TEI.spanGrp]{spanGrp} \hyperref[TEI.w]{w}\par 
    \item[core: ]
   \hyperref[TEI.abbr]{abbr} \hyperref[TEI.add]{add} \hyperref[TEI.address]{address} \hyperref[TEI.bibl]{bibl} \hyperref[TEI.biblStruct]{biblStruct} \hyperref[TEI.binaryObject]{binaryObject} \hyperref[TEI.cb]{cb} \hyperref[TEI.choice]{choice} \hyperref[TEI.cit]{cit} \hyperref[TEI.corr]{corr} \hyperref[TEI.date]{date} \hyperref[TEI.del]{del} \hyperref[TEI.desc]{desc} \hyperref[TEI.distinct]{distinct} \hyperref[TEI.email]{email} \hyperref[TEI.emph]{emph} \hyperref[TEI.expan]{expan} \hyperref[TEI.foreign]{foreign} \hyperref[TEI.gap]{gap} \hyperref[TEI.gb]{gb} \hyperref[TEI.gloss]{gloss} \hyperref[TEI.graphic]{graphic} \hyperref[TEI.hi]{hi} \hyperref[TEI.index]{index} \hyperref[TEI.l]{l} \hyperref[TEI.label]{label} \hyperref[TEI.lb]{lb} \hyperref[TEI.lg]{lg} \hyperref[TEI.list]{list} \hyperref[TEI.listBibl]{listBibl} \hyperref[TEI.measure]{measure} \hyperref[TEI.measureGrp]{measureGrp} \hyperref[TEI.media]{media} \hyperref[TEI.mentioned]{mentioned} \hyperref[TEI.milestone]{milestone} \hyperref[TEI.name]{name} \hyperref[TEI.note]{note} \hyperref[TEI.num]{num} \hyperref[TEI.orig]{orig} \hyperref[TEI.pb]{pb} \hyperref[TEI.ptr]{ptr} \hyperref[TEI.q]{q} \hyperref[TEI.quote]{quote} \hyperref[TEI.ref]{ref} \hyperref[TEI.reg]{reg} \hyperref[TEI.rs]{rs} \hyperref[TEI.said]{said} \hyperref[TEI.sic]{sic} \hyperref[TEI.soCalled]{soCalled} \hyperref[TEI.stage]{stage} \hyperref[TEI.term]{term} \hyperref[TEI.time]{time} \hyperref[TEI.title]{title} \hyperref[TEI.unclear]{unclear}\par 
    \item[derived-module-tei.istex: ]
   \hyperref[TEI.math]{math} \hyperref[TEI.mrow]{mrow}\par 
    \item[figures: ]
   \hyperref[TEI.figure]{figure} \hyperref[TEI.formula]{formula} \hyperref[TEI.notatedMusic]{notatedMusic} \hyperref[TEI.table]{table}\par 
    \item[header: ]
   \hyperref[TEI.biblFull]{biblFull} \hyperref[TEI.idno]{idno}\par 
    \item[iso-fs: ]
   \hyperref[TEI.fLib]{fLib} \hyperref[TEI.fs]{fs} \hyperref[TEI.fvLib]{fvLib}\par 
    \item[linking: ]
   \hyperref[TEI.alt]{alt} \hyperref[TEI.altGrp]{altGrp} \hyperref[TEI.anchor]{anchor} \hyperref[TEI.join]{join} \hyperref[TEI.joinGrp]{joinGrp} \hyperref[TEI.link]{link} \hyperref[TEI.linkGrp]{linkGrp} \hyperref[TEI.seg]{seg} \hyperref[TEI.timeline]{timeline}\par 
    \item[msdescription: ]
   \hyperref[TEI.catchwords]{catchwords} \hyperref[TEI.depth]{depth} \hyperref[TEI.dim]{dim} \hyperref[TEI.dimensions]{dimensions} \hyperref[TEI.height]{height} \hyperref[TEI.heraldry]{heraldry} \hyperref[TEI.locus]{locus} \hyperref[TEI.locusGrp]{locusGrp} \hyperref[TEI.material]{material} \hyperref[TEI.msDesc]{msDesc} \hyperref[TEI.objectType]{objectType} \hyperref[TEI.origDate]{origDate} \hyperref[TEI.origPlace]{origPlace} \hyperref[TEI.secFol]{secFol} \hyperref[TEI.signatures]{signatures} \hyperref[TEI.source]{source} \hyperref[TEI.stamp]{stamp} \hyperref[TEI.watermark]{watermark} \hyperref[TEI.width]{width}\par 
    \item[namesdates: ]
   \hyperref[TEI.addName]{addName} \hyperref[TEI.affiliation]{affiliation} \hyperref[TEI.country]{country} \hyperref[TEI.forename]{forename} \hyperref[TEI.genName]{genName} \hyperref[TEI.geogName]{geogName} \hyperref[TEI.listOrg]{listOrg} \hyperref[TEI.listPlace]{listPlace} \hyperref[TEI.location]{location} \hyperref[TEI.nameLink]{nameLink} \hyperref[TEI.orgName]{orgName} \hyperref[TEI.persName]{persName} \hyperref[TEI.placeName]{placeName} \hyperref[TEI.region]{region} \hyperref[TEI.roleName]{roleName} \hyperref[TEI.settlement]{settlement} \hyperref[TEI.state]{state} \hyperref[TEI.surname]{surname}\par 
    \item[spoken: ]
   \hyperref[TEI.annotationBlock]{annotationBlock}\par 
    \item[textstructure: ]
   \hyperref[TEI.floatingText]{floatingText}\par 
    \item[transcr: ]
   \hyperref[TEI.addSpan]{addSpan} \hyperref[TEI.am]{am} \hyperref[TEI.damage]{damage} \hyperref[TEI.damageSpan]{damageSpan} \hyperref[TEI.delSpan]{delSpan} \hyperref[TEI.ex]{ex} \hyperref[TEI.fw]{fw} \hyperref[TEI.handShift]{handShift} \hyperref[TEI.listTranspose]{listTranspose} \hyperref[TEI.metamark]{metamark} \hyperref[TEI.mod]{mod} \hyperref[TEI.redo]{redo} \hyperref[TEI.restore]{restore} \hyperref[TEI.retrace]{retrace} \hyperref[TEI.secl]{secl} \hyperref[TEI.space]{space} \hyperref[TEI.subst]{subst} \hyperref[TEI.substJoin]{substJoin} \hyperref[TEI.supplied]{supplied} \hyperref[TEI.surplus]{surplus} \hyperref[TEI.undo]{undo}\par des données textuelles
    \item[{Note}]
  \par
L'élément \hyperref[TEI.head]{<head>} est utilisé pour les titres de tous niveaux ; un logiciel qui traitera différemment, par exemple, les titres de chapitres, les titres de sections et les titres de listes, devra déterminer le traitement approprié de l'élément \hyperref[TEI.head]{<head>} rencontré en fonction de sa position dans la structure XML. Un élément \hyperref[TEI.head]{<head>} qui est le premier élément d'une liste est le titre de cette liste ; si l'élément \hyperref[TEI.head]{<head>} apparaît comme le premier élément d'un élément \texttt{<div1>}, il sera considéré comme le titre de ce chapitre ou de cette section.
    \item[{Exemple}]
  L'élément\hyperref[TEI.head]{<head>}est employé habituellement pour marquer les titres de sections. Dans d'anciens textes , les titres des textes conclusifs seront précédés de l'élément \texttt{<trailer>}, comme dans cet exemple :\leavevmode\bgroup\exampleFont \begin{shaded}\noindent\mbox{}{<\textbf{div}\hspace*{6pt}{type}="{chapitre}">}\mbox{}\newline 
\hspace*{6pt}{<\textbf{head}>}Les Mille et une Nuits{</\textbf{head}>}\mbox{}\newline 
\hspace*{6pt}{<\textbf{p}>}LES chroniques des Sassaniens, anciens rois de Perse, qui avaient étendu leur empire\mbox{}\newline 
\hspace*{6pt}\hspace*{6pt} dans les Indes, [...]{</\textbf{p}>}\mbox{}\newline 
\hspace*{6pt}{<\textbf{div}\hspace*{6pt}{type}="{histoire}">}\mbox{}\newline 
\hspace*{6pt}\hspace*{6pt}{<\textbf{head}>}Histoire du Vizir puni{</\textbf{head}>}\mbox{}\newline 
\hspace*{6pt}\hspace*{6pt}{<\textbf{p}>}IL était autrefois un roi, poursuivit-il, qui avait un{<\textbf{lb}/>} fils qui aimait\mbox{}\newline 
\hspace*{6pt}\hspace*{6pt}\hspace*{6pt}\hspace*{6pt} passionnément la chasse. Il lui permettait{<\textbf{lb}/>} de prendre souvent ce divertissement ;\mbox{}\newline 
\hspace*{6pt}\hspace*{6pt}\hspace*{6pt}\hspace*{6pt} [...] {</\textbf{p}>}\mbox{}\newline 
\hspace*{6pt}{</\textbf{div}>}\mbox{}\newline 
{</\textbf{div}>}\mbox{}\newline 
{<\textbf{trailer}>}\mbox{}\newline 
\hspace*{6pt}{<\textbf{hi}\hspace*{6pt}{rend}="{majuscule}">}fin du tome premier.{</\textbf{hi}>}\mbox{}\newline 
{</\textbf{trailer}>}\end{shaded}\egroup 


    \item[{Exemple}]
  L'élément \hyperref[TEI.head]{<head>} est aussi employé pour donner un titre à d'autres éléments, dans une liste par exemple :\leavevmode\bgroup\exampleFont \begin{shaded}\noindent\mbox{} Les déictiques\mbox{}\newline 
 sont des termes qui ne prennent leur sens que dans le cadre de la situation d'énonciation.\mbox{}\newline 
{<\textbf{list}\hspace*{6pt}{rend}="{bulleted}">}\mbox{}\newline 
\hspace*{6pt}{<\textbf{head}>}déictiques{</\textbf{head}>}\mbox{}\newline 
\hspace*{6pt}{<\textbf{item}>}ici{</\textbf{item}>}\mbox{}\newline 
\hspace*{6pt}{<\textbf{item}>}hier{</\textbf{item}>}\mbox{}\newline 
\hspace*{6pt}{<\textbf{item}>}là{</\textbf{item}>}\mbox{}\newline 
\hspace*{6pt}{<\textbf{item}>}je{</\textbf{item}>}\mbox{}\newline 
\hspace*{6pt}{<\textbf{item}>}tu{</\textbf{item}>}\mbox{}\newline 
\hspace*{6pt}{<\textbf{item}>}nous{</\textbf{item}>}\mbox{}\newline 
\hspace*{6pt}{<\textbf{item}>}vous{</\textbf{item}>}\mbox{}\newline 
\hspace*{6pt}{<\textbf{item}/>}\mbox{}\newline 
{</\textbf{list}>}\end{shaded}\egroup 


    \item[{Modèle de contenu}]
  \mbox{}\hfill\\[-10pt]\begin{Verbatim}[fontsize=\small]
<content>
 <alternate maxOccurs="unbounded"
  minOccurs="0">
  <textNode/>
  <elementRef key="lg"/>
  <classRef key="model.gLike"/>
  <classRef key="model.phrase"/>
  <classRef key="model.inter"/>
  <classRef key="model.lLike"/>
  <classRef key="model.global"/>
 </alternate>
</content>
    
\end{Verbatim}

    \item[{Schéma Declaration}]
  \mbox{}\hfill\\[-10pt]\begin{Verbatim}[fontsize=\small]
element head
{
   tei_att.global.attributes,
   tei_att.typed.attributes,
   tei_att.placement.attributes,
   tei_att.written.attributes,
   (
      text
    | tei_lg    | tei_model.gLike    | tei_model.phrase    | tei_model.inter    | tei_model.lLike    | tei_model.global   )*
}
\end{Verbatim}

\end{reflist}  \index{headItem=<headItem>|oddindex}
\begin{reflist}
\item[]\begin{specHead}{TEI.headItem}{<headItem> }(intitulé d'une liste d'items ) contient l'intitulé pour la colonne d'items ou de gloses dans un glossaire ou dans une liste semblablement structurée. [\xref{http://www.tei-c.org/release/doc/tei-p5-doc/en/html/CO.html\#COLI}{3.7. Lists}]\end{specHead} 
    \item[{Module}]
  core
    \item[{Attributs}]
  Attributs \hyperref[TEI.att.global]{att.global} (\textit{@xml:id}, \textit{@n}, \textit{@xml:lang}, \textit{@xml:base}, \textit{@xml:space})  (\hyperref[TEI.att.global.rendition]{att.global.rendition} (\textit{@rend}, \textit{@style}, \textit{@rendition})) (\hyperref[TEI.att.global.linking]{att.global.linking} (\textit{@corresp}, \textit{@synch}, \textit{@sameAs}, \textit{@copyOf}, \textit{@next}, \textit{@prev}, \textit{@exclude}, \textit{@select})) (\hyperref[TEI.att.global.analytic]{att.global.analytic} (\textit{@ana})) (\hyperref[TEI.att.global.facs]{att.global.facs} (\textit{@facs})) (\hyperref[TEI.att.global.change]{att.global.change} (\textit{@change})) (\hyperref[TEI.att.global.responsibility]{att.global.responsibility} (\textit{@cert}, \textit{@resp})) (\hyperref[TEI.att.global.source]{att.global.source} (\textit{@source}))
    \item[{Contenu dans}]
  
    \item[core: ]
   \hyperref[TEI.list]{list}
    \item[{Peut contenir}]
  
    \item[analysis: ]
   \hyperref[TEI.c]{c} \hyperref[TEI.cl]{cl} \hyperref[TEI.interp]{interp} \hyperref[TEI.interpGrp]{interpGrp} \hyperref[TEI.m]{m} \hyperref[TEI.pc]{pc} \hyperref[TEI.phr]{phr} \hyperref[TEI.s]{s} \hyperref[TEI.span]{span} \hyperref[TEI.spanGrp]{spanGrp} \hyperref[TEI.w]{w}\par 
    \item[core: ]
   \hyperref[TEI.abbr]{abbr} \hyperref[TEI.add]{add} \hyperref[TEI.address]{address} \hyperref[TEI.binaryObject]{binaryObject} \hyperref[TEI.cb]{cb} \hyperref[TEI.choice]{choice} \hyperref[TEI.corr]{corr} \hyperref[TEI.date]{date} \hyperref[TEI.del]{del} \hyperref[TEI.distinct]{distinct} \hyperref[TEI.email]{email} \hyperref[TEI.emph]{emph} \hyperref[TEI.expan]{expan} \hyperref[TEI.foreign]{foreign} \hyperref[TEI.gap]{gap} \hyperref[TEI.gb]{gb} \hyperref[TEI.gloss]{gloss} \hyperref[TEI.graphic]{graphic} \hyperref[TEI.hi]{hi} \hyperref[TEI.index]{index} \hyperref[TEI.lb]{lb} \hyperref[TEI.measure]{measure} \hyperref[TEI.measureGrp]{measureGrp} \hyperref[TEI.media]{media} \hyperref[TEI.mentioned]{mentioned} \hyperref[TEI.milestone]{milestone} \hyperref[TEI.name]{name} \hyperref[TEI.note]{note} \hyperref[TEI.num]{num} \hyperref[TEI.orig]{orig} \hyperref[TEI.pb]{pb} \hyperref[TEI.ptr]{ptr} \hyperref[TEI.ref]{ref} \hyperref[TEI.reg]{reg} \hyperref[TEI.rs]{rs} \hyperref[TEI.sic]{sic} \hyperref[TEI.soCalled]{soCalled} \hyperref[TEI.term]{term} \hyperref[TEI.time]{time} \hyperref[TEI.title]{title} \hyperref[TEI.unclear]{unclear}\par 
    \item[derived-module-tei.istex: ]
   \hyperref[TEI.math]{math} \hyperref[TEI.mrow]{mrow}\par 
    \item[figures: ]
   \hyperref[TEI.figure]{figure} \hyperref[TEI.formula]{formula} \hyperref[TEI.notatedMusic]{notatedMusic}\par 
    \item[header: ]
   \hyperref[TEI.idno]{idno}\par 
    \item[iso-fs: ]
   \hyperref[TEI.fLib]{fLib} \hyperref[TEI.fs]{fs} \hyperref[TEI.fvLib]{fvLib}\par 
    \item[linking: ]
   \hyperref[TEI.alt]{alt} \hyperref[TEI.altGrp]{altGrp} \hyperref[TEI.anchor]{anchor} \hyperref[TEI.join]{join} \hyperref[TEI.joinGrp]{joinGrp} \hyperref[TEI.link]{link} \hyperref[TEI.linkGrp]{linkGrp} \hyperref[TEI.seg]{seg} \hyperref[TEI.timeline]{timeline}\par 
    \item[msdescription: ]
   \hyperref[TEI.catchwords]{catchwords} \hyperref[TEI.depth]{depth} \hyperref[TEI.dim]{dim} \hyperref[TEI.dimensions]{dimensions} \hyperref[TEI.height]{height} \hyperref[TEI.heraldry]{heraldry} \hyperref[TEI.locus]{locus} \hyperref[TEI.locusGrp]{locusGrp} \hyperref[TEI.material]{material} \hyperref[TEI.objectType]{objectType} \hyperref[TEI.origDate]{origDate} \hyperref[TEI.origPlace]{origPlace} \hyperref[TEI.secFol]{secFol} \hyperref[TEI.signatures]{signatures} \hyperref[TEI.source]{source} \hyperref[TEI.stamp]{stamp} \hyperref[TEI.watermark]{watermark} \hyperref[TEI.width]{width}\par 
    \item[namesdates: ]
   \hyperref[TEI.addName]{addName} \hyperref[TEI.affiliation]{affiliation} \hyperref[TEI.country]{country} \hyperref[TEI.forename]{forename} \hyperref[TEI.genName]{genName} \hyperref[TEI.geogName]{geogName} \hyperref[TEI.location]{location} \hyperref[TEI.nameLink]{nameLink} \hyperref[TEI.orgName]{orgName} \hyperref[TEI.persName]{persName} \hyperref[TEI.placeName]{placeName} \hyperref[TEI.region]{region} \hyperref[TEI.roleName]{roleName} \hyperref[TEI.settlement]{settlement} \hyperref[TEI.state]{state} \hyperref[TEI.surname]{surname}\par 
    \item[spoken: ]
   \hyperref[TEI.annotationBlock]{annotationBlock}\par 
    \item[transcr: ]
   \hyperref[TEI.addSpan]{addSpan} \hyperref[TEI.am]{am} \hyperref[TEI.damage]{damage} \hyperref[TEI.damageSpan]{damageSpan} \hyperref[TEI.delSpan]{delSpan} \hyperref[TEI.ex]{ex} \hyperref[TEI.fw]{fw} \hyperref[TEI.handShift]{handShift} \hyperref[TEI.listTranspose]{listTranspose} \hyperref[TEI.metamark]{metamark} \hyperref[TEI.mod]{mod} \hyperref[TEI.redo]{redo} \hyperref[TEI.restore]{restore} \hyperref[TEI.retrace]{retrace} \hyperref[TEI.secl]{secl} \hyperref[TEI.space]{space} \hyperref[TEI.subst]{subst} \hyperref[TEI.substJoin]{substJoin} \hyperref[TEI.supplied]{supplied} \hyperref[TEI.surplus]{surplus} \hyperref[TEI.undo]{undo}\par des données textuelles
    \item[{Note}]
  \par
L'élément \hyperref[TEI.headItem]{<headItem>} est utilisé uniquement si chacun des items d'une liste est précédé d'un élément \hyperref[TEI.label]{<label>}.
    \item[{Exemple}]
  \leavevmode\bgroup\exampleFont \begin{shaded}\noindent\mbox{} Parlez-vous\mbox{}\newline 
 anglosnob? Liste de quelques mots franglais et des propositions pour les remplacer : : {<\textbf{list}\hspace*{6pt}{type}="{gloss}">}\mbox{}\newline 
\hspace*{6pt}{<\textbf{headLabel}\hspace*{6pt}{rend}="{smallcaps}">}Ne dites pas{</\textbf{headLabel}>}\mbox{}\newline 
\hspace*{6pt}{<\textbf{headItem}\hspace*{6pt}{rend}="{smallcaps}">}Mais dites...{</\textbf{headItem}>}\mbox{}\newline 
\hspace*{6pt}{<\textbf{label}>}abstract {</\textbf{label}>}\mbox{}\newline 
\hspace*{6pt}{<\textbf{item}>} résumé, abrégé {</\textbf{item}>}\mbox{}\newline 
\hspace*{6pt}{<\textbf{label}>}baby-boom {</\textbf{label}>}\mbox{}\newline 
\hspace*{6pt}{<\textbf{item}>}printemps démographique {</\textbf{item}>}\mbox{}\newline 
\hspace*{6pt}{<\textbf{label}>}carjacking {</\textbf{label}>}\mbox{}\newline 
\hspace*{6pt}{<\textbf{item}>}dévoituration (comme défenestration), dévoiturage(comme cambriolage, braquage) {</\textbf{item}>}\mbox{}\newline 
\hspace*{6pt}{<\textbf{label}>}bug {</\textbf{label}>}\mbox{}\newline 
\hspace*{6pt}{<\textbf{item}>}erreur, défaut, insecte, ("bogue" est inutile) {</\textbf{item}>}\mbox{}\newline 
\hspace*{6pt}{<\textbf{label}>}mixer{</\textbf{label}>}\mbox{}\newline 
\hspace*{6pt}{<\textbf{item}>}mélanger (sauf si c'est avec un mixeur){</\textbf{item}>}\mbox{}\newline 
{</\textbf{list}>}\end{shaded}\egroup 


    \item[{Modèle de contenu}]
  \mbox{}\hfill\\[-10pt]\begin{Verbatim}[fontsize=\small]
<content>
 <macroRef key="macro.phraseSeq"/>
</content>
    
\end{Verbatim}

    \item[{Schéma Declaration}]
  \mbox{}\hfill\\[-10pt]\begin{Verbatim}[fontsize=\small]
element headItem { tei_att.global.attributes, tei_macro.phraseSeq }
\end{Verbatim}

\end{reflist}  \index{headLabel=<headLabel>|oddindex}
\begin{reflist}
\item[]\begin{specHead}{TEI.headLabel}{<headLabel> }(intitulé d'une liste d'étiquettes) contient l'intitulé pour la colonne d'étiquettes ou de termes dans un glossaire ou dans une liste structurée de la même manière. [\xref{http://www.tei-c.org/release/doc/tei-p5-doc/en/html/CO.html\#COLI}{3.7. Lists}]\end{specHead} 
    \item[{Module}]
  core
    \item[{Attributs}]
  Attributs \hyperref[TEI.att.global]{att.global} (\textit{@xml:id}, \textit{@n}, \textit{@xml:lang}, \textit{@xml:base}, \textit{@xml:space})  (\hyperref[TEI.att.global.rendition]{att.global.rendition} (\textit{@rend}, \textit{@style}, \textit{@rendition})) (\hyperref[TEI.att.global.linking]{att.global.linking} (\textit{@corresp}, \textit{@synch}, \textit{@sameAs}, \textit{@copyOf}, \textit{@next}, \textit{@prev}, \textit{@exclude}, \textit{@select})) (\hyperref[TEI.att.global.analytic]{att.global.analytic} (\textit{@ana})) (\hyperref[TEI.att.global.facs]{att.global.facs} (\textit{@facs})) (\hyperref[TEI.att.global.change]{att.global.change} (\textit{@change})) (\hyperref[TEI.att.global.responsibility]{att.global.responsibility} (\textit{@cert}, \textit{@resp})) (\hyperref[TEI.att.global.source]{att.global.source} (\textit{@source}))
    \item[{Contenu dans}]
  
    \item[core: ]
   \hyperref[TEI.list]{list}
    \item[{Peut contenir}]
  
    \item[analysis: ]
   \hyperref[TEI.c]{c} \hyperref[TEI.cl]{cl} \hyperref[TEI.interp]{interp} \hyperref[TEI.interpGrp]{interpGrp} \hyperref[TEI.m]{m} \hyperref[TEI.pc]{pc} \hyperref[TEI.phr]{phr} \hyperref[TEI.s]{s} \hyperref[TEI.span]{span} \hyperref[TEI.spanGrp]{spanGrp} \hyperref[TEI.w]{w}\par 
    \item[core: ]
   \hyperref[TEI.abbr]{abbr} \hyperref[TEI.add]{add} \hyperref[TEI.address]{address} \hyperref[TEI.binaryObject]{binaryObject} \hyperref[TEI.cb]{cb} \hyperref[TEI.choice]{choice} \hyperref[TEI.corr]{corr} \hyperref[TEI.date]{date} \hyperref[TEI.del]{del} \hyperref[TEI.distinct]{distinct} \hyperref[TEI.email]{email} \hyperref[TEI.emph]{emph} \hyperref[TEI.expan]{expan} \hyperref[TEI.foreign]{foreign} \hyperref[TEI.gap]{gap} \hyperref[TEI.gb]{gb} \hyperref[TEI.gloss]{gloss} \hyperref[TEI.graphic]{graphic} \hyperref[TEI.hi]{hi} \hyperref[TEI.index]{index} \hyperref[TEI.lb]{lb} \hyperref[TEI.measure]{measure} \hyperref[TEI.measureGrp]{measureGrp} \hyperref[TEI.media]{media} \hyperref[TEI.mentioned]{mentioned} \hyperref[TEI.milestone]{milestone} \hyperref[TEI.name]{name} \hyperref[TEI.note]{note} \hyperref[TEI.num]{num} \hyperref[TEI.orig]{orig} \hyperref[TEI.pb]{pb} \hyperref[TEI.ptr]{ptr} \hyperref[TEI.ref]{ref} \hyperref[TEI.reg]{reg} \hyperref[TEI.rs]{rs} \hyperref[TEI.sic]{sic} \hyperref[TEI.soCalled]{soCalled} \hyperref[TEI.term]{term} \hyperref[TEI.time]{time} \hyperref[TEI.title]{title} \hyperref[TEI.unclear]{unclear}\par 
    \item[derived-module-tei.istex: ]
   \hyperref[TEI.math]{math} \hyperref[TEI.mrow]{mrow}\par 
    \item[figures: ]
   \hyperref[TEI.figure]{figure} \hyperref[TEI.formula]{formula} \hyperref[TEI.notatedMusic]{notatedMusic}\par 
    \item[header: ]
   \hyperref[TEI.idno]{idno}\par 
    \item[iso-fs: ]
   \hyperref[TEI.fLib]{fLib} \hyperref[TEI.fs]{fs} \hyperref[TEI.fvLib]{fvLib}\par 
    \item[linking: ]
   \hyperref[TEI.alt]{alt} \hyperref[TEI.altGrp]{altGrp} \hyperref[TEI.anchor]{anchor} \hyperref[TEI.join]{join} \hyperref[TEI.joinGrp]{joinGrp} \hyperref[TEI.link]{link} \hyperref[TEI.linkGrp]{linkGrp} \hyperref[TEI.seg]{seg} \hyperref[TEI.timeline]{timeline}\par 
    \item[msdescription: ]
   \hyperref[TEI.catchwords]{catchwords} \hyperref[TEI.depth]{depth} \hyperref[TEI.dim]{dim} \hyperref[TEI.dimensions]{dimensions} \hyperref[TEI.height]{height} \hyperref[TEI.heraldry]{heraldry} \hyperref[TEI.locus]{locus} \hyperref[TEI.locusGrp]{locusGrp} \hyperref[TEI.material]{material} \hyperref[TEI.objectType]{objectType} \hyperref[TEI.origDate]{origDate} \hyperref[TEI.origPlace]{origPlace} \hyperref[TEI.secFol]{secFol} \hyperref[TEI.signatures]{signatures} \hyperref[TEI.source]{source} \hyperref[TEI.stamp]{stamp} \hyperref[TEI.watermark]{watermark} \hyperref[TEI.width]{width}\par 
    \item[namesdates: ]
   \hyperref[TEI.addName]{addName} \hyperref[TEI.affiliation]{affiliation} \hyperref[TEI.country]{country} \hyperref[TEI.forename]{forename} \hyperref[TEI.genName]{genName} \hyperref[TEI.geogName]{geogName} \hyperref[TEI.location]{location} \hyperref[TEI.nameLink]{nameLink} \hyperref[TEI.orgName]{orgName} \hyperref[TEI.persName]{persName} \hyperref[TEI.placeName]{placeName} \hyperref[TEI.region]{region} \hyperref[TEI.roleName]{roleName} \hyperref[TEI.settlement]{settlement} \hyperref[TEI.state]{state} \hyperref[TEI.surname]{surname}\par 
    \item[spoken: ]
   \hyperref[TEI.annotationBlock]{annotationBlock}\par 
    \item[transcr: ]
   \hyperref[TEI.addSpan]{addSpan} \hyperref[TEI.am]{am} \hyperref[TEI.damage]{damage} \hyperref[TEI.damageSpan]{damageSpan} \hyperref[TEI.delSpan]{delSpan} \hyperref[TEI.ex]{ex} \hyperref[TEI.fw]{fw} \hyperref[TEI.handShift]{handShift} \hyperref[TEI.listTranspose]{listTranspose} \hyperref[TEI.metamark]{metamark} \hyperref[TEI.mod]{mod} \hyperref[TEI.redo]{redo} \hyperref[TEI.restore]{restore} \hyperref[TEI.retrace]{retrace} \hyperref[TEI.secl]{secl} \hyperref[TEI.space]{space} \hyperref[TEI.subst]{subst} \hyperref[TEI.substJoin]{substJoin} \hyperref[TEI.supplied]{supplied} \hyperref[TEI.surplus]{surplus} \hyperref[TEI.undo]{undo}\par des données textuelles
    \item[{Note}]
  \par
L'élément \hyperref[TEI.headLabel]{<headLabel>} ne peut apparaître que si chaque item de la liste est précédé d'un \hyperref[TEI.label]{<label>}.
    \item[{Exemple}]
  \leavevmode\bgroup\exampleFont \begin{shaded}\noindent\mbox{} Parlez-vous\mbox{}\newline 
 anglosnob? Liste de quelques mots franglais et des propositions pour les remplacer : {<\textbf{list}\hspace*{6pt}{type}="{gloss}">}\mbox{}\newline 
\hspace*{6pt}{<\textbf{headLabel}\hspace*{6pt}{rend}="{smallcaps}">}Ne dites pas{</\textbf{headLabel}>}\mbox{}\newline 
\hspace*{6pt}{<\textbf{headItem}\hspace*{6pt}{rend}="{smallcaps}">}Mais dites...{</\textbf{headItem}>}\mbox{}\newline 
\hspace*{6pt}{<\textbf{label}>}abstract {</\textbf{label}>}\mbox{}\newline 
\hspace*{6pt}{<\textbf{item}>} résumé, abrégé {</\textbf{item}>}\mbox{}\newline 
\hspace*{6pt}{<\textbf{label}>}baby-boom {</\textbf{label}>}\mbox{}\newline 
\hspace*{6pt}{<\textbf{item}>}printemps démographique {</\textbf{item}>}\mbox{}\newline 
\hspace*{6pt}{<\textbf{label}>}carjacking {</\textbf{label}>}\mbox{}\newline 
\hspace*{6pt}{<\textbf{item}>}dévoituration (comme défenestration), dévoiturage(comme cambriolage, braquage) {</\textbf{item}>}\mbox{}\newline 
\hspace*{6pt}{<\textbf{label}>}bug {</\textbf{label}>}\mbox{}\newline 
\hspace*{6pt}{<\textbf{item}>}erreur, défaut, insecte, ("bogue" est inutile) {</\textbf{item}>}\mbox{}\newline 
\hspace*{6pt}{<\textbf{label}>}mixer{</\textbf{label}>}\mbox{}\newline 
\hspace*{6pt}{<\textbf{item}>}mélanger (sauf si c'est avec un mixeur){</\textbf{item}>}\mbox{}\newline 
{</\textbf{list}>}\end{shaded}\egroup 


    \item[{Modèle de contenu}]
  \mbox{}\hfill\\[-10pt]\begin{Verbatim}[fontsize=\small]
<content>
 <macroRef key="macro.phraseSeq"/>
</content>
    
\end{Verbatim}

    \item[{Schéma Declaration}]
  \mbox{}\hfill\\[-10pt]\begin{Verbatim}[fontsize=\small]
element headLabel { tei_att.global.attributes, tei_macro.phraseSeq }
\end{Verbatim}

\end{reflist}  \index{height=<height>|oddindex}
\begin{reflist}
\item[]\begin{specHead}{TEI.height}{<height> }(hauteur) contient une dimension prise sur l'axe parallèle au dos du manuscrit. [\xref{http://www.tei-c.org/release/doc/tei-p5-doc/en/html/MS.html\#msdim}{10.3.4. Dimensions}]\end{specHead} 
    \item[{Module}]
  msdescription
    \item[{Attributs}]
  Attributs \hyperref[TEI.att.global]{att.global} (\textit{@xml:id}, \textit{@n}, \textit{@xml:lang}, \textit{@xml:base}, \textit{@xml:space})  (\hyperref[TEI.att.global.rendition]{att.global.rendition} (\textit{@rend}, \textit{@style}, \textit{@rendition})) (\hyperref[TEI.att.global.linking]{att.global.linking} (\textit{@corresp}, \textit{@synch}, \textit{@sameAs}, \textit{@copyOf}, \textit{@next}, \textit{@prev}, \textit{@exclude}, \textit{@select})) (\hyperref[TEI.att.global.analytic]{att.global.analytic} (\textit{@ana})) (\hyperref[TEI.att.global.facs]{att.global.facs} (\textit{@facs})) (\hyperref[TEI.att.global.change]{att.global.change} (\textit{@change})) (\hyperref[TEI.att.global.responsibility]{att.global.responsibility} (\textit{@cert}, \textit{@resp})) (\hyperref[TEI.att.global.source]{att.global.source} (\textit{@source})) \hyperref[TEI.att.dimensions]{att.dimensions} (\textit{@unit}, \textit{@quantity}, \textit{@extent}, \textit{@precision}, \textit{@scope})  (\hyperref[TEI.att.ranging]{att.ranging} (\textit{@atLeast}, \textit{@atMost}, \textit{@min}, \textit{@max}, \textit{@confidence}))
    \item[{Membre du}]
  \hyperref[TEI.model.dimLike]{model.dimLike} \hyperref[TEI.model.measureLike]{model.measureLike}Elément: \begin{itemize}
\item \hyperref[TEI.mpadded]{mpadded}/@height
\end{itemize} 
    \item[{Contenu dans}]
  
    \item[analysis: ]
   \hyperref[TEI.cl]{cl} \hyperref[TEI.phr]{phr} \hyperref[TEI.s]{s} \hyperref[TEI.span]{span}\par 
    \item[core: ]
   \hyperref[TEI.abbr]{abbr} \hyperref[TEI.add]{add} \hyperref[TEI.addrLine]{addrLine} \hyperref[TEI.author]{author} \hyperref[TEI.bibl]{bibl} \hyperref[TEI.biblScope]{biblScope} \hyperref[TEI.citedRange]{citedRange} \hyperref[TEI.corr]{corr} \hyperref[TEI.date]{date} \hyperref[TEI.del]{del} \hyperref[TEI.desc]{desc} \hyperref[TEI.distinct]{distinct} \hyperref[TEI.editor]{editor} \hyperref[TEI.email]{email} \hyperref[TEI.emph]{emph} \hyperref[TEI.expan]{expan} \hyperref[TEI.foreign]{foreign} \hyperref[TEI.gloss]{gloss} \hyperref[TEI.head]{head} \hyperref[TEI.headItem]{headItem} \hyperref[TEI.headLabel]{headLabel} \hyperref[TEI.hi]{hi} \hyperref[TEI.item]{item} \hyperref[TEI.l]{l} \hyperref[TEI.label]{label} \hyperref[TEI.measure]{measure} \hyperref[TEI.measureGrp]{measureGrp} \hyperref[TEI.meeting]{meeting} \hyperref[TEI.mentioned]{mentioned} \hyperref[TEI.name]{name} \hyperref[TEI.note]{note} \hyperref[TEI.num]{num} \hyperref[TEI.orig]{orig} \hyperref[TEI.p]{p} \hyperref[TEI.pubPlace]{pubPlace} \hyperref[TEI.publisher]{publisher} \hyperref[TEI.q]{q} \hyperref[TEI.quote]{quote} \hyperref[TEI.ref]{ref} \hyperref[TEI.reg]{reg} \hyperref[TEI.resp]{resp} \hyperref[TEI.rs]{rs} \hyperref[TEI.said]{said} \hyperref[TEI.sic]{sic} \hyperref[TEI.soCalled]{soCalled} \hyperref[TEI.speaker]{speaker} \hyperref[TEI.stage]{stage} \hyperref[TEI.street]{street} \hyperref[TEI.term]{term} \hyperref[TEI.textLang]{textLang} \hyperref[TEI.time]{time} \hyperref[TEI.title]{title} \hyperref[TEI.unclear]{unclear}\par 
    \item[figures: ]
   \hyperref[TEI.cell]{cell} \hyperref[TEI.figDesc]{figDesc}\par 
    \item[header: ]
   \hyperref[TEI.authority]{authority} \hyperref[TEI.change]{change} \hyperref[TEI.classCode]{classCode} \hyperref[TEI.creation]{creation} \hyperref[TEI.distributor]{distributor} \hyperref[TEI.edition]{edition} \hyperref[TEI.extent]{extent} \hyperref[TEI.funder]{funder} \hyperref[TEI.language]{language} \hyperref[TEI.licence]{licence} \hyperref[TEI.rendition]{rendition}\par 
    \item[iso-fs: ]
   \hyperref[TEI.fDescr]{fDescr} \hyperref[TEI.fsDescr]{fsDescr}\par 
    \item[linking: ]
   \hyperref[TEI.ab]{ab} \hyperref[TEI.seg]{seg}\par 
    \item[msdescription: ]
   \hyperref[TEI.accMat]{accMat} \hyperref[TEI.acquisition]{acquisition} \hyperref[TEI.additions]{additions} \hyperref[TEI.catchwords]{catchwords} \hyperref[TEI.collation]{collation} \hyperref[TEI.colophon]{colophon} \hyperref[TEI.condition]{condition} \hyperref[TEI.custEvent]{custEvent} \hyperref[TEI.decoNote]{decoNote} \hyperref[TEI.dimensions]{dimensions} \hyperref[TEI.explicit]{explicit} \hyperref[TEI.filiation]{filiation} \hyperref[TEI.finalRubric]{finalRubric} \hyperref[TEI.foliation]{foliation} \hyperref[TEI.heraldry]{heraldry} \hyperref[TEI.incipit]{incipit} \hyperref[TEI.layout]{layout} \hyperref[TEI.material]{material} \hyperref[TEI.musicNotation]{musicNotation} \hyperref[TEI.objectType]{objectType} \hyperref[TEI.origDate]{origDate} \hyperref[TEI.origPlace]{origPlace} \hyperref[TEI.origin]{origin} \hyperref[TEI.provenance]{provenance} \hyperref[TEI.rubric]{rubric} \hyperref[TEI.secFol]{secFol} \hyperref[TEI.signatures]{signatures} \hyperref[TEI.source]{source} \hyperref[TEI.stamp]{stamp} \hyperref[TEI.summary]{summary} \hyperref[TEI.support]{support} \hyperref[TEI.surrogates]{surrogates} \hyperref[TEI.typeNote]{typeNote} \hyperref[TEI.watermark]{watermark}\par 
    \item[namesdates: ]
   \hyperref[TEI.addName]{addName} \hyperref[TEI.affiliation]{affiliation} \hyperref[TEI.country]{country} \hyperref[TEI.forename]{forename} \hyperref[TEI.genName]{genName} \hyperref[TEI.geogName]{geogName} \hyperref[TEI.location]{location} \hyperref[TEI.nameLink]{nameLink} \hyperref[TEI.orgName]{orgName} \hyperref[TEI.persName]{persName} \hyperref[TEI.placeName]{placeName} \hyperref[TEI.region]{region} \hyperref[TEI.roleName]{roleName} \hyperref[TEI.settlement]{settlement} \hyperref[TEI.surname]{surname}\par 
    \item[textstructure: ]
   \hyperref[TEI.docAuthor]{docAuthor} \hyperref[TEI.docDate]{docDate} \hyperref[TEI.docEdition]{docEdition} \hyperref[TEI.titlePart]{titlePart}\par 
    \item[transcr: ]
   \hyperref[TEI.damage]{damage} \hyperref[TEI.fw]{fw} \hyperref[TEI.metamark]{metamark} \hyperref[TEI.mod]{mod} \hyperref[TEI.restore]{restore} \hyperref[TEI.retrace]{retrace} \hyperref[TEI.secl]{secl} \hyperref[TEI.supplied]{supplied} \hyperref[TEI.surplus]{surplus}
    \item[{Peut contenir}]
  Des données textuelles uniquement
    \item[{Note}]
  \par
If used to specify the height of a non text-bearing portion of some object, for example a monument, this element conventionally refers to the axis perpendicular to the surface of the earth.
    \item[{Exemple}]
  \leavevmode\bgroup\exampleFont \begin{shaded}\noindent\mbox{}{<\textbf{height}\hspace*{6pt}{quantity}="{7}"\hspace*{6pt}{unit}="{in}"/>}\end{shaded}\egroup 


    \item[{Exemple}]
  \leavevmode\bgroup\exampleFont \begin{shaded}\noindent\mbox{}{<\textbf{height}\hspace*{6pt}{quantity}="{7}"\hspace*{6pt}{unit}="{in}"/>}\end{shaded}\egroup 


    \item[{Modèle de contenu}]
  \fbox{\ttfamily <content>\newline
 <macroRef key="macro.xtext"/>\newline
</content>\newline
    } 
    \item[{Schéma Declaration}]
  \mbox{}\hfill\\[-10pt]\begin{Verbatim}[fontsize=\small]
element height
{
   tei_att.global.attributes,
   tei_att.dimensions.attributes,
   tei_macro.xtext}
\end{Verbatim}

\end{reflist}  \index{heraldry=<heraldry>|oddindex}
\begin{reflist}
\item[]\begin{specHead}{TEI.heraldry}{<heraldry> }(héraldique) contient une devise ou une formule d'héraldique, comme celles qu'on trouve sur un blason, des armoiries, etc. [\xref{http://www.tei-c.org/release/doc/tei-p5-doc/en/html/MS.html\#mshera}{10.3.8. Heraldry}]\end{specHead} 
    \item[{Module}]
  msdescription
    \item[{Attributs}]
  Attributs \hyperref[TEI.att.global]{att.global} (\textit{@xml:id}, \textit{@n}, \textit{@xml:lang}, \textit{@xml:base}, \textit{@xml:space})  (\hyperref[TEI.att.global.rendition]{att.global.rendition} (\textit{@rend}, \textit{@style}, \textit{@rendition})) (\hyperref[TEI.att.global.linking]{att.global.linking} (\textit{@corresp}, \textit{@synch}, \textit{@sameAs}, \textit{@copyOf}, \textit{@next}, \textit{@prev}, \textit{@exclude}, \textit{@select})) (\hyperref[TEI.att.global.analytic]{att.global.analytic} (\textit{@ana})) (\hyperref[TEI.att.global.facs]{att.global.facs} (\textit{@facs})) (\hyperref[TEI.att.global.change]{att.global.change} (\textit{@change})) (\hyperref[TEI.att.global.responsibility]{att.global.responsibility} (\textit{@cert}, \textit{@resp})) (\hyperref[TEI.att.global.source]{att.global.source} (\textit{@source}))
    \item[{Membre du}]
  \hyperref[TEI.model.pPart.msdesc]{model.pPart.msdesc}
    \item[{Contenu dans}]
  
    \item[analysis: ]
   \hyperref[TEI.cl]{cl} \hyperref[TEI.phr]{phr} \hyperref[TEI.s]{s} \hyperref[TEI.span]{span}\par 
    \item[core: ]
   \hyperref[TEI.abbr]{abbr} \hyperref[TEI.add]{add} \hyperref[TEI.addrLine]{addrLine} \hyperref[TEI.author]{author} \hyperref[TEI.biblScope]{biblScope} \hyperref[TEI.citedRange]{citedRange} \hyperref[TEI.corr]{corr} \hyperref[TEI.date]{date} \hyperref[TEI.del]{del} \hyperref[TEI.desc]{desc} \hyperref[TEI.distinct]{distinct} \hyperref[TEI.editor]{editor} \hyperref[TEI.email]{email} \hyperref[TEI.emph]{emph} \hyperref[TEI.expan]{expan} \hyperref[TEI.foreign]{foreign} \hyperref[TEI.gloss]{gloss} \hyperref[TEI.head]{head} \hyperref[TEI.headItem]{headItem} \hyperref[TEI.headLabel]{headLabel} \hyperref[TEI.hi]{hi} \hyperref[TEI.item]{item} \hyperref[TEI.l]{l} \hyperref[TEI.label]{label} \hyperref[TEI.measure]{measure} \hyperref[TEI.meeting]{meeting} \hyperref[TEI.mentioned]{mentioned} \hyperref[TEI.name]{name} \hyperref[TEI.note]{note} \hyperref[TEI.num]{num} \hyperref[TEI.orig]{orig} \hyperref[TEI.p]{p} \hyperref[TEI.pubPlace]{pubPlace} \hyperref[TEI.publisher]{publisher} \hyperref[TEI.q]{q} \hyperref[TEI.quote]{quote} \hyperref[TEI.ref]{ref} \hyperref[TEI.reg]{reg} \hyperref[TEI.resp]{resp} \hyperref[TEI.rs]{rs} \hyperref[TEI.said]{said} \hyperref[TEI.sic]{sic} \hyperref[TEI.soCalled]{soCalled} \hyperref[TEI.speaker]{speaker} \hyperref[TEI.stage]{stage} \hyperref[TEI.street]{street} \hyperref[TEI.term]{term} \hyperref[TEI.textLang]{textLang} \hyperref[TEI.time]{time} \hyperref[TEI.title]{title} \hyperref[TEI.unclear]{unclear}\par 
    \item[figures: ]
   \hyperref[TEI.cell]{cell} \hyperref[TEI.figDesc]{figDesc}\par 
    \item[header: ]
   \hyperref[TEI.authority]{authority} \hyperref[TEI.change]{change} \hyperref[TEI.classCode]{classCode} \hyperref[TEI.creation]{creation} \hyperref[TEI.distributor]{distributor} \hyperref[TEI.edition]{edition} \hyperref[TEI.extent]{extent} \hyperref[TEI.funder]{funder} \hyperref[TEI.language]{language} \hyperref[TEI.licence]{licence} \hyperref[TEI.rendition]{rendition}\par 
    \item[iso-fs: ]
   \hyperref[TEI.fDescr]{fDescr} \hyperref[TEI.fsDescr]{fsDescr}\par 
    \item[linking: ]
   \hyperref[TEI.ab]{ab} \hyperref[TEI.seg]{seg}\par 
    \item[msdescription: ]
   \hyperref[TEI.accMat]{accMat} \hyperref[TEI.acquisition]{acquisition} \hyperref[TEI.additions]{additions} \hyperref[TEI.catchwords]{catchwords} \hyperref[TEI.collation]{collation} \hyperref[TEI.colophon]{colophon} \hyperref[TEI.condition]{condition} \hyperref[TEI.custEvent]{custEvent} \hyperref[TEI.decoNote]{decoNote} \hyperref[TEI.explicit]{explicit} \hyperref[TEI.filiation]{filiation} \hyperref[TEI.finalRubric]{finalRubric} \hyperref[TEI.foliation]{foliation} \hyperref[TEI.heraldry]{heraldry} \hyperref[TEI.incipit]{incipit} \hyperref[TEI.layout]{layout} \hyperref[TEI.material]{material} \hyperref[TEI.musicNotation]{musicNotation} \hyperref[TEI.objectType]{objectType} \hyperref[TEI.origDate]{origDate} \hyperref[TEI.origPlace]{origPlace} \hyperref[TEI.origin]{origin} \hyperref[TEI.provenance]{provenance} \hyperref[TEI.rubric]{rubric} \hyperref[TEI.secFol]{secFol} \hyperref[TEI.signatures]{signatures} \hyperref[TEI.source]{source} \hyperref[TEI.stamp]{stamp} \hyperref[TEI.summary]{summary} \hyperref[TEI.support]{support} \hyperref[TEI.surrogates]{surrogates} \hyperref[TEI.typeNote]{typeNote} \hyperref[TEI.watermark]{watermark}\par 
    \item[namesdates: ]
   \hyperref[TEI.addName]{addName} \hyperref[TEI.affiliation]{affiliation} \hyperref[TEI.country]{country} \hyperref[TEI.forename]{forename} \hyperref[TEI.genName]{genName} \hyperref[TEI.geogName]{geogName} \hyperref[TEI.nameLink]{nameLink} \hyperref[TEI.orgName]{orgName} \hyperref[TEI.persName]{persName} \hyperref[TEI.placeName]{placeName} \hyperref[TEI.region]{region} \hyperref[TEI.roleName]{roleName} \hyperref[TEI.settlement]{settlement} \hyperref[TEI.surname]{surname}\par 
    \item[textstructure: ]
   \hyperref[TEI.docAuthor]{docAuthor} \hyperref[TEI.docDate]{docDate} \hyperref[TEI.docEdition]{docEdition} \hyperref[TEI.titlePart]{titlePart}\par 
    \item[transcr: ]
   \hyperref[TEI.damage]{damage} \hyperref[TEI.fw]{fw} \hyperref[TEI.metamark]{metamark} \hyperref[TEI.mod]{mod} \hyperref[TEI.restore]{restore} \hyperref[TEI.retrace]{retrace} \hyperref[TEI.secl]{secl} \hyperref[TEI.supplied]{supplied} \hyperref[TEI.surplus]{surplus}
    \item[{Peut contenir}]
  
    \item[analysis: ]
   \hyperref[TEI.c]{c} \hyperref[TEI.cl]{cl} \hyperref[TEI.interp]{interp} \hyperref[TEI.interpGrp]{interpGrp} \hyperref[TEI.m]{m} \hyperref[TEI.pc]{pc} \hyperref[TEI.phr]{phr} \hyperref[TEI.s]{s} \hyperref[TEI.span]{span} \hyperref[TEI.spanGrp]{spanGrp} \hyperref[TEI.w]{w}\par 
    \item[core: ]
   \hyperref[TEI.abbr]{abbr} \hyperref[TEI.add]{add} \hyperref[TEI.address]{address} \hyperref[TEI.binaryObject]{binaryObject} \hyperref[TEI.cb]{cb} \hyperref[TEI.choice]{choice} \hyperref[TEI.corr]{corr} \hyperref[TEI.date]{date} \hyperref[TEI.del]{del} \hyperref[TEI.distinct]{distinct} \hyperref[TEI.email]{email} \hyperref[TEI.emph]{emph} \hyperref[TEI.expan]{expan} \hyperref[TEI.foreign]{foreign} \hyperref[TEI.gap]{gap} \hyperref[TEI.gb]{gb} \hyperref[TEI.gloss]{gloss} \hyperref[TEI.graphic]{graphic} \hyperref[TEI.hi]{hi} \hyperref[TEI.index]{index} \hyperref[TEI.lb]{lb} \hyperref[TEI.measure]{measure} \hyperref[TEI.measureGrp]{measureGrp} \hyperref[TEI.media]{media} \hyperref[TEI.mentioned]{mentioned} \hyperref[TEI.milestone]{milestone} \hyperref[TEI.name]{name} \hyperref[TEI.note]{note} \hyperref[TEI.num]{num} \hyperref[TEI.orig]{orig} \hyperref[TEI.pb]{pb} \hyperref[TEI.ptr]{ptr} \hyperref[TEI.ref]{ref} \hyperref[TEI.reg]{reg} \hyperref[TEI.rs]{rs} \hyperref[TEI.sic]{sic} \hyperref[TEI.soCalled]{soCalled} \hyperref[TEI.term]{term} \hyperref[TEI.time]{time} \hyperref[TEI.title]{title} \hyperref[TEI.unclear]{unclear}\par 
    \item[derived-module-tei.istex: ]
   \hyperref[TEI.math]{math} \hyperref[TEI.mrow]{mrow}\par 
    \item[figures: ]
   \hyperref[TEI.figure]{figure} \hyperref[TEI.formula]{formula} \hyperref[TEI.notatedMusic]{notatedMusic}\par 
    \item[header: ]
   \hyperref[TEI.idno]{idno}\par 
    \item[iso-fs: ]
   \hyperref[TEI.fLib]{fLib} \hyperref[TEI.fs]{fs} \hyperref[TEI.fvLib]{fvLib}\par 
    \item[linking: ]
   \hyperref[TEI.alt]{alt} \hyperref[TEI.altGrp]{altGrp} \hyperref[TEI.anchor]{anchor} \hyperref[TEI.join]{join} \hyperref[TEI.joinGrp]{joinGrp} \hyperref[TEI.link]{link} \hyperref[TEI.linkGrp]{linkGrp} \hyperref[TEI.seg]{seg} \hyperref[TEI.timeline]{timeline}\par 
    \item[msdescription: ]
   \hyperref[TEI.catchwords]{catchwords} \hyperref[TEI.depth]{depth} \hyperref[TEI.dim]{dim} \hyperref[TEI.dimensions]{dimensions} \hyperref[TEI.height]{height} \hyperref[TEI.heraldry]{heraldry} \hyperref[TEI.locus]{locus} \hyperref[TEI.locusGrp]{locusGrp} \hyperref[TEI.material]{material} \hyperref[TEI.objectType]{objectType} \hyperref[TEI.origDate]{origDate} \hyperref[TEI.origPlace]{origPlace} \hyperref[TEI.secFol]{secFol} \hyperref[TEI.signatures]{signatures} \hyperref[TEI.source]{source} \hyperref[TEI.stamp]{stamp} \hyperref[TEI.watermark]{watermark} \hyperref[TEI.width]{width}\par 
    \item[namesdates: ]
   \hyperref[TEI.addName]{addName} \hyperref[TEI.affiliation]{affiliation} \hyperref[TEI.country]{country} \hyperref[TEI.forename]{forename} \hyperref[TEI.genName]{genName} \hyperref[TEI.geogName]{geogName} \hyperref[TEI.location]{location} \hyperref[TEI.nameLink]{nameLink} \hyperref[TEI.orgName]{orgName} \hyperref[TEI.persName]{persName} \hyperref[TEI.placeName]{placeName} \hyperref[TEI.region]{region} \hyperref[TEI.roleName]{roleName} \hyperref[TEI.settlement]{settlement} \hyperref[TEI.state]{state} \hyperref[TEI.surname]{surname}\par 
    \item[spoken: ]
   \hyperref[TEI.annotationBlock]{annotationBlock}\par 
    \item[transcr: ]
   \hyperref[TEI.addSpan]{addSpan} \hyperref[TEI.am]{am} \hyperref[TEI.damage]{damage} \hyperref[TEI.damageSpan]{damageSpan} \hyperref[TEI.delSpan]{delSpan} \hyperref[TEI.ex]{ex} \hyperref[TEI.fw]{fw} \hyperref[TEI.handShift]{handShift} \hyperref[TEI.listTranspose]{listTranspose} \hyperref[TEI.metamark]{metamark} \hyperref[TEI.mod]{mod} \hyperref[TEI.redo]{redo} \hyperref[TEI.restore]{restore} \hyperref[TEI.retrace]{retrace} \hyperref[TEI.secl]{secl} \hyperref[TEI.space]{space} \hyperref[TEI.subst]{subst} \hyperref[TEI.substJoin]{substJoin} \hyperref[TEI.supplied]{supplied} \hyperref[TEI.surplus]{surplus} \hyperref[TEI.undo]{undo}\par des données textuelles
    \item[{Exemple}]
  \leavevmode\bgroup\exampleFont \begin{shaded}\noindent\mbox{}{<\textbf{p}>}Barbey, en Normandie : {<\textbf{heraldry}>}d'azur, à deux bars adossés d'argent ; au chef cousu de\mbox{}\newline 
\hspace*{6pt}\hspace*{6pt} gueules, chargé de trois tourteaux d'or.{</\textbf{heraldry}>}.{</\textbf{p}>}\end{shaded}\egroup 


    \item[{Modèle de contenu}]
  \mbox{}\hfill\\[-10pt]\begin{Verbatim}[fontsize=\small]
<content>
 <macroRef key="macro.phraseSeq"/>
</content>
    
\end{Verbatim}

    \item[{Schéma Declaration}]
  \mbox{}\hfill\\[-10pt]\begin{Verbatim}[fontsize=\small]
element heraldry { tei_att.global.attributes, tei_macro.phraseSeq }
\end{Verbatim}

\end{reflist}  \index{hi=<hi>|oddindex}
\begin{reflist}
\item[]\begin{specHead}{TEI.hi}{<hi> }(mis en évidence) distingue un mot ou une expression comme graphiquement distincte du texte environnant, sans en donner la raison. [\xref{http://www.tei-c.org/release/doc/tei-p5-doc/en/html/CO.html\#COHQHE}{3.3.2.2. Emphatic Words and Phrases} \xref{http://www.tei-c.org/release/doc/tei-p5-doc/en/html/CO.html\#COHQH}{3.3.2. Emphasis, Foreign Words, and Unusual Language}]\end{specHead} 
    \item[{Module}]
  core
    \item[{Attributs}]
  Attributs \hyperref[TEI.att.global]{att.global} (\textit{@xml:id}, \textit{@n}, \textit{@xml:lang}, \textit{@xml:base}, \textit{@xml:space})  (\hyperref[TEI.att.global.rendition]{att.global.rendition} (\textit{@rend}, \textit{@style}, \textit{@rendition})) (\hyperref[TEI.att.global.linking]{att.global.linking} (\textit{@corresp}, \textit{@synch}, \textit{@sameAs}, \textit{@copyOf}, \textit{@next}, \textit{@prev}, \textit{@exclude}, \textit{@select})) (\hyperref[TEI.att.global.analytic]{att.global.analytic} (\textit{@ana})) (\hyperref[TEI.att.global.facs]{att.global.facs} (\textit{@facs})) (\hyperref[TEI.att.global.change]{att.global.change} (\textit{@change})) (\hyperref[TEI.att.global.responsibility]{att.global.responsibility} (\textit{@cert}, \textit{@resp})) (\hyperref[TEI.att.global.source]{att.global.source} (\textit{@source})) \hyperref[TEI.att.written]{att.written} (\textit{@hand}) 
    \item[{Membre du}]
  \hyperref[TEI.model.hiLike]{model.hiLike}
    \item[{Contenu dans}]
  
    \item[analysis: ]
   \hyperref[TEI.cl]{cl} \hyperref[TEI.m]{m} \hyperref[TEI.phr]{phr} \hyperref[TEI.s]{s} \hyperref[TEI.span]{span} \hyperref[TEI.w]{w}\par 
    \item[core: ]
   \hyperref[TEI.abbr]{abbr} \hyperref[TEI.add]{add} \hyperref[TEI.addrLine]{addrLine} \hyperref[TEI.author]{author} \hyperref[TEI.bibl]{bibl} \hyperref[TEI.biblScope]{biblScope} \hyperref[TEI.citedRange]{citedRange} \hyperref[TEI.corr]{corr} \hyperref[TEI.date]{date} \hyperref[TEI.del]{del} \hyperref[TEI.desc]{desc} \hyperref[TEI.distinct]{distinct} \hyperref[TEI.editor]{editor} \hyperref[TEI.email]{email} \hyperref[TEI.emph]{emph} \hyperref[TEI.expan]{expan} \hyperref[TEI.foreign]{foreign} \hyperref[TEI.gloss]{gloss} \hyperref[TEI.head]{head} \hyperref[TEI.headItem]{headItem} \hyperref[TEI.headLabel]{headLabel} \hyperref[TEI.hi]{hi} \hyperref[TEI.item]{item} \hyperref[TEI.l]{l} \hyperref[TEI.label]{label} \hyperref[TEI.measure]{measure} \hyperref[TEI.meeting]{meeting} \hyperref[TEI.mentioned]{mentioned} \hyperref[TEI.name]{name} \hyperref[TEI.note]{note} \hyperref[TEI.num]{num} \hyperref[TEI.orig]{orig} \hyperref[TEI.p]{p} \hyperref[TEI.pubPlace]{pubPlace} \hyperref[TEI.publisher]{publisher} \hyperref[TEI.q]{q} \hyperref[TEI.quote]{quote} \hyperref[TEI.ref]{ref} \hyperref[TEI.reg]{reg} \hyperref[TEI.resp]{resp} \hyperref[TEI.rs]{rs} \hyperref[TEI.said]{said} \hyperref[TEI.sic]{sic} \hyperref[TEI.soCalled]{soCalled} \hyperref[TEI.speaker]{speaker} \hyperref[TEI.stage]{stage} \hyperref[TEI.street]{street} \hyperref[TEI.term]{term} \hyperref[TEI.textLang]{textLang} \hyperref[TEI.time]{time} \hyperref[TEI.title]{title} \hyperref[TEI.unclear]{unclear}\par 
    \item[figures: ]
   \hyperref[TEI.cell]{cell} \hyperref[TEI.figDesc]{figDesc} \hyperref[TEI.formula]{formula}\par 
    \item[header: ]
   \hyperref[TEI.authority]{authority} \hyperref[TEI.change]{change} \hyperref[TEI.classCode]{classCode} \hyperref[TEI.creation]{creation} \hyperref[TEI.distributor]{distributor} \hyperref[TEI.edition]{edition} \hyperref[TEI.extent]{extent} \hyperref[TEI.funder]{funder} \hyperref[TEI.language]{language} \hyperref[TEI.licence]{licence} \hyperref[TEI.rendition]{rendition}\par 
    \item[iso-fs: ]
   \hyperref[TEI.fDescr]{fDescr} \hyperref[TEI.fsDescr]{fsDescr}\par 
    \item[linking: ]
   \hyperref[TEI.ab]{ab} \hyperref[TEI.seg]{seg}\par 
    \item[msdescription: ]
   \hyperref[TEI.accMat]{accMat} \hyperref[TEI.acquisition]{acquisition} \hyperref[TEI.additions]{additions} \hyperref[TEI.catchwords]{catchwords} \hyperref[TEI.collation]{collation} \hyperref[TEI.colophon]{colophon} \hyperref[TEI.condition]{condition} \hyperref[TEI.custEvent]{custEvent} \hyperref[TEI.decoNote]{decoNote} \hyperref[TEI.explicit]{explicit} \hyperref[TEI.filiation]{filiation} \hyperref[TEI.finalRubric]{finalRubric} \hyperref[TEI.foliation]{foliation} \hyperref[TEI.heraldry]{heraldry} \hyperref[TEI.incipit]{incipit} \hyperref[TEI.layout]{layout} \hyperref[TEI.locus]{locus} \hyperref[TEI.material]{material} \hyperref[TEI.musicNotation]{musicNotation} \hyperref[TEI.objectType]{objectType} \hyperref[TEI.origDate]{origDate} \hyperref[TEI.origPlace]{origPlace} \hyperref[TEI.origin]{origin} \hyperref[TEI.provenance]{provenance} \hyperref[TEI.rubric]{rubric} \hyperref[TEI.secFol]{secFol} \hyperref[TEI.signatures]{signatures} \hyperref[TEI.source]{source} \hyperref[TEI.stamp]{stamp} \hyperref[TEI.summary]{summary} \hyperref[TEI.support]{support} \hyperref[TEI.surrogates]{surrogates} \hyperref[TEI.typeNote]{typeNote} \hyperref[TEI.watermark]{watermark}\par 
    \item[namesdates: ]
   \hyperref[TEI.addName]{addName} \hyperref[TEI.affiliation]{affiliation} \hyperref[TEI.country]{country} \hyperref[TEI.forename]{forename} \hyperref[TEI.genName]{genName} \hyperref[TEI.geogName]{geogName} \hyperref[TEI.nameLink]{nameLink} \hyperref[TEI.orgName]{orgName} \hyperref[TEI.persName]{persName} \hyperref[TEI.placeName]{placeName} \hyperref[TEI.region]{region} \hyperref[TEI.roleName]{roleName} \hyperref[TEI.settlement]{settlement} \hyperref[TEI.surname]{surname}\par 
    \item[textstructure: ]
   \hyperref[TEI.docAuthor]{docAuthor} \hyperref[TEI.docDate]{docDate} \hyperref[TEI.docEdition]{docEdition} \hyperref[TEI.titlePart]{titlePart}\par 
    \item[transcr: ]
   \hyperref[TEI.damage]{damage} \hyperref[TEI.fw]{fw} \hyperref[TEI.line]{line} \hyperref[TEI.metamark]{metamark} \hyperref[TEI.mod]{mod} \hyperref[TEI.restore]{restore} \hyperref[TEI.retrace]{retrace} \hyperref[TEI.secl]{secl} \hyperref[TEI.supplied]{supplied} \hyperref[TEI.surplus]{surplus} \hyperref[TEI.zone]{zone}
    \item[{Peut contenir}]
  
    \item[analysis: ]
   \hyperref[TEI.c]{c} \hyperref[TEI.cl]{cl} \hyperref[TEI.interp]{interp} \hyperref[TEI.interpGrp]{interpGrp} \hyperref[TEI.m]{m} \hyperref[TEI.pc]{pc} \hyperref[TEI.phr]{phr} \hyperref[TEI.s]{s} \hyperref[TEI.span]{span} \hyperref[TEI.spanGrp]{spanGrp} \hyperref[TEI.w]{w}\par 
    \item[core: ]
   \hyperref[TEI.abbr]{abbr} \hyperref[TEI.add]{add} \hyperref[TEI.address]{address} \hyperref[TEI.bibl]{bibl} \hyperref[TEI.biblStruct]{biblStruct} \hyperref[TEI.binaryObject]{binaryObject} \hyperref[TEI.cb]{cb} \hyperref[TEI.choice]{choice} \hyperref[TEI.cit]{cit} \hyperref[TEI.corr]{corr} \hyperref[TEI.date]{date} \hyperref[TEI.del]{del} \hyperref[TEI.desc]{desc} \hyperref[TEI.distinct]{distinct} \hyperref[TEI.email]{email} \hyperref[TEI.emph]{emph} \hyperref[TEI.expan]{expan} \hyperref[TEI.foreign]{foreign} \hyperref[TEI.gap]{gap} \hyperref[TEI.gb]{gb} \hyperref[TEI.gloss]{gloss} \hyperref[TEI.graphic]{graphic} \hyperref[TEI.hi]{hi} \hyperref[TEI.index]{index} \hyperref[TEI.l]{l} \hyperref[TEI.label]{label} \hyperref[TEI.lb]{lb} \hyperref[TEI.lg]{lg} \hyperref[TEI.list]{list} \hyperref[TEI.listBibl]{listBibl} \hyperref[TEI.measure]{measure} \hyperref[TEI.measureGrp]{measureGrp} \hyperref[TEI.media]{media} \hyperref[TEI.mentioned]{mentioned} \hyperref[TEI.milestone]{milestone} \hyperref[TEI.name]{name} \hyperref[TEI.note]{note} \hyperref[TEI.num]{num} \hyperref[TEI.orig]{orig} \hyperref[TEI.pb]{pb} \hyperref[TEI.ptr]{ptr} \hyperref[TEI.q]{q} \hyperref[TEI.quote]{quote} \hyperref[TEI.ref]{ref} \hyperref[TEI.reg]{reg} \hyperref[TEI.rs]{rs} \hyperref[TEI.said]{said} \hyperref[TEI.sic]{sic} \hyperref[TEI.soCalled]{soCalled} \hyperref[TEI.stage]{stage} \hyperref[TEI.term]{term} \hyperref[TEI.time]{time} \hyperref[TEI.title]{title} \hyperref[TEI.unclear]{unclear}\par 
    \item[derived-module-tei.istex: ]
   \hyperref[TEI.math]{math} \hyperref[TEI.mrow]{mrow}\par 
    \item[figures: ]
   \hyperref[TEI.figure]{figure} \hyperref[TEI.formula]{formula} \hyperref[TEI.notatedMusic]{notatedMusic} \hyperref[TEI.table]{table}\par 
    \item[header: ]
   \hyperref[TEI.biblFull]{biblFull} \hyperref[TEI.idno]{idno}\par 
    \item[iso-fs: ]
   \hyperref[TEI.fLib]{fLib} \hyperref[TEI.fs]{fs} \hyperref[TEI.fvLib]{fvLib}\par 
    \item[linking: ]
   \hyperref[TEI.alt]{alt} \hyperref[TEI.altGrp]{altGrp} \hyperref[TEI.anchor]{anchor} \hyperref[TEI.join]{join} \hyperref[TEI.joinGrp]{joinGrp} \hyperref[TEI.link]{link} \hyperref[TEI.linkGrp]{linkGrp} \hyperref[TEI.seg]{seg} \hyperref[TEI.timeline]{timeline}\par 
    \item[msdescription: ]
   \hyperref[TEI.catchwords]{catchwords} \hyperref[TEI.depth]{depth} \hyperref[TEI.dim]{dim} \hyperref[TEI.dimensions]{dimensions} \hyperref[TEI.height]{height} \hyperref[TEI.heraldry]{heraldry} \hyperref[TEI.locus]{locus} \hyperref[TEI.locusGrp]{locusGrp} \hyperref[TEI.material]{material} \hyperref[TEI.msDesc]{msDesc} \hyperref[TEI.objectType]{objectType} \hyperref[TEI.origDate]{origDate} \hyperref[TEI.origPlace]{origPlace} \hyperref[TEI.secFol]{secFol} \hyperref[TEI.signatures]{signatures} \hyperref[TEI.source]{source} \hyperref[TEI.stamp]{stamp} \hyperref[TEI.watermark]{watermark} \hyperref[TEI.width]{width}\par 
    \item[namesdates: ]
   \hyperref[TEI.addName]{addName} \hyperref[TEI.affiliation]{affiliation} \hyperref[TEI.country]{country} \hyperref[TEI.forename]{forename} \hyperref[TEI.genName]{genName} \hyperref[TEI.geogName]{geogName} \hyperref[TEI.listOrg]{listOrg} \hyperref[TEI.listPlace]{listPlace} \hyperref[TEI.location]{location} \hyperref[TEI.nameLink]{nameLink} \hyperref[TEI.orgName]{orgName} \hyperref[TEI.persName]{persName} \hyperref[TEI.placeName]{placeName} \hyperref[TEI.region]{region} \hyperref[TEI.roleName]{roleName} \hyperref[TEI.settlement]{settlement} \hyperref[TEI.state]{state} \hyperref[TEI.surname]{surname}\par 
    \item[spoken: ]
   \hyperref[TEI.annotationBlock]{annotationBlock}\par 
    \item[textstructure: ]
   \hyperref[TEI.floatingText]{floatingText}\par 
    \item[transcr: ]
   \hyperref[TEI.addSpan]{addSpan} \hyperref[TEI.am]{am} \hyperref[TEI.damage]{damage} \hyperref[TEI.damageSpan]{damageSpan} \hyperref[TEI.delSpan]{delSpan} \hyperref[TEI.ex]{ex} \hyperref[TEI.fw]{fw} \hyperref[TEI.handShift]{handShift} \hyperref[TEI.listTranspose]{listTranspose} \hyperref[TEI.metamark]{metamark} \hyperref[TEI.mod]{mod} \hyperref[TEI.redo]{redo} \hyperref[TEI.restore]{restore} \hyperref[TEI.retrace]{retrace} \hyperref[TEI.secl]{secl} \hyperref[TEI.space]{space} \hyperref[TEI.subst]{subst} \hyperref[TEI.substJoin]{substJoin} \hyperref[TEI.supplied]{supplied} \hyperref[TEI.surplus]{surplus} \hyperref[TEI.undo]{undo}\par des données textuelles
    \item[{Exemple}]
  \leavevmode\bgroup\exampleFont \begin{shaded}\noindent\mbox{}{<\textbf{p}>}Au fronton, on lit cette inscription : {<\textbf{hi}\hspace*{6pt}{rend}="{uppercase}">}attends. Tu verras.{</\textbf{hi}>} Le\mbox{}\newline 
 notaire encore prétend qu' elle ne saurait être antérieure au XVIII siècle, car, sinon, l'\mbox{}\newline 
 on eût écrit --{<\textbf{q}>}tu voiras{</\textbf{q}>}--. {</\textbf{p}>}\end{shaded}\egroup 


    \item[{Modèle de contenu}]
  \mbox{}\hfill\\[-10pt]\begin{Verbatim}[fontsize=\small]
<content>
 <macroRef key="macro.paraContent"/>
</content>
    
\end{Verbatim}

    \item[{Schéma Declaration}]
  \mbox{}\hfill\\[-10pt]\begin{Verbatim}[fontsize=\small]
element hi
{
   tei_att.global.attributes,
   tei_att.written.attributes,
   tei_macro.paraContent}
\end{Verbatim}

\end{reflist}  \index{history=<history>|oddindex}
\begin{reflist}
\item[]\begin{specHead}{TEI.history}{<history> }(histoire) rassemble les éléments servant à donner un historique complet du manuscrit ou d'une partie du manuscrit. [\xref{http://www.tei-c.org/release/doc/tei-p5-doc/en/html/MS.html\#mshy}{10.8. History}]\end{specHead} 
    \item[{Module}]
  msdescription
    \item[{Attributs}]
  Attributs \hyperref[TEI.att.global]{att.global} (\textit{@xml:id}, \textit{@n}, \textit{@xml:lang}, \textit{@xml:base}, \textit{@xml:space})  (\hyperref[TEI.att.global.rendition]{att.global.rendition} (\textit{@rend}, \textit{@style}, \textit{@rendition})) (\hyperref[TEI.att.global.linking]{att.global.linking} (\textit{@corresp}, \textit{@synch}, \textit{@sameAs}, \textit{@copyOf}, \textit{@next}, \textit{@prev}, \textit{@exclude}, \textit{@select})) (\hyperref[TEI.att.global.analytic]{att.global.analytic} (\textit{@ana})) (\hyperref[TEI.att.global.facs]{att.global.facs} (\textit{@facs})) (\hyperref[TEI.att.global.change]{att.global.change} (\textit{@change})) (\hyperref[TEI.att.global.responsibility]{att.global.responsibility} (\textit{@cert}, \textit{@resp})) (\hyperref[TEI.att.global.source]{att.global.source} (\textit{@source}))
    \item[{Contenu dans}]
  
    \item[msdescription: ]
   \hyperref[TEI.msDesc]{msDesc} \hyperref[TEI.msFrag]{msFrag} \hyperref[TEI.msPart]{msPart}
    \item[{Peut contenir}]
  
    \item[core: ]
   \hyperref[TEI.p]{p}\par 
    \item[linking: ]
   \hyperref[TEI.ab]{ab}\par 
    \item[msdescription: ]
   \hyperref[TEI.acquisition]{acquisition} \hyperref[TEI.origin]{origin} \hyperref[TEI.provenance]{provenance} \hyperref[TEI.summary]{summary}
    \item[{Exemple}]
  \leavevmode\bgroup\exampleFont \begin{shaded}\noindent\mbox{}{<\textbf{history}>}\mbox{}\newline 
\hspace*{6pt}{<\textbf{origin}>}\mbox{}\newline 
\hspace*{6pt}\hspace*{6pt}{<\textbf{p}>}Written in Durham during the mid twelfth\mbox{}\newline 
\hspace*{6pt}\hspace*{6pt}\hspace*{6pt}\hspace*{6pt} century.{</\textbf{p}>}\mbox{}\newline 
\hspace*{6pt}{</\textbf{origin}>}\mbox{}\newline 
\hspace*{6pt}{<\textbf{provenance}>}\mbox{}\newline 
\hspace*{6pt}\hspace*{6pt}{<\textbf{p}>}Recorded in two medieval\mbox{}\newline 
\hspace*{6pt}\hspace*{6pt}\hspace*{6pt}\hspace*{6pt} catalogues of the books belonging to Durham Priory, made in 1391 and\mbox{}\newline 
\hspace*{6pt}\hspace*{6pt}\hspace*{6pt}\hspace*{6pt} 1405.{</\textbf{p}>}\mbox{}\newline 
\hspace*{6pt}{</\textbf{provenance}>}\mbox{}\newline 
\hspace*{6pt}{<\textbf{provenance}>}\mbox{}\newline 
\hspace*{6pt}\hspace*{6pt}{<\textbf{p}>}Given to W. Olleyf by William Ebchester, Prior (1446-56)\mbox{}\newline 
\hspace*{6pt}\hspace*{6pt}\hspace*{6pt}\hspace*{6pt} and later belonged to Henry Dalton, Prior of Holy Island (Lindisfarne)\mbox{}\newline 
\hspace*{6pt}\hspace*{6pt}\hspace*{6pt}\hspace*{6pt} according to inscriptions on ff. 4v and 5.{</\textbf{p}>}\mbox{}\newline 
\hspace*{6pt}{</\textbf{provenance}>}\mbox{}\newline 
\hspace*{6pt}{<\textbf{acquisition}>}\mbox{}\newline 
\hspace*{6pt}\hspace*{6pt}{<\textbf{p}>}Presented to Trinity College in 1738 by\mbox{}\newline 
\hspace*{6pt}\hspace*{6pt}\hspace*{6pt}\hspace*{6pt} Thomas Gale and his son Roger.{</\textbf{p}>}\mbox{}\newline 
\hspace*{6pt}{</\textbf{acquisition}>}\mbox{}\newline 
{</\textbf{history}>}\end{shaded}\egroup 


    \item[{Modèle de contenu}]
  \mbox{}\hfill\\[-10pt]\begin{Verbatim}[fontsize=\small]
<content>
 <alternate maxOccurs="1" minOccurs="1">
  <classRef key="model.pLike"
   maxOccurs="unbounded" minOccurs="1"/>
  <sequence maxOccurs="1" minOccurs="1">
   <elementRef key="summary" minOccurs="0"/>
   <elementRef key="origin" minOccurs="0"/>
   <elementRef key="provenance"
    maxOccurs="unbounded" minOccurs="0"/>
   <elementRef key="acquisition"
    minOccurs="0"/>
  </sequence>
 </alternate>
</content>
    
\end{Verbatim}

    \item[{Schéma Declaration}]
  \mbox{}\hfill\\[-10pt]\begin{Verbatim}[fontsize=\small]
element history
{
   tei_att.global.attributes,
   (
      tei_model.pLike+
    | ( tei_summary?, tei_origin?, tei_provenance*, tei_acquisition? )
   )
}
\end{Verbatim}

\end{reflist}  \index{idno=<idno>|oddindex}\index{type=@type!<idno>|oddindex}
\begin{reflist}
\item[]\begin{specHead}{TEI.idno}{<idno> }(identifiant) donne un numéro normalisé ou non qui peut être utilisé pour identifier une référence bibliographique. [\xref{http://www.tei-c.org/release/doc/tei-p5-doc/en/html/HD.html\#HD24}{2.2.4. Publication, Distribution, Licensing, etc.} \xref{http://www.tei-c.org/release/doc/tei-p5-doc/en/html/HD.html\#HD26}{2.2.5. The Series Statement} \xref{http://www.tei-c.org/release/doc/tei-p5-doc/en/html/CO.html\#COBICOI}{3.11.2.4. Imprint, Size of a Document, and Reprint Information}]\end{specHead} 
    \item[{Module}]
  header
    \item[{Attributs}]
  Attributs \hyperref[TEI.att.global]{att.global} (\textit{@xml:id}, \textit{@n}, \textit{@xml:lang}, \textit{@xml:base}, \textit{@xml:space})  (\hyperref[TEI.att.global.rendition]{att.global.rendition} (\textit{@rend}, \textit{@style}, \textit{@rendition})) (\hyperref[TEI.att.global.linking]{att.global.linking} (\textit{@corresp}, \textit{@synch}, \textit{@sameAs}, \textit{@copyOf}, \textit{@next}, \textit{@prev}, \textit{@exclude}, \textit{@select})) (\hyperref[TEI.att.global.analytic]{att.global.analytic} (\textit{@ana})) (\hyperref[TEI.att.global.facs]{att.global.facs} (\textit{@facs})) (\hyperref[TEI.att.global.change]{att.global.change} (\textit{@change})) (\hyperref[TEI.att.global.responsibility]{att.global.responsibility} (\textit{@cert}, \textit{@resp})) (\hyperref[TEI.att.global.source]{att.global.source} (\textit{@source})) \hyperref[TEI.att.sortable]{att.sortable} (\textit{@sortKey}) \hyperref[TEI.att.datable]{att.datable} (\textit{@calendar}, \textit{@period})  (\hyperref[TEI.att.datable.w3c]{att.datable.w3c} (\textit{@when}, \textit{@notBefore}, \textit{@notAfter}, \textit{@from}, \textit{@to})) (\hyperref[TEI.att.datable.iso]{att.datable.iso} (\textit{@when-iso}, \textit{@notBefore-iso}, \textit{@notAfter-iso}, \textit{@from-iso}, \textit{@to-iso})) (\hyperref[TEI.att.datable.custom]{att.datable.custom} (\textit{@when-custom}, \textit{@notBefore-custom}, \textit{@notAfter-custom}, \textit{@from-custom}, \textit{@to-custom}, \textit{@datingPoint}, \textit{@datingMethod})) \hyperref[TEI.att.typed]{att.typed} (\unusedattribute{type}, @subtype) \hfil\\[-10pt]\begin{sansreflist}
    \item[@type]
  classe un numéro dans une catégorie, par exemple comme étant un numéro ISBN ou comme appartenant une autre série normalisée.
\begin{reflist}
    \item[{Dérivé de}]
  \hyperref[TEI.att.typed]{att.typed}
    \item[{Statut}]
  Optionel
    \item[{Type de données}]
  \hyperref[TEI.teidata.enumerated]{teidata.enumerated}
    \item[{Les valeurs suggérées comprennent:}]
  \begin{description}

\item[{ISBN}]International Standard Book Number: a 13- or (if assigned prior to 2007) 10-digit identifying number assigned by the publishing industry to a published book or similar item, registered with the \xref{https://www.isbn-international.org}{International ISBN Agency.}
\item[{ISSN}]International Standard Serial Number: an eight-digit number to uniquely identify a serial publication.
\item[{DOI}]Digital Object Identifier: a unique string of letters and numbers assigned to an electronic document.
\item[{URI}]Uniform Resource Identifier: a string of characters to uniquely identify a resource which usually contains indication of the means of accessing that resource, the name of its host, and its filepath.
\item[{VIAF}]A data number in the Virtual Internet Authority File assigned to link different names in catalogs around the world for the same entity.
\item[{ESTC}]English Short-Title Catalogue number: an identifying number assigned to a document in English printed in the British Isles or North America before 1801.
\item[{OCLC}]union catalog number in WorldCat representing a resource held by one or more of the member libraries in the global cooperative Online Computer Library Center.
\end{description} 
\end{reflist}  
\end{sansreflist}  
    \item[{Membre du}]
  \hyperref[TEI.model.msItemPart]{model.msItemPart} \hyperref[TEI.model.nameLike]{model.nameLike} \hyperref[TEI.model.personPart]{model.personPart} \hyperref[TEI.model.publicationStmtPart.detail]{model.publicationStmtPart.detail} 
    \item[{Contenu dans}]
  
    \item[analysis: ]
   \hyperref[TEI.cl]{cl} \hyperref[TEI.phr]{phr} \hyperref[TEI.s]{s} \hyperref[TEI.span]{span}\par 
    \item[core: ]
   \hyperref[TEI.abbr]{abbr} \hyperref[TEI.add]{add} \hyperref[TEI.addrLine]{addrLine} \hyperref[TEI.address]{address} \hyperref[TEI.analytic]{analytic} \hyperref[TEI.author]{author} \hyperref[TEI.bibl]{bibl} \hyperref[TEI.biblScope]{biblScope} \hyperref[TEI.citedRange]{citedRange} \hyperref[TEI.corr]{corr} \hyperref[TEI.date]{date} \hyperref[TEI.del]{del} \hyperref[TEI.desc]{desc} \hyperref[TEI.distinct]{distinct} \hyperref[TEI.editor]{editor} \hyperref[TEI.email]{email} \hyperref[TEI.emph]{emph} \hyperref[TEI.expan]{expan} \hyperref[TEI.foreign]{foreign} \hyperref[TEI.gloss]{gloss} \hyperref[TEI.head]{head} \hyperref[TEI.headItem]{headItem} \hyperref[TEI.headLabel]{headLabel} \hyperref[TEI.hi]{hi} \hyperref[TEI.item]{item} \hyperref[TEI.l]{l} \hyperref[TEI.label]{label} \hyperref[TEI.measure]{measure} \hyperref[TEI.meeting]{meeting} \hyperref[TEI.mentioned]{mentioned} \hyperref[TEI.monogr]{monogr} \hyperref[TEI.name]{name} \hyperref[TEI.note]{note} \hyperref[TEI.num]{num} \hyperref[TEI.orig]{orig} \hyperref[TEI.p]{p} \hyperref[TEI.pubPlace]{pubPlace} \hyperref[TEI.publisher]{publisher} \hyperref[TEI.q]{q} \hyperref[TEI.quote]{quote} \hyperref[TEI.ref]{ref} \hyperref[TEI.reg]{reg} \hyperref[TEI.resp]{resp} \hyperref[TEI.rs]{rs} \hyperref[TEI.said]{said} \hyperref[TEI.series]{series} \hyperref[TEI.sic]{sic} \hyperref[TEI.soCalled]{soCalled} \hyperref[TEI.speaker]{speaker} \hyperref[TEI.stage]{stage} \hyperref[TEI.street]{street} \hyperref[TEI.term]{term} \hyperref[TEI.textLang]{textLang} \hyperref[TEI.time]{time} \hyperref[TEI.title]{title} \hyperref[TEI.unclear]{unclear}\par 
    \item[figures: ]
   \hyperref[TEI.cell]{cell} \hyperref[TEI.figDesc]{figDesc}\par 
    \item[header: ]
   \hyperref[TEI.authority]{authority} \hyperref[TEI.change]{change} \hyperref[TEI.classCode]{classCode} \hyperref[TEI.creation]{creation} \hyperref[TEI.distributor]{distributor} \hyperref[TEI.edition]{edition} \hyperref[TEI.extent]{extent} \hyperref[TEI.funder]{funder} \hyperref[TEI.idno]{idno} \hyperref[TEI.language]{language} \hyperref[TEI.licence]{licence} \hyperref[TEI.publicationStmt]{publicationStmt} \hyperref[TEI.rendition]{rendition} \hyperref[TEI.seriesStmt]{seriesStmt}\par 
    \item[iso-fs: ]
   \hyperref[TEI.fDescr]{fDescr} \hyperref[TEI.fsDescr]{fsDescr}\par 
    \item[linking: ]
   \hyperref[TEI.ab]{ab} \hyperref[TEI.seg]{seg}\par 
    \item[msdescription: ]
   \hyperref[TEI.accMat]{accMat} \hyperref[TEI.acquisition]{acquisition} \hyperref[TEI.additions]{additions} \hyperref[TEI.altIdentifier]{altIdentifier} \hyperref[TEI.catchwords]{catchwords} \hyperref[TEI.collation]{collation} \hyperref[TEI.colophon]{colophon} \hyperref[TEI.condition]{condition} \hyperref[TEI.custEvent]{custEvent} \hyperref[TEI.decoNote]{decoNote} \hyperref[TEI.explicit]{explicit} \hyperref[TEI.filiation]{filiation} \hyperref[TEI.finalRubric]{finalRubric} \hyperref[TEI.foliation]{foliation} \hyperref[TEI.heraldry]{heraldry} \hyperref[TEI.incipit]{incipit} \hyperref[TEI.layout]{layout} \hyperref[TEI.material]{material} \hyperref[TEI.msIdentifier]{msIdentifier} \hyperref[TEI.msItem]{msItem} \hyperref[TEI.musicNotation]{musicNotation} \hyperref[TEI.objectType]{objectType} \hyperref[TEI.origDate]{origDate} \hyperref[TEI.origPlace]{origPlace} \hyperref[TEI.origin]{origin} \hyperref[TEI.provenance]{provenance} \hyperref[TEI.rubric]{rubric} \hyperref[TEI.secFol]{secFol} \hyperref[TEI.signatures]{signatures} \hyperref[TEI.source]{source} \hyperref[TEI.stamp]{stamp} \hyperref[TEI.summary]{summary} \hyperref[TEI.support]{support} \hyperref[TEI.surrogates]{surrogates} \hyperref[TEI.typeNote]{typeNote} \hyperref[TEI.watermark]{watermark}\par 
    \item[namesdates: ]
   \hyperref[TEI.addName]{addName} \hyperref[TEI.affiliation]{affiliation} \hyperref[TEI.country]{country} \hyperref[TEI.forename]{forename} \hyperref[TEI.genName]{genName} \hyperref[TEI.geogName]{geogName} \hyperref[TEI.nameLink]{nameLink} \hyperref[TEI.org]{org} \hyperref[TEI.orgName]{orgName} \hyperref[TEI.persName]{persName} \hyperref[TEI.person]{person} \hyperref[TEI.personGrp]{personGrp} \hyperref[TEI.persona]{persona} \hyperref[TEI.place]{place} \hyperref[TEI.placeName]{placeName} \hyperref[TEI.region]{region} \hyperref[TEI.roleName]{roleName} \hyperref[TEI.settlement]{settlement} \hyperref[TEI.surname]{surname}\par 
    \item[spoken: ]
   \hyperref[TEI.annotationBlock]{annotationBlock}\par 
    \item[standOff: ]
   \hyperref[TEI.listAnnotation]{listAnnotation}\par 
    \item[textstructure: ]
   \hyperref[TEI.docAuthor]{docAuthor} \hyperref[TEI.docDate]{docDate} \hyperref[TEI.docEdition]{docEdition} \hyperref[TEI.titlePart]{titlePart}\par 
    \item[transcr: ]
   \hyperref[TEI.damage]{damage} \hyperref[TEI.fw]{fw} \hyperref[TEI.metamark]{metamark} \hyperref[TEI.mod]{mod} \hyperref[TEI.restore]{restore} \hyperref[TEI.retrace]{retrace} \hyperref[TEI.secl]{secl} \hyperref[TEI.supplied]{supplied} \hyperref[TEI.surplus]{surplus}
    \item[{Peut contenir}]
  
    \item[header: ]
   \hyperref[TEI.idno]{idno}\par des données textuelles
    \item[{Note}]
  \par
\hyperref[TEI.idno]{<idno>} should be used for labels which identify an object or concept in a formal cataloguing system such as a database or an RDF store, or in a distributed system such as the World Wide Web. Some suggested values for {\itshape type} on \hyperref[TEI.idno]{<idno>} are ISBN, ISSN, DOI, and URI.
    \item[{Exemple}]
  \leavevmode\bgroup\exampleFont \begin{shaded}\noindent\mbox{}{<\textbf{idno}\hspace*{6pt}{type}="{ISBN}">}978-1-906964-22-1{</\textbf{idno}>}\mbox{}\newline 
{<\textbf{idno}\hspace*{6pt}{type}="{ISSN}">}0143-3385{</\textbf{idno}>}\mbox{}\newline 
{<\textbf{idno}\hspace*{6pt}{type}="{DOI}">}10.1000/123{</\textbf{idno}>}\mbox{}\newline 
{<\textbf{idno}\hspace*{6pt}{type}="{URI}">}http://www.worldcat.org/oclc/185922478{</\textbf{idno}>}\mbox{}\newline 
{<\textbf{idno}\hspace*{6pt}{type}="{URI}">}http://authority.nzetc.org/463/{</\textbf{idno}>}\mbox{}\newline 
{<\textbf{idno}\hspace*{6pt}{type}="{LT}">}Thomason Tract E.537(17){</\textbf{idno}>}\mbox{}\newline 
{<\textbf{idno}\hspace*{6pt}{type}="{Wing}">}C695{</\textbf{idno}>}\mbox{}\newline 
{<\textbf{idno}\hspace*{6pt}{type}="{oldCat}">}\mbox{}\newline 
\hspace*{6pt}{<\textbf{g}\hspace*{6pt}{ref}="{\#sym}"/>}345\mbox{}\newline 
{</\textbf{idno}>}\end{shaded}\egroup 

In the last case, the identifier includes a non-Unicode character which is defined elsewhere by means of a \texttt{<glyph>} or \texttt{<char>} element referenced here as \texttt{\#sym}.
    \item[{Exemple}]
  \leavevmode\bgroup\exampleFont \begin{shaded}\noindent\mbox{}{<\textbf{idno}\hspace*{6pt}{type}="{ISSN}">}0143-3385{</\textbf{idno}>}\mbox{}\newline 
{<\textbf{idno}\hspace*{6pt}{type}="{OTA}">}116{</\textbf{idno}>}\mbox{}\newline 
{<\textbf{idno}\hspace*{6pt}{type}="{ISBN}">}1-896016-00-6{</\textbf{idno}>}\end{shaded}\egroup 


    \item[{Modèle de contenu}]
  \mbox{}\hfill\\[-10pt]\begin{Verbatim}[fontsize=\small]
<content>
 <alternate maxOccurs="unbounded"
  minOccurs="0">
  <textNode/>
  <classRef key="model.gLike"/>
  <elementRef key="idno"/>
 </alternate>
</content>
    
\end{Verbatim}

    \item[{Schéma Declaration}]
  \mbox{}\hfill\\[-10pt]\begin{Verbatim}[fontsize=\small]
element idno
{
   tei_att.global.attributes,
   tei_att.sortable.attributes,
   tei_att.datable.attributes,
   tei_att.typed.attribute.subtype,
   attribute type
   {
      "ISBN" | "ISSN" | "DOI" | "URI" | "VIAF" | "ESTC" | "OCLC"
   }?,
   ( text | tei_model.gLike | tei_idno )*
}
\end{Verbatim}

\end{reflist}  \index{if=<if>|oddindex}
\begin{reflist}
\item[]\begin{specHead}{TEI.if}{<if> }définit une valeur conditionnelle par défaut pour un trait ; la condition est indiquée comme une structure de traits et remplie si elle englobe la structure de traits dans le texte pour lequel on cherche une valeur par défaut. [\xref{http://www.tei-c.org/release/doc/tei-p5-doc/en/html/FS.html\#FD}{18.11. Feature System Declaration}]\end{specHead} 
    \item[{Module}]
  iso-fs
    \item[{Attributs}]
  Attributs \hyperref[TEI.att.global]{att.global} (\textit{@xml:id}, \textit{@n}, \textit{@xml:lang}, \textit{@xml:base}, \textit{@xml:space})  (\hyperref[TEI.att.global.rendition]{att.global.rendition} (\textit{@rend}, \textit{@style}, \textit{@rendition})) (\hyperref[TEI.att.global.linking]{att.global.linking} (\textit{@corresp}, \textit{@synch}, \textit{@sameAs}, \textit{@copyOf}, \textit{@next}, \textit{@prev}, \textit{@exclude}, \textit{@select})) (\hyperref[TEI.att.global.analytic]{att.global.analytic} (\textit{@ana})) (\hyperref[TEI.att.global.facs]{att.global.facs} (\textit{@facs})) (\hyperref[TEI.att.global.change]{att.global.change} (\textit{@change})) (\hyperref[TEI.att.global.responsibility]{att.global.responsibility} (\textit{@cert}, \textit{@resp})) (\hyperref[TEI.att.global.source]{att.global.source} (\textit{@source}))
    \item[{Contenu dans}]
  
    \item[iso-fs: ]
   \hyperref[TEI.vDefault]{vDefault}
    \item[{Peut contenir}]
  
    \item[iso-fs: ]
   \hyperref[TEI.binary]{binary} \hyperref[TEI.default]{default} \hyperref[TEI.f]{f} \hyperref[TEI.fs]{fs} \hyperref[TEI.numeric]{numeric} \hyperref[TEI.string]{string} \hyperref[TEI.symbol]{symbol} \hyperref[TEI.then]{then} \hyperref[TEI.vAlt]{vAlt} \hyperref[TEI.vColl]{vColl} \hyperref[TEI.vLabel]{vLabel} \hyperref[TEI.vMerge]{vMerge} \hyperref[TEI.vNot]{vNot}
    \item[{Note}]
  \par
Peut contenir une structure de traits suivie d'une valeur de trait ; les deux sont séparées par un élément \hyperref[TEI.then]{<then>}.
    \item[{Exemple}]
  \leavevmode\bgroup\exampleFont \begin{shaded}\noindent\mbox{}{<\textbf{vDefault}>}\mbox{}\newline 
\hspace*{6pt}{<\textbf{if}>}\mbox{}\newline 
\hspace*{6pt}\hspace*{6pt}{<\textbf{fs}>}\mbox{}\newline 
\hspace*{6pt}\hspace*{6pt}\hspace*{6pt}{<\textbf{f}\hspace*{6pt}{name}="{VFORM}">}\mbox{}\newline 
\hspace*{6pt}\hspace*{6pt}\hspace*{6pt}\hspace*{6pt}{<\textbf{symbol}\hspace*{6pt}{value}="{INF}"/>}\mbox{}\newline 
\hspace*{6pt}\hspace*{6pt}\hspace*{6pt}{</\textbf{f}>}\mbox{}\newline 
\hspace*{6pt}\hspace*{6pt}\hspace*{6pt}{<\textbf{f}\hspace*{6pt}{name}="{SUBJ}">}\mbox{}\newline 
\hspace*{6pt}\hspace*{6pt}\hspace*{6pt}\hspace*{6pt}{<\textbf{binary}\hspace*{6pt}{value}="{true}"/>}\mbox{}\newline 
\hspace*{6pt}\hspace*{6pt}\hspace*{6pt}{</\textbf{f}>}\mbox{}\newline 
\hspace*{6pt}\hspace*{6pt}{</\textbf{fs}>}\mbox{}\newline 
\hspace*{6pt}\hspace*{6pt}{<\textbf{then}/>}\mbox{}\newline 
\hspace*{6pt}\hspace*{6pt}{<\textbf{symbol}\hspace*{6pt}{value}="{for}"/>}\mbox{}\newline 
\hspace*{6pt}{</\textbf{if}>}\mbox{}\newline 
{</\textbf{vDefault}>}\end{shaded}\egroup 


    \item[{Modèle de contenu}]
  \mbox{}\hfill\\[-10pt]\begin{Verbatim}[fontsize=\small]
<content>
 <sequence maxOccurs="1" minOccurs="1">
  <alternate maxOccurs="1" minOccurs="1">
   <elementRef key="fs"/>
   <elementRef key="f"/>
  </alternate>
  <elementRef key="then"/>
  <classRef key="model.featureVal"/>
 </sequence>
</content>
    
\end{Verbatim}

    \item[{Schéma Declaration}]
  \mbox{}\hfill\\[-10pt]\begin{Verbatim}[fontsize=\small]
element if
{
   tei_att.global.attributes,
   ( ( tei_fs | tei_f ), tei_then, tei_model.featureVal )
}
\end{Verbatim}

\end{reflist}  \index{iff=<iff>|oddindex}
\begin{reflist}
\item[]\begin{specHead}{TEI.iff}{<iff> }(si et seulement si) sépare la condition de la conséquence dans un élément bicond [\xref{http://www.tei-c.org/release/doc/tei-p5-doc/en/html/FS.html\#FD}{18.11. Feature System Declaration}]\end{specHead} 
    \item[{Module}]
  iso-fs
    \item[{Attributs}]
  Attributs \hyperref[TEI.att.global]{att.global} (\textit{@xml:id}, \textit{@n}, \textit{@xml:lang}, \textit{@xml:base}, \textit{@xml:space})  (\hyperref[TEI.att.global.rendition]{att.global.rendition} (\textit{@rend}, \textit{@style}, \textit{@rendition})) (\hyperref[TEI.att.global.linking]{att.global.linking} (\textit{@corresp}, \textit{@synch}, \textit{@sameAs}, \textit{@copyOf}, \textit{@next}, \textit{@prev}, \textit{@exclude}, \textit{@select})) (\hyperref[TEI.att.global.analytic]{att.global.analytic} (\textit{@ana})) (\hyperref[TEI.att.global.facs]{att.global.facs} (\textit{@facs})) (\hyperref[TEI.att.global.change]{att.global.change} (\textit{@change})) (\hyperref[TEI.att.global.responsibility]{att.global.responsibility} (\textit{@cert}, \textit{@resp})) (\hyperref[TEI.att.global.source]{att.global.source} (\textit{@source}))
    \item[{Contenu dans}]
  
    \item[iso-fs: ]
   \hyperref[TEI.bicond]{bicond}
    \item[{Peut contenir}]
  Elément vide
    \item[{Note}]
  \par
Cet élément est fourni essentiellement pour rendre plus lisible par l'homme une déclaration d'un système de traits.
    \item[{Exemple}]
  \leavevmode\bgroup\exampleFont \begin{shaded}\noindent\mbox{}{<\textbf{bicond}>}\mbox{}\newline 
\hspace*{6pt}{<\textbf{fs}>}\mbox{}\newline 
\hspace*{6pt}\hspace*{6pt}{<\textbf{f}\hspace*{6pt}{name}="{FOO}">}\mbox{}\newline 
\hspace*{6pt}\hspace*{6pt}\hspace*{6pt}{<\textbf{symbol}\hspace*{6pt}{value}="{42}"/>}\mbox{}\newline 
\hspace*{6pt}\hspace*{6pt}{</\textbf{f}>}\mbox{}\newline 
\hspace*{6pt}{</\textbf{fs}>}\mbox{}\newline 
\hspace*{6pt}{<\textbf{iff}/>}\mbox{}\newline 
\hspace*{6pt}{<\textbf{fs}>}\mbox{}\newline 
\hspace*{6pt}\hspace*{6pt}{<\textbf{f}\hspace*{6pt}{name}="{BAR}">}\mbox{}\newline 
\hspace*{6pt}\hspace*{6pt}\hspace*{6pt}{<\textbf{binary}\hspace*{6pt}{value}="{true}"/>}\mbox{}\newline 
\hspace*{6pt}\hspace*{6pt}{</\textbf{f}>}\mbox{}\newline 
\hspace*{6pt}{</\textbf{fs}>}\mbox{}\newline 
{</\textbf{bicond}>}\end{shaded}\egroup 


    \item[{Modèle de contenu}]
  \fbox{\ttfamily <content>\newline
</content>\newline
    } 
    \item[{Schéma Declaration}]
  \fbox{\ttfamily element iff ❴ tei\textunderscore att.global.attributes, empty ❵} 
\end{reflist}  \index{imprint=<imprint>|oddindex}
\begin{reflist}
\item[]\begin{specHead}{TEI.imprint}{<imprint> }regroupe des informations relatives à la publication ou à la distribution d'un élément bibliographique. [\xref{http://www.tei-c.org/release/doc/tei-p5-doc/en/html/CO.html\#COBICOI}{3.11.2.4. Imprint, Size of a Document, and Reprint Information}]\end{specHead} 
    \item[{Module}]
  core
    \item[{Attributs}]
  Attributs \hyperref[TEI.att.global]{att.global} (\textit{@xml:id}, \textit{@n}, \textit{@xml:lang}, \textit{@xml:base}, \textit{@xml:space})  (\hyperref[TEI.att.global.rendition]{att.global.rendition} (\textit{@rend}, \textit{@style}, \textit{@rendition})) (\hyperref[TEI.att.global.linking]{att.global.linking} (\textit{@corresp}, \textit{@synch}, \textit{@sameAs}, \textit{@copyOf}, \textit{@next}, \textit{@prev}, \textit{@exclude}, \textit{@select})) (\hyperref[TEI.att.global.analytic]{att.global.analytic} (\textit{@ana})) (\hyperref[TEI.att.global.facs]{att.global.facs} (\textit{@facs})) (\hyperref[TEI.att.global.change]{att.global.change} (\textit{@change})) (\hyperref[TEI.att.global.responsibility]{att.global.responsibility} (\textit{@cert}, \textit{@resp})) (\hyperref[TEI.att.global.source]{att.global.source} (\textit{@source}))
    \item[{Contenu dans}]
  
    \item[core: ]
   \hyperref[TEI.monogr]{monogr}
    \item[{Peut contenir}]
  
    \item[analysis: ]
   \hyperref[TEI.interp]{interp} \hyperref[TEI.interpGrp]{interpGrp} \hyperref[TEI.span]{span} \hyperref[TEI.spanGrp]{spanGrp}\par 
    \item[core: ]
   \hyperref[TEI.biblScope]{biblScope} \hyperref[TEI.cb]{cb} \hyperref[TEI.date]{date} \hyperref[TEI.gap]{gap} \hyperref[TEI.gb]{gb} \hyperref[TEI.index]{index} \hyperref[TEI.lb]{lb} \hyperref[TEI.milestone]{milestone} \hyperref[TEI.note]{note} \hyperref[TEI.pb]{pb} \hyperref[TEI.pubPlace]{pubPlace} \hyperref[TEI.publisher]{publisher} \hyperref[TEI.respStmt]{respStmt} \hyperref[TEI.time]{time}\par 
    \item[figures: ]
   \hyperref[TEI.figure]{figure} \hyperref[TEI.notatedMusic]{notatedMusic}\par 
    \item[header: ]
   \hyperref[TEI.classCode]{classCode} \hyperref[TEI.distributor]{distributor}\par 
    \item[iso-fs: ]
   \hyperref[TEI.fLib]{fLib} \hyperref[TEI.fs]{fs} \hyperref[TEI.fvLib]{fvLib}\par 
    \item[linking: ]
   \hyperref[TEI.alt]{alt} \hyperref[TEI.altGrp]{altGrp} \hyperref[TEI.anchor]{anchor} \hyperref[TEI.join]{join} \hyperref[TEI.joinGrp]{joinGrp} \hyperref[TEI.link]{link} \hyperref[TEI.linkGrp]{linkGrp} \hyperref[TEI.timeline]{timeline}\par 
    \item[msdescription: ]
   \hyperref[TEI.source]{source}\par 
    \item[transcr: ]
   \hyperref[TEI.addSpan]{addSpan} \hyperref[TEI.damageSpan]{damageSpan} \hyperref[TEI.delSpan]{delSpan} \hyperref[TEI.fw]{fw} \hyperref[TEI.listTranspose]{listTranspose} \hyperref[TEI.metamark]{metamark} \hyperref[TEI.space]{space} \hyperref[TEI.substJoin]{substJoin}
    \item[{Exemple}]
  \leavevmode\bgroup\exampleFont \begin{shaded}\noindent\mbox{}{<\textbf{imprint}>}\mbox{}\newline 
\hspace*{6pt}{<\textbf{pubPlace}>}Paris{</\textbf{pubPlace}>}\mbox{}\newline 
\hspace*{6pt}{<\textbf{publisher}>}Les Éd. de Minuit{</\textbf{publisher}>}\mbox{}\newline 
\hspace*{6pt}{<\textbf{date}>}2001{</\textbf{date}>}\mbox{}\newline 
{</\textbf{imprint}>}\end{shaded}\egroup 


    \item[{Modèle de contenu}]
  \mbox{}\hfill\\[-10pt]\begin{Verbatim}[fontsize=\small]
<content>
 <sequence maxOccurs="1" minOccurs="1">
  <alternate maxOccurs="unbounded"
   minOccurs="0">
   <elementRef key="classCode"/>
   <elementRef key="catRef"/>
  </alternate>
  <sequence maxOccurs="unbounded"
   minOccurs="1">
   <alternate maxOccurs="1" minOccurs="1">
    <classRef key="model.imprintPart"/>
    <classRef key="model.dateLike"/>
   </alternate>
   <elementRef key="respStmt"
    maxOccurs="unbounded" minOccurs="0"/>
   <classRef key="model.global"
    maxOccurs="unbounded" minOccurs="0"/>
  </sequence>
 </sequence>
</content>
    
\end{Verbatim}

    \item[{Schéma Declaration}]
  \mbox{}\hfill\\[-10pt]\begin{Verbatim}[fontsize=\small]
element imprint
{
   tei_att.global.attributes,
   (
      ( tei_classCode | catRef )*,
      (
         ( tei_model.imprintPart | tei_model.dateLike ),
         tei_respStmt*,
         tei_model.global*
      )+
   )
}
\end{Verbatim}

\end{reflist}  \index{incipit=<incipit>|oddindex}
\begin{reflist}
\item[]\begin{specHead}{TEI.incipit}{<incipit> }contient l'\textit{incipit} d'une section d'un manuscrit, c'est-à-dire les mots commençant le texte proprement dit, à l'exclusion de toute \textit{rubrique} qui pourrait les précéder, la transcription étant de longueur suffisante pour permettre l'identification de l'œuvre. De tels incipit étaient autrefois souvent utilisés à la place du titre de l'œuvre, pour l'identifier. [\xref{http://www.tei-c.org/release/doc/tei-p5-doc/en/html/MS.html\#mscoit}{10.6.1. The msItem and msItemStruct Elements}]\end{specHead} 
    \item[{Module}]
  msdescription
    \item[{Attributs}]
  Attributs \hyperref[TEI.att.global]{att.global} (\textit{@xml:id}, \textit{@n}, \textit{@xml:lang}, \textit{@xml:base}, \textit{@xml:space})  (\hyperref[TEI.att.global.rendition]{att.global.rendition} (\textit{@rend}, \textit{@style}, \textit{@rendition})) (\hyperref[TEI.att.global.linking]{att.global.linking} (\textit{@corresp}, \textit{@synch}, \textit{@sameAs}, \textit{@copyOf}, \textit{@next}, \textit{@prev}, \textit{@exclude}, \textit{@select})) (\hyperref[TEI.att.global.analytic]{att.global.analytic} (\textit{@ana})) (\hyperref[TEI.att.global.facs]{att.global.facs} (\textit{@facs})) (\hyperref[TEI.att.global.change]{att.global.change} (\textit{@change})) (\hyperref[TEI.att.global.responsibility]{att.global.responsibility} (\textit{@cert}, \textit{@resp})) (\hyperref[TEI.att.global.source]{att.global.source} (\textit{@source})) \hyperref[TEI.att.typed]{att.typed} (\textit{@type}, \textit{@subtype}) \hyperref[TEI.att.msExcerpt]{att.msExcerpt} (\textit{@defective}) 
    \item[{Membre du}]
  \hyperref[TEI.model.msQuoteLike]{model.msQuoteLike} 
    \item[{Contenu dans}]
  
    \item[msdescription: ]
   \hyperref[TEI.msItem]{msItem} \hyperref[TEI.msItemStruct]{msItemStruct}
    \item[{Peut contenir}]
  
    \item[analysis: ]
   \hyperref[TEI.c]{c} \hyperref[TEI.cl]{cl} \hyperref[TEI.interp]{interp} \hyperref[TEI.interpGrp]{interpGrp} \hyperref[TEI.m]{m} \hyperref[TEI.pc]{pc} \hyperref[TEI.phr]{phr} \hyperref[TEI.s]{s} \hyperref[TEI.span]{span} \hyperref[TEI.spanGrp]{spanGrp} \hyperref[TEI.w]{w}\par 
    \item[core: ]
   \hyperref[TEI.abbr]{abbr} \hyperref[TEI.add]{add} \hyperref[TEI.address]{address} \hyperref[TEI.binaryObject]{binaryObject} \hyperref[TEI.cb]{cb} \hyperref[TEI.choice]{choice} \hyperref[TEI.corr]{corr} \hyperref[TEI.date]{date} \hyperref[TEI.del]{del} \hyperref[TEI.distinct]{distinct} \hyperref[TEI.email]{email} \hyperref[TEI.emph]{emph} \hyperref[TEI.expan]{expan} \hyperref[TEI.foreign]{foreign} \hyperref[TEI.gap]{gap} \hyperref[TEI.gb]{gb} \hyperref[TEI.gloss]{gloss} \hyperref[TEI.graphic]{graphic} \hyperref[TEI.hi]{hi} \hyperref[TEI.index]{index} \hyperref[TEI.lb]{lb} \hyperref[TEI.measure]{measure} \hyperref[TEI.measureGrp]{measureGrp} \hyperref[TEI.media]{media} \hyperref[TEI.mentioned]{mentioned} \hyperref[TEI.milestone]{milestone} \hyperref[TEI.name]{name} \hyperref[TEI.note]{note} \hyperref[TEI.num]{num} \hyperref[TEI.orig]{orig} \hyperref[TEI.pb]{pb} \hyperref[TEI.ptr]{ptr} \hyperref[TEI.ref]{ref} \hyperref[TEI.reg]{reg} \hyperref[TEI.rs]{rs} \hyperref[TEI.sic]{sic} \hyperref[TEI.soCalled]{soCalled} \hyperref[TEI.term]{term} \hyperref[TEI.time]{time} \hyperref[TEI.title]{title} \hyperref[TEI.unclear]{unclear}\par 
    \item[derived-module-tei.istex: ]
   \hyperref[TEI.math]{math} \hyperref[TEI.mrow]{mrow}\par 
    \item[figures: ]
   \hyperref[TEI.figure]{figure} \hyperref[TEI.formula]{formula} \hyperref[TEI.notatedMusic]{notatedMusic}\par 
    \item[header: ]
   \hyperref[TEI.idno]{idno}\par 
    \item[iso-fs: ]
   \hyperref[TEI.fLib]{fLib} \hyperref[TEI.fs]{fs} \hyperref[TEI.fvLib]{fvLib}\par 
    \item[linking: ]
   \hyperref[TEI.alt]{alt} \hyperref[TEI.altGrp]{altGrp} \hyperref[TEI.anchor]{anchor} \hyperref[TEI.join]{join} \hyperref[TEI.joinGrp]{joinGrp} \hyperref[TEI.link]{link} \hyperref[TEI.linkGrp]{linkGrp} \hyperref[TEI.seg]{seg} \hyperref[TEI.timeline]{timeline}\par 
    \item[msdescription: ]
   \hyperref[TEI.catchwords]{catchwords} \hyperref[TEI.depth]{depth} \hyperref[TEI.dim]{dim} \hyperref[TEI.dimensions]{dimensions} \hyperref[TEI.height]{height} \hyperref[TEI.heraldry]{heraldry} \hyperref[TEI.locus]{locus} \hyperref[TEI.locusGrp]{locusGrp} \hyperref[TEI.material]{material} \hyperref[TEI.objectType]{objectType} \hyperref[TEI.origDate]{origDate} \hyperref[TEI.origPlace]{origPlace} \hyperref[TEI.secFol]{secFol} \hyperref[TEI.signatures]{signatures} \hyperref[TEI.source]{source} \hyperref[TEI.stamp]{stamp} \hyperref[TEI.watermark]{watermark} \hyperref[TEI.width]{width}\par 
    \item[namesdates: ]
   \hyperref[TEI.addName]{addName} \hyperref[TEI.affiliation]{affiliation} \hyperref[TEI.country]{country} \hyperref[TEI.forename]{forename} \hyperref[TEI.genName]{genName} \hyperref[TEI.geogName]{geogName} \hyperref[TEI.location]{location} \hyperref[TEI.nameLink]{nameLink} \hyperref[TEI.orgName]{orgName} \hyperref[TEI.persName]{persName} \hyperref[TEI.placeName]{placeName} \hyperref[TEI.region]{region} \hyperref[TEI.roleName]{roleName} \hyperref[TEI.settlement]{settlement} \hyperref[TEI.state]{state} \hyperref[TEI.surname]{surname}\par 
    \item[spoken: ]
   \hyperref[TEI.annotationBlock]{annotationBlock}\par 
    \item[transcr: ]
   \hyperref[TEI.addSpan]{addSpan} \hyperref[TEI.am]{am} \hyperref[TEI.damage]{damage} \hyperref[TEI.damageSpan]{damageSpan} \hyperref[TEI.delSpan]{delSpan} \hyperref[TEI.ex]{ex} \hyperref[TEI.fw]{fw} \hyperref[TEI.handShift]{handShift} \hyperref[TEI.listTranspose]{listTranspose} \hyperref[TEI.metamark]{metamark} \hyperref[TEI.mod]{mod} \hyperref[TEI.redo]{redo} \hyperref[TEI.restore]{restore} \hyperref[TEI.retrace]{retrace} \hyperref[TEI.secl]{secl} \hyperref[TEI.space]{space} \hyperref[TEI.subst]{subst} \hyperref[TEI.substJoin]{substJoin} \hyperref[TEI.supplied]{supplied} \hyperref[TEI.surplus]{surplus} \hyperref[TEI.undo]{undo}\par des données textuelles
    \item[{Exemple}]
  \leavevmode\bgroup\exampleFont \begin{shaded}\noindent\mbox{}{<\textbf{incipit}>}Pater noster qui es in celis{</\textbf{incipit}>}\mbox{}\newline 
{<\textbf{incipit}\hspace*{6pt}{defective}="{true}">}tatem dedit hominibus alleluia.{</\textbf{incipit}>}\mbox{}\newline 
{<\textbf{incipit}\hspace*{6pt}{type}="{biblical}">}Ghif ons huden onse dagelix broet{</\textbf{incipit}>}\mbox{}\newline 
{<\textbf{incipit}>}O ongehoerde gewerdighe christi{</\textbf{incipit}>}\mbox{}\newline 
{<\textbf{incipit}\hspace*{6pt}{type}="{lemma}">}Firmiter{</\textbf{incipit}>}\mbox{}\newline 
{<\textbf{incipit}>}Ideo dicit firmiter quia ordo fidei nostre probari non potest{</\textbf{incipit}>}\end{shaded}\egroup 


    \item[{Exemple}]
  \leavevmode\bgroup\exampleFont \begin{shaded}\noindent\mbox{}{<\textbf{incipit}>}Pater noster qui es in celis{</\textbf{incipit}>}\mbox{}\newline 
{<\textbf{incipit}\hspace*{6pt}{defective}="{true}">}tatem dedit hominibus alleluia.{</\textbf{incipit}>}\mbox{}\newline 
{<\textbf{incipit}\hspace*{6pt}{type}="{biblical}">}Ghif ons huden onse dagelix broet{</\textbf{incipit}>}\mbox{}\newline 
{<\textbf{incipit}>}O ongehoerde gewerdighe christi{</\textbf{incipit}>}\mbox{}\newline 
{<\textbf{incipit}\hspace*{6pt}{type}="{lemma}">}Firmiter{</\textbf{incipit}>}\mbox{}\newline 
{<\textbf{incipit}>}Ideo dicit firmiter quia ordo fidei nostre probari non potest{</\textbf{incipit}>}\end{shaded}\egroup 


    \item[{Modèle de contenu}]
  \mbox{}\hfill\\[-10pt]\begin{Verbatim}[fontsize=\small]
<content>
 <macroRef key="macro.phraseSeq"/>
</content>
    
\end{Verbatim}

    \item[{Schéma Declaration}]
  \mbox{}\hfill\\[-10pt]\begin{Verbatim}[fontsize=\small]
element incipit
{
   tei_att.global.attributes,
   tei_att.typed.attributes,
   tei_att.msExcerpt.attributes,
   tei_macro.phraseSeq}
\end{Verbatim}

\end{reflist}  \index{index=<index>|oddindex}\index{indexName=@indexName!<index>|oddindex}
\begin{reflist}
\item[]\begin{specHead}{TEI.index}{<index> }(entrée d'index) marque un emplacement à indexer dans un but quelconque. [\xref{http://www.tei-c.org/release/doc/tei-p5-doc/en/html/CO.html\#CONOIX}{3.8.2. Index Entries}]\end{specHead} 
    \item[{Module}]
  core
    \item[{Attributs}]
  Attributs \hyperref[TEI.att.global]{att.global} (\textit{@xml:id}, \textit{@n}, \textit{@xml:lang}, \textit{@xml:base}, \textit{@xml:space})  (\hyperref[TEI.att.global.rendition]{att.global.rendition} (\textit{@rend}, \textit{@style}, \textit{@rendition})) (\hyperref[TEI.att.global.linking]{att.global.linking} (\textit{@corresp}, \textit{@synch}, \textit{@sameAs}, \textit{@copyOf}, \textit{@next}, \textit{@prev}, \textit{@exclude}, \textit{@select})) (\hyperref[TEI.att.global.analytic]{att.global.analytic} (\textit{@ana})) (\hyperref[TEI.att.global.facs]{att.global.facs} (\textit{@facs})) (\hyperref[TEI.att.global.change]{att.global.change} (\textit{@change})) (\hyperref[TEI.att.global.responsibility]{att.global.responsibility} (\textit{@cert}, \textit{@resp})) (\hyperref[TEI.att.global.source]{att.global.source} (\textit{@source})) \hyperref[TEI.att.spanning]{att.spanning} (\textit{@spanTo}) \hfil\\[-10pt]\begin{sansreflist}
    \item[@indexName]
  donne un nom pour préciser à quel index (parmi plusieurs) appartient l'entrée d'index.
\begin{reflist}
    \item[{Statut}]
  Optionel
    \item[{Type de données}]
  \hyperref[TEI.teidata.name]{teidata.name}
    \item[{Note}]
  \par
Cet attribut permet de créer plusieurs index pour un texte donné. 
\end{reflist}  
\end{sansreflist}  
    \item[{Membre du}]
  \hyperref[TEI.model.global.meta]{model.global.meta}
    \item[{Contenu dans}]
  
    \item[analysis: ]
   \hyperref[TEI.cl]{cl} \hyperref[TEI.m]{m} \hyperref[TEI.phr]{phr} \hyperref[TEI.s]{s} \hyperref[TEI.span]{span} \hyperref[TEI.w]{w}\par 
    \item[core: ]
   \hyperref[TEI.abbr]{abbr} \hyperref[TEI.add]{add} \hyperref[TEI.addrLine]{addrLine} \hyperref[TEI.address]{address} \hyperref[TEI.author]{author} \hyperref[TEI.bibl]{bibl} \hyperref[TEI.biblScope]{biblScope} \hyperref[TEI.cit]{cit} \hyperref[TEI.citedRange]{citedRange} \hyperref[TEI.corr]{corr} \hyperref[TEI.date]{date} \hyperref[TEI.del]{del} \hyperref[TEI.distinct]{distinct} \hyperref[TEI.editor]{editor} \hyperref[TEI.email]{email} \hyperref[TEI.emph]{emph} \hyperref[TEI.expan]{expan} \hyperref[TEI.foreign]{foreign} \hyperref[TEI.gloss]{gloss} \hyperref[TEI.head]{head} \hyperref[TEI.headItem]{headItem} \hyperref[TEI.headLabel]{headLabel} \hyperref[TEI.hi]{hi} \hyperref[TEI.imprint]{imprint} \hyperref[TEI.index]{index} \hyperref[TEI.item]{item} \hyperref[TEI.l]{l} \hyperref[TEI.label]{label} \hyperref[TEI.lg]{lg} \hyperref[TEI.list]{list} \hyperref[TEI.measure]{measure} \hyperref[TEI.mentioned]{mentioned} \hyperref[TEI.name]{name} \hyperref[TEI.note]{note} \hyperref[TEI.num]{num} \hyperref[TEI.orig]{orig} \hyperref[TEI.p]{p} \hyperref[TEI.pubPlace]{pubPlace} \hyperref[TEI.publisher]{publisher} \hyperref[TEI.q]{q} \hyperref[TEI.quote]{quote} \hyperref[TEI.ref]{ref} \hyperref[TEI.reg]{reg} \hyperref[TEI.resp]{resp} \hyperref[TEI.rs]{rs} \hyperref[TEI.said]{said} \hyperref[TEI.series]{series} \hyperref[TEI.sic]{sic} \hyperref[TEI.soCalled]{soCalled} \hyperref[TEI.sp]{sp} \hyperref[TEI.speaker]{speaker} \hyperref[TEI.stage]{stage} \hyperref[TEI.street]{street} \hyperref[TEI.term]{term} \hyperref[TEI.textLang]{textLang} \hyperref[TEI.time]{time} \hyperref[TEI.title]{title} \hyperref[TEI.unclear]{unclear}\par 
    \item[figures: ]
   \hyperref[TEI.cell]{cell} \hyperref[TEI.figure]{figure} \hyperref[TEI.table]{table}\par 
    \item[header: ]
   \hyperref[TEI.authority]{authority} \hyperref[TEI.change]{change} \hyperref[TEI.classCode]{classCode} \hyperref[TEI.distributor]{distributor} \hyperref[TEI.edition]{edition} \hyperref[TEI.extent]{extent} \hyperref[TEI.funder]{funder} \hyperref[TEI.language]{language} \hyperref[TEI.licence]{licence}\par 
    \item[linking: ]
   \hyperref[TEI.ab]{ab} \hyperref[TEI.seg]{seg}\par 
    \item[msdescription: ]
   \hyperref[TEI.accMat]{accMat} \hyperref[TEI.acquisition]{acquisition} \hyperref[TEI.additions]{additions} \hyperref[TEI.catchwords]{catchwords} \hyperref[TEI.collation]{collation} \hyperref[TEI.colophon]{colophon} \hyperref[TEI.condition]{condition} \hyperref[TEI.custEvent]{custEvent} \hyperref[TEI.decoNote]{decoNote} \hyperref[TEI.explicit]{explicit} \hyperref[TEI.filiation]{filiation} \hyperref[TEI.finalRubric]{finalRubric} \hyperref[TEI.foliation]{foliation} \hyperref[TEI.heraldry]{heraldry} \hyperref[TEI.incipit]{incipit} \hyperref[TEI.layout]{layout} \hyperref[TEI.material]{material} \hyperref[TEI.msItem]{msItem} \hyperref[TEI.musicNotation]{musicNotation} \hyperref[TEI.objectType]{objectType} \hyperref[TEI.origDate]{origDate} \hyperref[TEI.origPlace]{origPlace} \hyperref[TEI.origin]{origin} \hyperref[TEI.provenance]{provenance} \hyperref[TEI.rubric]{rubric} \hyperref[TEI.secFol]{secFol} \hyperref[TEI.signatures]{signatures} \hyperref[TEI.source]{source} \hyperref[TEI.stamp]{stamp} \hyperref[TEI.summary]{summary} \hyperref[TEI.support]{support} \hyperref[TEI.surrogates]{surrogates} \hyperref[TEI.typeNote]{typeNote} \hyperref[TEI.watermark]{watermark}\par 
    \item[namesdates: ]
   \hyperref[TEI.addName]{addName} \hyperref[TEI.affiliation]{affiliation} \hyperref[TEI.country]{country} \hyperref[TEI.forename]{forename} \hyperref[TEI.genName]{genName} \hyperref[TEI.geogName]{geogName} \hyperref[TEI.nameLink]{nameLink} \hyperref[TEI.orgName]{orgName} \hyperref[TEI.persName]{persName} \hyperref[TEI.person]{person} \hyperref[TEI.personGrp]{personGrp} \hyperref[TEI.persona]{persona} \hyperref[TEI.placeName]{placeName} \hyperref[TEI.region]{region} \hyperref[TEI.roleName]{roleName} \hyperref[TEI.settlement]{settlement} \hyperref[TEI.surname]{surname}\par 
    \item[spoken: ]
   \hyperref[TEI.annotationBlock]{annotationBlock}\par 
    \item[standOff: ]
   \hyperref[TEI.listAnnotation]{listAnnotation}\par 
    \item[textstructure: ]
   \hyperref[TEI.back]{back} \hyperref[TEI.body]{body} \hyperref[TEI.div]{div} \hyperref[TEI.docAuthor]{docAuthor} \hyperref[TEI.docDate]{docDate} \hyperref[TEI.docEdition]{docEdition} \hyperref[TEI.docTitle]{docTitle} \hyperref[TEI.floatingText]{floatingText} \hyperref[TEI.front]{front} \hyperref[TEI.group]{group} \hyperref[TEI.text]{text} \hyperref[TEI.titlePage]{titlePage} \hyperref[TEI.titlePart]{titlePart}\par 
    \item[transcr: ]
   \hyperref[TEI.damage]{damage} \hyperref[TEI.fw]{fw} \hyperref[TEI.line]{line} \hyperref[TEI.metamark]{metamark} \hyperref[TEI.mod]{mod} \hyperref[TEI.restore]{restore} \hyperref[TEI.retrace]{retrace} \hyperref[TEI.secl]{secl} \hyperref[TEI.sourceDoc]{sourceDoc} \hyperref[TEI.supplied]{supplied} \hyperref[TEI.surface]{surface} \hyperref[TEI.surfaceGrp]{surfaceGrp} \hyperref[TEI.surplus]{surplus} \hyperref[TEI.zone]{zone}
    \item[{Peut contenir}]
  
    \item[core: ]
   \hyperref[TEI.index]{index} \hyperref[TEI.term]{term}
    \item[{Exemple}]
  \leavevmode\bgroup\exampleFont \begin{shaded}\noindent\mbox{} Ils [les\mbox{}\newline 
 onagres] me venaient de mon grand-père maternel, l'empereur {<\textbf{index}\hspace*{6pt}{indexName}="{NAMES}">}\mbox{}\newline 
\hspace*{6pt}{<\textbf{term}>}Saharil{</\textbf{term}>}\mbox{}\newline 
{</\textbf{index}>}, fils d'Iakhschab, fils d'{<\textbf{index}\hspace*{6pt}{indexName}="{NAMES}">}\mbox{}\newline 
\hspace*{6pt}{<\textbf{term}>}Iaarab{</\textbf{term}>}\mbox{}\newline 
{</\textbf{index}>}, fils de \mbox{}\newline 
{<\textbf{index}\hspace*{6pt}{indexName}="{NAMES}">}\mbox{}\newline 
\hspace*{6pt}{<\textbf{term}>}Kastan{</\textbf{term}>}\mbox{}\newline 
{</\textbf{index}>}\end{shaded}\egroup 


    \item[{Modèle de contenu}]
  \mbox{}\hfill\\[-10pt]\begin{Verbatim}[fontsize=\small]
<content>
 <sequence maxOccurs="unbounded"
  minOccurs="0">
  <elementRef key="term"/>
  <elementRef key="index" minOccurs="0"/>
 </sequence>
</content>
    
\end{Verbatim}

    \item[{Schéma Declaration}]
  \mbox{}\hfill\\[-10pt]\begin{Verbatim}[fontsize=\small]
element index
{
   tei_att.global.attributes,
   tei_att.spanning.attributes,
   attribute indexName { text }?,
   ( tei_term, tei_index? )*
}
\end{Verbatim}

\end{reflist}  \index{institution=<institution>|oddindex}
\begin{reflist}
\item[]\begin{specHead}{TEI.institution}{<institution> }(institution) Contient le nom d'un organisme (comme une université ou une bibliothèque), avec lequel un manuscrit est identifié ; en général c'est le nom de l'institution qui conserve ce manuscrit. [\xref{http://www.tei-c.org/release/doc/tei-p5-doc/en/html/MS.html\#msid}{10.4. The Manuscript Identifier}]\end{specHead} 
    \item[{Module}]
  msdescription
    \item[{Attributs}]
  Attributs \hyperref[TEI.att.global]{att.global} (\textit{@xml:id}, \textit{@n}, \textit{@xml:lang}, \textit{@xml:base}, \textit{@xml:space})  (\hyperref[TEI.att.global.rendition]{att.global.rendition} (\textit{@rend}, \textit{@style}, \textit{@rendition})) (\hyperref[TEI.att.global.linking]{att.global.linking} (\textit{@corresp}, \textit{@synch}, \textit{@sameAs}, \textit{@copyOf}, \textit{@next}, \textit{@prev}, \textit{@exclude}, \textit{@select})) (\hyperref[TEI.att.global.analytic]{att.global.analytic} (\textit{@ana})) (\hyperref[TEI.att.global.facs]{att.global.facs} (\textit{@facs})) (\hyperref[TEI.att.global.change]{att.global.change} (\textit{@change})) (\hyperref[TEI.att.global.responsibility]{att.global.responsibility} (\textit{@cert}, \textit{@resp})) (\hyperref[TEI.att.global.source]{att.global.source} (\textit{@source})) \hyperref[TEI.att.naming]{att.naming} (\textit{@role}, \textit{@nymRef})  (\hyperref[TEI.att.canonical]{att.canonical} (\textit{@key}, \textit{@ref}))
    \item[{Contenu dans}]
  
    \item[msdescription: ]
   \hyperref[TEI.altIdentifier]{altIdentifier} \hyperref[TEI.msIdentifier]{msIdentifier}
    \item[{Peut contenir}]
  Des données textuelles uniquement
    \item[{Modèle de contenu}]
  \fbox{\ttfamily <content>\newline
 <macroRef key="macro.xtext"/>\newline
</content>\newline
    } 
    \item[{Schéma Declaration}]
  \mbox{}\hfill\\[-10pt]\begin{Verbatim}[fontsize=\small]
element institution
{
   tei_att.global.attributes,
   tei_att.naming.attributes,
   tei_macro.xtext}
\end{Verbatim}

\end{reflist}  \index{interp=<interp>|oddindex}
\begin{reflist}
\item[]\begin{specHead}{TEI.interp}{<interp> }(interprétation) interprétation sous la forme d'une annotation concise, pouvant être liée à un passage dans un texte [\xref{http://www.tei-c.org/release/doc/tei-p5-doc/en/html/AI.html\#AISP}{17.3. Spans and Interpretations}]\end{specHead} 
    \item[{Module}]
  analysis
    \item[{Attributs}]
  Attributs \hyperref[TEI.att.global]{att.global} (\textit{@xml:id}, \textit{@n}, \textit{@xml:lang}, \textit{@xml:base}, \textit{@xml:space})  (\hyperref[TEI.att.global.rendition]{att.global.rendition} (\textit{@rend}, \textit{@style}, \textit{@rendition})) (\hyperref[TEI.att.global.linking]{att.global.linking} (\textit{@corresp}, \textit{@synch}, \textit{@sameAs}, \textit{@copyOf}, \textit{@next}, \textit{@prev}, \textit{@exclude}, \textit{@select})) (\hyperref[TEI.att.global.analytic]{att.global.analytic} (\textit{@ana})) (\hyperref[TEI.att.global.facs]{att.global.facs} (\textit{@facs})) (\hyperref[TEI.att.global.change]{att.global.change} (\textit{@change})) (\hyperref[TEI.att.global.responsibility]{att.global.responsibility} (\textit{@cert}, \textit{@resp})) (\hyperref[TEI.att.global.source]{att.global.source} (\textit{@source})) \hyperref[TEI.att.interpLike]{att.interpLike} (\textit{@type}, \textit{@inst}) 
    \item[{Membre du}]
  \hyperref[TEI.model.OAAnnotation]{model.OAAnnotation} \hyperref[TEI.model.global.meta]{model.global.meta}
    \item[{Contenu dans}]
  
    \item[analysis: ]
   \hyperref[TEI.cl]{cl} \hyperref[TEI.interpGrp]{interpGrp} \hyperref[TEI.m]{m} \hyperref[TEI.phr]{phr} \hyperref[TEI.s]{s} \hyperref[TEI.span]{span} \hyperref[TEI.w]{w}\par 
    \item[core: ]
   \hyperref[TEI.abbr]{abbr} \hyperref[TEI.add]{add} \hyperref[TEI.addrLine]{addrLine} \hyperref[TEI.address]{address} \hyperref[TEI.author]{author} \hyperref[TEI.bibl]{bibl} \hyperref[TEI.biblScope]{biblScope} \hyperref[TEI.cit]{cit} \hyperref[TEI.citedRange]{citedRange} \hyperref[TEI.corr]{corr} \hyperref[TEI.date]{date} \hyperref[TEI.del]{del} \hyperref[TEI.distinct]{distinct} \hyperref[TEI.editor]{editor} \hyperref[TEI.email]{email} \hyperref[TEI.emph]{emph} \hyperref[TEI.expan]{expan} \hyperref[TEI.foreign]{foreign} \hyperref[TEI.gloss]{gloss} \hyperref[TEI.head]{head} \hyperref[TEI.headItem]{headItem} \hyperref[TEI.headLabel]{headLabel} \hyperref[TEI.hi]{hi} \hyperref[TEI.imprint]{imprint} \hyperref[TEI.item]{item} \hyperref[TEI.l]{l} \hyperref[TEI.label]{label} \hyperref[TEI.lg]{lg} \hyperref[TEI.list]{list} \hyperref[TEI.measure]{measure} \hyperref[TEI.mentioned]{mentioned} \hyperref[TEI.name]{name} \hyperref[TEI.note]{note} \hyperref[TEI.num]{num} \hyperref[TEI.orig]{orig} \hyperref[TEI.p]{p} \hyperref[TEI.pubPlace]{pubPlace} \hyperref[TEI.publisher]{publisher} \hyperref[TEI.q]{q} \hyperref[TEI.quote]{quote} \hyperref[TEI.ref]{ref} \hyperref[TEI.reg]{reg} \hyperref[TEI.resp]{resp} \hyperref[TEI.rs]{rs} \hyperref[TEI.said]{said} \hyperref[TEI.series]{series} \hyperref[TEI.sic]{sic} \hyperref[TEI.soCalled]{soCalled} \hyperref[TEI.sp]{sp} \hyperref[TEI.speaker]{speaker} \hyperref[TEI.stage]{stage} \hyperref[TEI.street]{street} \hyperref[TEI.term]{term} \hyperref[TEI.textLang]{textLang} \hyperref[TEI.time]{time} \hyperref[TEI.title]{title} \hyperref[TEI.unclear]{unclear}\par 
    \item[figures: ]
   \hyperref[TEI.cell]{cell} \hyperref[TEI.figure]{figure} \hyperref[TEI.table]{table}\par 
    \item[header: ]
   \hyperref[TEI.authority]{authority} \hyperref[TEI.change]{change} \hyperref[TEI.classCode]{classCode} \hyperref[TEI.distributor]{distributor} \hyperref[TEI.edition]{edition} \hyperref[TEI.extent]{extent} \hyperref[TEI.funder]{funder} \hyperref[TEI.language]{language} \hyperref[TEI.licence]{licence}\par 
    \item[linking: ]
   \hyperref[TEI.ab]{ab} \hyperref[TEI.seg]{seg}\par 
    \item[msdescription: ]
   \hyperref[TEI.accMat]{accMat} \hyperref[TEI.acquisition]{acquisition} \hyperref[TEI.additions]{additions} \hyperref[TEI.catchwords]{catchwords} \hyperref[TEI.collation]{collation} \hyperref[TEI.colophon]{colophon} \hyperref[TEI.condition]{condition} \hyperref[TEI.custEvent]{custEvent} \hyperref[TEI.decoNote]{decoNote} \hyperref[TEI.explicit]{explicit} \hyperref[TEI.filiation]{filiation} \hyperref[TEI.finalRubric]{finalRubric} \hyperref[TEI.foliation]{foliation} \hyperref[TEI.heraldry]{heraldry} \hyperref[TEI.incipit]{incipit} \hyperref[TEI.layout]{layout} \hyperref[TEI.material]{material} \hyperref[TEI.msItem]{msItem} \hyperref[TEI.musicNotation]{musicNotation} \hyperref[TEI.objectType]{objectType} \hyperref[TEI.origDate]{origDate} \hyperref[TEI.origPlace]{origPlace} \hyperref[TEI.origin]{origin} \hyperref[TEI.provenance]{provenance} \hyperref[TEI.rubric]{rubric} \hyperref[TEI.secFol]{secFol} \hyperref[TEI.signatures]{signatures} \hyperref[TEI.source]{source} \hyperref[TEI.stamp]{stamp} \hyperref[TEI.summary]{summary} \hyperref[TEI.support]{support} \hyperref[TEI.surrogates]{surrogates} \hyperref[TEI.typeNote]{typeNote} \hyperref[TEI.watermark]{watermark}\par 
    \item[namesdates: ]
   \hyperref[TEI.addName]{addName} \hyperref[TEI.affiliation]{affiliation} \hyperref[TEI.country]{country} \hyperref[TEI.forename]{forename} \hyperref[TEI.genName]{genName} \hyperref[TEI.geogName]{geogName} \hyperref[TEI.nameLink]{nameLink} \hyperref[TEI.orgName]{orgName} \hyperref[TEI.persName]{persName} \hyperref[TEI.person]{person} \hyperref[TEI.personGrp]{personGrp} \hyperref[TEI.persona]{persona} \hyperref[TEI.placeName]{placeName} \hyperref[TEI.region]{region} \hyperref[TEI.roleName]{roleName} \hyperref[TEI.settlement]{settlement} \hyperref[TEI.surname]{surname}\par 
    \item[spoken: ]
   \hyperref[TEI.annotationBlock]{annotationBlock}\par 
    \item[standOff: ]
   \hyperref[TEI.listAnnotation]{listAnnotation}\par 
    \item[textstructure: ]
   \hyperref[TEI.back]{back} \hyperref[TEI.body]{body} \hyperref[TEI.div]{div} \hyperref[TEI.docAuthor]{docAuthor} \hyperref[TEI.docDate]{docDate} \hyperref[TEI.docEdition]{docEdition} \hyperref[TEI.docTitle]{docTitle} \hyperref[TEI.floatingText]{floatingText} \hyperref[TEI.front]{front} \hyperref[TEI.group]{group} \hyperref[TEI.text]{text} \hyperref[TEI.titlePage]{titlePage} \hyperref[TEI.titlePart]{titlePart}\par 
    \item[transcr: ]
   \hyperref[TEI.damage]{damage} \hyperref[TEI.fw]{fw} \hyperref[TEI.line]{line} \hyperref[TEI.metamark]{metamark} \hyperref[TEI.mod]{mod} \hyperref[TEI.restore]{restore} \hyperref[TEI.retrace]{retrace} \hyperref[TEI.secl]{secl} \hyperref[TEI.sourceDoc]{sourceDoc} \hyperref[TEI.supplied]{supplied} \hyperref[TEI.surface]{surface} \hyperref[TEI.surfaceGrp]{surfaceGrp} \hyperref[TEI.surplus]{surplus} \hyperref[TEI.zone]{zone}
    \item[{Peut contenir}]
  
    \item[core: ]
   \hyperref[TEI.desc]{desc}\par des données textuelles
    \item[{Note}]
  \par
Generally, each \hyperref[TEI.interp]{<interp>} element carries an {\itshape xml:id} attribute. This permits the encoder to explicitly associate the interpretation represented by the content of an \hyperref[TEI.interp]{<interp>} with any textual element through its {\itshape ana} attribute.\par
Alternatively (or, in addition) an \hyperref[TEI.interp]{<interp>} may carry an {\itshape inst} attribute which points to one or more textual elements to which the analysis represented by the content of the \hyperref[TEI.interp]{<interp>} applies.
    \item[{Exemple}]
  \leavevmode\bgroup\exampleFont \begin{shaded}\noindent\mbox{}{<\textbf{interp}\hspace*{6pt}{type}="{structuralunit}">}conséquence{</\textbf{interp}>}\end{shaded}\egroup 


    \item[{Modèle de contenu}]
  \mbox{}\hfill\\[-10pt]\begin{Verbatim}[fontsize=\small]
<content>
 <alternate maxOccurs="unbounded"
  minOccurs="0">
  <textNode/>
  <classRef key="model.gLike"/>
  <classRef key="model.descLike"/>
  <classRef key="model.certLike"/>
 </alternate>
</content>
    
\end{Verbatim}

    \item[{Schéma Declaration}]
  \mbox{}\hfill\\[-10pt]\begin{Verbatim}[fontsize=\small]
element interp
{
   tei_att.global.attributes,
   tei_att.interpLike.attributes,
   ( text | tei_model.gLike | tei_model.descLike | tei_model.certLike )*
}
\end{Verbatim}

\end{reflist}  \index{interpGrp=<interpGrp>|oddindex}
\begin{reflist}
\item[]\begin{specHead}{TEI.interpGrp}{<interpGrp> }(groupe d'interprétations) regroupe un ensemble d'interprétations ayant en commun une mention de responsabilité ou un type [\xref{http://www.tei-c.org/release/doc/tei-p5-doc/en/html/AI.html\#AISP}{17.3. Spans and Interpretations}]\end{specHead} 
    \item[{Module}]
  analysis
    \item[{Attributs}]
  Attributs \hyperref[TEI.att.global]{att.global} (\textit{@xml:id}, \textit{@n}, \textit{@xml:lang}, \textit{@xml:base}, \textit{@xml:space})  (\hyperref[TEI.att.global.rendition]{att.global.rendition} (\textit{@rend}, \textit{@style}, \textit{@rendition})) (\hyperref[TEI.att.global.linking]{att.global.linking} (\textit{@corresp}, \textit{@synch}, \textit{@sameAs}, \textit{@copyOf}, \textit{@next}, \textit{@prev}, \textit{@exclude}, \textit{@select})) (\hyperref[TEI.att.global.analytic]{att.global.analytic} (\textit{@ana})) (\hyperref[TEI.att.global.facs]{att.global.facs} (\textit{@facs})) (\hyperref[TEI.att.global.change]{att.global.change} (\textit{@change})) (\hyperref[TEI.att.global.responsibility]{att.global.responsibility} (\textit{@cert}, \textit{@resp})) (\hyperref[TEI.att.global.source]{att.global.source} (\textit{@source})) \hyperref[TEI.att.interpLike]{att.interpLike} (\textit{@type}, \textit{@inst}) 
    \item[{Membre du}]
  \hyperref[TEI.model.global.meta]{model.global.meta}
    \item[{Contenu dans}]
  
    \item[analysis: ]
   \hyperref[TEI.cl]{cl} \hyperref[TEI.m]{m} \hyperref[TEI.phr]{phr} \hyperref[TEI.s]{s} \hyperref[TEI.span]{span} \hyperref[TEI.w]{w}\par 
    \item[core: ]
   \hyperref[TEI.abbr]{abbr} \hyperref[TEI.add]{add} \hyperref[TEI.addrLine]{addrLine} \hyperref[TEI.address]{address} \hyperref[TEI.author]{author} \hyperref[TEI.bibl]{bibl} \hyperref[TEI.biblScope]{biblScope} \hyperref[TEI.cit]{cit} \hyperref[TEI.citedRange]{citedRange} \hyperref[TEI.corr]{corr} \hyperref[TEI.date]{date} \hyperref[TEI.del]{del} \hyperref[TEI.distinct]{distinct} \hyperref[TEI.editor]{editor} \hyperref[TEI.email]{email} \hyperref[TEI.emph]{emph} \hyperref[TEI.expan]{expan} \hyperref[TEI.foreign]{foreign} \hyperref[TEI.gloss]{gloss} \hyperref[TEI.head]{head} \hyperref[TEI.headItem]{headItem} \hyperref[TEI.headLabel]{headLabel} \hyperref[TEI.hi]{hi} \hyperref[TEI.imprint]{imprint} \hyperref[TEI.item]{item} \hyperref[TEI.l]{l} \hyperref[TEI.label]{label} \hyperref[TEI.lg]{lg} \hyperref[TEI.list]{list} \hyperref[TEI.measure]{measure} \hyperref[TEI.mentioned]{mentioned} \hyperref[TEI.name]{name} \hyperref[TEI.note]{note} \hyperref[TEI.num]{num} \hyperref[TEI.orig]{orig} \hyperref[TEI.p]{p} \hyperref[TEI.pubPlace]{pubPlace} \hyperref[TEI.publisher]{publisher} \hyperref[TEI.q]{q} \hyperref[TEI.quote]{quote} \hyperref[TEI.ref]{ref} \hyperref[TEI.reg]{reg} \hyperref[TEI.resp]{resp} \hyperref[TEI.rs]{rs} \hyperref[TEI.said]{said} \hyperref[TEI.series]{series} \hyperref[TEI.sic]{sic} \hyperref[TEI.soCalled]{soCalled} \hyperref[TEI.sp]{sp} \hyperref[TEI.speaker]{speaker} \hyperref[TEI.stage]{stage} \hyperref[TEI.street]{street} \hyperref[TEI.term]{term} \hyperref[TEI.textLang]{textLang} \hyperref[TEI.time]{time} \hyperref[TEI.title]{title} \hyperref[TEI.unclear]{unclear}\par 
    \item[figures: ]
   \hyperref[TEI.cell]{cell} \hyperref[TEI.figure]{figure} \hyperref[TEI.table]{table}\par 
    \item[header: ]
   \hyperref[TEI.authority]{authority} \hyperref[TEI.change]{change} \hyperref[TEI.classCode]{classCode} \hyperref[TEI.distributor]{distributor} \hyperref[TEI.edition]{edition} \hyperref[TEI.extent]{extent} \hyperref[TEI.funder]{funder} \hyperref[TEI.language]{language} \hyperref[TEI.licence]{licence}\par 
    \item[linking: ]
   \hyperref[TEI.ab]{ab} \hyperref[TEI.seg]{seg}\par 
    \item[msdescription: ]
   \hyperref[TEI.accMat]{accMat} \hyperref[TEI.acquisition]{acquisition} \hyperref[TEI.additions]{additions} \hyperref[TEI.catchwords]{catchwords} \hyperref[TEI.collation]{collation} \hyperref[TEI.colophon]{colophon} \hyperref[TEI.condition]{condition} \hyperref[TEI.custEvent]{custEvent} \hyperref[TEI.decoNote]{decoNote} \hyperref[TEI.explicit]{explicit} \hyperref[TEI.filiation]{filiation} \hyperref[TEI.finalRubric]{finalRubric} \hyperref[TEI.foliation]{foliation} \hyperref[TEI.heraldry]{heraldry} \hyperref[TEI.incipit]{incipit} \hyperref[TEI.layout]{layout} \hyperref[TEI.material]{material} \hyperref[TEI.msItem]{msItem} \hyperref[TEI.musicNotation]{musicNotation} \hyperref[TEI.objectType]{objectType} \hyperref[TEI.origDate]{origDate} \hyperref[TEI.origPlace]{origPlace} \hyperref[TEI.origin]{origin} \hyperref[TEI.provenance]{provenance} \hyperref[TEI.rubric]{rubric} \hyperref[TEI.secFol]{secFol} \hyperref[TEI.signatures]{signatures} \hyperref[TEI.source]{source} \hyperref[TEI.stamp]{stamp} \hyperref[TEI.summary]{summary} \hyperref[TEI.support]{support} \hyperref[TEI.surrogates]{surrogates} \hyperref[TEI.typeNote]{typeNote} \hyperref[TEI.watermark]{watermark}\par 
    \item[namesdates: ]
   \hyperref[TEI.addName]{addName} \hyperref[TEI.affiliation]{affiliation} \hyperref[TEI.country]{country} \hyperref[TEI.forename]{forename} \hyperref[TEI.genName]{genName} \hyperref[TEI.geogName]{geogName} \hyperref[TEI.nameLink]{nameLink} \hyperref[TEI.orgName]{orgName} \hyperref[TEI.persName]{persName} \hyperref[TEI.person]{person} \hyperref[TEI.personGrp]{personGrp} \hyperref[TEI.persona]{persona} \hyperref[TEI.placeName]{placeName} \hyperref[TEI.region]{region} \hyperref[TEI.roleName]{roleName} \hyperref[TEI.settlement]{settlement} \hyperref[TEI.surname]{surname}\par 
    \item[spoken: ]
   \hyperref[TEI.annotationBlock]{annotationBlock}\par 
    \item[standOff: ]
   \hyperref[TEI.listAnnotation]{listAnnotation}\par 
    \item[textstructure: ]
   \hyperref[TEI.back]{back} \hyperref[TEI.body]{body} \hyperref[TEI.div]{div} \hyperref[TEI.docAuthor]{docAuthor} \hyperref[TEI.docDate]{docDate} \hyperref[TEI.docEdition]{docEdition} \hyperref[TEI.docTitle]{docTitle} \hyperref[TEI.floatingText]{floatingText} \hyperref[TEI.front]{front} \hyperref[TEI.group]{group} \hyperref[TEI.text]{text} \hyperref[TEI.titlePage]{titlePage} \hyperref[TEI.titlePart]{titlePart}\par 
    \item[transcr: ]
   \hyperref[TEI.damage]{damage} \hyperref[TEI.fw]{fw} \hyperref[TEI.line]{line} \hyperref[TEI.metamark]{metamark} \hyperref[TEI.mod]{mod} \hyperref[TEI.restore]{restore} \hyperref[TEI.retrace]{retrace} \hyperref[TEI.secl]{secl} \hyperref[TEI.sourceDoc]{sourceDoc} \hyperref[TEI.supplied]{supplied} \hyperref[TEI.surface]{surface} \hyperref[TEI.surfaceGrp]{surfaceGrp} \hyperref[TEI.surplus]{surplus} \hyperref[TEI.zone]{zone}
    \item[{Peut contenir}]
  
    \item[analysis: ]
   \hyperref[TEI.interp]{interp}\par 
    \item[core: ]
   \hyperref[TEI.desc]{desc}
    \item[{Note}]
  \par
Un nombre quelconque d'éléments \hyperref[TEI.interp]{<interp>}.
    \item[{Exemple}]
  \leavevmode\bgroup\exampleFont \begin{shaded}\noindent\mbox{}{<\textbf{interpGrp}\hspace*{6pt}{resp}="{\#fr\textunderscore TMA}"\mbox{}\newline 
\hspace*{6pt}{type}="{structuralunit}">}\mbox{}\newline 
\hspace*{6pt}{<\textbf{desc}>}Organisation structurelle de base{</\textbf{desc}>}\mbox{}\newline 
\hspace*{6pt}{<\textbf{interp}\hspace*{6pt}{xml:id}="{fr\textunderscore I1}">}introduction{</\textbf{interp}>}\mbox{}\newline 
\hspace*{6pt}{<\textbf{interp}\hspace*{6pt}{xml:id}="{fr\textunderscore I2}">}conflit{</\textbf{interp}>}\mbox{}\newline 
\hspace*{6pt}{<\textbf{interp}\hspace*{6pt}{xml:id}="{fr\textunderscore I3}">}apogée{</\textbf{interp}>}\mbox{}\newline 
\hspace*{6pt}{<\textbf{interp}\hspace*{6pt}{xml:id}="{fr\textunderscore I4}">}vengeance{</\textbf{interp}>}\mbox{}\newline 
\hspace*{6pt}{<\textbf{interp}\hspace*{6pt}{xml:id}="{fr\textunderscore I5}">}reconciliation{</\textbf{interp}>}\mbox{}\newline 
\hspace*{6pt}{<\textbf{interp}\hspace*{6pt}{xml:id}="{fr\textunderscore I6}">}conséquence{</\textbf{interp}>}\mbox{}\newline 
{</\textbf{interpGrp}>}\end{shaded}\egroup 


    \item[{Modèle de contenu}]
  \mbox{}\hfill\\[-10pt]\begin{Verbatim}[fontsize=\small]
<content>
 <sequence maxOccurs="1" minOccurs="1">
  <classRef key="model.descLike"
   maxOccurs="unbounded" minOccurs="0"/>
  <elementRef key="interp"
   maxOccurs="unbounded" minOccurs="1"/>
 </sequence>
</content>
    
\end{Verbatim}

    \item[{Schéma Declaration}]
  \mbox{}\hfill\\[-10pt]\begin{Verbatim}[fontsize=\small]
element interpGrp
{
   tei_att.global.attributes,
   tei_att.interpLike.attributes,
   ( tei_model.descLike*, tei_interp+ )
}
\end{Verbatim}

\end{reflist}  \index{item=<item>|oddindex}
\begin{reflist}
\item[]\begin{specHead}{TEI.item}{<item> }contient un composant d'une liste. [\xref{http://www.tei-c.org/release/doc/tei-p5-doc/en/html/CO.html\#COLI}{3.7. Lists} \xref{http://www.tei-c.org/release/doc/tei-p5-doc/en/html/HD.html\#HD6}{2.6. The Revision Description}]\end{specHead} 
    \item[{Module}]
  core
    \item[{Attributs}]
  Attributs \hyperref[TEI.att.global]{att.global} (\textit{@xml:id}, \textit{@n}, \textit{@xml:lang}, \textit{@xml:base}, \textit{@xml:space})  (\hyperref[TEI.att.global.rendition]{att.global.rendition} (\textit{@rend}, \textit{@style}, \textit{@rendition})) (\hyperref[TEI.att.global.linking]{att.global.linking} (\textit{@corresp}, \textit{@synch}, \textit{@sameAs}, \textit{@copyOf}, \textit{@next}, \textit{@prev}, \textit{@exclude}, \textit{@select})) (\hyperref[TEI.att.global.analytic]{att.global.analytic} (\textit{@ana})) (\hyperref[TEI.att.global.facs]{att.global.facs} (\textit{@facs})) (\hyperref[TEI.att.global.change]{att.global.change} (\textit{@change})) (\hyperref[TEI.att.global.responsibility]{att.global.responsibility} (\textit{@cert}, \textit{@resp})) (\hyperref[TEI.att.global.source]{att.global.source} (\textit{@source})) \hyperref[TEI.att.sortable]{att.sortable} (\textit{@sortKey}) 
    \item[{Contenu dans}]
  
    \item[core: ]
   \hyperref[TEI.list]{list}
    \item[{Peut contenir}]
  
    \item[analysis: ]
   \hyperref[TEI.c]{c} \hyperref[TEI.cl]{cl} \hyperref[TEI.interp]{interp} \hyperref[TEI.interpGrp]{interpGrp} \hyperref[TEI.m]{m} \hyperref[TEI.pc]{pc} \hyperref[TEI.phr]{phr} \hyperref[TEI.s]{s} \hyperref[TEI.span]{span} \hyperref[TEI.spanGrp]{spanGrp} \hyperref[TEI.w]{w}\par 
    \item[core: ]
   \hyperref[TEI.abbr]{abbr} \hyperref[TEI.add]{add} \hyperref[TEI.address]{address} \hyperref[TEI.bibl]{bibl} \hyperref[TEI.biblStruct]{biblStruct} \hyperref[TEI.binaryObject]{binaryObject} \hyperref[TEI.cb]{cb} \hyperref[TEI.choice]{choice} \hyperref[TEI.cit]{cit} \hyperref[TEI.corr]{corr} \hyperref[TEI.date]{date} \hyperref[TEI.del]{del} \hyperref[TEI.desc]{desc} \hyperref[TEI.distinct]{distinct} \hyperref[TEI.email]{email} \hyperref[TEI.emph]{emph} \hyperref[TEI.expan]{expan} \hyperref[TEI.foreign]{foreign} \hyperref[TEI.gap]{gap} \hyperref[TEI.gb]{gb} \hyperref[TEI.gloss]{gloss} \hyperref[TEI.graphic]{graphic} \hyperref[TEI.hi]{hi} \hyperref[TEI.index]{index} \hyperref[TEI.l]{l} \hyperref[TEI.label]{label} \hyperref[TEI.lb]{lb} \hyperref[TEI.lg]{lg} \hyperref[TEI.list]{list} \hyperref[TEI.listBibl]{listBibl} \hyperref[TEI.measure]{measure} \hyperref[TEI.measureGrp]{measureGrp} \hyperref[TEI.media]{media} \hyperref[TEI.mentioned]{mentioned} \hyperref[TEI.milestone]{milestone} \hyperref[TEI.name]{name} \hyperref[TEI.note]{note} \hyperref[TEI.num]{num} \hyperref[TEI.orig]{orig} \hyperref[TEI.p]{p} \hyperref[TEI.pb]{pb} \hyperref[TEI.ptr]{ptr} \hyperref[TEI.q]{q} \hyperref[TEI.quote]{quote} \hyperref[TEI.ref]{ref} \hyperref[TEI.reg]{reg} \hyperref[TEI.rs]{rs} \hyperref[TEI.said]{said} \hyperref[TEI.sic]{sic} \hyperref[TEI.soCalled]{soCalled} \hyperref[TEI.sp]{sp} \hyperref[TEI.stage]{stage} \hyperref[TEI.term]{term} \hyperref[TEI.time]{time} \hyperref[TEI.title]{title} \hyperref[TEI.unclear]{unclear}\par 
    \item[derived-module-tei.istex: ]
   \hyperref[TEI.math]{math} \hyperref[TEI.mrow]{mrow}\par 
    \item[figures: ]
   \hyperref[TEI.figure]{figure} \hyperref[TEI.formula]{formula} \hyperref[TEI.notatedMusic]{notatedMusic} \hyperref[TEI.table]{table}\par 
    \item[header: ]
   \hyperref[TEI.biblFull]{biblFull} \hyperref[TEI.idno]{idno}\par 
    \item[iso-fs: ]
   \hyperref[TEI.fLib]{fLib} \hyperref[TEI.fs]{fs} \hyperref[TEI.fvLib]{fvLib}\par 
    \item[linking: ]
   \hyperref[TEI.ab]{ab} \hyperref[TEI.alt]{alt} \hyperref[TEI.altGrp]{altGrp} \hyperref[TEI.anchor]{anchor} \hyperref[TEI.join]{join} \hyperref[TEI.joinGrp]{joinGrp} \hyperref[TEI.link]{link} \hyperref[TEI.linkGrp]{linkGrp} \hyperref[TEI.seg]{seg} \hyperref[TEI.timeline]{timeline}\par 
    \item[msdescription: ]
   \hyperref[TEI.catchwords]{catchwords} \hyperref[TEI.depth]{depth} \hyperref[TEI.dim]{dim} \hyperref[TEI.dimensions]{dimensions} \hyperref[TEI.height]{height} \hyperref[TEI.heraldry]{heraldry} \hyperref[TEI.locus]{locus} \hyperref[TEI.locusGrp]{locusGrp} \hyperref[TEI.material]{material} \hyperref[TEI.msDesc]{msDesc} \hyperref[TEI.objectType]{objectType} \hyperref[TEI.origDate]{origDate} \hyperref[TEI.origPlace]{origPlace} \hyperref[TEI.secFol]{secFol} \hyperref[TEI.signatures]{signatures} \hyperref[TEI.source]{source} \hyperref[TEI.stamp]{stamp} \hyperref[TEI.watermark]{watermark} \hyperref[TEI.width]{width}\par 
    \item[namesdates: ]
   \hyperref[TEI.addName]{addName} \hyperref[TEI.affiliation]{affiliation} \hyperref[TEI.country]{country} \hyperref[TEI.forename]{forename} \hyperref[TEI.genName]{genName} \hyperref[TEI.geogName]{geogName} \hyperref[TEI.listOrg]{listOrg} \hyperref[TEI.listPlace]{listPlace} \hyperref[TEI.location]{location} \hyperref[TEI.nameLink]{nameLink} \hyperref[TEI.orgName]{orgName} \hyperref[TEI.persName]{persName} \hyperref[TEI.placeName]{placeName} \hyperref[TEI.region]{region} \hyperref[TEI.roleName]{roleName} \hyperref[TEI.settlement]{settlement} \hyperref[TEI.state]{state} \hyperref[TEI.surname]{surname}\par 
    \item[spoken: ]
   \hyperref[TEI.annotationBlock]{annotationBlock}\par 
    \item[textstructure: ]
   \hyperref[TEI.floatingText]{floatingText}\par 
    \item[transcr: ]
   \hyperref[TEI.addSpan]{addSpan} \hyperref[TEI.am]{am} \hyperref[TEI.damage]{damage} \hyperref[TEI.damageSpan]{damageSpan} \hyperref[TEI.delSpan]{delSpan} \hyperref[TEI.ex]{ex} \hyperref[TEI.fw]{fw} \hyperref[TEI.handShift]{handShift} \hyperref[TEI.listTranspose]{listTranspose} \hyperref[TEI.metamark]{metamark} \hyperref[TEI.mod]{mod} \hyperref[TEI.redo]{redo} \hyperref[TEI.restore]{restore} \hyperref[TEI.retrace]{retrace} \hyperref[TEI.secl]{secl} \hyperref[TEI.space]{space} \hyperref[TEI.subst]{subst} \hyperref[TEI.substJoin]{substJoin} \hyperref[TEI.supplied]{supplied} \hyperref[TEI.surplus]{surplus} \hyperref[TEI.undo]{undo}\par des données textuelles
    \item[{Note}]
  \par
Peut contenir un texte ou une succession d'extraits.\par
Toute chaîne de caractères utilisée pour étiqueter un item de liste dans le texte peut être utilisée comme valeur de l'attribut global {\itshape n}, mais il n'est pas obligatoire de noter explicitement cette numérotation. Dans les listes ordonnées, l'attribut {\itshape n} de l'élément \hyperref[TEI.item]{<item>} est par définition synonyme de l'utilisation de l'élément \hyperref[TEI.label]{<label>} pour noter le numéro de l'item de la liste. Pour les glossaires toutefois, le terme défini doit être donné avec l'élément \hyperref[TEI.label]{<label>}, et non pas {\itshape n}.
    \item[{Exemple}]
  \leavevmode\bgroup\exampleFont \begin{shaded}\noindent\mbox{}{<\textbf{list}\hspace*{6pt}{type}="{unordered}">}\mbox{}\newline 
\hspace*{6pt}{<\textbf{head}>}Tentative d'inventaire de quelques-unes des choses qui ont été trouvées dans\mbox{}\newline 
\hspace*{6pt}\hspace*{6pt} les escaliers au fil des ans.{</\textbf{head}>}\mbox{}\newline 
\hspace*{6pt}{<\textbf{item}>}Plusieurs photos, dont celle d'une jeune fille de quinze{<\textbf{lb}/>} ans vêtue d'un slip\mbox{}\newline 
\hspace*{6pt}\hspace*{6pt} de bain noir et d'un chandail blanc, agenouillée sur une plage,{</\textbf{item}>}\mbox{}\newline 
\hspace*{6pt}{<\textbf{item}>}un réveil radio de toute évidence destiné à un réparateur, dans un sac plastique\mbox{}\newline 
\hspace*{6pt}\hspace*{6pt} des établissements Nicolas,{</\textbf{item}>}\mbox{}\newline 
\hspace*{6pt}{<\textbf{item}>}un soulier noir orné de brillants,{</\textbf{item}>}\mbox{}\newline 
\hspace*{6pt}{<\textbf{item}>}une mule en chevreau doré,{</\textbf{item}>}\mbox{}\newline 
\hspace*{6pt}{<\textbf{item}>}une boîte de pastilles Géraudel contre la toux.{</\textbf{item}>}\mbox{}\newline 
{</\textbf{list}>}\end{shaded}\egroup 


    \item[{Modèle de contenu}]
  \mbox{}\hfill\\[-10pt]\begin{Verbatim}[fontsize=\small]
<content>
 <macroRef key="macro.specialPara"/>
</content>
    
\end{Verbatim}

    \item[{Schéma Declaration}]
  \mbox{}\hfill\\[-10pt]\begin{Verbatim}[fontsize=\small]
element item
{
   tei_att.global.attributes,
   tei_att.sortable.attributes,
   tei_macro.specialPara}
\end{Verbatim}

\end{reflist}  \index{join=<join>|oddindex}\index{result=@result!<join>|oddindex}\index{scope=@scope!<join>|oddindex}
\begin{reflist}
\item[]\begin{specHead}{TEI.join}{<join> }(jointure) identifie un segment de texte, qui peut être fragmenté, en pointant vers les éléments éventuellement dispersés qui le composent. [\xref{http://www.tei-c.org/release/doc/tei-p5-doc/en/html/SA.html\#SAAG}{16.7. Aggregation}]\end{specHead} 
    \item[{Module}]
  linking
    \item[{Attributs}]
  Attributs \hyperref[TEI.att.global]{att.global} (\textit{@xml:id}, \textit{@n}, \textit{@xml:lang}, \textit{@xml:base}, \textit{@xml:space})  (\hyperref[TEI.att.global.rendition]{att.global.rendition} (\textit{@rend}, \textit{@style}, \textit{@rendition})) (\hyperref[TEI.att.global.linking]{att.global.linking} (\textit{@corresp}, \textit{@synch}, \textit{@sameAs}, \textit{@copyOf}, \textit{@next}, \textit{@prev}, \textit{@exclude}, \textit{@select})) (\hyperref[TEI.att.global.analytic]{att.global.analytic} (\textit{@ana})) (\hyperref[TEI.att.global.facs]{att.global.facs} (\textit{@facs})) (\hyperref[TEI.att.global.change]{att.global.change} (\textit{@change})) (\hyperref[TEI.att.global.responsibility]{att.global.responsibility} (\textit{@cert}, \textit{@resp})) (\hyperref[TEI.att.global.source]{att.global.source} (\textit{@source})) \hyperref[TEI.att.pointing]{att.pointing} (\textit{@targetLang}, \textit{@target}, \textit{@evaluate}) \hyperref[TEI.att.typed]{att.typed} (\textit{@type}, \textit{@subtype}) \hfil\\[-10pt]\begin{sansreflist}
    \item[@result]
  spécifie le nom de l'élément que cette agrégation est censée former.
\begin{reflist}
    \item[{Statut}]
  Optionel
    \item[{Type de données}]
  \hyperref[TEI.teidata.name]{teidata.name}
\end{reflist}  
    \item[@scope]
  indique si les cibles à réunir incluent l'intégralité de l'élément indiqué (le sous-arbre entier y compris sa racine) ou seulement les enfants de la cible (les branches du sous-arbre).
\begin{reflist}
    \item[{Statut}]
  Recommendé
    \item[{Type de données}]
  \hyperref[TEI.teidata.enumerated]{teidata.enumerated}
    \item[{Les valeurs autorisées sont:}]
  \begin{description}

\item[{root}]les sous-arbres dotés de leur racine qui sont désignés par l'attribut {\itshape target} sont joints ; chaque sous-arbre devient un fils de l'élément virtuel créé par la jointure.{[Valeur par défaut] }
\item[{branches}]les fils des sous-arbres désignés par l'attribut {\itshape target} deviennent les fils de l'élément virtuel (c'est-à-dire que les racines des sous-arbres disparaissent).
\end{description} 
\end{reflist}  
\end{sansreflist}  
    \item[{Membre du}]
  \hyperref[TEI.model.global.meta]{model.global.meta}
    \item[{Contenu dans}]
  
    \item[analysis: ]
   \hyperref[TEI.cl]{cl} \hyperref[TEI.m]{m} \hyperref[TEI.phr]{phr} \hyperref[TEI.s]{s} \hyperref[TEI.span]{span} \hyperref[TEI.w]{w}\par 
    \item[core: ]
   \hyperref[TEI.abbr]{abbr} \hyperref[TEI.add]{add} \hyperref[TEI.addrLine]{addrLine} \hyperref[TEI.address]{address} \hyperref[TEI.author]{author} \hyperref[TEI.bibl]{bibl} \hyperref[TEI.biblScope]{biblScope} \hyperref[TEI.cit]{cit} \hyperref[TEI.citedRange]{citedRange} \hyperref[TEI.corr]{corr} \hyperref[TEI.date]{date} \hyperref[TEI.del]{del} \hyperref[TEI.distinct]{distinct} \hyperref[TEI.editor]{editor} \hyperref[TEI.email]{email} \hyperref[TEI.emph]{emph} \hyperref[TEI.expan]{expan} \hyperref[TEI.foreign]{foreign} \hyperref[TEI.gloss]{gloss} \hyperref[TEI.head]{head} \hyperref[TEI.headItem]{headItem} \hyperref[TEI.headLabel]{headLabel} \hyperref[TEI.hi]{hi} \hyperref[TEI.imprint]{imprint} \hyperref[TEI.item]{item} \hyperref[TEI.l]{l} \hyperref[TEI.label]{label} \hyperref[TEI.lg]{lg} \hyperref[TEI.list]{list} \hyperref[TEI.measure]{measure} \hyperref[TEI.mentioned]{mentioned} \hyperref[TEI.name]{name} \hyperref[TEI.note]{note} \hyperref[TEI.num]{num} \hyperref[TEI.orig]{orig} \hyperref[TEI.p]{p} \hyperref[TEI.pubPlace]{pubPlace} \hyperref[TEI.publisher]{publisher} \hyperref[TEI.q]{q} \hyperref[TEI.quote]{quote} \hyperref[TEI.ref]{ref} \hyperref[TEI.reg]{reg} \hyperref[TEI.resp]{resp} \hyperref[TEI.rs]{rs} \hyperref[TEI.said]{said} \hyperref[TEI.series]{series} \hyperref[TEI.sic]{sic} \hyperref[TEI.soCalled]{soCalled} \hyperref[TEI.sp]{sp} \hyperref[TEI.speaker]{speaker} \hyperref[TEI.stage]{stage} \hyperref[TEI.street]{street} \hyperref[TEI.term]{term} \hyperref[TEI.textLang]{textLang} \hyperref[TEI.time]{time} \hyperref[TEI.title]{title} \hyperref[TEI.unclear]{unclear}\par 
    \item[figures: ]
   \hyperref[TEI.cell]{cell} \hyperref[TEI.figure]{figure} \hyperref[TEI.table]{table}\par 
    \item[header: ]
   \hyperref[TEI.authority]{authority} \hyperref[TEI.change]{change} \hyperref[TEI.classCode]{classCode} \hyperref[TEI.distributor]{distributor} \hyperref[TEI.edition]{edition} \hyperref[TEI.extent]{extent} \hyperref[TEI.funder]{funder} \hyperref[TEI.language]{language} \hyperref[TEI.licence]{licence}\par 
    \item[linking: ]
   \hyperref[TEI.ab]{ab} \hyperref[TEI.joinGrp]{joinGrp} \hyperref[TEI.seg]{seg}\par 
    \item[msdescription: ]
   \hyperref[TEI.accMat]{accMat} \hyperref[TEI.acquisition]{acquisition} \hyperref[TEI.additions]{additions} \hyperref[TEI.catchwords]{catchwords} \hyperref[TEI.collation]{collation} \hyperref[TEI.colophon]{colophon} \hyperref[TEI.condition]{condition} \hyperref[TEI.custEvent]{custEvent} \hyperref[TEI.decoNote]{decoNote} \hyperref[TEI.explicit]{explicit} \hyperref[TEI.filiation]{filiation} \hyperref[TEI.finalRubric]{finalRubric} \hyperref[TEI.foliation]{foliation} \hyperref[TEI.heraldry]{heraldry} \hyperref[TEI.incipit]{incipit} \hyperref[TEI.layout]{layout} \hyperref[TEI.material]{material} \hyperref[TEI.msItem]{msItem} \hyperref[TEI.musicNotation]{musicNotation} \hyperref[TEI.objectType]{objectType} \hyperref[TEI.origDate]{origDate} \hyperref[TEI.origPlace]{origPlace} \hyperref[TEI.origin]{origin} \hyperref[TEI.provenance]{provenance} \hyperref[TEI.rubric]{rubric} \hyperref[TEI.secFol]{secFol} \hyperref[TEI.signatures]{signatures} \hyperref[TEI.source]{source} \hyperref[TEI.stamp]{stamp} \hyperref[TEI.summary]{summary} \hyperref[TEI.support]{support} \hyperref[TEI.surrogates]{surrogates} \hyperref[TEI.typeNote]{typeNote} \hyperref[TEI.watermark]{watermark}\par 
    \item[namesdates: ]
   \hyperref[TEI.addName]{addName} \hyperref[TEI.affiliation]{affiliation} \hyperref[TEI.country]{country} \hyperref[TEI.forename]{forename} \hyperref[TEI.genName]{genName} \hyperref[TEI.geogName]{geogName} \hyperref[TEI.nameLink]{nameLink} \hyperref[TEI.orgName]{orgName} \hyperref[TEI.persName]{persName} \hyperref[TEI.person]{person} \hyperref[TEI.personGrp]{personGrp} \hyperref[TEI.persona]{persona} \hyperref[TEI.placeName]{placeName} \hyperref[TEI.region]{region} \hyperref[TEI.roleName]{roleName} \hyperref[TEI.settlement]{settlement} \hyperref[TEI.surname]{surname}\par 
    \item[spoken: ]
   \hyperref[TEI.annotationBlock]{annotationBlock}\par 
    \item[standOff: ]
   \hyperref[TEI.listAnnotation]{listAnnotation}\par 
    \item[textstructure: ]
   \hyperref[TEI.back]{back} \hyperref[TEI.body]{body} \hyperref[TEI.div]{div} \hyperref[TEI.docAuthor]{docAuthor} \hyperref[TEI.docDate]{docDate} \hyperref[TEI.docEdition]{docEdition} \hyperref[TEI.docTitle]{docTitle} \hyperref[TEI.floatingText]{floatingText} \hyperref[TEI.front]{front} \hyperref[TEI.group]{group} \hyperref[TEI.text]{text} \hyperref[TEI.titlePage]{titlePage} \hyperref[TEI.titlePart]{titlePart}\par 
    \item[transcr: ]
   \hyperref[TEI.damage]{damage} \hyperref[TEI.fw]{fw} \hyperref[TEI.line]{line} \hyperref[TEI.metamark]{metamark} \hyperref[TEI.mod]{mod} \hyperref[TEI.restore]{restore} \hyperref[TEI.retrace]{retrace} \hyperref[TEI.secl]{secl} \hyperref[TEI.sourceDoc]{sourceDoc} \hyperref[TEI.supplied]{supplied} \hyperref[TEI.surface]{surface} \hyperref[TEI.surfaceGrp]{surfaceGrp} \hyperref[TEI.surplus]{surplus} \hyperref[TEI.zone]{zone}
    \item[{Peut contenir}]
  
    \item[core: ]
   \hyperref[TEI.desc]{desc}
    \item[{Exemple}]
  The following example is discussed in section SAAG:\leavevmode\bgroup\exampleFont \begin{shaded}\noindent\mbox{}{<\textbf{sp}>}\mbox{}\newline 
\hspace*{6pt}{<\textbf{speaker}>}Hughie{</\textbf{speaker}>}\mbox{}\newline 
\hspace*{6pt}{<\textbf{p}>}How does it go? {<\textbf{q}>}\mbox{}\newline 
\hspace*{6pt}\hspace*{6pt}\hspace*{6pt}{<\textbf{l}\hspace*{6pt}{xml:id}="{fr\textunderscore frog\textunderscore x1}">}da-da-da{</\textbf{l}>}\mbox{}\newline 
\hspace*{6pt}\hspace*{6pt}\hspace*{6pt}{<\textbf{l}\hspace*{6pt}{xml:id}="{fr\textunderscore frog\textunderscore l2}">}gets a new frog{</\textbf{l}>}\mbox{}\newline 
\hspace*{6pt}\hspace*{6pt}\hspace*{6pt}{<\textbf{l}>}...{</\textbf{l}>}\mbox{}\newline 
\hspace*{6pt}\hspace*{6pt}{</\textbf{q}>}\mbox{}\newline 
\hspace*{6pt}{</\textbf{p}>}\mbox{}\newline 
{</\textbf{sp}>}\mbox{}\newline 
{<\textbf{sp}>}\mbox{}\newline 
\hspace*{6pt}{<\textbf{speaker}>}Louie{</\textbf{speaker}>}\mbox{}\newline 
\hspace*{6pt}{<\textbf{p}>}\mbox{}\newline 
\hspace*{6pt}\hspace*{6pt}{<\textbf{q}>}\mbox{}\newline 
\hspace*{6pt}\hspace*{6pt}\hspace*{6pt}{<\textbf{l}\hspace*{6pt}{xml:id}="{fr\textunderscore frog\textunderscore l1}">}When the old pond{</\textbf{l}>}\mbox{}\newline 
\hspace*{6pt}\hspace*{6pt}\hspace*{6pt}{<\textbf{l}>}...{</\textbf{l}>}\mbox{}\newline 
\hspace*{6pt}\hspace*{6pt}{</\textbf{q}>}\mbox{}\newline 
\hspace*{6pt}{</\textbf{p}>}\mbox{}\newline 
{</\textbf{sp}>}\mbox{}\newline 
{<\textbf{sp}>}\mbox{}\newline 
\hspace*{6pt}{<\textbf{speaker}>}Dewey{</\textbf{speaker}>}\mbox{}\newline 
\hspace*{6pt}{<\textbf{p}>}\mbox{}\newline 
\hspace*{6pt}\hspace*{6pt}{<\textbf{q}>}... {<\textbf{l}\hspace*{6pt}{xml:id}="{fr\textunderscore frog\textunderscore l3}">}It's a new pond.{</\textbf{l}>}\mbox{}\newline 
\hspace*{6pt}\hspace*{6pt}{</\textbf{q}>}\mbox{}\newline 
\hspace*{6pt}{</\textbf{p}>}\mbox{}\newline 
\hspace*{6pt}{<\textbf{join}\hspace*{6pt}{result}="{lg}"\hspace*{6pt}{scope}="{root}"\mbox{}\newline 
\hspace*{6pt}\hspace*{6pt}{target}="{\#fr\textunderscore frog\textunderscore l1 \#fr\textunderscore frog\textunderscore l2 \#fr\textunderscore frog\textunderscore l3}"/>}\mbox{}\newline 
{</\textbf{sp}>}\end{shaded}\egroup 

The \hyperref[TEI.join]{<join>} element here identifies a linegroup (\hyperref[TEI.lg]{<lg>}) comprising the three lines indicated by the {\itshape target} attribute. The value \texttt{root} for the{\itshape scope} attribute indicates that the resulting virtual element contains the three \hyperref[TEI.l]{<l>} elements linked to at \#frog\textunderscore l1 \#frog\textunderscore l2 \#frog\textunderscore l3, rather than their character data content.
    \item[{Exemple}]
  In this example, the attribute {\itshape scope} is specified with the value of \texttt{branches} to indicate that the virtual list being constructed is to be made by taking the lists indicated by the {\itshape target} attribute of the \hyperref[TEI.join]{<join>} element, discarding the \hyperref[TEI.list]{<list>} tags which enclose them, and combining the items contained within the lists into a single virtual list:\leavevmode\bgroup\exampleFont \begin{shaded}\noindent\mbox{}{<\textbf{p}>}Southern dialect (my own variety, at least) has only {<\textbf{list}\hspace*{6pt}{xml:id}="{fr\textunderscore LP1}">}\mbox{}\newline 
\hspace*{6pt}\hspace*{6pt}{<\textbf{item}>}\mbox{}\newline 
\hspace*{6pt}\hspace*{6pt}\hspace*{6pt}{<\textbf{s}>}I done gone{</\textbf{s}>}\mbox{}\newline 
\hspace*{6pt}\hspace*{6pt}{</\textbf{item}>}\mbox{}\newline 
\hspace*{6pt}\hspace*{6pt}{<\textbf{item}>}\mbox{}\newline 
\hspace*{6pt}\hspace*{6pt}\hspace*{6pt}{<\textbf{s}>}I done went{</\textbf{s}>}\mbox{}\newline 
\hspace*{6pt}\hspace*{6pt}{</\textbf{item}>}\mbox{}\newline 
\hspace*{6pt}{</\textbf{list}>} whereas Negro Non-Standard basilect has both these and {<\textbf{list}\hspace*{6pt}{xml:id}="{fr\textunderscore LP2}">}\mbox{}\newline 
\hspace*{6pt}\hspace*{6pt}{<\textbf{item}>}\mbox{}\newline 
\hspace*{6pt}\hspace*{6pt}\hspace*{6pt}{<\textbf{s}>}I done go{</\textbf{s}>}\mbox{}\newline 
\hspace*{6pt}\hspace*{6pt}{</\textbf{item}>}\mbox{}\newline 
\hspace*{6pt}{</\textbf{list}>}.{</\textbf{p}>}\mbox{}\newline 
{<\textbf{p}>}White Southern dialect also has {<\textbf{list}\hspace*{6pt}{xml:id}="{fr\textunderscore LP3}">}\mbox{}\newline 
\hspace*{6pt}\hspace*{6pt}{<\textbf{item}>}\mbox{}\newline 
\hspace*{6pt}\hspace*{6pt}\hspace*{6pt}{<\textbf{s}>}I've done gone{</\textbf{s}>}\mbox{}\newline 
\hspace*{6pt}\hspace*{6pt}{</\textbf{item}>}\mbox{}\newline 
\hspace*{6pt}\hspace*{6pt}{<\textbf{item}>}\mbox{}\newline 
\hspace*{6pt}\hspace*{6pt}\hspace*{6pt}{<\textbf{s}>}I've done went{</\textbf{s}>}\mbox{}\newline 
\hspace*{6pt}\hspace*{6pt}{</\textbf{item}>}\mbox{}\newline 
\hspace*{6pt}{</\textbf{list}>} which, when they occur in Negro dialect, should probably be considered as\mbox{}\newline 
 borrowings from other varieties of English.{</\textbf{p}>}\mbox{}\newline 
{<\textbf{join}\hspace*{6pt}{result}="{list}"\hspace*{6pt}{scope}="{branches}"\mbox{}\newline 
\hspace*{6pt}{target}="{\#fr\textunderscore LP1 \#fr\textunderscore LP2 \#fr\textunderscore LP3}"\hspace*{6pt}{xml:id}="{fr\textunderscore LST1}">}\mbox{}\newline 
\hspace*{6pt}{<\textbf{desc}>}Sample sentences in Southern speech{</\textbf{desc}>}\mbox{}\newline 
{</\textbf{join}>}\end{shaded}\egroup 


    \item[{Schematron}]
   <s:assert test="contains(@target,' ')">You must supply at least two values for @target on <s:name/> </s:assert>
    \item[{Modèle de contenu}]
  \mbox{}\hfill\\[-10pt]\begin{Verbatim}[fontsize=\small]
<content>
 <alternate maxOccurs="unbounded"
  minOccurs="0">
  <classRef key="model.descLike"/>
  <classRef key="model.certLike"/>
 </alternate>
</content>
    
\end{Verbatim}

    \item[{Schéma Declaration}]
  \mbox{}\hfill\\[-10pt]\begin{Verbatim}[fontsize=\small]
element join
{
   tei_att.global.attributes,
   tei_att.pointing.attributes,
   tei_att.typed.attributes,
   attribute result { text }?,
   attribute scope { "root" | "branches" }?,
   ( tei_model.descLike | tei_model.certLike )*
}
\end{Verbatim}

\end{reflist}  \index{joinGrp=<joinGrp>|oddindex}\index{result=@result!<joinGrp>|oddindex}
\begin{reflist}
\item[]\begin{specHead}{TEI.joinGrp}{<joinGrp> }(groupe de jointures) regroupe une collection d'éléments \hyperref[TEI.join]{<join>} ainsi que, éventuellement, des pointeurs. [\xref{http://www.tei-c.org/release/doc/tei-p5-doc/en/html/SA.html\#SAAG}{16.7. Aggregation}]\end{specHead} 
    \item[{Module}]
  linking
    \item[{Attributs}]
  Attributs \hyperref[TEI.att.global]{att.global} (\textit{@xml:id}, \textit{@n}, \textit{@xml:lang}, \textit{@xml:base}, \textit{@xml:space})  (\hyperref[TEI.att.global.rendition]{att.global.rendition} (\textit{@rend}, \textit{@style}, \textit{@rendition})) (\hyperref[TEI.att.global.linking]{att.global.linking} (\textit{@corresp}, \textit{@synch}, \textit{@sameAs}, \textit{@copyOf}, \textit{@next}, \textit{@prev}, \textit{@exclude}, \textit{@select})) (\hyperref[TEI.att.global.analytic]{att.global.analytic} (\textit{@ana})) (\hyperref[TEI.att.global.facs]{att.global.facs} (\textit{@facs})) (\hyperref[TEI.att.global.change]{att.global.change} (\textit{@change})) (\hyperref[TEI.att.global.responsibility]{att.global.responsibility} (\textit{@cert}, \textit{@resp})) (\hyperref[TEI.att.global.source]{att.global.source} (\textit{@source})) \hyperref[TEI.att.pointing.group]{att.pointing.group} (\textit{@domains}, \textit{@targFunc})  (\hyperref[TEI.att.pointing]{att.pointing} (\textit{@targetLang}, \textit{@target}, \textit{@evaluate})) (\hyperref[TEI.att.typed]{att.typed} (\textit{@type}, \textit{@subtype})) \hfil\\[-10pt]\begin{sansreflist}
    \item[@result]
  décrit le résultat produit par le rassemblement dans cette collection des éléments \hyperref[TEI.join]{<join>}.
\begin{reflist}
    \item[{Statut}]
  Optionel
    \item[{Type de données}]
  \hyperref[TEI.teidata.name]{teidata.name}
\end{reflist}  
\end{sansreflist}  
    \item[{Membre du}]
  \hyperref[TEI.model.global.meta]{model.global.meta}
    \item[{Contenu dans}]
  
    \item[analysis: ]
   \hyperref[TEI.cl]{cl} \hyperref[TEI.m]{m} \hyperref[TEI.phr]{phr} \hyperref[TEI.s]{s} \hyperref[TEI.span]{span} \hyperref[TEI.w]{w}\par 
    \item[core: ]
   \hyperref[TEI.abbr]{abbr} \hyperref[TEI.add]{add} \hyperref[TEI.addrLine]{addrLine} \hyperref[TEI.address]{address} \hyperref[TEI.author]{author} \hyperref[TEI.bibl]{bibl} \hyperref[TEI.biblScope]{biblScope} \hyperref[TEI.cit]{cit} \hyperref[TEI.citedRange]{citedRange} \hyperref[TEI.corr]{corr} \hyperref[TEI.date]{date} \hyperref[TEI.del]{del} \hyperref[TEI.distinct]{distinct} \hyperref[TEI.editor]{editor} \hyperref[TEI.email]{email} \hyperref[TEI.emph]{emph} \hyperref[TEI.expan]{expan} \hyperref[TEI.foreign]{foreign} \hyperref[TEI.gloss]{gloss} \hyperref[TEI.head]{head} \hyperref[TEI.headItem]{headItem} \hyperref[TEI.headLabel]{headLabel} \hyperref[TEI.hi]{hi} \hyperref[TEI.imprint]{imprint} \hyperref[TEI.item]{item} \hyperref[TEI.l]{l} \hyperref[TEI.label]{label} \hyperref[TEI.lg]{lg} \hyperref[TEI.list]{list} \hyperref[TEI.measure]{measure} \hyperref[TEI.mentioned]{mentioned} \hyperref[TEI.name]{name} \hyperref[TEI.note]{note} \hyperref[TEI.num]{num} \hyperref[TEI.orig]{orig} \hyperref[TEI.p]{p} \hyperref[TEI.pubPlace]{pubPlace} \hyperref[TEI.publisher]{publisher} \hyperref[TEI.q]{q} \hyperref[TEI.quote]{quote} \hyperref[TEI.ref]{ref} \hyperref[TEI.reg]{reg} \hyperref[TEI.resp]{resp} \hyperref[TEI.rs]{rs} \hyperref[TEI.said]{said} \hyperref[TEI.series]{series} \hyperref[TEI.sic]{sic} \hyperref[TEI.soCalled]{soCalled} \hyperref[TEI.sp]{sp} \hyperref[TEI.speaker]{speaker} \hyperref[TEI.stage]{stage} \hyperref[TEI.street]{street} \hyperref[TEI.term]{term} \hyperref[TEI.textLang]{textLang} \hyperref[TEI.time]{time} \hyperref[TEI.title]{title} \hyperref[TEI.unclear]{unclear}\par 
    \item[figures: ]
   \hyperref[TEI.cell]{cell} \hyperref[TEI.figure]{figure} \hyperref[TEI.table]{table}\par 
    \item[header: ]
   \hyperref[TEI.authority]{authority} \hyperref[TEI.change]{change} \hyperref[TEI.classCode]{classCode} \hyperref[TEI.distributor]{distributor} \hyperref[TEI.edition]{edition} \hyperref[TEI.extent]{extent} \hyperref[TEI.funder]{funder} \hyperref[TEI.language]{language} \hyperref[TEI.licence]{licence}\par 
    \item[linking: ]
   \hyperref[TEI.ab]{ab} \hyperref[TEI.seg]{seg}\par 
    \item[msdescription: ]
   \hyperref[TEI.accMat]{accMat} \hyperref[TEI.acquisition]{acquisition} \hyperref[TEI.additions]{additions} \hyperref[TEI.catchwords]{catchwords} \hyperref[TEI.collation]{collation} \hyperref[TEI.colophon]{colophon} \hyperref[TEI.condition]{condition} \hyperref[TEI.custEvent]{custEvent} \hyperref[TEI.decoNote]{decoNote} \hyperref[TEI.explicit]{explicit} \hyperref[TEI.filiation]{filiation} \hyperref[TEI.finalRubric]{finalRubric} \hyperref[TEI.foliation]{foliation} \hyperref[TEI.heraldry]{heraldry} \hyperref[TEI.incipit]{incipit} \hyperref[TEI.layout]{layout} \hyperref[TEI.material]{material} \hyperref[TEI.msItem]{msItem} \hyperref[TEI.musicNotation]{musicNotation} \hyperref[TEI.objectType]{objectType} \hyperref[TEI.origDate]{origDate} \hyperref[TEI.origPlace]{origPlace} \hyperref[TEI.origin]{origin} \hyperref[TEI.provenance]{provenance} \hyperref[TEI.rubric]{rubric} \hyperref[TEI.secFol]{secFol} \hyperref[TEI.signatures]{signatures} \hyperref[TEI.source]{source} \hyperref[TEI.stamp]{stamp} \hyperref[TEI.summary]{summary} \hyperref[TEI.support]{support} \hyperref[TEI.surrogates]{surrogates} \hyperref[TEI.typeNote]{typeNote} \hyperref[TEI.watermark]{watermark}\par 
    \item[namesdates: ]
   \hyperref[TEI.addName]{addName} \hyperref[TEI.affiliation]{affiliation} \hyperref[TEI.country]{country} \hyperref[TEI.forename]{forename} \hyperref[TEI.genName]{genName} \hyperref[TEI.geogName]{geogName} \hyperref[TEI.nameLink]{nameLink} \hyperref[TEI.orgName]{orgName} \hyperref[TEI.persName]{persName} \hyperref[TEI.person]{person} \hyperref[TEI.personGrp]{personGrp} \hyperref[TEI.persona]{persona} \hyperref[TEI.placeName]{placeName} \hyperref[TEI.region]{region} \hyperref[TEI.roleName]{roleName} \hyperref[TEI.settlement]{settlement} \hyperref[TEI.surname]{surname}\par 
    \item[spoken: ]
   \hyperref[TEI.annotationBlock]{annotationBlock}\par 
    \item[standOff: ]
   \hyperref[TEI.listAnnotation]{listAnnotation}\par 
    \item[textstructure: ]
   \hyperref[TEI.back]{back} \hyperref[TEI.body]{body} \hyperref[TEI.div]{div} \hyperref[TEI.docAuthor]{docAuthor} \hyperref[TEI.docDate]{docDate} \hyperref[TEI.docEdition]{docEdition} \hyperref[TEI.docTitle]{docTitle} \hyperref[TEI.floatingText]{floatingText} \hyperref[TEI.front]{front} \hyperref[TEI.group]{group} \hyperref[TEI.text]{text} \hyperref[TEI.titlePage]{titlePage} \hyperref[TEI.titlePart]{titlePart}\par 
    \item[transcr: ]
   \hyperref[TEI.damage]{damage} \hyperref[TEI.fw]{fw} \hyperref[TEI.line]{line} \hyperref[TEI.metamark]{metamark} \hyperref[TEI.mod]{mod} \hyperref[TEI.restore]{restore} \hyperref[TEI.retrace]{retrace} \hyperref[TEI.secl]{secl} \hyperref[TEI.sourceDoc]{sourceDoc} \hyperref[TEI.supplied]{supplied} \hyperref[TEI.surface]{surface} \hyperref[TEI.surfaceGrp]{surfaceGrp} \hyperref[TEI.surplus]{surplus} \hyperref[TEI.zone]{zone}
    \item[{Peut contenir}]
  
    \item[core: ]
   \hyperref[TEI.gloss]{gloss} \hyperref[TEI.ptr]{ptr}\par 
    \item[linking: ]
   \hyperref[TEI.join]{join}
    \item[{Note}]
  \par
Un nombre quelconque d'éléments \hyperref[TEI.join]{<join>} ou \hyperref[TEI.ptr]{<ptr>}.
    \item[{Exemple}]
  \leavevmode\bgroup\exampleFont \begin{shaded}\noindent\mbox{}{<\textbf{joinGrp}\hspace*{6pt}{domains}="{\#zuitxt1 \#zuitxt2 \#zuitxt3}"\mbox{}\newline 
\hspace*{6pt}{result}="{q}">}\mbox{}\newline 
\hspace*{6pt}{<\textbf{join}\hspace*{6pt}{target}="{\#zuiq1 \#zuiq2 \#zuiq6}"/>}\mbox{}\newline 
\hspace*{6pt}{<\textbf{join}\hspace*{6pt}{target}="{\#zuiq3 \#zuiq4 \#zuiq5}"/>}\mbox{}\newline 
{</\textbf{joinGrp}>}\end{shaded}\egroup 


    \item[{Exemple}]
  \leavevmode\bgroup\exampleFont \begin{shaded}\noindent\mbox{}{<\textbf{joinGrp}\hspace*{6pt}{domains}="{\#zuitxt1 \#zuitxt2 \#zuitxt3}"\mbox{}\newline 
\hspace*{6pt}{result}="{q}">}\mbox{}\newline 
\hspace*{6pt}{<\textbf{join}\hspace*{6pt}{target}="{\#fr\textunderscore zuiq1 \#fr\textunderscore zuiq2 \#fr\textunderscore zuiq6}"/>}\mbox{}\newline 
\hspace*{6pt}{<\textbf{join}\hspace*{6pt}{target}="{\#fr\textunderscore zuiq3 \#fr\textunderscore zuiq4 \#fr\textunderscore zuiq5}"/>}\mbox{}\newline 
{</\textbf{joinGrp}>}\end{shaded}\egroup 


    \item[{Modèle de contenu}]
  \mbox{}\hfill\\[-10pt]\begin{Verbatim}[fontsize=\small]
<content>
 <sequence maxOccurs="1" minOccurs="1">
  <classRef key="model.glossLike"
   maxOccurs="unbounded" minOccurs="0"/>
  <alternate maxOccurs="unbounded"
   minOccurs="1">
   <elementRef key="join"/>
   <elementRef key="ptr"/>
  </alternate>
 </sequence>
</content>
    
\end{Verbatim}

    \item[{Schéma Declaration}]
  \mbox{}\hfill\\[-10pt]\begin{Verbatim}[fontsize=\small]
element joinGrp
{
   tei_att.global.attributes,
   tei_att.pointing.group.attributes,
   attribute result { text }?,
   ( tei_model.glossLike*, ( tei_join | tei_ptr )+ )
}
\end{Verbatim}

\end{reflist}  \index{keywords=<keywords>|oddindex}\index{scheme=@scheme!<keywords>|oddindex}
\begin{reflist}
\item[]\begin{specHead}{TEI.keywords}{<keywords> }(mot clé) contient une liste de mots clés ou d’expressions décrivant la nature ou le sujet d’un texte. [\xref{http://www.tei-c.org/release/doc/tei-p5-doc/en/html/HD.html\#HD43}{2.4.3. The Text Classification}]\end{specHead} 
    \item[{Module}]
  header
    \item[{Attributs}]
  Attributs \hyperref[TEI.att.global]{att.global} (\textit{@xml:id}, \textit{@n}, \textit{@xml:lang}, \textit{@xml:base}, \textit{@xml:space})  (\hyperref[TEI.att.global.rendition]{att.global.rendition} (\textit{@rend}, \textit{@style}, \textit{@rendition})) (\hyperref[TEI.att.global.linking]{att.global.linking} (\textit{@corresp}, \textit{@synch}, \textit{@sameAs}, \textit{@copyOf}, \textit{@next}, \textit{@prev}, \textit{@exclude}, \textit{@select})) (\hyperref[TEI.att.global.analytic]{att.global.analytic} (\textit{@ana})) (\hyperref[TEI.att.global.facs]{att.global.facs} (\textit{@facs})) (\hyperref[TEI.att.global.change]{att.global.change} (\textit{@change})) (\hyperref[TEI.att.global.responsibility]{att.global.responsibility} (\textit{@cert}, \textit{@resp})) (\hyperref[TEI.att.global.source]{att.global.source} (\textit{@source})) \hfil\\[-10pt]\begin{sansreflist}
    \item[@scheme]
  désigne la liste close de mots dans lequel l'ensemble des mots-clés concernés est défini.
\begin{reflist}
    \item[{Statut}]
  Optionel
    \item[{Type de données}]
  \hyperref[TEI.teidata.pointer]{teidata.pointer}
\end{reflist}  
\end{sansreflist}  
    \item[{Membre du}]
  \hyperref[TEI.model.annotation]{model.annotation} 
    \item[{Contenu dans}]
  
    \item[header: ]
   \hyperref[TEI.textClass]{textClass}\par 
    \item[spoken: ]
   \hyperref[TEI.annotationBlock]{annotationBlock}\par 
    \item[standOff: ]
   \hyperref[TEI.listAnnotation]{listAnnotation}
    \item[{Peut contenir}]
  
    \item[core: ]
   \hyperref[TEI.list]{list} \hyperref[TEI.term]{term}
    \item[{Exemple}]
  StandOff enrichissement subject hub de métadonnées Abes\leavevmode\bgroup\exampleFont \begin{shaded}\noindent\mbox{}{<\textbf{annotationBlock}\hspace*{6pt}{corresp}="{\#subject-01}">}\mbox{}\newline 
\hspace*{6pt}{<\textbf{keywords}\hspace*{6pt}{change}="{\#abes-01}"\hspace*{6pt}{resp}="{\#abes}">}\mbox{}\newline 
\hspace*{6pt}\hspace*{6pt}{<\textbf{term}>}Pharmacologie{</\textbf{term}>}\mbox{}\newline 
\hspace*{6pt}{</\textbf{keywords}>}\mbox{}\newline 
{</\textbf{annotationBlock}>}\end{shaded}\egroup 


    \item[{Exemple}]
  StandOff enrichissement indexation rd-teeft\leavevmode\bgroup\exampleFont \begin{shaded}\noindent\mbox{}{<\textbf{annotationBlock}\hspace*{6pt}{corresp}="{text}">}\mbox{}\newline 
\hspace*{6pt}{<\textbf{keywords}\hspace*{6pt}{change}="{\#istex}"\hspace*{6pt}{resp}="{\#istex}">}\mbox{}\newline 
\hspace*{6pt}\hspace*{6pt}{<\textbf{term}>}\mbox{}\newline 
\hspace*{6pt}\hspace*{6pt}\hspace*{6pt}{<\textbf{term}>}prepulse inhibition{</\textbf{term}>}\mbox{}\newline 
\hspace*{6pt}\hspace*{6pt}\hspace*{6pt}{<\textbf{fs}\hspace*{6pt}{type}="{statistics}">}\mbox{}\newline 
\hspace*{6pt}\hspace*{6pt}\hspace*{6pt}\hspace*{6pt}{<\textbf{f}\hspace*{6pt}{name}="{frequency}">}\mbox{}\newline 
\hspace*{6pt}\hspace*{6pt}\hspace*{6pt}\hspace*{6pt}\hspace*{6pt}{<\textbf{numeric}\hspace*{6pt}{value}="{4}"/>}\mbox{}\newline 
\hspace*{6pt}\hspace*{6pt}\hspace*{6pt}\hspace*{6pt}{</\textbf{f}>}\mbox{}\newline 
\hspace*{6pt}\hspace*{6pt}\hspace*{6pt}\hspace*{6pt}{<\textbf{f}\hspace*{6pt}{name}="{specificity}">}\mbox{}\newline 
\hspace*{6pt}\hspace*{6pt}\hspace*{6pt}\hspace*{6pt}\hspace*{6pt}{<\textbf{numeric}\hspace*{6pt}{value}="{1}"/>}\mbox{}\newline 
\hspace*{6pt}\hspace*{6pt}\hspace*{6pt}\hspace*{6pt}{</\textbf{f}>}\mbox{}\newline 
\hspace*{6pt}\hspace*{6pt}\hspace*{6pt}{</\textbf{fs}>}\mbox{}\newline 
\hspace*{6pt}\hspace*{6pt}{</\textbf{term}>}\mbox{}\newline 
\hspace*{6pt}\hspace*{6pt}{<\textbf{term}>}\mbox{}\newline 
\hspace*{6pt}\hspace*{6pt}\hspace*{6pt}{<\textbf{term}>}decompensated inpatients{</\textbf{term}>}\mbox{}\newline 
\hspace*{6pt}\hspace*{6pt}\hspace*{6pt}{<\textbf{fs}\hspace*{6pt}{type}="{statistics}">}\mbox{}\newline 
\hspace*{6pt}\hspace*{6pt}\hspace*{6pt}\hspace*{6pt}{<\textbf{f}\hspace*{6pt}{name}="{frequency}">}\mbox{}\newline 
\hspace*{6pt}\hspace*{6pt}\hspace*{6pt}\hspace*{6pt}\hspace*{6pt}{<\textbf{numeric}\hspace*{6pt}{value}="{2}"/>}\mbox{}\newline 
\hspace*{6pt}\hspace*{6pt}\hspace*{6pt}\hspace*{6pt}{</\textbf{f}>}\mbox{}\newline 
\hspace*{6pt}\hspace*{6pt}\hspace*{6pt}\hspace*{6pt}{<\textbf{f}\hspace*{6pt}{name}="{specificity}">}\mbox{}\newline 
\hspace*{6pt}\hspace*{6pt}\hspace*{6pt}\hspace*{6pt}\hspace*{6pt}{<\textbf{numeric}\hspace*{6pt}{value}="{0.5}"/>}\mbox{}\newline 
\hspace*{6pt}\hspace*{6pt}\hspace*{6pt}\hspace*{6pt}{</\textbf{f}>}\mbox{}\newline 
\hspace*{6pt}\hspace*{6pt}\hspace*{6pt}{</\textbf{fs}>}\mbox{}\newline 
\hspace*{6pt}\hspace*{6pt}{</\textbf{term}>}\mbox{}\newline 
\hspace*{6pt}{</\textbf{keywords}>}\mbox{}\newline 
{</\textbf{annotationBlock}>}\end{shaded}\egroup 


    \item[{Modèle de contenu}]
  \mbox{}\hfill\\[-10pt]\begin{Verbatim}[fontsize=\small]
<content>
 <alternate maxOccurs="1" minOccurs="1">
  <elementRef key="term"
   maxOccurs="unbounded" minOccurs="1"/>
  <elementRef key="list"/>
 </alternate>
</content>
    
\end{Verbatim}

    \item[{Schéma Declaration}]
  \mbox{}\hfill\\[-10pt]\begin{Verbatim}[fontsize=\small]
element keywords
{
   tei_att.global.attributes,
   attribute scheme { text }?,
   ( tei_term+ | tei_list )
}
\end{Verbatim}

\end{reflist}  \index{l=<l>|oddindex}
\begin{reflist}
\item[]\begin{specHead}{TEI.l}{<l> }(vers) contient un seul vers, éventuellement incomplet. [\xref{http://www.tei-c.org/release/doc/tei-p5-doc/en/html/CO.html\#COVE}{3.12.1. Core Tags for Verse} \xref{http://www.tei-c.org/release/doc/tei-p5-doc/en/html/CO.html\#CODV}{3.12. Passages of Verse or Drama} \xref{http://www.tei-c.org/release/doc/tei-p5-doc/en/html/DR.html\#DRPAL}{7.2.5. Speech Contents}]\end{specHead} 
    \item[{Module}]
  core
    \item[{Attributs}]
  Attributs \hyperref[TEI.att.global]{att.global} (\textit{@xml:id}, \textit{@n}, \textit{@xml:lang}, \textit{@xml:base}, \textit{@xml:space})  (\hyperref[TEI.att.global.rendition]{att.global.rendition} (\textit{@rend}, \textit{@style}, \textit{@rendition})) (\hyperref[TEI.att.global.linking]{att.global.linking} (\textit{@corresp}, \textit{@synch}, \textit{@sameAs}, \textit{@copyOf}, \textit{@next}, \textit{@prev}, \textit{@exclude}, \textit{@select})) (\hyperref[TEI.att.global.analytic]{att.global.analytic} (\textit{@ana})) (\hyperref[TEI.att.global.facs]{att.global.facs} (\textit{@facs})) (\hyperref[TEI.att.global.change]{att.global.change} (\textit{@change})) (\hyperref[TEI.att.global.responsibility]{att.global.responsibility} (\textit{@cert}, \textit{@resp})) (\hyperref[TEI.att.global.source]{att.global.source} (\textit{@source})) \hyperref[TEI.att.fragmentable]{att.fragmentable} (\textit{@part}) 
    \item[{Membre du}]
  \hyperref[TEI.model.lLike]{model.lLike}
    \item[{Contenu dans}]
  
    \item[core: ]
   \hyperref[TEI.add]{add} \hyperref[TEI.corr]{corr} \hyperref[TEI.del]{del} \hyperref[TEI.emph]{emph} \hyperref[TEI.head]{head} \hyperref[TEI.hi]{hi} \hyperref[TEI.item]{item} \hyperref[TEI.lg]{lg} \hyperref[TEI.note]{note} \hyperref[TEI.orig]{orig} \hyperref[TEI.p]{p} \hyperref[TEI.q]{q} \hyperref[TEI.quote]{quote} \hyperref[TEI.ref]{ref} \hyperref[TEI.reg]{reg} \hyperref[TEI.said]{said} \hyperref[TEI.sic]{sic} \hyperref[TEI.sp]{sp} \hyperref[TEI.stage]{stage} \hyperref[TEI.title]{title} \hyperref[TEI.unclear]{unclear}\par 
    \item[figures: ]
   \hyperref[TEI.cell]{cell} \hyperref[TEI.figure]{figure}\par 
    \item[header: ]
   \hyperref[TEI.change]{change} \hyperref[TEI.licence]{licence}\par 
    \item[linking: ]
   \hyperref[TEI.ab]{ab} \hyperref[TEI.seg]{seg}\par 
    \item[msdescription: ]
   \hyperref[TEI.accMat]{accMat} \hyperref[TEI.acquisition]{acquisition} \hyperref[TEI.additions]{additions} \hyperref[TEI.collation]{collation} \hyperref[TEI.condition]{condition} \hyperref[TEI.custEvent]{custEvent} \hyperref[TEI.decoNote]{decoNote} \hyperref[TEI.filiation]{filiation} \hyperref[TEI.foliation]{foliation} \hyperref[TEI.layout]{layout} \hyperref[TEI.musicNotation]{musicNotation} \hyperref[TEI.origin]{origin} \hyperref[TEI.provenance]{provenance} \hyperref[TEI.signatures]{signatures} \hyperref[TEI.source]{source} \hyperref[TEI.summary]{summary} \hyperref[TEI.support]{support} \hyperref[TEI.surrogates]{surrogates} \hyperref[TEI.typeNote]{typeNote}\par 
    \item[textstructure: ]
   \hyperref[TEI.body]{body} \hyperref[TEI.div]{div} \hyperref[TEI.docEdition]{docEdition} \hyperref[TEI.titlePart]{titlePart}\par 
    \item[transcr: ]
   \hyperref[TEI.damage]{damage} \hyperref[TEI.metamark]{metamark} \hyperref[TEI.mod]{mod} \hyperref[TEI.restore]{restore} \hyperref[TEI.retrace]{retrace} \hyperref[TEI.secl]{secl} \hyperref[TEI.supplied]{supplied} \hyperref[TEI.surplus]{surplus}
    \item[{Peut contenir}]
  
    \item[analysis: ]
   \hyperref[TEI.c]{c} \hyperref[TEI.cl]{cl} \hyperref[TEI.interp]{interp} \hyperref[TEI.interpGrp]{interpGrp} \hyperref[TEI.m]{m} \hyperref[TEI.pc]{pc} \hyperref[TEI.phr]{phr} \hyperref[TEI.s]{s} \hyperref[TEI.span]{span} \hyperref[TEI.spanGrp]{spanGrp} \hyperref[TEI.w]{w}\par 
    \item[core: ]
   \hyperref[TEI.abbr]{abbr} \hyperref[TEI.add]{add} \hyperref[TEI.address]{address} \hyperref[TEI.bibl]{bibl} \hyperref[TEI.biblStruct]{biblStruct} \hyperref[TEI.binaryObject]{binaryObject} \hyperref[TEI.cb]{cb} \hyperref[TEI.choice]{choice} \hyperref[TEI.cit]{cit} \hyperref[TEI.corr]{corr} \hyperref[TEI.date]{date} \hyperref[TEI.del]{del} \hyperref[TEI.desc]{desc} \hyperref[TEI.distinct]{distinct} \hyperref[TEI.email]{email} \hyperref[TEI.emph]{emph} \hyperref[TEI.expan]{expan} \hyperref[TEI.foreign]{foreign} \hyperref[TEI.gap]{gap} \hyperref[TEI.gb]{gb} \hyperref[TEI.gloss]{gloss} \hyperref[TEI.graphic]{graphic} \hyperref[TEI.hi]{hi} \hyperref[TEI.index]{index} \hyperref[TEI.label]{label} \hyperref[TEI.lb]{lb} \hyperref[TEI.list]{list} \hyperref[TEI.listBibl]{listBibl} \hyperref[TEI.measure]{measure} \hyperref[TEI.measureGrp]{measureGrp} \hyperref[TEI.media]{media} \hyperref[TEI.mentioned]{mentioned} \hyperref[TEI.milestone]{milestone} \hyperref[TEI.name]{name} \hyperref[TEI.note]{note} \hyperref[TEI.num]{num} \hyperref[TEI.orig]{orig} \hyperref[TEI.pb]{pb} \hyperref[TEI.ptr]{ptr} \hyperref[TEI.q]{q} \hyperref[TEI.quote]{quote} \hyperref[TEI.ref]{ref} \hyperref[TEI.reg]{reg} \hyperref[TEI.rs]{rs} \hyperref[TEI.said]{said} \hyperref[TEI.sic]{sic} \hyperref[TEI.soCalled]{soCalled} \hyperref[TEI.stage]{stage} \hyperref[TEI.term]{term} \hyperref[TEI.time]{time} \hyperref[TEI.title]{title} \hyperref[TEI.unclear]{unclear}\par 
    \item[derived-module-tei.istex: ]
   \hyperref[TEI.math]{math} \hyperref[TEI.mrow]{mrow}\par 
    \item[figures: ]
   \hyperref[TEI.figure]{figure} \hyperref[TEI.formula]{formula} \hyperref[TEI.notatedMusic]{notatedMusic} \hyperref[TEI.table]{table}\par 
    \item[header: ]
   \hyperref[TEI.biblFull]{biblFull} \hyperref[TEI.idno]{idno}\par 
    \item[iso-fs: ]
   \hyperref[TEI.fLib]{fLib} \hyperref[TEI.fs]{fs} \hyperref[TEI.fvLib]{fvLib}\par 
    \item[linking: ]
   \hyperref[TEI.alt]{alt} \hyperref[TEI.altGrp]{altGrp} \hyperref[TEI.anchor]{anchor} \hyperref[TEI.join]{join} \hyperref[TEI.joinGrp]{joinGrp} \hyperref[TEI.link]{link} \hyperref[TEI.linkGrp]{linkGrp} \hyperref[TEI.seg]{seg} \hyperref[TEI.timeline]{timeline}\par 
    \item[msdescription: ]
   \hyperref[TEI.catchwords]{catchwords} \hyperref[TEI.depth]{depth} \hyperref[TEI.dim]{dim} \hyperref[TEI.dimensions]{dimensions} \hyperref[TEI.height]{height} \hyperref[TEI.heraldry]{heraldry} \hyperref[TEI.locus]{locus} \hyperref[TEI.locusGrp]{locusGrp} \hyperref[TEI.material]{material} \hyperref[TEI.msDesc]{msDesc} \hyperref[TEI.objectType]{objectType} \hyperref[TEI.origDate]{origDate} \hyperref[TEI.origPlace]{origPlace} \hyperref[TEI.secFol]{secFol} \hyperref[TEI.signatures]{signatures} \hyperref[TEI.source]{source} \hyperref[TEI.stamp]{stamp} \hyperref[TEI.watermark]{watermark} \hyperref[TEI.width]{width}\par 
    \item[namesdates: ]
   \hyperref[TEI.addName]{addName} \hyperref[TEI.affiliation]{affiliation} \hyperref[TEI.country]{country} \hyperref[TEI.forename]{forename} \hyperref[TEI.genName]{genName} \hyperref[TEI.geogName]{geogName} \hyperref[TEI.listOrg]{listOrg} \hyperref[TEI.listPlace]{listPlace} \hyperref[TEI.location]{location} \hyperref[TEI.nameLink]{nameLink} \hyperref[TEI.orgName]{orgName} \hyperref[TEI.persName]{persName} \hyperref[TEI.placeName]{placeName} \hyperref[TEI.region]{region} \hyperref[TEI.roleName]{roleName} \hyperref[TEI.settlement]{settlement} \hyperref[TEI.state]{state} \hyperref[TEI.surname]{surname}\par 
    \item[spoken: ]
   \hyperref[TEI.annotationBlock]{annotationBlock}\par 
    \item[textstructure: ]
   \hyperref[TEI.floatingText]{floatingText}\par 
    \item[transcr: ]
   \hyperref[TEI.addSpan]{addSpan} \hyperref[TEI.am]{am} \hyperref[TEI.damage]{damage} \hyperref[TEI.damageSpan]{damageSpan} \hyperref[TEI.delSpan]{delSpan} \hyperref[TEI.ex]{ex} \hyperref[TEI.fw]{fw} \hyperref[TEI.handShift]{handShift} \hyperref[TEI.listTranspose]{listTranspose} \hyperref[TEI.metamark]{metamark} \hyperref[TEI.mod]{mod} \hyperref[TEI.redo]{redo} \hyperref[TEI.restore]{restore} \hyperref[TEI.retrace]{retrace} \hyperref[TEI.secl]{secl} \hyperref[TEI.space]{space} \hyperref[TEI.subst]{subst} \hyperref[TEI.substJoin]{substJoin} \hyperref[TEI.supplied]{supplied} \hyperref[TEI.surplus]{surplus} \hyperref[TEI.undo]{undo}\par des données textuelles
    \item[{Exemple}]
  \leavevmode\bgroup\exampleFont \begin{shaded}\noindent\mbox{}{<\textbf{l}\hspace*{6pt}{met}="{x/x/x/x/x/}"\hspace*{6pt}{real}="{/xx/x/x/x/}">}Shall I compare thee to a summer's day?{</\textbf{l}>}\end{shaded}\egroup 


    \item[{Exemple}]
  \leavevmode\bgroup\exampleFont \begin{shaded}\noindent\mbox{}{<\textbf{l}>}Que toujours, dans vos vers, le sens coupant les mots{</\textbf{l}>}\mbox{}\newline 
{<\textbf{l}>}Suspende l'hémistiche, en marque le repos.{</\textbf{l}>}\end{shaded}\egroup 


    \item[{Schematron}]
   <s:report test="ancestor::tei:l[not(.//tei:note//tei:l[. = current()])]"> Abstract model violation: Lines may not contain lines or lg elements. </s:report>
    \item[{Modèle de contenu}]
  \mbox{}\hfill\\[-10pt]\begin{Verbatim}[fontsize=\small]
<content>
 <alternate maxOccurs="unbounded"
  minOccurs="0">
  <textNode/>
  <classRef key="model.gLike"/>
  <classRef key="model.phrase"/>
  <classRef key="model.inter"/>
  <classRef key="model.global"/>
 </alternate>
</content>
    
\end{Verbatim}

    \item[{Schéma Declaration}]
  \mbox{}\hfill\\[-10pt]\begin{Verbatim}[fontsize=\small]
element l
{
   tei_att.global.attributes,
   tei_att.fragmentable.attributes,
   (
      text
    | tei_model.gLike    | tei_model.phrase    | tei_model.inter    | tei_model.global   )*
}
\end{Verbatim}

\end{reflist}  \index{label=<label>|oddindex}
\begin{reflist}
\item[]\begin{specHead}{TEI.label}{<label> }(étiquette) contient l'étiquette attachée à un item dans une liste ; dans les glossaires, il marque le terme qui est défini. [\xref{http://www.tei-c.org/release/doc/tei-p5-doc/en/html/CO.html\#COLI}{3.7. Lists}]\end{specHead} 
    \item[{Module}]
  core
    \item[{Attributs}]
  Attributs \hyperref[TEI.att.global]{att.global} (\textit{@xml:id}, \textit{@n}, \textit{@xml:lang}, \textit{@xml:base}, \textit{@xml:space})  (\hyperref[TEI.att.global.rendition]{att.global.rendition} (\textit{@rend}, \textit{@style}, \textit{@rendition})) (\hyperref[TEI.att.global.linking]{att.global.linking} (\textit{@corresp}, \textit{@synch}, \textit{@sameAs}, \textit{@copyOf}, \textit{@next}, \textit{@prev}, \textit{@exclude}, \textit{@select})) (\hyperref[TEI.att.global.analytic]{att.global.analytic} (\textit{@ana})) (\hyperref[TEI.att.global.facs]{att.global.facs} (\textit{@facs})) (\hyperref[TEI.att.global.change]{att.global.change} (\textit{@change})) (\hyperref[TEI.att.global.responsibility]{att.global.responsibility} (\textit{@cert}, \textit{@resp})) (\hyperref[TEI.att.global.source]{att.global.source} (\textit{@source})) \hyperref[TEI.att.typed]{att.typed} (\textit{@type}, \textit{@subtype}) \hyperref[TEI.att.placement]{att.placement} (\textit{@place}) \hyperref[TEI.att.written]{att.written} (\textit{@hand}) 
    \item[{Membre du}]
  \hyperref[TEI.model.labelLike]{model.labelLike}
    \item[{Contenu dans}]
  
    \item[core: ]
   \hyperref[TEI.add]{add} \hyperref[TEI.corr]{corr} \hyperref[TEI.del]{del} \hyperref[TEI.desc]{desc} \hyperref[TEI.emph]{emph} \hyperref[TEI.head]{head} \hyperref[TEI.hi]{hi} \hyperref[TEI.item]{item} \hyperref[TEI.l]{l} \hyperref[TEI.lg]{lg} \hyperref[TEI.list]{list} \hyperref[TEI.meeting]{meeting} \hyperref[TEI.note]{note} \hyperref[TEI.orig]{orig} \hyperref[TEI.p]{p} \hyperref[TEI.q]{q} \hyperref[TEI.quote]{quote} \hyperref[TEI.ref]{ref} \hyperref[TEI.reg]{reg} \hyperref[TEI.said]{said} \hyperref[TEI.sic]{sic} \hyperref[TEI.stage]{stage} \hyperref[TEI.title]{title} \hyperref[TEI.unclear]{unclear}\par 
    \item[figures: ]
   \hyperref[TEI.cell]{cell} \hyperref[TEI.figDesc]{figDesc} \hyperref[TEI.figure]{figure} \hyperref[TEI.notatedMusic]{notatedMusic}\par 
    \item[header: ]
   \hyperref[TEI.application]{application} \hyperref[TEI.change]{change} \hyperref[TEI.licence]{licence} \hyperref[TEI.rendition]{rendition}\par 
    \item[iso-fs: ]
   \hyperref[TEI.fDescr]{fDescr} \hyperref[TEI.fsDescr]{fsDescr}\par 
    \item[linking: ]
   \hyperref[TEI.ab]{ab} \hyperref[TEI.seg]{seg}\par 
    \item[msdescription: ]
   \hyperref[TEI.accMat]{accMat} \hyperref[TEI.acquisition]{acquisition} \hyperref[TEI.additions]{additions} \hyperref[TEI.collation]{collation} \hyperref[TEI.condition]{condition} \hyperref[TEI.custEvent]{custEvent} \hyperref[TEI.decoNote]{decoNote} \hyperref[TEI.filiation]{filiation} \hyperref[TEI.foliation]{foliation} \hyperref[TEI.layout]{layout} \hyperref[TEI.musicNotation]{musicNotation} \hyperref[TEI.origin]{origin} \hyperref[TEI.provenance]{provenance} \hyperref[TEI.signatures]{signatures} \hyperref[TEI.source]{source} \hyperref[TEI.summary]{summary} \hyperref[TEI.support]{support} \hyperref[TEI.surrogates]{surrogates} \hyperref[TEI.typeNote]{typeNote}\par 
    \item[namesdates: ]
   \hyperref[TEI.event]{event} \hyperref[TEI.location]{location} \hyperref[TEI.org]{org} \hyperref[TEI.place]{place} \hyperref[TEI.state]{state}\par 
    \item[textstructure: ]
   \hyperref[TEI.body]{body} \hyperref[TEI.div]{div} \hyperref[TEI.docEdition]{docEdition} \hyperref[TEI.titlePart]{titlePart}\par 
    \item[transcr: ]
   \hyperref[TEI.damage]{damage} \hyperref[TEI.metamark]{metamark} \hyperref[TEI.mod]{mod} \hyperref[TEI.restore]{restore} \hyperref[TEI.retrace]{retrace} \hyperref[TEI.secl]{secl} \hyperref[TEI.supplied]{supplied} \hyperref[TEI.surface]{surface} \hyperref[TEI.surplus]{surplus}
    \item[{Peut contenir}]
  
    \item[analysis: ]
   \hyperref[TEI.c]{c} \hyperref[TEI.cl]{cl} \hyperref[TEI.interp]{interp} \hyperref[TEI.interpGrp]{interpGrp} \hyperref[TEI.m]{m} \hyperref[TEI.pc]{pc} \hyperref[TEI.phr]{phr} \hyperref[TEI.s]{s} \hyperref[TEI.span]{span} \hyperref[TEI.spanGrp]{spanGrp} \hyperref[TEI.w]{w}\par 
    \item[core: ]
   \hyperref[TEI.abbr]{abbr} \hyperref[TEI.add]{add} \hyperref[TEI.address]{address} \hyperref[TEI.binaryObject]{binaryObject} \hyperref[TEI.cb]{cb} \hyperref[TEI.choice]{choice} \hyperref[TEI.corr]{corr} \hyperref[TEI.date]{date} \hyperref[TEI.del]{del} \hyperref[TEI.distinct]{distinct} \hyperref[TEI.email]{email} \hyperref[TEI.emph]{emph} \hyperref[TEI.expan]{expan} \hyperref[TEI.foreign]{foreign} \hyperref[TEI.gap]{gap} \hyperref[TEI.gb]{gb} \hyperref[TEI.gloss]{gloss} \hyperref[TEI.graphic]{graphic} \hyperref[TEI.hi]{hi} \hyperref[TEI.index]{index} \hyperref[TEI.lb]{lb} \hyperref[TEI.measure]{measure} \hyperref[TEI.measureGrp]{measureGrp} \hyperref[TEI.media]{media} \hyperref[TEI.mentioned]{mentioned} \hyperref[TEI.milestone]{milestone} \hyperref[TEI.name]{name} \hyperref[TEI.note]{note} \hyperref[TEI.num]{num} \hyperref[TEI.orig]{orig} \hyperref[TEI.pb]{pb} \hyperref[TEI.ptr]{ptr} \hyperref[TEI.ref]{ref} \hyperref[TEI.reg]{reg} \hyperref[TEI.rs]{rs} \hyperref[TEI.sic]{sic} \hyperref[TEI.soCalled]{soCalled} \hyperref[TEI.term]{term} \hyperref[TEI.time]{time} \hyperref[TEI.title]{title} \hyperref[TEI.unclear]{unclear}\par 
    \item[derived-module-tei.istex: ]
   \hyperref[TEI.math]{math} \hyperref[TEI.mrow]{mrow}\par 
    \item[figures: ]
   \hyperref[TEI.figure]{figure} \hyperref[TEI.formula]{formula} \hyperref[TEI.notatedMusic]{notatedMusic}\par 
    \item[header: ]
   \hyperref[TEI.idno]{idno}\par 
    \item[iso-fs: ]
   \hyperref[TEI.fLib]{fLib} \hyperref[TEI.fs]{fs} \hyperref[TEI.fvLib]{fvLib}\par 
    \item[linking: ]
   \hyperref[TEI.alt]{alt} \hyperref[TEI.altGrp]{altGrp} \hyperref[TEI.anchor]{anchor} \hyperref[TEI.join]{join} \hyperref[TEI.joinGrp]{joinGrp} \hyperref[TEI.link]{link} \hyperref[TEI.linkGrp]{linkGrp} \hyperref[TEI.seg]{seg} \hyperref[TEI.timeline]{timeline}\par 
    \item[msdescription: ]
   \hyperref[TEI.catchwords]{catchwords} \hyperref[TEI.depth]{depth} \hyperref[TEI.dim]{dim} \hyperref[TEI.dimensions]{dimensions} \hyperref[TEI.height]{height} \hyperref[TEI.heraldry]{heraldry} \hyperref[TEI.locus]{locus} \hyperref[TEI.locusGrp]{locusGrp} \hyperref[TEI.material]{material} \hyperref[TEI.objectType]{objectType} \hyperref[TEI.origDate]{origDate} \hyperref[TEI.origPlace]{origPlace} \hyperref[TEI.secFol]{secFol} \hyperref[TEI.signatures]{signatures} \hyperref[TEI.source]{source} \hyperref[TEI.stamp]{stamp} \hyperref[TEI.watermark]{watermark} \hyperref[TEI.width]{width}\par 
    \item[namesdates: ]
   \hyperref[TEI.addName]{addName} \hyperref[TEI.affiliation]{affiliation} \hyperref[TEI.country]{country} \hyperref[TEI.forename]{forename} \hyperref[TEI.genName]{genName} \hyperref[TEI.geogName]{geogName} \hyperref[TEI.location]{location} \hyperref[TEI.nameLink]{nameLink} \hyperref[TEI.orgName]{orgName} \hyperref[TEI.persName]{persName} \hyperref[TEI.placeName]{placeName} \hyperref[TEI.region]{region} \hyperref[TEI.roleName]{roleName} \hyperref[TEI.settlement]{settlement} \hyperref[TEI.state]{state} \hyperref[TEI.surname]{surname}\par 
    \item[spoken: ]
   \hyperref[TEI.annotationBlock]{annotationBlock}\par 
    \item[transcr: ]
   \hyperref[TEI.addSpan]{addSpan} \hyperref[TEI.am]{am} \hyperref[TEI.damage]{damage} \hyperref[TEI.damageSpan]{damageSpan} \hyperref[TEI.delSpan]{delSpan} \hyperref[TEI.ex]{ex} \hyperref[TEI.fw]{fw} \hyperref[TEI.handShift]{handShift} \hyperref[TEI.listTranspose]{listTranspose} \hyperref[TEI.metamark]{metamark} \hyperref[TEI.mod]{mod} \hyperref[TEI.redo]{redo} \hyperref[TEI.restore]{restore} \hyperref[TEI.retrace]{retrace} \hyperref[TEI.secl]{secl} \hyperref[TEI.space]{space} \hyperref[TEI.subst]{subst} \hyperref[TEI.substJoin]{substJoin} \hyperref[TEI.supplied]{supplied} \hyperref[TEI.surplus]{surplus} \hyperref[TEI.undo]{undo}\par des données textuelles
    \item[{Exemple}]
  \hyperref[TEI.label]{<label>} est souvent employé pour identifier les mots dans des listes de glossaire ; on note l'utilisation de l’attribut global {\itshape xml:lang} : l'ancien français est la langue par défaut du glossaire\leavevmode\bgroup\exampleFont \begin{shaded}\noindent\mbox{}{<\textbf{list}\hspace*{6pt}{type}="{gloss}"\hspace*{6pt}{xml:lang}="{fro}">}\mbox{}\newline 
\hspace*{6pt}{<\textbf{head}\hspace*{6pt}{xml:lang}="{en}">}Vocabulaire{</\textbf{head}>}\mbox{}\newline 
\hspace*{6pt}{<\textbf{headLabel}\hspace*{6pt}{xml:lang}="{fr}">}Ancien français{</\textbf{headLabel}>}\mbox{}\newline 
\hspace*{6pt}{<\textbf{headItem}\hspace*{6pt}{xml:lang}="{fr}">}Français moderne{</\textbf{headItem}>}\mbox{}\newline 
\hspace*{6pt}{<\textbf{label}>}amerté{</\textbf{label}>}\mbox{}\newline 
\hspace*{6pt}{<\textbf{item}\hspace*{6pt}{xml:lang}="{fr}">}amertume{</\textbf{item}>}\mbox{}\newline 
\hspace*{6pt}{<\textbf{label}>}barquer{</\textbf{label}>}\mbox{}\newline 
\hspace*{6pt}{<\textbf{item}\hspace*{6pt}{xml:lang}="{fr}">}conduire une barque{</\textbf{item}>}\mbox{}\newline 
\hspace*{6pt}{<\textbf{label}>}biberon{</\textbf{label}>}\mbox{}\newline 
\hspace*{6pt}{<\textbf{item}\hspace*{6pt}{xml:lang}="{fr}">}goulot d'un vase.{</\textbf{item}>}\mbox{}\newline 
\hspace*{6pt}{<\textbf{label}>}bugle{</\textbf{label}>}\mbox{}\newline 
\hspace*{6pt}{<\textbf{item}\hspace*{6pt}{xml:lang}="{fr}">}jeune boeuf{</\textbf{item}>}\mbox{}\newline 
\hspace*{6pt}{<\textbf{label}>}cestui{</\textbf{label}>}\mbox{}\newline 
\hspace*{6pt}{<\textbf{item}\hspace*{6pt}{xml:lang}="{fr}">}celui-ci{</\textbf{item}>}\mbox{}\newline 
\hspace*{6pt}{<\textbf{label}>}chaitiveté{</\textbf{label}>}\mbox{}\newline 
\hspace*{6pt}{<\textbf{item}\hspace*{6pt}{xml:lang}="{fr}">}captivité{</\textbf{item}>}\mbox{}\newline 
\hspace*{6pt}{<\textbf{label}>}duire{</\textbf{label}>}\mbox{}\newline 
\hspace*{6pt}{<\textbf{item}\hspace*{6pt}{xml:lang}="{fr}">}conduire{</\textbf{item}>}\mbox{}\newline 
\hspace*{6pt}{<\textbf{label}>}dangier{</\textbf{label}>}\mbox{}\newline 
\hspace*{6pt}{<\textbf{item}\hspace*{6pt}{xml:lang}="{fr}">}puissance, pouvoir{</\textbf{item}>}\mbox{}\newline 
\hspace*{6pt}{<\textbf{label}>}sangler{</\textbf{label}>}\mbox{}\newline 
\hspace*{6pt}{<\textbf{item}\hspace*{6pt}{xml:lang}="{la}">}singularis{</\textbf{item}>}\mbox{}\newline 
\hspace*{6pt}{<\textbf{label}>}vespre {</\textbf{label}>}\mbox{}\newline 
\hspace*{6pt}{<\textbf{item}\hspace*{6pt}{xml:lang}="{fr}">}tombée du jour (cf. {<\textbf{cit}>}\mbox{}\newline 
\hspace*{6pt}\hspace*{6pt}\hspace*{6pt}{<\textbf{ref}>}Lexique de Godefroy{</\textbf{ref}>}\mbox{}\newline 
\hspace*{6pt}\hspace*{6pt}\hspace*{6pt}{<\textbf{quote}>}de hautes vespres : tard dans la soirée.{</\textbf{quote}>}\mbox{}\newline 
\hspace*{6pt}\hspace*{6pt}{</\textbf{cit}>}){</\textbf{item}>}\mbox{}\newline 
{</\textbf{list}>}\end{shaded}\egroup 


    \item[{Exemple}]
  On emploie aussi \hyperref[TEI.label]{<label>} pour enregistrer le classement alphanumérique d'items dans des listes ordonnées.\leavevmode\bgroup\exampleFont \begin{shaded}\noindent\mbox{} L’unité mixte de\mbox{}\newline 
 recherche ATILF (Analyse et Traitement Informatique de la Langue Française) propose 3 types\mbox{}\newline 
 de ressources : {<\textbf{list}\hspace*{6pt}{rend}="{runon}"\hspace*{6pt}{type}="{ordered}">}\mbox{}\newline 
\hspace*{6pt}{<\textbf{label}>}(1){</\textbf{label}>}\mbox{}\newline 
\hspace*{6pt}{<\textbf{item}>}Des ressources linguistiques publiées.{</\textbf{item}>}\mbox{}\newline 
\hspace*{6pt}{<\textbf{label}>}(2) {</\textbf{label}>}\mbox{}\newline 
\hspace*{6pt}{<\textbf{item}>}Des ressources linguistiques informatisées.{</\textbf{item}>}\mbox{}\newline 
\hspace*{6pt}{<\textbf{label}>}(3) {</\textbf{label}>}\mbox{}\newline 
\hspace*{6pt}{<\textbf{item}>}Des ressources linguistiques logicielles.{</\textbf{item}>}\mbox{}\newline 
{</\textbf{list}>}\end{shaded}\egroup 


    \item[{Exemple}]
  On emploie aussi \hyperref[TEI.label]{<label>} dans d'autres listes structurées, comme dans cet extrait du journal de Jules Renard.\leavevmode\bgroup\exampleFont \begin{shaded}\noindent\mbox{}{<\textbf{list}\hspace*{6pt}{type}="{gloss}">}\mbox{}\newline 
\hspace*{6pt}{<\textbf{label}>}1er janvier..{</\textbf{label}>}\mbox{}\newline 
\hspace*{6pt}{<\textbf{item}>} L'esprit inquiet mais clairvoyant, c'est-à-dire actif et sain, de l'homme qui ne\mbox{}\newline 
\hspace*{6pt}\hspace*{6pt} travaille pas.{</\textbf{item}>}\mbox{}\newline 
\hspace*{6pt}{<\textbf{label}>}7 janvier. {</\textbf{label}>}\mbox{}\newline 
\hspace*{6pt}{<\textbf{item}>}On parle de Syveton. Elle aussi se rappelle avoir été, petite fille, poursuivie par\mbox{}\newline 
\hspace*{6pt}\hspace*{6pt} un homme tout décolleté du bas et qu'on appelait l'homme au nez rouge. {</\textbf{item}>}\mbox{}\newline 
\hspace*{6pt}{<\textbf{label}>}9 janvier. {</\textbf{label}>}\mbox{}\newline 
\hspace*{6pt}{<\textbf{item}>}Jaurès dit que Syveton n'avait aucun intérêt à se tuer mais, à y regarder de près,\mbox{}\newline 
\hspace*{6pt}\hspace*{6pt} oui, en cherchant bien, on trouve que nous avons tous un intérêt quelconque à nous tuer. {</\textbf{item}>}\mbox{}\newline 
\hspace*{6pt}{<\textbf{label}>}13 janvier. {</\textbf{label}>}\mbox{}\newline 
\hspace*{6pt}{<\textbf{item}>}Dans mon coeur froid, quelques rares jolis sentiments, comme des oiseaux aux petites\mbox{}\newline 
\hspace*{6pt}\hspace*{6pt} pattes sur de la neige.{</\textbf{item}>}\mbox{}\newline 
{</\textbf{list}>}\end{shaded}\egroup 


    \item[{Modèle de contenu}]
  \mbox{}\hfill\\[-10pt]\begin{Verbatim}[fontsize=\small]
<content>
 <macroRef key="macro.phraseSeq"/>
</content>
    
\end{Verbatim}

    \item[{Schéma Declaration}]
  \mbox{}\hfill\\[-10pt]\begin{Verbatim}[fontsize=\small]
element label
{
   tei_att.global.attributes,
   tei_att.typed.attributes,
   tei_att.placement.attributes,
   tei_att.written.attributes,
   tei_macro.phraseSeq}
\end{Verbatim}

\end{reflist}  \index{langUsage=<langUsage>|oddindex}
\begin{reflist}
\item[]\begin{specHead}{TEI.langUsage}{<langUsage> }(langue utilisée) décrit les langues, variétés de langues, registres, dialectes, etc. présents à l’intérieur d’un texte. [\xref{http://www.tei-c.org/release/doc/tei-p5-doc/en/html/HD.html\#HD41}{2.4.2. Language Usage} \xref{http://www.tei-c.org/release/doc/tei-p5-doc/en/html/HD.html\#HD4}{2.4. The Profile Description} \xref{http://www.tei-c.org/release/doc/tei-p5-doc/en/html/CC.html\#CCAS2}{15.3.2. Declarable Elements}]\end{specHead} 
    \item[{Module}]
  header
    \item[{Attributs}]
  Attributs \hyperref[TEI.att.global]{att.global} (\textit{@xml:id}, \textit{@n}, \textit{@xml:lang}, \textit{@xml:base}, \textit{@xml:space})  (\hyperref[TEI.att.global.rendition]{att.global.rendition} (\textit{@rend}, \textit{@style}, \textit{@rendition})) (\hyperref[TEI.att.global.linking]{att.global.linking} (\textit{@corresp}, \textit{@synch}, \textit{@sameAs}, \textit{@copyOf}, \textit{@next}, \textit{@prev}, \textit{@exclude}, \textit{@select})) (\hyperref[TEI.att.global.analytic]{att.global.analytic} (\textit{@ana})) (\hyperref[TEI.att.global.facs]{att.global.facs} (\textit{@facs})) (\hyperref[TEI.att.global.change]{att.global.change} (\textit{@change})) (\hyperref[TEI.att.global.responsibility]{att.global.responsibility} (\textit{@cert}, \textit{@resp})) (\hyperref[TEI.att.global.source]{att.global.source} (\textit{@source})) \hyperref[TEI.att.declarable]{att.declarable} (\textit{@default}) 
    \item[{Membre du}]
  \hyperref[TEI.model.profileDescPart]{model.profileDescPart}
    \item[{Contenu dans}]
  
    \item[header: ]
   \hyperref[TEI.profileDesc]{profileDesc}
    \item[{Peut contenir}]
  
    \item[core: ]
   \hyperref[TEI.p]{p}\par 
    \item[header: ]
   \hyperref[TEI.language]{language}\par 
    \item[linking: ]
   \hyperref[TEI.ab]{ab}
    \item[{Exemple}]
  \leavevmode\bgroup\exampleFont \begin{shaded}\noindent\mbox{}{<\textbf{langUsage}>}\mbox{}\newline 
\hspace*{6pt}{<\textbf{language}\hspace*{6pt}{ident}="{fr-CA}"\hspace*{6pt}{usage}="{60}">}Québecois{</\textbf{language}>}\mbox{}\newline 
\hspace*{6pt}{<\textbf{language}\hspace*{6pt}{ident}="{en-CA}"\hspace*{6pt}{usage}="{20}">}Anglais canadien des affaires{</\textbf{language}>}\mbox{}\newline 
\hspace*{6pt}{<\textbf{language}\hspace*{6pt}{ident}="{en-GB}"\hspace*{6pt}{usage}="{20}">}Anglais de Grande-Bretagne{</\textbf{language}>}\mbox{}\newline 
{</\textbf{langUsage}>}\end{shaded}\egroup 


    \item[{Modèle de contenu}]
  \mbox{}\hfill\\[-10pt]\begin{Verbatim}[fontsize=\small]
<content>
 <alternate maxOccurs="1" minOccurs="1">
  <classRef key="model.pLike"
   maxOccurs="unbounded" minOccurs="1"/>
  <elementRef key="language"
   maxOccurs="unbounded" minOccurs="1"/>
 </alternate>
</content>
    
\end{Verbatim}

    \item[{Schéma Declaration}]
  \mbox{}\hfill\\[-10pt]\begin{Verbatim}[fontsize=\small]
element langUsage
{
   tei_att.global.attributes,
   tei_att.declarable.attributes,
   ( tei_model.pLike+ | tei_language+ )
}
\end{Verbatim}

\end{reflist}  \index{language=<language>|oddindex}\index{ident=@ident!<language>|oddindex}\index{usage=@usage!<language>|oddindex}
\begin{reflist}
\item[]\begin{specHead}{TEI.language}{<language> }(langue) caractérise une langue ou une variété de langue utilisée dans un texte. [\xref{http://www.tei-c.org/release/doc/tei-p5-doc/en/html/HD.html\#HD41}{2.4.2. Language Usage}]\end{specHead} 
    \item[{Module}]
  header
    \item[{Attributs}]
  Attributs \hyperref[TEI.att.global]{att.global} (\textit{@xml:id}, \textit{@n}, \textit{@xml:lang}, \textit{@xml:base}, \textit{@xml:space})  (\hyperref[TEI.att.global.rendition]{att.global.rendition} (\textit{@rend}, \textit{@style}, \textit{@rendition})) (\hyperref[TEI.att.global.linking]{att.global.linking} (\textit{@corresp}, \textit{@synch}, \textit{@sameAs}, \textit{@copyOf}, \textit{@next}, \textit{@prev}, \textit{@exclude}, \textit{@select})) (\hyperref[TEI.att.global.analytic]{att.global.analytic} (\textit{@ana})) (\hyperref[TEI.att.global.facs]{att.global.facs} (\textit{@facs})) (\hyperref[TEI.att.global.change]{att.global.change} (\textit{@change})) (\hyperref[TEI.att.global.responsibility]{att.global.responsibility} (\textit{@cert}, \textit{@resp})) (\hyperref[TEI.att.global.source]{att.global.source} (\textit{@source})) \hfil\\[-10pt]\begin{sansreflist}
    \item[@ident]
  (identificateur) fournit un code de langue issu de la recommandation \xref{https://tools.ietf.org/html/bcp47}{BCP 47} (ou son successeur) utilisé pour identifier la langue précisée par cet élément, référencé par l’attribut global {\itshape xml:lang} s’appliquant à l’élément considéré.
\begin{reflist}
    \item[{Statut}]
  Requis
    \item[{Type de données}]
  \hyperref[TEI.teidata.language]{teidata.language}
\end{reflist}  
    \item[@usage]
  précise approximativement le pourcentage du volume de texte utilisant cette langue.
\begin{reflist}
    \item[{Statut}]
  Optionel
    \item[{Type de données}]
  \xref{https://www.w3.org/TR/xmlschema-2/\#nonNegativeInteger}{nonNegativeInteger}
\end{reflist}  
\end{sansreflist}  
    \item[{Contenu dans}]
  
    \item[header: ]
   \hyperref[TEI.langUsage]{langUsage}
    \item[{Peut contenir}]
  
    \item[analysis: ]
   \hyperref[TEI.interp]{interp} \hyperref[TEI.interpGrp]{interpGrp} \hyperref[TEI.span]{span} \hyperref[TEI.spanGrp]{spanGrp}\par 
    \item[core: ]
   \hyperref[TEI.abbr]{abbr} \hyperref[TEI.address]{address} \hyperref[TEI.cb]{cb} \hyperref[TEI.choice]{choice} \hyperref[TEI.date]{date} \hyperref[TEI.distinct]{distinct} \hyperref[TEI.email]{email} \hyperref[TEI.emph]{emph} \hyperref[TEI.expan]{expan} \hyperref[TEI.foreign]{foreign} \hyperref[TEI.gap]{gap} \hyperref[TEI.gb]{gb} \hyperref[TEI.gloss]{gloss} \hyperref[TEI.hi]{hi} \hyperref[TEI.index]{index} \hyperref[TEI.lb]{lb} \hyperref[TEI.measure]{measure} \hyperref[TEI.measureGrp]{measureGrp} \hyperref[TEI.mentioned]{mentioned} \hyperref[TEI.milestone]{milestone} \hyperref[TEI.name]{name} \hyperref[TEI.note]{note} \hyperref[TEI.num]{num} \hyperref[TEI.pb]{pb} \hyperref[TEI.ptr]{ptr} \hyperref[TEI.ref]{ref} \hyperref[TEI.rs]{rs} \hyperref[TEI.soCalled]{soCalled} \hyperref[TEI.term]{term} \hyperref[TEI.time]{time} \hyperref[TEI.title]{title}\par 
    \item[figures: ]
   \hyperref[TEI.figure]{figure} \hyperref[TEI.notatedMusic]{notatedMusic}\par 
    \item[header: ]
   \hyperref[TEI.idno]{idno}\par 
    \item[iso-fs: ]
   \hyperref[TEI.fLib]{fLib} \hyperref[TEI.fs]{fs} \hyperref[TEI.fvLib]{fvLib}\par 
    \item[linking: ]
   \hyperref[TEI.alt]{alt} \hyperref[TEI.altGrp]{altGrp} \hyperref[TEI.anchor]{anchor} \hyperref[TEI.join]{join} \hyperref[TEI.joinGrp]{joinGrp} \hyperref[TEI.link]{link} \hyperref[TEI.linkGrp]{linkGrp} \hyperref[TEI.timeline]{timeline}\par 
    \item[msdescription: ]
   \hyperref[TEI.catchwords]{catchwords} \hyperref[TEI.depth]{depth} \hyperref[TEI.dim]{dim} \hyperref[TEI.dimensions]{dimensions} \hyperref[TEI.height]{height} \hyperref[TEI.heraldry]{heraldry} \hyperref[TEI.locus]{locus} \hyperref[TEI.locusGrp]{locusGrp} \hyperref[TEI.material]{material} \hyperref[TEI.objectType]{objectType} \hyperref[TEI.origDate]{origDate} \hyperref[TEI.origPlace]{origPlace} \hyperref[TEI.secFol]{secFol} \hyperref[TEI.signatures]{signatures} \hyperref[TEI.source]{source} \hyperref[TEI.stamp]{stamp} \hyperref[TEI.watermark]{watermark} \hyperref[TEI.width]{width}\par 
    \item[namesdates: ]
   \hyperref[TEI.addName]{addName} \hyperref[TEI.affiliation]{affiliation} \hyperref[TEI.country]{country} \hyperref[TEI.forename]{forename} \hyperref[TEI.genName]{genName} \hyperref[TEI.geogName]{geogName} \hyperref[TEI.location]{location} \hyperref[TEI.nameLink]{nameLink} \hyperref[TEI.orgName]{orgName} \hyperref[TEI.persName]{persName} \hyperref[TEI.placeName]{placeName} \hyperref[TEI.region]{region} \hyperref[TEI.roleName]{roleName} \hyperref[TEI.settlement]{settlement} \hyperref[TEI.state]{state} \hyperref[TEI.surname]{surname}\par 
    \item[transcr: ]
   \hyperref[TEI.addSpan]{addSpan} \hyperref[TEI.am]{am} \hyperref[TEI.damageSpan]{damageSpan} \hyperref[TEI.delSpan]{delSpan} \hyperref[TEI.ex]{ex} \hyperref[TEI.fw]{fw} \hyperref[TEI.listTranspose]{listTranspose} \hyperref[TEI.metamark]{metamark} \hyperref[TEI.space]{space} \hyperref[TEI.subst]{subst} \hyperref[TEI.substJoin]{substJoin}\par des données textuelles
    \item[{Note}]
  \par
Dans le cas particulier des variétés de langues, l'élément contiendra un texte caractérisant mais non structuré.
    \item[{Exemple}]
  \leavevmode\bgroup\exampleFont \begin{shaded}\noindent\mbox{}{<\textbf{langUsage}>}\mbox{}\newline 
\hspace*{6pt}{<\textbf{language}\hspace*{6pt}{ident}="{en-US}"\hspace*{6pt}{usage}="{75}">}Anglais américain moderne{</\textbf{language}>}\mbox{}\newline 
\hspace*{6pt}{<\textbf{language}\hspace*{6pt}{ident}="{i-az-Arab}"\hspace*{6pt}{usage}="{20}">}Azerbaijanais en caractères arabes{</\textbf{language}>}\mbox{}\newline 
\hspace*{6pt}{<\textbf{language}\hspace*{6pt}{ident}="{x-verlan}"\hspace*{6pt}{usage}="{05}">}verlan{</\textbf{language}>}\mbox{}\newline 
{</\textbf{langUsage}>}\end{shaded}\egroup 


    \item[{Modèle de contenu}]
  \mbox{}\hfill\\[-10pt]\begin{Verbatim}[fontsize=\small]
<content>
 <macroRef key="macro.phraseSeq.limited"/>
</content>
    
\end{Verbatim}

    \item[{Schéma Declaration}]
  \mbox{}\hfill\\[-10pt]\begin{Verbatim}[fontsize=\small]
element language
{
   tei_att.global.attributes,
   attribute ident { text },
   attribute usage { text }?,
   tei_macro.phraseSeq.limited}
\end{Verbatim}

\end{reflist}  \index{layout=<layout>|oddindex}\index{columns=@columns!<layout>|oddindex}\index{ruledLines=@ruledLines!<layout>|oddindex}\index{writtenLines=@writtenLines!<layout>|oddindex}
\begin{reflist}
\item[]\begin{specHead}{TEI.layout}{<layout> }(mise en page) décrit comment le texte est disposé sur la page, ce qui inclut les informations sur d'éventuels systèmes de réglure, de piqûre ou d'autres techniques de préparation de la page. [\xref{http://www.tei-c.org/release/doc/tei-p5-doc/en/html/MS.html\#msph2}{10.7.2. Writing, Decoration, and Other Notations}]\end{specHead} 
    \item[{Module}]
  msdescription
    \item[{Attributs}]
  Attributs \hyperref[TEI.att.global]{att.global} (\textit{@xml:id}, \textit{@n}, \textit{@xml:lang}, \textit{@xml:base}, \textit{@xml:space})  (\hyperref[TEI.att.global.rendition]{att.global.rendition} (\textit{@rend}, \textit{@style}, \textit{@rendition})) (\hyperref[TEI.att.global.linking]{att.global.linking} (\textit{@corresp}, \textit{@synch}, \textit{@sameAs}, \textit{@copyOf}, \textit{@next}, \textit{@prev}, \textit{@exclude}, \textit{@select})) (\hyperref[TEI.att.global.analytic]{att.global.analytic} (\textit{@ana})) (\hyperref[TEI.att.global.facs]{att.global.facs} (\textit{@facs})) (\hyperref[TEI.att.global.change]{att.global.change} (\textit{@change})) (\hyperref[TEI.att.global.responsibility]{att.global.responsibility} (\textit{@cert}, \textit{@resp})) (\hyperref[TEI.att.global.source]{att.global.source} (\textit{@source})) \hfil\\[-10pt]\begin{sansreflist}
    \item[@columns]
  (colonnes) spécifie le nombre de colonnes présentes sur une page
\begin{reflist}
    \item[{Statut}]
  Optionel
    \item[{Type de données}]
  1–2 occurrences de \hyperref[TEI.teidata.count]{teidata.count} séparé par un espace
\end{reflist}  
    \item[@ruledLines]
  (lignes de réglure) spécifie le nombre de lignes de réglure présentes par colonne
\begin{reflist}
    \item[{Statut}]
  Optionel
    \item[{Type de données}]
  1–2 occurrences de \hyperref[TEI.teidata.count]{teidata.count} séparé par un espace
\end{reflist}  
    \item[@writtenLines]
  (lignes d'écriture) spécifie le nombre de lignes écrites par colonne
\begin{reflist}
    \item[{Statut}]
  Optionel
    \item[{Type de données}]
  1–2 occurrences de \hyperref[TEI.teidata.count]{teidata.count} séparé par un espace
\end{reflist}  
\end{sansreflist}  
    \item[{Contenu dans}]
  
    \item[msdescription: ]
   \hyperref[TEI.layoutDesc]{layoutDesc}
    \item[{Peut contenir}]
  
    \item[analysis: ]
   \hyperref[TEI.c]{c} \hyperref[TEI.cl]{cl} \hyperref[TEI.interp]{interp} \hyperref[TEI.interpGrp]{interpGrp} \hyperref[TEI.m]{m} \hyperref[TEI.pc]{pc} \hyperref[TEI.phr]{phr} \hyperref[TEI.s]{s} \hyperref[TEI.span]{span} \hyperref[TEI.spanGrp]{spanGrp} \hyperref[TEI.w]{w}\par 
    \item[core: ]
   \hyperref[TEI.abbr]{abbr} \hyperref[TEI.add]{add} \hyperref[TEI.address]{address} \hyperref[TEI.bibl]{bibl} \hyperref[TEI.biblStruct]{biblStruct} \hyperref[TEI.binaryObject]{binaryObject} \hyperref[TEI.cb]{cb} \hyperref[TEI.choice]{choice} \hyperref[TEI.cit]{cit} \hyperref[TEI.corr]{corr} \hyperref[TEI.date]{date} \hyperref[TEI.del]{del} \hyperref[TEI.desc]{desc} \hyperref[TEI.distinct]{distinct} \hyperref[TEI.email]{email} \hyperref[TEI.emph]{emph} \hyperref[TEI.expan]{expan} \hyperref[TEI.foreign]{foreign} \hyperref[TEI.gap]{gap} \hyperref[TEI.gb]{gb} \hyperref[TEI.gloss]{gloss} \hyperref[TEI.graphic]{graphic} \hyperref[TEI.hi]{hi} \hyperref[TEI.index]{index} \hyperref[TEI.l]{l} \hyperref[TEI.label]{label} \hyperref[TEI.lb]{lb} \hyperref[TEI.lg]{lg} \hyperref[TEI.list]{list} \hyperref[TEI.listBibl]{listBibl} \hyperref[TEI.measure]{measure} \hyperref[TEI.measureGrp]{measureGrp} \hyperref[TEI.media]{media} \hyperref[TEI.mentioned]{mentioned} \hyperref[TEI.milestone]{milestone} \hyperref[TEI.name]{name} \hyperref[TEI.note]{note} \hyperref[TEI.num]{num} \hyperref[TEI.orig]{orig} \hyperref[TEI.p]{p} \hyperref[TEI.pb]{pb} \hyperref[TEI.ptr]{ptr} \hyperref[TEI.q]{q} \hyperref[TEI.quote]{quote} \hyperref[TEI.ref]{ref} \hyperref[TEI.reg]{reg} \hyperref[TEI.rs]{rs} \hyperref[TEI.said]{said} \hyperref[TEI.sic]{sic} \hyperref[TEI.soCalled]{soCalled} \hyperref[TEI.sp]{sp} \hyperref[TEI.stage]{stage} \hyperref[TEI.term]{term} \hyperref[TEI.time]{time} \hyperref[TEI.title]{title} \hyperref[TEI.unclear]{unclear}\par 
    \item[derived-module-tei.istex: ]
   \hyperref[TEI.math]{math} \hyperref[TEI.mrow]{mrow}\par 
    \item[figures: ]
   \hyperref[TEI.figure]{figure} \hyperref[TEI.formula]{formula} \hyperref[TEI.notatedMusic]{notatedMusic} \hyperref[TEI.table]{table}\par 
    \item[header: ]
   \hyperref[TEI.biblFull]{biblFull} \hyperref[TEI.idno]{idno}\par 
    \item[iso-fs: ]
   \hyperref[TEI.fLib]{fLib} \hyperref[TEI.fs]{fs} \hyperref[TEI.fvLib]{fvLib}\par 
    \item[linking: ]
   \hyperref[TEI.ab]{ab} \hyperref[TEI.alt]{alt} \hyperref[TEI.altGrp]{altGrp} \hyperref[TEI.anchor]{anchor} \hyperref[TEI.join]{join} \hyperref[TEI.joinGrp]{joinGrp} \hyperref[TEI.link]{link} \hyperref[TEI.linkGrp]{linkGrp} \hyperref[TEI.seg]{seg} \hyperref[TEI.timeline]{timeline}\par 
    \item[msdescription: ]
   \hyperref[TEI.catchwords]{catchwords} \hyperref[TEI.depth]{depth} \hyperref[TEI.dim]{dim} \hyperref[TEI.dimensions]{dimensions} \hyperref[TEI.height]{height} \hyperref[TEI.heraldry]{heraldry} \hyperref[TEI.locus]{locus} \hyperref[TEI.locusGrp]{locusGrp} \hyperref[TEI.material]{material} \hyperref[TEI.msDesc]{msDesc} \hyperref[TEI.objectType]{objectType} \hyperref[TEI.origDate]{origDate} \hyperref[TEI.origPlace]{origPlace} \hyperref[TEI.secFol]{secFol} \hyperref[TEI.signatures]{signatures} \hyperref[TEI.source]{source} \hyperref[TEI.stamp]{stamp} \hyperref[TEI.watermark]{watermark} \hyperref[TEI.width]{width}\par 
    \item[namesdates: ]
   \hyperref[TEI.addName]{addName} \hyperref[TEI.affiliation]{affiliation} \hyperref[TEI.country]{country} \hyperref[TEI.forename]{forename} \hyperref[TEI.genName]{genName} \hyperref[TEI.geogName]{geogName} \hyperref[TEI.listOrg]{listOrg} \hyperref[TEI.listPlace]{listPlace} \hyperref[TEI.location]{location} \hyperref[TEI.nameLink]{nameLink} \hyperref[TEI.orgName]{orgName} \hyperref[TEI.persName]{persName} \hyperref[TEI.placeName]{placeName} \hyperref[TEI.region]{region} \hyperref[TEI.roleName]{roleName} \hyperref[TEI.settlement]{settlement} \hyperref[TEI.state]{state} \hyperref[TEI.surname]{surname}\par 
    \item[spoken: ]
   \hyperref[TEI.annotationBlock]{annotationBlock}\par 
    \item[textstructure: ]
   \hyperref[TEI.floatingText]{floatingText}\par 
    \item[transcr: ]
   \hyperref[TEI.addSpan]{addSpan} \hyperref[TEI.am]{am} \hyperref[TEI.damage]{damage} \hyperref[TEI.damageSpan]{damageSpan} \hyperref[TEI.delSpan]{delSpan} \hyperref[TEI.ex]{ex} \hyperref[TEI.fw]{fw} \hyperref[TEI.handShift]{handShift} \hyperref[TEI.listTranspose]{listTranspose} \hyperref[TEI.metamark]{metamark} \hyperref[TEI.mod]{mod} \hyperref[TEI.redo]{redo} \hyperref[TEI.restore]{restore} \hyperref[TEI.retrace]{retrace} \hyperref[TEI.secl]{secl} \hyperref[TEI.space]{space} \hyperref[TEI.subst]{subst} \hyperref[TEI.substJoin]{substJoin} \hyperref[TEI.supplied]{supplied} \hyperref[TEI.surplus]{surplus} \hyperref[TEI.undo]{undo}\par des données textuelles
    \item[{Exemple}]
  \leavevmode\bgroup\exampleFont \begin{shaded}\noindent\mbox{}{<\textbf{layout}\hspace*{6pt}{columns}="{1}"\hspace*{6pt}{ruledLines}="{25 32}">} Most pages have between 25 and 32 long lines ruled\mbox{}\newline 
 in lead.{</\textbf{layout}>}\end{shaded}\egroup 


    \item[{Exemple}]
  \leavevmode\bgroup\exampleFont \begin{shaded}\noindent\mbox{}{<\textbf{layout}\hspace*{6pt}{columns}="{2}"\hspace*{6pt}{ruledLines}="{42}">}\mbox{}\newline 
\hspace*{6pt}{<\textbf{p}>}2 columns of 42 lines ruled in ink, with central rule between the columns.{</\textbf{p}>}\mbox{}\newline 
{</\textbf{layout}>}\end{shaded}\egroup 


    \item[{Exemple}]
  \leavevmode\bgroup\exampleFont \begin{shaded}\noindent\mbox{}{<\textbf{layout}\hspace*{6pt}{columns}="{1 2}"\hspace*{6pt}{writtenLines}="{40 50}">}\mbox{}\newline 
\hspace*{6pt}{<\textbf{p}>}Some pages have 2 columns, with central rule between the columns; each column with\mbox{}\newline 
\hspace*{6pt}\hspace*{6pt} between 40 and 50 lines of writing.{</\textbf{p}>}\mbox{}\newline 
{</\textbf{layout}>}\end{shaded}\egroup 


    \item[{Modèle de contenu}]
  \mbox{}\hfill\\[-10pt]\begin{Verbatim}[fontsize=\small]
<content>
 <macroRef key="macro.specialPara"/>
</content>
    
\end{Verbatim}

    \item[{Schéma Declaration}]
  \mbox{}\hfill\\[-10pt]\begin{Verbatim}[fontsize=\small]
element layout
{
   tei_att.global.attributes,
   attribute columns { list { ? } }?,
   attribute ruledLines { list { ? } }?,
   attribute writtenLines { list { ? } }?,
   tei_macro.specialPara}
\end{Verbatim}

\end{reflist}  \index{layoutDesc=<layoutDesc>|oddindex}
\begin{reflist}
\item[]\begin{specHead}{TEI.layoutDesc}{<layoutDesc> }(description de la mise en page) rassemble les descriptions des mises en page d' un manuscrit. [\xref{http://www.tei-c.org/release/doc/tei-p5-doc/en/html/MS.html\#msph2}{10.7.2. Writing, Decoration, and Other Notations}]\end{specHead} 
    \item[{Module}]
  msdescription
    \item[{Attributs}]
  Attributs \hyperref[TEI.att.global]{att.global} (\textit{@xml:id}, \textit{@n}, \textit{@xml:lang}, \textit{@xml:base}, \textit{@xml:space})  (\hyperref[TEI.att.global.rendition]{att.global.rendition} (\textit{@rend}, \textit{@style}, \textit{@rendition})) (\hyperref[TEI.att.global.linking]{att.global.linking} (\textit{@corresp}, \textit{@synch}, \textit{@sameAs}, \textit{@copyOf}, \textit{@next}, \textit{@prev}, \textit{@exclude}, \textit{@select})) (\hyperref[TEI.att.global.analytic]{att.global.analytic} (\textit{@ana})) (\hyperref[TEI.att.global.facs]{att.global.facs} (\textit{@facs})) (\hyperref[TEI.att.global.change]{att.global.change} (\textit{@change})) (\hyperref[TEI.att.global.responsibility]{att.global.responsibility} (\textit{@cert}, \textit{@resp})) (\hyperref[TEI.att.global.source]{att.global.source} (\textit{@source}))
    \item[{Contenu dans}]
  
    \item[msdescription: ]
   \hyperref[TEI.objectDesc]{objectDesc}
    \item[{Peut contenir}]
  
    \item[core: ]
   \hyperref[TEI.p]{p}\par 
    \item[linking: ]
   \hyperref[TEI.ab]{ab}\par 
    \item[msdescription: ]
   \hyperref[TEI.layout]{layout} \hyperref[TEI.summary]{summary}
    \item[{Exemple}]
  \leavevmode\bgroup\exampleFont \begin{shaded}\noindent\mbox{}{<\textbf{layoutDesc}>}\mbox{}\newline 
\hspace*{6pt}{<\textbf{p}>}Most pages have between 25 and 32 long lines ruled in lead.{</\textbf{p}>}\mbox{}\newline 
{</\textbf{layoutDesc}>}\end{shaded}\egroup 


    \item[{Exemple}]
  \leavevmode\bgroup\exampleFont \begin{shaded}\noindent\mbox{}{<\textbf{layoutDesc}>}\mbox{}\newline 
\hspace*{6pt}{<\textbf{layout}\hspace*{6pt}{columns}="{2}"\hspace*{6pt}{ruledLines}="{42}">}\mbox{}\newline 
\hspace*{6pt}\hspace*{6pt}{<\textbf{p}>}\mbox{}\newline 
\hspace*{6pt}\hspace*{6pt}\hspace*{6pt}{<\textbf{locus}\hspace*{6pt}{from}="{f12r}"\hspace*{6pt}{to}="{f15v}"/>} 2 columns of 42 lines pricked and ruled in ink, with\mbox{}\newline 
\hspace*{6pt}\hspace*{6pt}\hspace*{6pt}\hspace*{6pt} central rule between the columns.{</\textbf{p}>}\mbox{}\newline 
\hspace*{6pt}{</\textbf{layout}>}\mbox{}\newline 
\hspace*{6pt}{<\textbf{layout}\hspace*{6pt}{columns}="{3}">}\mbox{}\newline 
\hspace*{6pt}\hspace*{6pt}{<\textbf{p}>}\mbox{}\newline 
\hspace*{6pt}\hspace*{6pt}\hspace*{6pt}{<\textbf{locus}\hspace*{6pt}{from}="{f16}"/>}Prickings for three columns are visible.{</\textbf{p}>}\mbox{}\newline 
\hspace*{6pt}{</\textbf{layout}>}\mbox{}\newline 
{</\textbf{layoutDesc}>}\end{shaded}\egroup 


    \item[{Modèle de contenu}]
  \mbox{}\hfill\\[-10pt]\begin{Verbatim}[fontsize=\small]
<content>
 <alternate maxOccurs="1" minOccurs="1">
  <classRef key="model.pLike"
   maxOccurs="unbounded" minOccurs="1"/>
  <sequence maxOccurs="1" minOccurs="1">
   <elementRef key="summary" minOccurs="0"/>
   <elementRef key="layout"
    maxOccurs="unbounded" minOccurs="1"/>
  </sequence>
 </alternate>
</content>
    
\end{Verbatim}

    \item[{Schéma Declaration}]
  \mbox{}\hfill\\[-10pt]\begin{Verbatim}[fontsize=\small]
element layoutDesc
{
   tei_att.global.attributes,
   ( tei_model.pLike+ | ( tei_summary?, tei_layout+ ) )
}
\end{Verbatim}

\end{reflist}  \index{lb=<lb>|oddindex}
\begin{reflist}
\item[]\begin{specHead}{TEI.lb}{<lb> }(saut de ligne) marque le début d'une nouvelle ligne (typographique) dans une édition ou dans une version d'un texte. [\xref{http://www.tei-c.org/release/doc/tei-p5-doc/en/html/CO.html\#CORS5}{3.10.3. Milestone Elements} \xref{http://www.tei-c.org/release/doc/tei-p5-doc/en/html/DR.html\#DRPAL}{7.2.5. Speech Contents}]\end{specHead} 
    \item[{Module}]
  core
    \item[{Attributs}]
  Attributs \hyperref[TEI.att.global]{att.global} (\textit{@xml:id}, \textit{@n}, \textit{@xml:lang}, \textit{@xml:base}, \textit{@xml:space})  (\hyperref[TEI.att.global.rendition]{att.global.rendition} (\textit{@rend}, \textit{@style}, \textit{@rendition})) (\hyperref[TEI.att.global.linking]{att.global.linking} (\textit{@corresp}, \textit{@synch}, \textit{@sameAs}, \textit{@copyOf}, \textit{@next}, \textit{@prev}, \textit{@exclude}, \textit{@select})) (\hyperref[TEI.att.global.analytic]{att.global.analytic} (\textit{@ana})) (\hyperref[TEI.att.global.facs]{att.global.facs} (\textit{@facs})) (\hyperref[TEI.att.global.change]{att.global.change} (\textit{@change})) (\hyperref[TEI.att.global.responsibility]{att.global.responsibility} (\textit{@cert}, \textit{@resp})) (\hyperref[TEI.att.global.source]{att.global.source} (\textit{@source})) \hyperref[TEI.att.typed]{att.typed} (\textit{@type}, \textit{@subtype}) \hyperref[TEI.att.edition]{att.edition} (\textit{@ed}, \textit{@edRef}) \hyperref[TEI.att.spanning]{att.spanning} (\textit{@spanTo}) \hyperref[TEI.att.breaking]{att.breaking} (\textit{@break}) 
    \item[{Membre du}]
  \hyperref[TEI.model.milestoneLike]{model.milestoneLike}
    \item[{Contenu dans}]
  
    \item[analysis: ]
   \hyperref[TEI.cl]{cl} \hyperref[TEI.m]{m} \hyperref[TEI.phr]{phr} \hyperref[TEI.s]{s} \hyperref[TEI.span]{span} \hyperref[TEI.w]{w}\par 
    \item[core: ]
   \hyperref[TEI.abbr]{abbr} \hyperref[TEI.add]{add} \hyperref[TEI.addrLine]{addrLine} \hyperref[TEI.address]{address} \hyperref[TEI.author]{author} \hyperref[TEI.bibl]{bibl} \hyperref[TEI.biblScope]{biblScope} \hyperref[TEI.cit]{cit} \hyperref[TEI.citedRange]{citedRange} \hyperref[TEI.corr]{corr} \hyperref[TEI.date]{date} \hyperref[TEI.del]{del} \hyperref[TEI.distinct]{distinct} \hyperref[TEI.editor]{editor} \hyperref[TEI.email]{email} \hyperref[TEI.emph]{emph} \hyperref[TEI.expan]{expan} \hyperref[TEI.foreign]{foreign} \hyperref[TEI.gloss]{gloss} \hyperref[TEI.head]{head} \hyperref[TEI.headItem]{headItem} \hyperref[TEI.headLabel]{headLabel} \hyperref[TEI.hi]{hi} \hyperref[TEI.imprint]{imprint} \hyperref[TEI.item]{item} \hyperref[TEI.l]{l} \hyperref[TEI.label]{label} \hyperref[TEI.lg]{lg} \hyperref[TEI.list]{list} \hyperref[TEI.listBibl]{listBibl} \hyperref[TEI.measure]{measure} \hyperref[TEI.mentioned]{mentioned} \hyperref[TEI.name]{name} \hyperref[TEI.note]{note} \hyperref[TEI.num]{num} \hyperref[TEI.orig]{orig} \hyperref[TEI.p]{p} \hyperref[TEI.pubPlace]{pubPlace} \hyperref[TEI.publisher]{publisher} \hyperref[TEI.q]{q} \hyperref[TEI.quote]{quote} \hyperref[TEI.ref]{ref} \hyperref[TEI.reg]{reg} \hyperref[TEI.resp]{resp} \hyperref[TEI.rs]{rs} \hyperref[TEI.said]{said} \hyperref[TEI.series]{series} \hyperref[TEI.sic]{sic} \hyperref[TEI.soCalled]{soCalled} \hyperref[TEI.sp]{sp} \hyperref[TEI.speaker]{speaker} \hyperref[TEI.stage]{stage} \hyperref[TEI.street]{street} \hyperref[TEI.term]{term} \hyperref[TEI.textLang]{textLang} \hyperref[TEI.time]{time} \hyperref[TEI.title]{title} \hyperref[TEI.unclear]{unclear}\par 
    \item[figures: ]
   \hyperref[TEI.cell]{cell} \hyperref[TEI.figure]{figure} \hyperref[TEI.table]{table}\par 
    \item[header: ]
   \hyperref[TEI.authority]{authority} \hyperref[TEI.change]{change} \hyperref[TEI.classCode]{classCode} \hyperref[TEI.distributor]{distributor} \hyperref[TEI.edition]{edition} \hyperref[TEI.extent]{extent} \hyperref[TEI.funder]{funder} \hyperref[TEI.language]{language} \hyperref[TEI.licence]{licence}\par 
    \item[linking: ]
   \hyperref[TEI.ab]{ab} \hyperref[TEI.seg]{seg}\par 
    \item[msdescription: ]
   \hyperref[TEI.accMat]{accMat} \hyperref[TEI.acquisition]{acquisition} \hyperref[TEI.additions]{additions} \hyperref[TEI.catchwords]{catchwords} \hyperref[TEI.collation]{collation} \hyperref[TEI.colophon]{colophon} \hyperref[TEI.condition]{condition} \hyperref[TEI.custEvent]{custEvent} \hyperref[TEI.decoNote]{decoNote} \hyperref[TEI.explicit]{explicit} \hyperref[TEI.filiation]{filiation} \hyperref[TEI.finalRubric]{finalRubric} \hyperref[TEI.foliation]{foliation} \hyperref[TEI.heraldry]{heraldry} \hyperref[TEI.incipit]{incipit} \hyperref[TEI.layout]{layout} \hyperref[TEI.material]{material} \hyperref[TEI.msItem]{msItem} \hyperref[TEI.musicNotation]{musicNotation} \hyperref[TEI.objectType]{objectType} \hyperref[TEI.origDate]{origDate} \hyperref[TEI.origPlace]{origPlace} \hyperref[TEI.origin]{origin} \hyperref[TEI.provenance]{provenance} \hyperref[TEI.rubric]{rubric} \hyperref[TEI.secFol]{secFol} \hyperref[TEI.signatures]{signatures} \hyperref[TEI.source]{source} \hyperref[TEI.stamp]{stamp} \hyperref[TEI.summary]{summary} \hyperref[TEI.support]{support} \hyperref[TEI.surrogates]{surrogates} \hyperref[TEI.typeNote]{typeNote} \hyperref[TEI.watermark]{watermark}\par 
    \item[namesdates: ]
   \hyperref[TEI.addName]{addName} \hyperref[TEI.affiliation]{affiliation} \hyperref[TEI.country]{country} \hyperref[TEI.forename]{forename} \hyperref[TEI.genName]{genName} \hyperref[TEI.geogName]{geogName} \hyperref[TEI.nameLink]{nameLink} \hyperref[TEI.org]{org} \hyperref[TEI.orgName]{orgName} \hyperref[TEI.persName]{persName} \hyperref[TEI.person]{person} \hyperref[TEI.personGrp]{personGrp} \hyperref[TEI.persona]{persona} \hyperref[TEI.placeName]{placeName} \hyperref[TEI.region]{region} \hyperref[TEI.roleName]{roleName} \hyperref[TEI.settlement]{settlement} \hyperref[TEI.surname]{surname}\par 
    \item[textstructure: ]
   \hyperref[TEI.back]{back} \hyperref[TEI.body]{body} \hyperref[TEI.div]{div} \hyperref[TEI.docAuthor]{docAuthor} \hyperref[TEI.docDate]{docDate} \hyperref[TEI.docEdition]{docEdition} \hyperref[TEI.docTitle]{docTitle} \hyperref[TEI.floatingText]{floatingText} \hyperref[TEI.front]{front} \hyperref[TEI.group]{group} \hyperref[TEI.text]{text} \hyperref[TEI.titlePage]{titlePage} \hyperref[TEI.titlePart]{titlePart}\par 
    \item[transcr: ]
   \hyperref[TEI.damage]{damage} \hyperref[TEI.fw]{fw} \hyperref[TEI.line]{line} \hyperref[TEI.metamark]{metamark} \hyperref[TEI.mod]{mod} \hyperref[TEI.restore]{restore} \hyperref[TEI.retrace]{retrace} \hyperref[TEI.secl]{secl} \hyperref[TEI.sourceDoc]{sourceDoc} \hyperref[TEI.subst]{subst} \hyperref[TEI.supplied]{supplied} \hyperref[TEI.surface]{surface} \hyperref[TEI.surfaceGrp]{surfaceGrp} \hyperref[TEI.surplus]{surplus} \hyperref[TEI.zone]{zone}
    \item[{Peut contenir}]
  Elément vide
    \item[{Note}]
  \par
Par convention, l'élément \hyperref[TEI.lb]{<lb>} apparaît à l’endroit du texte où commence une nouvelle ligne. L'attribut {\itshape n}, s’il est utilisé, donne un nombre ou une autre valeur associée au texte entre ce point et l’élément suivant \hyperref[TEI.lb]{<lb>}, spécifiquement le numéro de la ligne dans la page, ou une autre unité de mesure appropriée. Cet élément est prévu pour être employé pour marquer un saut de ligne sur un manuscrit ou sur une page imprimée, à l’endroit où il se survient; on n’utilisera pas de balisage structurel comme une succession de vers (pour lequel l’élément \hyperref[TEI.l]{<l>} est disponible) sauf dans le cas où des blocs structurés ne peuvent pas être marqués autrement.\par
L'attribut {\itshape type} sera employé pour caractériser toute espèce de caractéristiques du saut de ligne, sauf la coupure des mots (indique par l'attribut {\itshape break}) ou la source concernée.
    \item[{Exemple}]
  Cet exemple montre les sauts de ligne dans des vers, qui apparaissent à différents endroits selon les éditions.\leavevmode\bgroup\exampleFont \begin{shaded}\noindent\mbox{}{<\textbf{l}>}Of Mans First Disobedience,{<\textbf{lb}\hspace*{6pt}{ed}="{1674}"/>} and{<\textbf{lb}\hspace*{6pt}{ed}="{1667}"/>} the Fruit{</\textbf{l}>}\mbox{}\newline 
{<\textbf{l}>}Of that Forbidden Tree, whose{<\textbf{lb}\hspace*{6pt}{ed}="{1667 1674}"/>} mortal tast{</\textbf{l}>}\mbox{}\newline 
{<\textbf{l}>}Brought Death into the World,{<\textbf{lb}\hspace*{6pt}{ed}="{1667}"/>} and all{<\textbf{lb}\hspace*{6pt}{ed}="{1674}"/>} our woe,{</\textbf{l}>}\end{shaded}\egroup 


    \item[{Exemple}]
  Cet exemple encode les sauts de ligne pour montre l'apparence visuelle d'une page titre. L'attribut {\itshape break} est utilisé pour montrer que le saut de ligne ne marque pas le début d'un nouveau mot.\leavevmode\bgroup\exampleFont \begin{shaded}\noindent\mbox{}{<\textbf{titlePart}\hspace*{6pt}{rend}="{italic}">}\mbox{}\newline 
\hspace*{6pt}{<\textbf{lb}/>}L'auteur susdict supplie les Lecteurs\mbox{}\newline 
{<\textbf{lb}/>}benevoles, soy reserver à rire au\mbox{}\newline 
 soi-{<\textbf{lb}\hspace*{6pt}{break}="{no}"/>}xante \& dixhuytiesme livre.\mbox{}\newline 
\mbox{}\newline 
{</\textbf{titlePart}>}\end{shaded}\egroup 


    \item[{Modèle de contenu}]
  \fbox{\ttfamily <content>\newline
</content>\newline
    } 
    \item[{Schéma Declaration}]
  \mbox{}\hfill\\[-10pt]\begin{Verbatim}[fontsize=\small]
element lb
{
   tei_att.global.attributes,
   tei_att.typed.attributes,
   tei_att.edition.attributes,
   tei_att.spanning.attributes,
   tei_att.breaking.attributes,
   empty
}
\end{Verbatim}

\end{reflist}  \index{lg=<lg>|oddindex}
\begin{reflist}
\item[]\begin{specHead}{TEI.lg}{<lg> }(groupe de vers) contient un groupe de vers fonctionnant comme une unité formelle, par exemple une strophe, un refrain, un paragraphe en vers, etc. [\xref{http://www.tei-c.org/release/doc/tei-p5-doc/en/html/CO.html\#COVE}{3.12.1. Core Tags for Verse} \xref{http://www.tei-c.org/release/doc/tei-p5-doc/en/html/CO.html\#CODV}{3.12. Passages of Verse or Drama} \xref{http://www.tei-c.org/release/doc/tei-p5-doc/en/html/DR.html\#DRPAL}{7.2.5. Speech Contents}]\end{specHead} 
    \item[{Module}]
  core
    \item[{Attributs}]
  Attributs \hyperref[TEI.att.global]{att.global} (\textit{@xml:id}, \textit{@n}, \textit{@xml:lang}, \textit{@xml:base}, \textit{@xml:space})  (\hyperref[TEI.att.global.rendition]{att.global.rendition} (\textit{@rend}, \textit{@style}, \textit{@rendition})) (\hyperref[TEI.att.global.linking]{att.global.linking} (\textit{@corresp}, \textit{@synch}, \textit{@sameAs}, \textit{@copyOf}, \textit{@next}, \textit{@prev}, \textit{@exclude}, \textit{@select})) (\hyperref[TEI.att.global.analytic]{att.global.analytic} (\textit{@ana})) (\hyperref[TEI.att.global.facs]{att.global.facs} (\textit{@facs})) (\hyperref[TEI.att.global.change]{att.global.change} (\textit{@change})) (\hyperref[TEI.att.global.responsibility]{att.global.responsibility} (\textit{@cert}, \textit{@resp})) (\hyperref[TEI.att.global.source]{att.global.source} (\textit{@source})) \hyperref[TEI.att.divLike]{att.divLike} (\textit{@org}, \textit{@sample})  (\hyperref[TEI.att.fragmentable]{att.fragmentable} (\textit{@part})) \hyperref[TEI.att.typed]{att.typed} (\textit{@type}, \textit{@subtype}) \hyperref[TEI.att.declaring]{att.declaring} (\textit{@decls}) 
    \item[{Membre du}]
  \hyperref[TEI.macro.paraContent]{macro.paraContent} \hyperref[TEI.model.divPart]{model.divPart} 
    \item[{Contenu dans}]
  
    \item[core: ]
   \hyperref[TEI.add]{add} \hyperref[TEI.corr]{corr} \hyperref[TEI.del]{del} \hyperref[TEI.emph]{emph} \hyperref[TEI.head]{head} \hyperref[TEI.hi]{hi} \hyperref[TEI.item]{item} \hyperref[TEI.lg]{lg} \hyperref[TEI.note]{note} \hyperref[TEI.orig]{orig} \hyperref[TEI.p]{p} \hyperref[TEI.q]{q} \hyperref[TEI.quote]{quote} \hyperref[TEI.ref]{ref} \hyperref[TEI.reg]{reg} \hyperref[TEI.said]{said} \hyperref[TEI.sic]{sic} \hyperref[TEI.sp]{sp} \hyperref[TEI.stage]{stage} \hyperref[TEI.title]{title} \hyperref[TEI.unclear]{unclear}\par 
    \item[figures: ]
   \hyperref[TEI.cell]{cell} \hyperref[TEI.figure]{figure}\par 
    \item[header: ]
   \hyperref[TEI.change]{change} \hyperref[TEI.licence]{licence}\par 
    \item[linking: ]
   \hyperref[TEI.ab]{ab} \hyperref[TEI.seg]{seg}\par 
    \item[msdescription: ]
   \hyperref[TEI.accMat]{accMat} \hyperref[TEI.acquisition]{acquisition} \hyperref[TEI.additions]{additions} \hyperref[TEI.collation]{collation} \hyperref[TEI.condition]{condition} \hyperref[TEI.custEvent]{custEvent} \hyperref[TEI.decoNote]{decoNote} \hyperref[TEI.filiation]{filiation} \hyperref[TEI.foliation]{foliation} \hyperref[TEI.layout]{layout} \hyperref[TEI.musicNotation]{musicNotation} \hyperref[TEI.origin]{origin} \hyperref[TEI.provenance]{provenance} \hyperref[TEI.signatures]{signatures} \hyperref[TEI.source]{source} \hyperref[TEI.summary]{summary} \hyperref[TEI.support]{support} \hyperref[TEI.surrogates]{surrogates} \hyperref[TEI.typeNote]{typeNote}\par 
    \item[textstructure: ]
   \hyperref[TEI.body]{body} \hyperref[TEI.div]{div} \hyperref[TEI.docEdition]{docEdition} \hyperref[TEI.titlePart]{titlePart}\par 
    \item[transcr: ]
   \hyperref[TEI.damage]{damage} \hyperref[TEI.metamark]{metamark} \hyperref[TEI.mod]{mod} \hyperref[TEI.restore]{restore} \hyperref[TEI.retrace]{retrace} \hyperref[TEI.secl]{secl} \hyperref[TEI.supplied]{supplied} \hyperref[TEI.surplus]{surplus}
    \item[{Peut contenir}]
  
    \item[analysis: ]
   \hyperref[TEI.interp]{interp} \hyperref[TEI.interpGrp]{interpGrp} \hyperref[TEI.span]{span} \hyperref[TEI.spanGrp]{spanGrp}\par 
    \item[core: ]
   \hyperref[TEI.cb]{cb} \hyperref[TEI.desc]{desc} \hyperref[TEI.gap]{gap} \hyperref[TEI.gb]{gb} \hyperref[TEI.head]{head} \hyperref[TEI.index]{index} \hyperref[TEI.l]{l} \hyperref[TEI.label]{label} \hyperref[TEI.lb]{lb} \hyperref[TEI.lg]{lg} \hyperref[TEI.meeting]{meeting} \hyperref[TEI.milestone]{milestone} \hyperref[TEI.note]{note} \hyperref[TEI.pb]{pb} \hyperref[TEI.stage]{stage}\par 
    \item[figures: ]
   \hyperref[TEI.figure]{figure} \hyperref[TEI.notatedMusic]{notatedMusic}\par 
    \item[iso-fs: ]
   \hyperref[TEI.fLib]{fLib} \hyperref[TEI.fs]{fs} \hyperref[TEI.fvLib]{fvLib}\par 
    \item[linking: ]
   \hyperref[TEI.alt]{alt} \hyperref[TEI.altGrp]{altGrp} \hyperref[TEI.anchor]{anchor} \hyperref[TEI.join]{join} \hyperref[TEI.joinGrp]{joinGrp} \hyperref[TEI.link]{link} \hyperref[TEI.linkGrp]{linkGrp} \hyperref[TEI.timeline]{timeline}\par 
    \item[msdescription: ]
   \hyperref[TEI.source]{source}\par 
    \item[textstructure: ]
   \hyperref[TEI.docAuthor]{docAuthor} \hyperref[TEI.docDate]{docDate}\par 
    \item[transcr: ]
   \hyperref[TEI.addSpan]{addSpan} \hyperref[TEI.damageSpan]{damageSpan} \hyperref[TEI.delSpan]{delSpan} \hyperref[TEI.fw]{fw} \hyperref[TEI.listTranspose]{listTranspose} \hyperref[TEI.metamark]{metamark} \hyperref[TEI.space]{space} \hyperref[TEI.substJoin]{substJoin}
    \item[{Note}]
  \par
ne contient que des vers ou des groupes de vers enchâssés, éventuellement précédés d'un titre.
    \item[{Exemple}]
  \leavevmode\bgroup\exampleFont \begin{shaded}\noindent\mbox{}{<\textbf{div}\hspace*{6pt}{type}="{sonnet}">}\mbox{}\newline 
\hspace*{6pt}{<\textbf{lg}\hspace*{6pt}{type}="{quatrain}">}\mbox{}\newline 
\hspace*{6pt}\hspace*{6pt}{<\textbf{l}>}Les amoureux fervents et les savants austères{</\textbf{l}>}\mbox{}\newline 
\hspace*{6pt}\hspace*{6pt}{<\textbf{l}>}Aiment également, dans leur mûre saison,{</\textbf{l}>}\mbox{}\newline 
\hspace*{6pt}\hspace*{6pt}{<\textbf{l}>}Les chats puissants et doux, orgueil de la maison,{</\textbf{l}>}\mbox{}\newline 
\hspace*{6pt}\hspace*{6pt}{<\textbf{l}>}Qui comme eux sont frileux et comme eux sédentaires.{</\textbf{l}>}\mbox{}\newline 
\hspace*{6pt}{</\textbf{lg}>}\mbox{}\newline 
\hspace*{6pt}{<\textbf{lg}\hspace*{6pt}{type}="{quatrain}">}\mbox{}\newline 
\hspace*{6pt}\hspace*{6pt}{<\textbf{l}>}Amis de la science et de la volupté{</\textbf{l}>}\mbox{}\newline 
\hspace*{6pt}\hspace*{6pt}{<\textbf{l}>}Ils cherchent le silence et l'horreur des ténèbres ;{</\textbf{l}>}\mbox{}\newline 
\hspace*{6pt}\hspace*{6pt}{<\textbf{l}>}L'Erèbe les eût pris pour ses coursiers funèbres,{</\textbf{l}>}\mbox{}\newline 
\hspace*{6pt}\hspace*{6pt}{<\textbf{l}>}S'ils pouvaient au servage incliner leur fierté.{</\textbf{l}>}\mbox{}\newline 
\hspace*{6pt}{</\textbf{lg}>}\mbox{}\newline 
\hspace*{6pt}{<\textbf{lg}\hspace*{6pt}{type}="{tercet}">}\mbox{}\newline 
\hspace*{6pt}\hspace*{6pt}{<\textbf{l}>}Ils prennent en songeant les nobles attitudes{</\textbf{l}>}\mbox{}\newline 
\hspace*{6pt}\hspace*{6pt}{<\textbf{l}>}Des grands sphinx allongés au fond des solitudes,{</\textbf{l}>}\mbox{}\newline 
\hspace*{6pt}\hspace*{6pt}{<\textbf{l}>}Qui semblent s'endormir dans un rêve sans fin ;{</\textbf{l}>}\mbox{}\newline 
\hspace*{6pt}{</\textbf{lg}>}\mbox{}\newline 
\hspace*{6pt}{<\textbf{lg}\hspace*{6pt}{type}="{tercet}">}\mbox{}\newline 
\hspace*{6pt}\hspace*{6pt}{<\textbf{l}>}Leurs reins féconds sont pleins d'étincelles magiques,{</\textbf{l}>}\mbox{}\newline 
\hspace*{6pt}\hspace*{6pt}{<\textbf{l}>}Et des parcelles d'or, ainsi qu'un sable fin,{</\textbf{l}>}\mbox{}\newline 
\hspace*{6pt}\hspace*{6pt}{<\textbf{l}>}Etoilent vaguement leurs prunelles mystiques.{</\textbf{l}>}\mbox{}\newline 
\hspace*{6pt}{</\textbf{lg}>}\mbox{}\newline 
{</\textbf{div}>}\end{shaded}\egroup 


    \item[{Schematron}]
   <sch:assert test="count(descendant::tei:lg|descendant::tei:l|descendant::tei:gap) >   0">An lg element  must contain at least one child l, lg or gap element.</sch:assert>
    \item[{Schematron}]
   <s:report test="ancestor::tei:l[not(.//tei:note//tei:lg[. = current()])]"> Abstract model violation: Lines may not contain line groups. </s:report>
    \item[{Modèle de contenu}]
  \mbox{}\hfill\\[-10pt]\begin{Verbatim}[fontsize=\small]
<content>
 <sequence maxOccurs="1" minOccurs="1">
  <alternate maxOccurs="unbounded"
   minOccurs="0">
   <classRef key="model.divTop"/>
   <classRef key="model.global"/>
  </alternate>
  <alternate maxOccurs="1" minOccurs="1">
   <classRef key="model.lLike"/>
   <classRef key="model.stageLike"/>
   <classRef key="model.labelLike"/>
   <elementRef key="lg"/>
  </alternate>
  <alternate maxOccurs="unbounded"
   minOccurs="0">
   <classRef key="model.lLike"/>
   <classRef key="model.stageLike"/>
   <classRef key="model.labelLike"/>
   <classRef key="model.global"/>
   <elementRef key="lg"/>
  </alternate>
  <sequence maxOccurs="unbounded"
   minOccurs="0">
   <classRef key="model.divBottom"/>
   <classRef key="model.global"
    maxOccurs="unbounded" minOccurs="0"/>
  </sequence>
 </sequence>
</content>
    
\end{Verbatim}

    \item[{Schéma Declaration}]
  \mbox{}\hfill\\[-10pt]\begin{Verbatim}[fontsize=\small]
element lg
{
   tei_att.global.attributes,
   tei_att.divLike.attributes,
   tei_att.typed.attributes,
   tei_att.declaring.attributes,
   (
      ( tei_model.divTop | tei_model.global )*,
      ( tei_model.lLike | tei_model.stageLike | tei_model.labelLike | tei_lg ),
      (
         tei_model.lLike       | tei_model.stageLike       | tei_model.labelLike       | tei_model.global       | tei_lg      )*,
      ( tei_model.divBottom, tei_model.global* )*
   )
}
\end{Verbatim}

\end{reflist}  \index{licence=<licence>|oddindex}
\begin{reflist}
\item[]\begin{specHead}{TEI.licence}{<licence> }contient des informations légales applicables au texte, notamment le contrat de licence définissant les droits d'utilisation. [\xref{http://www.tei-c.org/release/doc/tei-p5-doc/en/html/HD.html\#HD24}{2.2.4. Publication, Distribution, Licensing, etc.}]\end{specHead} 
    \item[{Module}]
  header
    \item[{Attributs}]
  Attributs \hyperref[TEI.att.global]{att.global} (\textit{@xml:id}, \textit{@n}, \textit{@xml:lang}, \textit{@xml:base}, \textit{@xml:space})  (\hyperref[TEI.att.global.rendition]{att.global.rendition} (\textit{@rend}, \textit{@style}, \textit{@rendition})) (\hyperref[TEI.att.global.linking]{att.global.linking} (\textit{@corresp}, \textit{@synch}, \textit{@sameAs}, \textit{@copyOf}, \textit{@next}, \textit{@prev}, \textit{@exclude}, \textit{@select})) (\hyperref[TEI.att.global.analytic]{att.global.analytic} (\textit{@ana})) (\hyperref[TEI.att.global.facs]{att.global.facs} (\textit{@facs})) (\hyperref[TEI.att.global.change]{att.global.change} (\textit{@change})) (\hyperref[TEI.att.global.responsibility]{att.global.responsibility} (\textit{@cert}, \textit{@resp})) (\hyperref[TEI.att.global.source]{att.global.source} (\textit{@source})) \hyperref[TEI.att.pointing]{att.pointing} (\textit{@targetLang}, \textit{@target}, \textit{@evaluate}) \hyperref[TEI.att.datable]{att.datable} (\textit{@calendar}, \textit{@period})  (\hyperref[TEI.att.datable.w3c]{att.datable.w3c} (\textit{@when}, \textit{@notBefore}, \textit{@notAfter}, \textit{@from}, \textit{@to})) (\hyperref[TEI.att.datable.iso]{att.datable.iso} (\textit{@when-iso}, \textit{@notBefore-iso}, \textit{@notAfter-iso}, \textit{@from-iso}, \textit{@to-iso})) (\hyperref[TEI.att.datable.custom]{att.datable.custom} (\textit{@when-custom}, \textit{@notBefore-custom}, \textit{@notAfter-custom}, \textit{@from-custom}, \textit{@to-custom}, \textit{@datingPoint}, \textit{@datingMethod}))
    \item[{Membre du}]
  \hyperref[TEI.model.availabilityPart]{model.availabilityPart}
    \item[{Contenu dans}]
  
    \item[header: ]
   \hyperref[TEI.availability]{availability}
    \item[{Peut contenir}]
  
    \item[analysis: ]
   \hyperref[TEI.c]{c} \hyperref[TEI.cl]{cl} \hyperref[TEI.interp]{interp} \hyperref[TEI.interpGrp]{interpGrp} \hyperref[TEI.m]{m} \hyperref[TEI.pc]{pc} \hyperref[TEI.phr]{phr} \hyperref[TEI.s]{s} \hyperref[TEI.span]{span} \hyperref[TEI.spanGrp]{spanGrp} \hyperref[TEI.w]{w}\par 
    \item[core: ]
   \hyperref[TEI.abbr]{abbr} \hyperref[TEI.add]{add} \hyperref[TEI.address]{address} \hyperref[TEI.bibl]{bibl} \hyperref[TEI.biblStruct]{biblStruct} \hyperref[TEI.binaryObject]{binaryObject} \hyperref[TEI.cb]{cb} \hyperref[TEI.choice]{choice} \hyperref[TEI.cit]{cit} \hyperref[TEI.corr]{corr} \hyperref[TEI.date]{date} \hyperref[TEI.del]{del} \hyperref[TEI.desc]{desc} \hyperref[TEI.distinct]{distinct} \hyperref[TEI.email]{email} \hyperref[TEI.emph]{emph} \hyperref[TEI.expan]{expan} \hyperref[TEI.foreign]{foreign} \hyperref[TEI.gap]{gap} \hyperref[TEI.gb]{gb} \hyperref[TEI.gloss]{gloss} \hyperref[TEI.graphic]{graphic} \hyperref[TEI.hi]{hi} \hyperref[TEI.index]{index} \hyperref[TEI.l]{l} \hyperref[TEI.label]{label} \hyperref[TEI.lb]{lb} \hyperref[TEI.lg]{lg} \hyperref[TEI.list]{list} \hyperref[TEI.listBibl]{listBibl} \hyperref[TEI.measure]{measure} \hyperref[TEI.measureGrp]{measureGrp} \hyperref[TEI.media]{media} \hyperref[TEI.mentioned]{mentioned} \hyperref[TEI.milestone]{milestone} \hyperref[TEI.name]{name} \hyperref[TEI.note]{note} \hyperref[TEI.num]{num} \hyperref[TEI.orig]{orig} \hyperref[TEI.p]{p} \hyperref[TEI.pb]{pb} \hyperref[TEI.ptr]{ptr} \hyperref[TEI.q]{q} \hyperref[TEI.quote]{quote} \hyperref[TEI.ref]{ref} \hyperref[TEI.reg]{reg} \hyperref[TEI.rs]{rs} \hyperref[TEI.said]{said} \hyperref[TEI.sic]{sic} \hyperref[TEI.soCalled]{soCalled} \hyperref[TEI.sp]{sp} \hyperref[TEI.stage]{stage} \hyperref[TEI.term]{term} \hyperref[TEI.time]{time} \hyperref[TEI.title]{title} \hyperref[TEI.unclear]{unclear}\par 
    \item[derived-module-tei.istex: ]
   \hyperref[TEI.math]{math} \hyperref[TEI.mrow]{mrow}\par 
    \item[figures: ]
   \hyperref[TEI.figure]{figure} \hyperref[TEI.formula]{formula} \hyperref[TEI.notatedMusic]{notatedMusic} \hyperref[TEI.table]{table}\par 
    \item[header: ]
   \hyperref[TEI.biblFull]{biblFull} \hyperref[TEI.idno]{idno}\par 
    \item[iso-fs: ]
   \hyperref[TEI.fLib]{fLib} \hyperref[TEI.fs]{fs} \hyperref[TEI.fvLib]{fvLib}\par 
    \item[linking: ]
   \hyperref[TEI.ab]{ab} \hyperref[TEI.alt]{alt} \hyperref[TEI.altGrp]{altGrp} \hyperref[TEI.anchor]{anchor} \hyperref[TEI.join]{join} \hyperref[TEI.joinGrp]{joinGrp} \hyperref[TEI.link]{link} \hyperref[TEI.linkGrp]{linkGrp} \hyperref[TEI.seg]{seg} \hyperref[TEI.timeline]{timeline}\par 
    \item[msdescription: ]
   \hyperref[TEI.catchwords]{catchwords} \hyperref[TEI.depth]{depth} \hyperref[TEI.dim]{dim} \hyperref[TEI.dimensions]{dimensions} \hyperref[TEI.height]{height} \hyperref[TEI.heraldry]{heraldry} \hyperref[TEI.locus]{locus} \hyperref[TEI.locusGrp]{locusGrp} \hyperref[TEI.material]{material} \hyperref[TEI.msDesc]{msDesc} \hyperref[TEI.objectType]{objectType} \hyperref[TEI.origDate]{origDate} \hyperref[TEI.origPlace]{origPlace} \hyperref[TEI.secFol]{secFol} \hyperref[TEI.signatures]{signatures} \hyperref[TEI.source]{source} \hyperref[TEI.stamp]{stamp} \hyperref[TEI.watermark]{watermark} \hyperref[TEI.width]{width}\par 
    \item[namesdates: ]
   \hyperref[TEI.addName]{addName} \hyperref[TEI.affiliation]{affiliation} \hyperref[TEI.country]{country} \hyperref[TEI.forename]{forename} \hyperref[TEI.genName]{genName} \hyperref[TEI.geogName]{geogName} \hyperref[TEI.listOrg]{listOrg} \hyperref[TEI.listPlace]{listPlace} \hyperref[TEI.location]{location} \hyperref[TEI.nameLink]{nameLink} \hyperref[TEI.orgName]{orgName} \hyperref[TEI.persName]{persName} \hyperref[TEI.placeName]{placeName} \hyperref[TEI.region]{region} \hyperref[TEI.roleName]{roleName} \hyperref[TEI.settlement]{settlement} \hyperref[TEI.state]{state} \hyperref[TEI.surname]{surname}\par 
    \item[spoken: ]
   \hyperref[TEI.annotationBlock]{annotationBlock}\par 
    \item[textstructure: ]
   \hyperref[TEI.floatingText]{floatingText}\par 
    \item[transcr: ]
   \hyperref[TEI.addSpan]{addSpan} \hyperref[TEI.am]{am} \hyperref[TEI.damage]{damage} \hyperref[TEI.damageSpan]{damageSpan} \hyperref[TEI.delSpan]{delSpan} \hyperref[TEI.ex]{ex} \hyperref[TEI.fw]{fw} \hyperref[TEI.handShift]{handShift} \hyperref[TEI.listTranspose]{listTranspose} \hyperref[TEI.metamark]{metamark} \hyperref[TEI.mod]{mod} \hyperref[TEI.redo]{redo} \hyperref[TEI.restore]{restore} \hyperref[TEI.retrace]{retrace} \hyperref[TEI.secl]{secl} \hyperref[TEI.space]{space} \hyperref[TEI.subst]{subst} \hyperref[TEI.substJoin]{substJoin} \hyperref[TEI.supplied]{supplied} \hyperref[TEI.surplus]{surplus} \hyperref[TEI.undo]{undo}\par des données textuelles
    \item[{Note}]
  \par
A \hyperref[TEI.licence]{<licence>} element should be supplied for each licence agreement applicable to the text in question. The {\itshape target} attribute may be used to reference a full version of the licence. The {\itshape when}, {\itshape notBefore}, {\itshape notAfter}, {\itshape from} or {\itshape to} attributes may be used in combination to indicate the date or dates of applicability of the licence.
    \item[{Exemple}]
  \leavevmode\bgroup\exampleFont \begin{shaded}\noindent\mbox{}{<\textbf{licence}\hspace*{6pt}{target}="{http://creativecommons.org/licenses/by/3.0/deed.fr}">} Creative Commons Attribution 3.0 non transposé (CC BY 3.0)\mbox{}\newline 
{</\textbf{licence}>}\end{shaded}\egroup 


    \item[{Exemple}]
  \leavevmode\bgroup\exampleFont \begin{shaded}\noindent\mbox{}{<\textbf{licence}\hspace*{6pt}{target}="{http://creativecommons.org/licenses/by-sa/2.0/}">} Ce document\mbox{}\newline 
 est publié librement sur le web à destination de la communauté scientifique\mbox{}\newline 
 dans le cadre de la licence Creative Commons « Paternité-Pas d’Utilisation\mbox{}\newline 
 Commerciale-Partage des Conditions Initiales à l’Identique 2.0 France ».\mbox{}\newline 
{</\textbf{licence}>}\end{shaded}\egroup 


    \item[{Modèle de contenu}]
  \mbox{}\hfill\\[-10pt]\begin{Verbatim}[fontsize=\small]
<content>
 <macroRef key="macro.specialPara"/>
</content>
    
\end{Verbatim}

    \item[{Schéma Declaration}]
  \mbox{}\hfill\\[-10pt]\begin{Verbatim}[fontsize=\small]
element licence
{
   tei_att.global.attributes,
   tei_att.pointing.attributes,
   tei_att.datable.attributes,
   tei_macro.specialPara}
\end{Verbatim}

\end{reflist}  \index{line=<line>|oddindex}
\begin{reflist}
\item[]\begin{specHead}{TEI.line}{<line> }contains the transcription of a topographic line in the source document [\xref{http://www.tei-c.org/release/doc/tei-p5-doc/en/html/PH.html\#PHZLAB}{11.2.2. Embedded Transcription}]\end{specHead} 
    \item[{Module}]
  transcr
    \item[{Attributs}]
  Attributs \hyperref[TEI.att.typed]{att.typed} (\textit{@type}, \textit{@subtype}) \hyperref[TEI.att.global]{att.global} (\textit{@xml:id}, \textit{@n}, \textit{@xml:lang}, \textit{@xml:base}, \textit{@xml:space})  (\hyperref[TEI.att.global.rendition]{att.global.rendition} (\textit{@rend}, \textit{@style}, \textit{@rendition})) (\hyperref[TEI.att.global.linking]{att.global.linking} (\textit{@corresp}, \textit{@synch}, \textit{@sameAs}, \textit{@copyOf}, \textit{@next}, \textit{@prev}, \textit{@exclude}, \textit{@select})) (\hyperref[TEI.att.global.analytic]{att.global.analytic} (\textit{@ana})) (\hyperref[TEI.att.global.facs]{att.global.facs} (\textit{@facs})) (\hyperref[TEI.att.global.change]{att.global.change} (\textit{@change})) (\hyperref[TEI.att.global.responsibility]{att.global.responsibility} (\textit{@cert}, \textit{@resp})) (\hyperref[TEI.att.global.source]{att.global.source} (\textit{@source})) \hyperref[TEI.att.coordinated]{att.coordinated} (\textit{@start}, \textit{@ulx}, \textit{@uly}, \textit{@lrx}, \textit{@lry}, \textit{@points}) \hyperref[TEI.att.written]{att.written} (\textit{@hand}) 
    \item[{Membre du}]
  \hyperref[TEI.model.linePart]{model.linePart} 
    \item[{Contenu dans}]
  
    \item[transcr: ]
   \hyperref[TEI.line]{line} \hyperref[TEI.surface]{surface} \hyperref[TEI.zone]{zone}
    \item[{Peut contenir}]
  
    \item[analysis: ]
   \hyperref[TEI.c]{c} \hyperref[TEI.interp]{interp} \hyperref[TEI.interpGrp]{interpGrp} \hyperref[TEI.pc]{pc} \hyperref[TEI.span]{span} \hyperref[TEI.spanGrp]{spanGrp} \hyperref[TEI.w]{w}\par 
    \item[core: ]
   \hyperref[TEI.add]{add} \hyperref[TEI.cb]{cb} \hyperref[TEI.choice]{choice} \hyperref[TEI.del]{del} \hyperref[TEI.gap]{gap} \hyperref[TEI.gb]{gb} \hyperref[TEI.hi]{hi} \hyperref[TEI.index]{index} \hyperref[TEI.lb]{lb} \hyperref[TEI.milestone]{milestone} \hyperref[TEI.note]{note} \hyperref[TEI.pb]{pb} \hyperref[TEI.unclear]{unclear}\par 
    \item[figures: ]
   \hyperref[TEI.figure]{figure} \hyperref[TEI.notatedMusic]{notatedMusic}\par 
    \item[iso-fs: ]
   \hyperref[TEI.fLib]{fLib} \hyperref[TEI.fs]{fs} \hyperref[TEI.fvLib]{fvLib}\par 
    \item[linking: ]
   \hyperref[TEI.alt]{alt} \hyperref[TEI.altGrp]{altGrp} \hyperref[TEI.anchor]{anchor} \hyperref[TEI.join]{join} \hyperref[TEI.joinGrp]{joinGrp} \hyperref[TEI.link]{link} \hyperref[TEI.linkGrp]{linkGrp} \hyperref[TEI.seg]{seg} \hyperref[TEI.timeline]{timeline}\par 
    \item[msdescription: ]
   \hyperref[TEI.source]{source}\par 
    \item[transcr: ]
   \hyperref[TEI.addSpan]{addSpan} \hyperref[TEI.damage]{damage} \hyperref[TEI.damageSpan]{damageSpan} \hyperref[TEI.delSpan]{delSpan} \hyperref[TEI.fw]{fw} \hyperref[TEI.handShift]{handShift} \hyperref[TEI.line]{line} \hyperref[TEI.listTranspose]{listTranspose} \hyperref[TEI.metamark]{metamark} \hyperref[TEI.mod]{mod} \hyperref[TEI.redo]{redo} \hyperref[TEI.restore]{restore} \hyperref[TEI.retrace]{retrace} \hyperref[TEI.space]{space} \hyperref[TEI.substJoin]{substJoin} \hyperref[TEI.undo]{undo} \hyperref[TEI.zone]{zone}\par des données textuelles
    \item[{Note}]
  \par
This element should be used only to mark up writing which is topographically organized as a series of lines, horizontal or vertical. It should not be used to mark lines of verse (for which use \hyperref[TEI.l]{<l>}) nor to mark linebreaks within text which has been encoded using structural elements such as \hyperref[TEI.p]{<p>} (for which use \hyperref[TEI.lb]{<lb>}).
    \item[{Exemple}]
  \leavevmode\bgroup\exampleFont \begin{shaded}\noindent\mbox{}{<\textbf{surface}>}\mbox{}\newline 
\hspace*{6pt}{<\textbf{zone}>}\mbox{}\newline 
\hspace*{6pt}\hspace*{6pt}{<\textbf{line}>}Poem{</\textbf{line}>}\mbox{}\newline 
\hspace*{6pt}\hspace*{6pt}{<\textbf{line}>}As in Visions of — at{</\textbf{line}>}\mbox{}\newline 
\hspace*{6pt}\hspace*{6pt}{<\textbf{line}>}night —{</\textbf{line}>}\mbox{}\newline 
\hspace*{6pt}\hspace*{6pt}{<\textbf{line}>}All sorts of fancies running through{</\textbf{line}>}\mbox{}\newline 
\hspace*{6pt}\hspace*{6pt}{<\textbf{line}>}the head{</\textbf{line}>}\mbox{}\newline 
\hspace*{6pt}{</\textbf{zone}>}\mbox{}\newline 
{</\textbf{surface}>}\end{shaded}\egroup 


    \item[{Exemple}]
  \leavevmode\bgroup\exampleFont \begin{shaded}\noindent\mbox{}{<\textbf{surface}>}\mbox{}\newline 
\hspace*{6pt}{<\textbf{zone}>}\mbox{}\newline 
\hspace*{6pt}\hspace*{6pt}{<\textbf{line}>}Hope you enjoyed{</\textbf{line}>}\mbox{}\newline 
\hspace*{6pt}\hspace*{6pt}{<\textbf{line}>}Wales, as they\mbox{}\newline 
\hspace*{6pt}\hspace*{6pt}\hspace*{6pt}\hspace*{6pt} said{</\textbf{line}>}\mbox{}\newline 
\hspace*{6pt}\hspace*{6pt}{<\textbf{line}>}to Mrs FitzHerbert{</\textbf{line}>}\mbox{}\newline 
\hspace*{6pt}\hspace*{6pt}{<\textbf{line}>}Mama{</\textbf{line}>}\mbox{}\newline 
\hspace*{6pt}{</\textbf{zone}>}\mbox{}\newline 
\hspace*{6pt}{<\textbf{zone}>}\mbox{}\newline 
\hspace*{6pt}\hspace*{6pt}{<\textbf{line}>}Printed in England{</\textbf{line}>}\mbox{}\newline 
\hspace*{6pt}{</\textbf{zone}>}\mbox{}\newline 
{</\textbf{surface}>}\end{shaded}\egroup 


    \item[{Modèle de contenu}]
  \mbox{}\hfill\\[-10pt]\begin{Verbatim}[fontsize=\small]
<content>
 <alternate maxOccurs="unbounded"
  minOccurs="0">
  <textNode/>
  <classRef key="model.global"/>
  <classRef key="model.gLike"/>
  <classRef key="model.linePart"/>
 </alternate>
</content>
    
\end{Verbatim}

    \item[{Schéma Declaration}]
  \mbox{}\hfill\\[-10pt]\begin{Verbatim}[fontsize=\small]
element line
{
   tei_att.typed.attributes,
   tei_att.global.attributes,
   tei_att.coordinated.attributes,
   tei_att.written.attributes,
   ( text | tei_model.global | tei_model.gLike | tei_model.linePart )*
}
\end{Verbatim}

\end{reflist}  \index{link=<link>|oddindex}
\begin{reflist}
\item[]\begin{specHead}{TEI.link}{<link> }(lien) définit une association ou un lien hypertextuel entre des éléments ou des passages, lien dont le type ne peut être spécifié précisément par d'autres éléments. [\xref{http://www.tei-c.org/release/doc/tei-p5-doc/en/html/SA.html\#SAPT}{16.1. Links}]\end{specHead} 
    \item[{Module}]
  linking
    \item[{Attributs}]
  Attributs \hyperref[TEI.att.global]{att.global} (\textit{@xml:id}, \textit{@n}, \textit{@xml:lang}, \textit{@xml:base}, \textit{@xml:space})  (\hyperref[TEI.att.global.rendition]{att.global.rendition} (\textit{@rend}, \textit{@style}, \textit{@rendition})) (\hyperref[TEI.att.global.linking]{att.global.linking} (\textit{@corresp}, \textit{@synch}, \textit{@sameAs}, \textit{@copyOf}, \textit{@next}, \textit{@prev}, \textit{@exclude}, \textit{@select})) (\hyperref[TEI.att.global.analytic]{att.global.analytic} (\textit{@ana})) (\hyperref[TEI.att.global.facs]{att.global.facs} (\textit{@facs})) (\hyperref[TEI.att.global.change]{att.global.change} (\textit{@change})) (\hyperref[TEI.att.global.responsibility]{att.global.responsibility} (\textit{@cert}, \textit{@resp})) (\hyperref[TEI.att.global.source]{att.global.source} (\textit{@source})) \hyperref[TEI.att.pointing]{att.pointing} (\textit{@targetLang}, \textit{@target}, \textit{@evaluate}) \hyperref[TEI.att.typed]{att.typed} (\textit{@type}, \textit{@subtype}) 
    \item[{Membre du}]
  \hyperref[TEI.model.global.meta]{model.global.meta} 
    \item[{Contenu dans}]
  
    \item[analysis: ]
   \hyperref[TEI.cl]{cl} \hyperref[TEI.m]{m} \hyperref[TEI.phr]{phr} \hyperref[TEI.s]{s} \hyperref[TEI.span]{span} \hyperref[TEI.w]{w}\par 
    \item[core: ]
   \hyperref[TEI.abbr]{abbr} \hyperref[TEI.add]{add} \hyperref[TEI.addrLine]{addrLine} \hyperref[TEI.address]{address} \hyperref[TEI.author]{author} \hyperref[TEI.bibl]{bibl} \hyperref[TEI.biblScope]{biblScope} \hyperref[TEI.cit]{cit} \hyperref[TEI.citedRange]{citedRange} \hyperref[TEI.corr]{corr} \hyperref[TEI.date]{date} \hyperref[TEI.del]{del} \hyperref[TEI.distinct]{distinct} \hyperref[TEI.editor]{editor} \hyperref[TEI.email]{email} \hyperref[TEI.emph]{emph} \hyperref[TEI.expan]{expan} \hyperref[TEI.foreign]{foreign} \hyperref[TEI.gloss]{gloss} \hyperref[TEI.head]{head} \hyperref[TEI.headItem]{headItem} \hyperref[TEI.headLabel]{headLabel} \hyperref[TEI.hi]{hi} \hyperref[TEI.imprint]{imprint} \hyperref[TEI.item]{item} \hyperref[TEI.l]{l} \hyperref[TEI.label]{label} \hyperref[TEI.lg]{lg} \hyperref[TEI.list]{list} \hyperref[TEI.measure]{measure} \hyperref[TEI.mentioned]{mentioned} \hyperref[TEI.name]{name} \hyperref[TEI.note]{note} \hyperref[TEI.num]{num} \hyperref[TEI.orig]{orig} \hyperref[TEI.p]{p} \hyperref[TEI.pubPlace]{pubPlace} \hyperref[TEI.publisher]{publisher} \hyperref[TEI.q]{q} \hyperref[TEI.quote]{quote} \hyperref[TEI.ref]{ref} \hyperref[TEI.reg]{reg} \hyperref[TEI.resp]{resp} \hyperref[TEI.rs]{rs} \hyperref[TEI.said]{said} \hyperref[TEI.series]{series} \hyperref[TEI.sic]{sic} \hyperref[TEI.soCalled]{soCalled} \hyperref[TEI.sp]{sp} \hyperref[TEI.speaker]{speaker} \hyperref[TEI.stage]{stage} \hyperref[TEI.street]{street} \hyperref[TEI.term]{term} \hyperref[TEI.textLang]{textLang} \hyperref[TEI.time]{time} \hyperref[TEI.title]{title} \hyperref[TEI.unclear]{unclear}\par 
    \item[figures: ]
   \hyperref[TEI.cell]{cell} \hyperref[TEI.figure]{figure} \hyperref[TEI.table]{table}\par 
    \item[header: ]
   \hyperref[TEI.authority]{authority} \hyperref[TEI.change]{change} \hyperref[TEI.classCode]{classCode} \hyperref[TEI.distributor]{distributor} \hyperref[TEI.edition]{edition} \hyperref[TEI.extent]{extent} \hyperref[TEI.funder]{funder} \hyperref[TEI.language]{language} \hyperref[TEI.licence]{licence}\par 
    \item[linking: ]
   \hyperref[TEI.ab]{ab} \hyperref[TEI.linkGrp]{linkGrp} \hyperref[TEI.seg]{seg}\par 
    \item[msdescription: ]
   \hyperref[TEI.accMat]{accMat} \hyperref[TEI.acquisition]{acquisition} \hyperref[TEI.additions]{additions} \hyperref[TEI.catchwords]{catchwords} \hyperref[TEI.collation]{collation} \hyperref[TEI.colophon]{colophon} \hyperref[TEI.condition]{condition} \hyperref[TEI.custEvent]{custEvent} \hyperref[TEI.decoNote]{decoNote} \hyperref[TEI.explicit]{explicit} \hyperref[TEI.filiation]{filiation} \hyperref[TEI.finalRubric]{finalRubric} \hyperref[TEI.foliation]{foliation} \hyperref[TEI.heraldry]{heraldry} \hyperref[TEI.incipit]{incipit} \hyperref[TEI.layout]{layout} \hyperref[TEI.material]{material} \hyperref[TEI.msItem]{msItem} \hyperref[TEI.musicNotation]{musicNotation} \hyperref[TEI.objectType]{objectType} \hyperref[TEI.origDate]{origDate} \hyperref[TEI.origPlace]{origPlace} \hyperref[TEI.origin]{origin} \hyperref[TEI.provenance]{provenance} \hyperref[TEI.rubric]{rubric} \hyperref[TEI.secFol]{secFol} \hyperref[TEI.signatures]{signatures} \hyperref[TEI.source]{source} \hyperref[TEI.stamp]{stamp} \hyperref[TEI.summary]{summary} \hyperref[TEI.support]{support} \hyperref[TEI.surrogates]{surrogates} \hyperref[TEI.typeNote]{typeNote} \hyperref[TEI.watermark]{watermark}\par 
    \item[namesdates: ]
   \hyperref[TEI.addName]{addName} \hyperref[TEI.affiliation]{affiliation} \hyperref[TEI.country]{country} \hyperref[TEI.event]{event} \hyperref[TEI.forename]{forename} \hyperref[TEI.genName]{genName} \hyperref[TEI.geogName]{geogName} \hyperref[TEI.nameLink]{nameLink} \hyperref[TEI.org]{org} \hyperref[TEI.orgName]{orgName} \hyperref[TEI.persName]{persName} \hyperref[TEI.person]{person} \hyperref[TEI.personGrp]{personGrp} \hyperref[TEI.persona]{persona} \hyperref[TEI.place]{place} \hyperref[TEI.placeName]{placeName} \hyperref[TEI.region]{region} \hyperref[TEI.roleName]{roleName} \hyperref[TEI.settlement]{settlement} \hyperref[TEI.surname]{surname}\par 
    \item[spoken: ]
   \hyperref[TEI.annotationBlock]{annotationBlock}\par 
    \item[standOff: ]
   \hyperref[TEI.listAnnotation]{listAnnotation}\par 
    \item[textstructure: ]
   \hyperref[TEI.back]{back} \hyperref[TEI.body]{body} \hyperref[TEI.div]{div} \hyperref[TEI.docAuthor]{docAuthor} \hyperref[TEI.docDate]{docDate} \hyperref[TEI.docEdition]{docEdition} \hyperref[TEI.docTitle]{docTitle} \hyperref[TEI.floatingText]{floatingText} \hyperref[TEI.front]{front} \hyperref[TEI.group]{group} \hyperref[TEI.text]{text} \hyperref[TEI.titlePage]{titlePage} \hyperref[TEI.titlePart]{titlePart}\par 
    \item[transcr: ]
   \hyperref[TEI.damage]{damage} \hyperref[TEI.fw]{fw} \hyperref[TEI.line]{line} \hyperref[TEI.metamark]{metamark} \hyperref[TEI.mod]{mod} \hyperref[TEI.restore]{restore} \hyperref[TEI.retrace]{retrace} \hyperref[TEI.secl]{secl} \hyperref[TEI.sourceDoc]{sourceDoc} \hyperref[TEI.supplied]{supplied} \hyperref[TEI.surface]{surface} \hyperref[TEI.surfaceGrp]{surfaceGrp} \hyperref[TEI.surplus]{surplus} \hyperref[TEI.zone]{zone}
    \item[{Peut contenir}]
  Elément vide
    \item[{Note}]
  \par
Cet élément n’est utilisé que pour encoder des associations ; il n’est pas préconisé pour d’autres éléments plus spécifiques. \par
L’emplacement de cet élément dans un document n'a aucune signification, à moins qu'il ne soit inclus dans un élément \hyperref[TEI.linkGrp]{<linkGrp>} ; dans ce cas il peut hériter de la valeur donnée à l’attribut {\itshape type} de l’élément \hyperref[TEI.linkGrp]{<linkGrp>} .
    \item[{Exemple}]
  \leavevmode\bgroup\exampleFont \begin{shaded}\noindent\mbox{}{<\textbf{s}\hspace*{6pt}{n}="{1}">}The state Supreme Court has refused to release {<\textbf{rs}\hspace*{6pt}{xml:id}="{R1}">}\mbox{}\newline 
\hspace*{6pt}\hspace*{6pt}{<\textbf{rs}\hspace*{6pt}{xml:id}="{R2}">}Rahway State Prison{</\textbf{rs}>} inmate{</\textbf{rs}>}\mbox{}\newline 
\hspace*{6pt}{<\textbf{rs}\hspace*{6pt}{xml:id}="{R3}">}James Scott{</\textbf{rs}>} on bail.{</\textbf{s}>}\mbox{}\newline 
{<\textbf{s}\hspace*{6pt}{n}="{2}">}\mbox{}\newline 
\hspace*{6pt}{<\textbf{rs}\hspace*{6pt}{xml:id}="{R4}">}The fighter{</\textbf{rs}>} is serving 30-40 years\mbox{}\newline 
 for a 1975 armed robbery conviction in {<\textbf{rs}\hspace*{6pt}{xml:id}="{R5}">}the penitentiary{</\textbf{rs}>}.\mbox{}\newline 
{</\textbf{s}>}\mbox{}\newline 
\textit{<!-- ... -->}\mbox{}\newline 
{<\textbf{linkGrp}\hspace*{6pt}{type}="{periphrasis}">}\mbox{}\newline 
\hspace*{6pt}{<\textbf{link}\hspace*{6pt}{target}="{\#R1 \#R3 \#R4}"/>}\mbox{}\newline 
\hspace*{6pt}{<\textbf{link}\hspace*{6pt}{target}="{\#R2 \#R5}"/>}\mbox{}\newline 
{</\textbf{linkGrp}>}\end{shaded}\egroup 


    \item[{Schematron}]
   <sch:assert test="contains(normalize-space(@target),' ')">You must supply at least two values for @target or on <sch:name/> </sch:assert>
    \item[{Modèle de contenu}]
  \fbox{\ttfamily <content>\newline
</content>\newline
    } 
    \item[{Schéma Declaration}]
  \mbox{}\hfill\\[-10pt]\begin{Verbatim}[fontsize=\small]
element link
{
   tei_att.global.attributes,
   tei_att.pointing.attributes,
   tei_att.typed.attributes,
   empty
}
\end{Verbatim}

\end{reflist}  \index{linkGrp=<linkGrp>|oddindex}
\begin{reflist}
\item[]\begin{specHead}{TEI.linkGrp}{<linkGrp> }(groupe de liens) définit un ensemble d'associations ou de liens hypertextuels. [\xref{http://www.tei-c.org/release/doc/tei-p5-doc/en/html/SA.html\#SAPT}{16.1. Links}]\end{specHead} 
    \item[{Module}]
  linking
    \item[{Attributs}]
  Attributs \hyperref[TEI.att.global]{att.global} (\textit{@xml:id}, \textit{@n}, \textit{@xml:lang}, \textit{@xml:base}, \textit{@xml:space})  (\hyperref[TEI.att.global.rendition]{att.global.rendition} (\textit{@rend}, \textit{@style}, \textit{@rendition})) (\hyperref[TEI.att.global.linking]{att.global.linking} (\textit{@corresp}, \textit{@synch}, \textit{@sameAs}, \textit{@copyOf}, \textit{@next}, \textit{@prev}, \textit{@exclude}, \textit{@select})) (\hyperref[TEI.att.global.analytic]{att.global.analytic} (\textit{@ana})) (\hyperref[TEI.att.global.facs]{att.global.facs} (\textit{@facs})) (\hyperref[TEI.att.global.change]{att.global.change} (\textit{@change})) (\hyperref[TEI.att.global.responsibility]{att.global.responsibility} (\textit{@cert}, \textit{@resp})) (\hyperref[TEI.att.global.source]{att.global.source} (\textit{@source})) \hyperref[TEI.att.pointing.group]{att.pointing.group} (\textit{@domains}, \textit{@targFunc})  (\hyperref[TEI.att.pointing]{att.pointing} (\textit{@targetLang}, \textit{@target}, \textit{@evaluate})) (\hyperref[TEI.att.typed]{att.typed} (\textit{@type}, \textit{@subtype}))
    \item[{Membre du}]
  \hyperref[TEI.model.global.meta]{model.global.meta} 
    \item[{Contenu dans}]
  
    \item[analysis: ]
   \hyperref[TEI.cl]{cl} \hyperref[TEI.m]{m} \hyperref[TEI.phr]{phr} \hyperref[TEI.s]{s} \hyperref[TEI.span]{span} \hyperref[TEI.w]{w}\par 
    \item[core: ]
   \hyperref[TEI.abbr]{abbr} \hyperref[TEI.add]{add} \hyperref[TEI.addrLine]{addrLine} \hyperref[TEI.address]{address} \hyperref[TEI.author]{author} \hyperref[TEI.bibl]{bibl} \hyperref[TEI.biblScope]{biblScope} \hyperref[TEI.cit]{cit} \hyperref[TEI.citedRange]{citedRange} \hyperref[TEI.corr]{corr} \hyperref[TEI.date]{date} \hyperref[TEI.del]{del} \hyperref[TEI.distinct]{distinct} \hyperref[TEI.editor]{editor} \hyperref[TEI.email]{email} \hyperref[TEI.emph]{emph} \hyperref[TEI.expan]{expan} \hyperref[TEI.foreign]{foreign} \hyperref[TEI.gloss]{gloss} \hyperref[TEI.head]{head} \hyperref[TEI.headItem]{headItem} \hyperref[TEI.headLabel]{headLabel} \hyperref[TEI.hi]{hi} \hyperref[TEI.imprint]{imprint} \hyperref[TEI.item]{item} \hyperref[TEI.l]{l} \hyperref[TEI.label]{label} \hyperref[TEI.lg]{lg} \hyperref[TEI.list]{list} \hyperref[TEI.measure]{measure} \hyperref[TEI.mentioned]{mentioned} \hyperref[TEI.name]{name} \hyperref[TEI.note]{note} \hyperref[TEI.num]{num} \hyperref[TEI.orig]{orig} \hyperref[TEI.p]{p} \hyperref[TEI.pubPlace]{pubPlace} \hyperref[TEI.publisher]{publisher} \hyperref[TEI.q]{q} \hyperref[TEI.quote]{quote} \hyperref[TEI.ref]{ref} \hyperref[TEI.reg]{reg} \hyperref[TEI.resp]{resp} \hyperref[TEI.rs]{rs} \hyperref[TEI.said]{said} \hyperref[TEI.series]{series} \hyperref[TEI.sic]{sic} \hyperref[TEI.soCalled]{soCalled} \hyperref[TEI.sp]{sp} \hyperref[TEI.speaker]{speaker} \hyperref[TEI.stage]{stage} \hyperref[TEI.street]{street} \hyperref[TEI.term]{term} \hyperref[TEI.textLang]{textLang} \hyperref[TEI.time]{time} \hyperref[TEI.title]{title} \hyperref[TEI.unclear]{unclear}\par 
    \item[figures: ]
   \hyperref[TEI.cell]{cell} \hyperref[TEI.figure]{figure} \hyperref[TEI.table]{table}\par 
    \item[header: ]
   \hyperref[TEI.authority]{authority} \hyperref[TEI.change]{change} \hyperref[TEI.classCode]{classCode} \hyperref[TEI.distributor]{distributor} \hyperref[TEI.edition]{edition} \hyperref[TEI.extent]{extent} \hyperref[TEI.funder]{funder} \hyperref[TEI.language]{language} \hyperref[TEI.licence]{licence}\par 
    \item[linking: ]
   \hyperref[TEI.ab]{ab} \hyperref[TEI.seg]{seg}\par 
    \item[msdescription: ]
   \hyperref[TEI.accMat]{accMat} \hyperref[TEI.acquisition]{acquisition} \hyperref[TEI.additions]{additions} \hyperref[TEI.catchwords]{catchwords} \hyperref[TEI.collation]{collation} \hyperref[TEI.colophon]{colophon} \hyperref[TEI.condition]{condition} \hyperref[TEI.custEvent]{custEvent} \hyperref[TEI.decoNote]{decoNote} \hyperref[TEI.explicit]{explicit} \hyperref[TEI.filiation]{filiation} \hyperref[TEI.finalRubric]{finalRubric} \hyperref[TEI.foliation]{foliation} \hyperref[TEI.heraldry]{heraldry} \hyperref[TEI.incipit]{incipit} \hyperref[TEI.layout]{layout} \hyperref[TEI.material]{material} \hyperref[TEI.msItem]{msItem} \hyperref[TEI.musicNotation]{musicNotation} \hyperref[TEI.objectType]{objectType} \hyperref[TEI.origDate]{origDate} \hyperref[TEI.origPlace]{origPlace} \hyperref[TEI.origin]{origin} \hyperref[TEI.provenance]{provenance} \hyperref[TEI.rubric]{rubric} \hyperref[TEI.secFol]{secFol} \hyperref[TEI.signatures]{signatures} \hyperref[TEI.source]{source} \hyperref[TEI.stamp]{stamp} \hyperref[TEI.summary]{summary} \hyperref[TEI.support]{support} \hyperref[TEI.surrogates]{surrogates} \hyperref[TEI.typeNote]{typeNote} \hyperref[TEI.watermark]{watermark}\par 
    \item[namesdates: ]
   \hyperref[TEI.addName]{addName} \hyperref[TEI.affiliation]{affiliation} \hyperref[TEI.country]{country} \hyperref[TEI.event]{event} \hyperref[TEI.forename]{forename} \hyperref[TEI.genName]{genName} \hyperref[TEI.geogName]{geogName} \hyperref[TEI.nameLink]{nameLink} \hyperref[TEI.org]{org} \hyperref[TEI.orgName]{orgName} \hyperref[TEI.persName]{persName} \hyperref[TEI.person]{person} \hyperref[TEI.personGrp]{personGrp} \hyperref[TEI.persona]{persona} \hyperref[TEI.place]{place} \hyperref[TEI.placeName]{placeName} \hyperref[TEI.region]{region} \hyperref[TEI.roleName]{roleName} \hyperref[TEI.settlement]{settlement} \hyperref[TEI.surname]{surname}\par 
    \item[spoken: ]
   \hyperref[TEI.annotationBlock]{annotationBlock}\par 
    \item[standOff: ]
   \hyperref[TEI.listAnnotation]{listAnnotation}\par 
    \item[textstructure: ]
   \hyperref[TEI.back]{back} \hyperref[TEI.body]{body} \hyperref[TEI.div]{div} \hyperref[TEI.docAuthor]{docAuthor} \hyperref[TEI.docDate]{docDate} \hyperref[TEI.docEdition]{docEdition} \hyperref[TEI.docTitle]{docTitle} \hyperref[TEI.floatingText]{floatingText} \hyperref[TEI.front]{front} \hyperref[TEI.group]{group} \hyperref[TEI.text]{text} \hyperref[TEI.titlePage]{titlePage} \hyperref[TEI.titlePart]{titlePart}\par 
    \item[transcr: ]
   \hyperref[TEI.damage]{damage} \hyperref[TEI.fw]{fw} \hyperref[TEI.line]{line} \hyperref[TEI.metamark]{metamark} \hyperref[TEI.mod]{mod} \hyperref[TEI.restore]{restore} \hyperref[TEI.retrace]{retrace} \hyperref[TEI.secl]{secl} \hyperref[TEI.sourceDoc]{sourceDoc} \hyperref[TEI.supplied]{supplied} \hyperref[TEI.surface]{surface} \hyperref[TEI.surfaceGrp]{surfaceGrp} \hyperref[TEI.surplus]{surplus} \hyperref[TEI.zone]{zone}
    \item[{Peut contenir}]
  
    \item[core: ]
   \hyperref[TEI.ptr]{ptr}\par 
    \item[linking: ]
   \hyperref[TEI.link]{link}
    \item[{Note}]
  \par
Ne peut contenir qu’un ou plusieurs éléments \hyperref[TEI.link]{<link>}, et éventuellement divers éléments pointeurs.\par
Un groupe de liens facilite les opérations de gestion de ces liens ; on devrait l'utiliser pour regrouper un ensemble de liens dans un but précis, et non pas simplement pour donner une valeur par défaut à l'attribut {\itshape type}.
    \item[{Exemple}]
  \leavevmode\bgroup\exampleFont \begin{shaded}\noindent\mbox{}{<\textbf{linkGrp}\hspace*{6pt}{type}="{translation}">}\mbox{}\newline 
\hspace*{6pt}{<\textbf{link}\hspace*{6pt}{target}="{\#CCS1 \#SW1}"/>}\mbox{}\newline 
\hspace*{6pt}{<\textbf{link}\hspace*{6pt}{target}="{\#CCS2 \#SW2}"/>}\mbox{}\newline 
\hspace*{6pt}{<\textbf{link}\hspace*{6pt}{target}="{\#CCS \#SW}"/>}\mbox{}\newline 
{</\textbf{linkGrp}>}\mbox{}\newline 
{<\textbf{div}\hspace*{6pt}{type}="{volume}"\hspace*{6pt}{xml:id}="{CCS}"\mbox{}\newline 
\hspace*{6pt}{xml:lang}="{fr}">}\mbox{}\newline 
\hspace*{6pt}{<\textbf{p}>}\mbox{}\newline 
\hspace*{6pt}\hspace*{6pt}{<\textbf{s}\hspace*{6pt}{xml:id}="{CCS1}">}Longtemps, je me suis couché de bonne heure.{</\textbf{s}>}\mbox{}\newline 
\hspace*{6pt}\hspace*{6pt}{<\textbf{s}\hspace*{6pt}{xml:id}="{CCS2}">}Parfois, à peine ma bougie éteinte, mes yeux se fermaient si vite que je n'avais pas le temps de me dire : "Je m'endors."{</\textbf{s}>}\mbox{}\newline 
\hspace*{6pt}{</\textbf{p}>}\mbox{}\newline 
\textit{<!-- ... -->}\mbox{}\newline 
{</\textbf{div}>}\mbox{}\newline 
{<\textbf{div}\hspace*{6pt}{type}="{volume}"\hspace*{6pt}{xml:id}="{SW}"\hspace*{6pt}{xml:lang}="{en}">}\mbox{}\newline 
\hspace*{6pt}{<\textbf{p}>}\mbox{}\newline 
\hspace*{6pt}\hspace*{6pt}{<\textbf{s}\hspace*{6pt}{xml:id}="{SW1}">}For a long time I used to go to bed early.{</\textbf{s}>}\mbox{}\newline 
\hspace*{6pt}\hspace*{6pt}{<\textbf{s}\hspace*{6pt}{xml:id}="{SW2}">}Sometimes, when I had put out my candle, my eyes would close so quickly that I had not even time to say "I'm going to sleep."{</\textbf{s}>}\mbox{}\newline 
\hspace*{6pt}{</\textbf{p}>}\mbox{}\newline 
\textit{<!-- ... -->}\mbox{}\newline 
{</\textbf{div}>}\end{shaded}\egroup 


    \item[{Exemple}]
  \leavevmode\bgroup\exampleFont \begin{shaded}\noindent\mbox{}{<\textbf{linkGrp}\hspace*{6pt}{type}="{translation}">}\mbox{}\newline 
\hspace*{6pt}{<\textbf{link}\hspace*{6pt}{target}="{\#fr\textunderscore CCS1 \#fr\textunderscore SW1}"/>}\mbox{}\newline 
\hspace*{6pt}{<\textbf{link}\hspace*{6pt}{target}="{\#fr\textunderscore CCS2 \#fr\textunderscore SW2}"/>}\mbox{}\newline 
\hspace*{6pt}{<\textbf{link}\hspace*{6pt}{target}="{\#fr\textunderscore CCS \#fr\textunderscore SW}"/>}\mbox{}\newline 
{</\textbf{linkGrp}>}\mbox{}\newline 
{<\textbf{div}\hspace*{6pt}{type}="{volume}"\hspace*{6pt}{xml:id}="{fr\textunderscore CCS}"\mbox{}\newline 
\hspace*{6pt}{xml:lang}="{fr}">}\mbox{}\newline 
\hspace*{6pt}{<\textbf{p}>}\mbox{}\newline 
\hspace*{6pt}\hspace*{6pt}{<\textbf{s}\hspace*{6pt}{xml:id}="{fr\textunderscore CCS1}">}Longtemps, je me suis couché de bonne heure.{</\textbf{s}>}\mbox{}\newline 
\hspace*{6pt}\hspace*{6pt}{<\textbf{s}\hspace*{6pt}{xml:id}="{fr\textunderscore CCS2}">}Parfois, à peine ma bougie éteinte, mes yeux se fermaient si vite que\mbox{}\newline 
\hspace*{6pt}\hspace*{6pt}\hspace*{6pt}\hspace*{6pt} je n'avais pas le temps de me dire : "Je m'endors."{</\textbf{s}>}\mbox{}\newline 
\hspace*{6pt}{</\textbf{p}>}\mbox{}\newline 
{</\textbf{div}>}\mbox{}\newline 
{<\textbf{div}\hspace*{6pt}{type}="{volume}"\hspace*{6pt}{xml:id}="{fr\textunderscore SW}"\mbox{}\newline 
\hspace*{6pt}{xml:lang}="{en}">}\mbox{}\newline 
\hspace*{6pt}{<\textbf{p}>}\mbox{}\newline 
\hspace*{6pt}\hspace*{6pt}{<\textbf{s}\hspace*{6pt}{xml:id}="{fr\textunderscore SW1}">}For a long time I used to go to bed early.{</\textbf{s}>}\mbox{}\newline 
\hspace*{6pt}\hspace*{6pt}{<\textbf{s}\hspace*{6pt}{xml:id}="{fr\textunderscore SW2}">}Sometimes, when I had put out my candle, my eyes would close so quickly\mbox{}\newline 
\hspace*{6pt}\hspace*{6pt}\hspace*{6pt}\hspace*{6pt} that I had not even time to say "I'm going to sleep."{</\textbf{s}>}\mbox{}\newline 
\hspace*{6pt}{</\textbf{p}>}\mbox{}\newline 
{</\textbf{div}>}\end{shaded}\egroup 


    \item[{Modèle de contenu}]
  \mbox{}\hfill\\[-10pt]\begin{Verbatim}[fontsize=\small]
<content>
 <alternate maxOccurs="unbounded"
  minOccurs="1">
  <elementRef key="link"/>
  <elementRef key="ptr"/>
 </alternate>
</content>
    
\end{Verbatim}

    \item[{Schéma Declaration}]
  \mbox{}\hfill\\[-10pt]\begin{Verbatim}[fontsize=\small]
element linkGrp
{
   tei_att.global.attributes,
   tei_att.pointing.group.attributes,
   ( tei_link | tei_ptr )+
}
\end{Verbatim}

\end{reflist}  \index{list=<list>|oddindex}\index{type=@type!<list>|oddindex}
\begin{reflist}
\item[]\begin{specHead}{TEI.list}{<list> }(liste) contient une suite d'items ordonnés dans une liste. [\xref{http://www.tei-c.org/release/doc/tei-p5-doc/en/html/CO.html\#COLI}{3.7. Lists}]\end{specHead} 
    \item[{Module}]
  core
    \item[{Attributs}]
  Attributs \hyperref[TEI.att.global]{att.global} (\textit{@xml:id}, \textit{@n}, \textit{@xml:lang}, \textit{@xml:base}, \textit{@xml:space})  (\hyperref[TEI.att.global.rendition]{att.global.rendition} (\textit{@rend}, \textit{@style}, \textit{@rendition})) (\hyperref[TEI.att.global.linking]{att.global.linking} (\textit{@corresp}, \textit{@synch}, \textit{@sameAs}, \textit{@copyOf}, \textit{@next}, \textit{@prev}, \textit{@exclude}, \textit{@select})) (\hyperref[TEI.att.global.analytic]{att.global.analytic} (\textit{@ana})) (\hyperref[TEI.att.global.facs]{att.global.facs} (\textit{@facs})) (\hyperref[TEI.att.global.change]{att.global.change} (\textit{@change})) (\hyperref[TEI.att.global.responsibility]{att.global.responsibility} (\textit{@cert}, \textit{@resp})) (\hyperref[TEI.att.global.source]{att.global.source} (\textit{@source})) \hyperref[TEI.att.sortable]{att.sortable} (\textit{@sortKey}) \hyperref[TEI.att.typed]{att.typed} (\unusedattribute{type}, @subtype) \hfil\\[-10pt]\begin{sansreflist}
    \item[@type]
  describes the nature of the items in the list.
\begin{reflist}
    \item[{Dérivé de}]
  \hyperref[TEI.att.typed]{att.typed}
    \item[{Statut}]
  Optionel
    \item[{Type de données}]
  \hyperref[TEI.teidata.enumerated]{teidata.enumerated}
    \item[{Les valeurs suggérées comprennent:}]
  \begin{description}

\item[{gloss}]chaque item de la liste commente un terme ou un concept qui est donné par un élément \hyperref[TEI.label]{<label>} précédant l'item de la liste.
\item[{index}]each list item is an entry in an index such as the alphabetical topical index at the back of a print volume.
\item[{instructions}]each list item is a step in a sequence of instructions, as in a recipe.
\item[{litany}]each list item is one of a sequence of petitions, supplications or invocations, typically in a religious ritual.
\item[{syllogism}]each list item is part of an argument consisting of two or more propositions and a final conclusion derived from them.
\end{description} 
    \item[{Note}]
  \par
La syntaxe formelle des déclarations d'élément permet d'omettre les étiquettes de balises des listes balisées par <list type="gloss"> mais c'est une erreur sémantique.
\end{reflist}  
\end{sansreflist}  
    \item[{Membre du}]
  \hyperref[TEI.model.listLike]{model.listLike} 
    \item[{Contenu dans}]
  
    \item[core: ]
   \hyperref[TEI.add]{add} \hyperref[TEI.corr]{corr} \hyperref[TEI.del]{del} \hyperref[TEI.desc]{desc} \hyperref[TEI.emph]{emph} \hyperref[TEI.head]{head} \hyperref[TEI.hi]{hi} \hyperref[TEI.item]{item} \hyperref[TEI.l]{l} \hyperref[TEI.meeting]{meeting} \hyperref[TEI.note]{note} \hyperref[TEI.orig]{orig} \hyperref[TEI.p]{p} \hyperref[TEI.q]{q} \hyperref[TEI.quote]{quote} \hyperref[TEI.ref]{ref} \hyperref[TEI.reg]{reg} \hyperref[TEI.said]{said} \hyperref[TEI.sic]{sic} \hyperref[TEI.sp]{sp} \hyperref[TEI.stage]{stage} \hyperref[TEI.title]{title} \hyperref[TEI.unclear]{unclear}\par 
    \item[figures: ]
   \hyperref[TEI.cell]{cell} \hyperref[TEI.figDesc]{figDesc} \hyperref[TEI.figure]{figure}\par 
    \item[header: ]
   \hyperref[TEI.abstract]{abstract} \hyperref[TEI.change]{change} \hyperref[TEI.keywords]{keywords} \hyperref[TEI.licence]{licence} \hyperref[TEI.rendition]{rendition} \hyperref[TEI.revisionDesc]{revisionDesc} \hyperref[TEI.sourceDesc]{sourceDesc}\par 
    \item[iso-fs: ]
   \hyperref[TEI.fDescr]{fDescr} \hyperref[TEI.fsDescr]{fsDescr}\par 
    \item[linking: ]
   \hyperref[TEI.ab]{ab} \hyperref[TEI.seg]{seg}\par 
    \item[msdescription: ]
   \hyperref[TEI.accMat]{accMat} \hyperref[TEI.acquisition]{acquisition} \hyperref[TEI.additions]{additions} \hyperref[TEI.collation]{collation} \hyperref[TEI.condition]{condition} \hyperref[TEI.custEvent]{custEvent} \hyperref[TEI.decoNote]{decoNote} \hyperref[TEI.filiation]{filiation} \hyperref[TEI.foliation]{foliation} \hyperref[TEI.layout]{layout} \hyperref[TEI.musicNotation]{musicNotation} \hyperref[TEI.origin]{origin} \hyperref[TEI.provenance]{provenance} \hyperref[TEI.signatures]{signatures} \hyperref[TEI.source]{source} \hyperref[TEI.summary]{summary} \hyperref[TEI.support]{support} \hyperref[TEI.surrogates]{surrogates} \hyperref[TEI.typeNote]{typeNote}\par 
    \item[spoken: ]
   \hyperref[TEI.annotationBlock]{annotationBlock}\par 
    \item[standOff: ]
   \hyperref[TEI.listAnnotation]{listAnnotation}\par 
    \item[textstructure: ]
   \hyperref[TEI.back]{back} \hyperref[TEI.body]{body} \hyperref[TEI.div]{div} \hyperref[TEI.docEdition]{docEdition} \hyperref[TEI.titlePart]{titlePart}\par 
    \item[transcr: ]
   \hyperref[TEI.damage]{damage} \hyperref[TEI.metamark]{metamark} \hyperref[TEI.mod]{mod} \hyperref[TEI.restore]{restore} \hyperref[TEI.retrace]{retrace} \hyperref[TEI.secl]{secl} \hyperref[TEI.supplied]{supplied} \hyperref[TEI.surplus]{surplus}
    \item[{Peut contenir}]
  
    \item[analysis: ]
   \hyperref[TEI.interp]{interp} \hyperref[TEI.interpGrp]{interpGrp} \hyperref[TEI.span]{span} \hyperref[TEI.spanGrp]{spanGrp}\par 
    \item[core: ]
   \hyperref[TEI.cb]{cb} \hyperref[TEI.gap]{gap} \hyperref[TEI.gb]{gb} \hyperref[TEI.head]{head} \hyperref[TEI.headItem]{headItem} \hyperref[TEI.headLabel]{headLabel} \hyperref[TEI.index]{index} \hyperref[TEI.item]{item} \hyperref[TEI.label]{label} \hyperref[TEI.lb]{lb} \hyperref[TEI.meeting]{meeting} \hyperref[TEI.milestone]{milestone} \hyperref[TEI.note]{note} \hyperref[TEI.pb]{pb}\par 
    \item[figures: ]
   \hyperref[TEI.figure]{figure} \hyperref[TEI.notatedMusic]{notatedMusic}\par 
    \item[iso-fs: ]
   \hyperref[TEI.fLib]{fLib} \hyperref[TEI.fs]{fs} \hyperref[TEI.fvLib]{fvLib}\par 
    \item[linking: ]
   \hyperref[TEI.alt]{alt} \hyperref[TEI.altGrp]{altGrp} \hyperref[TEI.anchor]{anchor} \hyperref[TEI.join]{join} \hyperref[TEI.joinGrp]{joinGrp} \hyperref[TEI.link]{link} \hyperref[TEI.linkGrp]{linkGrp} \hyperref[TEI.timeline]{timeline}\par 
    \item[msdescription: ]
   \hyperref[TEI.source]{source}\par 
    \item[textstructure: ]
   \hyperref[TEI.docAuthor]{docAuthor} \hyperref[TEI.docDate]{docDate}\par 
    \item[transcr: ]
   \hyperref[TEI.addSpan]{addSpan} \hyperref[TEI.damageSpan]{damageSpan} \hyperref[TEI.delSpan]{delSpan} \hyperref[TEI.fw]{fw} \hyperref[TEI.listTranspose]{listTranspose} \hyperref[TEI.metamark]{metamark} \hyperref[TEI.space]{space} \hyperref[TEI.substJoin]{substJoin}
    \item[{Note}]
  \par
Peut contenir un titre facultatif suivi d'une succession d'items ou d'une succession de couples constitués d'une étiquette et d'un item, ce dernier type pouvant être précédé par un ou deux titres spécifiques.
    \item[{Exemple}]
  \leavevmode\bgroup\exampleFont \begin{shaded}\noindent\mbox{}{<\textbf{list}\hspace*{6pt}{rend}="{bulleted}">}\mbox{}\newline 
\hspace*{6pt}{<\textbf{item}>}Thé sans sucre et sans lait {</\textbf{item}>}\mbox{}\newline 
\hspace*{6pt}{<\textbf{item}>}Un jus d'ananas{</\textbf{item}>}\mbox{}\newline 
\hspace*{6pt}{<\textbf{item}>}Un yaourt{</\textbf{item}>}\mbox{}\newline 
\hspace*{6pt}{<\textbf{item}>}Trois biscuits de seigle {</\textbf{item}>}\mbox{}\newline 
\hspace*{6pt}{<\textbf{item}>}Carottes râpées{</\textbf{item}>}\mbox{}\newline 
\hspace*{6pt}{<\textbf{item}>}Côtelettes d'agneau (deux){</\textbf{item}>}\mbox{}\newline 
\hspace*{6pt}{<\textbf{item}>}Courgettes{</\textbf{item}>}\mbox{}\newline 
\hspace*{6pt}{<\textbf{item}>}Chèvre frais {</\textbf{item}>}\mbox{}\newline 
\hspace*{6pt}{<\textbf{item}>}Coings{</\textbf{item}>}\mbox{}\newline 
{</\textbf{list}>}\end{shaded}\egroup 


    \item[{Exemple}]
  \leavevmode\bgroup\exampleFont \begin{shaded}\noindent\mbox{}{<\textbf{div}>}\mbox{}\newline 
\hspace*{6pt}{<\textbf{p}>} Selon des critères qui n'appartiennent qu'à lui, Rémi Plassaert a classé ses buvards\mbox{}\newline 
\hspace*{6pt}\hspace*{6pt} en huit tas respectivement surmontés par :{</\textbf{p}>}\mbox{}\newline 
\hspace*{6pt}{<\textbf{list}\hspace*{6pt}{rend}="{bulleted}">}\mbox{}\newline 
\hspace*{6pt}\hspace*{6pt}{<\textbf{item}>}un toréador chantant (dentifrice émail Diamant){</\textbf{item}>}\mbox{}\newline 
\hspace*{6pt}\hspace*{6pt}{<\textbf{item}>}un tapis d'Orient du XVIIe siècle, provenant d'une basilique de Transylvanie\mbox{}\newline 
\hspace*{6pt}\hspace*{6pt}\hspace*{6pt}\hspace*{6pt} (Kalium-Sedaph, soluté de propionate de potassium){</\textbf{item}>}\mbox{}\newline 
\hspace*{6pt}\hspace*{6pt}{<\textbf{item}>}Le Renard et la Cicogne (sic), gravure de Jean-Baptiste Oudry (Papeteries\mbox{}\newline 
\hspace*{6pt}\hspace*{6pt}\hspace*{6pt}\hspace*{6pt} Marquaize, Stencyl, Reprographie){</\textbf{item}>}\mbox{}\newline 
\hspace*{6pt}{</\textbf{list}>}\mbox{}\newline 
{</\textbf{div}>}\end{shaded}\egroup 


    \item[{Exemple}]
  \leavevmode\bgroup\exampleFont \begin{shaded}\noindent\mbox{}{<\textbf{div}>}\mbox{}\newline 
\hspace*{6pt}{<\textbf{p}>} [...] et tout autour, la longue cohorte de ses personnages, avec leur histoire, leur\mbox{}\newline 
\hspace*{6pt}\hspace*{6pt} passé, leurs légendes :{</\textbf{p}>}\mbox{}\newline 
\hspace*{6pt}{<\textbf{list}\hspace*{6pt}{rend}="{numbered}">}\mbox{}\newline 
\hspace*{6pt}\hspace*{6pt}{<\textbf{item}\hspace*{6pt}{n}="{1}">}Pélage vainqueur d'Alkhamah se faisant couronner à Covadonga {</\textbf{item}>}\mbox{}\newline 
\hspace*{6pt}\hspace*{6pt}{<\textbf{item}\hspace*{6pt}{n}="{2}">}La cantatrice exilée de Russie suivant Schönberg à Amsterdam{</\textbf{item}>}\mbox{}\newline 
\hspace*{6pt}\hspace*{6pt}{<\textbf{item}\hspace*{6pt}{n}="{3}">}Le petit chat sourd aux yeux vairons vivant au dernier étage{</\textbf{item}>}\mbox{}\newline 
\hspace*{6pt}\hspace*{6pt}{<\textbf{item}\hspace*{6pt}{n}="{4}">}Le crétin chef d'îlot faisant préparer des tonneaux de sable{</\textbf{item}>}\mbox{}\newline 
\hspace*{6pt}{</\textbf{list}>}\mbox{}\newline 
{</\textbf{div}>}\end{shaded}\egroup 


    \item[{Schematron}]
   <sch:rule context="tei:list[@type='gloss']"> <sch:assert test="tei:label">The content of a "gloss" list should include a sequence of one or more pairs of a label element followed by an item element</sch:assert> </sch:rule>
    \item[{Modèle de contenu}]
  \mbox{}\hfill\\[-10pt]\begin{Verbatim}[fontsize=\small]
<content>
 <sequence maxOccurs="1" minOccurs="1">
  <alternate maxOccurs="unbounded"
   minOccurs="0">
   <classRef key="model.divTop"/>
   <classRef key="model.global"/>
  </alternate>
  <alternate maxOccurs="1" minOccurs="1">
   <sequence maxOccurs="unbounded"
    minOccurs="1">
    <elementRef key="item"/>
    <classRef key="model.global"
     maxOccurs="unbounded" minOccurs="0"/>
   </sequence>
   <sequence maxOccurs="1" minOccurs="1">
    <elementRef key="headLabel"
     minOccurs="0"/>
    <elementRef key="headItem"
     minOccurs="0"/>
    <sequence maxOccurs="unbounded"
     minOccurs="1">
     <elementRef key="label"/>
     <classRef key="model.global"
      maxOccurs="unbounded" minOccurs="0"/>
     <elementRef key="item"/>
     <classRef key="model.global"
      maxOccurs="unbounded" minOccurs="0"/>
    </sequence>
   </sequence>
  </alternate>
  <sequence maxOccurs="unbounded"
   minOccurs="0">
   <classRef key="model.divBottom"/>
   <classRef key="model.global"
    maxOccurs="unbounded" minOccurs="0"/>
  </sequence>
 </sequence>
</content>
    
\end{Verbatim}

    \item[{Schéma Declaration}]
  \mbox{}\hfill\\[-10pt]\begin{Verbatim}[fontsize=\small]
element list
{
   tei_att.global.attributes,
   tei_att.sortable.attributes,
   tei_att.typed.attribute.subtype,
   attribute type
   {
      "gloss" | "index" | "instructions" | "litany" | "syllogism"
   }?,
   (
      ( tei_model.divTop | tei_model.global )*,
      (
         ( tei_item, tei_model.global* )+
       | (
            tei_headLabel?,
            tei_headItem?,
            ( tei_label, tei_model.global*, tei_item, tei_model.global* )+
         )
      ),
      ( tei_model.divBottom, tei_model.global* )*
   )
}
\end{Verbatim}

\end{reflist}  \index{listAnnotation=<listAnnotation>|oddindex}
\begin{reflist}
\item[]\begin{specHead}{TEI.listAnnotation}{<listAnnotation> }Groups together various annotations, for instance for parallel interpretations of a spoken segment\end{specHead} 
    \item[{Namespace}]
  https://xml-schema.delivery.istex.fr/formats/ns1.xsd
    \item[{Module}]
  standOff
    \item[{Attributs}]
  Attributs \hyperref[TEI.att.global]{att.global} (\textit{@xml:id}, \textit{@n}, \textit{@xml:lang}, \textit{@xml:base}, \textit{@xml:space})  (\hyperref[TEI.att.global.rendition]{att.global.rendition} (\textit{@rend}, \textit{@style}, \textit{@rendition})) (\hyperref[TEI.att.global.linking]{att.global.linking} (\textit{@corresp}, \textit{@synch}, \textit{@sameAs}, \textit{@copyOf}, \textit{@next}, \textit{@prev}, \textit{@exclude}, \textit{@select})) (\hyperref[TEI.att.global.analytic]{att.global.analytic} (\textit{@ana})) (\hyperref[TEI.att.global.facs]{att.global.facs} (\textit{@facs})) (\hyperref[TEI.att.global.change]{att.global.change} (\textit{@change})) (\hyperref[TEI.att.global.responsibility]{att.global.responsibility} (\textit{@cert}, \textit{@resp})) (\hyperref[TEI.att.global.source]{att.global.source} (\textit{@source})) \hyperref[TEI.att.typed]{att.typed} (\textit{@type}, \textit{@subtype}) \hyperref[TEI.att.notated]{att.notated} (\textit{@notation}) \hyperref[TEI.att.timed]{att.timed} (\textit{@start}, \textit{@end})  (\hyperref[TEI.att.duration]{att.duration} (\hyperref[TEI.att.duration.w3c]{att.duration.w3c} (\textit{@dur})) (\hyperref[TEI.att.duration.iso]{att.duration.iso} (\textit{@dur-iso})) ) \hyperref[TEI.att.declaring]{att.declaring} (\textit{@decls}) 
    \item[{Membre du}]
  \hyperref[TEI.model.annotation]{model.annotation} 
    \item[{Contenu dans}]
  
    \item[spoken: ]
   \hyperref[TEI.annotationBlock]{annotationBlock}\par 
    \item[standOff: ]
   \hyperref[TEI.listAnnotation]{listAnnotation} \hyperref[TEI.standOff]{standOff}
    \item[{Peut contenir}]
  
    \item[analysis: ]
   \hyperref[TEI.interp]{interp} \hyperref[TEI.interpGrp]{interpGrp} \hyperref[TEI.span]{span} \hyperref[TEI.spanGrp]{spanGrp}\par 
    \item[core: ]
   \hyperref[TEI.author]{author} \hyperref[TEI.bibl]{bibl} \hyperref[TEI.biblStruct]{biblStruct} \hyperref[TEI.date]{date} \hyperref[TEI.head]{head} \hyperref[TEI.index]{index} \hyperref[TEI.list]{list} \hyperref[TEI.listBibl]{listBibl} \hyperref[TEI.name]{name} \hyperref[TEI.ptr]{ptr} \hyperref[TEI.ref]{ref} \hyperref[TEI.rs]{rs} \hyperref[TEI.time]{time}\par 
    \item[figures: ]
   \hyperref[TEI.table]{table}\par 
    \item[header: ]
   \hyperref[TEI.biblFull]{biblFull} \hyperref[TEI.idno]{idno} \hyperref[TEI.keywords]{keywords}\par 
    \item[iso-fs: ]
   \hyperref[TEI.fLib]{fLib} \hyperref[TEI.fs]{fs} \hyperref[TEI.fvLib]{fvLib}\par 
    \item[linking: ]
   \hyperref[TEI.alt]{alt} \hyperref[TEI.altGrp]{altGrp} \hyperref[TEI.join]{join} \hyperref[TEI.joinGrp]{joinGrp} \hyperref[TEI.link]{link} \hyperref[TEI.linkGrp]{linkGrp} \hyperref[TEI.seg]{seg} \hyperref[TEI.timeline]{timeline}\par 
    \item[msdescription: ]
   \hyperref[TEI.msDesc]{msDesc} \hyperref[TEI.source]{source}\par 
    \item[namesdates: ]
   \hyperref[TEI.addName]{addName} \hyperref[TEI.country]{country} \hyperref[TEI.forename]{forename} \hyperref[TEI.genName]{genName} \hyperref[TEI.geogName]{geogName} \hyperref[TEI.listOrg]{listOrg} \hyperref[TEI.listPlace]{listPlace} \hyperref[TEI.location]{location} \hyperref[TEI.nameLink]{nameLink} \hyperref[TEI.orgName]{orgName} \hyperref[TEI.persName]{persName} \hyperref[TEI.placeName]{placeName} \hyperref[TEI.region]{region} \hyperref[TEI.roleName]{roleName} \hyperref[TEI.settlement]{settlement} \hyperref[TEI.state]{state} \hyperref[TEI.surname]{surname}\par 
    \item[spoken: ]
   \hyperref[TEI.annotationBlock]{annotationBlock}\par 
    \item[standOff: ]
   \hyperref[TEI.listAnnotation]{listAnnotation}\par 
    \item[textstructure: ]
   \hyperref[TEI.text]{text}\par 
    \item[transcr: ]
   \hyperref[TEI.listTranspose]{listTranspose} \hyperref[TEI.substJoin]{substJoin} \hyperref[TEI.zone]{zone}
    \item[{Modèle de contenu}]
  \mbox{}\hfill\\[-10pt]\begin{Verbatim}[fontsize=\small]
<content>
 <sequence maxOccurs="1" minOccurs="1">
  <classRef key="model.headLike"
   maxOccurs="unbounded" minOccurs="0"/>
  <classRef key="model.annotation"
   maxOccurs="unbounded" minOccurs="1"/>
 </sequence>
</content>
    
\end{Verbatim}

    \item[{Schéma Declaration}]
  \mbox{}\hfill\\[-10pt]\begin{Verbatim}[fontsize=\small]
element listAnnotation
{
   tei_att.global.attributes,
   tei_att.typed.attributes,
   tei_att.notated.attributes,
   tei_att.timed.attributes,
   tei_att.declaring.attributes,
   ( tei_model.headLike*, tei_model.annotation+ )
}
\end{Verbatim}

\end{reflist}  \index{listBibl=<listBibl>|oddindex}
\begin{reflist}
\item[]\begin{specHead}{TEI.listBibl}{<listBibl> }(liste de références bibliographiques) contient une liste de références bibliographiques de toute nature. [\xref{http://www.tei-c.org/release/doc/tei-p5-doc/en/html/CO.html\#COBITY}{3.11.1. Methods of Encoding Bibliographic References and Lists of References} \xref{http://www.tei-c.org/release/doc/tei-p5-doc/en/html/HD.html\#HD3}{2.2.7. The Source Description} \xref{http://www.tei-c.org/release/doc/tei-p5-doc/en/html/CC.html\#CCAS2}{15.3.2. Declarable Elements}]\end{specHead} 
    \item[{Module}]
  core
    \item[{Attributs}]
  Attributs \hyperref[TEI.att.global]{att.global} (\textit{@xml:id}, \textit{@n}, \textit{@xml:lang}, \textit{@xml:base}, \textit{@xml:space})  (\hyperref[TEI.att.global.rendition]{att.global.rendition} (\textit{@rend}, \textit{@style}, \textit{@rendition})) (\hyperref[TEI.att.global.linking]{att.global.linking} (\textit{@corresp}, \textit{@synch}, \textit{@sameAs}, \textit{@copyOf}, \textit{@next}, \textit{@prev}, \textit{@exclude}, \textit{@select})) (\hyperref[TEI.att.global.analytic]{att.global.analytic} (\textit{@ana})) (\hyperref[TEI.att.global.facs]{att.global.facs} (\textit{@facs})) (\hyperref[TEI.att.global.change]{att.global.change} (\textit{@change})) (\hyperref[TEI.att.global.responsibility]{att.global.responsibility} (\textit{@cert}, \textit{@resp})) (\hyperref[TEI.att.global.source]{att.global.source} (\textit{@source})) \hyperref[TEI.att.sortable]{att.sortable} (\textit{@sortKey}) \hyperref[TEI.att.declarable]{att.declarable} (\textit{@default}) \hyperref[TEI.att.typed]{att.typed} (\textit{@type}, \textit{@subtype}) 
    \item[{Membre du}]
  \hyperref[TEI.model.biblLike]{model.biblLike} \hyperref[TEI.model.frontPart]{model.frontPart} 
    \item[{Contenu dans}]
  
    \item[core: ]
   \hyperref[TEI.add]{add} \hyperref[TEI.cit]{cit} \hyperref[TEI.corr]{corr} \hyperref[TEI.del]{del} \hyperref[TEI.desc]{desc} \hyperref[TEI.emph]{emph} \hyperref[TEI.head]{head} \hyperref[TEI.hi]{hi} \hyperref[TEI.item]{item} \hyperref[TEI.l]{l} \hyperref[TEI.listBibl]{listBibl} \hyperref[TEI.meeting]{meeting} \hyperref[TEI.note]{note} \hyperref[TEI.orig]{orig} \hyperref[TEI.p]{p} \hyperref[TEI.q]{q} \hyperref[TEI.quote]{quote} \hyperref[TEI.ref]{ref} \hyperref[TEI.reg]{reg} \hyperref[TEI.relatedItem]{relatedItem} \hyperref[TEI.said]{said} \hyperref[TEI.sic]{sic} \hyperref[TEI.stage]{stage} \hyperref[TEI.title]{title} \hyperref[TEI.unclear]{unclear}\par 
    \item[figures: ]
   \hyperref[TEI.cell]{cell} \hyperref[TEI.figDesc]{figDesc} \hyperref[TEI.figure]{figure}\par 
    \item[header: ]
   \hyperref[TEI.change]{change} \hyperref[TEI.licence]{licence} \hyperref[TEI.rendition]{rendition} \hyperref[TEI.sourceDesc]{sourceDesc} \hyperref[TEI.taxonomy]{taxonomy}\par 
    \item[iso-fs: ]
   \hyperref[TEI.fDescr]{fDescr} \hyperref[TEI.fsDescr]{fsDescr}\par 
    \item[linking: ]
   \hyperref[TEI.ab]{ab} \hyperref[TEI.seg]{seg}\par 
    \item[msdescription: ]
   \hyperref[TEI.accMat]{accMat} \hyperref[TEI.acquisition]{acquisition} \hyperref[TEI.additional]{additional} \hyperref[TEI.additions]{additions} \hyperref[TEI.collation]{collation} \hyperref[TEI.condition]{condition} \hyperref[TEI.custEvent]{custEvent} \hyperref[TEI.decoNote]{decoNote} \hyperref[TEI.filiation]{filiation} \hyperref[TEI.foliation]{foliation} \hyperref[TEI.layout]{layout} \hyperref[TEI.msItem]{msItem} \hyperref[TEI.msItemStruct]{msItemStruct} \hyperref[TEI.musicNotation]{musicNotation} \hyperref[TEI.origin]{origin} \hyperref[TEI.provenance]{provenance} \hyperref[TEI.signatures]{signatures} \hyperref[TEI.source]{source} \hyperref[TEI.summary]{summary} \hyperref[TEI.support]{support} \hyperref[TEI.surrogates]{surrogates} \hyperref[TEI.typeNote]{typeNote}\par 
    \item[namesdates: ]
   \hyperref[TEI.event]{event} \hyperref[TEI.location]{location} \hyperref[TEI.org]{org} \hyperref[TEI.person]{person} \hyperref[TEI.personGrp]{personGrp} \hyperref[TEI.persona]{persona} \hyperref[TEI.place]{place} \hyperref[TEI.state]{state}\par 
    \item[spoken: ]
   \hyperref[TEI.annotationBlock]{annotationBlock}\par 
    \item[standOff: ]
   \hyperref[TEI.listAnnotation]{listAnnotation}\par 
    \item[textstructure: ]
   \hyperref[TEI.back]{back} \hyperref[TEI.body]{body} \hyperref[TEI.div]{div} \hyperref[TEI.docEdition]{docEdition} \hyperref[TEI.front]{front} \hyperref[TEI.titlePart]{titlePart}\par 
    \item[transcr: ]
   \hyperref[TEI.damage]{damage} \hyperref[TEI.metamark]{metamark} \hyperref[TEI.mod]{mod} \hyperref[TEI.restore]{restore} \hyperref[TEI.retrace]{retrace} \hyperref[TEI.secl]{secl} \hyperref[TEI.supplied]{supplied} \hyperref[TEI.surplus]{surplus}
    \item[{Peut contenir}]
  
    \item[core: ]
   \hyperref[TEI.bibl]{bibl} \hyperref[TEI.biblStruct]{biblStruct} \hyperref[TEI.cb]{cb} \hyperref[TEI.gb]{gb} \hyperref[TEI.head]{head} \hyperref[TEI.lb]{lb} \hyperref[TEI.listBibl]{listBibl} \hyperref[TEI.milestone]{milestone} \hyperref[TEI.pb]{pb}\par 
    \item[header: ]
   \hyperref[TEI.biblFull]{biblFull}\par 
    \item[linking: ]
   \hyperref[TEI.anchor]{anchor}\par 
    \item[msdescription: ]
   \hyperref[TEI.msDesc]{msDesc}\par 
    \item[transcr: ]
   \hyperref[TEI.fw]{fw}
    \item[{Exemple}]
  \leavevmode\bgroup\exampleFont \begin{shaded}\noindent\mbox{}{<\textbf{listBibl}>}\mbox{}\newline 
\hspace*{6pt}{<\textbf{head}>}Liste des ouvrages cités{</\textbf{head}>}\mbox{}\newline 
\hspace*{6pt}{<\textbf{bibl}>}Les Petits Romantiques {</\textbf{bibl}>}\mbox{}\newline 
\hspace*{6pt}{<\textbf{biblStruct}>}\mbox{}\newline 
\hspace*{6pt}\hspace*{6pt}{<\textbf{analytic}>}\mbox{}\newline 
\hspace*{6pt}\hspace*{6pt}\hspace*{6pt}{<\textbf{title}>}La poésie en prose{</\textbf{title}>}\mbox{}\newline 
\hspace*{6pt}\hspace*{6pt}{</\textbf{analytic}>}\mbox{}\newline 
\hspace*{6pt}\hspace*{6pt}{<\textbf{monogr}>}\mbox{}\newline 
\hspace*{6pt}\hspace*{6pt}\hspace*{6pt}{<\textbf{title}>}Aloysius Bertrand, "inventeur" du poème en prose{</\textbf{title}>}\mbox{}\newline 
\hspace*{6pt}\hspace*{6pt}\hspace*{6pt}{<\textbf{author}>}Bert Guégand{</\textbf{author}>}\mbox{}\newline 
\hspace*{6pt}\hspace*{6pt}\hspace*{6pt}{<\textbf{imprint}>}\mbox{}\newline 
\hspace*{6pt}\hspace*{6pt}\hspace*{6pt}\hspace*{6pt}{<\textbf{publisher}>}PUN{</\textbf{publisher}>}\mbox{}\newline 
\hspace*{6pt}\hspace*{6pt}\hspace*{6pt}\hspace*{6pt}{<\textbf{date}>}2000{</\textbf{date}>}\mbox{}\newline 
\hspace*{6pt}\hspace*{6pt}\hspace*{6pt}{</\textbf{imprint}>}\mbox{}\newline 
\hspace*{6pt}\hspace*{6pt}{</\textbf{monogr}>}\mbox{}\newline 
\hspace*{6pt}{</\textbf{biblStruct}>}\mbox{}\newline 
{</\textbf{listBibl}>}\end{shaded}\egroup 


    \item[{Modèle de contenu}]
  \mbox{}\hfill\\[-10pt]\begin{Verbatim}[fontsize=\small]
<content>
 <sequence maxOccurs="1" minOccurs="1">
  <classRef key="model.headLike"
   maxOccurs="unbounded" minOccurs="0"/>
  <alternate maxOccurs="unbounded"
   minOccurs="1">
   <classRef key="model.biblLike"/>
   <classRef key="model.milestoneLike"/>
  </alternate>
  <alternate maxOccurs="unbounded"
   minOccurs="0">
   <elementRef key="relation"/>
   <elementRef key="listRelation"/>
  </alternate>
 </sequence>
</content>
    
\end{Verbatim}

    \item[{Schéma Declaration}]
  \mbox{}\hfill\\[-10pt]\begin{Verbatim}[fontsize=\small]
element listBibl
{
   tei_att.global.attributes,
   tei_att.sortable.attributes,
   tei_att.declarable.attributes,
   tei_att.typed.attributes,
   (
      tei_model.headLike*,
      ( tei_model.biblLike | tei_model.milestoneLike )+,
      ( relation | listRelation )*
   )
}
\end{Verbatim}

\end{reflist}  \index{listOrg=<listOrg>|oddindex}
\begin{reflist}
\item[]\begin{specHead}{TEI.listOrg}{<listOrg> }(liste d'organisations) contient une liste d'éléments, chacun d'eux fournissant des informations sur une organisation identifiable. [\xref{http://www.tei-c.org/release/doc/tei-p5-doc/en/html/ND.html\#NDORG}{13.2.2. Organizational Names}]\end{specHead} 
    \item[{Module}]
  namesdates
    \item[{Attributs}]
  Attributs \hyperref[TEI.att.global]{att.global} (\textit{@xml:id}, \textit{@n}, \textit{@xml:lang}, \textit{@xml:base}, \textit{@xml:space})  (\hyperref[TEI.att.global.rendition]{att.global.rendition} (\textit{@rend}, \textit{@style}, \textit{@rendition})) (\hyperref[TEI.att.global.linking]{att.global.linking} (\textit{@corresp}, \textit{@synch}, \textit{@sameAs}, \textit{@copyOf}, \textit{@next}, \textit{@prev}, \textit{@exclude}, \textit{@select})) (\hyperref[TEI.att.global.analytic]{att.global.analytic} (\textit{@ana})) (\hyperref[TEI.att.global.facs]{att.global.facs} (\textit{@facs})) (\hyperref[TEI.att.global.change]{att.global.change} (\textit{@change})) (\hyperref[TEI.att.global.responsibility]{att.global.responsibility} (\textit{@cert}, \textit{@resp})) (\hyperref[TEI.att.global.source]{att.global.source} (\textit{@source})) \hyperref[TEI.att.typed]{att.typed} (\textit{@type}, \textit{@subtype}) \hyperref[TEI.att.declarable]{att.declarable} (\textit{@default}) \hyperref[TEI.att.sortable]{att.sortable} (\textit{@sortKey}) 
    \item[{Membre du}]
  \hyperref[TEI.model.listLike]{model.listLike} \hyperref[TEI.model.orgPart]{model.orgPart}
    \item[{Contenu dans}]
  
    \item[core: ]
   \hyperref[TEI.add]{add} \hyperref[TEI.corr]{corr} \hyperref[TEI.del]{del} \hyperref[TEI.desc]{desc} \hyperref[TEI.emph]{emph} \hyperref[TEI.head]{head} \hyperref[TEI.hi]{hi} \hyperref[TEI.item]{item} \hyperref[TEI.l]{l} \hyperref[TEI.meeting]{meeting} \hyperref[TEI.note]{note} \hyperref[TEI.orig]{orig} \hyperref[TEI.p]{p} \hyperref[TEI.q]{q} \hyperref[TEI.quote]{quote} \hyperref[TEI.ref]{ref} \hyperref[TEI.reg]{reg} \hyperref[TEI.said]{said} \hyperref[TEI.sic]{sic} \hyperref[TEI.sp]{sp} \hyperref[TEI.stage]{stage} \hyperref[TEI.title]{title} \hyperref[TEI.unclear]{unclear}\par 
    \item[figures: ]
   \hyperref[TEI.cell]{cell} \hyperref[TEI.figDesc]{figDesc} \hyperref[TEI.figure]{figure}\par 
    \item[header: ]
   \hyperref[TEI.abstract]{abstract} \hyperref[TEI.change]{change} \hyperref[TEI.licence]{licence} \hyperref[TEI.rendition]{rendition} \hyperref[TEI.sourceDesc]{sourceDesc}\par 
    \item[iso-fs: ]
   \hyperref[TEI.fDescr]{fDescr} \hyperref[TEI.fsDescr]{fsDescr}\par 
    \item[linking: ]
   \hyperref[TEI.ab]{ab} \hyperref[TEI.seg]{seg}\par 
    \item[msdescription: ]
   \hyperref[TEI.accMat]{accMat} \hyperref[TEI.acquisition]{acquisition} \hyperref[TEI.additions]{additions} \hyperref[TEI.collation]{collation} \hyperref[TEI.condition]{condition} \hyperref[TEI.custEvent]{custEvent} \hyperref[TEI.decoNote]{decoNote} \hyperref[TEI.filiation]{filiation} \hyperref[TEI.foliation]{foliation} \hyperref[TEI.layout]{layout} \hyperref[TEI.musicNotation]{musicNotation} \hyperref[TEI.origin]{origin} \hyperref[TEI.provenance]{provenance} \hyperref[TEI.signatures]{signatures} \hyperref[TEI.source]{source} \hyperref[TEI.summary]{summary} \hyperref[TEI.support]{support} \hyperref[TEI.surrogates]{surrogates} \hyperref[TEI.typeNote]{typeNote}\par 
    \item[namesdates: ]
   \hyperref[TEI.listOrg]{listOrg} \hyperref[TEI.org]{org}\par 
    \item[spoken: ]
   \hyperref[TEI.annotationBlock]{annotationBlock}\par 
    \item[standOff: ]
   \hyperref[TEI.listAnnotation]{listAnnotation}\par 
    \item[textstructure: ]
   \hyperref[TEI.back]{back} \hyperref[TEI.body]{body} \hyperref[TEI.div]{div} \hyperref[TEI.docEdition]{docEdition} \hyperref[TEI.titlePart]{titlePart}\par 
    \item[transcr: ]
   \hyperref[TEI.damage]{damage} \hyperref[TEI.metamark]{metamark} \hyperref[TEI.mod]{mod} \hyperref[TEI.restore]{restore} \hyperref[TEI.retrace]{retrace} \hyperref[TEI.secl]{secl} \hyperref[TEI.supplied]{supplied} \hyperref[TEI.surplus]{surplus}
    \item[{Peut contenir}]
  
    \item[core: ]
   \hyperref[TEI.head]{head}\par 
    \item[namesdates: ]
   \hyperref[TEI.listOrg]{listOrg} \hyperref[TEI.org]{org}
    \item[{Note}]
  \par
L'attribut type peut être utilisé pour établir des listes par type d'organisation si cela présente un intérêt.
    \item[{Exemple}]
  \leavevmode\bgroup\exampleFont \begin{shaded}\noindent\mbox{}{<\textbf{listOrg}>}\mbox{}\newline 
\hspace*{6pt}{<\textbf{head}>}Libyans{</\textbf{head}>}\mbox{}\newline 
\hspace*{6pt}{<\textbf{org}>}\mbox{}\newline 
\hspace*{6pt}\hspace*{6pt}{<\textbf{orgName}>}Adyrmachidae{</\textbf{orgName}>}\mbox{}\newline 
\hspace*{6pt}\hspace*{6pt}{<\textbf{desc}>}These people have, in most points, the same customs as the Egyptians, but\mbox{}\newline 
\hspace*{6pt}\hspace*{6pt}\hspace*{6pt}\hspace*{6pt} use the costume of the Libyans. Their women wear on each leg a ring made of\mbox{}\newline 
\hspace*{6pt}\hspace*{6pt}\hspace*{6pt}\hspace*{6pt} bronze [...]{</\textbf{desc}>}\mbox{}\newline 
\hspace*{6pt}{</\textbf{org}>}\mbox{}\newline 
\hspace*{6pt}{<\textbf{org}>}\mbox{}\newline 
\hspace*{6pt}\hspace*{6pt}{<\textbf{orgName}>}Nasamonians{</\textbf{orgName}>}\mbox{}\newline 
\hspace*{6pt}\hspace*{6pt}{<\textbf{desc}>}In summer they leave their flocks and herds upon the sea-shore, and go up\mbox{}\newline 
\hspace*{6pt}\hspace*{6pt}\hspace*{6pt}\hspace*{6pt} the country to a place called Augila, where they gather the dates from the\mbox{}\newline 
\hspace*{6pt}\hspace*{6pt}\hspace*{6pt}\hspace*{6pt} palms [...]{</\textbf{desc}>}\mbox{}\newline 
\hspace*{6pt}{</\textbf{org}>}\mbox{}\newline 
\hspace*{6pt}{<\textbf{org}>}\mbox{}\newline 
\hspace*{6pt}\hspace*{6pt}{<\textbf{orgName}>}Garamantians{</\textbf{orgName}>}\mbox{}\newline 
\hspace*{6pt}\hspace*{6pt}{<\textbf{desc}>}[...] avoid all society or intercourse with their fellow-men, have no\mbox{}\newline 
\hspace*{6pt}\hspace*{6pt}\hspace*{6pt}\hspace*{6pt} weapon of war, and do not know how to defend themselves. [...]{</\textbf{desc}>}\mbox{}\newline 
\textit{<!-- ... -->}\mbox{}\newline 
\hspace*{6pt}{</\textbf{org}>}\mbox{}\newline 
{</\textbf{listOrg}>}\end{shaded}\egroup 


    \item[{Modèle de contenu}]
  \mbox{}\hfill\\[-10pt]\begin{Verbatim}[fontsize=\small]
<content>
 <sequence maxOccurs="1" minOccurs="1">
  <classRef key="model.headLike"
   maxOccurs="unbounded" minOccurs="0"/>
  <alternate maxOccurs="unbounded"
   minOccurs="1">
   <elementRef key="org"/>
   <elementRef key="listOrg"/>
  </alternate>
  <alternate maxOccurs="unbounded"
   minOccurs="0">
   <elementRef key="relation"/>
   <elementRef key="listRelation"/>
  </alternate>
 </sequence>
</content>
    
\end{Verbatim}

    \item[{Schéma Declaration}]
  \mbox{}\hfill\\[-10pt]\begin{Verbatim}[fontsize=\small]
element listOrg
{
   tei_att.global.attributes,
   tei_att.typed.attributes,
   tei_att.declarable.attributes,
   tei_att.sortable.attributes,
   (
      tei_model.headLike*,
      ( tei_org | tei_listOrg )+,
      ( relation | listRelation )*
   )
}
\end{Verbatim}

\end{reflist}  \index{listPlace=<listPlace>|oddindex}
\begin{reflist}
\item[]\begin{specHead}{TEI.listPlace}{<listPlace> }(liste de lieux) contient une liste de lieux, qui peut être suivie d'une liste de relations définies entre les lieux (autres que la relation d'inclusion). [\xref{http://www.tei-c.org/release/doc/tei-p5-doc/en/html/HD.html\#HD3}{2.2.7. The Source Description} \xref{http://www.tei-c.org/release/doc/tei-p5-doc/en/html/ND.html\#NDGEOG}{13.3.4. Places}]\end{specHead} 
    \item[{Module}]
  namesdates
    \item[{Attributs}]
  Attributs \hyperref[TEI.att.global]{att.global} (\textit{@xml:id}, \textit{@n}, \textit{@xml:lang}, \textit{@xml:base}, \textit{@xml:space})  (\hyperref[TEI.att.global.rendition]{att.global.rendition} (\textit{@rend}, \textit{@style}, \textit{@rendition})) (\hyperref[TEI.att.global.linking]{att.global.linking} (\textit{@corresp}, \textit{@synch}, \textit{@sameAs}, \textit{@copyOf}, \textit{@next}, \textit{@prev}, \textit{@exclude}, \textit{@select})) (\hyperref[TEI.att.global.analytic]{att.global.analytic} (\textit{@ana})) (\hyperref[TEI.att.global.facs]{att.global.facs} (\textit{@facs})) (\hyperref[TEI.att.global.change]{att.global.change} (\textit{@change})) (\hyperref[TEI.att.global.responsibility]{att.global.responsibility} (\textit{@cert}, \textit{@resp})) (\hyperref[TEI.att.global.source]{att.global.source} (\textit{@source})) \hyperref[TEI.att.typed]{att.typed} (\textit{@type}, \textit{@subtype}) \hyperref[TEI.att.declarable]{att.declarable} (\textit{@default}) \hyperref[TEI.att.sortable]{att.sortable} (\textit{@sortKey}) 
    \item[{Membre du}]
  \hyperref[TEI.model.listLike]{model.listLike} \hyperref[TEI.model.orgPart]{model.orgPart} 
    \item[{Contenu dans}]
  
    \item[core: ]
   \hyperref[TEI.add]{add} \hyperref[TEI.corr]{corr} \hyperref[TEI.del]{del} \hyperref[TEI.desc]{desc} \hyperref[TEI.emph]{emph} \hyperref[TEI.head]{head} \hyperref[TEI.hi]{hi} \hyperref[TEI.item]{item} \hyperref[TEI.l]{l} \hyperref[TEI.meeting]{meeting} \hyperref[TEI.note]{note} \hyperref[TEI.orig]{orig} \hyperref[TEI.p]{p} \hyperref[TEI.q]{q} \hyperref[TEI.quote]{quote} \hyperref[TEI.ref]{ref} \hyperref[TEI.reg]{reg} \hyperref[TEI.said]{said} \hyperref[TEI.sic]{sic} \hyperref[TEI.sp]{sp} \hyperref[TEI.stage]{stage} \hyperref[TEI.title]{title} \hyperref[TEI.unclear]{unclear}\par 
    \item[figures: ]
   \hyperref[TEI.cell]{cell} \hyperref[TEI.figDesc]{figDesc} \hyperref[TEI.figure]{figure}\par 
    \item[header: ]
   \hyperref[TEI.abstract]{abstract} \hyperref[TEI.change]{change} \hyperref[TEI.licence]{licence} \hyperref[TEI.rendition]{rendition} \hyperref[TEI.sourceDesc]{sourceDesc}\par 
    \item[iso-fs: ]
   \hyperref[TEI.fDescr]{fDescr} \hyperref[TEI.fsDescr]{fsDescr}\par 
    \item[linking: ]
   \hyperref[TEI.ab]{ab} \hyperref[TEI.seg]{seg}\par 
    \item[msdescription: ]
   \hyperref[TEI.accMat]{accMat} \hyperref[TEI.acquisition]{acquisition} \hyperref[TEI.additions]{additions} \hyperref[TEI.collation]{collation} \hyperref[TEI.condition]{condition} \hyperref[TEI.custEvent]{custEvent} \hyperref[TEI.decoNote]{decoNote} \hyperref[TEI.filiation]{filiation} \hyperref[TEI.foliation]{foliation} \hyperref[TEI.layout]{layout} \hyperref[TEI.musicNotation]{musicNotation} \hyperref[TEI.origin]{origin} \hyperref[TEI.provenance]{provenance} \hyperref[TEI.signatures]{signatures} \hyperref[TEI.source]{source} \hyperref[TEI.summary]{summary} \hyperref[TEI.support]{support} \hyperref[TEI.surrogates]{surrogates} \hyperref[TEI.typeNote]{typeNote}\par 
    \item[namesdates: ]
   \hyperref[TEI.listPlace]{listPlace} \hyperref[TEI.org]{org} \hyperref[TEI.place]{place}\par 
    \item[spoken: ]
   \hyperref[TEI.annotationBlock]{annotationBlock}\par 
    \item[standOff: ]
   \hyperref[TEI.listAnnotation]{listAnnotation}\par 
    \item[textstructure: ]
   \hyperref[TEI.back]{back} \hyperref[TEI.body]{body} \hyperref[TEI.div]{div} \hyperref[TEI.docEdition]{docEdition} \hyperref[TEI.titlePart]{titlePart}\par 
    \item[transcr: ]
   \hyperref[TEI.damage]{damage} \hyperref[TEI.metamark]{metamark} \hyperref[TEI.mod]{mod} \hyperref[TEI.restore]{restore} \hyperref[TEI.retrace]{retrace} \hyperref[TEI.secl]{secl} \hyperref[TEI.supplied]{supplied} \hyperref[TEI.surplus]{surplus}
    \item[{Peut contenir}]
  
    \item[core: ]
   \hyperref[TEI.head]{head}\par 
    \item[namesdates: ]
   \hyperref[TEI.listPlace]{listPlace} \hyperref[TEI.place]{place}
    \item[{Exemple}]
  \leavevmode\bgroup\exampleFont \begin{shaded}\noindent\mbox{}{<\textbf{listPlace}\hspace*{6pt}{type}="{offshoreIslands}">}\mbox{}\newline 
\hspace*{6pt}{<\textbf{place}>}\mbox{}\newline 
\hspace*{6pt}\hspace*{6pt}{<\textbf{placeName}>}La roche qui pleure{</\textbf{placeName}>}\mbox{}\newline 
\hspace*{6pt}{</\textbf{place}>}\mbox{}\newline 
\hspace*{6pt}{<\textbf{place}>}\mbox{}\newline 
\hspace*{6pt}\hspace*{6pt}{<\textbf{placeName}>}Ile aux cerfs{</\textbf{placeName}>}\mbox{}\newline 
\hspace*{6pt}{</\textbf{place}>}\mbox{}\newline 
{</\textbf{listPlace}>}\end{shaded}\egroup 


    \item[{Modèle de contenu}]
  \mbox{}\hfill\\[-10pt]\begin{Verbatim}[fontsize=\small]
<content>
 <sequence maxOccurs="1" minOccurs="1">
  <classRef key="model.headLike"
   maxOccurs="unbounded" minOccurs="0"/>
  <alternate maxOccurs="unbounded"
   minOccurs="1">
   <classRef key="model.placeLike"/>
   <elementRef key="listPlace"/>
  </alternate>
  <alternate maxOccurs="unbounded"
   minOccurs="0">
   <elementRef key="relation"/>
   <elementRef key="listRelation"/>
  </alternate>
 </sequence>
</content>
    
\end{Verbatim}

    \item[{Schéma Declaration}]
  \mbox{}\hfill\\[-10pt]\begin{Verbatim}[fontsize=\small]
element listPlace
{
   tei_att.global.attributes,
   tei_att.typed.attributes,
   tei_att.declarable.attributes,
   tei_att.sortable.attributes,
   (
      tei_model.headLike*,
      ( tei_model.placeLike | tei_listPlace )+,
      ( relation | listRelation )*
   )
}
\end{Verbatim}

\end{reflist}  \index{listTranspose=<listTranspose>|oddindex}
\begin{reflist}
\item[]\begin{specHead}{TEI.listTranspose}{<listTranspose> }supplies a list of transpositions, each of which is indicated at some point in a document typically by means of metamarks. [\xref{http://www.tei-c.org/release/doc/tei-p5-doc/en/html/PH.html\#transpo}{11.3.4.5. Transpositions}]\end{specHead} 
    \item[{Module}]
  transcr
    \item[{Attributs}]
  Attributs \hyperref[TEI.att.global]{att.global} (\textit{@xml:id}, \textit{@n}, \textit{@xml:lang}, \textit{@xml:base}, \textit{@xml:space})  (\hyperref[TEI.att.global.rendition]{att.global.rendition} (\textit{@rend}, \textit{@style}, \textit{@rendition})) (\hyperref[TEI.att.global.linking]{att.global.linking} (\textit{@corresp}, \textit{@synch}, \textit{@sameAs}, \textit{@copyOf}, \textit{@next}, \textit{@prev}, \textit{@exclude}, \textit{@select})) (\hyperref[TEI.att.global.analytic]{att.global.analytic} (\textit{@ana})) (\hyperref[TEI.att.global.facs]{att.global.facs} (\textit{@facs})) (\hyperref[TEI.att.global.change]{att.global.change} (\textit{@change})) (\hyperref[TEI.att.global.responsibility]{att.global.responsibility} (\textit{@cert}, \textit{@resp})) (\hyperref[TEI.att.global.source]{att.global.source} (\textit{@source}))
    \item[{Membre du}]
  \hyperref[TEI.model.global.meta]{model.global.meta} \hyperref[TEI.model.profileDescPart]{model.profileDescPart}
    \item[{Contenu dans}]
  
    \item[analysis: ]
   \hyperref[TEI.cl]{cl} \hyperref[TEI.m]{m} \hyperref[TEI.phr]{phr} \hyperref[TEI.s]{s} \hyperref[TEI.span]{span} \hyperref[TEI.w]{w}\par 
    \item[core: ]
   \hyperref[TEI.abbr]{abbr} \hyperref[TEI.add]{add} \hyperref[TEI.addrLine]{addrLine} \hyperref[TEI.address]{address} \hyperref[TEI.author]{author} \hyperref[TEI.bibl]{bibl} \hyperref[TEI.biblScope]{biblScope} \hyperref[TEI.cit]{cit} \hyperref[TEI.citedRange]{citedRange} \hyperref[TEI.corr]{corr} \hyperref[TEI.date]{date} \hyperref[TEI.del]{del} \hyperref[TEI.distinct]{distinct} \hyperref[TEI.editor]{editor} \hyperref[TEI.email]{email} \hyperref[TEI.emph]{emph} \hyperref[TEI.expan]{expan} \hyperref[TEI.foreign]{foreign} \hyperref[TEI.gloss]{gloss} \hyperref[TEI.head]{head} \hyperref[TEI.headItem]{headItem} \hyperref[TEI.headLabel]{headLabel} \hyperref[TEI.hi]{hi} \hyperref[TEI.imprint]{imprint} \hyperref[TEI.item]{item} \hyperref[TEI.l]{l} \hyperref[TEI.label]{label} \hyperref[TEI.lg]{lg} \hyperref[TEI.list]{list} \hyperref[TEI.measure]{measure} \hyperref[TEI.mentioned]{mentioned} \hyperref[TEI.name]{name} \hyperref[TEI.note]{note} \hyperref[TEI.num]{num} \hyperref[TEI.orig]{orig} \hyperref[TEI.p]{p} \hyperref[TEI.pubPlace]{pubPlace} \hyperref[TEI.publisher]{publisher} \hyperref[TEI.q]{q} \hyperref[TEI.quote]{quote} \hyperref[TEI.ref]{ref} \hyperref[TEI.reg]{reg} \hyperref[TEI.resp]{resp} \hyperref[TEI.rs]{rs} \hyperref[TEI.said]{said} \hyperref[TEI.series]{series} \hyperref[TEI.sic]{sic} \hyperref[TEI.soCalled]{soCalled} \hyperref[TEI.sp]{sp} \hyperref[TEI.speaker]{speaker} \hyperref[TEI.stage]{stage} \hyperref[TEI.street]{street} \hyperref[TEI.term]{term} \hyperref[TEI.textLang]{textLang} \hyperref[TEI.time]{time} \hyperref[TEI.title]{title} \hyperref[TEI.unclear]{unclear}\par 
    \item[figures: ]
   \hyperref[TEI.cell]{cell} \hyperref[TEI.figure]{figure} \hyperref[TEI.table]{table}\par 
    \item[header: ]
   \hyperref[TEI.authority]{authority} \hyperref[TEI.change]{change} \hyperref[TEI.classCode]{classCode} \hyperref[TEI.distributor]{distributor} \hyperref[TEI.edition]{edition} \hyperref[TEI.extent]{extent} \hyperref[TEI.funder]{funder} \hyperref[TEI.language]{language} \hyperref[TEI.licence]{licence} \hyperref[TEI.profileDesc]{profileDesc}\par 
    \item[linking: ]
   \hyperref[TEI.ab]{ab} \hyperref[TEI.seg]{seg}\par 
    \item[msdescription: ]
   \hyperref[TEI.accMat]{accMat} \hyperref[TEI.acquisition]{acquisition} \hyperref[TEI.additions]{additions} \hyperref[TEI.catchwords]{catchwords} \hyperref[TEI.collation]{collation} \hyperref[TEI.colophon]{colophon} \hyperref[TEI.condition]{condition} \hyperref[TEI.custEvent]{custEvent} \hyperref[TEI.decoNote]{decoNote} \hyperref[TEI.explicit]{explicit} \hyperref[TEI.filiation]{filiation} \hyperref[TEI.finalRubric]{finalRubric} \hyperref[TEI.foliation]{foliation} \hyperref[TEI.heraldry]{heraldry} \hyperref[TEI.incipit]{incipit} \hyperref[TEI.layout]{layout} \hyperref[TEI.material]{material} \hyperref[TEI.msItem]{msItem} \hyperref[TEI.musicNotation]{musicNotation} \hyperref[TEI.objectType]{objectType} \hyperref[TEI.origDate]{origDate} \hyperref[TEI.origPlace]{origPlace} \hyperref[TEI.origin]{origin} \hyperref[TEI.provenance]{provenance} \hyperref[TEI.rubric]{rubric} \hyperref[TEI.secFol]{secFol} \hyperref[TEI.signatures]{signatures} \hyperref[TEI.source]{source} \hyperref[TEI.stamp]{stamp} \hyperref[TEI.summary]{summary} \hyperref[TEI.support]{support} \hyperref[TEI.surrogates]{surrogates} \hyperref[TEI.typeNote]{typeNote} \hyperref[TEI.watermark]{watermark}\par 
    \item[namesdates: ]
   \hyperref[TEI.addName]{addName} \hyperref[TEI.affiliation]{affiliation} \hyperref[TEI.country]{country} \hyperref[TEI.forename]{forename} \hyperref[TEI.genName]{genName} \hyperref[TEI.geogName]{geogName} \hyperref[TEI.nameLink]{nameLink} \hyperref[TEI.orgName]{orgName} \hyperref[TEI.persName]{persName} \hyperref[TEI.person]{person} \hyperref[TEI.personGrp]{personGrp} \hyperref[TEI.persona]{persona} \hyperref[TEI.placeName]{placeName} \hyperref[TEI.region]{region} \hyperref[TEI.roleName]{roleName} \hyperref[TEI.settlement]{settlement} \hyperref[TEI.surname]{surname}\par 
    \item[spoken: ]
   \hyperref[TEI.annotationBlock]{annotationBlock}\par 
    \item[standOff: ]
   \hyperref[TEI.listAnnotation]{listAnnotation}\par 
    \item[textstructure: ]
   \hyperref[TEI.back]{back} \hyperref[TEI.body]{body} \hyperref[TEI.div]{div} \hyperref[TEI.docAuthor]{docAuthor} \hyperref[TEI.docDate]{docDate} \hyperref[TEI.docEdition]{docEdition} \hyperref[TEI.docTitle]{docTitle} \hyperref[TEI.floatingText]{floatingText} \hyperref[TEI.front]{front} \hyperref[TEI.group]{group} \hyperref[TEI.text]{text} \hyperref[TEI.titlePage]{titlePage} \hyperref[TEI.titlePart]{titlePart}\par 
    \item[transcr: ]
   \hyperref[TEI.damage]{damage} \hyperref[TEI.fw]{fw} \hyperref[TEI.line]{line} \hyperref[TEI.metamark]{metamark} \hyperref[TEI.mod]{mod} \hyperref[TEI.restore]{restore} \hyperref[TEI.retrace]{retrace} \hyperref[TEI.secl]{secl} \hyperref[TEI.sourceDoc]{sourceDoc} \hyperref[TEI.supplied]{supplied} \hyperref[TEI.surface]{surface} \hyperref[TEI.surfaceGrp]{surfaceGrp} \hyperref[TEI.surplus]{surplus} \hyperref[TEI.zone]{zone}
    \item[{Peut contenir}]
  
    \item[transcr: ]
   \hyperref[TEI.transpose]{transpose}
    \item[{Exemple}]
  \leavevmode\bgroup\exampleFont \begin{shaded}\noindent\mbox{}{<\textbf{listTranspose}>}\mbox{}\newline 
\hspace*{6pt}{<\textbf{transpose}>}\mbox{}\newline 
\hspace*{6pt}\hspace*{6pt}{<\textbf{ptr}\hspace*{6pt}{target}="{\#ib02}"/>}\mbox{}\newline 
\hspace*{6pt}\hspace*{6pt}{<\textbf{ptr}\hspace*{6pt}{target}="{\#ib01}"/>}\mbox{}\newline 
\hspace*{6pt}{</\textbf{transpose}>}\mbox{}\newline 
{</\textbf{listTranspose}>}\end{shaded}\egroup 

This example might be used for a source document which indicates in some way that the elements identified by \texttt{ib02} and code \texttt{ib01} should be read in that order (ib02 followed by ib01), rather than in the reading order in which they are presented in the source.
    \item[{Modèle de contenu}]
  \mbox{}\hfill\\[-10pt]\begin{Verbatim}[fontsize=\small]
<content>
 <elementRef key="transpose"
  maxOccurs="unbounded" minOccurs="1"/>
</content>
    
\end{Verbatim}

    \item[{Schéma Declaration}]
  \mbox{}\hfill\\[-10pt]\begin{Verbatim}[fontsize=\small]
element listTranspose { tei_att.global.attributes, tei_transpose+ }
\end{Verbatim}

\end{reflist}  \index{location=<location>|oddindex}
\begin{reflist}
\item[]\begin{specHead}{TEI.location}{<location> }(localisation) définit l'emplacement d'un lieu par des coordonnées géographiques, en termes d'entités nommées dites géopolitiques, ou par une adresse. [\xref{http://www.tei-c.org/release/doc/tei-p5-doc/en/html/ND.html\#NDGEOG}{13.3.4. Places}]\end{specHead} 
    \item[{Module}]
  namesdates
    \item[{Attributs}]
  Attributs \hyperref[TEI.att.global]{att.global} (\textit{@xml:id}, \textit{@n}, \textit{@xml:lang}, \textit{@xml:base}, \textit{@xml:space})  (\hyperref[TEI.att.global.rendition]{att.global.rendition} (\textit{@rend}, \textit{@style}, \textit{@rendition})) (\hyperref[TEI.att.global.linking]{att.global.linking} (\textit{@corresp}, \textit{@synch}, \textit{@sameAs}, \textit{@copyOf}, \textit{@next}, \textit{@prev}, \textit{@exclude}, \textit{@select})) (\hyperref[TEI.att.global.analytic]{att.global.analytic} (\textit{@ana})) (\hyperref[TEI.att.global.facs]{att.global.facs} (\textit{@facs})) (\hyperref[TEI.att.global.change]{att.global.change} (\textit{@change})) (\hyperref[TEI.att.global.responsibility]{att.global.responsibility} (\textit{@cert}, \textit{@resp})) (\hyperref[TEI.att.global.source]{att.global.source} (\textit{@source})) \hyperref[TEI.att.typed]{att.typed} (\textit{@type}, \textit{@subtype}) \hyperref[TEI.att.datable]{att.datable} (\textit{@calendar}, \textit{@period})  (\hyperref[TEI.att.datable.w3c]{att.datable.w3c} (\textit{@when}, \textit{@notBefore}, \textit{@notAfter}, \textit{@from}, \textit{@to})) (\hyperref[TEI.att.datable.iso]{att.datable.iso} (\textit{@when-iso}, \textit{@notBefore-iso}, \textit{@notAfter-iso}, \textit{@from-iso}, \textit{@to-iso})) (\hyperref[TEI.att.datable.custom]{att.datable.custom} (\textit{@when-custom}, \textit{@notBefore-custom}, \textit{@notAfter-custom}, \textit{@from-custom}, \textit{@to-custom}, \textit{@datingPoint}, \textit{@datingMethod})) \hyperref[TEI.att.editLike]{att.editLike} (\textit{@evidence}, \textit{@instant})  (\hyperref[TEI.att.dimensions]{att.dimensions} (\textit{@unit}, \textit{@quantity}, \textit{@extent}, \textit{@precision}, \textit{@scope}) (\hyperref[TEI.att.ranging]{att.ranging} (\textit{@atLeast}, \textit{@atMost}, \textit{@min}, \textit{@max}, \textit{@confidence})) )
    \item[{Membre du}]
  \hyperref[TEI.model.placeStateLike]{model.placeStateLike}Elément: \begin{itemize}
\item \hyperref[TEI.math]{math}/@location
\end{itemize} 
    \item[{Contenu dans}]
  
    \item[analysis: ]
   \hyperref[TEI.cl]{cl} \hyperref[TEI.phr]{phr} \hyperref[TEI.s]{s} \hyperref[TEI.span]{span}\par 
    \item[core: ]
   \hyperref[TEI.abbr]{abbr} \hyperref[TEI.add]{add} \hyperref[TEI.addrLine]{addrLine} \hyperref[TEI.address]{address} \hyperref[TEI.author]{author} \hyperref[TEI.bibl]{bibl} \hyperref[TEI.biblScope]{biblScope} \hyperref[TEI.citedRange]{citedRange} \hyperref[TEI.corr]{corr} \hyperref[TEI.date]{date} \hyperref[TEI.del]{del} \hyperref[TEI.desc]{desc} \hyperref[TEI.distinct]{distinct} \hyperref[TEI.editor]{editor} \hyperref[TEI.email]{email} \hyperref[TEI.emph]{emph} \hyperref[TEI.expan]{expan} \hyperref[TEI.foreign]{foreign} \hyperref[TEI.gloss]{gloss} \hyperref[TEI.head]{head} \hyperref[TEI.headItem]{headItem} \hyperref[TEI.headLabel]{headLabel} \hyperref[TEI.hi]{hi} \hyperref[TEI.item]{item} \hyperref[TEI.l]{l} \hyperref[TEI.label]{label} \hyperref[TEI.measure]{measure} \hyperref[TEI.meeting]{meeting} \hyperref[TEI.mentioned]{mentioned} \hyperref[TEI.name]{name} \hyperref[TEI.note]{note} \hyperref[TEI.num]{num} \hyperref[TEI.orig]{orig} \hyperref[TEI.p]{p} \hyperref[TEI.pubPlace]{pubPlace} \hyperref[TEI.publisher]{publisher} \hyperref[TEI.q]{q} \hyperref[TEI.quote]{quote} \hyperref[TEI.ref]{ref} \hyperref[TEI.reg]{reg} \hyperref[TEI.resp]{resp} \hyperref[TEI.rs]{rs} \hyperref[TEI.said]{said} \hyperref[TEI.sic]{sic} \hyperref[TEI.soCalled]{soCalled} \hyperref[TEI.speaker]{speaker} \hyperref[TEI.stage]{stage} \hyperref[TEI.street]{street} \hyperref[TEI.term]{term} \hyperref[TEI.textLang]{textLang} \hyperref[TEI.time]{time} \hyperref[TEI.title]{title} \hyperref[TEI.unclear]{unclear}\par 
    \item[figures: ]
   \hyperref[TEI.cell]{cell} \hyperref[TEI.figDesc]{figDesc}\par 
    \item[header: ]
   \hyperref[TEI.authority]{authority} \hyperref[TEI.change]{change} \hyperref[TEI.classCode]{classCode} \hyperref[TEI.creation]{creation} \hyperref[TEI.distributor]{distributor} \hyperref[TEI.edition]{edition} \hyperref[TEI.extent]{extent} \hyperref[TEI.funder]{funder} \hyperref[TEI.language]{language} \hyperref[TEI.licence]{licence} \hyperref[TEI.rendition]{rendition}\par 
    \item[iso-fs: ]
   \hyperref[TEI.fDescr]{fDescr} \hyperref[TEI.fsDescr]{fsDescr}\par 
    \item[linking: ]
   \hyperref[TEI.ab]{ab} \hyperref[TEI.seg]{seg}\par 
    \item[msdescription: ]
   \hyperref[TEI.accMat]{accMat} \hyperref[TEI.acquisition]{acquisition} \hyperref[TEI.additions]{additions} \hyperref[TEI.catchwords]{catchwords} \hyperref[TEI.collation]{collation} \hyperref[TEI.colophon]{colophon} \hyperref[TEI.condition]{condition} \hyperref[TEI.custEvent]{custEvent} \hyperref[TEI.decoNote]{decoNote} \hyperref[TEI.explicit]{explicit} \hyperref[TEI.filiation]{filiation} \hyperref[TEI.finalRubric]{finalRubric} \hyperref[TEI.foliation]{foliation} \hyperref[TEI.heraldry]{heraldry} \hyperref[TEI.incipit]{incipit} \hyperref[TEI.layout]{layout} \hyperref[TEI.material]{material} \hyperref[TEI.musicNotation]{musicNotation} \hyperref[TEI.objectType]{objectType} \hyperref[TEI.origDate]{origDate} \hyperref[TEI.origPlace]{origPlace} \hyperref[TEI.origin]{origin} \hyperref[TEI.provenance]{provenance} \hyperref[TEI.rubric]{rubric} \hyperref[TEI.secFol]{secFol} \hyperref[TEI.signatures]{signatures} \hyperref[TEI.source]{source} \hyperref[TEI.stamp]{stamp} \hyperref[TEI.summary]{summary} \hyperref[TEI.support]{support} \hyperref[TEI.surrogates]{surrogates} \hyperref[TEI.typeNote]{typeNote} \hyperref[TEI.watermark]{watermark}\par 
    \item[namesdates: ]
   \hyperref[TEI.addName]{addName} \hyperref[TEI.affiliation]{affiliation} \hyperref[TEI.country]{country} \hyperref[TEI.forename]{forename} \hyperref[TEI.genName]{genName} \hyperref[TEI.geogName]{geogName} \hyperref[TEI.nameLink]{nameLink} \hyperref[TEI.org]{org} \hyperref[TEI.orgName]{orgName} \hyperref[TEI.persName]{persName} \hyperref[TEI.place]{place} \hyperref[TEI.placeName]{placeName} \hyperref[TEI.region]{region} \hyperref[TEI.roleName]{roleName} \hyperref[TEI.settlement]{settlement} \hyperref[TEI.surname]{surname}\par 
    \item[spoken: ]
   \hyperref[TEI.annotationBlock]{annotationBlock}\par 
    \item[standOff: ]
   \hyperref[TEI.listAnnotation]{listAnnotation}\par 
    \item[textstructure: ]
   \hyperref[TEI.docAuthor]{docAuthor} \hyperref[TEI.docDate]{docDate} \hyperref[TEI.docEdition]{docEdition} \hyperref[TEI.titlePart]{titlePart}\par 
    \item[transcr: ]
   \hyperref[TEI.damage]{damage} \hyperref[TEI.fw]{fw} \hyperref[TEI.metamark]{metamark} \hyperref[TEI.mod]{mod} \hyperref[TEI.restore]{restore} \hyperref[TEI.retrace]{retrace} \hyperref[TEI.secl]{secl} \hyperref[TEI.supplied]{supplied} \hyperref[TEI.surplus]{surplus}
    \item[{Peut contenir}]
  
    \item[core: ]
   \hyperref[TEI.address]{address} \hyperref[TEI.bibl]{bibl} \hyperref[TEI.biblStruct]{biblStruct} \hyperref[TEI.desc]{desc} \hyperref[TEI.email]{email} \hyperref[TEI.label]{label} \hyperref[TEI.listBibl]{listBibl} \hyperref[TEI.measure]{measure} \hyperref[TEI.measureGrp]{measureGrp} \hyperref[TEI.note]{note} \hyperref[TEI.num]{num}\par 
    \item[header: ]
   \hyperref[TEI.biblFull]{biblFull}\par 
    \item[msdescription: ]
   \hyperref[TEI.depth]{depth} \hyperref[TEI.dim]{dim} \hyperref[TEI.height]{height} \hyperref[TEI.msDesc]{msDesc} \hyperref[TEI.width]{width}\par 
    \item[namesdates: ]
   \hyperref[TEI.affiliation]{affiliation} \hyperref[TEI.country]{country} \hyperref[TEI.geogName]{geogName} \hyperref[TEI.placeName]{placeName} \hyperref[TEI.region]{region} \hyperref[TEI.settlement]{settlement}
    \item[{Exemple}]
  \leavevmode\bgroup\exampleFont \begin{shaded}\noindent\mbox{}{<\textbf{place}>}\mbox{}\newline 
\hspace*{6pt}{<\textbf{placeName}>}Abbey Dore{</\textbf{placeName}>}\mbox{}\newline 
\hspace*{6pt}{<\textbf{location}>}\mbox{}\newline 
\hspace*{6pt}\hspace*{6pt}{<\textbf{geo}>}51.969604 -2.893146{</\textbf{geo}>}\mbox{}\newline 
\hspace*{6pt}{</\textbf{location}>}\mbox{}\newline 
{</\textbf{place}>}\end{shaded}\egroup 


    \item[{Exemple}]
  \leavevmode\bgroup\exampleFont \begin{shaded}\noindent\mbox{}{<\textbf{place}\hspace*{6pt}{type}="{building}"\hspace*{6pt}{xml:id}="{BGbuilding}">}\mbox{}\newline 
\hspace*{6pt}{<\textbf{placeName}>}Brasserie Georges{</\textbf{placeName}>}\mbox{}\newline 
\hspace*{6pt}{<\textbf{location}>}\mbox{}\newline 
\hspace*{6pt}\hspace*{6pt}{<\textbf{country}\hspace*{6pt}{key}="{FR}"/>}\mbox{}\newline 
\hspace*{6pt}\hspace*{6pt}{<\textbf{settlement}\hspace*{6pt}{type}="{city}">}Lyon{</\textbf{settlement}>}\mbox{}\newline 
\hspace*{6pt}\hspace*{6pt}{<\textbf{district}\hspace*{6pt}{type}="{arrondissement}">}IIème{</\textbf{district}>}\mbox{}\newline 
\hspace*{6pt}\hspace*{6pt}{<\textbf{district}\hspace*{6pt}{type}="{quartier}">}Perrache{</\textbf{district}>}\mbox{}\newline 
\hspace*{6pt}\hspace*{6pt}{<\textbf{placeName}\hspace*{6pt}{type}="{street}">}\mbox{}\newline 
\hspace*{6pt}\hspace*{6pt}\hspace*{6pt}{<\textbf{num}>}30{</\textbf{num}>}, Cours de Verdun{</\textbf{placeName}>}\mbox{}\newline 
\hspace*{6pt}{</\textbf{location}>}\mbox{}\newline 
{</\textbf{place}>}\end{shaded}\egroup 


    \item[{Exemple}]
  \leavevmode\bgroup\exampleFont \begin{shaded}\noindent\mbox{}{<\textbf{place}\hspace*{6pt}{type}="{imaginary}">}\mbox{}\newline 
\hspace*{6pt}{<\textbf{placeName}>}Atlantis{</\textbf{placeName}>}\mbox{}\newline 
\hspace*{6pt}{<\textbf{location}>}\mbox{}\newline 
\hspace*{6pt}\hspace*{6pt}{<\textbf{offset}>}beyond{</\textbf{offset}>}\mbox{}\newline 
\hspace*{6pt}\hspace*{6pt}{<\textbf{placeName}>}The Pillars of {<\textbf{persName}>}Hercules{</\textbf{persName}>}\mbox{}\newline 
\hspace*{6pt}\hspace*{6pt}{</\textbf{placeName}>}\mbox{}\newline 
\hspace*{6pt}{</\textbf{location}>}\mbox{}\newline 
{</\textbf{place}>}\end{shaded}\egroup 


    \item[{Modèle de contenu}]
  \mbox{}\hfill\\[-10pt]\begin{Verbatim}[fontsize=\small]
<content>
 <alternate maxOccurs="unbounded"
  minOccurs="0">
  <elementRef key="precision"/>
  <classRef key="model.labelLike"/>
  <classRef key="model.placeNamePart"/>
  <classRef key="model.offsetLike"/>
  <classRef key="model.measureLike"/>
  <classRef key="model.addressLike"/>
  <classRef key="model.noteLike"/>
  <classRef key="model.biblLike"/>
 </alternate>
</content>
    
\end{Verbatim}

    \item[{Schéma Declaration}]
  \mbox{}\hfill\\[-10pt]\begin{Verbatim}[fontsize=\small]
element location
{
   tei_att.global.attributes,
   tei_att.typed.attributes,
   tei_att.datable.attributes,
   tei_att.editLike.attributes,
   (
      precision    | tei_model.labelLike    | tei_model.placeNamePart    | tei_model.offsetLike    | tei_model.measureLike    | tei_model.addressLike    | tei_model.noteLike    | tei_model.biblLike   )*
}
\end{Verbatim}

\end{reflist}  \index{locus=<locus>|oddindex}\index{scheme=@scheme!<locus>|oddindex}\index{from=@from!<locus>|oddindex}\index{to=@to!<locus>|oddindex}
\begin{reflist}
\item[]\begin{specHead}{TEI.locus}{<locus> }(locus) définit un emplacement au sein d'un manuscrit ou d'une partie de manuscrit, souvent une séquence, éventuellement discontinue, de références de feuillets. [\xref{http://www.tei-c.org/release/doc/tei-p5-doc/en/html/MS.html\#msloc}{10.3.5. References to Locations within a Manuscript}]\end{specHead} 
    \item[{Module}]
  msdescription
    \item[{Attributs}]
  Attributs \hyperref[TEI.att.global]{att.global} (\textit{@xml:id}, \textit{@n}, \textit{@xml:lang}, \textit{@xml:base}, \textit{@xml:space})  (\hyperref[TEI.att.global.rendition]{att.global.rendition} (\textit{@rend}, \textit{@style}, \textit{@rendition})) (\hyperref[TEI.att.global.linking]{att.global.linking} (\textit{@corresp}, \textit{@synch}, \textit{@sameAs}, \textit{@copyOf}, \textit{@next}, \textit{@prev}, \textit{@exclude}, \textit{@select})) (\hyperref[TEI.att.global.analytic]{att.global.analytic} (\textit{@ana})) (\hyperref[TEI.att.global.facs]{att.global.facs} (\textit{@facs})) (\hyperref[TEI.att.global.change]{att.global.change} (\textit{@change})) (\hyperref[TEI.att.global.responsibility]{att.global.responsibility} (\textit{@cert}, \textit{@resp})) (\hyperref[TEI.att.global.source]{att.global.source} (\textit{@source})) \hyperref[TEI.att.pointing]{att.pointing} (\textit{@targetLang}, \textit{@target}, \textit{@evaluate}) \hyperref[TEI.att.typed]{att.typed} (\textit{@type}, \textit{@subtype}) \hfil\\[-10pt]\begin{sansreflist}
    \item[@scheme]
  (système) désigne le système de foliotation utilisé pour localiser la subdivision du manuscrit qui est en cours de description.
\begin{reflist}
    \item[{Statut}]
  Optionel
    \item[{Type de données}]
  \hyperref[TEI.teidata.pointer]{teidata.pointer}
\end{reflist}  
    \item[@from]
  (depuis) Spécifie, sous une forme normalisée, le point de départ de la localisation.
\begin{reflist}
    \item[{Statut}]
  Optionel
    \item[{Type de données}]
  \hyperref[TEI.teidata.word]{teidata.word}
\end{reflist}  
    \item[@to]
  (jusqu'à) Spécifie, sous une forme normalisée, la borne de fin pour la localisation.
\begin{reflist}
    \item[{Statut}]
  Optionel
    \item[{Type de données}]
  \hyperref[TEI.teidata.word]{teidata.word}
\end{reflist}  
\end{sansreflist}  
    \item[{Membre du}]
  \hyperref[TEI.model.pPart.msdesc]{model.pPart.msdesc} 
    \item[{Contenu dans}]
  
    \item[analysis: ]
   \hyperref[TEI.cl]{cl} \hyperref[TEI.phr]{phr} \hyperref[TEI.s]{s} \hyperref[TEI.span]{span}\par 
    \item[core: ]
   \hyperref[TEI.abbr]{abbr} \hyperref[TEI.add]{add} \hyperref[TEI.addrLine]{addrLine} \hyperref[TEI.author]{author} \hyperref[TEI.biblScope]{biblScope} \hyperref[TEI.citedRange]{citedRange} \hyperref[TEI.corr]{corr} \hyperref[TEI.date]{date} \hyperref[TEI.del]{del} \hyperref[TEI.desc]{desc} \hyperref[TEI.distinct]{distinct} \hyperref[TEI.editor]{editor} \hyperref[TEI.email]{email} \hyperref[TEI.emph]{emph} \hyperref[TEI.expan]{expan} \hyperref[TEI.foreign]{foreign} \hyperref[TEI.gloss]{gloss} \hyperref[TEI.head]{head} \hyperref[TEI.headItem]{headItem} \hyperref[TEI.headLabel]{headLabel} \hyperref[TEI.hi]{hi} \hyperref[TEI.item]{item} \hyperref[TEI.l]{l} \hyperref[TEI.label]{label} \hyperref[TEI.measure]{measure} \hyperref[TEI.meeting]{meeting} \hyperref[TEI.mentioned]{mentioned} \hyperref[TEI.name]{name} \hyperref[TEI.note]{note} \hyperref[TEI.num]{num} \hyperref[TEI.orig]{orig} \hyperref[TEI.p]{p} \hyperref[TEI.pubPlace]{pubPlace} \hyperref[TEI.publisher]{publisher} \hyperref[TEI.q]{q} \hyperref[TEI.quote]{quote} \hyperref[TEI.ref]{ref} \hyperref[TEI.reg]{reg} \hyperref[TEI.resp]{resp} \hyperref[TEI.rs]{rs} \hyperref[TEI.said]{said} \hyperref[TEI.sic]{sic} \hyperref[TEI.soCalled]{soCalled} \hyperref[TEI.speaker]{speaker} \hyperref[TEI.stage]{stage} \hyperref[TEI.street]{street} \hyperref[TEI.term]{term} \hyperref[TEI.textLang]{textLang} \hyperref[TEI.time]{time} \hyperref[TEI.title]{title} \hyperref[TEI.unclear]{unclear}\par 
    \item[figures: ]
   \hyperref[TEI.cell]{cell} \hyperref[TEI.figDesc]{figDesc}\par 
    \item[header: ]
   \hyperref[TEI.authority]{authority} \hyperref[TEI.change]{change} \hyperref[TEI.classCode]{classCode} \hyperref[TEI.creation]{creation} \hyperref[TEI.distributor]{distributor} \hyperref[TEI.edition]{edition} \hyperref[TEI.extent]{extent} \hyperref[TEI.funder]{funder} \hyperref[TEI.language]{language} \hyperref[TEI.licence]{licence} \hyperref[TEI.rendition]{rendition}\par 
    \item[iso-fs: ]
   \hyperref[TEI.fDescr]{fDescr} \hyperref[TEI.fsDescr]{fsDescr}\par 
    \item[linking: ]
   \hyperref[TEI.ab]{ab} \hyperref[TEI.seg]{seg}\par 
    \item[msdescription: ]
   \hyperref[TEI.accMat]{accMat} \hyperref[TEI.acquisition]{acquisition} \hyperref[TEI.additions]{additions} \hyperref[TEI.catchwords]{catchwords} \hyperref[TEI.collation]{collation} \hyperref[TEI.colophon]{colophon} \hyperref[TEI.condition]{condition} \hyperref[TEI.custEvent]{custEvent} \hyperref[TEI.decoNote]{decoNote} \hyperref[TEI.explicit]{explicit} \hyperref[TEI.filiation]{filiation} \hyperref[TEI.finalRubric]{finalRubric} \hyperref[TEI.foliation]{foliation} \hyperref[TEI.heraldry]{heraldry} \hyperref[TEI.incipit]{incipit} \hyperref[TEI.layout]{layout} \hyperref[TEI.locus]{locus} \hyperref[TEI.locusGrp]{locusGrp} \hyperref[TEI.material]{material} \hyperref[TEI.msItem]{msItem} \hyperref[TEI.msItemStruct]{msItemStruct} \hyperref[TEI.musicNotation]{musicNotation} \hyperref[TEI.objectType]{objectType} \hyperref[TEI.origDate]{origDate} \hyperref[TEI.origPlace]{origPlace} \hyperref[TEI.origin]{origin} \hyperref[TEI.provenance]{provenance} \hyperref[TEI.rubric]{rubric} \hyperref[TEI.secFol]{secFol} \hyperref[TEI.signatures]{signatures} \hyperref[TEI.source]{source} \hyperref[TEI.stamp]{stamp} \hyperref[TEI.summary]{summary} \hyperref[TEI.support]{support} \hyperref[TEI.surrogates]{surrogates} \hyperref[TEI.typeNote]{typeNote} \hyperref[TEI.watermark]{watermark}\par 
    \item[namesdates: ]
   \hyperref[TEI.addName]{addName} \hyperref[TEI.affiliation]{affiliation} \hyperref[TEI.country]{country} \hyperref[TEI.forename]{forename} \hyperref[TEI.genName]{genName} \hyperref[TEI.geogName]{geogName} \hyperref[TEI.nameLink]{nameLink} \hyperref[TEI.orgName]{orgName} \hyperref[TEI.persName]{persName} \hyperref[TEI.placeName]{placeName} \hyperref[TEI.region]{region} \hyperref[TEI.roleName]{roleName} \hyperref[TEI.settlement]{settlement} \hyperref[TEI.surname]{surname}\par 
    \item[textstructure: ]
   \hyperref[TEI.docAuthor]{docAuthor} \hyperref[TEI.docDate]{docDate} \hyperref[TEI.docEdition]{docEdition} \hyperref[TEI.titlePart]{titlePart}\par 
    \item[transcr: ]
   \hyperref[TEI.damage]{damage} \hyperref[TEI.fw]{fw} \hyperref[TEI.metamark]{metamark} \hyperref[TEI.mod]{mod} \hyperref[TEI.restore]{restore} \hyperref[TEI.retrace]{retrace} \hyperref[TEI.secl]{secl} \hyperref[TEI.supplied]{supplied} \hyperref[TEI.surplus]{surplus}
    \item[{Peut contenir}]
  
    \item[core: ]
   \hyperref[TEI.hi]{hi}\par 
    \item[msdescription: ]
   \hyperref[TEI.locus]{locus}\par des données textuelles
    \item[{Note}]
  \par
L'attribut {\itshape target} doit être utilisé uniquement pour pointer vers des éléments contenant ou référençant une transcription de la partie du manuscrit ainsi localisée, comme dans le premier exemple ci-dessus. Pour associer un élément \hyperref[TEI.locus]{<locus>} avec l'image d'une page ou avec une autre représentation similaire, on doit utiliser l'attribut global {\itshape facs}, comme le montre le deuxième exemple. L'attribut {\itshape target} est déprécié pour établir un lien vers une image. On utilise l'attribut {\itshape facs}, soit pour établir un lien vers un ou plusieurs fichiers image, comme ci-dessus, soit pour pointer vers un ou plusieurs éléments dédiés, tels que \hyperref[TEI.surface]{<surface>}, \hyperref[TEI.zone]{<zone>}, \hyperref[TEI.graphic]{<graphic>} ou \hyperref[TEI.binaryObject]{<binaryObject>}.
    \item[{Exemple}]
  \leavevmode\bgroup\exampleFont \begin{shaded}\noindent\mbox{}{<\textbf{msItem}\hspace*{6pt}{n}="{1}">}\mbox{}\newline 
\hspace*{6pt}{<\textbf{locus}\hspace*{6pt}{target}="{\#fr\textunderscore F1r \#fr\textunderscore F1v \#fr\textunderscore F2r}">}ff. 1r-2r{</\textbf{locus}>}\mbox{}\newline 
\hspace*{6pt}{<\textbf{author}>}Ben Jonson{</\textbf{author}>}\mbox{}\newline 
\hspace*{6pt}{<\textbf{title}>}Ode to himself{</\textbf{title}>}\mbox{}\newline 
\hspace*{6pt}{<\textbf{rubric}\hspace*{6pt}{rend}="{italics}">} An Ode{<\textbf{lb}/>} to him selfe.{</\textbf{rubric}>}\mbox{}\newline 
\hspace*{6pt}{<\textbf{incipit}>}Com leaue the loathed stage{</\textbf{incipit}>}\mbox{}\newline 
\hspace*{6pt}{<\textbf{explicit}>}And see his chariot triumph ore his wayne.{</\textbf{explicit}>}\mbox{}\newline 
\hspace*{6pt}{<\textbf{bibl}>}\mbox{}\newline 
\hspace*{6pt}\hspace*{6pt}{<\textbf{name}>}Beal{</\textbf{name}>}, {<\textbf{title}>}Index 1450-1625{</\textbf{title}>}, JnB 380{</\textbf{bibl}>}\mbox{}\newline 
{</\textbf{msItem}>}\mbox{}\newline 
{<\textbf{pb}\hspace*{6pt}{xml:id}="{fr\textunderscore F1r}"/>}\mbox{}\newline 
{<\textbf{pb}\hspace*{6pt}{xml:id}="{fr\textunderscore F1v}"/>}\mbox{}\newline 
{<\textbf{pb}\hspace*{6pt}{xml:id}="{fr\textunderscore F2r}"/>}\end{shaded}\egroup 


    \item[{Exemple}]
  The {\itshape facs} attribute is available globally when the \textsf{transcr} module is included in a schema. It may be used to point directly to an image file, as in the following example:\leavevmode\bgroup\exampleFont \begin{shaded}\noindent\mbox{}{<\textbf{msItem}>}\mbox{}\newline 
\hspace*{6pt}{<\textbf{locus}\hspace*{6pt}{facs}="{images/08v.jpg images/09r.jpg images/09v.jpg images/10r.jpg images/10v.jpg}">}fols. 8v-10v{</\textbf{locus}>}\mbox{}\newline 
\hspace*{6pt}{<\textbf{title}>}Birds Praise of Love{</\textbf{title}>}\mbox{}\newline 
\hspace*{6pt}{<\textbf{bibl}>}\mbox{}\newline 
\hspace*{6pt}\hspace*{6pt}{<\textbf{title}>}IMEV{</\textbf{title}>}\mbox{}\newline 
\hspace*{6pt}\hspace*{6pt}{<\textbf{biblScope}>}1506{</\textbf{biblScope}>}\mbox{}\newline 
\hspace*{6pt}{</\textbf{bibl}>}\mbox{}\newline 
{</\textbf{msItem}>}\end{shaded}\egroup 


    \item[{Modèle de contenu}]
  \mbox{}\hfill\\[-10pt]\begin{Verbatim}[fontsize=\small]
<content>
 <alternate maxOccurs="unbounded"
  minOccurs="0">
  <textNode/>
  <classRef key="model.gLike"/>
  <elementRef key="hi"/>
  <elementRef key="locus"/>
 </alternate>
</content>
    
\end{Verbatim}

    \item[{Schéma Declaration}]
  \mbox{}\hfill\\[-10pt]\begin{Verbatim}[fontsize=\small]
element locus
{
   tei_att.global.attributes,
   tei_att.pointing.attributes,
   tei_att.typed.attributes,
   attribute scheme { text }?,
   attribute from { text }?,
   attribute to { text }?,
   ( text | tei_model.gLike | tei_hi | tei_locus )*
}
\end{Verbatim}

\end{reflist}  \index{locusGrp=<locusGrp>|oddindex}\index{scheme=@scheme!<locusGrp>|oddindex}
\begin{reflist}
\item[]\begin{specHead}{TEI.locusGrp}{<locusGrp> }(groupe d'emplacements) regroupe un certain nombre d'emplacements qui forment ensemble un item identifiable bien que discontinu dans un manuscrit ou une partie de manuscrit selon une foliotation spécifique. [\xref{http://www.tei-c.org/release/doc/tei-p5-doc/en/html/MS.html\#msloc}{10.3.5. References to Locations within a Manuscript}]\end{specHead} 
    \item[{Module}]
  msdescription
    \item[{Attributs}]
  Attributs \hyperref[TEI.att.global]{att.global} (\textit{@xml:id}, \textit{@n}, \textit{@xml:lang}, \textit{@xml:base}, \textit{@xml:space})  (\hyperref[TEI.att.global.rendition]{att.global.rendition} (\textit{@rend}, \textit{@style}, \textit{@rendition})) (\hyperref[TEI.att.global.linking]{att.global.linking} (\textit{@corresp}, \textit{@synch}, \textit{@sameAs}, \textit{@copyOf}, \textit{@next}, \textit{@prev}, \textit{@exclude}, \textit{@select})) (\hyperref[TEI.att.global.analytic]{att.global.analytic} (\textit{@ana})) (\hyperref[TEI.att.global.facs]{att.global.facs} (\textit{@facs})) (\hyperref[TEI.att.global.change]{att.global.change} (\textit{@change})) (\hyperref[TEI.att.global.responsibility]{att.global.responsibility} (\textit{@cert}, \textit{@resp})) (\hyperref[TEI.att.global.source]{att.global.source} (\textit{@source})) \hfil\\[-10pt]\begin{sansreflist}
    \item[@scheme]
  (système) désigne le système de foliotation selon lequel les emplacements contenus dans le groupe sont définis.
\begin{reflist}
    \item[{Statut}]
  Optionel
    \item[{Type de données}]
  \hyperref[TEI.teidata.pointer]{teidata.pointer}
\end{reflist}  
\end{sansreflist}  
    \item[{Membre du}]
  \hyperref[TEI.model.pPart.msdesc]{model.pPart.msdesc} 
    \item[{Contenu dans}]
  
    \item[analysis: ]
   \hyperref[TEI.cl]{cl} \hyperref[TEI.phr]{phr} \hyperref[TEI.s]{s} \hyperref[TEI.span]{span}\par 
    \item[core: ]
   \hyperref[TEI.abbr]{abbr} \hyperref[TEI.add]{add} \hyperref[TEI.addrLine]{addrLine} \hyperref[TEI.author]{author} \hyperref[TEI.biblScope]{biblScope} \hyperref[TEI.citedRange]{citedRange} \hyperref[TEI.corr]{corr} \hyperref[TEI.date]{date} \hyperref[TEI.del]{del} \hyperref[TEI.desc]{desc} \hyperref[TEI.distinct]{distinct} \hyperref[TEI.editor]{editor} \hyperref[TEI.email]{email} \hyperref[TEI.emph]{emph} \hyperref[TEI.expan]{expan} \hyperref[TEI.foreign]{foreign} \hyperref[TEI.gloss]{gloss} \hyperref[TEI.head]{head} \hyperref[TEI.headItem]{headItem} \hyperref[TEI.headLabel]{headLabel} \hyperref[TEI.hi]{hi} \hyperref[TEI.item]{item} \hyperref[TEI.l]{l} \hyperref[TEI.label]{label} \hyperref[TEI.measure]{measure} \hyperref[TEI.meeting]{meeting} \hyperref[TEI.mentioned]{mentioned} \hyperref[TEI.name]{name} \hyperref[TEI.note]{note} \hyperref[TEI.num]{num} \hyperref[TEI.orig]{orig} \hyperref[TEI.p]{p} \hyperref[TEI.pubPlace]{pubPlace} \hyperref[TEI.publisher]{publisher} \hyperref[TEI.q]{q} \hyperref[TEI.quote]{quote} \hyperref[TEI.ref]{ref} \hyperref[TEI.reg]{reg} \hyperref[TEI.resp]{resp} \hyperref[TEI.rs]{rs} \hyperref[TEI.said]{said} \hyperref[TEI.sic]{sic} \hyperref[TEI.soCalled]{soCalled} \hyperref[TEI.speaker]{speaker} \hyperref[TEI.stage]{stage} \hyperref[TEI.street]{street} \hyperref[TEI.term]{term} \hyperref[TEI.textLang]{textLang} \hyperref[TEI.time]{time} \hyperref[TEI.title]{title} \hyperref[TEI.unclear]{unclear}\par 
    \item[figures: ]
   \hyperref[TEI.cell]{cell} \hyperref[TEI.figDesc]{figDesc}\par 
    \item[header: ]
   \hyperref[TEI.authority]{authority} \hyperref[TEI.change]{change} \hyperref[TEI.classCode]{classCode} \hyperref[TEI.creation]{creation} \hyperref[TEI.distributor]{distributor} \hyperref[TEI.edition]{edition} \hyperref[TEI.extent]{extent} \hyperref[TEI.funder]{funder} \hyperref[TEI.language]{language} \hyperref[TEI.licence]{licence} \hyperref[TEI.rendition]{rendition}\par 
    \item[iso-fs: ]
   \hyperref[TEI.fDescr]{fDescr} \hyperref[TEI.fsDescr]{fsDescr}\par 
    \item[linking: ]
   \hyperref[TEI.ab]{ab} \hyperref[TEI.seg]{seg}\par 
    \item[msdescription: ]
   \hyperref[TEI.accMat]{accMat} \hyperref[TEI.acquisition]{acquisition} \hyperref[TEI.additions]{additions} \hyperref[TEI.catchwords]{catchwords} \hyperref[TEI.collation]{collation} \hyperref[TEI.colophon]{colophon} \hyperref[TEI.condition]{condition} \hyperref[TEI.custEvent]{custEvent} \hyperref[TEI.decoNote]{decoNote} \hyperref[TEI.explicit]{explicit} \hyperref[TEI.filiation]{filiation} \hyperref[TEI.finalRubric]{finalRubric} \hyperref[TEI.foliation]{foliation} \hyperref[TEI.heraldry]{heraldry} \hyperref[TEI.incipit]{incipit} \hyperref[TEI.layout]{layout} \hyperref[TEI.material]{material} \hyperref[TEI.msItem]{msItem} \hyperref[TEI.msItemStruct]{msItemStruct} \hyperref[TEI.musicNotation]{musicNotation} \hyperref[TEI.objectType]{objectType} \hyperref[TEI.origDate]{origDate} \hyperref[TEI.origPlace]{origPlace} \hyperref[TEI.origin]{origin} \hyperref[TEI.provenance]{provenance} \hyperref[TEI.rubric]{rubric} \hyperref[TEI.secFol]{secFol} \hyperref[TEI.signatures]{signatures} \hyperref[TEI.source]{source} \hyperref[TEI.stamp]{stamp} \hyperref[TEI.summary]{summary} \hyperref[TEI.support]{support} \hyperref[TEI.surrogates]{surrogates} \hyperref[TEI.typeNote]{typeNote} \hyperref[TEI.watermark]{watermark}\par 
    \item[namesdates: ]
   \hyperref[TEI.addName]{addName} \hyperref[TEI.affiliation]{affiliation} \hyperref[TEI.country]{country} \hyperref[TEI.forename]{forename} \hyperref[TEI.genName]{genName} \hyperref[TEI.geogName]{geogName} \hyperref[TEI.nameLink]{nameLink} \hyperref[TEI.orgName]{orgName} \hyperref[TEI.persName]{persName} \hyperref[TEI.placeName]{placeName} \hyperref[TEI.region]{region} \hyperref[TEI.roleName]{roleName} \hyperref[TEI.settlement]{settlement} \hyperref[TEI.surname]{surname}\par 
    \item[textstructure: ]
   \hyperref[TEI.docAuthor]{docAuthor} \hyperref[TEI.docDate]{docDate} \hyperref[TEI.docEdition]{docEdition} \hyperref[TEI.titlePart]{titlePart}\par 
    \item[transcr: ]
   \hyperref[TEI.damage]{damage} \hyperref[TEI.fw]{fw} \hyperref[TEI.metamark]{metamark} \hyperref[TEI.mod]{mod} \hyperref[TEI.restore]{restore} \hyperref[TEI.retrace]{retrace} \hyperref[TEI.secl]{secl} \hyperref[TEI.supplied]{supplied} \hyperref[TEI.surplus]{surplus}
    \item[{Peut contenir}]
  
    \item[msdescription: ]
   \hyperref[TEI.locus]{locus}
    \item[{Exemple}]
  \leavevmode\bgroup\exampleFont \begin{shaded}\noindent\mbox{}{<\textbf{msItem}>}\mbox{}\newline 
\hspace*{6pt}{<\textbf{locusGrp}>}\mbox{}\newline 
\hspace*{6pt}\hspace*{6pt}{<\textbf{locus}\hspace*{6pt}{from}="{13}"\hspace*{6pt}{to}="{26}">}Bl. 13--26{</\textbf{locus}>}\mbox{}\newline 
\hspace*{6pt}\hspace*{6pt}{<\textbf{locus}\hspace*{6pt}{from}="{37}"\hspace*{6pt}{to}="{58}">}37--58{</\textbf{locus}>}\mbox{}\newline 
\hspace*{6pt}\hspace*{6pt}{<\textbf{locus}\hspace*{6pt}{from}="{82}"\hspace*{6pt}{to}="{96}">}82--96{</\textbf{locus}>}\mbox{}\newline 
\hspace*{6pt}{</\textbf{locusGrp}>}\mbox{}\newline 
\hspace*{6pt}{<\textbf{note}>}Stücke von Daniel Ecklin’s Reise ins h. Land{</\textbf{note}>}\mbox{}\newline 
{</\textbf{msItem}>}\end{shaded}\egroup 


    \item[{Modèle de contenu}]
  \mbox{}\hfill\\[-10pt]\begin{Verbatim}[fontsize=\small]
<content>
 <elementRef key="locus"
  maxOccurs="unbounded" minOccurs="1"/>
</content>
    
\end{Verbatim}

    \item[{Schéma Declaration}]
  \mbox{}\hfill\\[-10pt]\begin{Verbatim}[fontsize=\small]
element locusGrp
{
   tei_att.global.attributes,
   attribute scheme { text }?,
   tei_locus+
}
\end{Verbatim}

\end{reflist}  \index{m=<m>|oddindex}\index{baseForm=@baseForm!<m>|oddindex}
\begin{reflist}
\item[]\begin{specHead}{TEI.m}{<m> }(morphème) représente un morphème grammatical [\xref{http://www.tei-c.org/release/doc/tei-p5-doc/en/html/AI.html\#AILC}{17.1. Linguistic Segment Categories}]\end{specHead} 
    \item[{Module}]
  analysis
    \item[{Attributs}]
  Attributs \hyperref[TEI.att.global]{att.global} (\textit{@xml:id}, \textit{@n}, \textit{@xml:lang}, \textit{@xml:base}, \textit{@xml:space})  (\hyperref[TEI.att.global.rendition]{att.global.rendition} (\textit{@rend}, \textit{@style}, \textit{@rendition})) (\hyperref[TEI.att.global.linking]{att.global.linking} (\textit{@corresp}, \textit{@synch}, \textit{@sameAs}, \textit{@copyOf}, \textit{@next}, \textit{@prev}, \textit{@exclude}, \textit{@select})) (\hyperref[TEI.att.global.analytic]{att.global.analytic} (\textit{@ana})) (\hyperref[TEI.att.global.facs]{att.global.facs} (\textit{@facs})) (\hyperref[TEI.att.global.change]{att.global.change} (\textit{@change})) (\hyperref[TEI.att.global.responsibility]{att.global.responsibility} (\textit{@cert}, \textit{@resp})) (\hyperref[TEI.att.global.source]{att.global.source} (\textit{@source})) \hyperref[TEI.att.segLike]{att.segLike} (\textit{@function})  (\hyperref[TEI.att.datcat]{att.datcat} (\textit{@datcat}, \textit{@valueDatcat})) (\hyperref[TEI.att.fragmentable]{att.fragmentable} (\textit{@part})) \hyperref[TEI.att.typed]{att.typed} (\textit{@type}, \textit{@subtype}) \hfil\\[-10pt]\begin{sansreflist}
    \item[@baseForm]
  identifie la forme de base du morphème
\begin{reflist}
    \item[{Statut}]
  Optionel
    \item[{Type de données}]
  \hyperref[TEI.teidata.word]{teidata.word}
\end{reflist}  
\end{sansreflist}  
    \item[{Membre du}]
  \hyperref[TEI.model.segLike]{model.segLike} 
    \item[{Contenu dans}]
  
    \item[analysis: ]
   \hyperref[TEI.cl]{cl} \hyperref[TEI.m]{m} \hyperref[TEI.phr]{phr} \hyperref[TEI.s]{s} \hyperref[TEI.w]{w}\par 
    \item[core: ]
   \hyperref[TEI.abbr]{abbr} \hyperref[TEI.add]{add} \hyperref[TEI.addrLine]{addrLine} \hyperref[TEI.author]{author} \hyperref[TEI.bibl]{bibl} \hyperref[TEI.biblScope]{biblScope} \hyperref[TEI.citedRange]{citedRange} \hyperref[TEI.corr]{corr} \hyperref[TEI.date]{date} \hyperref[TEI.del]{del} \hyperref[TEI.distinct]{distinct} \hyperref[TEI.editor]{editor} \hyperref[TEI.email]{email} \hyperref[TEI.emph]{emph} \hyperref[TEI.expan]{expan} \hyperref[TEI.foreign]{foreign} \hyperref[TEI.gloss]{gloss} \hyperref[TEI.head]{head} \hyperref[TEI.headItem]{headItem} \hyperref[TEI.headLabel]{headLabel} \hyperref[TEI.hi]{hi} \hyperref[TEI.item]{item} \hyperref[TEI.l]{l} \hyperref[TEI.label]{label} \hyperref[TEI.measure]{measure} \hyperref[TEI.mentioned]{mentioned} \hyperref[TEI.name]{name} \hyperref[TEI.note]{note} \hyperref[TEI.num]{num} \hyperref[TEI.orig]{orig} \hyperref[TEI.p]{p} \hyperref[TEI.pubPlace]{pubPlace} \hyperref[TEI.publisher]{publisher} \hyperref[TEI.q]{q} \hyperref[TEI.quote]{quote} \hyperref[TEI.ref]{ref} \hyperref[TEI.reg]{reg} \hyperref[TEI.rs]{rs} \hyperref[TEI.said]{said} \hyperref[TEI.sic]{sic} \hyperref[TEI.soCalled]{soCalled} \hyperref[TEI.speaker]{speaker} \hyperref[TEI.stage]{stage} \hyperref[TEI.street]{street} \hyperref[TEI.term]{term} \hyperref[TEI.textLang]{textLang} \hyperref[TEI.time]{time} \hyperref[TEI.title]{title} \hyperref[TEI.unclear]{unclear}\par 
    \item[figures: ]
   \hyperref[TEI.cell]{cell}\par 
    \item[header: ]
   \hyperref[TEI.change]{change} \hyperref[TEI.distributor]{distributor} \hyperref[TEI.edition]{edition} \hyperref[TEI.extent]{extent} \hyperref[TEI.licence]{licence}\par 
    \item[linking: ]
   \hyperref[TEI.ab]{ab} \hyperref[TEI.seg]{seg}\par 
    \item[msdescription: ]
   \hyperref[TEI.accMat]{accMat} \hyperref[TEI.acquisition]{acquisition} \hyperref[TEI.additions]{additions} \hyperref[TEI.catchwords]{catchwords} \hyperref[TEI.collation]{collation} \hyperref[TEI.colophon]{colophon} \hyperref[TEI.condition]{condition} \hyperref[TEI.custEvent]{custEvent} \hyperref[TEI.decoNote]{decoNote} \hyperref[TEI.explicit]{explicit} \hyperref[TEI.filiation]{filiation} \hyperref[TEI.finalRubric]{finalRubric} \hyperref[TEI.foliation]{foliation} \hyperref[TEI.heraldry]{heraldry} \hyperref[TEI.incipit]{incipit} \hyperref[TEI.layout]{layout} \hyperref[TEI.material]{material} \hyperref[TEI.musicNotation]{musicNotation} \hyperref[TEI.objectType]{objectType} \hyperref[TEI.origDate]{origDate} \hyperref[TEI.origPlace]{origPlace} \hyperref[TEI.origin]{origin} \hyperref[TEI.provenance]{provenance} \hyperref[TEI.rubric]{rubric} \hyperref[TEI.secFol]{secFol} \hyperref[TEI.signatures]{signatures} \hyperref[TEI.source]{source} \hyperref[TEI.stamp]{stamp} \hyperref[TEI.summary]{summary} \hyperref[TEI.support]{support} \hyperref[TEI.surrogates]{surrogates} \hyperref[TEI.typeNote]{typeNote} \hyperref[TEI.watermark]{watermark}\par 
    \item[namesdates: ]
   \hyperref[TEI.addName]{addName} \hyperref[TEI.affiliation]{affiliation} \hyperref[TEI.country]{country} \hyperref[TEI.forename]{forename} \hyperref[TEI.genName]{genName} \hyperref[TEI.geogName]{geogName} \hyperref[TEI.nameLink]{nameLink} \hyperref[TEI.orgName]{orgName} \hyperref[TEI.persName]{persName} \hyperref[TEI.placeName]{placeName} \hyperref[TEI.region]{region} \hyperref[TEI.roleName]{roleName} \hyperref[TEI.settlement]{settlement} \hyperref[TEI.surname]{surname}\par 
    \item[textstructure: ]
   \hyperref[TEI.docAuthor]{docAuthor} \hyperref[TEI.docDate]{docDate} \hyperref[TEI.docEdition]{docEdition} \hyperref[TEI.titlePart]{titlePart}\par 
    \item[transcr: ]
   \hyperref[TEI.damage]{damage} \hyperref[TEI.fw]{fw} \hyperref[TEI.metamark]{metamark} \hyperref[TEI.mod]{mod} \hyperref[TEI.restore]{restore} \hyperref[TEI.retrace]{retrace} \hyperref[TEI.secl]{secl} \hyperref[TEI.supplied]{supplied} \hyperref[TEI.surplus]{surplus}
    \item[{Peut contenir}]
  
    \item[analysis: ]
   \hyperref[TEI.c]{c} \hyperref[TEI.interp]{interp} \hyperref[TEI.interpGrp]{interpGrp} \hyperref[TEI.m]{m} \hyperref[TEI.span]{span} \hyperref[TEI.spanGrp]{spanGrp}\par 
    \item[core: ]
   \hyperref[TEI.cb]{cb} \hyperref[TEI.gap]{gap} \hyperref[TEI.gb]{gb} \hyperref[TEI.hi]{hi} \hyperref[TEI.index]{index} \hyperref[TEI.lb]{lb} \hyperref[TEI.milestone]{milestone} \hyperref[TEI.note]{note} \hyperref[TEI.pb]{pb}\par 
    \item[figures: ]
   \hyperref[TEI.figure]{figure} \hyperref[TEI.notatedMusic]{notatedMusic}\par 
    \item[iso-fs: ]
   \hyperref[TEI.fLib]{fLib} \hyperref[TEI.fs]{fs} \hyperref[TEI.fvLib]{fvLib}\par 
    \item[linking: ]
   \hyperref[TEI.alt]{alt} \hyperref[TEI.altGrp]{altGrp} \hyperref[TEI.anchor]{anchor} \hyperref[TEI.join]{join} \hyperref[TEI.joinGrp]{joinGrp} \hyperref[TEI.link]{link} \hyperref[TEI.linkGrp]{linkGrp} \hyperref[TEI.seg]{seg} \hyperref[TEI.timeline]{timeline}\par 
    \item[msdescription: ]
   \hyperref[TEI.source]{source}\par 
    \item[transcr: ]
   \hyperref[TEI.addSpan]{addSpan} \hyperref[TEI.damageSpan]{damageSpan} \hyperref[TEI.delSpan]{delSpan} \hyperref[TEI.fw]{fw} \hyperref[TEI.listTranspose]{listTranspose} \hyperref[TEI.metamark]{metamark} \hyperref[TEI.space]{space} \hyperref[TEI.substJoin]{substJoin}\par des données textuelles
    \item[{Note}]
  \par
L'attribut {\itshape type} peut être utilisé pour préciser le type de morphème, avec des valeurs telles que clitique, préfixe, stemma, etc. selon le cas.
    \item[{Exemple}]
  \leavevmode\bgroup\exampleFont \begin{shaded}\noindent\mbox{}{<\textbf{w}\hspace*{6pt}{type}="{adjective}">}\mbox{}\newline 
\hspace*{6pt}{<\textbf{w}\hspace*{6pt}{type}="{noun}">}\mbox{}\newline 
\hspace*{6pt}\hspace*{6pt}{<\textbf{m}\hspace*{6pt}{baseForm}="{con}"\hspace*{6pt}{type}="{prefix}">}con{</\textbf{m}>}\mbox{}\newline 
\hspace*{6pt}\hspace*{6pt}{<\textbf{m}\hspace*{6pt}{type}="{root}">}fort{</\textbf{m}>}\mbox{}\newline 
\hspace*{6pt}{</\textbf{w}>}\mbox{}\newline 
\hspace*{6pt}{<\textbf{m}\hspace*{6pt}{type}="{suffix}">}able{</\textbf{m}>}\mbox{}\newline 
{</\textbf{w}>}\end{shaded}\egroup 


    \item[{Modèle de contenu}]
  \mbox{}\hfill\\[-10pt]\begin{Verbatim}[fontsize=\small]
<content>
 <alternate maxOccurs="unbounded"
  minOccurs="0">
  <textNode/>
  <classRef key="model.gLike"/>
  <classRef key="model.hiLike"/>
  <elementRef key="seg"/>
  <elementRef key="m"/>
  <elementRef key="c"/>
  <classRef key="model.global"/>
 </alternate>
</content>
    
\end{Verbatim}

    \item[{Schéma Declaration}]
  \mbox{}\hfill\\[-10pt]\begin{Verbatim}[fontsize=\small]
element m
{
   tei_att.global.attributes,
   tei_att.segLike.attributes,
   tei_att.typed.attributes,
   attribute baseForm { text }?,
   (
      text
    | tei_model.gLike    | tei_model.hiLike    | tei_seg    | tei_m    | tei_c    | tei_model.global   )*
}
\end{Verbatim}

\end{reflist}  \index{material=<material>|oddindex}
\begin{reflist}
\item[]\begin{specHead}{TEI.material}{<material> }(matériau) Contient un mot ou une expression décrivant le ou les matériau(x) utilisé(s) pour fabriquer un manuscrit (ou une partie d'un manuscrit). [\xref{http://www.tei-c.org/release/doc/tei-p5-doc/en/html/MS.html\#msmat}{10.3.2. Material and Object Type}]\end{specHead} 
    \item[{Module}]
  msdescription
    \item[{Attributs}]
  Attributs \hyperref[TEI.att.global]{att.global} (\textit{@xml:id}, \textit{@n}, \textit{@xml:lang}, \textit{@xml:base}, \textit{@xml:space})  (\hyperref[TEI.att.global.rendition]{att.global.rendition} (\textit{@rend}, \textit{@style}, \textit{@rendition})) (\hyperref[TEI.att.global.linking]{att.global.linking} (\textit{@corresp}, \textit{@synch}, \textit{@sameAs}, \textit{@copyOf}, \textit{@next}, \textit{@prev}, \textit{@exclude}, \textit{@select})) (\hyperref[TEI.att.global.analytic]{att.global.analytic} (\textit{@ana})) (\hyperref[TEI.att.global.facs]{att.global.facs} (\textit{@facs})) (\hyperref[TEI.att.global.change]{att.global.change} (\textit{@change})) (\hyperref[TEI.att.global.responsibility]{att.global.responsibility} (\textit{@cert}, \textit{@resp})) (\hyperref[TEI.att.global.source]{att.global.source} (\textit{@source})) \hyperref[TEI.att.canonical]{att.canonical} (\textit{@key}, \textit{@ref}) 
    \item[{Membre du}]
  \hyperref[TEI.model.pPart.msdesc]{model.pPart.msdesc}
    \item[{Contenu dans}]
  
    \item[analysis: ]
   \hyperref[TEI.cl]{cl} \hyperref[TEI.phr]{phr} \hyperref[TEI.s]{s} \hyperref[TEI.span]{span}\par 
    \item[core: ]
   \hyperref[TEI.abbr]{abbr} \hyperref[TEI.add]{add} \hyperref[TEI.addrLine]{addrLine} \hyperref[TEI.author]{author} \hyperref[TEI.biblScope]{biblScope} \hyperref[TEI.citedRange]{citedRange} \hyperref[TEI.corr]{corr} \hyperref[TEI.date]{date} \hyperref[TEI.del]{del} \hyperref[TEI.desc]{desc} \hyperref[TEI.distinct]{distinct} \hyperref[TEI.editor]{editor} \hyperref[TEI.email]{email} \hyperref[TEI.emph]{emph} \hyperref[TEI.expan]{expan} \hyperref[TEI.foreign]{foreign} \hyperref[TEI.gloss]{gloss} \hyperref[TEI.head]{head} \hyperref[TEI.headItem]{headItem} \hyperref[TEI.headLabel]{headLabel} \hyperref[TEI.hi]{hi} \hyperref[TEI.item]{item} \hyperref[TEI.l]{l} \hyperref[TEI.label]{label} \hyperref[TEI.measure]{measure} \hyperref[TEI.meeting]{meeting} \hyperref[TEI.mentioned]{mentioned} \hyperref[TEI.name]{name} \hyperref[TEI.note]{note} \hyperref[TEI.num]{num} \hyperref[TEI.orig]{orig} \hyperref[TEI.p]{p} \hyperref[TEI.pubPlace]{pubPlace} \hyperref[TEI.publisher]{publisher} \hyperref[TEI.q]{q} \hyperref[TEI.quote]{quote} \hyperref[TEI.ref]{ref} \hyperref[TEI.reg]{reg} \hyperref[TEI.resp]{resp} \hyperref[TEI.rs]{rs} \hyperref[TEI.said]{said} \hyperref[TEI.sic]{sic} \hyperref[TEI.soCalled]{soCalled} \hyperref[TEI.speaker]{speaker} \hyperref[TEI.stage]{stage} \hyperref[TEI.street]{street} \hyperref[TEI.term]{term} \hyperref[TEI.textLang]{textLang} \hyperref[TEI.time]{time} \hyperref[TEI.title]{title} \hyperref[TEI.unclear]{unclear}\par 
    \item[figures: ]
   \hyperref[TEI.cell]{cell} \hyperref[TEI.figDesc]{figDesc}\par 
    \item[header: ]
   \hyperref[TEI.authority]{authority} \hyperref[TEI.change]{change} \hyperref[TEI.classCode]{classCode} \hyperref[TEI.creation]{creation} \hyperref[TEI.distributor]{distributor} \hyperref[TEI.edition]{edition} \hyperref[TEI.extent]{extent} \hyperref[TEI.funder]{funder} \hyperref[TEI.language]{language} \hyperref[TEI.licence]{licence} \hyperref[TEI.rendition]{rendition}\par 
    \item[iso-fs: ]
   \hyperref[TEI.fDescr]{fDescr} \hyperref[TEI.fsDescr]{fsDescr}\par 
    \item[linking: ]
   \hyperref[TEI.ab]{ab} \hyperref[TEI.seg]{seg}\par 
    \item[msdescription: ]
   \hyperref[TEI.accMat]{accMat} \hyperref[TEI.acquisition]{acquisition} \hyperref[TEI.additions]{additions} \hyperref[TEI.catchwords]{catchwords} \hyperref[TEI.collation]{collation} \hyperref[TEI.colophon]{colophon} \hyperref[TEI.condition]{condition} \hyperref[TEI.custEvent]{custEvent} \hyperref[TEI.decoNote]{decoNote} \hyperref[TEI.explicit]{explicit} \hyperref[TEI.filiation]{filiation} \hyperref[TEI.finalRubric]{finalRubric} \hyperref[TEI.foliation]{foliation} \hyperref[TEI.heraldry]{heraldry} \hyperref[TEI.incipit]{incipit} \hyperref[TEI.layout]{layout} \hyperref[TEI.material]{material} \hyperref[TEI.musicNotation]{musicNotation} \hyperref[TEI.objectType]{objectType} \hyperref[TEI.origDate]{origDate} \hyperref[TEI.origPlace]{origPlace} \hyperref[TEI.origin]{origin} \hyperref[TEI.provenance]{provenance} \hyperref[TEI.rubric]{rubric} \hyperref[TEI.secFol]{secFol} \hyperref[TEI.signatures]{signatures} \hyperref[TEI.source]{source} \hyperref[TEI.stamp]{stamp} \hyperref[TEI.summary]{summary} \hyperref[TEI.support]{support} \hyperref[TEI.surrogates]{surrogates} \hyperref[TEI.typeNote]{typeNote} \hyperref[TEI.watermark]{watermark}\par 
    \item[namesdates: ]
   \hyperref[TEI.addName]{addName} \hyperref[TEI.affiliation]{affiliation} \hyperref[TEI.country]{country} \hyperref[TEI.forename]{forename} \hyperref[TEI.genName]{genName} \hyperref[TEI.geogName]{geogName} \hyperref[TEI.nameLink]{nameLink} \hyperref[TEI.orgName]{orgName} \hyperref[TEI.persName]{persName} \hyperref[TEI.placeName]{placeName} \hyperref[TEI.region]{region} \hyperref[TEI.roleName]{roleName} \hyperref[TEI.settlement]{settlement} \hyperref[TEI.surname]{surname}\par 
    \item[textstructure: ]
   \hyperref[TEI.docAuthor]{docAuthor} \hyperref[TEI.docDate]{docDate} \hyperref[TEI.docEdition]{docEdition} \hyperref[TEI.titlePart]{titlePart}\par 
    \item[transcr: ]
   \hyperref[TEI.damage]{damage} \hyperref[TEI.fw]{fw} \hyperref[TEI.metamark]{metamark} \hyperref[TEI.mod]{mod} \hyperref[TEI.restore]{restore} \hyperref[TEI.retrace]{retrace} \hyperref[TEI.secl]{secl} \hyperref[TEI.supplied]{supplied} \hyperref[TEI.surplus]{surplus}
    \item[{Peut contenir}]
  
    \item[analysis: ]
   \hyperref[TEI.c]{c} \hyperref[TEI.cl]{cl} \hyperref[TEI.interp]{interp} \hyperref[TEI.interpGrp]{interpGrp} \hyperref[TEI.m]{m} \hyperref[TEI.pc]{pc} \hyperref[TEI.phr]{phr} \hyperref[TEI.s]{s} \hyperref[TEI.span]{span} \hyperref[TEI.spanGrp]{spanGrp} \hyperref[TEI.w]{w}\par 
    \item[core: ]
   \hyperref[TEI.abbr]{abbr} \hyperref[TEI.add]{add} \hyperref[TEI.address]{address} \hyperref[TEI.binaryObject]{binaryObject} \hyperref[TEI.cb]{cb} \hyperref[TEI.choice]{choice} \hyperref[TEI.corr]{corr} \hyperref[TEI.date]{date} \hyperref[TEI.del]{del} \hyperref[TEI.distinct]{distinct} \hyperref[TEI.email]{email} \hyperref[TEI.emph]{emph} \hyperref[TEI.expan]{expan} \hyperref[TEI.foreign]{foreign} \hyperref[TEI.gap]{gap} \hyperref[TEI.gb]{gb} \hyperref[TEI.gloss]{gloss} \hyperref[TEI.graphic]{graphic} \hyperref[TEI.hi]{hi} \hyperref[TEI.index]{index} \hyperref[TEI.lb]{lb} \hyperref[TEI.measure]{measure} \hyperref[TEI.measureGrp]{measureGrp} \hyperref[TEI.media]{media} \hyperref[TEI.mentioned]{mentioned} \hyperref[TEI.milestone]{milestone} \hyperref[TEI.name]{name} \hyperref[TEI.note]{note} \hyperref[TEI.num]{num} \hyperref[TEI.orig]{orig} \hyperref[TEI.pb]{pb} \hyperref[TEI.ptr]{ptr} \hyperref[TEI.ref]{ref} \hyperref[TEI.reg]{reg} \hyperref[TEI.rs]{rs} \hyperref[TEI.sic]{sic} \hyperref[TEI.soCalled]{soCalled} \hyperref[TEI.term]{term} \hyperref[TEI.time]{time} \hyperref[TEI.title]{title} \hyperref[TEI.unclear]{unclear}\par 
    \item[derived-module-tei.istex: ]
   \hyperref[TEI.math]{math} \hyperref[TEI.mrow]{mrow}\par 
    \item[figures: ]
   \hyperref[TEI.figure]{figure} \hyperref[TEI.formula]{formula} \hyperref[TEI.notatedMusic]{notatedMusic}\par 
    \item[header: ]
   \hyperref[TEI.idno]{idno}\par 
    \item[iso-fs: ]
   \hyperref[TEI.fLib]{fLib} \hyperref[TEI.fs]{fs} \hyperref[TEI.fvLib]{fvLib}\par 
    \item[linking: ]
   \hyperref[TEI.alt]{alt} \hyperref[TEI.altGrp]{altGrp} \hyperref[TEI.anchor]{anchor} \hyperref[TEI.join]{join} \hyperref[TEI.joinGrp]{joinGrp} \hyperref[TEI.link]{link} \hyperref[TEI.linkGrp]{linkGrp} \hyperref[TEI.seg]{seg} \hyperref[TEI.timeline]{timeline}\par 
    \item[msdescription: ]
   \hyperref[TEI.catchwords]{catchwords} \hyperref[TEI.depth]{depth} \hyperref[TEI.dim]{dim} \hyperref[TEI.dimensions]{dimensions} \hyperref[TEI.height]{height} \hyperref[TEI.heraldry]{heraldry} \hyperref[TEI.locus]{locus} \hyperref[TEI.locusGrp]{locusGrp} \hyperref[TEI.material]{material} \hyperref[TEI.objectType]{objectType} \hyperref[TEI.origDate]{origDate} \hyperref[TEI.origPlace]{origPlace} \hyperref[TEI.secFol]{secFol} \hyperref[TEI.signatures]{signatures} \hyperref[TEI.source]{source} \hyperref[TEI.stamp]{stamp} \hyperref[TEI.watermark]{watermark} \hyperref[TEI.width]{width}\par 
    \item[namesdates: ]
   \hyperref[TEI.addName]{addName} \hyperref[TEI.affiliation]{affiliation} \hyperref[TEI.country]{country} \hyperref[TEI.forename]{forename} \hyperref[TEI.genName]{genName} \hyperref[TEI.geogName]{geogName} \hyperref[TEI.location]{location} \hyperref[TEI.nameLink]{nameLink} \hyperref[TEI.orgName]{orgName} \hyperref[TEI.persName]{persName} \hyperref[TEI.placeName]{placeName} \hyperref[TEI.region]{region} \hyperref[TEI.roleName]{roleName} \hyperref[TEI.settlement]{settlement} \hyperref[TEI.state]{state} \hyperref[TEI.surname]{surname}\par 
    \item[spoken: ]
   \hyperref[TEI.annotationBlock]{annotationBlock}\par 
    \item[transcr: ]
   \hyperref[TEI.addSpan]{addSpan} \hyperref[TEI.am]{am} \hyperref[TEI.damage]{damage} \hyperref[TEI.damageSpan]{damageSpan} \hyperref[TEI.delSpan]{delSpan} \hyperref[TEI.ex]{ex} \hyperref[TEI.fw]{fw} \hyperref[TEI.handShift]{handShift} \hyperref[TEI.listTranspose]{listTranspose} \hyperref[TEI.metamark]{metamark} \hyperref[TEI.mod]{mod} \hyperref[TEI.redo]{redo} \hyperref[TEI.restore]{restore} \hyperref[TEI.retrace]{retrace} \hyperref[TEI.secl]{secl} \hyperref[TEI.space]{space} \hyperref[TEI.subst]{subst} \hyperref[TEI.substJoin]{substJoin} \hyperref[TEI.supplied]{supplied} \hyperref[TEI.surplus]{surplus} \hyperref[TEI.undo]{undo}\par des données textuelles
    \item[{Note}]
  \par
The {\itshape ref} attribute may be used to point to one or more items within a taxonomy of types of material, defined either internally or externally.
    \item[{Exemple}]
  \leavevmode\bgroup\exampleFont \begin{shaded}\noindent\mbox{}{<\textbf{p}>}\mbox{}\newline 
\hspace*{6pt}{<\textbf{index}\hspace*{6pt}{indexName}="{typo\textunderscore decor}">}\mbox{}\newline 
\hspace*{6pt}\hspace*{6pt}{<\textbf{term}>}Entrelacs géométriques{</\textbf{term}>}\mbox{}\newline 
\hspace*{6pt}{</\textbf{index}>} Reliure en {<\textbf{material}>}maroquin{</\textbf{material}>} brun jaspé\mbox{}\newline 
{</\textbf{p}>}\end{shaded}\egroup 


    \item[{Modèle de contenu}]
  \mbox{}\hfill\\[-10pt]\begin{Verbatim}[fontsize=\small]
<content>
 <macroRef key="macro.phraseSeq"/>
</content>
    
\end{Verbatim}

    \item[{Schéma Declaration}]
  \mbox{}\hfill\\[-10pt]\begin{Verbatim}[fontsize=\small]
element material
{
   tei_att.global.attributes,
   tei_att.canonical.attributes,
   tei_macro.phraseSeq}
\end{Verbatim}

\end{reflist}  \index{math=<math>|oddindex}\index{display=@display!<math>|oddindex}\index{altimg=@altimg!<math>|oddindex}\index{location=@location!<math>|oddindex}\index{overflow=@overflow!<math>|oddindex}\index{alttext=@alttext!<math>|oddindex}\index{maxwidth=@maxwidth!<math>|oddindex}\index{altimg-width=@altimg-width!<math>|oddindex}\index{altimg-height=@altimg-height!<math>|oddindex}\index{altimg-valign=@altimg-valign!<math>|oddindex}\index{cdgroup=@cdgroup!<math>|oddindex}
\begin{reflist}
\item[]\begin{specHead}{TEI.math}{<math> }MathML\end{specHead} 
    \item[{Namespace}]
  http://www.w3.org/1998/Math/MathML
    \item[{Module}]
  derived-module-tei.istex
    \item[{Attributs}]
  Attributs\hfil\\[-10pt]\begin{sansreflist}
    \item[@display]
  
\begin{reflist}
    \item[{Statut}]
  Optionel
    \item[{Type de données}]
  \xref{https://www.w3.org/TR/xmlschema-2/\#}{}
\end{reflist}  
    \item[@altimg]
  
\begin{reflist}
    \item[{Statut}]
  Optionel
    \item[{Type de données}]
  \xref{https://www.w3.org/TR/xmlschema-2/\#}{}
\end{reflist}  
    \item[@location]
  
\begin{reflist}
    \item[{Statut}]
  Optionel
    \item[{Type de données}]
  \xref{https://www.w3.org/TR/xmlschema-2/\#}{}
\end{reflist}  
    \item[@overflow]
  
\begin{reflist}
    \item[{Statut}]
  Optionel
    \item[{Type de données}]
  \xref{https://www.w3.org/TR/xmlschema-2/\#}{}
\end{reflist}  
    \item[@alttext]
  
\begin{reflist}
    \item[{Statut}]
  Optionel
    \item[{Type de données}]
  \xref{https://www.w3.org/TR/xmlschema-2/\#}{}
\end{reflist}  
    \item[@maxwidth]
  
\begin{reflist}
    \item[{Statut}]
  Optionel
    \item[{Type de données}]
  \xref{https://www.w3.org/TR/xmlschema-2/\#}{}
\end{reflist}  
    \item[@altimg-width]
  
\begin{reflist}
    \item[{Statut}]
  Optionel
    \item[{Type de données}]
  \xref{https://www.w3.org/TR/xmlschema-2/\#}{}
\end{reflist}  
    \item[@altimg-height]
  
\begin{reflist}
    \item[{Statut}]
  Optionel
    \item[{Type de données}]
  \xref{https://www.w3.org/TR/xmlschema-2/\#}{}
\end{reflist}  
    \item[@altimg-valign]
  
\begin{reflist}
    \item[{Statut}]
  Optionel
    \item[{Type de données}]
  \xref{https://www.w3.org/TR/xmlschema-2/\#}{}
\end{reflist}  
    \item[@cdgroup]
  
\begin{reflist}
    \item[{Statut}]
  Optionel
    \item[{Type de données}]
  \xref{https://www.w3.org/TR/xmlschema-2/\#}{}
\end{reflist}  
\end{sansreflist}  
    \item[{Membre du}]
  \hyperref[TEI.model.graphicLike]{model.graphicLike}
    \item[{Contenu dans}]
  
    \item[analysis: ]
   \hyperref[TEI.cl]{cl} \hyperref[TEI.phr]{phr} \hyperref[TEI.s]{s}\par 
    \item[core: ]
   \hyperref[TEI.abbr]{abbr} \hyperref[TEI.add]{add} \hyperref[TEI.addrLine]{addrLine} \hyperref[TEI.author]{author} \hyperref[TEI.biblScope]{biblScope} \hyperref[TEI.citedRange]{citedRange} \hyperref[TEI.corr]{corr} \hyperref[TEI.date]{date} \hyperref[TEI.del]{del} \hyperref[TEI.distinct]{distinct} \hyperref[TEI.editor]{editor} \hyperref[TEI.email]{email} \hyperref[TEI.emph]{emph} \hyperref[TEI.expan]{expan} \hyperref[TEI.foreign]{foreign} \hyperref[TEI.gloss]{gloss} \hyperref[TEI.head]{head} \hyperref[TEI.headItem]{headItem} \hyperref[TEI.headLabel]{headLabel} \hyperref[TEI.hi]{hi} \hyperref[TEI.item]{item} \hyperref[TEI.l]{l} \hyperref[TEI.label]{label} \hyperref[TEI.measure]{measure} \hyperref[TEI.mentioned]{mentioned} \hyperref[TEI.name]{name} \hyperref[TEI.note]{note} \hyperref[TEI.num]{num} \hyperref[TEI.orig]{orig} \hyperref[TEI.p]{p} \hyperref[TEI.pubPlace]{pubPlace} \hyperref[TEI.publisher]{publisher} \hyperref[TEI.q]{q} \hyperref[TEI.quote]{quote} \hyperref[TEI.ref]{ref} \hyperref[TEI.reg]{reg} \hyperref[TEI.rs]{rs} \hyperref[TEI.said]{said} \hyperref[TEI.sic]{sic} \hyperref[TEI.soCalled]{soCalled} \hyperref[TEI.speaker]{speaker} \hyperref[TEI.stage]{stage} \hyperref[TEI.street]{street} \hyperref[TEI.term]{term} \hyperref[TEI.textLang]{textLang} \hyperref[TEI.time]{time} \hyperref[TEI.title]{title} \hyperref[TEI.unclear]{unclear}\par 
    \item[figures: ]
   \hyperref[TEI.cell]{cell} \hyperref[TEI.figDesc]{figDesc} \hyperref[TEI.figure]{figure} \hyperref[TEI.formula]{formula} \hyperref[TEI.table]{table}\par 
    \item[header: ]
   \hyperref[TEI.change]{change} \hyperref[TEI.distributor]{distributor} \hyperref[TEI.edition]{edition} \hyperref[TEI.extent]{extent} \hyperref[TEI.licence]{licence}\par 
    \item[linking: ]
   \hyperref[TEI.ab]{ab} \hyperref[TEI.seg]{seg}\par 
    \item[msdescription: ]
   \hyperref[TEI.accMat]{accMat} \hyperref[TEI.acquisition]{acquisition} \hyperref[TEI.additions]{additions} \hyperref[TEI.catchwords]{catchwords} \hyperref[TEI.collation]{collation} \hyperref[TEI.colophon]{colophon} \hyperref[TEI.condition]{condition} \hyperref[TEI.custEvent]{custEvent} \hyperref[TEI.decoNote]{decoNote} \hyperref[TEI.explicit]{explicit} \hyperref[TEI.filiation]{filiation} \hyperref[TEI.finalRubric]{finalRubric} \hyperref[TEI.foliation]{foliation} \hyperref[TEI.heraldry]{heraldry} \hyperref[TEI.incipit]{incipit} \hyperref[TEI.layout]{layout} \hyperref[TEI.material]{material} \hyperref[TEI.musicNotation]{musicNotation} \hyperref[TEI.objectType]{objectType} \hyperref[TEI.origDate]{origDate} \hyperref[TEI.origPlace]{origPlace} \hyperref[TEI.origin]{origin} \hyperref[TEI.provenance]{provenance} \hyperref[TEI.rubric]{rubric} \hyperref[TEI.secFol]{secFol} \hyperref[TEI.signatures]{signatures} \hyperref[TEI.source]{source} \hyperref[TEI.stamp]{stamp} \hyperref[TEI.summary]{summary} \hyperref[TEI.support]{support} \hyperref[TEI.surrogates]{surrogates} \hyperref[TEI.typeNote]{typeNote} \hyperref[TEI.watermark]{watermark}\par 
    \item[namesdates: ]
   \hyperref[TEI.addName]{addName} \hyperref[TEI.affiliation]{affiliation} \hyperref[TEI.country]{country} \hyperref[TEI.forename]{forename} \hyperref[TEI.genName]{genName} \hyperref[TEI.geogName]{geogName} \hyperref[TEI.nameLink]{nameLink} \hyperref[TEI.orgName]{orgName} \hyperref[TEI.persName]{persName} \hyperref[TEI.placeName]{placeName} \hyperref[TEI.region]{region} \hyperref[TEI.roleName]{roleName} \hyperref[TEI.settlement]{settlement} \hyperref[TEI.surname]{surname}\par 
    \item[textstructure: ]
   \hyperref[TEI.docAuthor]{docAuthor} \hyperref[TEI.docDate]{docDate} \hyperref[TEI.docEdition]{docEdition} \hyperref[TEI.titlePart]{titlePart}\par 
    \item[transcr: ]
   \hyperref[TEI.damage]{damage} \hyperref[TEI.facsimile]{facsimile} \hyperref[TEI.fw]{fw} \hyperref[TEI.metamark]{metamark} \hyperref[TEI.mod]{mod} \hyperref[TEI.restore]{restore} \hyperref[TEI.retrace]{retrace} \hyperref[TEI.secl]{secl} \hyperref[TEI.sourceDoc]{sourceDoc} \hyperref[TEI.supplied]{supplied} \hyperref[TEI.surface]{surface} \hyperref[TEI.surplus]{surplus} \hyperref[TEI.zone]{zone}
    \item[{Peut contenir}]
  
    \item[derived-module-tei.istex: ]
   \hyperref[TEI.menclose]{menclose} \hyperref[TEI.mfenced]{mfenced} \hyperref[TEI.mfrac]{mfrac} \hyperref[TEI.mi]{mi} \hyperref[TEI.mn]{mn} \hyperref[TEI.mo]{mo} \hyperref[TEI.mpadded]{mpadded} \hyperref[TEI.mphantom]{mphantom} \hyperref[TEI.mprescripts]{mprescripts} \hyperref[TEI.mrow]{mrow} \hyperref[TEI.mspace]{mspace} \hyperref[TEI.msqrt]{msqrt} \hyperref[TEI.mstyle]{mstyle} \hyperref[TEI.msub]{msub} \hyperref[TEI.msubsup]{msubsup} \hyperref[TEI.msup]{msup} \hyperref[TEI.msupsub]{msupsub} \hyperref[TEI.mtable]{mtable} \hyperref[TEI.mtext]{mtext} \hyperref[TEI.munder]{munder} \hyperref[TEI.munderover]{munderover} \hyperref[TEI.none]{none} \hyperref[TEI.semantics]{semantics}\par des données textuelles
    \item[{Modèle de contenu}]
  \mbox{}\hfill\\[-10pt]\begin{Verbatim}[fontsize=\small]
<content>
 <alternate maxOccurs="unbounded"
  minOccurs="0">
  <textNode/>
  <elementRef key="mrow"/>
  <elementRef key="mstyle"/>
  <elementRef key="mi"/>
  <elementRef key="mn"/>
  <elementRef key="mtext"/>
  <elementRef key="mfrac"/>
  <elementRef key="mspace"/>
  <elementRef key="msqrt"/>
  <elementRef key="msub"/>
  <elementRef key="msubsup"/>
  <elementRef key="msupsub"/>
  <elementRef key="msup"/>
  <elementRef key="mo"/>
  <elementRef key="mfenced"/>
  <elementRef key="mtable"/>
  <elementRef key="munderover"/>
  <elementRef key="mprescripts"/>
  <elementRef key="none"/>
  <elementRef key="munder"/>
  <elementRef key="mphantom"/>
  <elementRef key="mpadded"/>
  <elementRef key="menclose"/>
  <elementRef key="semantics"/>
 </alternate>
</content>
    
\end{Verbatim}

    \item[{Schéma Declaration}]
  \mbox{}\hfill\\[-10pt]\begin{Verbatim}[fontsize=\small]
element math
{
   attribute display { display }?,
   attribute altimg { altimg }?,
   attribute [http://www.wiley.com/namespaces/wiley]location { location }?,
   attribute overflow { overflow }?,
   attribute alttext { alttext }?,
   attribute maxwidth { maxwidth }?,
   attribute altimg-width { altimg-width }?,
   attribute altimg-height { altimg-height }?,
   attribute altimg-valign { altimg-valign }?,
   attribute cdgroup { cdgroup }?,
   (
      text
    | tei_mrow    | tei_mstyle    | tei_mi    | tei_mn    | tei_mtext    | tei_mfrac    | tei_mspace    | tei_msqrt    | tei_msub    | tei_msubsup    | tei_msupsub    | tei_msup    | tei_mo    | tei_mfenced    | tei_mtable    | tei_munderover    | tei_mprescripts    | tei_none    | tei_munder    | tei_mphantom    | tei_mpadded    | tei_menclose    | tei_semantics   )*
}
\end{Verbatim}

\end{reflist}  \index{measure=<measure>|oddindex}\index{type=@type!<measure>|oddindex}
\begin{reflist}
\item[]\begin{specHead}{TEI.measure}{<measure> }(mesure) contient un mot ou une expression faisant référence à la quantité d'un objet ou d'un produit, comprenant en général un nombre, une unité et le nom d'un produit. [\xref{http://www.tei-c.org/release/doc/tei-p5-doc/en/html/CO.html\#CONANU}{3.5.3. Numbers and Measures}]\end{specHead} 
    \item[{Module}]
  core
    \item[{Attributs}]
  Attributs \hyperref[TEI.att.global]{att.global} (\textit{@xml:id}, \textit{@n}, \textit{@xml:lang}, \textit{@xml:base}, \textit{@xml:space})  (\hyperref[TEI.att.global.rendition]{att.global.rendition} (\textit{@rend}, \textit{@style}, \textit{@rendition})) (\hyperref[TEI.att.global.linking]{att.global.linking} (\textit{@corresp}, \textit{@synch}, \textit{@sameAs}, \textit{@copyOf}, \textit{@next}, \textit{@prev}, \textit{@exclude}, \textit{@select})) (\hyperref[TEI.att.global.analytic]{att.global.analytic} (\textit{@ana})) (\hyperref[TEI.att.global.facs]{att.global.facs} (\textit{@facs})) (\hyperref[TEI.att.global.change]{att.global.change} (\textit{@change})) (\hyperref[TEI.att.global.responsibility]{att.global.responsibility} (\textit{@cert}, \textit{@resp})) (\hyperref[TEI.att.global.source]{att.global.source} (\textit{@source})) \hyperref[TEI.att.measurement]{att.measurement} (\textit{@unit}, \textit{@quantity}, \textit{@commodity}) \hfil\\[-10pt]\begin{sansreflist}
    \item[@type]
  précise le type de mesure exprimée dans la typologie adaptée.
\begin{reflist}
    \item[{Statut}]
  Optionel
    \item[{Type de données}]
  \hyperref[TEI.teidata.enumerated]{teidata.enumerated}
\end{reflist}  
\end{sansreflist}  
    \item[{Membre du}]
  \hyperref[TEI.model.OABody]{model.OABody} \hyperref[TEI.model.measureLike]{model.measureLike}
    \item[{Contenu dans}]
  
    \item[analysis: ]
   \hyperref[TEI.cl]{cl} \hyperref[TEI.phr]{phr} \hyperref[TEI.s]{s} \hyperref[TEI.span]{span}\par 
    \item[core: ]
   \hyperref[TEI.abbr]{abbr} \hyperref[TEI.add]{add} \hyperref[TEI.addrLine]{addrLine} \hyperref[TEI.author]{author} \hyperref[TEI.bibl]{bibl} \hyperref[TEI.biblScope]{biblScope} \hyperref[TEI.citedRange]{citedRange} \hyperref[TEI.corr]{corr} \hyperref[TEI.date]{date} \hyperref[TEI.del]{del} \hyperref[TEI.desc]{desc} \hyperref[TEI.distinct]{distinct} \hyperref[TEI.editor]{editor} \hyperref[TEI.email]{email} \hyperref[TEI.emph]{emph} \hyperref[TEI.expan]{expan} \hyperref[TEI.foreign]{foreign} \hyperref[TEI.gloss]{gloss} \hyperref[TEI.head]{head} \hyperref[TEI.headItem]{headItem} \hyperref[TEI.headLabel]{headLabel} \hyperref[TEI.hi]{hi} \hyperref[TEI.item]{item} \hyperref[TEI.l]{l} \hyperref[TEI.label]{label} \hyperref[TEI.measure]{measure} \hyperref[TEI.measureGrp]{measureGrp} \hyperref[TEI.meeting]{meeting} \hyperref[TEI.mentioned]{mentioned} \hyperref[TEI.name]{name} \hyperref[TEI.note]{note} \hyperref[TEI.num]{num} \hyperref[TEI.orig]{orig} \hyperref[TEI.p]{p} \hyperref[TEI.pubPlace]{pubPlace} \hyperref[TEI.publisher]{publisher} \hyperref[TEI.q]{q} \hyperref[TEI.quote]{quote} \hyperref[TEI.ref]{ref} \hyperref[TEI.reg]{reg} \hyperref[TEI.resp]{resp} \hyperref[TEI.rs]{rs} \hyperref[TEI.said]{said} \hyperref[TEI.sic]{sic} \hyperref[TEI.soCalled]{soCalled} \hyperref[TEI.speaker]{speaker} \hyperref[TEI.stage]{stage} \hyperref[TEI.street]{street} \hyperref[TEI.term]{term} \hyperref[TEI.textLang]{textLang} \hyperref[TEI.time]{time} \hyperref[TEI.title]{title} \hyperref[TEI.unclear]{unclear}\par 
    \item[figures: ]
   \hyperref[TEI.cell]{cell} \hyperref[TEI.figDesc]{figDesc}\par 
    \item[header: ]
   \hyperref[TEI.authority]{authority} \hyperref[TEI.change]{change} \hyperref[TEI.classCode]{classCode} \hyperref[TEI.creation]{creation} \hyperref[TEI.distributor]{distributor} \hyperref[TEI.edition]{edition} \hyperref[TEI.extent]{extent} \hyperref[TEI.funder]{funder} \hyperref[TEI.language]{language} \hyperref[TEI.licence]{licence} \hyperref[TEI.rendition]{rendition}\par 
    \item[iso-fs: ]
   \hyperref[TEI.fDescr]{fDescr} \hyperref[TEI.fsDescr]{fsDescr}\par 
    \item[linking: ]
   \hyperref[TEI.ab]{ab} \hyperref[TEI.seg]{seg}\par 
    \item[msdescription: ]
   \hyperref[TEI.accMat]{accMat} \hyperref[TEI.acquisition]{acquisition} \hyperref[TEI.additions]{additions} \hyperref[TEI.catchwords]{catchwords} \hyperref[TEI.collation]{collation} \hyperref[TEI.colophon]{colophon} \hyperref[TEI.condition]{condition} \hyperref[TEI.custEvent]{custEvent} \hyperref[TEI.decoNote]{decoNote} \hyperref[TEI.explicit]{explicit} \hyperref[TEI.filiation]{filiation} \hyperref[TEI.finalRubric]{finalRubric} \hyperref[TEI.foliation]{foliation} \hyperref[TEI.heraldry]{heraldry} \hyperref[TEI.incipit]{incipit} \hyperref[TEI.layout]{layout} \hyperref[TEI.material]{material} \hyperref[TEI.musicNotation]{musicNotation} \hyperref[TEI.objectType]{objectType} \hyperref[TEI.origDate]{origDate} \hyperref[TEI.origPlace]{origPlace} \hyperref[TEI.origin]{origin} \hyperref[TEI.provenance]{provenance} \hyperref[TEI.rubric]{rubric} \hyperref[TEI.secFol]{secFol} \hyperref[TEI.signatures]{signatures} \hyperref[TEI.source]{source} \hyperref[TEI.stamp]{stamp} \hyperref[TEI.summary]{summary} \hyperref[TEI.support]{support} \hyperref[TEI.surrogates]{surrogates} \hyperref[TEI.typeNote]{typeNote} \hyperref[TEI.watermark]{watermark}\par 
    \item[namesdates: ]
   \hyperref[TEI.addName]{addName} \hyperref[TEI.affiliation]{affiliation} \hyperref[TEI.country]{country} \hyperref[TEI.forename]{forename} \hyperref[TEI.genName]{genName} \hyperref[TEI.geogName]{geogName} \hyperref[TEI.location]{location} \hyperref[TEI.nameLink]{nameLink} \hyperref[TEI.orgName]{orgName} \hyperref[TEI.persName]{persName} \hyperref[TEI.placeName]{placeName} \hyperref[TEI.region]{region} \hyperref[TEI.roleName]{roleName} \hyperref[TEI.settlement]{settlement} \hyperref[TEI.surname]{surname}\par 
    \item[spoken: ]
   \hyperref[TEI.annotationBlock]{annotationBlock}\par 
    \item[textstructure: ]
   \hyperref[TEI.docAuthor]{docAuthor} \hyperref[TEI.docDate]{docDate} \hyperref[TEI.docEdition]{docEdition} \hyperref[TEI.titlePart]{titlePart}\par 
    \item[transcr: ]
   \hyperref[TEI.damage]{damage} \hyperref[TEI.fw]{fw} \hyperref[TEI.metamark]{metamark} \hyperref[TEI.mod]{mod} \hyperref[TEI.restore]{restore} \hyperref[TEI.retrace]{retrace} \hyperref[TEI.secl]{secl} \hyperref[TEI.supplied]{supplied} \hyperref[TEI.surplus]{surplus}
    \item[{Peut contenir}]
  
    \item[analysis: ]
   \hyperref[TEI.c]{c} \hyperref[TEI.cl]{cl} \hyperref[TEI.interp]{interp} \hyperref[TEI.interpGrp]{interpGrp} \hyperref[TEI.m]{m} \hyperref[TEI.pc]{pc} \hyperref[TEI.phr]{phr} \hyperref[TEI.s]{s} \hyperref[TEI.span]{span} \hyperref[TEI.spanGrp]{spanGrp} \hyperref[TEI.w]{w}\par 
    \item[core: ]
   \hyperref[TEI.abbr]{abbr} \hyperref[TEI.add]{add} \hyperref[TEI.address]{address} \hyperref[TEI.binaryObject]{binaryObject} \hyperref[TEI.cb]{cb} \hyperref[TEI.choice]{choice} \hyperref[TEI.corr]{corr} \hyperref[TEI.date]{date} \hyperref[TEI.del]{del} \hyperref[TEI.distinct]{distinct} \hyperref[TEI.email]{email} \hyperref[TEI.emph]{emph} \hyperref[TEI.expan]{expan} \hyperref[TEI.foreign]{foreign} \hyperref[TEI.gap]{gap} \hyperref[TEI.gb]{gb} \hyperref[TEI.gloss]{gloss} \hyperref[TEI.graphic]{graphic} \hyperref[TEI.hi]{hi} \hyperref[TEI.index]{index} \hyperref[TEI.lb]{lb} \hyperref[TEI.measure]{measure} \hyperref[TEI.measureGrp]{measureGrp} \hyperref[TEI.media]{media} \hyperref[TEI.mentioned]{mentioned} \hyperref[TEI.milestone]{milestone} \hyperref[TEI.name]{name} \hyperref[TEI.note]{note} \hyperref[TEI.num]{num} \hyperref[TEI.orig]{orig} \hyperref[TEI.pb]{pb} \hyperref[TEI.ptr]{ptr} \hyperref[TEI.ref]{ref} \hyperref[TEI.reg]{reg} \hyperref[TEI.rs]{rs} \hyperref[TEI.sic]{sic} \hyperref[TEI.soCalled]{soCalled} \hyperref[TEI.term]{term} \hyperref[TEI.time]{time} \hyperref[TEI.title]{title} \hyperref[TEI.unclear]{unclear}\par 
    \item[derived-module-tei.istex: ]
   \hyperref[TEI.math]{math} \hyperref[TEI.mrow]{mrow}\par 
    \item[figures: ]
   \hyperref[TEI.figure]{figure} \hyperref[TEI.formula]{formula} \hyperref[TEI.notatedMusic]{notatedMusic}\par 
    \item[header: ]
   \hyperref[TEI.idno]{idno}\par 
    \item[iso-fs: ]
   \hyperref[TEI.fLib]{fLib} \hyperref[TEI.fs]{fs} \hyperref[TEI.fvLib]{fvLib}\par 
    \item[linking: ]
   \hyperref[TEI.alt]{alt} \hyperref[TEI.altGrp]{altGrp} \hyperref[TEI.anchor]{anchor} \hyperref[TEI.join]{join} \hyperref[TEI.joinGrp]{joinGrp} \hyperref[TEI.link]{link} \hyperref[TEI.linkGrp]{linkGrp} \hyperref[TEI.seg]{seg} \hyperref[TEI.timeline]{timeline}\par 
    \item[msdescription: ]
   \hyperref[TEI.catchwords]{catchwords} \hyperref[TEI.depth]{depth} \hyperref[TEI.dim]{dim} \hyperref[TEI.dimensions]{dimensions} \hyperref[TEI.height]{height} \hyperref[TEI.heraldry]{heraldry} \hyperref[TEI.locus]{locus} \hyperref[TEI.locusGrp]{locusGrp} \hyperref[TEI.material]{material} \hyperref[TEI.objectType]{objectType} \hyperref[TEI.origDate]{origDate} \hyperref[TEI.origPlace]{origPlace} \hyperref[TEI.secFol]{secFol} \hyperref[TEI.signatures]{signatures} \hyperref[TEI.source]{source} \hyperref[TEI.stamp]{stamp} \hyperref[TEI.watermark]{watermark} \hyperref[TEI.width]{width}\par 
    \item[namesdates: ]
   \hyperref[TEI.addName]{addName} \hyperref[TEI.affiliation]{affiliation} \hyperref[TEI.country]{country} \hyperref[TEI.forename]{forename} \hyperref[TEI.genName]{genName} \hyperref[TEI.geogName]{geogName} \hyperref[TEI.location]{location} \hyperref[TEI.nameLink]{nameLink} \hyperref[TEI.orgName]{orgName} \hyperref[TEI.persName]{persName} \hyperref[TEI.placeName]{placeName} \hyperref[TEI.region]{region} \hyperref[TEI.roleName]{roleName} \hyperref[TEI.settlement]{settlement} \hyperref[TEI.state]{state} \hyperref[TEI.surname]{surname}\par 
    \item[spoken: ]
   \hyperref[TEI.annotationBlock]{annotationBlock}\par 
    \item[transcr: ]
   \hyperref[TEI.addSpan]{addSpan} \hyperref[TEI.am]{am} \hyperref[TEI.damage]{damage} \hyperref[TEI.damageSpan]{damageSpan} \hyperref[TEI.delSpan]{delSpan} \hyperref[TEI.ex]{ex} \hyperref[TEI.fw]{fw} \hyperref[TEI.handShift]{handShift} \hyperref[TEI.listTranspose]{listTranspose} \hyperref[TEI.metamark]{metamark} \hyperref[TEI.mod]{mod} \hyperref[TEI.redo]{redo} \hyperref[TEI.restore]{restore} \hyperref[TEI.retrace]{retrace} \hyperref[TEI.secl]{secl} \hyperref[TEI.space]{space} \hyperref[TEI.subst]{subst} \hyperref[TEI.substJoin]{substJoin} \hyperref[TEI.supplied]{supplied} \hyperref[TEI.surplus]{surplus} \hyperref[TEI.undo]{undo}\par des données textuelles
    \item[{Exemple}]
  \leavevmode\bgroup\exampleFont \begin{shaded}\noindent\mbox{}{<\textbf{measure}\hspace*{6pt}{type}="{weight}">}\mbox{}\newline 
\hspace*{6pt}{<\textbf{num}>}2{</\textbf{num}>} kilos de sucre\mbox{}\newline 
{</\textbf{measure}>}\mbox{}\newline 
{<\textbf{measure}\hspace*{6pt}{type}="{currency}">}16,99 € {</\textbf{measure}>}\mbox{}\newline 
{<\textbf{measure}\hspace*{6pt}{type}="{area}">}5 hectares{</\textbf{measure}>}\end{shaded}\egroup 


    \item[{Exemple}]
  \leavevmode\bgroup\exampleFont \begin{shaded}\noindent\mbox{}{<\textbf{measure}\hspace*{6pt}{commodity}="{vin}"\hspace*{6pt}{quantity}="{1}"\mbox{}\newline 
\hspace*{6pt}{unit}="{liquide}">}un hectolitre de vin{</\textbf{measure}>}\mbox{}\newline 
{<\textbf{measure}\hspace*{6pt}{commodity}="{rose}"\hspace*{6pt}{quantity}="{12}"\mbox{}\newline 
\hspace*{6pt}{unit}="{nombre}">}1 douzaine de roses{</\textbf{measure}>}\mbox{}\newline 
{<\textbf{measure}\hspace*{6pt}{commodity}="{moule}"\hspace*{6pt}{quantity}="{1}"\mbox{}\newline 
\hspace*{6pt}{unit}="{liquide}">} un litre de moules {</\textbf{measure}>}\end{shaded}\egroup 


    \item[{Modèle de contenu}]
  \mbox{}\hfill\\[-10pt]\begin{Verbatim}[fontsize=\small]
<content>
 <macroRef key="macro.phraseSeq"/>
</content>
    
\end{Verbatim}

    \item[{Schéma Declaration}]
  \mbox{}\hfill\\[-10pt]\begin{Verbatim}[fontsize=\small]
element measure
{
   tei_att.global.attributes,
   tei_att.measurement.attributes,
   attribute type { text }?,
   tei_macro.phraseSeq}
\end{Verbatim}

\end{reflist}  \index{measureGrp=<measureGrp>|oddindex}
\begin{reflist}
\item[]\begin{specHead}{TEI.measureGrp}{<measureGrp> }(groupe de mesures) contient un groupe de spécifications des dimensions qui concernent un même objet, par exemple la hauteur et la largeur d'une page d'un manuscrit. [\xref{http://www.tei-c.org/release/doc/tei-p5-doc/en/html/MS.html\#msdim}{10.3.4. Dimensions}]\end{specHead} 
    \item[{Module}]
  core
    \item[{Attributs}]
  Attributs \hyperref[TEI.att.global]{att.global} (\textit{@xml:id}, \textit{@n}, \textit{@xml:lang}, \textit{@xml:base}, \textit{@xml:space})  (\hyperref[TEI.att.global.rendition]{att.global.rendition} (\textit{@rend}, \textit{@style}, \textit{@rendition})) (\hyperref[TEI.att.global.linking]{att.global.linking} (\textit{@corresp}, \textit{@synch}, \textit{@sameAs}, \textit{@copyOf}, \textit{@next}, \textit{@prev}, \textit{@exclude}, \textit{@select})) (\hyperref[TEI.att.global.analytic]{att.global.analytic} (\textit{@ana})) (\hyperref[TEI.att.global.facs]{att.global.facs} (\textit{@facs})) (\hyperref[TEI.att.global.change]{att.global.change} (\textit{@change})) (\hyperref[TEI.att.global.responsibility]{att.global.responsibility} (\textit{@cert}, \textit{@resp})) (\hyperref[TEI.att.global.source]{att.global.source} (\textit{@source})) \hyperref[TEI.att.measurement]{att.measurement} (\textit{@unit}, \textit{@quantity}, \textit{@commodity}) \hyperref[TEI.att.typed]{att.typed} (\textit{@type}, \textit{@subtype}) 
    \item[{Membre du}]
  \hyperref[TEI.model.measureLike]{model.measureLike}
    \item[{Contenu dans}]
  
    \item[analysis: ]
   \hyperref[TEI.cl]{cl} \hyperref[TEI.phr]{phr} \hyperref[TEI.s]{s} \hyperref[TEI.span]{span}\par 
    \item[core: ]
   \hyperref[TEI.abbr]{abbr} \hyperref[TEI.add]{add} \hyperref[TEI.addrLine]{addrLine} \hyperref[TEI.author]{author} \hyperref[TEI.bibl]{bibl} \hyperref[TEI.biblScope]{biblScope} \hyperref[TEI.citedRange]{citedRange} \hyperref[TEI.corr]{corr} \hyperref[TEI.date]{date} \hyperref[TEI.del]{del} \hyperref[TEI.desc]{desc} \hyperref[TEI.distinct]{distinct} \hyperref[TEI.editor]{editor} \hyperref[TEI.email]{email} \hyperref[TEI.emph]{emph} \hyperref[TEI.expan]{expan} \hyperref[TEI.foreign]{foreign} \hyperref[TEI.gloss]{gloss} \hyperref[TEI.head]{head} \hyperref[TEI.headItem]{headItem} \hyperref[TEI.headLabel]{headLabel} \hyperref[TEI.hi]{hi} \hyperref[TEI.item]{item} \hyperref[TEI.l]{l} \hyperref[TEI.label]{label} \hyperref[TEI.measure]{measure} \hyperref[TEI.measureGrp]{measureGrp} \hyperref[TEI.meeting]{meeting} \hyperref[TEI.mentioned]{mentioned} \hyperref[TEI.name]{name} \hyperref[TEI.note]{note} \hyperref[TEI.num]{num} \hyperref[TEI.orig]{orig} \hyperref[TEI.p]{p} \hyperref[TEI.pubPlace]{pubPlace} \hyperref[TEI.publisher]{publisher} \hyperref[TEI.q]{q} \hyperref[TEI.quote]{quote} \hyperref[TEI.ref]{ref} \hyperref[TEI.reg]{reg} \hyperref[TEI.resp]{resp} \hyperref[TEI.rs]{rs} \hyperref[TEI.said]{said} \hyperref[TEI.sic]{sic} \hyperref[TEI.soCalled]{soCalled} \hyperref[TEI.speaker]{speaker} \hyperref[TEI.stage]{stage} \hyperref[TEI.street]{street} \hyperref[TEI.term]{term} \hyperref[TEI.textLang]{textLang} \hyperref[TEI.time]{time} \hyperref[TEI.title]{title} \hyperref[TEI.unclear]{unclear}\par 
    \item[figures: ]
   \hyperref[TEI.cell]{cell} \hyperref[TEI.figDesc]{figDesc}\par 
    \item[header: ]
   \hyperref[TEI.authority]{authority} \hyperref[TEI.change]{change} \hyperref[TEI.classCode]{classCode} \hyperref[TEI.creation]{creation} \hyperref[TEI.distributor]{distributor} \hyperref[TEI.edition]{edition} \hyperref[TEI.extent]{extent} \hyperref[TEI.funder]{funder} \hyperref[TEI.language]{language} \hyperref[TEI.licence]{licence} \hyperref[TEI.rendition]{rendition}\par 
    \item[iso-fs: ]
   \hyperref[TEI.fDescr]{fDescr} \hyperref[TEI.fsDescr]{fsDescr}\par 
    \item[linking: ]
   \hyperref[TEI.ab]{ab} \hyperref[TEI.seg]{seg}\par 
    \item[msdescription: ]
   \hyperref[TEI.accMat]{accMat} \hyperref[TEI.acquisition]{acquisition} \hyperref[TEI.additions]{additions} \hyperref[TEI.catchwords]{catchwords} \hyperref[TEI.collation]{collation} \hyperref[TEI.colophon]{colophon} \hyperref[TEI.condition]{condition} \hyperref[TEI.custEvent]{custEvent} \hyperref[TEI.decoNote]{decoNote} \hyperref[TEI.explicit]{explicit} \hyperref[TEI.filiation]{filiation} \hyperref[TEI.finalRubric]{finalRubric} \hyperref[TEI.foliation]{foliation} \hyperref[TEI.heraldry]{heraldry} \hyperref[TEI.incipit]{incipit} \hyperref[TEI.layout]{layout} \hyperref[TEI.material]{material} \hyperref[TEI.musicNotation]{musicNotation} \hyperref[TEI.objectType]{objectType} \hyperref[TEI.origDate]{origDate} \hyperref[TEI.origPlace]{origPlace} \hyperref[TEI.origin]{origin} \hyperref[TEI.provenance]{provenance} \hyperref[TEI.rubric]{rubric} \hyperref[TEI.secFol]{secFol} \hyperref[TEI.signatures]{signatures} \hyperref[TEI.source]{source} \hyperref[TEI.stamp]{stamp} \hyperref[TEI.summary]{summary} \hyperref[TEI.support]{support} \hyperref[TEI.surrogates]{surrogates} \hyperref[TEI.typeNote]{typeNote} \hyperref[TEI.watermark]{watermark}\par 
    \item[namesdates: ]
   \hyperref[TEI.addName]{addName} \hyperref[TEI.affiliation]{affiliation} \hyperref[TEI.country]{country} \hyperref[TEI.forename]{forename} \hyperref[TEI.genName]{genName} \hyperref[TEI.geogName]{geogName} \hyperref[TEI.location]{location} \hyperref[TEI.nameLink]{nameLink} \hyperref[TEI.orgName]{orgName} \hyperref[TEI.persName]{persName} \hyperref[TEI.placeName]{placeName} \hyperref[TEI.region]{region} \hyperref[TEI.roleName]{roleName} \hyperref[TEI.settlement]{settlement} \hyperref[TEI.surname]{surname}\par 
    \item[textstructure: ]
   \hyperref[TEI.docAuthor]{docAuthor} \hyperref[TEI.docDate]{docDate} \hyperref[TEI.docEdition]{docEdition} \hyperref[TEI.titlePart]{titlePart}\par 
    \item[transcr: ]
   \hyperref[TEI.damage]{damage} \hyperref[TEI.fw]{fw} \hyperref[TEI.metamark]{metamark} \hyperref[TEI.mod]{mod} \hyperref[TEI.restore]{restore} \hyperref[TEI.retrace]{retrace} \hyperref[TEI.secl]{secl} \hyperref[TEI.supplied]{supplied} \hyperref[TEI.surplus]{surplus}
    \item[{Peut contenir}]
  
    \item[core: ]
   \hyperref[TEI.measure]{measure} \hyperref[TEI.measureGrp]{measureGrp} \hyperref[TEI.num]{num}\par 
    \item[msdescription: ]
   \hyperref[TEI.depth]{depth} \hyperref[TEI.dim]{dim} \hyperref[TEI.height]{height} \hyperref[TEI.width]{width}\par des données textuelles
    \item[{Exemple}]
  \leavevmode\bgroup\exampleFont \begin{shaded}\noindent\mbox{}{<\textbf{measureGrp}\hspace*{6pt}{type}="{leaves}"\hspace*{6pt}{unit}="{mm}">}\mbox{}\newline 
\hspace*{6pt}{<\textbf{height}\hspace*{6pt}{scope}="{range}">}157-160{</\textbf{height}>}\mbox{}\newline 
\hspace*{6pt}{<\textbf{width}\hspace*{6pt}{quantity}="{105}"/>}\mbox{}\newline 
{</\textbf{measureGrp}>}\mbox{}\newline 
{<\textbf{measureGrp}\hspace*{6pt}{type}="{ruledArea}"\hspace*{6pt}{unit}="{mm}">}\mbox{}\newline 
\hspace*{6pt}{<\textbf{height}\hspace*{6pt}{quantity}="{90}"\hspace*{6pt}{scope}="{most}"/>}\mbox{}\newline 
\hspace*{6pt}{<\textbf{width}\hspace*{6pt}{quantity}="{48}"\hspace*{6pt}{scope}="{most}"/>}\mbox{}\newline 
{</\textbf{measureGrp}>}\mbox{}\newline 
{<\textbf{measureGrp}\hspace*{6pt}{type}="{box}"\hspace*{6pt}{unit}="{in}">}\mbox{}\newline 
\hspace*{6pt}{<\textbf{height}\hspace*{6pt}{quantity}="{12}"/>}\mbox{}\newline 
\hspace*{6pt}{<\textbf{width}\hspace*{6pt}{quantity}="{10}"/>}\mbox{}\newline 
\hspace*{6pt}{<\textbf{depth}\hspace*{6pt}{quantity}="{6}"/>}\mbox{}\newline 
{</\textbf{measureGrp}>}\end{shaded}\egroup 


    \item[{Modèle de contenu}]
  \mbox{}\hfill\\[-10pt]\begin{Verbatim}[fontsize=\small]
<content>
 <alternate maxOccurs="unbounded"
  minOccurs="0">
  <textNode/>
  <classRef key="model.gLike"/>
  <classRef key="model.measureLike"/>
 </alternate>
</content>
    
\end{Verbatim}

    \item[{Schéma Declaration}]
  \mbox{}\hfill\\[-10pt]\begin{Verbatim}[fontsize=\small]
element measureGrp
{
   tei_att.global.attributes,
   tei_att.measurement.attributes,
   tei_att.typed.attributes,
   ( text | tei_model.gLike | tei_model.measureLike )*
}
\end{Verbatim}

\end{reflist}  \index{media=<media>|oddindex}\index{mimeType=@mimeType!<media>|oddindex}
\begin{reflist}
\item[]\begin{specHead}{TEI.media}{<media> }indicates the location of any form of external media such as an audio or video clip etc. [\xref{http://www.tei-c.org/release/doc/tei-p5-doc/en/html/CO.html\#COGR}{3.9. Graphics and Other Non-textual Components}]\end{specHead} 
    \item[{Module}]
  core
    \item[{Attributs}]
  Attributs \hyperref[TEI.att.typed]{att.typed} (\textit{@type}, \textit{@subtype}) \hyperref[TEI.att.global]{att.global} (\textit{@xml:id}, \textit{@n}, \textit{@xml:lang}, \textit{@xml:base}, \textit{@xml:space})  (\hyperref[TEI.att.global.rendition]{att.global.rendition} (\textit{@rend}, \textit{@style}, \textit{@rendition})) (\hyperref[TEI.att.global.linking]{att.global.linking} (\textit{@corresp}, \textit{@synch}, \textit{@sameAs}, \textit{@copyOf}, \textit{@next}, \textit{@prev}, \textit{@exclude}, \textit{@select})) (\hyperref[TEI.att.global.analytic]{att.global.analytic} (\textit{@ana})) (\hyperref[TEI.att.global.facs]{att.global.facs} (\textit{@facs})) (\hyperref[TEI.att.global.change]{att.global.change} (\textit{@change})) (\hyperref[TEI.att.global.responsibility]{att.global.responsibility} (\textit{@cert}, \textit{@resp})) (\hyperref[TEI.att.global.source]{att.global.source} (\textit{@source})) \hyperref[TEI.att.resourced]{att.resourced} (\textit{@url}) \hyperref[TEI.att.declaring]{att.declaring} (\textit{@decls}) \hyperref[TEI.att.timed]{att.timed} (\textit{@start}, \textit{@end})  (\hyperref[TEI.att.duration]{att.duration} (\hyperref[TEI.att.duration.w3c]{att.duration.w3c} (\textit{@dur})) (\hyperref[TEI.att.duration.iso]{att.duration.iso} (\textit{@dur-iso})) ) \hyperref[TEI.att.media]{att.media} (@width, @height, @scale) \hfil\\[-10pt]\begin{sansreflist}
    \item[@mimeType]
  (type de média MIME) spécifie le type MIME (multipurpose internet mail extension) applicable.
\begin{reflist}
    \item[{Dérivé de}]
  \hyperref[TEI.att.internetMedia]{att.internetMedia}
    \item[{Statut}]
  Requis
    \item[{Type de données}]
  1–∞ occurrences de \hyperref[TEI.teidata.word]{teidata.word} séparé par un espace
\end{reflist}  
\end{sansreflist}  
    \item[{Membre du}]
  \hyperref[TEI.model.graphicLike]{model.graphicLike}
    \item[{Contenu dans}]
  
    \item[analysis: ]
   \hyperref[TEI.cl]{cl} \hyperref[TEI.phr]{phr} \hyperref[TEI.s]{s}\par 
    \item[core: ]
   \hyperref[TEI.abbr]{abbr} \hyperref[TEI.add]{add} \hyperref[TEI.addrLine]{addrLine} \hyperref[TEI.author]{author} \hyperref[TEI.biblScope]{biblScope} \hyperref[TEI.citedRange]{citedRange} \hyperref[TEI.corr]{corr} \hyperref[TEI.date]{date} \hyperref[TEI.del]{del} \hyperref[TEI.distinct]{distinct} \hyperref[TEI.editor]{editor} \hyperref[TEI.email]{email} \hyperref[TEI.emph]{emph} \hyperref[TEI.expan]{expan} \hyperref[TEI.foreign]{foreign} \hyperref[TEI.gloss]{gloss} \hyperref[TEI.head]{head} \hyperref[TEI.headItem]{headItem} \hyperref[TEI.headLabel]{headLabel} \hyperref[TEI.hi]{hi} \hyperref[TEI.item]{item} \hyperref[TEI.l]{l} \hyperref[TEI.label]{label} \hyperref[TEI.measure]{measure} \hyperref[TEI.mentioned]{mentioned} \hyperref[TEI.name]{name} \hyperref[TEI.note]{note} \hyperref[TEI.num]{num} \hyperref[TEI.orig]{orig} \hyperref[TEI.p]{p} \hyperref[TEI.pubPlace]{pubPlace} \hyperref[TEI.publisher]{publisher} \hyperref[TEI.q]{q} \hyperref[TEI.quote]{quote} \hyperref[TEI.ref]{ref} \hyperref[TEI.reg]{reg} \hyperref[TEI.rs]{rs} \hyperref[TEI.said]{said} \hyperref[TEI.sic]{sic} \hyperref[TEI.soCalled]{soCalled} \hyperref[TEI.speaker]{speaker} \hyperref[TEI.stage]{stage} \hyperref[TEI.street]{street} \hyperref[TEI.term]{term} \hyperref[TEI.textLang]{textLang} \hyperref[TEI.time]{time} \hyperref[TEI.title]{title} \hyperref[TEI.unclear]{unclear}\par 
    \item[figures: ]
   \hyperref[TEI.cell]{cell} \hyperref[TEI.figDesc]{figDesc} \hyperref[TEI.figure]{figure} \hyperref[TEI.formula]{formula} \hyperref[TEI.table]{table}\par 
    \item[header: ]
   \hyperref[TEI.change]{change} \hyperref[TEI.distributor]{distributor} \hyperref[TEI.edition]{edition} \hyperref[TEI.extent]{extent} \hyperref[TEI.licence]{licence}\par 
    \item[linking: ]
   \hyperref[TEI.ab]{ab} \hyperref[TEI.seg]{seg}\par 
    \item[msdescription: ]
   \hyperref[TEI.accMat]{accMat} \hyperref[TEI.acquisition]{acquisition} \hyperref[TEI.additions]{additions} \hyperref[TEI.catchwords]{catchwords} \hyperref[TEI.collation]{collation} \hyperref[TEI.colophon]{colophon} \hyperref[TEI.condition]{condition} \hyperref[TEI.custEvent]{custEvent} \hyperref[TEI.decoNote]{decoNote} \hyperref[TEI.explicit]{explicit} \hyperref[TEI.filiation]{filiation} \hyperref[TEI.finalRubric]{finalRubric} \hyperref[TEI.foliation]{foliation} \hyperref[TEI.heraldry]{heraldry} \hyperref[TEI.incipit]{incipit} \hyperref[TEI.layout]{layout} \hyperref[TEI.material]{material} \hyperref[TEI.musicNotation]{musicNotation} \hyperref[TEI.objectType]{objectType} \hyperref[TEI.origDate]{origDate} \hyperref[TEI.origPlace]{origPlace} \hyperref[TEI.origin]{origin} \hyperref[TEI.provenance]{provenance} \hyperref[TEI.rubric]{rubric} \hyperref[TEI.secFol]{secFol} \hyperref[TEI.signatures]{signatures} \hyperref[TEI.source]{source} \hyperref[TEI.stamp]{stamp} \hyperref[TEI.summary]{summary} \hyperref[TEI.support]{support} \hyperref[TEI.surrogates]{surrogates} \hyperref[TEI.typeNote]{typeNote} \hyperref[TEI.watermark]{watermark}\par 
    \item[namesdates: ]
   \hyperref[TEI.addName]{addName} \hyperref[TEI.affiliation]{affiliation} \hyperref[TEI.country]{country} \hyperref[TEI.forename]{forename} \hyperref[TEI.genName]{genName} \hyperref[TEI.geogName]{geogName} \hyperref[TEI.nameLink]{nameLink} \hyperref[TEI.orgName]{orgName} \hyperref[TEI.persName]{persName} \hyperref[TEI.placeName]{placeName} \hyperref[TEI.region]{region} \hyperref[TEI.roleName]{roleName} \hyperref[TEI.settlement]{settlement} \hyperref[TEI.surname]{surname}\par 
    \item[textstructure: ]
   \hyperref[TEI.docAuthor]{docAuthor} \hyperref[TEI.docDate]{docDate} \hyperref[TEI.docEdition]{docEdition} \hyperref[TEI.titlePart]{titlePart}\par 
    \item[transcr: ]
   \hyperref[TEI.damage]{damage} \hyperref[TEI.facsimile]{facsimile} \hyperref[TEI.fw]{fw} \hyperref[TEI.metamark]{metamark} \hyperref[TEI.mod]{mod} \hyperref[TEI.restore]{restore} \hyperref[TEI.retrace]{retrace} \hyperref[TEI.secl]{secl} \hyperref[TEI.sourceDoc]{sourceDoc} \hyperref[TEI.supplied]{supplied} \hyperref[TEI.surface]{surface} \hyperref[TEI.surplus]{surplus} \hyperref[TEI.zone]{zone}
    \item[{Peut contenir}]
  
    \item[core: ]
   \hyperref[TEI.desc]{desc}
    \item[{Note}]
  \par
The attributes available for this element are not appropriate in all cases. For example, it makes no sense to specify the temporal duration of a graphic. Such errors are not currently detected.\par
The {\itshape mimeType} attribute must be used to specify the MIME media type of the resource specified by the {\itshape url} attribute.
    \item[{Exemple}]
  \leavevmode\bgroup\exampleFont \begin{shaded}\noindent\mbox{}{<\textbf{figure}>}\mbox{}\newline 
\hspace*{6pt}{<\textbf{media}\hspace*{6pt}{mimeType}="{image/png}"\hspace*{6pt}{url}="{fig1.png}"/>}\mbox{}\newline 
\hspace*{6pt}{<\textbf{head}>}Figure One: The View from the Bridge{</\textbf{head}>}\mbox{}\newline 
\hspace*{6pt}{<\textbf{figDesc}>}A Whistleresque view showing four or five sailing boats in the foreground, and a\mbox{}\newline 
\hspace*{6pt}\hspace*{6pt} series of buoys strung out between them.{</\textbf{figDesc}>}\mbox{}\newline 
{</\textbf{figure}>}\end{shaded}\egroup 


    \item[{Exemple}]
  \leavevmode\bgroup\exampleFont \begin{shaded}\noindent\mbox{}{<\textbf{media}\hspace*{6pt}{dur}="{PT10S}"\hspace*{6pt}{mimeType}="{audio/wav}"\mbox{}\newline 
\hspace*{6pt}{url}="{dingDong.wav}">}\mbox{}\newline 
\hspace*{6pt}{<\textbf{desc}>}Ten seconds of bellringing sound{</\textbf{desc}>}\mbox{}\newline 
{</\textbf{media}>}\end{shaded}\egroup 


    \item[{Exemple}]
  \leavevmode\bgroup\exampleFont \begin{shaded}\noindent\mbox{}{<\textbf{media}\hspace*{6pt}{dur}="{PT45M}"\hspace*{6pt}{mimeType}="{video/mp4}"\mbox{}\newline 
\hspace*{6pt}{url}="{clip45.mp4}"\hspace*{6pt}{width}="{500px}">}\mbox{}\newline 
\hspace*{6pt}{<\textbf{desc}>}A 45 minute video clip to be displayed in a window 500\mbox{}\newline 
\hspace*{6pt}\hspace*{6pt} px wide{</\textbf{desc}>}\mbox{}\newline 
{</\textbf{media}>}\end{shaded}\egroup 


    \item[{Modèle de contenu}]
  \mbox{}\hfill\\[-10pt]\begin{Verbatim}[fontsize=\small]
<content>
 <classRef key="model.descLike"
  maxOccurs="unbounded" minOccurs="0"/>
</content>
    
\end{Verbatim}

    \item[{Schéma Declaration}]
  \mbox{}\hfill\\[-10pt]\begin{Verbatim}[fontsize=\small]
element media
{
   tei_att.typed.attributes,
   tei_att.global.attributes,
   tei_att.media.attribute.width,
   tei_att.media.attribute.height,
   tei_att.media.attribute.scale,
   tei_att.resourced.attributes,
   tei_att.declaring.attributes,
   tei_att.timed.attributes,
   attribute mimeType { list { + } },
   tei_model.descLike*
}
\end{Verbatim}

\end{reflist}  \index{meeting=<meeting>|oddindex}
\begin{reflist}
\item[]\begin{specHead}{TEI.meeting}{<meeting> }contient le titre descriptif formalisé d’une réunion ou d’une conférence, employé dans une description bibliographique pour un article provenant d'une telle réunion, ou comme le titre ou le préambule aux publications qui en émanent. [\xref{http://www.tei-c.org/release/doc/tei-p5-doc/en/html/CO.html\#COBICOR}{3.11.2.2. Titles, Authors, and Editors}]\end{specHead} 
    \item[{Module}]
  core
    \item[{Attributs}]
  Attributs \hyperref[TEI.att.global]{att.global} (\textit{@xml:id}, \textit{@n}, \textit{@xml:lang}, \textit{@xml:base}, \textit{@xml:space})  (\hyperref[TEI.att.global.rendition]{att.global.rendition} (\textit{@rend}, \textit{@style}, \textit{@rendition})) (\hyperref[TEI.att.global.linking]{att.global.linking} (\textit{@corresp}, \textit{@synch}, \textit{@sameAs}, \textit{@copyOf}, \textit{@next}, \textit{@prev}, \textit{@exclude}, \textit{@select})) (\hyperref[TEI.att.global.analytic]{att.global.analytic} (\textit{@ana})) (\hyperref[TEI.att.global.facs]{att.global.facs} (\textit{@facs})) (\hyperref[TEI.att.global.change]{att.global.change} (\textit{@change})) (\hyperref[TEI.att.global.responsibility]{att.global.responsibility} (\textit{@cert}, \textit{@resp})) (\hyperref[TEI.att.global.source]{att.global.source} (\textit{@source})) \hyperref[TEI.att.canonical]{att.canonical} (\textit{@key}, \textit{@ref}) 
    \item[{Membre du}]
  \hyperref[TEI.model.divWrapper]{model.divWrapper} \hyperref[TEI.model.respLike]{model.respLike} 
    \item[{Contenu dans}]
  
    \item[core: ]
   \hyperref[TEI.bibl]{bibl} \hyperref[TEI.lg]{lg} \hyperref[TEI.list]{list} \hyperref[TEI.monogr]{monogr}\par 
    \item[figures: ]
   \hyperref[TEI.figure]{figure} \hyperref[TEI.table]{table}\par 
    \item[header: ]
   \hyperref[TEI.editionStmt]{editionStmt} \hyperref[TEI.titleStmt]{titleStmt}\par 
    \item[msdescription: ]
   \hyperref[TEI.msItem]{msItem}\par 
    \item[textstructure: ]
   \hyperref[TEI.body]{body} \hyperref[TEI.div]{div} \hyperref[TEI.front]{front} \hyperref[TEI.group]{group}
    \item[{Peut contenir}]
  
    \item[core: ]
   \hyperref[TEI.abbr]{abbr} \hyperref[TEI.address]{address} \hyperref[TEI.bibl]{bibl} \hyperref[TEI.biblStruct]{biblStruct} \hyperref[TEI.choice]{choice} \hyperref[TEI.cit]{cit} \hyperref[TEI.date]{date} \hyperref[TEI.desc]{desc} \hyperref[TEI.distinct]{distinct} \hyperref[TEI.email]{email} \hyperref[TEI.emph]{emph} \hyperref[TEI.expan]{expan} \hyperref[TEI.foreign]{foreign} \hyperref[TEI.gloss]{gloss} \hyperref[TEI.hi]{hi} \hyperref[TEI.label]{label} \hyperref[TEI.list]{list} \hyperref[TEI.listBibl]{listBibl} \hyperref[TEI.measure]{measure} \hyperref[TEI.measureGrp]{measureGrp} \hyperref[TEI.mentioned]{mentioned} \hyperref[TEI.name]{name} \hyperref[TEI.num]{num} \hyperref[TEI.ptr]{ptr} \hyperref[TEI.q]{q} \hyperref[TEI.quote]{quote} \hyperref[TEI.ref]{ref} \hyperref[TEI.rs]{rs} \hyperref[TEI.said]{said} \hyperref[TEI.soCalled]{soCalled} \hyperref[TEI.stage]{stage} \hyperref[TEI.term]{term} \hyperref[TEI.time]{time} \hyperref[TEI.title]{title}\par 
    \item[figures: ]
   \hyperref[TEI.table]{table}\par 
    \item[header: ]
   \hyperref[TEI.biblFull]{biblFull} \hyperref[TEI.idno]{idno}\par 
    \item[msdescription: ]
   \hyperref[TEI.catchwords]{catchwords} \hyperref[TEI.depth]{depth} \hyperref[TEI.dim]{dim} \hyperref[TEI.dimensions]{dimensions} \hyperref[TEI.height]{height} \hyperref[TEI.heraldry]{heraldry} \hyperref[TEI.locus]{locus} \hyperref[TEI.locusGrp]{locusGrp} \hyperref[TEI.material]{material} \hyperref[TEI.msDesc]{msDesc} \hyperref[TEI.objectType]{objectType} \hyperref[TEI.origDate]{origDate} \hyperref[TEI.origPlace]{origPlace} \hyperref[TEI.secFol]{secFol} \hyperref[TEI.signatures]{signatures} \hyperref[TEI.stamp]{stamp} \hyperref[TEI.watermark]{watermark} \hyperref[TEI.width]{width}\par 
    \item[namesdates: ]
   \hyperref[TEI.addName]{addName} \hyperref[TEI.affiliation]{affiliation} \hyperref[TEI.country]{country} \hyperref[TEI.forename]{forename} \hyperref[TEI.genName]{genName} \hyperref[TEI.geogName]{geogName} \hyperref[TEI.listOrg]{listOrg} \hyperref[TEI.listPlace]{listPlace} \hyperref[TEI.location]{location} \hyperref[TEI.nameLink]{nameLink} \hyperref[TEI.orgName]{orgName} \hyperref[TEI.persName]{persName} \hyperref[TEI.placeName]{placeName} \hyperref[TEI.region]{region} \hyperref[TEI.roleName]{roleName} \hyperref[TEI.settlement]{settlement} \hyperref[TEI.state]{state} \hyperref[TEI.surname]{surname}\par 
    \item[textstructure: ]
   \hyperref[TEI.floatingText]{floatingText}\par 
    \item[transcr: ]
   \hyperref[TEI.am]{am} \hyperref[TEI.ex]{ex} \hyperref[TEI.subst]{subst}\par des données textuelles
    \item[{Exemple}]
  \leavevmode\bgroup\exampleFont \begin{shaded}\noindent\mbox{}{<\textbf{div}>}\mbox{}\newline 
\hspace*{6pt}{<\textbf{meeting}>}Colloque international : Duras, marges et transgressions, Nancy, 1er et 2 avril\mbox{}\newline 
\hspace*{6pt}\hspace*{6pt} 2005{</\textbf{meeting}>}\mbox{}\newline 
\hspace*{6pt}{<\textbf{list}\hspace*{6pt}{type}="{attendance}">}\mbox{}\newline 
\hspace*{6pt}\hspace*{6pt}{<\textbf{head}>}liste des participants{</\textbf{head}>}\mbox{}\newline 
\hspace*{6pt}\hspace*{6pt}{<\textbf{item}>}\mbox{}\newline 
\hspace*{6pt}\hspace*{6pt}\hspace*{6pt}{<\textbf{persName}>}...{</\textbf{persName}>}\mbox{}\newline 
\hspace*{6pt}\hspace*{6pt}{</\textbf{item}>}\mbox{}\newline 
\hspace*{6pt}\hspace*{6pt}{<\textbf{item}>}\mbox{}\newline 
\hspace*{6pt}\hspace*{6pt}\hspace*{6pt}{<\textbf{persName}>}...{</\textbf{persName}>}\mbox{}\newline 
\hspace*{6pt}\hspace*{6pt}{</\textbf{item}>}\mbox{}\newline 
\hspace*{6pt}{</\textbf{list}>}\mbox{}\newline 
\hspace*{6pt}{<\textbf{p}>}...{</\textbf{p}>}\mbox{}\newline 
{</\textbf{div}>}\end{shaded}\egroup 


    \item[{Modèle de contenu}]
  \mbox{}\hfill\\[-10pt]\begin{Verbatim}[fontsize=\small]
<content>
 <macroRef key="macro.limitedContent"/>
</content>
    
\end{Verbatim}

    \item[{Schéma Declaration}]
  \mbox{}\hfill\\[-10pt]\begin{Verbatim}[fontsize=\small]
element meeting
{
   tei_att.global.attributes,
   tei_att.canonical.attributes,
   tei_macro.limitedContent}
\end{Verbatim}

\end{reflist}  \index{menclose=<menclose>|oddindex}\index{notation=@notation!<menclose>|oddindex}
\begin{reflist}
\item[]\begin{specHead}{TEI.menclose}{<menclose> }\end{specHead} 
    \item[{Namespace}]
  http://www.w3.org/1998/Math/MathML
    \item[{Module}]
  derived-module-tei.istex
    \item[{Attributs}]
  Attributs\hfil\\[-10pt]\begin{sansreflist}
    \item[@notation]
  
\begin{reflist}
    \item[{Statut}]
  Optionel
    \item[{Type de données}]
  \xref{https://www.w3.org/TR/xmlschema-2/\#}{}
\end{reflist}  
\end{sansreflist}  
    \item[{Contenu dans}]
  
    \item[derived-module-tei.istex: ]
   \hyperref[TEI.math]{math} \hyperref[TEI.menclose]{menclose} \hyperref[TEI.mfenced]{mfenced} \hyperref[TEI.mfrac]{mfrac} \hyperref[TEI.mmultiscripts]{mmultiscripts} \hyperref[TEI.mover]{mover} \hyperref[TEI.mpadded]{mpadded} \hyperref[TEI.mphantom]{mphantom} \hyperref[TEI.mprescripts]{mprescripts} \hyperref[TEI.mrow]{mrow} \hyperref[TEI.msqrt]{msqrt} \hyperref[TEI.mstyle]{mstyle} \hyperref[TEI.msub]{msub} \hyperref[TEI.msubsup]{msubsup} \hyperref[TEI.msup]{msup} \hyperref[TEI.msupsub]{msupsub} \hyperref[TEI.mtable]{mtable} \hyperref[TEI.mtd]{mtd} \hyperref[TEI.mtr]{mtr} \hyperref[TEI.munder]{munder} \hyperref[TEI.munderover]{munderover} \hyperref[TEI.semantics]{semantics}
    \item[{Peut contenir}]
  
    \item[derived-module-tei.istex: ]
   \hyperref[TEI.menclose]{menclose} \hyperref[TEI.mfenced]{mfenced} \hyperref[TEI.mfrac]{mfrac} \hyperref[TEI.mi]{mi} \hyperref[TEI.mmultiscripts]{mmultiscripts} \hyperref[TEI.mn]{mn} \hyperref[TEI.mo]{mo} \hyperref[TEI.mover]{mover} \hyperref[TEI.mpadded]{mpadded} \hyperref[TEI.mphantom]{mphantom} \hyperref[TEI.mprescripts]{mprescripts} \hyperref[TEI.mrow]{mrow} \hyperref[TEI.mspace]{mspace} \hyperref[TEI.msqrt]{msqrt} \hyperref[TEI.mstyle]{mstyle} \hyperref[TEI.msub]{msub} \hyperref[TEI.msubsup]{msubsup} \hyperref[TEI.msup]{msup} \hyperref[TEI.msupsub]{msupsub} \hyperref[TEI.mtable]{mtable} \hyperref[TEI.mtd]{mtd} \hyperref[TEI.mtext]{mtext} \hyperref[TEI.mtr]{mtr} \hyperref[TEI.munder]{munder} \hyperref[TEI.munderover]{munderover} \hyperref[TEI.none]{none}\par des données textuelles
    \item[{Modèle de contenu}]
  \mbox{}\hfill\\[-10pt]\begin{Verbatim}[fontsize=\small]
<content>
 <alternate maxOccurs="unbounded"
  minOccurs="0">
  <textNode/>
  <elementRef key="mstyle"/>
  <elementRef key="mtr"/>
  <elementRef key="mtd"/>
  <elementRef key="mrow"/>
  <elementRef key="mi"/>
  <elementRef key="mn"/>
  <elementRef key="mtext"/>
  <elementRef key="mfrac"/>
  <elementRef key="mspace"/>
  <elementRef key="msqrt"/>
  <elementRef key="msub"/>
  <elementRef key="msup"/>
  <elementRef key="mo"/>
  <elementRef key="mover"/>
  <elementRef key="mfenced"/>
  <elementRef key="mtable"/>
  <elementRef key="msubsup"/>
  <elementRef key="msupsub"/>
  <elementRef key="mmultiscripts"/>
  <elementRef key="munderover"/>
  <elementRef key="mprescripts"/>
  <elementRef key="none"/>
  <elementRef key="munder"/>
  <elementRef key="mphantom"/>
  <elementRef key="mpadded"/>
  <elementRef key="menclose"/>
 </alternate>
</content>
    
\end{Verbatim}

    \item[{Schéma Declaration}]
  \mbox{}\hfill\\[-10pt]\begin{Verbatim}[fontsize=\small]
element menclose
{
   attribute notation { notation }?,
   (
      text
    | tei_mstyle    | tei_mtr    | tei_mtd    | tei_mrow    | tei_mi    | tei_mn    | tei_mtext    | tei_mfrac    | tei_mspace    | tei_msqrt    | tei_msub    | tei_msup    | tei_mo    | tei_mover    | tei_mfenced    | tei_mtable    | tei_msubsup    | tei_msupsub    | tei_mmultiscripts    | tei_munderover    | tei_mprescripts    | tei_none    | tei_munder    | tei_mphantom    | tei_mpadded    | tei_menclose   )*
}
\end{Verbatim}

\end{reflist}  \index{mentioned=<mentioned>|oddindex}
\begin{reflist}
\item[]\begin{specHead}{TEI.mentioned}{<mentioned> }marque des mots ou des expressions employés métalinguistiquement [\xref{http://www.tei-c.org/release/doc/tei-p5-doc/en/html/CO.html\#COHQQ}{3.3.3. Quotation}]\end{specHead} 
    \item[{Module}]
  core
    \item[{Attributs}]
  Attributs \hyperref[TEI.att.global]{att.global} (\textit{@xml:id}, \textit{@n}, \textit{@xml:lang}, \textit{@xml:base}, \textit{@xml:space})  (\hyperref[TEI.att.global.rendition]{att.global.rendition} (\textit{@rend}, \textit{@style}, \textit{@rendition})) (\hyperref[TEI.att.global.linking]{att.global.linking} (\textit{@corresp}, \textit{@synch}, \textit{@sameAs}, \textit{@copyOf}, \textit{@next}, \textit{@prev}, \textit{@exclude}, \textit{@select})) (\hyperref[TEI.att.global.analytic]{att.global.analytic} (\textit{@ana})) (\hyperref[TEI.att.global.facs]{att.global.facs} (\textit{@facs})) (\hyperref[TEI.att.global.change]{att.global.change} (\textit{@change})) (\hyperref[TEI.att.global.responsibility]{att.global.responsibility} (\textit{@cert}, \textit{@resp})) (\hyperref[TEI.att.global.source]{att.global.source} (\textit{@source}))
    \item[{Membre du}]
  \hyperref[TEI.model.emphLike]{model.emphLike}
    \item[{Contenu dans}]
  
    \item[analysis: ]
   \hyperref[TEI.cl]{cl} \hyperref[TEI.phr]{phr} \hyperref[TEI.s]{s} \hyperref[TEI.span]{span}\par 
    \item[core: ]
   \hyperref[TEI.abbr]{abbr} \hyperref[TEI.add]{add} \hyperref[TEI.addrLine]{addrLine} \hyperref[TEI.author]{author} \hyperref[TEI.bibl]{bibl} \hyperref[TEI.biblScope]{biblScope} \hyperref[TEI.citedRange]{citedRange} \hyperref[TEI.corr]{corr} \hyperref[TEI.date]{date} \hyperref[TEI.del]{del} \hyperref[TEI.desc]{desc} \hyperref[TEI.distinct]{distinct} \hyperref[TEI.editor]{editor} \hyperref[TEI.email]{email} \hyperref[TEI.emph]{emph} \hyperref[TEI.expan]{expan} \hyperref[TEI.foreign]{foreign} \hyperref[TEI.gloss]{gloss} \hyperref[TEI.head]{head} \hyperref[TEI.headItem]{headItem} \hyperref[TEI.headLabel]{headLabel} \hyperref[TEI.hi]{hi} \hyperref[TEI.item]{item} \hyperref[TEI.l]{l} \hyperref[TEI.label]{label} \hyperref[TEI.measure]{measure} \hyperref[TEI.meeting]{meeting} \hyperref[TEI.mentioned]{mentioned} \hyperref[TEI.name]{name} \hyperref[TEI.note]{note} \hyperref[TEI.num]{num} \hyperref[TEI.orig]{orig} \hyperref[TEI.p]{p} \hyperref[TEI.pubPlace]{pubPlace} \hyperref[TEI.publisher]{publisher} \hyperref[TEI.q]{q} \hyperref[TEI.quote]{quote} \hyperref[TEI.ref]{ref} \hyperref[TEI.reg]{reg} \hyperref[TEI.resp]{resp} \hyperref[TEI.rs]{rs} \hyperref[TEI.said]{said} \hyperref[TEI.sic]{sic} \hyperref[TEI.soCalled]{soCalled} \hyperref[TEI.speaker]{speaker} \hyperref[TEI.stage]{stage} \hyperref[TEI.street]{street} \hyperref[TEI.term]{term} \hyperref[TEI.textLang]{textLang} \hyperref[TEI.time]{time} \hyperref[TEI.title]{title} \hyperref[TEI.unclear]{unclear}\par 
    \item[figures: ]
   \hyperref[TEI.cell]{cell} \hyperref[TEI.figDesc]{figDesc}\par 
    \item[header: ]
   \hyperref[TEI.authority]{authority} \hyperref[TEI.change]{change} \hyperref[TEI.classCode]{classCode} \hyperref[TEI.creation]{creation} \hyperref[TEI.distributor]{distributor} \hyperref[TEI.edition]{edition} \hyperref[TEI.extent]{extent} \hyperref[TEI.funder]{funder} \hyperref[TEI.language]{language} \hyperref[TEI.licence]{licence} \hyperref[TEI.rendition]{rendition}\par 
    \item[iso-fs: ]
   \hyperref[TEI.fDescr]{fDescr} \hyperref[TEI.fsDescr]{fsDescr}\par 
    \item[linking: ]
   \hyperref[TEI.ab]{ab} \hyperref[TEI.seg]{seg}\par 
    \item[msdescription: ]
   \hyperref[TEI.accMat]{accMat} \hyperref[TEI.acquisition]{acquisition} \hyperref[TEI.additions]{additions} \hyperref[TEI.catchwords]{catchwords} \hyperref[TEI.collation]{collation} \hyperref[TEI.colophon]{colophon} \hyperref[TEI.condition]{condition} \hyperref[TEI.custEvent]{custEvent} \hyperref[TEI.decoNote]{decoNote} \hyperref[TEI.explicit]{explicit} \hyperref[TEI.filiation]{filiation} \hyperref[TEI.finalRubric]{finalRubric} \hyperref[TEI.foliation]{foliation} \hyperref[TEI.heraldry]{heraldry} \hyperref[TEI.incipit]{incipit} \hyperref[TEI.layout]{layout} \hyperref[TEI.material]{material} \hyperref[TEI.musicNotation]{musicNotation} \hyperref[TEI.objectType]{objectType} \hyperref[TEI.origDate]{origDate} \hyperref[TEI.origPlace]{origPlace} \hyperref[TEI.origin]{origin} \hyperref[TEI.provenance]{provenance} \hyperref[TEI.rubric]{rubric} \hyperref[TEI.secFol]{secFol} \hyperref[TEI.signatures]{signatures} \hyperref[TEI.source]{source} \hyperref[TEI.stamp]{stamp} \hyperref[TEI.summary]{summary} \hyperref[TEI.support]{support} \hyperref[TEI.surrogates]{surrogates} \hyperref[TEI.typeNote]{typeNote} \hyperref[TEI.watermark]{watermark}\par 
    \item[namesdates: ]
   \hyperref[TEI.addName]{addName} \hyperref[TEI.affiliation]{affiliation} \hyperref[TEI.country]{country} \hyperref[TEI.forename]{forename} \hyperref[TEI.genName]{genName} \hyperref[TEI.geogName]{geogName} \hyperref[TEI.nameLink]{nameLink} \hyperref[TEI.orgName]{orgName} \hyperref[TEI.persName]{persName} \hyperref[TEI.placeName]{placeName} \hyperref[TEI.region]{region} \hyperref[TEI.roleName]{roleName} \hyperref[TEI.settlement]{settlement} \hyperref[TEI.surname]{surname}\par 
    \item[textstructure: ]
   \hyperref[TEI.docAuthor]{docAuthor} \hyperref[TEI.docDate]{docDate} \hyperref[TEI.docEdition]{docEdition} \hyperref[TEI.titlePart]{titlePart}\par 
    \item[transcr: ]
   \hyperref[TEI.damage]{damage} \hyperref[TEI.fw]{fw} \hyperref[TEI.metamark]{metamark} \hyperref[TEI.mod]{mod} \hyperref[TEI.restore]{restore} \hyperref[TEI.retrace]{retrace} \hyperref[TEI.secl]{secl} \hyperref[TEI.supplied]{supplied} \hyperref[TEI.surplus]{surplus}
    \item[{Peut contenir}]
  
    \item[analysis: ]
   \hyperref[TEI.c]{c} \hyperref[TEI.cl]{cl} \hyperref[TEI.interp]{interp} \hyperref[TEI.interpGrp]{interpGrp} \hyperref[TEI.m]{m} \hyperref[TEI.pc]{pc} \hyperref[TEI.phr]{phr} \hyperref[TEI.s]{s} \hyperref[TEI.span]{span} \hyperref[TEI.spanGrp]{spanGrp} \hyperref[TEI.w]{w}\par 
    \item[core: ]
   \hyperref[TEI.abbr]{abbr} \hyperref[TEI.add]{add} \hyperref[TEI.address]{address} \hyperref[TEI.binaryObject]{binaryObject} \hyperref[TEI.cb]{cb} \hyperref[TEI.choice]{choice} \hyperref[TEI.corr]{corr} \hyperref[TEI.date]{date} \hyperref[TEI.del]{del} \hyperref[TEI.distinct]{distinct} \hyperref[TEI.email]{email} \hyperref[TEI.emph]{emph} \hyperref[TEI.expan]{expan} \hyperref[TEI.foreign]{foreign} \hyperref[TEI.gap]{gap} \hyperref[TEI.gb]{gb} \hyperref[TEI.gloss]{gloss} \hyperref[TEI.graphic]{graphic} \hyperref[TEI.hi]{hi} \hyperref[TEI.index]{index} \hyperref[TEI.lb]{lb} \hyperref[TEI.measure]{measure} \hyperref[TEI.measureGrp]{measureGrp} \hyperref[TEI.media]{media} \hyperref[TEI.mentioned]{mentioned} \hyperref[TEI.milestone]{milestone} \hyperref[TEI.name]{name} \hyperref[TEI.note]{note} \hyperref[TEI.num]{num} \hyperref[TEI.orig]{orig} \hyperref[TEI.pb]{pb} \hyperref[TEI.ptr]{ptr} \hyperref[TEI.ref]{ref} \hyperref[TEI.reg]{reg} \hyperref[TEI.rs]{rs} \hyperref[TEI.sic]{sic} \hyperref[TEI.soCalled]{soCalled} \hyperref[TEI.term]{term} \hyperref[TEI.time]{time} \hyperref[TEI.title]{title} \hyperref[TEI.unclear]{unclear}\par 
    \item[derived-module-tei.istex: ]
   \hyperref[TEI.math]{math} \hyperref[TEI.mrow]{mrow}\par 
    \item[figures: ]
   \hyperref[TEI.figure]{figure} \hyperref[TEI.formula]{formula} \hyperref[TEI.notatedMusic]{notatedMusic}\par 
    \item[header: ]
   \hyperref[TEI.idno]{idno}\par 
    \item[iso-fs: ]
   \hyperref[TEI.fLib]{fLib} \hyperref[TEI.fs]{fs} \hyperref[TEI.fvLib]{fvLib}\par 
    \item[linking: ]
   \hyperref[TEI.alt]{alt} \hyperref[TEI.altGrp]{altGrp} \hyperref[TEI.anchor]{anchor} \hyperref[TEI.join]{join} \hyperref[TEI.joinGrp]{joinGrp} \hyperref[TEI.link]{link} \hyperref[TEI.linkGrp]{linkGrp} \hyperref[TEI.seg]{seg} \hyperref[TEI.timeline]{timeline}\par 
    \item[msdescription: ]
   \hyperref[TEI.catchwords]{catchwords} \hyperref[TEI.depth]{depth} \hyperref[TEI.dim]{dim} \hyperref[TEI.dimensions]{dimensions} \hyperref[TEI.height]{height} \hyperref[TEI.heraldry]{heraldry} \hyperref[TEI.locus]{locus} \hyperref[TEI.locusGrp]{locusGrp} \hyperref[TEI.material]{material} \hyperref[TEI.objectType]{objectType} \hyperref[TEI.origDate]{origDate} \hyperref[TEI.origPlace]{origPlace} \hyperref[TEI.secFol]{secFol} \hyperref[TEI.signatures]{signatures} \hyperref[TEI.source]{source} \hyperref[TEI.stamp]{stamp} \hyperref[TEI.watermark]{watermark} \hyperref[TEI.width]{width}\par 
    \item[namesdates: ]
   \hyperref[TEI.addName]{addName} \hyperref[TEI.affiliation]{affiliation} \hyperref[TEI.country]{country} \hyperref[TEI.forename]{forename} \hyperref[TEI.genName]{genName} \hyperref[TEI.geogName]{geogName} \hyperref[TEI.location]{location} \hyperref[TEI.nameLink]{nameLink} \hyperref[TEI.orgName]{orgName} \hyperref[TEI.persName]{persName} \hyperref[TEI.placeName]{placeName} \hyperref[TEI.region]{region} \hyperref[TEI.roleName]{roleName} \hyperref[TEI.settlement]{settlement} \hyperref[TEI.state]{state} \hyperref[TEI.surname]{surname}\par 
    \item[spoken: ]
   \hyperref[TEI.annotationBlock]{annotationBlock}\par 
    \item[transcr: ]
   \hyperref[TEI.addSpan]{addSpan} \hyperref[TEI.am]{am} \hyperref[TEI.damage]{damage} \hyperref[TEI.damageSpan]{damageSpan} \hyperref[TEI.delSpan]{delSpan} \hyperref[TEI.ex]{ex} \hyperref[TEI.fw]{fw} \hyperref[TEI.handShift]{handShift} \hyperref[TEI.listTranspose]{listTranspose} \hyperref[TEI.metamark]{metamark} \hyperref[TEI.mod]{mod} \hyperref[TEI.redo]{redo} \hyperref[TEI.restore]{restore} \hyperref[TEI.retrace]{retrace} \hyperref[TEI.secl]{secl} \hyperref[TEI.space]{space} \hyperref[TEI.subst]{subst} \hyperref[TEI.substJoin]{substJoin} \hyperref[TEI.supplied]{supplied} \hyperref[TEI.surplus]{surplus} \hyperref[TEI.undo]{undo}\par des données textuelles
    \item[{Exemple}]
  \leavevmode\bgroup\exampleFont \begin{shaded}\noindent\mbox{}Aucune ville ne répond mieux à\mbox{}\newline 
 l'expressioin {<\textbf{mentioned}>}sortie de terre{</\textbf{mentioned}>} que New York\mbox{}\newline 
 (ou faudrait-il plutôt dire {<\textbf{mentioned}>}jaillie{</\textbf{mentioned}>}) :\end{shaded}\egroup 


    \item[{Exemple}]
  \leavevmode\bgroup\exampleFont \begin{shaded}\noindent\mbox{} L’harmonisation\mbox{}\newline 
 vocalique régressive empêche que {<\textbf{mentioned}>}agwêdê{</\textbf{mentioned}>} puisse être interprété comme\mbox{}\newline 
 un dérivé de {<\textbf{mentioned}>}gwada{</\textbf{mentioned}>}, qui pourtant est de même racine.\end{shaded}\egroup 


    \item[{Modèle de contenu}]
  \mbox{}\hfill\\[-10pt]\begin{Verbatim}[fontsize=\small]
<content>
 <macroRef key="macro.phraseSeq"/>
</content>
    
\end{Verbatim}

    \item[{Schéma Declaration}]
  \mbox{}\hfill\\[-10pt]\begin{Verbatim}[fontsize=\small]
element mentioned { tei_att.global.attributes, tei_macro.phraseSeq }
\end{Verbatim}

\end{reflist}  \index{metamark=<metamark>|oddindex}\index{function=@function!<metamark>|oddindex}\index{target=@target!<metamark>|oddindex}
\begin{reflist}
\item[]\begin{specHead}{TEI.metamark}{<metamark> }contains or describes any kind of graphic or written signal within a document the function of which is to determine how it should be read rather than forming part of the actual content of the document. [\xref{http://www.tei-c.org/release/doc/tei-p5-doc/en/html/PH.html\#PH-meta}{11.3.4.2. Metamarks}]\end{specHead} 
    \item[{Module}]
  transcr
    \item[{Attributs}]
  Attributs \hyperref[TEI.att.spanning]{att.spanning} (\textit{@spanTo}) \hyperref[TEI.att.placement]{att.placement} (\textit{@place}) \hyperref[TEI.att.global]{att.global} (\textit{@xml:id}, \textit{@n}, \textit{@xml:lang}, \textit{@xml:base}, \textit{@xml:space})  (\hyperref[TEI.att.global.rendition]{att.global.rendition} (\textit{@rend}, \textit{@style}, \textit{@rendition})) (\hyperref[TEI.att.global.linking]{att.global.linking} (\textit{@corresp}, \textit{@synch}, \textit{@sameAs}, \textit{@copyOf}, \textit{@next}, \textit{@prev}, \textit{@exclude}, \textit{@select})) (\hyperref[TEI.att.global.analytic]{att.global.analytic} (\textit{@ana})) (\hyperref[TEI.att.global.facs]{att.global.facs} (\textit{@facs})) (\hyperref[TEI.att.global.change]{att.global.change} (\textit{@change})) (\hyperref[TEI.att.global.responsibility]{att.global.responsibility} (\textit{@cert}, \textit{@resp})) (\hyperref[TEI.att.global.source]{att.global.source} (\textit{@source})) \hfil\\[-10pt]\begin{sansreflist}
    \item[@function]
  describes the function (for example status, insertion, deletion, transposition) of the metamark.
\begin{reflist}
    \item[{Statut}]
  Optionel
    \item[{Type de données}]
  \hyperref[TEI.teidata.word]{teidata.word}
\end{reflist}  
    \item[@target]
  identifies one or more elements to which the metamark applies.
\begin{reflist}
    \item[{Statut}]
  Optionel
    \item[{Type de données}]
  1–∞ occurrences de \hyperref[TEI.teidata.pointer]{teidata.pointer} séparé par un espace
\end{reflist}  
\end{sansreflist}  
    \item[{Membre du}]
  \hyperref[TEI.model.global]{model.global}
    \item[{Contenu dans}]
  
    \item[analysis: ]
   \hyperref[TEI.cl]{cl} \hyperref[TEI.m]{m} \hyperref[TEI.phr]{phr} \hyperref[TEI.s]{s} \hyperref[TEI.span]{span} \hyperref[TEI.w]{w}\par 
    \item[core: ]
   \hyperref[TEI.abbr]{abbr} \hyperref[TEI.add]{add} \hyperref[TEI.addrLine]{addrLine} \hyperref[TEI.address]{address} \hyperref[TEI.author]{author} \hyperref[TEI.bibl]{bibl} \hyperref[TEI.biblScope]{biblScope} \hyperref[TEI.cit]{cit} \hyperref[TEI.citedRange]{citedRange} \hyperref[TEI.corr]{corr} \hyperref[TEI.date]{date} \hyperref[TEI.del]{del} \hyperref[TEI.distinct]{distinct} \hyperref[TEI.editor]{editor} \hyperref[TEI.email]{email} \hyperref[TEI.emph]{emph} \hyperref[TEI.expan]{expan} \hyperref[TEI.foreign]{foreign} \hyperref[TEI.gloss]{gloss} \hyperref[TEI.head]{head} \hyperref[TEI.headItem]{headItem} \hyperref[TEI.headLabel]{headLabel} \hyperref[TEI.hi]{hi} \hyperref[TEI.imprint]{imprint} \hyperref[TEI.item]{item} \hyperref[TEI.l]{l} \hyperref[TEI.label]{label} \hyperref[TEI.lg]{lg} \hyperref[TEI.list]{list} \hyperref[TEI.measure]{measure} \hyperref[TEI.mentioned]{mentioned} \hyperref[TEI.name]{name} \hyperref[TEI.note]{note} \hyperref[TEI.num]{num} \hyperref[TEI.orig]{orig} \hyperref[TEI.p]{p} \hyperref[TEI.pubPlace]{pubPlace} \hyperref[TEI.publisher]{publisher} \hyperref[TEI.q]{q} \hyperref[TEI.quote]{quote} \hyperref[TEI.ref]{ref} \hyperref[TEI.reg]{reg} \hyperref[TEI.resp]{resp} \hyperref[TEI.rs]{rs} \hyperref[TEI.said]{said} \hyperref[TEI.series]{series} \hyperref[TEI.sic]{sic} \hyperref[TEI.soCalled]{soCalled} \hyperref[TEI.sp]{sp} \hyperref[TEI.speaker]{speaker} \hyperref[TEI.stage]{stage} \hyperref[TEI.street]{street} \hyperref[TEI.term]{term} \hyperref[TEI.textLang]{textLang} \hyperref[TEI.time]{time} \hyperref[TEI.title]{title} \hyperref[TEI.unclear]{unclear}\par 
    \item[figures: ]
   \hyperref[TEI.cell]{cell} \hyperref[TEI.figure]{figure} \hyperref[TEI.table]{table}\par 
    \item[header: ]
   \hyperref[TEI.authority]{authority} \hyperref[TEI.change]{change} \hyperref[TEI.classCode]{classCode} \hyperref[TEI.distributor]{distributor} \hyperref[TEI.edition]{edition} \hyperref[TEI.extent]{extent} \hyperref[TEI.funder]{funder} \hyperref[TEI.language]{language} \hyperref[TEI.licence]{licence}\par 
    \item[linking: ]
   \hyperref[TEI.ab]{ab} \hyperref[TEI.seg]{seg}\par 
    \item[msdescription: ]
   \hyperref[TEI.accMat]{accMat} \hyperref[TEI.acquisition]{acquisition} \hyperref[TEI.additions]{additions} \hyperref[TEI.catchwords]{catchwords} \hyperref[TEI.collation]{collation} \hyperref[TEI.colophon]{colophon} \hyperref[TEI.condition]{condition} \hyperref[TEI.custEvent]{custEvent} \hyperref[TEI.decoNote]{decoNote} \hyperref[TEI.explicit]{explicit} \hyperref[TEI.filiation]{filiation} \hyperref[TEI.finalRubric]{finalRubric} \hyperref[TEI.foliation]{foliation} \hyperref[TEI.heraldry]{heraldry} \hyperref[TEI.incipit]{incipit} \hyperref[TEI.layout]{layout} \hyperref[TEI.material]{material} \hyperref[TEI.msItem]{msItem} \hyperref[TEI.musicNotation]{musicNotation} \hyperref[TEI.objectType]{objectType} \hyperref[TEI.origDate]{origDate} \hyperref[TEI.origPlace]{origPlace} \hyperref[TEI.origin]{origin} \hyperref[TEI.provenance]{provenance} \hyperref[TEI.rubric]{rubric} \hyperref[TEI.secFol]{secFol} \hyperref[TEI.signatures]{signatures} \hyperref[TEI.source]{source} \hyperref[TEI.stamp]{stamp} \hyperref[TEI.summary]{summary} \hyperref[TEI.support]{support} \hyperref[TEI.surrogates]{surrogates} \hyperref[TEI.typeNote]{typeNote} \hyperref[TEI.watermark]{watermark}\par 
    \item[namesdates: ]
   \hyperref[TEI.addName]{addName} \hyperref[TEI.affiliation]{affiliation} \hyperref[TEI.country]{country} \hyperref[TEI.forename]{forename} \hyperref[TEI.genName]{genName} \hyperref[TEI.geogName]{geogName} \hyperref[TEI.nameLink]{nameLink} \hyperref[TEI.orgName]{orgName} \hyperref[TEI.persName]{persName} \hyperref[TEI.person]{person} \hyperref[TEI.personGrp]{personGrp} \hyperref[TEI.persona]{persona} \hyperref[TEI.placeName]{placeName} \hyperref[TEI.region]{region} \hyperref[TEI.roleName]{roleName} \hyperref[TEI.settlement]{settlement} \hyperref[TEI.surname]{surname}\par 
    \item[textstructure: ]
   \hyperref[TEI.back]{back} \hyperref[TEI.body]{body} \hyperref[TEI.div]{div} \hyperref[TEI.docAuthor]{docAuthor} \hyperref[TEI.docDate]{docDate} \hyperref[TEI.docEdition]{docEdition} \hyperref[TEI.docTitle]{docTitle} \hyperref[TEI.floatingText]{floatingText} \hyperref[TEI.front]{front} \hyperref[TEI.group]{group} \hyperref[TEI.text]{text} \hyperref[TEI.titlePage]{titlePage} \hyperref[TEI.titlePart]{titlePart}\par 
    \item[transcr: ]
   \hyperref[TEI.damage]{damage} \hyperref[TEI.fw]{fw} \hyperref[TEI.line]{line} \hyperref[TEI.metamark]{metamark} \hyperref[TEI.mod]{mod} \hyperref[TEI.restore]{restore} \hyperref[TEI.retrace]{retrace} \hyperref[TEI.secl]{secl} \hyperref[TEI.sourceDoc]{sourceDoc} \hyperref[TEI.supplied]{supplied} \hyperref[TEI.surface]{surface} \hyperref[TEI.surfaceGrp]{surfaceGrp} \hyperref[TEI.surplus]{surplus} \hyperref[TEI.zone]{zone}
    \item[{Peut contenir}]
  
    \item[analysis: ]
   \hyperref[TEI.c]{c} \hyperref[TEI.cl]{cl} \hyperref[TEI.interp]{interp} \hyperref[TEI.interpGrp]{interpGrp} \hyperref[TEI.m]{m} \hyperref[TEI.pc]{pc} \hyperref[TEI.phr]{phr} \hyperref[TEI.s]{s} \hyperref[TEI.span]{span} \hyperref[TEI.spanGrp]{spanGrp} \hyperref[TEI.w]{w}\par 
    \item[core: ]
   \hyperref[TEI.abbr]{abbr} \hyperref[TEI.add]{add} \hyperref[TEI.address]{address} \hyperref[TEI.bibl]{bibl} \hyperref[TEI.biblStruct]{biblStruct} \hyperref[TEI.binaryObject]{binaryObject} \hyperref[TEI.cb]{cb} \hyperref[TEI.choice]{choice} \hyperref[TEI.cit]{cit} \hyperref[TEI.corr]{corr} \hyperref[TEI.date]{date} \hyperref[TEI.del]{del} \hyperref[TEI.desc]{desc} \hyperref[TEI.distinct]{distinct} \hyperref[TEI.email]{email} \hyperref[TEI.emph]{emph} \hyperref[TEI.expan]{expan} \hyperref[TEI.foreign]{foreign} \hyperref[TEI.gap]{gap} \hyperref[TEI.gb]{gb} \hyperref[TEI.gloss]{gloss} \hyperref[TEI.graphic]{graphic} \hyperref[TEI.hi]{hi} \hyperref[TEI.index]{index} \hyperref[TEI.l]{l} \hyperref[TEI.label]{label} \hyperref[TEI.lb]{lb} \hyperref[TEI.lg]{lg} \hyperref[TEI.list]{list} \hyperref[TEI.listBibl]{listBibl} \hyperref[TEI.measure]{measure} \hyperref[TEI.measureGrp]{measureGrp} \hyperref[TEI.media]{media} \hyperref[TEI.mentioned]{mentioned} \hyperref[TEI.milestone]{milestone} \hyperref[TEI.name]{name} \hyperref[TEI.note]{note} \hyperref[TEI.num]{num} \hyperref[TEI.orig]{orig} \hyperref[TEI.p]{p} \hyperref[TEI.pb]{pb} \hyperref[TEI.ptr]{ptr} \hyperref[TEI.q]{q} \hyperref[TEI.quote]{quote} \hyperref[TEI.ref]{ref} \hyperref[TEI.reg]{reg} \hyperref[TEI.rs]{rs} \hyperref[TEI.said]{said} \hyperref[TEI.sic]{sic} \hyperref[TEI.soCalled]{soCalled} \hyperref[TEI.sp]{sp} \hyperref[TEI.stage]{stage} \hyperref[TEI.term]{term} \hyperref[TEI.time]{time} \hyperref[TEI.title]{title} \hyperref[TEI.unclear]{unclear}\par 
    \item[derived-module-tei.istex: ]
   \hyperref[TEI.math]{math} \hyperref[TEI.mrow]{mrow}\par 
    \item[figures: ]
   \hyperref[TEI.figure]{figure} \hyperref[TEI.formula]{formula} \hyperref[TEI.notatedMusic]{notatedMusic} \hyperref[TEI.table]{table}\par 
    \item[header: ]
   \hyperref[TEI.biblFull]{biblFull} \hyperref[TEI.idno]{idno}\par 
    \item[iso-fs: ]
   \hyperref[TEI.fLib]{fLib} \hyperref[TEI.fs]{fs} \hyperref[TEI.fvLib]{fvLib}\par 
    \item[linking: ]
   \hyperref[TEI.ab]{ab} \hyperref[TEI.alt]{alt} \hyperref[TEI.altGrp]{altGrp} \hyperref[TEI.anchor]{anchor} \hyperref[TEI.join]{join} \hyperref[TEI.joinGrp]{joinGrp} \hyperref[TEI.link]{link} \hyperref[TEI.linkGrp]{linkGrp} \hyperref[TEI.seg]{seg} \hyperref[TEI.timeline]{timeline}\par 
    \item[msdescription: ]
   \hyperref[TEI.catchwords]{catchwords} \hyperref[TEI.depth]{depth} \hyperref[TEI.dim]{dim} \hyperref[TEI.dimensions]{dimensions} \hyperref[TEI.height]{height} \hyperref[TEI.heraldry]{heraldry} \hyperref[TEI.locus]{locus} \hyperref[TEI.locusGrp]{locusGrp} \hyperref[TEI.material]{material} \hyperref[TEI.msDesc]{msDesc} \hyperref[TEI.objectType]{objectType} \hyperref[TEI.origDate]{origDate} \hyperref[TEI.origPlace]{origPlace} \hyperref[TEI.secFol]{secFol} \hyperref[TEI.signatures]{signatures} \hyperref[TEI.source]{source} \hyperref[TEI.stamp]{stamp} \hyperref[TEI.watermark]{watermark} \hyperref[TEI.width]{width}\par 
    \item[namesdates: ]
   \hyperref[TEI.addName]{addName} \hyperref[TEI.affiliation]{affiliation} \hyperref[TEI.country]{country} \hyperref[TEI.forename]{forename} \hyperref[TEI.genName]{genName} \hyperref[TEI.geogName]{geogName} \hyperref[TEI.listOrg]{listOrg} \hyperref[TEI.listPlace]{listPlace} \hyperref[TEI.location]{location} \hyperref[TEI.nameLink]{nameLink} \hyperref[TEI.orgName]{orgName} \hyperref[TEI.persName]{persName} \hyperref[TEI.placeName]{placeName} \hyperref[TEI.region]{region} \hyperref[TEI.roleName]{roleName} \hyperref[TEI.settlement]{settlement} \hyperref[TEI.state]{state} \hyperref[TEI.surname]{surname}\par 
    \item[spoken: ]
   \hyperref[TEI.annotationBlock]{annotationBlock}\par 
    \item[textstructure: ]
   \hyperref[TEI.floatingText]{floatingText}\par 
    \item[transcr: ]
   \hyperref[TEI.addSpan]{addSpan} \hyperref[TEI.am]{am} \hyperref[TEI.damage]{damage} \hyperref[TEI.damageSpan]{damageSpan} \hyperref[TEI.delSpan]{delSpan} \hyperref[TEI.ex]{ex} \hyperref[TEI.fw]{fw} \hyperref[TEI.handShift]{handShift} \hyperref[TEI.listTranspose]{listTranspose} \hyperref[TEI.metamark]{metamark} \hyperref[TEI.mod]{mod} \hyperref[TEI.redo]{redo} \hyperref[TEI.restore]{restore} \hyperref[TEI.retrace]{retrace} \hyperref[TEI.secl]{secl} \hyperref[TEI.space]{space} \hyperref[TEI.subst]{subst} \hyperref[TEI.substJoin]{substJoin} \hyperref[TEI.supplied]{supplied} \hyperref[TEI.surplus]{surplus} \hyperref[TEI.undo]{undo}\par des données textuelles
    \item[{Exemple}]
  \leavevmode\bgroup\exampleFont \begin{shaded}\noindent\mbox{}{<\textbf{surface}>}\mbox{}\newline 
\hspace*{6pt}{<\textbf{metamark}\hspace*{6pt}{function}="{used}"\hspace*{6pt}{rend}="{line}"\mbox{}\newline 
\hspace*{6pt}\hspace*{6pt}{target}="{\#X2}"/>}\mbox{}\newline 
\hspace*{6pt}{<\textbf{zone}\hspace*{6pt}{xml:id}="{zone-X2}">}\mbox{}\newline 
\hspace*{6pt}\hspace*{6pt}{<\textbf{line}>}I am that halfgrown {<\textbf{add}>}angry{</\textbf{add}>} boy, fallen asleep{</\textbf{line}>}\mbox{}\newline 
\hspace*{6pt}\hspace*{6pt}{<\textbf{line}>}The tears of foolish passion yet undried{</\textbf{line}>}\mbox{}\newline 
\hspace*{6pt}\hspace*{6pt}{<\textbf{line}>}upon my cheeks.{</\textbf{line}>}\mbox{}\newline 
\textit{<!-- ... -->}\mbox{}\newline 
\hspace*{6pt}\hspace*{6pt}{<\textbf{line}>}I pass through {<\textbf{add}>}the{</\textbf{add}>} travels and {<\textbf{del}>}fortunes{</\textbf{del}>} of\mbox{}\newline 
\hspace*{6pt}\hspace*{6pt}{<\textbf{retrace}>}thirty{</\textbf{retrace}>}\mbox{}\newline 
\hspace*{6pt}\hspace*{6pt}{</\textbf{line}>}\mbox{}\newline 
\hspace*{6pt}\hspace*{6pt}{<\textbf{line}>}years and become old,{</\textbf{line}>}\mbox{}\newline 
\hspace*{6pt}\hspace*{6pt}{<\textbf{line}>}Each in its due order comes and goes,{</\textbf{line}>}\mbox{}\newline 
\hspace*{6pt}\hspace*{6pt}{<\textbf{line}>}And thus a message for me comes.{</\textbf{line}>}\mbox{}\newline 
\hspace*{6pt}\hspace*{6pt}{<\textbf{line}>}The{</\textbf{line}>}\mbox{}\newline 
\hspace*{6pt}{</\textbf{zone}>}\mbox{}\newline 
\hspace*{6pt}{<\textbf{metamark}\hspace*{6pt}{function}="{used}"\mbox{}\newline 
\hspace*{6pt}\hspace*{6pt}{target}="{\#zone-X2}">}Entered - Yes{</\textbf{metamark}>}\mbox{}\newline 
{</\textbf{surface}>}\end{shaded}\egroup 


    \item[{Modèle de contenu}]
  \mbox{}\hfill\\[-10pt]\begin{Verbatim}[fontsize=\small]
<content>
 <macroRef key="macro.specialPara"/>
</content>
    
\end{Verbatim}

    \item[{Schéma Declaration}]
  \mbox{}\hfill\\[-10pt]\begin{Verbatim}[fontsize=\small]
element metamark
{
   tei_att.spanning.attributes,
   tei_att.placement.attributes,
   tei_att.global.attributes,
   attribute function { text }?,
   attribute target { list { + } }?,
   tei_macro.specialPara}
\end{Verbatim}

\end{reflist}  \index{mfenced=<mfenced>|oddindex}\index{open=@open!<mfenced>|oddindex}\index{close=@close!<mfenced>|oddindex}\index{separators=@separators!<mfenced>|oddindex}
\begin{reflist}
\item[]\begin{specHead}{TEI.mfenced}{<mfenced> }\end{specHead} 
    \item[{Namespace}]
  http://www.w3.org/1998/Math/MathML
    \item[{Module}]
  derived-module-tei.istex
    \item[{Attributs}]
  Attributs\hfil\\[-10pt]\begin{sansreflist}
    \item[@open]
  
\begin{reflist}
    \item[{Statut}]
  Optionel
    \item[{Type de données}]
  \xref{https://www.w3.org/TR/xmlschema-2/\#}{}
\end{reflist}  
    \item[@close]
  
\begin{reflist}
    \item[{Statut}]
  Optionel
    \item[{Type de données}]
  \xref{https://www.w3.org/TR/xmlschema-2/\#}{}
\end{reflist}  
    \item[@separators]
  
\begin{reflist}
    \item[{Statut}]
  Optionel
    \item[{Type de données}]
  \xref{https://www.w3.org/TR/xmlschema-2/\#}{}
\end{reflist}  
\end{sansreflist}  
    \item[{Contenu dans}]
  
    \item[derived-module-tei.istex: ]
   \hyperref[TEI.math]{math} \hyperref[TEI.menclose]{menclose} \hyperref[TEI.mfenced]{mfenced} \hyperref[TEI.mfrac]{mfrac} \hyperref[TEI.mmultiscripts]{mmultiscripts} \hyperref[TEI.mover]{mover} \hyperref[TEI.mpadded]{mpadded} \hyperref[TEI.mphantom]{mphantom} \hyperref[TEI.mprescripts]{mprescripts} \hyperref[TEI.mrow]{mrow} \hyperref[TEI.msqrt]{msqrt} \hyperref[TEI.mstyle]{mstyle} \hyperref[TEI.msub]{msub} \hyperref[TEI.msubsup]{msubsup} \hyperref[TEI.msup]{msup} \hyperref[TEI.msupsub]{msupsub} \hyperref[TEI.mtable]{mtable} \hyperref[TEI.mtd]{mtd} \hyperref[TEI.mtr]{mtr} \hyperref[TEI.munder]{munder} \hyperref[TEI.munderover]{munderover} \hyperref[TEI.semantics]{semantics}
    \item[{Peut contenir}]
  
    \item[derived-module-tei.istex: ]
   \hyperref[TEI.menclose]{menclose} \hyperref[TEI.mfenced]{mfenced} \hyperref[TEI.mfrac]{mfrac} \hyperref[TEI.mi]{mi} \hyperref[TEI.mmultiscripts]{mmultiscripts} \hyperref[TEI.mn]{mn} \hyperref[TEI.mo]{mo} \hyperref[TEI.mover]{mover} \hyperref[TEI.mpadded]{mpadded} \hyperref[TEI.mphantom]{mphantom} \hyperref[TEI.mprescripts]{mprescripts} \hyperref[TEI.mrow]{mrow} \hyperref[TEI.mspace]{mspace} \hyperref[TEI.msqrt]{msqrt} \hyperref[TEI.mstyle]{mstyle} \hyperref[TEI.msub]{msub} \hyperref[TEI.msubsup]{msubsup} \hyperref[TEI.msup]{msup} \hyperref[TEI.msupsub]{msupsub} \hyperref[TEI.mtable]{mtable} \hyperref[TEI.mtd]{mtd} \hyperref[TEI.mtext]{mtext} \hyperref[TEI.mtr]{mtr} \hyperref[TEI.munder]{munder} \hyperref[TEI.munderover]{munderover} \hyperref[TEI.none]{none}\par des données textuelles
    \item[{Modèle de contenu}]
  \mbox{}\hfill\\[-10pt]\begin{Verbatim}[fontsize=\small]
<content>
 <alternate maxOccurs="unbounded"
  minOccurs="0">
  <textNode/>
  <elementRef key="mstyle"/>
  <elementRef key="mtr"/>
  <elementRef key="mtd"/>
  <elementRef key="mrow"/>
  <elementRef key="mi"/>
  <elementRef key="mn"/>
  <elementRef key="mtext"/>
  <elementRef key="mfrac"/>
  <elementRef key="mspace"/>
  <elementRef key="msqrt"/>
  <elementRef key="msub"/>
  <elementRef key="msup"/>
  <elementRef key="mo"/>
  <elementRef key="mover"/>
  <elementRef key="mfenced"/>
  <elementRef key="mtable"/>
  <elementRef key="msubsup"/>
  <elementRef key="msupsub"/>
  <elementRef key="mmultiscripts"/>
  <elementRef key="munderover"/>
  <elementRef key="mprescripts"/>
  <elementRef key="none"/>
  <elementRef key="munder"/>
  <elementRef key="mphantom"/>
  <elementRef key="mpadded"/>
  <elementRef key="menclose"/>
 </alternate>
</content>
    
\end{Verbatim}

    \item[{Schéma Declaration}]
  \mbox{}\hfill\\[-10pt]\begin{Verbatim}[fontsize=\small]
element mfenced
{
   attribute open { open }?,
   attribute close { close }?,
   attribute separators { separators }?,
   (
      text
    | tei_mstyle    | tei_mtr    | tei_mtd    | tei_mrow    | tei_mi    | tei_mn    | tei_mtext    | tei_mfrac    | tei_mspace    | tei_msqrt    | tei_msub    | tei_msup    | tei_mo    | tei_mover    | tei_mfenced    | tei_mtable    | tei_msubsup    | tei_msupsub    | tei_mmultiscripts    | tei_munderover    | tei_mprescripts    | tei_none    | tei_munder    | tei_mphantom    | tei_mpadded    | tei_menclose   )*
}
\end{Verbatim}

\end{reflist}  \index{mfrac=<mfrac>|oddindex}\index{bevelled=@bevelled!<mfrac>|oddindex}\index{linethickness=@linethickness!<mfrac>|oddindex}
\begin{reflist}
\item[]\begin{specHead}{TEI.mfrac}{<mfrac> }\end{specHead} 
    \item[{Namespace}]
  http://www.w3.org/1998/Math/MathML
    \item[{Module}]
  derived-module-tei.istex
    \item[{Attributs}]
  Attributs\hfil\\[-10pt]\begin{sansreflist}
    \item[@bevelled]
  
\begin{reflist}
    \item[{Statut}]
  Optionel
    \item[{Type de données}]
  \xref{https://www.w3.org/TR/xmlschema-2/\#}{}
\end{reflist}  
    \item[@linethickness]
  
\begin{reflist}
    \item[{Statut}]
  Optionel
    \item[{Type de données}]
  \xref{https://www.w3.org/TR/xmlschema-2/\#}{}
\end{reflist}  
\end{sansreflist}  
    \item[{Contenu dans}]
  
    \item[derived-module-tei.istex: ]
   \hyperref[TEI.math]{math} \hyperref[TEI.menclose]{menclose} \hyperref[TEI.mfenced]{mfenced} \hyperref[TEI.mfrac]{mfrac} \hyperref[TEI.mmultiscripts]{mmultiscripts} \hyperref[TEI.mover]{mover} \hyperref[TEI.mpadded]{mpadded} \hyperref[TEI.mphantom]{mphantom} \hyperref[TEI.mprescripts]{mprescripts} \hyperref[TEI.mrow]{mrow} \hyperref[TEI.msqrt]{msqrt} \hyperref[TEI.mstyle]{mstyle} \hyperref[TEI.msub]{msub} \hyperref[TEI.msubsup]{msubsup} \hyperref[TEI.msup]{msup} \hyperref[TEI.msupsub]{msupsub} \hyperref[TEI.mtable]{mtable} \hyperref[TEI.mtd]{mtd} \hyperref[TEI.mtr]{mtr} \hyperref[TEI.munder]{munder} \hyperref[TEI.munderover]{munderover} \hyperref[TEI.semantics]{semantics}
    \item[{Peut contenir}]
  
    \item[derived-module-tei.istex: ]
   \hyperref[TEI.menclose]{menclose} \hyperref[TEI.mfenced]{mfenced} \hyperref[TEI.mfrac]{mfrac} \hyperref[TEI.mi]{mi} \hyperref[TEI.mmultiscripts]{mmultiscripts} \hyperref[TEI.mn]{mn} \hyperref[TEI.mo]{mo} \hyperref[TEI.mover]{mover} \hyperref[TEI.mpadded]{mpadded} \hyperref[TEI.mphantom]{mphantom} \hyperref[TEI.mprescripts]{mprescripts} \hyperref[TEI.mrow]{mrow} \hyperref[TEI.mspace]{mspace} \hyperref[TEI.msqrt]{msqrt} \hyperref[TEI.mstyle]{mstyle} \hyperref[TEI.msub]{msub} \hyperref[TEI.msubsup]{msubsup} \hyperref[TEI.msup]{msup} \hyperref[TEI.msupsub]{msupsub} \hyperref[TEI.mtable]{mtable} \hyperref[TEI.mtd]{mtd} \hyperref[TEI.mtext]{mtext} \hyperref[TEI.mtr]{mtr} \hyperref[TEI.munder]{munder} \hyperref[TEI.munderover]{munderover} \hyperref[TEI.none]{none}\par des données textuelles
    \item[{Modèle de contenu}]
  \mbox{}\hfill\\[-10pt]\begin{Verbatim}[fontsize=\small]
<content>
 <alternate maxOccurs="unbounded"
  minOccurs="0">
  <textNode/>
  <elementRef key="mstyle"/>
  <elementRef key="mtr"/>
  <elementRef key="mtd"/>
  <elementRef key="mrow"/>
  <elementRef key="mi"/>
  <elementRef key="mn"/>
  <elementRef key="mtext"/>
  <elementRef key="mfrac"/>
  <elementRef key="mspace"/>
  <elementRef key="msqrt"/>
  <elementRef key="msub"/>
  <elementRef key="msup"/>
  <elementRef key="mo"/>
  <elementRef key="mover"/>
  <elementRef key="mfenced"/>
  <elementRef key="mtable"/>
  <elementRef key="msubsup"/>
  <elementRef key="msupsub"/>
  <elementRef key="mmultiscripts"/>
  <elementRef key="munderover"/>
  <elementRef key="mprescripts"/>
  <elementRef key="none"/>
  <elementRef key="munder"/>
  <elementRef key="mphantom"/>
  <elementRef key="mpadded"/>
  <elementRef key="menclose"/>
 </alternate>
</content>
    
\end{Verbatim}

    \item[{Schéma Declaration}]
  \mbox{}\hfill\\[-10pt]\begin{Verbatim}[fontsize=\small]
element mfrac
{
   attribute bevelled { bevelled }?,
   attribute linethickness { linethickness }?,
   (
      text
    | tei_mstyle    | tei_mtr    | tei_mtd    | tei_mrow    | tei_mi    | tei_mn    | tei_mtext    | tei_mfrac    | tei_mspace    | tei_msqrt    | tei_msub    | tei_msup    | tei_mo    | tei_mover    | tei_mfenced    | tei_mtable    | tei_msubsup    | tei_msupsub    | tei_mmultiscripts    | tei_munderover    | tei_mprescripts    | tei_none    | tei_munder    | tei_mphantom    | tei_mpadded    | tei_menclose   )*
}
\end{Verbatim}

\end{reflist}  \index{mi=<mi>|oddindex}\index{stretchy=@stretchy!<mi>|oddindex}\index{mathvariant=@mathvariant!<mi>|oddindex}\index{fontstyle=@fontstyle!<mi>|oddindex}\index{fontweight=@fontweight!<mi>|oddindex}\index{mathsize=@mathsize!<mi>|oddindex}\index{movablelimits=@movablelimits!<mi>|oddindex}\index{class=@class!<mi>|oddindex}\index{accent=@accent!<mi>|oddindex}\index{form=@form!<mi>|oddindex}\index{fence=@fence!<mi>|oddindex}
\begin{reflist}
\item[]\begin{specHead}{TEI.mi}{<mi> }\end{specHead} 
    \item[{Namespace}]
  http://www.w3.org/1998/Math/MathML
    \item[{Module}]
  derived-module-tei.istex
    \item[{Attributs}]
  Attributs\hfil\\[-10pt]\begin{sansreflist}
    \item[@stretchy]
  
\begin{reflist}
    \item[{Statut}]
  Optionel
    \item[{Type de données}]
  \xref{https://www.w3.org/TR/xmlschema-2/\#}{}
\end{reflist}  
    \item[@mathvariant]
  
\begin{reflist}
    \item[{Statut}]
  Optionel
    \item[{Type de données}]
  \xref{https://www.w3.org/TR/xmlschema-2/\#}{}
\end{reflist}  
    \item[@fontstyle]
  
\begin{reflist}
    \item[{Statut}]
  Optionel
    \item[{Type de données}]
  \xref{https://www.w3.org/TR/xmlschema-2/\#}{}
\end{reflist}  
    \item[@fontweight]
  
\begin{reflist}
    \item[{Statut}]
  Optionel
    \item[{Type de données}]
  \xref{https://www.w3.org/TR/xmlschema-2/\#}{}
\end{reflist}  
    \item[@mathsize]
  
\begin{reflist}
    \item[{Statut}]
  Optionel
    \item[{Type de données}]
  \xref{https://www.w3.org/TR/xmlschema-2/\#}{}
\end{reflist}  
    \item[@movablelimits]
  
\begin{reflist}
    \item[{Statut}]
  Optionel
    \item[{Type de données}]
  \xref{https://www.w3.org/TR/xmlschema-2/\#}{}
\end{reflist}  
    \item[@class]
  
\begin{reflist}
    \item[{Statut}]
  Optionel
    \item[{Type de données}]
  \xref{https://www.w3.org/TR/xmlschema-2/\#}{}
\end{reflist}  
    \item[@accent]
  
\begin{reflist}
    \item[{Statut}]
  Optionel
    \item[{Type de données}]
  \xref{https://www.w3.org/TR/xmlschema-2/\#}{}
\end{reflist}  
    \item[@form]
  
\begin{reflist}
    \item[{Statut}]
  Optionel
    \item[{Type de données}]
  \xref{https://www.w3.org/TR/xmlschema-2/\#}{}
\end{reflist}  
    \item[@fence]
  
\begin{reflist}
    \item[{Statut}]
  Optionel
    \item[{Type de données}]
  \xref{https://www.w3.org/TR/xmlschema-2/\#}{}
\end{reflist}  
\end{sansreflist}  
    \item[{Contenu dans}]
  
    \item[derived-module-tei.istex: ]
   \hyperref[TEI.math]{math} \hyperref[TEI.menclose]{menclose} \hyperref[TEI.mfenced]{mfenced} \hyperref[TEI.mfrac]{mfrac} \hyperref[TEI.mmultiscripts]{mmultiscripts} \hyperref[TEI.mover]{mover} \hyperref[TEI.mpadded]{mpadded} \hyperref[TEI.mphantom]{mphantom} \hyperref[TEI.mprescripts]{mprescripts} \hyperref[TEI.mrow]{mrow} \hyperref[TEI.msqrt]{msqrt} \hyperref[TEI.mstyle]{mstyle} \hyperref[TEI.msub]{msub} \hyperref[TEI.msubsup]{msubsup} \hyperref[TEI.msup]{msup} \hyperref[TEI.msupsub]{msupsub} \hyperref[TEI.mtable]{mtable} \hyperref[TEI.mtd]{mtd} \hyperref[TEI.mtr]{mtr} \hyperref[TEI.munder]{munder} \hyperref[TEI.munderover]{munderover} \hyperref[TEI.semantics]{semantics}
    \item[{Peut contenir}]
  Des données textuelles uniquement
    \item[{Modèle de contenu}]
  \fbox{\ttfamily <content>\newline
 <textNode/>\newline
</content>\newline
    } 
    \item[{Schéma Declaration}]
  \mbox{}\hfill\\[-10pt]\begin{Verbatim}[fontsize=\small]
element mi
{
   attribute stretchy { stretchy }?,
   attribute mathvariant { mathvariant }?,
   attribute fontstyle { fontstyle }?,
   attribute fontweight { fontweight }?,
   attribute mathsize { mathsize }?,
   attribute movablelimits { movablelimits }?,
   attribute class { class }?,
   attribute accent { accent }?,
   attribute form { form }?,
   attribute fence { fence }?,
   text
}
\end{Verbatim}

\end{reflist}  \index{milestone=<milestone>|oddindex}
\begin{reflist}
\item[]\begin{specHead}{TEI.milestone}{<milestone> }(borne) marque un point permettant de délimiter les sections d'un texte selon un autre systeme que les éléments de structure ; une balise de ce type marque une frontière. [\xref{http://www.tei-c.org/release/doc/tei-p5-doc/en/html/CO.html\#CORS5}{3.10.3. Milestone Elements}]\end{specHead} 
    \item[{Module}]
  core
    \item[{Attributs}]
  Attributs \hyperref[TEI.att.global]{att.global} (\textit{@xml:id}, \textit{@n}, \textit{@xml:lang}, \textit{@xml:base}, \textit{@xml:space})  (\hyperref[TEI.att.global.rendition]{att.global.rendition} (\textit{@rend}, \textit{@style}, \textit{@rendition})) (\hyperref[TEI.att.global.linking]{att.global.linking} (\textit{@corresp}, \textit{@synch}, \textit{@sameAs}, \textit{@copyOf}, \textit{@next}, \textit{@prev}, \textit{@exclude}, \textit{@select})) (\hyperref[TEI.att.global.analytic]{att.global.analytic} (\textit{@ana})) (\hyperref[TEI.att.global.facs]{att.global.facs} (\textit{@facs})) (\hyperref[TEI.att.global.change]{att.global.change} (\textit{@change})) (\hyperref[TEI.att.global.responsibility]{att.global.responsibility} (\textit{@cert}, \textit{@resp})) (\hyperref[TEI.att.global.source]{att.global.source} (\textit{@source})) \hyperref[TEI.att.milestoneUnit]{att.milestoneUnit} (\textit{@unit}) \hyperref[TEI.att.typed]{att.typed} (\textit{@type}, \textit{@subtype}) \hyperref[TEI.att.edition]{att.edition} (\textit{@ed}, \textit{@edRef}) \hyperref[TEI.att.spanning]{att.spanning} (\textit{@spanTo}) \hyperref[TEI.att.breaking]{att.breaking} (\textit{@break}) 
    \item[{Membre du}]
  \hyperref[TEI.model.milestoneLike]{model.milestoneLike}
    \item[{Contenu dans}]
  
    \item[analysis: ]
   \hyperref[TEI.cl]{cl} \hyperref[TEI.m]{m} \hyperref[TEI.phr]{phr} \hyperref[TEI.s]{s} \hyperref[TEI.span]{span} \hyperref[TEI.w]{w}\par 
    \item[core: ]
   \hyperref[TEI.abbr]{abbr} \hyperref[TEI.add]{add} \hyperref[TEI.addrLine]{addrLine} \hyperref[TEI.address]{address} \hyperref[TEI.author]{author} \hyperref[TEI.bibl]{bibl} \hyperref[TEI.biblScope]{biblScope} \hyperref[TEI.cit]{cit} \hyperref[TEI.citedRange]{citedRange} \hyperref[TEI.corr]{corr} \hyperref[TEI.date]{date} \hyperref[TEI.del]{del} \hyperref[TEI.distinct]{distinct} \hyperref[TEI.editor]{editor} \hyperref[TEI.email]{email} \hyperref[TEI.emph]{emph} \hyperref[TEI.expan]{expan} \hyperref[TEI.foreign]{foreign} \hyperref[TEI.gloss]{gloss} \hyperref[TEI.head]{head} \hyperref[TEI.headItem]{headItem} \hyperref[TEI.headLabel]{headLabel} \hyperref[TEI.hi]{hi} \hyperref[TEI.imprint]{imprint} \hyperref[TEI.item]{item} \hyperref[TEI.l]{l} \hyperref[TEI.label]{label} \hyperref[TEI.lg]{lg} \hyperref[TEI.list]{list} \hyperref[TEI.listBibl]{listBibl} \hyperref[TEI.measure]{measure} \hyperref[TEI.mentioned]{mentioned} \hyperref[TEI.name]{name} \hyperref[TEI.note]{note} \hyperref[TEI.num]{num} \hyperref[TEI.orig]{orig} \hyperref[TEI.p]{p} \hyperref[TEI.pubPlace]{pubPlace} \hyperref[TEI.publisher]{publisher} \hyperref[TEI.q]{q} \hyperref[TEI.quote]{quote} \hyperref[TEI.ref]{ref} \hyperref[TEI.reg]{reg} \hyperref[TEI.resp]{resp} \hyperref[TEI.rs]{rs} \hyperref[TEI.said]{said} \hyperref[TEI.series]{series} \hyperref[TEI.sic]{sic} \hyperref[TEI.soCalled]{soCalled} \hyperref[TEI.sp]{sp} \hyperref[TEI.speaker]{speaker} \hyperref[TEI.stage]{stage} \hyperref[TEI.street]{street} \hyperref[TEI.term]{term} \hyperref[TEI.textLang]{textLang} \hyperref[TEI.time]{time} \hyperref[TEI.title]{title} \hyperref[TEI.unclear]{unclear}\par 
    \item[figures: ]
   \hyperref[TEI.cell]{cell} \hyperref[TEI.figure]{figure} \hyperref[TEI.table]{table}\par 
    \item[header: ]
   \hyperref[TEI.authority]{authority} \hyperref[TEI.change]{change} \hyperref[TEI.classCode]{classCode} \hyperref[TEI.distributor]{distributor} \hyperref[TEI.edition]{edition} \hyperref[TEI.extent]{extent} \hyperref[TEI.funder]{funder} \hyperref[TEI.language]{language} \hyperref[TEI.licence]{licence}\par 
    \item[linking: ]
   \hyperref[TEI.ab]{ab} \hyperref[TEI.seg]{seg}\par 
    \item[msdescription: ]
   \hyperref[TEI.accMat]{accMat} \hyperref[TEI.acquisition]{acquisition} \hyperref[TEI.additions]{additions} \hyperref[TEI.catchwords]{catchwords} \hyperref[TEI.collation]{collation} \hyperref[TEI.colophon]{colophon} \hyperref[TEI.condition]{condition} \hyperref[TEI.custEvent]{custEvent} \hyperref[TEI.decoNote]{decoNote} \hyperref[TEI.explicit]{explicit} \hyperref[TEI.filiation]{filiation} \hyperref[TEI.finalRubric]{finalRubric} \hyperref[TEI.foliation]{foliation} \hyperref[TEI.heraldry]{heraldry} \hyperref[TEI.incipit]{incipit} \hyperref[TEI.layout]{layout} \hyperref[TEI.material]{material} \hyperref[TEI.msItem]{msItem} \hyperref[TEI.musicNotation]{musicNotation} \hyperref[TEI.objectType]{objectType} \hyperref[TEI.origDate]{origDate} \hyperref[TEI.origPlace]{origPlace} \hyperref[TEI.origin]{origin} \hyperref[TEI.provenance]{provenance} \hyperref[TEI.rubric]{rubric} \hyperref[TEI.secFol]{secFol} \hyperref[TEI.signatures]{signatures} \hyperref[TEI.source]{source} \hyperref[TEI.stamp]{stamp} \hyperref[TEI.summary]{summary} \hyperref[TEI.support]{support} \hyperref[TEI.surrogates]{surrogates} \hyperref[TEI.typeNote]{typeNote} \hyperref[TEI.watermark]{watermark}\par 
    \item[namesdates: ]
   \hyperref[TEI.addName]{addName} \hyperref[TEI.affiliation]{affiliation} \hyperref[TEI.country]{country} \hyperref[TEI.forename]{forename} \hyperref[TEI.genName]{genName} \hyperref[TEI.geogName]{geogName} \hyperref[TEI.nameLink]{nameLink} \hyperref[TEI.org]{org} \hyperref[TEI.orgName]{orgName} \hyperref[TEI.persName]{persName} \hyperref[TEI.person]{person} \hyperref[TEI.personGrp]{personGrp} \hyperref[TEI.persona]{persona} \hyperref[TEI.placeName]{placeName} \hyperref[TEI.region]{region} \hyperref[TEI.roleName]{roleName} \hyperref[TEI.settlement]{settlement} \hyperref[TEI.surname]{surname}\par 
    \item[textstructure: ]
   \hyperref[TEI.back]{back} \hyperref[TEI.body]{body} \hyperref[TEI.div]{div} \hyperref[TEI.docAuthor]{docAuthor} \hyperref[TEI.docDate]{docDate} \hyperref[TEI.docEdition]{docEdition} \hyperref[TEI.docTitle]{docTitle} \hyperref[TEI.floatingText]{floatingText} \hyperref[TEI.front]{front} \hyperref[TEI.group]{group} \hyperref[TEI.text]{text} \hyperref[TEI.titlePage]{titlePage} \hyperref[TEI.titlePart]{titlePart}\par 
    \item[transcr: ]
   \hyperref[TEI.damage]{damage} \hyperref[TEI.fw]{fw} \hyperref[TEI.line]{line} \hyperref[TEI.metamark]{metamark} \hyperref[TEI.mod]{mod} \hyperref[TEI.restore]{restore} \hyperref[TEI.retrace]{retrace} \hyperref[TEI.secl]{secl} \hyperref[TEI.sourceDoc]{sourceDoc} \hyperref[TEI.subst]{subst} \hyperref[TEI.supplied]{supplied} \hyperref[TEI.surface]{surface} \hyperref[TEI.surfaceGrp]{surfaceGrp} \hyperref[TEI.surplus]{surplus} \hyperref[TEI.zone]{zone}
    \item[{Peut contenir}]
  Elément vide
    \item[{Note}]
  \par
Pour cet élément, l'attribut global {\itshape n} affecte un nouveau numéro ou une autre valeur à l'unité qui change à partir de l'élément \hyperref[TEI.milestone]{<milestone>}. La valeur \textit{unnumbered} doit être utilisée pour les passages qui sortent du système normal de numérotation (par ex. titres de chapitres, numéros ou titres de poèmes, ou noms des personnages qui prennent la parole dans une pièce de théâtre).\par
L'ordre dans lequel apparaissent les éléments \hyperref[TEI.milestone]{<milestone>} à un endroit donné n'est en principe pas signifiant.
    \item[{Exemple}]
  \leavevmode\bgroup\exampleFont \begin{shaded}\noindent\mbox{}{<\textbf{milestone}\hspace*{6pt}{ed}="{La}"\hspace*{6pt}{n}="{23}"\hspace*{6pt}{unit}="{Dreissiger}"/>}\mbox{}\newline 
 ... {<\textbf{milestone}\hspace*{6pt}{ed}="{AV}"\hspace*{6pt}{n}="{24}"\hspace*{6pt}{unit}="{verse}"/>} ...\end{shaded}\egroup 


    \item[{Exemple}]
  \leavevmode\bgroup\exampleFont \begin{shaded}\noindent\mbox{}{<\textbf{milestone}\hspace*{6pt}{ed}="{La}"\hspace*{6pt}{n}="{23}"\hspace*{6pt}{unit}="{Dreissiger}"/>} ... {<\textbf{milestone}\hspace*{6pt}{ed}="{AV}"\hspace*{6pt}{n}="{24}"\hspace*{6pt}{unit}="{verse}"/>}\mbox{}\newline 
 ...\end{shaded}\egroup 


    \item[{Modèle de contenu}]
  \fbox{\ttfamily <content>\newline
</content>\newline
    } 
    \item[{Schéma Declaration}]
  \mbox{}\hfill\\[-10pt]\begin{Verbatim}[fontsize=\small]
element milestone
{
   tei_att.global.attributes,
   tei_att.milestoneUnit.attributes,
   tei_att.typed.attributes,
   tei_att.edition.attributes,
   tei_att.spanning.attributes,
   tei_att.breaking.attributes,
   empty
}
\end{Verbatim}

\end{reflist}  \index{mmultiscripts=<mmultiscripts>|oddindex}
\begin{reflist}
\item[]\begin{specHead}{TEI.mmultiscripts}{<mmultiscripts> }\end{specHead} 
    \item[{Namespace}]
  http://www.w3.org/1998/Math/MathML
    \item[{Module}]
  derived-module-tei.istex
    \item[{Contenu dans}]
  
    \item[derived-module-tei.istex: ]
   \hyperref[TEI.menclose]{menclose} \hyperref[TEI.mfenced]{mfenced} \hyperref[TEI.mfrac]{mfrac} \hyperref[TEI.mmultiscripts]{mmultiscripts} \hyperref[TEI.mover]{mover} \hyperref[TEI.mpadded]{mpadded} \hyperref[TEI.mphantom]{mphantom} \hyperref[TEI.mprescripts]{mprescripts} \hyperref[TEI.mrow]{mrow} \hyperref[TEI.msqrt]{msqrt} \hyperref[TEI.mstyle]{mstyle} \hyperref[TEI.msub]{msub} \hyperref[TEI.msubsup]{msubsup} \hyperref[TEI.msup]{msup} \hyperref[TEI.msupsub]{msupsub} \hyperref[TEI.mtable]{mtable} \hyperref[TEI.mtd]{mtd} \hyperref[TEI.mtr]{mtr} \hyperref[TEI.munder]{munder} \hyperref[TEI.munderover]{munderover} \hyperref[TEI.semantics]{semantics}
    \item[{Peut contenir}]
  
    \item[derived-module-tei.istex: ]
   \hyperref[TEI.menclose]{menclose} \hyperref[TEI.mfenced]{mfenced} \hyperref[TEI.mfrac]{mfrac} \hyperref[TEI.mi]{mi} \hyperref[TEI.mmultiscripts]{mmultiscripts} \hyperref[TEI.mn]{mn} \hyperref[TEI.mo]{mo} \hyperref[TEI.mover]{mover} \hyperref[TEI.mpadded]{mpadded} \hyperref[TEI.mphantom]{mphantom} \hyperref[TEI.mprescripts]{mprescripts} \hyperref[TEI.mrow]{mrow} \hyperref[TEI.mspace]{mspace} \hyperref[TEI.msqrt]{msqrt} \hyperref[TEI.mstyle]{mstyle} \hyperref[TEI.msub]{msub} \hyperref[TEI.msubsup]{msubsup} \hyperref[TEI.msup]{msup} \hyperref[TEI.msupsub]{msupsub} \hyperref[TEI.mtable]{mtable} \hyperref[TEI.mtd]{mtd} \hyperref[TEI.mtext]{mtext} \hyperref[TEI.mtr]{mtr} \hyperref[TEI.munder]{munder} \hyperref[TEI.munderover]{munderover} \hyperref[TEI.none]{none}\par des données textuelles
    \item[{Modèle de contenu}]
  \mbox{}\hfill\\[-10pt]\begin{Verbatim}[fontsize=\small]
<content>
 <alternate maxOccurs="unbounded"
  minOccurs="0">
  <textNode/>
  <elementRef key="mstyle"/>
  <elementRef key="mtr"/>
  <elementRef key="mtd"/>
  <elementRef key="mrow"/>
  <elementRef key="mi"/>
  <elementRef key="mn"/>
  <elementRef key="mtext"/>
  <elementRef key="mfrac"/>
  <elementRef key="mspace"/>
  <elementRef key="msqrt"/>
  <elementRef key="msub"/>
  <elementRef key="msup"/>
  <elementRef key="mo"/>
  <elementRef key="mover"/>
  <elementRef key="mfenced"/>
  <elementRef key="mtable"/>
  <elementRef key="msubsup"/>
  <elementRef key="msupsub"/>
  <elementRef key="mmultiscripts"/>
  <elementRef key="munderover"/>
  <elementRef key="mprescripts"/>
  <elementRef key="none"/>
  <elementRef key="munder"/>
  <elementRef key="mphantom"/>
  <elementRef key="mpadded"/>
  <elementRef key="menclose"/>
 </alternate>
</content>
    
\end{Verbatim}

    \item[{Schéma Declaration}]
  \mbox{}\hfill\\[-10pt]\begin{Verbatim}[fontsize=\small]
element mmultiscripts
{
   (
      text
    | tei_mstyle    | tei_mtr    | tei_mtd    | tei_mrow    | tei_mi    | tei_mn    | tei_mtext    | tei_mfrac    | tei_mspace    | tei_msqrt    | tei_msub    | tei_msup    | tei_mo    | tei_mover    | tei_mfenced    | tei_mtable    | tei_msubsup    | tei_msupsub    | tei_mmultiscripts    | tei_munderover    | tei_mprescripts    | tei_none    | tei_munder    | tei_mphantom    | tei_mpadded    | tei_menclose   )*
}
\end{Verbatim}

\end{reflist}  \index{mn=<mn>|oddindex}\index{stretchy=@stretchy!<mn>|oddindex}\index{mathvariant=@mathvariant!<mn>|oddindex}\index{fontstyle=@fontstyle!<mn>|oddindex}\index{fontweight=@fontweight!<mn>|oddindex}\index{mathsize=@mathsize!<mn>|oddindex}\index{movablelimits=@movablelimits!<mn>|oddindex}\index{class=@class!<mn>|oddindex}\index{accent=@accent!<mn>|oddindex}\index{form=@form!<mn>|oddindex}\index{fence=@fence!<mn>|oddindex}
\begin{reflist}
\item[]\begin{specHead}{TEI.mn}{<mn> }\end{specHead} 
    \item[{Namespace}]
  http://www.w3.org/1998/Math/MathML
    \item[{Module}]
  derived-module-tei.istex
    \item[{Attributs}]
  Attributs\hfil\\[-10pt]\begin{sansreflist}
    \item[@stretchy]
  
\begin{reflist}
    \item[{Statut}]
  Optionel
    \item[{Type de données}]
  \xref{https://www.w3.org/TR/xmlschema-2/\#}{}
\end{reflist}  
    \item[@mathvariant]
  
\begin{reflist}
    \item[{Statut}]
  Optionel
    \item[{Type de données}]
  \xref{https://www.w3.org/TR/xmlschema-2/\#}{}
\end{reflist}  
    \item[@fontstyle]
  
\begin{reflist}
    \item[{Statut}]
  Optionel
    \item[{Type de données}]
  \xref{https://www.w3.org/TR/xmlschema-2/\#}{}
\end{reflist}  
    \item[@fontweight]
  
\begin{reflist}
    \item[{Statut}]
  Optionel
    \item[{Type de données}]
  \xref{https://www.w3.org/TR/xmlschema-2/\#}{}
\end{reflist}  
    \item[@mathsize]
  
\begin{reflist}
    \item[{Statut}]
  Optionel
    \item[{Type de données}]
  \xref{https://www.w3.org/TR/xmlschema-2/\#}{}
\end{reflist}  
    \item[@movablelimits]
  
\begin{reflist}
    \item[{Statut}]
  Optionel
    \item[{Type de données}]
  \xref{https://www.w3.org/TR/xmlschema-2/\#}{}
\end{reflist}  
    \item[@class]
  
\begin{reflist}
    \item[{Statut}]
  Optionel
    \item[{Type de données}]
  \xref{https://www.w3.org/TR/xmlschema-2/\#}{}
\end{reflist}  
    \item[@accent]
  
\begin{reflist}
    \item[{Statut}]
  Optionel
    \item[{Type de données}]
  \xref{https://www.w3.org/TR/xmlschema-2/\#}{}
\end{reflist}  
    \item[@form]
  
\begin{reflist}
    \item[{Statut}]
  Optionel
    \item[{Type de données}]
  \xref{https://www.w3.org/TR/xmlschema-2/\#}{}
\end{reflist}  
    \item[@fence]
  
\begin{reflist}
    \item[{Statut}]
  Optionel
    \item[{Type de données}]
  \xref{https://www.w3.org/TR/xmlschema-2/\#}{}
\end{reflist}  
\end{sansreflist}  
    \item[{Contenu dans}]
  
    \item[derived-module-tei.istex: ]
   \hyperref[TEI.math]{math} \hyperref[TEI.menclose]{menclose} \hyperref[TEI.mfenced]{mfenced} \hyperref[TEI.mfrac]{mfrac} \hyperref[TEI.mmultiscripts]{mmultiscripts} \hyperref[TEI.mover]{mover} \hyperref[TEI.mpadded]{mpadded} \hyperref[TEI.mphantom]{mphantom} \hyperref[TEI.mprescripts]{mprescripts} \hyperref[TEI.mrow]{mrow} \hyperref[TEI.msqrt]{msqrt} \hyperref[TEI.mstyle]{mstyle} \hyperref[TEI.msub]{msub} \hyperref[TEI.msubsup]{msubsup} \hyperref[TEI.msup]{msup} \hyperref[TEI.msupsub]{msupsub} \hyperref[TEI.mtable]{mtable} \hyperref[TEI.mtd]{mtd} \hyperref[TEI.mtr]{mtr} \hyperref[TEI.munder]{munder} \hyperref[TEI.munderover]{munderover} \hyperref[TEI.semantics]{semantics}
    \item[{Peut contenir}]
  Des données textuelles uniquement
    \item[{Modèle de contenu}]
  \fbox{\ttfamily <content>\newline
 <textNode/>\newline
</content>\newline
    } 
    \item[{Schéma Declaration}]
  \mbox{}\hfill\\[-10pt]\begin{Verbatim}[fontsize=\small]
element mn
{
   attribute stretchy { stretchy }?,
   attribute mathvariant { mathvariant }?,
   attribute fontstyle { fontstyle }?,
   attribute fontweight { fontweight }?,
   attribute mathsize { mathsize }?,
   attribute movablelimits { movablelimits }?,
   attribute class { class }?,
   attribute accent { accent }?,
   attribute form { form }?,
   attribute fence { fence }?,
   text
}
\end{Verbatim}

\end{reflist}  \index{mo=<mo>|oddindex}\index{stretchy=@stretchy!<mo>|oddindex}\index{fontstyle=@fontstyle!<mo>|oddindex}\index{fontweight=@fontweight!<mo>|oddindex}\index{mathvariant=@mathvariant!<mo>|oddindex}\index{movablelimits=@movablelimits!<mo>|oddindex}\index{class=@class!<mo>|oddindex}\index{accent=@accent!<mo>|oddindex}\index{form=@form!<mo>|oddindex}\index{fence=@fence!<mo>|oddindex}\index{mathsize=@mathsize!<mo>|oddindex}
\begin{reflist}
\item[]\begin{specHead}{TEI.mo}{<mo> }\end{specHead} 
    \item[{Namespace}]
  http://www.w3.org/1998/Math/MathML
    \item[{Module}]
  derived-module-tei.istex
    \item[{Attributs}]
  Attributs\hfil\\[-10pt]\begin{sansreflist}
    \item[@stretchy]
  
\begin{reflist}
    \item[{Statut}]
  Optionel
    \item[{Type de données}]
  \xref{https://www.w3.org/TR/xmlschema-2/\#}{}
\end{reflist}  
    \item[@fontstyle]
  
\begin{reflist}
    \item[{Statut}]
  Optionel
    \item[{Type de données}]
  \xref{https://www.w3.org/TR/xmlschema-2/\#}{}
\end{reflist}  
    \item[@fontweight]
  
\begin{reflist}
    \item[{Statut}]
  Optionel
    \item[{Type de données}]
  \xref{https://www.w3.org/TR/xmlschema-2/\#}{}
\end{reflist}  
    \item[@mathvariant]
  
\begin{reflist}
    \item[{Statut}]
  Optionel
    \item[{Type de données}]
  \xref{https://www.w3.org/TR/xmlschema-2/\#}{}
\end{reflist}  
    \item[@movablelimits]
  
\begin{reflist}
    \item[{Statut}]
  Optionel
    \item[{Type de données}]
  \xref{https://www.w3.org/TR/xmlschema-2/\#}{}
\end{reflist}  
    \item[@class]
  
\begin{reflist}
    \item[{Statut}]
  Optionel
    \item[{Type de données}]
  \xref{https://www.w3.org/TR/xmlschema-2/\#}{}
\end{reflist}  
    \item[@accent]
  
\begin{reflist}
    \item[{Statut}]
  Optionel
    \item[{Type de données}]
  \xref{https://www.w3.org/TR/xmlschema-2/\#}{}
\end{reflist}  
    \item[@form]
  
\begin{reflist}
    \item[{Statut}]
  Optionel
    \item[{Type de données}]
  \xref{https://www.w3.org/TR/xmlschema-2/\#}{}
\end{reflist}  
    \item[@fence]
  
\begin{reflist}
    \item[{Statut}]
  Optionel
    \item[{Type de données}]
  \xref{https://www.w3.org/TR/xmlschema-2/\#}{}
\end{reflist}  
    \item[@mathsize]
  
\begin{reflist}
    \item[{Statut}]
  Optionel
    \item[{Type de données}]
  \xref{https://www.w3.org/TR/xmlschema-2/\#}{}
\end{reflist}  
\end{sansreflist}  
    \item[{Contenu dans}]
  
    \item[derived-module-tei.istex: ]
   \hyperref[TEI.math]{math} \hyperref[TEI.menclose]{menclose} \hyperref[TEI.mfenced]{mfenced} \hyperref[TEI.mfrac]{mfrac} \hyperref[TEI.mmultiscripts]{mmultiscripts} \hyperref[TEI.mover]{mover} \hyperref[TEI.mpadded]{mpadded} \hyperref[TEI.mphantom]{mphantom} \hyperref[TEI.mprescripts]{mprescripts} \hyperref[TEI.mrow]{mrow} \hyperref[TEI.msqrt]{msqrt} \hyperref[TEI.mstyle]{mstyle} \hyperref[TEI.msub]{msub} \hyperref[TEI.msubsup]{msubsup} \hyperref[TEI.msup]{msup} \hyperref[TEI.msupsub]{msupsub} \hyperref[TEI.mtable]{mtable} \hyperref[TEI.mtd]{mtd} \hyperref[TEI.mtr]{mtr} \hyperref[TEI.munder]{munder} \hyperref[TEI.munderover]{munderover} \hyperref[TEI.semantics]{semantics}
    \item[{Peut contenir}]
  Des données textuelles uniquement
    \item[{Modèle de contenu}]
  \fbox{\ttfamily <content>\newline
 <textNode/>\newline
</content>\newline
    } 
    \item[{Schéma Declaration}]
  \mbox{}\hfill\\[-10pt]\begin{Verbatim}[fontsize=\small]
element mo
{
   attribute stretchy { stretchy }?,
   attribute fontstyle { fontstyle }?,
   attribute fontweight { fontweight }?,
   attribute mathvariant { mathvariant }?,
   attribute movablelimits { movablelimits }?,
   attribute class { class }?,
   attribute accent { accent }?,
   attribute form { form }?,
   attribute fence { fence }?,
   attribute mathsize { mathsize }?,
   text
}
\end{Verbatim}

\end{reflist}  \index{mod=<mod>|oddindex}
\begin{reflist}
\item[]\begin{specHead}{TEI.mod}{<mod> }represents any kind of modification identified within a single document. [\xref{http://www.tei-c.org/release/doc/tei-p5-doc/en/html/PH.html\#PH-mod}{11.3.4.1. Generic Modification}]\end{specHead} 
    \item[{Module}]
  transcr
    \item[{Attributs}]
  Attributs \hyperref[TEI.att.global]{att.global} (\textit{@xml:id}, \textit{@n}, \textit{@xml:lang}, \textit{@xml:base}, \textit{@xml:space})  (\hyperref[TEI.att.global.rendition]{att.global.rendition} (\textit{@rend}, \textit{@style}, \textit{@rendition})) (\hyperref[TEI.att.global.linking]{att.global.linking} (\textit{@corresp}, \textit{@synch}, \textit{@sameAs}, \textit{@copyOf}, \textit{@next}, \textit{@prev}, \textit{@exclude}, \textit{@select})) (\hyperref[TEI.att.global.analytic]{att.global.analytic} (\textit{@ana})) (\hyperref[TEI.att.global.facs]{att.global.facs} (\textit{@facs})) (\hyperref[TEI.att.global.change]{att.global.change} (\textit{@change})) (\hyperref[TEI.att.global.responsibility]{att.global.responsibility} (\textit{@cert}, \textit{@resp})) (\hyperref[TEI.att.global.source]{att.global.source} (\textit{@source})) \hyperref[TEI.att.transcriptional]{att.transcriptional} (\textit{@status}, \textit{@cause}, \textit{@seq})  (\hyperref[TEI.att.editLike]{att.editLike} (\textit{@evidence}, \textit{@instant}) (\hyperref[TEI.att.dimensions]{att.dimensions} (\textit{@unit}, \textit{@quantity}, \textit{@extent}, \textit{@precision}, \textit{@scope}) (\hyperref[TEI.att.ranging]{att.ranging} (\textit{@atLeast}, \textit{@atMost}, \textit{@min}, \textit{@max}, \textit{@confidence})) ) ) (\hyperref[TEI.att.written]{att.written} (\textit{@hand})) \hyperref[TEI.att.typed]{att.typed} (\textit{@type}, \textit{@subtype}) \hyperref[TEI.att.spanning]{att.spanning} (\textit{@spanTo}) 
    \item[{Membre du}]
  \hyperref[TEI.model.linePart]{model.linePart} \hyperref[TEI.model.pPart.transcriptional]{model.pPart.transcriptional}
    \item[{Contenu dans}]
  
    \item[analysis: ]
   \hyperref[TEI.cl]{cl} \hyperref[TEI.pc]{pc} \hyperref[TEI.phr]{phr} \hyperref[TEI.s]{s} \hyperref[TEI.w]{w}\par 
    \item[core: ]
   \hyperref[TEI.abbr]{abbr} \hyperref[TEI.add]{add} \hyperref[TEI.addrLine]{addrLine} \hyperref[TEI.author]{author} \hyperref[TEI.bibl]{bibl} \hyperref[TEI.biblScope]{biblScope} \hyperref[TEI.citedRange]{citedRange} \hyperref[TEI.corr]{corr} \hyperref[TEI.date]{date} \hyperref[TEI.del]{del} \hyperref[TEI.distinct]{distinct} \hyperref[TEI.editor]{editor} \hyperref[TEI.email]{email} \hyperref[TEI.emph]{emph} \hyperref[TEI.expan]{expan} \hyperref[TEI.foreign]{foreign} \hyperref[TEI.gloss]{gloss} \hyperref[TEI.head]{head} \hyperref[TEI.headItem]{headItem} \hyperref[TEI.headLabel]{headLabel} \hyperref[TEI.hi]{hi} \hyperref[TEI.item]{item} \hyperref[TEI.l]{l} \hyperref[TEI.label]{label} \hyperref[TEI.measure]{measure} \hyperref[TEI.mentioned]{mentioned} \hyperref[TEI.name]{name} \hyperref[TEI.note]{note} \hyperref[TEI.num]{num} \hyperref[TEI.orig]{orig} \hyperref[TEI.p]{p} \hyperref[TEI.pubPlace]{pubPlace} \hyperref[TEI.publisher]{publisher} \hyperref[TEI.q]{q} \hyperref[TEI.quote]{quote} \hyperref[TEI.ref]{ref} \hyperref[TEI.reg]{reg} \hyperref[TEI.rs]{rs} \hyperref[TEI.said]{said} \hyperref[TEI.sic]{sic} \hyperref[TEI.soCalled]{soCalled} \hyperref[TEI.speaker]{speaker} \hyperref[TEI.stage]{stage} \hyperref[TEI.street]{street} \hyperref[TEI.term]{term} \hyperref[TEI.textLang]{textLang} \hyperref[TEI.time]{time} \hyperref[TEI.title]{title} \hyperref[TEI.unclear]{unclear}\par 
    \item[figures: ]
   \hyperref[TEI.cell]{cell}\par 
    \item[header: ]
   \hyperref[TEI.change]{change} \hyperref[TEI.distributor]{distributor} \hyperref[TEI.edition]{edition} \hyperref[TEI.extent]{extent} \hyperref[TEI.licence]{licence}\par 
    \item[linking: ]
   \hyperref[TEI.ab]{ab} \hyperref[TEI.seg]{seg}\par 
    \item[msdescription: ]
   \hyperref[TEI.accMat]{accMat} \hyperref[TEI.acquisition]{acquisition} \hyperref[TEI.additions]{additions} \hyperref[TEI.catchwords]{catchwords} \hyperref[TEI.collation]{collation} \hyperref[TEI.colophon]{colophon} \hyperref[TEI.condition]{condition} \hyperref[TEI.custEvent]{custEvent} \hyperref[TEI.decoNote]{decoNote} \hyperref[TEI.explicit]{explicit} \hyperref[TEI.filiation]{filiation} \hyperref[TEI.finalRubric]{finalRubric} \hyperref[TEI.foliation]{foliation} \hyperref[TEI.heraldry]{heraldry} \hyperref[TEI.incipit]{incipit} \hyperref[TEI.layout]{layout} \hyperref[TEI.material]{material} \hyperref[TEI.musicNotation]{musicNotation} \hyperref[TEI.objectType]{objectType} \hyperref[TEI.origDate]{origDate} \hyperref[TEI.origPlace]{origPlace} \hyperref[TEI.origin]{origin} \hyperref[TEI.provenance]{provenance} \hyperref[TEI.rubric]{rubric} \hyperref[TEI.secFol]{secFol} \hyperref[TEI.signatures]{signatures} \hyperref[TEI.source]{source} \hyperref[TEI.stamp]{stamp} \hyperref[TEI.summary]{summary} \hyperref[TEI.support]{support} \hyperref[TEI.surrogates]{surrogates} \hyperref[TEI.typeNote]{typeNote} \hyperref[TEI.watermark]{watermark}\par 
    \item[namesdates: ]
   \hyperref[TEI.addName]{addName} \hyperref[TEI.affiliation]{affiliation} \hyperref[TEI.country]{country} \hyperref[TEI.forename]{forename} \hyperref[TEI.genName]{genName} \hyperref[TEI.geogName]{geogName} \hyperref[TEI.nameLink]{nameLink} \hyperref[TEI.orgName]{orgName} \hyperref[TEI.persName]{persName} \hyperref[TEI.placeName]{placeName} \hyperref[TEI.region]{region} \hyperref[TEI.roleName]{roleName} \hyperref[TEI.settlement]{settlement} \hyperref[TEI.surname]{surname}\par 
    \item[textstructure: ]
   \hyperref[TEI.docAuthor]{docAuthor} \hyperref[TEI.docDate]{docDate} \hyperref[TEI.docEdition]{docEdition} \hyperref[TEI.titlePart]{titlePart}\par 
    \item[transcr: ]
   \hyperref[TEI.am]{am} \hyperref[TEI.damage]{damage} \hyperref[TEI.fw]{fw} \hyperref[TEI.line]{line} \hyperref[TEI.metamark]{metamark} \hyperref[TEI.mod]{mod} \hyperref[TEI.restore]{restore} \hyperref[TEI.retrace]{retrace} \hyperref[TEI.secl]{secl} \hyperref[TEI.supplied]{supplied} \hyperref[TEI.surplus]{surplus} \hyperref[TEI.zone]{zone}
    \item[{Peut contenir}]
  
    \item[analysis: ]
   \hyperref[TEI.c]{c} \hyperref[TEI.cl]{cl} \hyperref[TEI.interp]{interp} \hyperref[TEI.interpGrp]{interpGrp} \hyperref[TEI.m]{m} \hyperref[TEI.pc]{pc} \hyperref[TEI.phr]{phr} \hyperref[TEI.s]{s} \hyperref[TEI.span]{span} \hyperref[TEI.spanGrp]{spanGrp} \hyperref[TEI.w]{w}\par 
    \item[core: ]
   \hyperref[TEI.abbr]{abbr} \hyperref[TEI.add]{add} \hyperref[TEI.address]{address} \hyperref[TEI.bibl]{bibl} \hyperref[TEI.biblStruct]{biblStruct} \hyperref[TEI.binaryObject]{binaryObject} \hyperref[TEI.cb]{cb} \hyperref[TEI.choice]{choice} \hyperref[TEI.cit]{cit} \hyperref[TEI.corr]{corr} \hyperref[TEI.date]{date} \hyperref[TEI.del]{del} \hyperref[TEI.desc]{desc} \hyperref[TEI.distinct]{distinct} \hyperref[TEI.email]{email} \hyperref[TEI.emph]{emph} \hyperref[TEI.expan]{expan} \hyperref[TEI.foreign]{foreign} \hyperref[TEI.gap]{gap} \hyperref[TEI.gb]{gb} \hyperref[TEI.gloss]{gloss} \hyperref[TEI.graphic]{graphic} \hyperref[TEI.hi]{hi} \hyperref[TEI.index]{index} \hyperref[TEI.l]{l} \hyperref[TEI.label]{label} \hyperref[TEI.lb]{lb} \hyperref[TEI.lg]{lg} \hyperref[TEI.list]{list} \hyperref[TEI.listBibl]{listBibl} \hyperref[TEI.measure]{measure} \hyperref[TEI.measureGrp]{measureGrp} \hyperref[TEI.media]{media} \hyperref[TEI.mentioned]{mentioned} \hyperref[TEI.milestone]{milestone} \hyperref[TEI.name]{name} \hyperref[TEI.note]{note} \hyperref[TEI.num]{num} \hyperref[TEI.orig]{orig} \hyperref[TEI.pb]{pb} \hyperref[TEI.ptr]{ptr} \hyperref[TEI.q]{q} \hyperref[TEI.quote]{quote} \hyperref[TEI.ref]{ref} \hyperref[TEI.reg]{reg} \hyperref[TEI.rs]{rs} \hyperref[TEI.said]{said} \hyperref[TEI.sic]{sic} \hyperref[TEI.soCalled]{soCalled} \hyperref[TEI.stage]{stage} \hyperref[TEI.term]{term} \hyperref[TEI.time]{time} \hyperref[TEI.title]{title} \hyperref[TEI.unclear]{unclear}\par 
    \item[derived-module-tei.istex: ]
   \hyperref[TEI.math]{math} \hyperref[TEI.mrow]{mrow}\par 
    \item[figures: ]
   \hyperref[TEI.figure]{figure} \hyperref[TEI.formula]{formula} \hyperref[TEI.notatedMusic]{notatedMusic} \hyperref[TEI.table]{table}\par 
    \item[header: ]
   \hyperref[TEI.biblFull]{biblFull} \hyperref[TEI.idno]{idno}\par 
    \item[iso-fs: ]
   \hyperref[TEI.fLib]{fLib} \hyperref[TEI.fs]{fs} \hyperref[TEI.fvLib]{fvLib}\par 
    \item[linking: ]
   \hyperref[TEI.alt]{alt} \hyperref[TEI.altGrp]{altGrp} \hyperref[TEI.anchor]{anchor} \hyperref[TEI.join]{join} \hyperref[TEI.joinGrp]{joinGrp} \hyperref[TEI.link]{link} \hyperref[TEI.linkGrp]{linkGrp} \hyperref[TEI.seg]{seg} \hyperref[TEI.timeline]{timeline}\par 
    \item[msdescription: ]
   \hyperref[TEI.catchwords]{catchwords} \hyperref[TEI.depth]{depth} \hyperref[TEI.dim]{dim} \hyperref[TEI.dimensions]{dimensions} \hyperref[TEI.height]{height} \hyperref[TEI.heraldry]{heraldry} \hyperref[TEI.locus]{locus} \hyperref[TEI.locusGrp]{locusGrp} \hyperref[TEI.material]{material} \hyperref[TEI.msDesc]{msDesc} \hyperref[TEI.objectType]{objectType} \hyperref[TEI.origDate]{origDate} \hyperref[TEI.origPlace]{origPlace} \hyperref[TEI.secFol]{secFol} \hyperref[TEI.signatures]{signatures} \hyperref[TEI.source]{source} \hyperref[TEI.stamp]{stamp} \hyperref[TEI.watermark]{watermark} \hyperref[TEI.width]{width}\par 
    \item[namesdates: ]
   \hyperref[TEI.addName]{addName} \hyperref[TEI.affiliation]{affiliation} \hyperref[TEI.country]{country} \hyperref[TEI.forename]{forename} \hyperref[TEI.genName]{genName} \hyperref[TEI.geogName]{geogName} \hyperref[TEI.listOrg]{listOrg} \hyperref[TEI.listPlace]{listPlace} \hyperref[TEI.location]{location} \hyperref[TEI.nameLink]{nameLink} \hyperref[TEI.orgName]{orgName} \hyperref[TEI.persName]{persName} \hyperref[TEI.placeName]{placeName} \hyperref[TEI.region]{region} \hyperref[TEI.roleName]{roleName} \hyperref[TEI.settlement]{settlement} \hyperref[TEI.state]{state} \hyperref[TEI.surname]{surname}\par 
    \item[spoken: ]
   \hyperref[TEI.annotationBlock]{annotationBlock}\par 
    \item[textstructure: ]
   \hyperref[TEI.floatingText]{floatingText}\par 
    \item[transcr: ]
   \hyperref[TEI.addSpan]{addSpan} \hyperref[TEI.am]{am} \hyperref[TEI.damage]{damage} \hyperref[TEI.damageSpan]{damageSpan} \hyperref[TEI.delSpan]{delSpan} \hyperref[TEI.ex]{ex} \hyperref[TEI.fw]{fw} \hyperref[TEI.handShift]{handShift} \hyperref[TEI.listTranspose]{listTranspose} \hyperref[TEI.metamark]{metamark} \hyperref[TEI.mod]{mod} \hyperref[TEI.redo]{redo} \hyperref[TEI.restore]{restore} \hyperref[TEI.retrace]{retrace} \hyperref[TEI.secl]{secl} \hyperref[TEI.space]{space} \hyperref[TEI.subst]{subst} \hyperref[TEI.substJoin]{substJoin} \hyperref[TEI.supplied]{supplied} \hyperref[TEI.surplus]{surplus} \hyperref[TEI.undo]{undo}\par des données textuelles
    \item[{Exemple}]
  \leavevmode\bgroup\exampleFont \begin{shaded}\noindent\mbox{}{<\textbf{mod}\hspace*{6pt}{type}="{subst}">}\mbox{}\newline 
\hspace*{6pt}{<\textbf{add}>}pleasing{</\textbf{add}>}\mbox{}\newline 
\hspace*{6pt}{<\textbf{del}>}agreable{</\textbf{del}>}\mbox{}\newline 
{</\textbf{mod}>}\end{shaded}\egroup 


    \item[{Modèle de contenu}]
  \mbox{}\hfill\\[-10pt]\begin{Verbatim}[fontsize=\small]
<content>
 <macroRef key="macro.paraContent"/>
</content>
    
\end{Verbatim}

    \item[{Schéma Declaration}]
  \mbox{}\hfill\\[-10pt]\begin{Verbatim}[fontsize=\small]
element mod
{
   tei_att.global.attributes,
   tei_att.transcriptional.attributes,
   tei_att.typed.attributes,
   tei_att.spanning.attributes,
   tei_macro.paraContent}
\end{Verbatim}

\end{reflist}  \index{monogr=<monogr>|oddindex}
\begin{reflist}
\item[]\begin{specHead}{TEI.monogr}{<monogr> }(niveau monographique) contient des données bibliographiques décrivant un objet (par exemple une monographie ou une revue) publié comme un élément indépendant (i.e. matériellement séparé. [\xref{http://www.tei-c.org/release/doc/tei-p5-doc/en/html/CO.html\#COBICOL}{3.11.2.1. Analytic, Monographic, and Series Levels}]\end{specHead} 
    \item[{Module}]
  core
    \item[{Attributs}]
  Attributs \hyperref[TEI.att.global]{att.global} (\textit{@xml:id}, \textit{@n}, \textit{@xml:lang}, \textit{@xml:base}, \textit{@xml:space})  (\hyperref[TEI.att.global.rendition]{att.global.rendition} (\textit{@rend}, \textit{@style}, \textit{@rendition})) (\hyperref[TEI.att.global.linking]{att.global.linking} (\textit{@corresp}, \textit{@synch}, \textit{@sameAs}, \textit{@copyOf}, \textit{@next}, \textit{@prev}, \textit{@exclude}, \textit{@select})) (\hyperref[TEI.att.global.analytic]{att.global.analytic} (\textit{@ana})) (\hyperref[TEI.att.global.facs]{att.global.facs} (\textit{@facs})) (\hyperref[TEI.att.global.change]{att.global.change} (\textit{@change})) (\hyperref[TEI.att.global.responsibility]{att.global.responsibility} (\textit{@cert}, \textit{@resp})) (\hyperref[TEI.att.global.source]{att.global.source} (\textit{@source}))
    \item[{Contenu dans}]
  
    \item[core: ]
   \hyperref[TEI.biblStruct]{biblStruct}
    \item[{Peut contenir}]
  
    \item[core: ]
   \hyperref[TEI.author]{author} \hyperref[TEI.biblScope]{biblScope} \hyperref[TEI.editor]{editor} \hyperref[TEI.imprint]{imprint} \hyperref[TEI.meeting]{meeting} \hyperref[TEI.note]{note} \hyperref[TEI.ptr]{ptr} \hyperref[TEI.ref]{ref} \hyperref[TEI.respStmt]{respStmt} \hyperref[TEI.textLang]{textLang} \hyperref[TEI.title]{title}\par 
    \item[header: ]
   \hyperref[TEI.authority]{authority} \hyperref[TEI.availability]{availability} \hyperref[TEI.edition]{edition} \hyperref[TEI.extent]{extent} \hyperref[TEI.funder]{funder} \hyperref[TEI.idno]{idno}
    \item[{Note}]
  \par
Cet élément contient des éléments de description bibliographique spécialisés, dans un ordre prescrit.\par
L'élément \hyperref[TEI.monogr]{<monogr>} n'est disponible que dans l'élément \hyperref[TEI.biblStruct]{<biblStruct>}, où il faut l'utiliser pour encoder la description bibliographique d'une monographie.
    \item[{Exemple}]
  \leavevmode\bgroup\exampleFont \begin{shaded}\noindent\mbox{}{<\textbf{biblStruct}>}\mbox{}\newline 
\hspace*{6pt}{<\textbf{analytic}>}\mbox{}\newline 
\hspace*{6pt}\hspace*{6pt}{<\textbf{author}>}Chesnutt, David{</\textbf{author}>}\mbox{}\newline 
\hspace*{6pt}\hspace*{6pt}{<\textbf{title}>}Historical Editions in the States{</\textbf{title}>}\mbox{}\newline 
\hspace*{6pt}{</\textbf{analytic}>}\mbox{}\newline 
\hspace*{6pt}{<\textbf{monogr}>}\mbox{}\newline 
\hspace*{6pt}\hspace*{6pt}{<\textbf{title}\hspace*{6pt}{level}="{j}">}Computers and the Humanities{</\textbf{title}>}\mbox{}\newline 
\hspace*{6pt}\hspace*{6pt}{<\textbf{imprint}>}\mbox{}\newline 
\hspace*{6pt}\hspace*{6pt}\hspace*{6pt}{<\textbf{date}\hspace*{6pt}{when}="{1991-12}">}(December, 1991):{</\textbf{date}>}\mbox{}\newline 
\hspace*{6pt}\hspace*{6pt}{</\textbf{imprint}>}\mbox{}\newline 
\hspace*{6pt}\hspace*{6pt}{<\textbf{biblScope}>}25.6{</\textbf{biblScope}>}\mbox{}\newline 
\hspace*{6pt}\hspace*{6pt}{<\textbf{biblScope}\hspace*{6pt}{from}="{377}"\hspace*{6pt}{to}="{380}"\hspace*{6pt}{unit}="{page}">}377–380{</\textbf{biblScope}>}\mbox{}\newline 
\hspace*{6pt}{</\textbf{monogr}>}\mbox{}\newline 
{</\textbf{biblStruct}>}\end{shaded}\egroup 


    \item[{Exemple}]
  \leavevmode\bgroup\exampleFont \begin{shaded}\noindent\mbox{}{<\textbf{biblStruct}\hspace*{6pt}{type}="{book}">}\mbox{}\newline 
\hspace*{6pt}{<\textbf{monogr}>}\mbox{}\newline 
\hspace*{6pt}\hspace*{6pt}{<\textbf{author}>}\mbox{}\newline 
\hspace*{6pt}\hspace*{6pt}\hspace*{6pt}{<\textbf{persName}>}\mbox{}\newline 
\hspace*{6pt}\hspace*{6pt}\hspace*{6pt}\hspace*{6pt}{<\textbf{forename}>}Leo Joachim{</\textbf{forename}>}\mbox{}\newline 
\hspace*{6pt}\hspace*{6pt}\hspace*{6pt}\hspace*{6pt}{<\textbf{surname}>}Frachtenberg{</\textbf{surname}>}\mbox{}\newline 
\hspace*{6pt}\hspace*{6pt}\hspace*{6pt}{</\textbf{persName}>}\mbox{}\newline 
\hspace*{6pt}\hspace*{6pt}{</\textbf{author}>}\mbox{}\newline 
\hspace*{6pt}\hspace*{6pt}{<\textbf{title}\hspace*{6pt}{level}="{m}"\hspace*{6pt}{type}="{main}">}Lower Umpqua Texts{</\textbf{title}>}\mbox{}\newline 
\hspace*{6pt}\hspace*{6pt}{<\textbf{imprint}>}\mbox{}\newline 
\hspace*{6pt}\hspace*{6pt}\hspace*{6pt}{<\textbf{pubPlace}>}New York{</\textbf{pubPlace}>}\mbox{}\newline 
\hspace*{6pt}\hspace*{6pt}\hspace*{6pt}{<\textbf{publisher}>}Columbia University Press{</\textbf{publisher}>}\mbox{}\newline 
\hspace*{6pt}\hspace*{6pt}\hspace*{6pt}{<\textbf{date}>}1914{</\textbf{date}>}\mbox{}\newline 
\hspace*{6pt}\hspace*{6pt}{</\textbf{imprint}>}\mbox{}\newline 
\hspace*{6pt}{</\textbf{monogr}>}\mbox{}\newline 
\hspace*{6pt}{<\textbf{series}>}\mbox{}\newline 
\hspace*{6pt}\hspace*{6pt}{<\textbf{title}\hspace*{6pt}{level}="{s}"\hspace*{6pt}{type}="{main}">}Columbia University Contributions to\mbox{}\newline 
\hspace*{6pt}\hspace*{6pt}\hspace*{6pt}\hspace*{6pt} Anthropology{</\textbf{title}>}\mbox{}\newline 
\hspace*{6pt}\hspace*{6pt}{<\textbf{biblScope}\hspace*{6pt}{unit}="{volume}">}4{</\textbf{biblScope}>}\mbox{}\newline 
\hspace*{6pt}{</\textbf{series}>}\mbox{}\newline 
{</\textbf{biblStruct}>}\end{shaded}\egroup 


    \item[{Modèle de contenu}]
  \mbox{}\hfill\\[-10pt]\begin{Verbatim}[fontsize=\small]
<content>
 <sequence maxOccurs="1" minOccurs="1">
  <alternate maxOccurs="1" minOccurs="0">
   <sequence maxOccurs="1" minOccurs="1">
    <alternate maxOccurs="1" minOccurs="1">
     <elementRef key="author"/>
     <elementRef key="editor"/>
     <elementRef key="meeting"/>
     <elementRef key="respStmt"/>
    </alternate>
    <alternate maxOccurs="unbounded"
     minOccurs="0">
     <elementRef key="author"/>
     <elementRef key="editor"/>
     <elementRef key="meeting"/>
     <elementRef key="respStmt"/>
    </alternate>
    <elementRef key="title"
     maxOccurs="unbounded" minOccurs="1"/>
    <alternate maxOccurs="unbounded"
     minOccurs="0">
     <classRef key="model.ptrLike"/>
     <elementRef key="idno"/>
     <elementRef key="textLang"/>
     <elementRef key="editor"/>
     <elementRef key="respStmt"/>
    </alternate>
   </sequence>
   <sequence maxOccurs="1" minOccurs="1">
    <alternate maxOccurs="unbounded"
     minOccurs="1">
     <elementRef key="title"/>
     <classRef key="model.ptrLike"/>
     <elementRef key="idno"/>
    </alternate>
    <alternate maxOccurs="unbounded"
     minOccurs="0">
     <elementRef key="textLang"/>
     <elementRef key="author"/>
     <elementRef key="editor"/>
     <elementRef key="meeting"/>
     <elementRef key="respStmt"/>
    </alternate>
   </sequence>
   <sequence maxOccurs="1" minOccurs="1">
    <elementRef key="authority"/>
    <elementRef key="idno"/>
   </sequence>
  </alternate>
  <elementRef key="availability"
   maxOccurs="unbounded" minOccurs="0"/>
  <classRef key="model.noteLike"
   maxOccurs="unbounded" minOccurs="0"/>
  <sequence maxOccurs="unbounded"
   minOccurs="0">
   <elementRef key="edition"/>
   <alternate maxOccurs="unbounded"
    minOccurs="0">
    <elementRef key="idno"/>
    <classRef key="model.ptrLike"/>
    <elementRef key="editor"/>
    <elementRef key="sponsor"/>
    <elementRef key="funder"/>
    <elementRef key="respStmt"/>
   </alternate>
  </sequence>
  <elementRef key="imprint"/>
  <alternate maxOccurs="unbounded"
   minOccurs="0">
   <elementRef key="imprint"/>
   <elementRef key="extent"/>
   <elementRef key="biblScope"/>
  </alternate>
 </sequence>
</content>
    
\end{Verbatim}

    \item[{Schéma Declaration}]
  \mbox{}\hfill\\[-10pt]\begin{Verbatim}[fontsize=\small]
element monogr
{
   tei_att.global.attributes,
   (
      (
         (
            ( tei_author | tei_editor | tei_meeting | tei_respStmt ),
            ( tei_author | tei_editor | tei_meeting | tei_respStmt )*,
            tei_title+,
            (
               tei_model.ptrLike             | tei_idno             | tei_textLang             | tei_editor             | tei_respStmt            )*
         )
       | (
            ( tei_title | tei_model.ptrLike | tei_idno )+,
            (
               tei_textLang             | tei_author             | tei_editor             | tei_meeting             | tei_respStmt            )*
         )
       | ( tei_authority, tei_idno )
      )?,
      tei_availability*,
      tei_model.noteLike*,
      (
         tei_edition,
         (
            tei_idno          | tei_model.ptrLike          | tei_editor          | sponsor          | tei_funder          | tei_respStmt         )*
      )*,
      tei_imprint,
      ( tei_imprint | tei_extent | tei_biblScope )*
   )
}
\end{Verbatim}

\end{reflist}  \index{mover=<mover>|oddindex}\index{accent=@accent!<mover>|oddindex}\index{class=@class!<mover>|oddindex}
\begin{reflist}
\item[]\begin{specHead}{TEI.mover}{<mover> }\end{specHead} 
    \item[{Namespace}]
  http://www.w3.org/1998/Math/MathML
    \item[{Module}]
  derived-module-tei.istex
    \item[{Attributs}]
  Attributs\hfil\\[-10pt]\begin{sansreflist}
    \item[@accent]
  
\begin{reflist}
    \item[{Statut}]
  Optionel
    \item[{Type de données}]
  \xref{https://www.w3.org/TR/xmlschema-2/\#}{}
\end{reflist}  
    \item[@class]
  
\begin{reflist}
    \item[{Statut}]
  Optionel
    \item[{Type de données}]
  \xref{https://www.w3.org/TR/xmlschema-2/\#}{}
\end{reflist}  
\end{sansreflist}  
    \item[{Contenu dans}]
  
    \item[derived-module-tei.istex: ]
   \hyperref[TEI.menclose]{menclose} \hyperref[TEI.mfenced]{mfenced} \hyperref[TEI.mfrac]{mfrac} \hyperref[TEI.mmultiscripts]{mmultiscripts} \hyperref[TEI.mover]{mover} \hyperref[TEI.mpadded]{mpadded} \hyperref[TEI.mphantom]{mphantom} \hyperref[TEI.mprescripts]{mprescripts} \hyperref[TEI.mrow]{mrow} \hyperref[TEI.msqrt]{msqrt} \hyperref[TEI.mstyle]{mstyle} \hyperref[TEI.msub]{msub} \hyperref[TEI.msubsup]{msubsup} \hyperref[TEI.msup]{msup} \hyperref[TEI.msupsub]{msupsub} \hyperref[TEI.mtable]{mtable} \hyperref[TEI.mtd]{mtd} \hyperref[TEI.mtr]{mtr} \hyperref[TEI.munder]{munder} \hyperref[TEI.munderover]{munderover} \hyperref[TEI.semantics]{semantics}
    \item[{Peut contenir}]
  
    \item[derived-module-tei.istex: ]
   \hyperref[TEI.menclose]{menclose} \hyperref[TEI.mfenced]{mfenced} \hyperref[TEI.mfrac]{mfrac} \hyperref[TEI.mi]{mi} \hyperref[TEI.mmultiscripts]{mmultiscripts} \hyperref[TEI.mn]{mn} \hyperref[TEI.mo]{mo} \hyperref[TEI.mover]{mover} \hyperref[TEI.mpadded]{mpadded} \hyperref[TEI.mphantom]{mphantom} \hyperref[TEI.mprescripts]{mprescripts} \hyperref[TEI.mrow]{mrow} \hyperref[TEI.mspace]{mspace} \hyperref[TEI.msqrt]{msqrt} \hyperref[TEI.mstyle]{mstyle} \hyperref[TEI.msub]{msub} \hyperref[TEI.msubsup]{msubsup} \hyperref[TEI.msup]{msup} \hyperref[TEI.msupsub]{msupsub} \hyperref[TEI.mtable]{mtable} \hyperref[TEI.mtd]{mtd} \hyperref[TEI.mtext]{mtext} \hyperref[TEI.mtr]{mtr} \hyperref[TEI.munder]{munder} \hyperref[TEI.munderover]{munderover} \hyperref[TEI.none]{none}\par des données textuelles
    \item[{Modèle de contenu}]
  \mbox{}\hfill\\[-10pt]\begin{Verbatim}[fontsize=\small]
<content>
 <alternate maxOccurs="unbounded"
  minOccurs="0">
  <textNode/>
  <elementRef key="mstyle"/>
  <elementRef key="mtr"/>
  <elementRef key="mtd"/>
  <elementRef key="mrow"/>
  <elementRef key="mi"/>
  <elementRef key="mn"/>
  <elementRef key="mtext"/>
  <elementRef key="mfrac"/>
  <elementRef key="mspace"/>
  <elementRef key="msqrt"/>
  <elementRef key="msub"/>
  <elementRef key="msup"/>
  <elementRef key="mo"/>
  <elementRef key="mover"/>
  <elementRef key="mfenced"/>
  <elementRef key="mtable"/>
  <elementRef key="msubsup"/>
  <elementRef key="msupsub"/>
  <elementRef key="mmultiscripts"/>
  <elementRef key="munderover"/>
  <elementRef key="mprescripts"/>
  <elementRef key="none"/>
  <elementRef key="munder"/>
  <elementRef key="mphantom"/>
  <elementRef key="mpadded"/>
  <elementRef key="menclose"/>
 </alternate>
</content>
    
\end{Verbatim}

    \item[{Schéma Declaration}]
  \mbox{}\hfill\\[-10pt]\begin{Verbatim}[fontsize=\small]
element mover
{
   attribute accent { accent }?,
   attribute class { class }?,
   (
      text
    | tei_mstyle    | tei_mtr    | tei_mtd    | tei_mrow    | tei_mi    | tei_mn    | tei_mtext    | tei_mfrac    | tei_mspace    | tei_msqrt    | tei_msub    | tei_msup    | tei_mo    | tei_mover    | tei_mfenced    | tei_mtable    | tei_msubsup    | tei_msupsub    | tei_mmultiscripts    | tei_munderover    | tei_mprescripts    | tei_none    | tei_munder    | tei_mphantom    | tei_mpadded    | tei_menclose   )*
}
\end{Verbatim}

\end{reflist}  \index{mpadded=<mpadded>|oddindex}\index{width=@width!<mpadded>|oddindex}\index{depth=@depth!<mpadded>|oddindex}\index{height=@height!<mpadded>|oddindex}
\begin{reflist}
\item[]\begin{specHead}{TEI.mpadded}{<mpadded> }\end{specHead} 
    \item[{Namespace}]
  http://www.w3.org/1998/Math/MathML
    \item[{Module}]
  derived-module-tei.istex
    \item[{Attributs}]
  Attributs\hfil\\[-10pt]\begin{sansreflist}
    \item[@width]
  
\begin{reflist}
    \item[{Statut}]
  Optionel
    \item[{Type de données}]
  \xref{https://www.w3.org/TR/xmlschema-2/\#}{}
\end{reflist}  
    \item[@depth]
  
\begin{reflist}
    \item[{Statut}]
  Optionel
    \item[{Type de données}]
  \xref{https://www.w3.org/TR/xmlschema-2/\#}{}
\end{reflist}  
    \item[@height]
  
\begin{reflist}
    \item[{Statut}]
  Optionel
    \item[{Type de données}]
  \xref{https://www.w3.org/TR/xmlschema-2/\#}{}
\end{reflist}  
\end{sansreflist}  
    \item[{Contenu dans}]
  
    \item[derived-module-tei.istex: ]
   \hyperref[TEI.math]{math} \hyperref[TEI.menclose]{menclose} \hyperref[TEI.mfenced]{mfenced} \hyperref[TEI.mfrac]{mfrac} \hyperref[TEI.mmultiscripts]{mmultiscripts} \hyperref[TEI.mover]{mover} \hyperref[TEI.mpadded]{mpadded} \hyperref[TEI.mphantom]{mphantom} \hyperref[TEI.mprescripts]{mprescripts} \hyperref[TEI.mrow]{mrow} \hyperref[TEI.msqrt]{msqrt} \hyperref[TEI.mstyle]{mstyle} \hyperref[TEI.msub]{msub} \hyperref[TEI.msubsup]{msubsup} \hyperref[TEI.msup]{msup} \hyperref[TEI.msupsub]{msupsub} \hyperref[TEI.mtable]{mtable} \hyperref[TEI.mtd]{mtd} \hyperref[TEI.mtr]{mtr} \hyperref[TEI.munder]{munder} \hyperref[TEI.munderover]{munderover} \hyperref[TEI.semantics]{semantics}
    \item[{Peut contenir}]
  
    \item[derived-module-tei.istex: ]
   \hyperref[TEI.menclose]{menclose} \hyperref[TEI.mfenced]{mfenced} \hyperref[TEI.mfrac]{mfrac} \hyperref[TEI.mi]{mi} \hyperref[TEI.mmultiscripts]{mmultiscripts} \hyperref[TEI.mn]{mn} \hyperref[TEI.mo]{mo} \hyperref[TEI.mover]{mover} \hyperref[TEI.mpadded]{mpadded} \hyperref[TEI.mphantom]{mphantom} \hyperref[TEI.mprescripts]{mprescripts} \hyperref[TEI.mrow]{mrow} \hyperref[TEI.mspace]{mspace} \hyperref[TEI.msqrt]{msqrt} \hyperref[TEI.mstyle]{mstyle} \hyperref[TEI.msub]{msub} \hyperref[TEI.msubsup]{msubsup} \hyperref[TEI.msup]{msup} \hyperref[TEI.msupsub]{msupsub} \hyperref[TEI.mtable]{mtable} \hyperref[TEI.mtd]{mtd} \hyperref[TEI.mtext]{mtext} \hyperref[TEI.mtr]{mtr} \hyperref[TEI.munder]{munder} \hyperref[TEI.munderover]{munderover} \hyperref[TEI.none]{none}\par des données textuelles
    \item[{Modèle de contenu}]
  \mbox{}\hfill\\[-10pt]\begin{Verbatim}[fontsize=\small]
<content>
 <alternate maxOccurs="unbounded"
  minOccurs="0">
  <textNode/>
  <elementRef key="mstyle"/>
  <elementRef key="mtr"/>
  <elementRef key="mtd"/>
  <elementRef key="mrow"/>
  <elementRef key="mi"/>
  <elementRef key="mn"/>
  <elementRef key="mtext"/>
  <elementRef key="mfrac"/>
  <elementRef key="mspace"/>
  <elementRef key="msqrt"/>
  <elementRef key="msub"/>
  <elementRef key="msup"/>
  <elementRef key="mo"/>
  <elementRef key="mover"/>
  <elementRef key="mfenced"/>
  <elementRef key="mtable"/>
  <elementRef key="msubsup"/>
  <elementRef key="msupsub"/>
  <elementRef key="mmultiscripts"/>
  <elementRef key="munderover"/>
  <elementRef key="mprescripts"/>
  <elementRef key="none"/>
  <elementRef key="munder"/>
  <elementRef key="mphantom"/>
  <elementRef key="mpadded"/>
  <elementRef key="menclose"/>
 </alternate>
</content>
    
\end{Verbatim}

    \item[{Schéma Declaration}]
  \mbox{}\hfill\\[-10pt]\begin{Verbatim}[fontsize=\small]
element mpadded
{
   attribute width { width }?,
   attribute depth { depth }?,
   attribute height { height }?,
   (
      text
    | tei_mstyle    | tei_mtr    | tei_mtd    | tei_mrow    | tei_mi    | tei_mn    | tei_mtext    | tei_mfrac    | tei_mspace    | tei_msqrt    | tei_msub    | tei_msup    | tei_mo    | tei_mover    | tei_mfenced    | tei_mtable    | tei_msubsup    | tei_msupsub    | tei_mmultiscripts    | tei_munderover    | tei_mprescripts    | tei_none    | tei_munder    | tei_mphantom    | tei_mpadded    | tei_menclose   )*
}
\end{Verbatim}

\end{reflist}  \index{mphantom=<mphantom>|oddindex}
\begin{reflist}
\item[]\begin{specHead}{TEI.mphantom}{<mphantom> }\end{specHead} 
    \item[{Namespace}]
  http://www.w3.org/1998/Math/MathML
    \item[{Module}]
  derived-module-tei.istex
    \item[{Contenu dans}]
  
    \item[derived-module-tei.istex: ]
   \hyperref[TEI.math]{math} \hyperref[TEI.menclose]{menclose} \hyperref[TEI.mfenced]{mfenced} \hyperref[TEI.mfrac]{mfrac} \hyperref[TEI.mmultiscripts]{mmultiscripts} \hyperref[TEI.mover]{mover} \hyperref[TEI.mpadded]{mpadded} \hyperref[TEI.mphantom]{mphantom} \hyperref[TEI.mprescripts]{mprescripts} \hyperref[TEI.mrow]{mrow} \hyperref[TEI.msqrt]{msqrt} \hyperref[TEI.mstyle]{mstyle} \hyperref[TEI.msub]{msub} \hyperref[TEI.msubsup]{msubsup} \hyperref[TEI.msup]{msup} \hyperref[TEI.msupsub]{msupsub} \hyperref[TEI.mtable]{mtable} \hyperref[TEI.mtd]{mtd} \hyperref[TEI.mtr]{mtr} \hyperref[TEI.munder]{munder} \hyperref[TEI.munderover]{munderover} \hyperref[TEI.semantics]{semantics}
    \item[{Peut contenir}]
  
    \item[derived-module-tei.istex: ]
   \hyperref[TEI.menclose]{menclose} \hyperref[TEI.mfenced]{mfenced} \hyperref[TEI.mfrac]{mfrac} \hyperref[TEI.mi]{mi} \hyperref[TEI.mmultiscripts]{mmultiscripts} \hyperref[TEI.mn]{mn} \hyperref[TEI.mo]{mo} \hyperref[TEI.mover]{mover} \hyperref[TEI.mpadded]{mpadded} \hyperref[TEI.mphantom]{mphantom} \hyperref[TEI.mprescripts]{mprescripts} \hyperref[TEI.mrow]{mrow} \hyperref[TEI.mspace]{mspace} \hyperref[TEI.msqrt]{msqrt} \hyperref[TEI.mstyle]{mstyle} \hyperref[TEI.msub]{msub} \hyperref[TEI.msubsup]{msubsup} \hyperref[TEI.msup]{msup} \hyperref[TEI.msupsub]{msupsub} \hyperref[TEI.mtable]{mtable} \hyperref[TEI.mtd]{mtd} \hyperref[TEI.mtext]{mtext} \hyperref[TEI.mtr]{mtr} \hyperref[TEI.munder]{munder} \hyperref[TEI.munderover]{munderover} \hyperref[TEI.none]{none}\par des données textuelles
    \item[{Modèle de contenu}]
  \mbox{}\hfill\\[-10pt]\begin{Verbatim}[fontsize=\small]
<content>
 <alternate maxOccurs="unbounded"
  minOccurs="0">
  <textNode/>
  <elementRef key="mstyle"/>
  <elementRef key="mtr"/>
  <elementRef key="mtd"/>
  <elementRef key="mrow"/>
  <elementRef key="mi"/>
  <elementRef key="mn"/>
  <elementRef key="mtext"/>
  <elementRef key="mfrac"/>
  <elementRef key="mspace"/>
  <elementRef key="msqrt"/>
  <elementRef key="msub"/>
  <elementRef key="msup"/>
  <elementRef key="mo"/>
  <elementRef key="mover"/>
  <elementRef key="mfenced"/>
  <elementRef key="mtable"/>
  <elementRef key="msubsup"/>
  <elementRef key="msupsub"/>
  <elementRef key="mmultiscripts"/>
  <elementRef key="munderover"/>
  <elementRef key="mprescripts"/>
  <elementRef key="none"/>
  <elementRef key="munder"/>
  <elementRef key="mphantom"/>
  <elementRef key="mpadded"/>
  <elementRef key="menclose"/>
 </alternate>
</content>
    
\end{Verbatim}

    \item[{Schéma Declaration}]
  \mbox{}\hfill\\[-10pt]\begin{Verbatim}[fontsize=\small]
element mphantom
{
   (
      text
    | tei_mstyle    | tei_mtr    | tei_mtd    | tei_mrow    | tei_mi    | tei_mn    | tei_mtext    | tei_mfrac    | tei_mspace    | tei_msqrt    | tei_msub    | tei_msup    | tei_mo    | tei_mover    | tei_mfenced    | tei_mtable    | tei_msubsup    | tei_msupsub    | tei_mmultiscripts    | tei_munderover    | tei_mprescripts    | tei_none    | tei_munder    | tei_mphantom    | tei_mpadded    | tei_menclose   )*
}
\end{Verbatim}

\end{reflist}  \index{mprescripts=<mprescripts>|oddindex}
\begin{reflist}
\item[]\begin{specHead}{TEI.mprescripts}{<mprescripts> }\end{specHead} 
    \item[{Namespace}]
  http://www.w3.org/1998/Math/MathML
    \item[{Module}]
  derived-module-tei.istex
    \item[{Contenu dans}]
  
    \item[derived-module-tei.istex: ]
   \hyperref[TEI.math]{math} \hyperref[TEI.menclose]{menclose} \hyperref[TEI.mfenced]{mfenced} \hyperref[TEI.mfrac]{mfrac} \hyperref[TEI.mmultiscripts]{mmultiscripts} \hyperref[TEI.mover]{mover} \hyperref[TEI.mpadded]{mpadded} \hyperref[TEI.mphantom]{mphantom} \hyperref[TEI.mprescripts]{mprescripts} \hyperref[TEI.mrow]{mrow} \hyperref[TEI.msqrt]{msqrt} \hyperref[TEI.mstyle]{mstyle} \hyperref[TEI.msub]{msub} \hyperref[TEI.msubsup]{msubsup} \hyperref[TEI.msup]{msup} \hyperref[TEI.msupsub]{msupsub} \hyperref[TEI.mtable]{mtable} \hyperref[TEI.mtd]{mtd} \hyperref[TEI.mtr]{mtr} \hyperref[TEI.munder]{munder} \hyperref[TEI.munderover]{munderover} \hyperref[TEI.semantics]{semantics}
    \item[{Peut contenir}]
  
    \item[derived-module-tei.istex: ]
   \hyperref[TEI.menclose]{menclose} \hyperref[TEI.mfenced]{mfenced} \hyperref[TEI.mfrac]{mfrac} \hyperref[TEI.mi]{mi} \hyperref[TEI.mmultiscripts]{mmultiscripts} \hyperref[TEI.mn]{mn} \hyperref[TEI.mo]{mo} \hyperref[TEI.mover]{mover} \hyperref[TEI.mpadded]{mpadded} \hyperref[TEI.mphantom]{mphantom} \hyperref[TEI.mprescripts]{mprescripts} \hyperref[TEI.mrow]{mrow} \hyperref[TEI.mspace]{mspace} \hyperref[TEI.msqrt]{msqrt} \hyperref[TEI.mstyle]{mstyle} \hyperref[TEI.msub]{msub} \hyperref[TEI.msubsup]{msubsup} \hyperref[TEI.msup]{msup} \hyperref[TEI.msupsub]{msupsub} \hyperref[TEI.mtable]{mtable} \hyperref[TEI.mtd]{mtd} \hyperref[TEI.mtext]{mtext} \hyperref[TEI.mtr]{mtr} \hyperref[TEI.munder]{munder} \hyperref[TEI.munderover]{munderover} \hyperref[TEI.none]{none}\par des données textuelles
    \item[{Modèle de contenu}]
  \mbox{}\hfill\\[-10pt]\begin{Verbatim}[fontsize=\small]
<content>
 <alternate maxOccurs="unbounded"
  minOccurs="0">
  <textNode/>
  <elementRef key="mstyle"/>
  <elementRef key="mtr"/>
  <elementRef key="mtd"/>
  <elementRef key="mrow"/>
  <elementRef key="mi"/>
  <elementRef key="mn"/>
  <elementRef key="mtext"/>
  <elementRef key="mfrac"/>
  <elementRef key="mspace"/>
  <elementRef key="msqrt"/>
  <elementRef key="msub"/>
  <elementRef key="msup"/>
  <elementRef key="mo"/>
  <elementRef key="mover"/>
  <elementRef key="mfenced"/>
  <elementRef key="mtable"/>
  <elementRef key="msubsup"/>
  <elementRef key="msupsub"/>
  <elementRef key="mmultiscripts"/>
  <elementRef key="munderover"/>
  <elementRef key="mprescripts"/>
  <elementRef key="none"/>
  <elementRef key="munder"/>
  <elementRef key="mphantom"/>
  <elementRef key="mpadded"/>
  <elementRef key="menclose"/>
 </alternate>
</content>
    
\end{Verbatim}

    \item[{Schéma Declaration}]
  \mbox{}\hfill\\[-10pt]\begin{Verbatim}[fontsize=\small]
element mprescripts
{
   (
      text
    | tei_mstyle    | tei_mtr    | tei_mtd    | tei_mrow    | tei_mi    | tei_mn    | tei_mtext    | tei_mfrac    | tei_mspace    | tei_msqrt    | tei_msub    | tei_msup    | tei_mo    | tei_mover    | tei_mfenced    | tei_mtable    | tei_msubsup    | tei_msupsub    | tei_mmultiscripts    | tei_munderover    | tei_mprescripts    | tei_none    | tei_munder    | tei_mphantom    | tei_mpadded    | tei_menclose   )*
}
\end{Verbatim}

\end{reflist}  \index{mrow=<mrow>|oddindex}
\begin{reflist}
\item[]\begin{specHead}{TEI.mrow}{<mrow> }\end{specHead} 
    \item[{Namespace}]
  http://www.w3.org/1998/Math/MathML
    \item[{Module}]
  derived-module-tei.istex
    \item[{Membre du}]
  \hyperref[TEI.model.graphicLike]{model.graphicLike} 
    \item[{Contenu dans}]
  
    \item[analysis: ]
   \hyperref[TEI.cl]{cl} \hyperref[TEI.phr]{phr} \hyperref[TEI.s]{s}\par 
    \item[core: ]
   \hyperref[TEI.abbr]{abbr} \hyperref[TEI.add]{add} \hyperref[TEI.addrLine]{addrLine} \hyperref[TEI.author]{author} \hyperref[TEI.biblScope]{biblScope} \hyperref[TEI.citedRange]{citedRange} \hyperref[TEI.corr]{corr} \hyperref[TEI.date]{date} \hyperref[TEI.del]{del} \hyperref[TEI.distinct]{distinct} \hyperref[TEI.editor]{editor} \hyperref[TEI.email]{email} \hyperref[TEI.emph]{emph} \hyperref[TEI.expan]{expan} \hyperref[TEI.foreign]{foreign} \hyperref[TEI.gloss]{gloss} \hyperref[TEI.head]{head} \hyperref[TEI.headItem]{headItem} \hyperref[TEI.headLabel]{headLabel} \hyperref[TEI.hi]{hi} \hyperref[TEI.item]{item} \hyperref[TEI.l]{l} \hyperref[TEI.label]{label} \hyperref[TEI.measure]{measure} \hyperref[TEI.mentioned]{mentioned} \hyperref[TEI.name]{name} \hyperref[TEI.note]{note} \hyperref[TEI.num]{num} \hyperref[TEI.orig]{orig} \hyperref[TEI.p]{p} \hyperref[TEI.pubPlace]{pubPlace} \hyperref[TEI.publisher]{publisher} \hyperref[TEI.q]{q} \hyperref[TEI.quote]{quote} \hyperref[TEI.ref]{ref} \hyperref[TEI.reg]{reg} \hyperref[TEI.rs]{rs} \hyperref[TEI.said]{said} \hyperref[TEI.sic]{sic} \hyperref[TEI.soCalled]{soCalled} \hyperref[TEI.speaker]{speaker} \hyperref[TEI.stage]{stage} \hyperref[TEI.street]{street} \hyperref[TEI.term]{term} \hyperref[TEI.textLang]{textLang} \hyperref[TEI.time]{time} \hyperref[TEI.title]{title} \hyperref[TEI.unclear]{unclear}\par 
    \item[derived-module-tei.istex: ]
   \hyperref[TEI.math]{math} \hyperref[TEI.menclose]{menclose} \hyperref[TEI.mfenced]{mfenced} \hyperref[TEI.mfrac]{mfrac} \hyperref[TEI.mmultiscripts]{mmultiscripts} \hyperref[TEI.mover]{mover} \hyperref[TEI.mpadded]{mpadded} \hyperref[TEI.mphantom]{mphantom} \hyperref[TEI.mprescripts]{mprescripts} \hyperref[TEI.mrow]{mrow} \hyperref[TEI.msqrt]{msqrt} \hyperref[TEI.mstyle]{mstyle} \hyperref[TEI.msub]{msub} \hyperref[TEI.msubsup]{msubsup} \hyperref[TEI.msup]{msup} \hyperref[TEI.msupsub]{msupsub} \hyperref[TEI.mtable]{mtable} \hyperref[TEI.mtd]{mtd} \hyperref[TEI.mtr]{mtr} \hyperref[TEI.munder]{munder} \hyperref[TEI.munderover]{munderover} \hyperref[TEI.semantics]{semantics}\par 
    \item[figures: ]
   \hyperref[TEI.cell]{cell} \hyperref[TEI.figDesc]{figDesc} \hyperref[TEI.figure]{figure} \hyperref[TEI.formula]{formula} \hyperref[TEI.table]{table}\par 
    \item[header: ]
   \hyperref[TEI.change]{change} \hyperref[TEI.distributor]{distributor} \hyperref[TEI.edition]{edition} \hyperref[TEI.extent]{extent} \hyperref[TEI.licence]{licence}\par 
    \item[linking: ]
   \hyperref[TEI.ab]{ab} \hyperref[TEI.seg]{seg}\par 
    \item[msdescription: ]
   \hyperref[TEI.accMat]{accMat} \hyperref[TEI.acquisition]{acquisition} \hyperref[TEI.additions]{additions} \hyperref[TEI.catchwords]{catchwords} \hyperref[TEI.collation]{collation} \hyperref[TEI.colophon]{colophon} \hyperref[TEI.condition]{condition} \hyperref[TEI.custEvent]{custEvent} \hyperref[TEI.decoNote]{decoNote} \hyperref[TEI.explicit]{explicit} \hyperref[TEI.filiation]{filiation} \hyperref[TEI.finalRubric]{finalRubric} \hyperref[TEI.foliation]{foliation} \hyperref[TEI.heraldry]{heraldry} \hyperref[TEI.incipit]{incipit} \hyperref[TEI.layout]{layout} \hyperref[TEI.material]{material} \hyperref[TEI.musicNotation]{musicNotation} \hyperref[TEI.objectType]{objectType} \hyperref[TEI.origDate]{origDate} \hyperref[TEI.origPlace]{origPlace} \hyperref[TEI.origin]{origin} \hyperref[TEI.provenance]{provenance} \hyperref[TEI.rubric]{rubric} \hyperref[TEI.secFol]{secFol} \hyperref[TEI.signatures]{signatures} \hyperref[TEI.source]{source} \hyperref[TEI.stamp]{stamp} \hyperref[TEI.summary]{summary} \hyperref[TEI.support]{support} \hyperref[TEI.surrogates]{surrogates} \hyperref[TEI.typeNote]{typeNote} \hyperref[TEI.watermark]{watermark}\par 
    \item[namesdates: ]
   \hyperref[TEI.addName]{addName} \hyperref[TEI.affiliation]{affiliation} \hyperref[TEI.country]{country} \hyperref[TEI.forename]{forename} \hyperref[TEI.genName]{genName} \hyperref[TEI.geogName]{geogName} \hyperref[TEI.nameLink]{nameLink} \hyperref[TEI.orgName]{orgName} \hyperref[TEI.persName]{persName} \hyperref[TEI.placeName]{placeName} \hyperref[TEI.region]{region} \hyperref[TEI.roleName]{roleName} \hyperref[TEI.settlement]{settlement} \hyperref[TEI.surname]{surname}\par 
    \item[textstructure: ]
   \hyperref[TEI.docAuthor]{docAuthor} \hyperref[TEI.docDate]{docDate} \hyperref[TEI.docEdition]{docEdition} \hyperref[TEI.titlePart]{titlePart}\par 
    \item[transcr: ]
   \hyperref[TEI.damage]{damage} \hyperref[TEI.facsimile]{facsimile} \hyperref[TEI.fw]{fw} \hyperref[TEI.metamark]{metamark} \hyperref[TEI.mod]{mod} \hyperref[TEI.restore]{restore} \hyperref[TEI.retrace]{retrace} \hyperref[TEI.secl]{secl} \hyperref[TEI.sourceDoc]{sourceDoc} \hyperref[TEI.supplied]{supplied} \hyperref[TEI.surface]{surface} \hyperref[TEI.surplus]{surplus} \hyperref[TEI.zone]{zone}
    \item[{Peut contenir}]
  
    \item[derived-module-tei.istex: ]
   \hyperref[TEI.menclose]{menclose} \hyperref[TEI.mfenced]{mfenced} \hyperref[TEI.mfrac]{mfrac} \hyperref[TEI.mi]{mi} \hyperref[TEI.mmultiscripts]{mmultiscripts} \hyperref[TEI.mn]{mn} \hyperref[TEI.mo]{mo} \hyperref[TEI.mover]{mover} \hyperref[TEI.mpadded]{mpadded} \hyperref[TEI.mphantom]{mphantom} \hyperref[TEI.mprescripts]{mprescripts} \hyperref[TEI.mrow]{mrow} \hyperref[TEI.mspace]{mspace} \hyperref[TEI.msqrt]{msqrt} \hyperref[TEI.mstyle]{mstyle} \hyperref[TEI.msub]{msub} \hyperref[TEI.msubsup]{msubsup} \hyperref[TEI.msup]{msup} \hyperref[TEI.msupsub]{msupsub} \hyperref[TEI.mtable]{mtable} \hyperref[TEI.mtd]{mtd} \hyperref[TEI.mtext]{mtext} \hyperref[TEI.mtr]{mtr} \hyperref[TEI.munder]{munder} \hyperref[TEI.munderover]{munderover} \hyperref[TEI.none]{none}\par des données textuelles
    \item[{Modèle de contenu}]
  \mbox{}\hfill\\[-10pt]\begin{Verbatim}[fontsize=\small]
<content>
 <alternate maxOccurs="unbounded"
  minOccurs="0">
  <textNode/>
  <elementRef key="mstyle"/>
  <elementRef key="mtr"/>
  <elementRef key="mtd"/>
  <elementRef key="mrow"/>
  <elementRef key="mi"/>
  <elementRef key="mn"/>
  <elementRef key="mtext"/>
  <elementRef key="mfrac"/>
  <elementRef key="mspace"/>
  <elementRef key="msqrt"/>
  <elementRef key="msub"/>
  <elementRef key="msup"/>
  <elementRef key="mo"/>
  <elementRef key="mover"/>
  <elementRef key="mfenced"/>
  <elementRef key="mtable"/>
  <elementRef key="msubsup"/>
  <elementRef key="msupsub"/>
  <elementRef key="mmultiscripts"/>
  <elementRef key="munderover"/>
  <elementRef key="mprescripts"/>
  <elementRef key="none"/>
  <elementRef key="munder"/>
  <elementRef key="mphantom"/>
  <elementRef key="mpadded"/>
  <elementRef key="menclose"/>
 </alternate>
</content>
    
\end{Verbatim}

    \item[{Schéma Declaration}]
  \mbox{}\hfill\\[-10pt]\begin{Verbatim}[fontsize=\small]
element mrow
{
   (
      text
    | tei_mstyle    | tei_mtr    | tei_mtd    | tei_mrow    | tei_mi    | tei_mn    | tei_mtext    | tei_mfrac    | tei_mspace    | tei_msqrt    | tei_msub    | tei_msup    | tei_mo    | tei_mover    | tei_mfenced    | tei_mtable    | tei_msubsup    | tei_msupsub    | tei_mmultiscripts    | tei_munderover    | tei_mprescripts    | tei_none    | tei_munder    | tei_mphantom    | tei_mpadded    | tei_menclose   )*
}
\end{Verbatim}

\end{reflist}  \index{msContents=<msContents>|oddindex}
\begin{reflist}
\item[]\begin{specHead}{TEI.msContents}{<msContents> }(contenu du manuscrit) décrit le contenu intellectuel d'un manuscrit ou d'une partie d'un manuscrit, soit en une série de paragraphes \textit{p}, soit sous la forme d'une série d'éléments structurés \textit{msItem} concernant les items du manuscrit. [\xref{http://www.tei-c.org/release/doc/tei-p5-doc/en/html/MS.html\#msco}{10.6. Intellectual Content}]\end{specHead} 
    \item[{Module}]
  msdescription
    \item[{Attributs}]
  Attributs \hyperref[TEI.att.global]{att.global} (\textit{@xml:id}, \textit{@n}, \textit{@xml:lang}, \textit{@xml:base}, \textit{@xml:space})  (\hyperref[TEI.att.global.rendition]{att.global.rendition} (\textit{@rend}, \textit{@style}, \textit{@rendition})) (\hyperref[TEI.att.global.linking]{att.global.linking} (\textit{@corresp}, \textit{@synch}, \textit{@sameAs}, \textit{@copyOf}, \textit{@next}, \textit{@prev}, \textit{@exclude}, \textit{@select})) (\hyperref[TEI.att.global.analytic]{att.global.analytic} (\textit{@ana})) (\hyperref[TEI.att.global.facs]{att.global.facs} (\textit{@facs})) (\hyperref[TEI.att.global.change]{att.global.change} (\textit{@change})) (\hyperref[TEI.att.global.responsibility]{att.global.responsibility} (\textit{@cert}, \textit{@resp})) (\hyperref[TEI.att.global.source]{att.global.source} (\textit{@source})) \hyperref[TEI.att.msExcerpt]{att.msExcerpt} (\textit{@defective}) \hyperref[TEI.att.msClass]{att.msClass} (\textit{@class}) 
    \item[{Contenu dans}]
  
    \item[msdescription: ]
   \hyperref[TEI.msDesc]{msDesc} \hyperref[TEI.msFrag]{msFrag} \hyperref[TEI.msPart]{msPart}
    \item[{Peut contenir}]
  
    \item[core: ]
   \hyperref[TEI.p]{p} \hyperref[TEI.textLang]{textLang}\par 
    \item[linking: ]
   \hyperref[TEI.ab]{ab}\par 
    \item[msdescription: ]
   \hyperref[TEI.msItem]{msItem} \hyperref[TEI.msItemStruct]{msItemStruct} \hyperref[TEI.summary]{summary}\par 
    \item[textstructure: ]
   \hyperref[TEI.titlePage]{titlePage}
    \item[{Note}]
  \par
A moins qu'il ne contienne une description en texte libre, cet élément doit contenir au moins l'un des éléments \hyperref[TEI.summary]{<summary>}, \hyperref[TEI.msItem]{<msItem>} ou \hyperref[TEI.msItemStruct]{<msItemStruct>}. Actuellement le schéma ne rend pas obligatoire cette contrainte.
    \item[{Exemple}]
  \leavevmode\bgroup\exampleFont \begin{shaded}\noindent\mbox{}{<\textbf{msContents}>}\mbox{}\newline 
\hspace*{6pt}{<\textbf{msItem}>}\mbox{}\newline 
\textit{<!-- pour le traitement des recueils la solution possible est de répéter l'élément <msItem>  -->}\mbox{}\newline 
\hspace*{6pt}\hspace*{6pt}{<\textbf{docAuthor}>}\mbox{}\newline 
\hspace*{6pt}\hspace*{6pt}\hspace*{6pt}{<\textbf{forename}>}Guillaume de Lorris {</\textbf{forename}>}\mbox{}\newline 
\hspace*{6pt}\hspace*{6pt}{</\textbf{docAuthor}>}\mbox{}\newline 
\hspace*{6pt}\hspace*{6pt}{<\textbf{docAuthor}>}\mbox{}\newline 
\hspace*{6pt}\hspace*{6pt}\hspace*{6pt}{<\textbf{forename}>}Jean de Meung{</\textbf{forename}>}\mbox{}\newline 
\hspace*{6pt}\hspace*{6pt}{</\textbf{docAuthor}>}\mbox{}\newline 
\hspace*{6pt}\hspace*{6pt}{<\textbf{docTitle}>}\mbox{}\newline 
\hspace*{6pt}\hspace*{6pt}\hspace*{6pt}{<\textbf{titlePart}\hspace*{6pt}{type}="{main}">}Le Rommant de la rose{</\textbf{titlePart}>}\mbox{}\newline 
\hspace*{6pt}\hspace*{6pt}\hspace*{6pt}{<\textbf{titlePart}\hspace*{6pt}{type}="{sub}"/>}\mbox{}\newline 
\hspace*{6pt}\hspace*{6pt}{</\textbf{docTitle}>}\mbox{}\newline 
\hspace*{6pt}\hspace*{6pt}{<\textbf{docImprint}>}\mbox{}\newline 
\hspace*{6pt}\hspace*{6pt}\hspace*{6pt}{<\textbf{pubPlace}>}Paris{</\textbf{pubPlace}>}\mbox{}\newline 
\hspace*{6pt}\hspace*{6pt}\hspace*{6pt}{<\textbf{publisher}>}Antoine Vérard{</\textbf{publisher}>}\mbox{}\newline 
\hspace*{6pt}\hspace*{6pt}{</\textbf{docImprint}>}\mbox{}\newline 
\hspace*{6pt}\hspace*{6pt}{<\textbf{docDate}\hspace*{6pt}{when}="{1497}">}1497 ou 1498{</\textbf{docDate}>}\mbox{}\newline 
\hspace*{6pt}\hspace*{6pt}{<\textbf{note}>}\mbox{}\newline 
\hspace*{6pt}\hspace*{6pt}\hspace*{6pt}{<\textbf{date}\hspace*{6pt}{notAfter}="{1498-12-31}"\mbox{}\newline 
\hspace*{6pt}\hspace*{6pt}\hspace*{6pt}\hspace*{6pt}{notBefore}="{1497-01-01}"/>}\mbox{}\newline 
\hspace*{6pt}\hspace*{6pt}{</\textbf{note}>}\mbox{}\newline 
\hspace*{6pt}\hspace*{6pt}{<\textbf{note}>}in-2°.{</\textbf{note}>}\mbox{}\newline 
\textit{<!-- in-32°; in-24°; in-16°; in-8°; in-4°; in-folio; gr. folio -->}\mbox{}\newline 
\hspace*{6pt}\hspace*{6pt}{<\textbf{note}>}Exemplaire sur vélin, enluminé, « vraisemblablement dans l’atelier d’Antoine\mbox{}\newline 
\hspace*{6pt}\hspace*{6pt}\hspace*{6pt}\hspace*{6pt} Vérard » {<\textbf{ref}\hspace*{6pt}{target}="{\#fr\textunderscore bib06}">}(Charon 1988, n° 3){</\textbf{ref}>}\mbox{}\newline 
\hspace*{6pt}\hspace*{6pt}{</\textbf{note}>}\mbox{}\newline 
\hspace*{6pt}\hspace*{6pt}{<\textbf{note}>}\mbox{}\newline 
\hspace*{6pt}\hspace*{6pt}\hspace*{6pt}{<\textbf{ref}\hspace*{6pt}{target}="{http://catalogue.bnf.fr/ark:/12148/cb305575966}">}Notice bibliographique\mbox{}\newline 
\hspace*{6pt}\hspace*{6pt}\hspace*{6pt}\hspace*{6pt}\hspace*{6pt}\hspace*{6pt} dans le Catalogue général{</\textbf{ref}>}\mbox{}\newline 
\hspace*{6pt}\hspace*{6pt}{</\textbf{note}>}\mbox{}\newline 
\hspace*{6pt}{</\textbf{msItem}>}\mbox{}\newline 
{</\textbf{msContents}>}\end{shaded}\egroup 


    \item[{Exemple}]
  \leavevmode\bgroup\exampleFont \begin{shaded}\noindent\mbox{}{<\textbf{msContents}>}\mbox{}\newline 
\hspace*{6pt}{<\textbf{msItem}>}\mbox{}\newline 
\textit{<!-- pour le traitement des recueils la solution possible est de répéter l'élément <msItem>  -->}\mbox{}\newline 
\hspace*{6pt}\hspace*{6pt}{<\textbf{docAuthor}>}\mbox{}\newline 
\hspace*{6pt}\hspace*{6pt}\hspace*{6pt}{<\textbf{surname}>}Longus{</\textbf{surname}>}\mbox{}\newline 
\hspace*{6pt}\hspace*{6pt}{</\textbf{docAuthor}>}\mbox{}\newline 
\hspace*{6pt}\hspace*{6pt}{<\textbf{docTitle}>}\mbox{}\newline 
\hspace*{6pt}\hspace*{6pt}\hspace*{6pt}{<\textbf{titlePart}\hspace*{6pt}{type}="{main}">}Les amours pastorales de Daphnis et Chloé{</\textbf{titlePart}>}\mbox{}\newline 
\hspace*{6pt}\hspace*{6pt}{</\textbf{docTitle}>}\mbox{}\newline 
\hspace*{6pt}\hspace*{6pt}{<\textbf{docImprint}>}\mbox{}\newline 
\hspace*{6pt}\hspace*{6pt}\hspace*{6pt}{<\textbf{pubPlace}>}Paris{</\textbf{pubPlace}>}\mbox{}\newline 
\hspace*{6pt}\hspace*{6pt}\hspace*{6pt}{<\textbf{publisher}>}[Jacques Quillau]{</\textbf{publisher}>}\mbox{}\newline 
\hspace*{6pt}\hspace*{6pt}{</\textbf{docImprint}>}\mbox{}\newline 
\hspace*{6pt}\hspace*{6pt}{<\textbf{docDate}\hspace*{6pt}{when}="{1718}">}1718{</\textbf{docDate}>}\mbox{}\newline 
\hspace*{6pt}\hspace*{6pt}{<\textbf{note}>}in-8°.{</\textbf{note}>}\mbox{}\newline 
\textit{<!-- in-32°; in-24°; in-16°; in-8°; in-4°; in-folio; gr. folio -->}\mbox{}\newline 
\hspace*{6pt}\hspace*{6pt}{<\textbf{note}>}Exemplaire réglé.{</\textbf{note}>}\mbox{}\newline 
\hspace*{6pt}\hspace*{6pt}{<\textbf{note}>}\mbox{}\newline 
\hspace*{6pt}\hspace*{6pt}\hspace*{6pt}{<\textbf{ref}\hspace*{6pt}{target}="{http://catalogue.bnf.fr/ark:/12148/cb30831232s}">}Notice bibliographique\mbox{}\newline 
\hspace*{6pt}\hspace*{6pt}\hspace*{6pt}\hspace*{6pt}\hspace*{6pt}\hspace*{6pt} dans le Catalogue général{</\textbf{ref}>}\mbox{}\newline 
\hspace*{6pt}\hspace*{6pt}{</\textbf{note}>}\mbox{}\newline 
\hspace*{6pt}{</\textbf{msItem}>}\mbox{}\newline 
{</\textbf{msContents}>}\end{shaded}\egroup 


    \item[{Modèle de contenu}]
  \mbox{}\hfill\\[-10pt]\begin{Verbatim}[fontsize=\small]
<content>
 <alternate maxOccurs="1" minOccurs="1">
  <classRef key="model.pLike"
   maxOccurs="unbounded" minOccurs="1"/>
  <sequence maxOccurs="1" minOccurs="1">
   <elementRef key="summary" minOccurs="0"/>
   <elementRef key="textLang" minOccurs="0"/>
   <elementRef key="titlePage"
    minOccurs="0"/>
   <alternate maxOccurs="unbounded"
    minOccurs="0">
    <elementRef key="msItem"/>
    <elementRef key="msItemStruct"/>
   </alternate>
  </sequence>
 </alternate>
</content>
    
\end{Verbatim}

    \item[{Schéma Declaration}]
  \mbox{}\hfill\\[-10pt]\begin{Verbatim}[fontsize=\small]
element msContents
{
   tei_att.global.attributes,
   tei_att.msExcerpt.attributes,
   tei_att.msClass.attributes,
   (
      tei_model.pLike+
    | (
         tei_summary?,
         tei_textLang?,
         tei_titlePage?,
         ( tei_msItem | tei_msItemStruct )*
      )
   )
}
\end{Verbatim}

\end{reflist}  \index{msDesc=<msDesc>|oddindex}
\begin{reflist}
\item[]\begin{specHead}{TEI.msDesc}{<msDesc> }(description d'un manuscrit) contient la description d'un manuscrit individuel [\xref{http://www.tei-c.org/release/doc/tei-p5-doc/en/html/MS.html\#msov}{10.1. Overview}]\end{specHead} 
    \item[{Module}]
  msdescription
    \item[{Attributs}]
  Attributs \hyperref[TEI.att.global]{att.global} (\textit{@xml:id}, \textit{@n}, \textit{@xml:lang}, \textit{@xml:base}, \textit{@xml:space})  (\hyperref[TEI.att.global.rendition]{att.global.rendition} (\textit{@rend}, \textit{@style}, \textit{@rendition})) (\hyperref[TEI.att.global.linking]{att.global.linking} (\textit{@corresp}, \textit{@synch}, \textit{@sameAs}, \textit{@copyOf}, \textit{@next}, \textit{@prev}, \textit{@exclude}, \textit{@select})) (\hyperref[TEI.att.global.analytic]{att.global.analytic} (\textit{@ana})) (\hyperref[TEI.att.global.facs]{att.global.facs} (\textit{@facs})) (\hyperref[TEI.att.global.change]{att.global.change} (\textit{@change})) (\hyperref[TEI.att.global.responsibility]{att.global.responsibility} (\textit{@cert}, \textit{@resp})) (\hyperref[TEI.att.global.source]{att.global.source} (\textit{@source})) \hyperref[TEI.att.sortable]{att.sortable} (\textit{@sortKey}) \hyperref[TEI.att.typed]{att.typed} (\textit{@type}, \textit{@subtype}) \hyperref[TEI.att.declaring]{att.declaring} (\textit{@decls}) \hyperref[TEI.att.docStatus]{att.docStatus} (\textit{@status}) 
    \item[{Membre du}]
  \hyperref[TEI.model.biblLike]{model.biblLike}
    \item[{Contenu dans}]
  
    \item[core: ]
   \hyperref[TEI.add]{add} \hyperref[TEI.cit]{cit} \hyperref[TEI.corr]{corr} \hyperref[TEI.del]{del} \hyperref[TEI.desc]{desc} \hyperref[TEI.emph]{emph} \hyperref[TEI.head]{head} \hyperref[TEI.hi]{hi} \hyperref[TEI.item]{item} \hyperref[TEI.l]{l} \hyperref[TEI.listBibl]{listBibl} \hyperref[TEI.meeting]{meeting} \hyperref[TEI.note]{note} \hyperref[TEI.orig]{orig} \hyperref[TEI.p]{p} \hyperref[TEI.q]{q} \hyperref[TEI.quote]{quote} \hyperref[TEI.ref]{ref} \hyperref[TEI.reg]{reg} \hyperref[TEI.relatedItem]{relatedItem} \hyperref[TEI.said]{said} \hyperref[TEI.sic]{sic} \hyperref[TEI.stage]{stage} \hyperref[TEI.title]{title} \hyperref[TEI.unclear]{unclear}\par 
    \item[figures: ]
   \hyperref[TEI.cell]{cell} \hyperref[TEI.figDesc]{figDesc} \hyperref[TEI.figure]{figure}\par 
    \item[header: ]
   \hyperref[TEI.change]{change} \hyperref[TEI.licence]{licence} \hyperref[TEI.rendition]{rendition} \hyperref[TEI.sourceDesc]{sourceDesc} \hyperref[TEI.taxonomy]{taxonomy}\par 
    \item[iso-fs: ]
   \hyperref[TEI.fDescr]{fDescr} \hyperref[TEI.fsDescr]{fsDescr}\par 
    \item[linking: ]
   \hyperref[TEI.ab]{ab} \hyperref[TEI.seg]{seg}\par 
    \item[msdescription: ]
   \hyperref[TEI.accMat]{accMat} \hyperref[TEI.acquisition]{acquisition} \hyperref[TEI.additions]{additions} \hyperref[TEI.collation]{collation} \hyperref[TEI.condition]{condition} \hyperref[TEI.custEvent]{custEvent} \hyperref[TEI.decoNote]{decoNote} \hyperref[TEI.filiation]{filiation} \hyperref[TEI.foliation]{foliation} \hyperref[TEI.layout]{layout} \hyperref[TEI.msItem]{msItem} \hyperref[TEI.musicNotation]{musicNotation} \hyperref[TEI.origin]{origin} \hyperref[TEI.provenance]{provenance} \hyperref[TEI.signatures]{signatures} \hyperref[TEI.source]{source} \hyperref[TEI.summary]{summary} \hyperref[TEI.support]{support} \hyperref[TEI.surrogates]{surrogates} \hyperref[TEI.typeNote]{typeNote}\par 
    \item[namesdates: ]
   \hyperref[TEI.event]{event} \hyperref[TEI.location]{location} \hyperref[TEI.org]{org} \hyperref[TEI.person]{person} \hyperref[TEI.personGrp]{personGrp} \hyperref[TEI.persona]{persona} \hyperref[TEI.place]{place} \hyperref[TEI.state]{state}\par 
    \item[spoken: ]
   \hyperref[TEI.annotationBlock]{annotationBlock}\par 
    \item[standOff: ]
   \hyperref[TEI.listAnnotation]{listAnnotation}\par 
    \item[textstructure: ]
   \hyperref[TEI.body]{body} \hyperref[TEI.div]{div} \hyperref[TEI.docEdition]{docEdition} \hyperref[TEI.titlePart]{titlePart}\par 
    \item[transcr: ]
   \hyperref[TEI.damage]{damage} \hyperref[TEI.metamark]{metamark} \hyperref[TEI.mod]{mod} \hyperref[TEI.restore]{restore} \hyperref[TEI.retrace]{retrace} \hyperref[TEI.secl]{secl} \hyperref[TEI.supplied]{supplied} \hyperref[TEI.surplus]{surplus}
    \item[{Peut contenir}]
  
    \item[core: ]
   \hyperref[TEI.head]{head} \hyperref[TEI.p]{p}\par 
    \item[linking: ]
   \hyperref[TEI.ab]{ab}\par 
    \item[msdescription: ]
   \hyperref[TEI.additional]{additional} \hyperref[TEI.history]{history} \hyperref[TEI.msContents]{msContents} \hyperref[TEI.msFrag]{msFrag} \hyperref[TEI.msIdentifier]{msIdentifier} \hyperref[TEI.msPart]{msPart} \hyperref[TEI.physDesc]{physDesc}
    \item[{Exemple}]
  \leavevmode\bgroup\exampleFont \begin{shaded}\noindent\mbox{}{<\textbf{msDesc}>}\mbox{}\newline 
\hspace*{6pt}{<\textbf{msIdentifier}>}\mbox{}\newline 
\hspace*{6pt}\hspace*{6pt}{<\textbf{country}>}France{</\textbf{country}>}\mbox{}\newline 
\hspace*{6pt}\hspace*{6pt}{<\textbf{settlement}>}Paris{</\textbf{settlement}>}\mbox{}\newline 
\hspace*{6pt}\hspace*{6pt}{<\textbf{repository}\hspace*{6pt}{xml:lang}="{fr}">}Bibliothèque nationale de France. Réserve des livres rares>{</\textbf{repository}>}\mbox{}\newline 
\hspace*{6pt}\hspace*{6pt}{<\textbf{idno}>}RES P- YC- 1275{</\textbf{idno}>}\mbox{}\newline 
\textit{<!-- dans le cas des recueils : cote uniquement sans les sous-cotes -->}\mbox{}\newline 
\hspace*{6pt}\hspace*{6pt}{<\textbf{altIdentifier}>}\mbox{}\newline 
\hspace*{6pt}\hspace*{6pt}\hspace*{6pt}{<\textbf{idno}>}Y. 1341{</\textbf{idno}>}\mbox{}\newline 
\hspace*{6pt}\hspace*{6pt}\hspace*{6pt}{<\textbf{note}>}Cote de la Bibliothèque royale au XVIIIe s. (Catalogue de 1750).{</\textbf{note}>}\mbox{}\newline 
\hspace*{6pt}\hspace*{6pt}{</\textbf{altIdentifier}>}\mbox{}\newline 
\hspace*{6pt}{</\textbf{msIdentifier}>}\mbox{}\newline 
\hspace*{6pt}{<\textbf{msContents}>}\mbox{}\newline 
\hspace*{6pt}\hspace*{6pt}{<\textbf{msItem}>}\mbox{}\newline 
\textit{<!-- pour le traitement des recueils la solution possible est de répéter l'élément <msItem>  -->}\mbox{}\newline 
\hspace*{6pt}\hspace*{6pt}\hspace*{6pt}{<\textbf{docAuthor}>}\mbox{}\newline 
\hspace*{6pt}\hspace*{6pt}\hspace*{6pt}\hspace*{6pt}{<\textbf{forename}>}Juvénal{</\textbf{forename}>}\mbox{}\newline 
\hspace*{6pt}\hspace*{6pt}\hspace*{6pt}{</\textbf{docAuthor}>}\mbox{}\newline 
\hspace*{6pt}\hspace*{6pt}\hspace*{6pt}{<\textbf{docAuthor}>}\mbox{}\newline 
\hspace*{6pt}\hspace*{6pt}\hspace*{6pt}\hspace*{6pt}{<\textbf{forename}>}Perse{</\textbf{forename}>}\mbox{}\newline 
\hspace*{6pt}\hspace*{6pt}\hspace*{6pt}{</\textbf{docAuthor}>}\mbox{}\newline 
\hspace*{6pt}\hspace*{6pt}\hspace*{6pt}{<\textbf{docTitle}>}\mbox{}\newline 
\hspace*{6pt}\hspace*{6pt}\hspace*{6pt}\hspace*{6pt}{<\textbf{titlePart}\hspace*{6pt}{type}="{main}"/>}\mbox{}\newline 
\hspace*{6pt}\hspace*{6pt}\hspace*{6pt}\hspace*{6pt}{<\textbf{titlePart}\hspace*{6pt}{type}="{sub}"/>}\mbox{}\newline 
\hspace*{6pt}\hspace*{6pt}\hspace*{6pt}{</\textbf{docTitle}>}\mbox{}\newline 
\hspace*{6pt}\hspace*{6pt}\hspace*{6pt}{<\textbf{docImprint}>}\mbox{}\newline 
\hspace*{6pt}\hspace*{6pt}\hspace*{6pt}\hspace*{6pt}{<\textbf{pubPlace}>}Venise{</\textbf{pubPlace}>}\mbox{}\newline 
\hspace*{6pt}\hspace*{6pt}\hspace*{6pt}\hspace*{6pt}{<\textbf{publisher}>}F. Torresani{</\textbf{publisher}>}\mbox{}\newline 
\textit{<!-- dans le Catalogue général: "in aedibus haeredum Aldi et Andreae soceri" -->}\mbox{}\newline 
\hspace*{6pt}\hspace*{6pt}\hspace*{6pt}\hspace*{6pt}{<\textbf{publisher}>}G.-F. Torresani{</\textbf{publisher}>}\mbox{}\newline 
\hspace*{6pt}\hspace*{6pt}\hspace*{6pt}{</\textbf{docImprint}>}\mbox{}\newline 
\hspace*{6pt}\hspace*{6pt}\hspace*{6pt}{<\textbf{docDate}\hspace*{6pt}{when}="{1535}">}1535{</\textbf{docDate}>}\mbox{}\newline 
\hspace*{6pt}\hspace*{6pt}\hspace*{6pt}{<\textbf{note}>}In-8°.{</\textbf{note}>}\mbox{}\newline 
\textit{<!-- in-32°; in-24°; in-16°; in-8°; in-4°; in-folio; gr. folio -->}\mbox{}\newline 
\hspace*{6pt}\hspace*{6pt}\hspace*{6pt}{<\textbf{note}>}Exemplaire avec rehauts peints en argent, or et bleu.{</\textbf{note}>}\mbox{}\newline 
\hspace*{6pt}\hspace*{6pt}\hspace*{6pt}{<\textbf{note}>}\mbox{}\newline 
\hspace*{6pt}\hspace*{6pt}\hspace*{6pt}\hspace*{6pt}{<\textbf{ref}\hspace*{6pt}{target}="{http://catalogue.bnf.fr/ark:/12148/cb31088624r}">}Notice bibliographique\mbox{}\newline 
\hspace*{6pt}\hspace*{6pt}\hspace*{6pt}\hspace*{6pt}\hspace*{6pt}\hspace*{6pt}\hspace*{6pt}\hspace*{6pt} dans le Catalogue général{</\textbf{ref}>}\mbox{}\newline 
\hspace*{6pt}\hspace*{6pt}\hspace*{6pt}{</\textbf{note}>}\mbox{}\newline 
\hspace*{6pt}\hspace*{6pt}{</\textbf{msItem}>}\mbox{}\newline 
\hspace*{6pt}{</\textbf{msContents}>}\mbox{}\newline 
\hspace*{6pt}{<\textbf{physDesc}>}\mbox{}\newline 
\hspace*{6pt}\hspace*{6pt}{<\textbf{p}>}\mbox{}\newline 
\hspace*{6pt}\hspace*{6pt}\hspace*{6pt}{<\textbf{ref}\hspace*{6pt}{target}="{http://bnf.fr/ark://}">}Image de la reliure dans l'iconothèque{</\textbf{ref}>}\mbox{}\newline 
\textit{<!-- RC-B-05225 (plat sup.) -->}\mbox{}\newline 
\hspace*{6pt}\hspace*{6pt}{</\textbf{p}>}\mbox{}\newline 
\hspace*{6pt}\hspace*{6pt}{<\textbf{objectDesc}>}\mbox{}\newline 
\hspace*{6pt}\hspace*{6pt}\hspace*{6pt}{<\textbf{supportDesc}>}\mbox{}\newline 
\hspace*{6pt}\hspace*{6pt}\hspace*{6pt}\hspace*{6pt}{<\textbf{extent}>}\mbox{}\newline 
\hspace*{6pt}\hspace*{6pt}\hspace*{6pt}\hspace*{6pt}\hspace*{6pt}{<\textbf{dimensions}\hspace*{6pt}{type}="{binding}">}\mbox{}\newline 
\hspace*{6pt}\hspace*{6pt}\hspace*{6pt}\hspace*{6pt}\hspace*{6pt}\hspace*{6pt}{<\textbf{height}\hspace*{6pt}{unit}="{mm}">}170{</\textbf{height}>}\mbox{}\newline 
\hspace*{6pt}\hspace*{6pt}\hspace*{6pt}\hspace*{6pt}\hspace*{6pt}\hspace*{6pt}{<\textbf{width}\hspace*{6pt}{unit}="{mm}">}98{</\textbf{width}>}\mbox{}\newline 
\hspace*{6pt}\hspace*{6pt}\hspace*{6pt}\hspace*{6pt}\hspace*{6pt}\hspace*{6pt}{<\textbf{depth}\hspace*{6pt}{unit}="{mm}">}15{</\textbf{depth}>}\mbox{}\newline 
\hspace*{6pt}\hspace*{6pt}\hspace*{6pt}\hspace*{6pt}\hspace*{6pt}{</\textbf{dimensions}>}\mbox{}\newline 
\hspace*{6pt}\hspace*{6pt}\hspace*{6pt}\hspace*{6pt}{</\textbf{extent}>}\mbox{}\newline 
\hspace*{6pt}\hspace*{6pt}\hspace*{6pt}{</\textbf{supportDesc}>}\mbox{}\newline 
\hspace*{6pt}\hspace*{6pt}{</\textbf{objectDesc}>}\mbox{}\newline 
\hspace*{6pt}\hspace*{6pt}{<\textbf{bindingDesc}>}\mbox{}\newline 
\hspace*{6pt}\hspace*{6pt}\hspace*{6pt}{<\textbf{binding}\hspace*{6pt}{contemporary}="{true}">}\mbox{}\newline 
\hspace*{6pt}\hspace*{6pt}\hspace*{6pt}\hspace*{6pt}{<\textbf{p}>}\mbox{}\newline 
\hspace*{6pt}\hspace*{6pt}\hspace*{6pt}\hspace*{6pt}\hspace*{6pt}{<\textbf{index}\hspace*{6pt}{indexName}="{typo\textunderscore reliure}">}\mbox{}\newline 
\hspace*{6pt}\hspace*{6pt}\hspace*{6pt}\hspace*{6pt}\hspace*{6pt}\hspace*{6pt}{<\textbf{term}>}Reliure à décor{</\textbf{term}>}\mbox{}\newline 
\hspace*{6pt}\hspace*{6pt}\hspace*{6pt}\hspace*{6pt}\hspace*{6pt}{</\textbf{index}>}\mbox{}\newline 
\hspace*{6pt}\hspace*{6pt}\hspace*{6pt}\hspace*{6pt}\hspace*{6pt}{<\textbf{index}\hspace*{6pt}{indexName}="{typo\textunderscore decor}">}\mbox{}\newline 
\hspace*{6pt}\hspace*{6pt}\hspace*{6pt}\hspace*{6pt}\hspace*{6pt}\hspace*{6pt}{<\textbf{term}>}Entrelacs géométriques{</\textbf{term}>}\mbox{}\newline 
\hspace*{6pt}\hspace*{6pt}\hspace*{6pt}\hspace*{6pt}\hspace*{6pt}{</\textbf{index}>} Reliure en {<\textbf{material}>}maroquin{</\textbf{material}>} brun jaspé{</\textbf{p}>}\mbox{}\newline 
\hspace*{6pt}\hspace*{6pt}\hspace*{6pt}\hspace*{6pt}{<\textbf{decoNote}\hspace*{6pt}{type}="{plats}">} à décor d’entrelacs géométriques (structure de losange et\mbox{}\newline 
\hspace*{6pt}\hspace*{6pt}\hspace*{6pt}\hspace*{6pt}\hspace*{6pt}\hspace*{6pt}\hspace*{6pt}\hspace*{6pt} rectangle) complété de fers évidés.{</\textbf{decoNote}>}\mbox{}\newline 
\hspace*{6pt}\hspace*{6pt}\hspace*{6pt}\hspace*{6pt}{<\textbf{decoNote}\hspace*{6pt}{type}="{plat\textunderscore sup}">}Titre {<\textbf{q}>}ivvenalis. persivs{</\textbf{q}>} et ex-libris de Jean\mbox{}\newline 
\hspace*{6pt}\hspace*{6pt}\hspace*{6pt}\hspace*{6pt}\hspace*{6pt}\hspace*{6pt}\hspace*{6pt}\hspace*{6pt} Grolier {<\textbf{q}>}io. grolierii et amicorvm.{</\textbf{q}>} dorés respectivement au centre et au bas\mbox{}\newline 
\hspace*{6pt}\hspace*{6pt}\hspace*{6pt}\hspace*{6pt}\hspace*{6pt}\hspace*{6pt}\hspace*{6pt}\hspace*{6pt} du plat supérieur. {</\textbf{decoNote}>}\mbox{}\newline 
\hspace*{6pt}\hspace*{6pt}\hspace*{6pt}\hspace*{6pt}{<\textbf{decoNote}\hspace*{6pt}{type}="{plat\textunderscore inf}">}Devise de Jean Grolier{<\textbf{q}>}portio mea sit in terra\mbox{}\newline 
\hspace*{6pt}\hspace*{6pt}\hspace*{6pt}\hspace*{6pt}\hspace*{6pt}\hspace*{6pt}\hspace*{6pt}\hspace*{6pt}\hspace*{6pt}\hspace*{6pt} viventivm{</\textbf{q}>} dorée au centre du plat inférieur.{</\textbf{decoNote}>}\mbox{}\newline 
\hspace*{6pt}\hspace*{6pt}\hspace*{6pt}\hspace*{6pt}{<\textbf{decoNote}\hspace*{6pt}{type}="{dos}">}Dos à cinq nerfs, sans décor ; simple filet doré sur chaque\mbox{}\newline 
\hspace*{6pt}\hspace*{6pt}\hspace*{6pt}\hspace*{6pt}\hspace*{6pt}\hspace*{6pt}\hspace*{6pt}\hspace*{6pt} nerf et en encadrement des caissons ; passages de chaînette marqués de même.{</\textbf{decoNote}>}\mbox{}\newline 
\hspace*{6pt}\hspace*{6pt}\hspace*{6pt}\hspace*{6pt}{<\textbf{decoNote}\hspace*{6pt}{type}="{tranchefiles}">}Tranchefiles simples unicolores, vert foncé.{</\textbf{decoNote}>}\mbox{}\newline 
\hspace*{6pt}\hspace*{6pt}\hspace*{6pt}\hspace*{6pt}{<\textbf{decoNote}\hspace*{6pt}{type}="{coupes}">}Filet doré sur les coupes.{</\textbf{decoNote}>}\mbox{}\newline 
\hspace*{6pt}\hspace*{6pt}\hspace*{6pt}\hspace*{6pt}{<\textbf{decoNote}\hspace*{6pt}{type}="{annexes}"/>}\mbox{}\newline 
\hspace*{6pt}\hspace*{6pt}\hspace*{6pt}\hspace*{6pt}{<\textbf{decoNote}\hspace*{6pt}{type}="{tranches}">}Tranches dorées.{</\textbf{decoNote}>}\mbox{}\newline 
\hspace*{6pt}\hspace*{6pt}\hspace*{6pt}\hspace*{6pt}{<\textbf{decoNote}\hspace*{6pt}{type}="{contreplats}">}Contreplats en vélin.{</\textbf{decoNote}>}\mbox{}\newline 
\hspace*{6pt}\hspace*{6pt}\hspace*{6pt}\hspace*{6pt}{<\textbf{decoNote}\hspace*{6pt}{type}="{chasses}">}Filet doré sur les chasses.{</\textbf{decoNote}>}\mbox{}\newline 
\textit{<!-- Description des gardes : gardes blanches ; gardes couleurs (marbrées, gaufrées, peintes, dominotées, etc.) généralement suivies de gardes blanches ; dans tous les cas, spécifier le nombre de gardes (début + fin du volume)-->}\mbox{}\newline 
\hspace*{6pt}\hspace*{6pt}\hspace*{6pt}\hspace*{6pt}{<\textbf{decoNote}\hspace*{6pt}{type}="{gardes}">}Gardes en papier et vélin (2+1+2 / 2+1+2) ; filigrane au\mbox{}\newline 
\hspace*{6pt}\hspace*{6pt}\hspace*{6pt}\hspace*{6pt}\hspace*{6pt}\hspace*{6pt}\hspace*{6pt}\hspace*{6pt} pot.{<\textbf{ref}>}Briquet N° XX{</\textbf{ref}>}\mbox{}\newline 
\hspace*{6pt}\hspace*{6pt}\hspace*{6pt}\hspace*{6pt}{</\textbf{decoNote}>}\mbox{}\newline 
\textit{<!-- Élément qui inclut aussi bien des remarques sur la couture que les charnières, claies ou modes d'attaches des plats : tous éléments de la structure dont la description est jugée utile à la description et l'identification de la reliure-->}\mbox{}\newline 
\hspace*{6pt}\hspace*{6pt}\hspace*{6pt}\hspace*{6pt}{<\textbf{decoNote}\hspace*{6pt}{type}="{structure}">}Defet manuscrit utilisé comme claie au contreplat\mbox{}\newline 
\hspace*{6pt}\hspace*{6pt}\hspace*{6pt}\hspace*{6pt}\hspace*{6pt}\hspace*{6pt}\hspace*{6pt}\hspace*{6pt} inférieur (visible par transparence, sous la contregarde en vélin).{</\textbf{decoNote}>}\mbox{}\newline 
\hspace*{6pt}\hspace*{6pt}\hspace*{6pt}\hspace*{6pt}{<\textbf{condition}>}Traces de mouillures anciennes plus ou moins importantes au bas des\mbox{}\newline 
\hspace*{6pt}\hspace*{6pt}\hspace*{6pt}\hspace*{6pt}\hspace*{6pt}\hspace*{6pt}\hspace*{6pt}\hspace*{6pt} feuillets, qui n'ont pas affecté la reliure ; éraflure en tête du plat\mbox{}\newline 
\hspace*{6pt}\hspace*{6pt}\hspace*{6pt}\hspace*{6pt}\hspace*{6pt}\hspace*{6pt}\hspace*{6pt}\hspace*{6pt} inférieur.{</\textbf{condition}>}\mbox{}\newline 
\hspace*{6pt}\hspace*{6pt}\hspace*{6pt}{</\textbf{binding}>}\mbox{}\newline 
\hspace*{6pt}\hspace*{6pt}{</\textbf{bindingDesc}>}\mbox{}\newline 
\hspace*{6pt}{</\textbf{physDesc}>}\mbox{}\newline 
\hspace*{6pt}{<\textbf{history}>}\mbox{}\newline 
\hspace*{6pt}\hspace*{6pt}{<\textbf{origin}\hspace*{6pt}{notAfter}="{1547-09-15}"\mbox{}\newline 
\hspace*{6pt}\hspace*{6pt}\hspace*{6pt}{notBefore}="{1540-01-01}">}\mbox{}\newline 
\hspace*{6pt}\hspace*{6pt}\hspace*{6pt}{<\textbf{p}>}Reliure exécutée pour Jean Grolier par Jean Picard, Paris, entre 1540 et 1547.{</\textbf{p}>}\mbox{}\newline 
\hspace*{6pt}\hspace*{6pt}{</\textbf{origin}>}\mbox{}\newline 
\hspace*{6pt}\hspace*{6pt}{<\textbf{provenance}>}\mbox{}\newline 
\hspace*{6pt}\hspace*{6pt}\hspace*{6pt}{<\textbf{p}/>}\mbox{}\newline 
\hspace*{6pt}\hspace*{6pt}{</\textbf{provenance}>}\mbox{}\newline 
\hspace*{6pt}\hspace*{6pt}{<\textbf{acquisition}\hspace*{6pt}{notAfter}="{1724-12-31}"\mbox{}\newline 
\hspace*{6pt}\hspace*{6pt}\hspace*{6pt}{notBefore}="{1680-12-31}">}Estampille n° 1, utilisée de\mbox{}\newline 
\hspace*{6pt}\hspace*{6pt}\hspace*{6pt}\hspace*{6pt} la fin du XVIIe siècle à 1724 (page de titre).{</\textbf{acquisition}>}\mbox{}\newline 
\hspace*{6pt}{</\textbf{history}>}\mbox{}\newline 
\hspace*{6pt}{<\textbf{additional}>}\mbox{}\newline 
\hspace*{6pt}\hspace*{6pt}{<\textbf{adminInfo}>}\mbox{}\newline 
\hspace*{6pt}\hspace*{6pt}\hspace*{6pt}{<\textbf{recordHist}>}\mbox{}\newline 
\hspace*{6pt}\hspace*{6pt}\hspace*{6pt}\hspace*{6pt}{<\textbf{source}>}Notice établie à partir du document original{</\textbf{source}>}\mbox{}\newline 
\hspace*{6pt}\hspace*{6pt}\hspace*{6pt}\hspace*{6pt}{<\textbf{change}\hspace*{6pt}{when}="{2009-10-05}"\hspace*{6pt}{who}="{Markova}">}Description mise à jour le {<\textbf{date}\hspace*{6pt}{type}="{crea}">}5 octobre 2009 {</\textbf{date}>}en vue de l'encodage en TEI des descriptions des reliure\mbox{}\newline 
\hspace*{6pt}\hspace*{6pt}\hspace*{6pt}\hspace*{6pt}\hspace*{6pt}\hspace*{6pt}\hspace*{6pt}\hspace*{6pt} de la Réserve des livres rares{</\textbf{change}>}\mbox{}\newline 
\hspace*{6pt}\hspace*{6pt}\hspace*{6pt}\hspace*{6pt}{<\textbf{change}\hspace*{6pt}{when}="{2009-06-01}"\hspace*{6pt}{who}="{Le Bars}">}Description revue le {<\textbf{date}\hspace*{6pt}{type}="{maj}">}1er\mbox{}\newline 
\hspace*{6pt}\hspace*{6pt}\hspace*{6pt}\hspace*{6pt}\hspace*{6pt}\hspace*{6pt}\hspace*{6pt}\hspace*{6pt}\hspace*{6pt}\hspace*{6pt} juin 2009 {</\textbf{date}>} par Fabienne Le Bars{</\textbf{change}>}\mbox{}\newline 
\hspace*{6pt}\hspace*{6pt}\hspace*{6pt}\hspace*{6pt}{<\textbf{change}\hspace*{6pt}{when}="{2009-06-25}"\hspace*{6pt}{who}="{Le Bars}">}Description validée le{<\textbf{date}\hspace*{6pt}{type}="{valid}">}25\mbox{}\newline 
\hspace*{6pt}\hspace*{6pt}\hspace*{6pt}\hspace*{6pt}\hspace*{6pt}\hspace*{6pt}\hspace*{6pt}\hspace*{6pt}\hspace*{6pt}\hspace*{6pt} juin 2009{</\textbf{date}>}par Fabienne Le Bars{</\textbf{change}>}\mbox{}\newline 
\hspace*{6pt}\hspace*{6pt}\hspace*{6pt}{</\textbf{recordHist}>}\mbox{}\newline 
\hspace*{6pt}\hspace*{6pt}{</\textbf{adminInfo}>}\mbox{}\newline 
\hspace*{6pt}{</\textbf{additional}>}\mbox{}\newline 
{</\textbf{msDesc}>}\end{shaded}\egroup 


    \item[{Modèle de contenu}]
  \mbox{}\hfill\\[-10pt]\begin{Verbatim}[fontsize=\small]
<content>
 <sequence maxOccurs="1" minOccurs="1">
  <elementRef key="msIdentifier"/>
  <classRef key="model.headLike"
   maxOccurs="unbounded" minOccurs="0"/>
  <alternate maxOccurs="1" minOccurs="1">
   <classRef key="model.pLike"
    maxOccurs="unbounded" minOccurs="1"/>
   <sequence maxOccurs="1" minOccurs="1">
    <elementRef key="msContents"
     minOccurs="0"/>
    <elementRef key="physDesc"
     minOccurs="0"/>
    <elementRef key="history" minOccurs="0"/>
    <elementRef key="additional"
     minOccurs="0"/>
    <alternate maxOccurs="1" minOccurs="1">
     <elementRef key="msPart"
      maxOccurs="unbounded" minOccurs="0"/>
     <elementRef key="msFrag"
      maxOccurs="unbounded" minOccurs="0"/>
    </alternate>
   </sequence>
  </alternate>
 </sequence>
</content>
    
\end{Verbatim}

    \item[{Schéma Declaration}]
  \mbox{}\hfill\\[-10pt]\begin{Verbatim}[fontsize=\small]
element msDesc
{
   tei_att.global.attributes,
   tei_att.sortable.attributes,
   tei_att.typed.attributes,
   tei_att.declaring.attributes,
   tei_att.docStatus.attributes,
   (
      tei_msIdentifier,
      tei_model.headLike*,
      (
         tei_model.pLike+
       | (
            tei_msContents?,
            tei_physDesc?,
            tei_history?,
            tei_additional?,
            ( tei_msPart* | tei_msFrag* )
         )
      )
   )
}
\end{Verbatim}

\end{reflist}  \index{msFrag=<msFrag>|oddindex}
\begin{reflist}
\item[]\begin{specHead}{TEI.msFrag}{<msFrag> }(fragment d'un manuscrit) contient des informations sur un fragment d'un manuscrit dispersé, aujourd'hui conservé séparément ou incorporé dans un autre manuscrit. [\xref{http://www.tei-c.org/release/doc/tei-p5-doc/en/html/MS.html\#msfg}{10.11. Manuscript Fragments}]\end{specHead} 
    \item[{Module}]
  msdescription
    \item[{Attributs}]
  Attributs \hyperref[TEI.att.global]{att.global} (\textit{@xml:id}, \textit{@n}, \textit{@xml:lang}, \textit{@xml:base}, \textit{@xml:space})  (\hyperref[TEI.att.global.rendition]{att.global.rendition} (\textit{@rend}, \textit{@style}, \textit{@rendition})) (\hyperref[TEI.att.global.linking]{att.global.linking} (\textit{@corresp}, \textit{@synch}, \textit{@sameAs}, \textit{@copyOf}, \textit{@next}, \textit{@prev}, \textit{@exclude}, \textit{@select})) (\hyperref[TEI.att.global.analytic]{att.global.analytic} (\textit{@ana})) (\hyperref[TEI.att.global.facs]{att.global.facs} (\textit{@facs})) (\hyperref[TEI.att.global.change]{att.global.change} (\textit{@change})) (\hyperref[TEI.att.global.responsibility]{att.global.responsibility} (\textit{@cert}, \textit{@resp})) (\hyperref[TEI.att.global.source]{att.global.source} (\textit{@source})) \hyperref[TEI.att.typed]{att.typed} (\textit{@type}, \textit{@subtype}) 
    \item[{Contenu dans}]
  
    \item[msdescription: ]
   \hyperref[TEI.msDesc]{msDesc}
    \item[{Peut contenir}]
  
    \item[core: ]
   \hyperref[TEI.head]{head} \hyperref[TEI.p]{p}\par 
    \item[linking: ]
   \hyperref[TEI.ab]{ab}\par 
    \item[msdescription: ]
   \hyperref[TEI.additional]{additional} \hyperref[TEI.altIdentifier]{altIdentifier} \hyperref[TEI.history]{history} \hyperref[TEI.msContents]{msContents} \hyperref[TEI.msIdentifier]{msIdentifier} \hyperref[TEI.physDesc]{physDesc}
    \item[{Exemple}]
  \leavevmode\bgroup\exampleFont \begin{shaded}\noindent\mbox{}{<\textbf{msDesc}>}\mbox{}\newline 
\hspace*{6pt}{<\textbf{msIdentifier}>}\mbox{}\newline 
\hspace*{6pt}\hspace*{6pt}{<\textbf{msName}\hspace*{6pt}{xml:lang}="{la}">}Codex Suprasliensis{</\textbf{msName}>}\mbox{}\newline 
\hspace*{6pt}{</\textbf{msIdentifier}>}\mbox{}\newline 
\hspace*{6pt}{<\textbf{msFrag}>}\mbox{}\newline 
\hspace*{6pt}\hspace*{6pt}{<\textbf{msIdentifier}>}\mbox{}\newline 
\hspace*{6pt}\hspace*{6pt}\hspace*{6pt}{<\textbf{settlement}>}Ljubljana{</\textbf{settlement}>}\mbox{}\newline 
\hspace*{6pt}\hspace*{6pt}\hspace*{6pt}{<\textbf{repository}>}Narodna in univerzitetna knjiznica{</\textbf{repository}>}\mbox{}\newline 
\hspace*{6pt}\hspace*{6pt}\hspace*{6pt}{<\textbf{idno}>}MS Kopitar 2{</\textbf{idno}>}\mbox{}\newline 
\hspace*{6pt}\hspace*{6pt}{</\textbf{msIdentifier}>}\mbox{}\newline 
\hspace*{6pt}\hspace*{6pt}{<\textbf{msContents}>}\mbox{}\newline 
\hspace*{6pt}\hspace*{6pt}\hspace*{6pt}{<\textbf{summary}>}Contains ff. 10 to 42 only{</\textbf{summary}>}\mbox{}\newline 
\hspace*{6pt}\hspace*{6pt}{</\textbf{msContents}>}\mbox{}\newline 
\hspace*{6pt}{</\textbf{msFrag}>}\mbox{}\newline 
\hspace*{6pt}{<\textbf{msFrag}>}\mbox{}\newline 
\hspace*{6pt}\hspace*{6pt}{<\textbf{msIdentifier}>}\mbox{}\newline 
\hspace*{6pt}\hspace*{6pt}\hspace*{6pt}{<\textbf{settlement}>}Warszawa{</\textbf{settlement}>}\mbox{}\newline 
\hspace*{6pt}\hspace*{6pt}\hspace*{6pt}{<\textbf{repository}>}Biblioteka Narodowa{</\textbf{repository}>}\mbox{}\newline 
\hspace*{6pt}\hspace*{6pt}\hspace*{6pt}{<\textbf{idno}>}BO 3.201{</\textbf{idno}>}\mbox{}\newline 
\hspace*{6pt}\hspace*{6pt}{</\textbf{msIdentifier}>}\mbox{}\newline 
\hspace*{6pt}{</\textbf{msFrag}>}\mbox{}\newline 
\hspace*{6pt}{<\textbf{msFrag}>}\mbox{}\newline 
\hspace*{6pt}\hspace*{6pt}{<\textbf{msIdentifier}>}\mbox{}\newline 
\hspace*{6pt}\hspace*{6pt}\hspace*{6pt}{<\textbf{settlement}>}Sankt-Peterburg{</\textbf{settlement}>}\mbox{}\newline 
\hspace*{6pt}\hspace*{6pt}\hspace*{6pt}{<\textbf{repository}>}Rossiiskaia natsional'naia biblioteka{</\textbf{repository}>}\mbox{}\newline 
\hspace*{6pt}\hspace*{6pt}\hspace*{6pt}{<\textbf{idno}>}Q.p.I.72{</\textbf{idno}>}\mbox{}\newline 
\hspace*{6pt}\hspace*{6pt}{</\textbf{msIdentifier}>}\mbox{}\newline 
\hspace*{6pt}{</\textbf{msFrag}>}\mbox{}\newline 
{</\textbf{msDesc}>}\end{shaded}\egroup 


    \item[{Modèle de contenu}]
  \mbox{}\hfill\\[-10pt]\begin{Verbatim}[fontsize=\small]
<content>
 <sequence maxOccurs="1" minOccurs="1">
  <alternate maxOccurs="1" minOccurs="1">
   <elementRef key="altIdentifier"/>
   <elementRef key="msIdentifier"/>
  </alternate>
  <classRef key="model.headLike"
   maxOccurs="unbounded" minOccurs="0"/>
  <alternate maxOccurs="1" minOccurs="1">
   <classRef key="model.pLike"
    maxOccurs="unbounded" minOccurs="1"/>
   <sequence maxOccurs="1" minOccurs="1">
    <elementRef key="msContents"
     minOccurs="0"/>
    <elementRef key="physDesc"
     minOccurs="0"/>
    <elementRef key="history" minOccurs="0"/>
    <elementRef key="additional"
     minOccurs="0"/>
   </sequence>
  </alternate>
 </sequence>
</content>
    
\end{Verbatim}

    \item[{Schéma Declaration}]
  \mbox{}\hfill\\[-10pt]\begin{Verbatim}[fontsize=\small]
element msFrag
{
   tei_att.global.attributes,
   tei_att.typed.attributes,
   (
      ( tei_altIdentifier | tei_msIdentifier ),
      tei_model.headLike*,
      (
         tei_model.pLike+
       | ( tei_msContents?, tei_physDesc?, tei_history?, tei_additional? )
      )
   )
}
\end{Verbatim}

\end{reflist}  \index{msIdentifier=<msIdentifier>|oddindex}
\begin{reflist}
\item[]\begin{specHead}{TEI.msIdentifier}{<msIdentifier> }(identifiant du manuscrit) Contient les informations requises pour identifier le manuscrit en cours de description. [\xref{http://www.tei-c.org/release/doc/tei-p5-doc/en/html/MS.html\#msid}{10.4. The Manuscript Identifier}]\end{specHead} 
    \item[{Module}]
  msdescription
    \item[{Attributs}]
  Attributs \hyperref[TEI.att.global]{att.global} (\textit{@xml:id}, \textit{@n}, \textit{@xml:lang}, \textit{@xml:base}, \textit{@xml:space})  (\hyperref[TEI.att.global.rendition]{att.global.rendition} (\textit{@rend}, \textit{@style}, \textit{@rendition})) (\hyperref[TEI.att.global.linking]{att.global.linking} (\textit{@corresp}, \textit{@synch}, \textit{@sameAs}, \textit{@copyOf}, \textit{@next}, \textit{@prev}, \textit{@exclude}, \textit{@select})) (\hyperref[TEI.att.global.analytic]{att.global.analytic} (\textit{@ana})) (\hyperref[TEI.att.global.facs]{att.global.facs} (\textit{@facs})) (\hyperref[TEI.att.global.change]{att.global.change} (\textit{@change})) (\hyperref[TEI.att.global.responsibility]{att.global.responsibility} (\textit{@cert}, \textit{@resp})) (\hyperref[TEI.att.global.source]{att.global.source} (\textit{@source}))
    \item[{Membre du}]
  \hyperref[TEI.model.biblPart]{model.biblPart} 
    \item[{Contenu dans}]
  
    \item[core: ]
   \hyperref[TEI.bibl]{bibl}\par 
    \item[msdescription: ]
   \hyperref[TEI.msDesc]{msDesc} \hyperref[TEI.msFrag]{msFrag} \hyperref[TEI.msPart]{msPart}
    \item[{Peut contenir}]
  
    \item[header: ]
   \hyperref[TEI.idno]{idno}\par 
    \item[msdescription: ]
   \hyperref[TEI.altIdentifier]{altIdentifier} \hyperref[TEI.collection]{collection} \hyperref[TEI.institution]{institution} \hyperref[TEI.msName]{msName} \hyperref[TEI.repository]{repository}\par 
    \item[namesdates: ]
   \hyperref[TEI.country]{country} \hyperref[TEI.geogName]{geogName} \hyperref[TEI.placeName]{placeName} \hyperref[TEI.region]{region} \hyperref[TEI.settlement]{settlement}
    \item[{Exemple}]
  \leavevmode\bgroup\exampleFont \begin{shaded}\noindent\mbox{}{<\textbf{msIdentifier}>}\mbox{}\newline 
\hspace*{6pt}{<\textbf{country}>}France{</\textbf{country}>}\mbox{}\newline 
\hspace*{6pt}{<\textbf{settlement}>}Paris{</\textbf{settlement}>}\mbox{}\newline 
\hspace*{6pt}{<\textbf{repository}\hspace*{6pt}{xml:lang}="{fr}">}Bibliothèque nationale de France. Réserve des livres rares>{</\textbf{repository}>}\mbox{}\newline 
\hspace*{6pt}{<\textbf{idno}>}B- 73{</\textbf{idno}>}\mbox{}\newline 
\textit{<!-- dans le cas des recueils : cote uniquement sans les sous-cotes -->}\mbox{}\newline 
\hspace*{6pt}{<\textbf{altIdentifier}>}\mbox{}\newline 
\hspace*{6pt}\hspace*{6pt}{<\textbf{idno}>}B-121{</\textbf{idno}>}\mbox{}\newline 
\hspace*{6pt}\hspace*{6pt}{<\textbf{note}>} Cote de la bibliothèque royale au XVIIIe siècle (inscrite à l'encre, sur la\mbox{}\newline 
\hspace*{6pt}\hspace*{6pt}\hspace*{6pt}\hspace*{6pt} doublure de tabis).{</\textbf{note}>}\mbox{}\newline 
\hspace*{6pt}{</\textbf{altIdentifier}>}\mbox{}\newline 
\hspace*{6pt}{<\textbf{altIdentifier}>}\mbox{}\newline 
\hspace*{6pt}\hspace*{6pt}{<\textbf{idno}>}Double de B. 274. A (Réserve){</\textbf{idno}>}\mbox{}\newline 
\hspace*{6pt}\hspace*{6pt}{<\textbf{note}>}Cote inscrite face à la page de titre, en remplacement de la cote "1541",\mbox{}\newline 
\hspace*{6pt}\hspace*{6pt}\hspace*{6pt}\hspace*{6pt} barrée{</\textbf{note}>}\mbox{}\newline 
\hspace*{6pt}{</\textbf{altIdentifier}>}\mbox{}\newline 
{</\textbf{msIdentifier}>}\end{shaded}\egroup 


    \item[{Schematron}]
   <s:report test="not(parent::tei:msPart) and (local-name(*[1])='idno' or local-name(*[1])='altIdentifier'   or normalize-space(.)='')">An msIdentifier must contain either a repository or location of some type, or a manuscript name</s:report>
    \item[{Modèle de contenu}]
  \mbox{}\hfill\\[-10pt]\begin{Verbatim}[fontsize=\small]
<content>
 <sequence maxOccurs="1" minOccurs="1">
  <sequence maxOccurs="1" minOccurs="1">
   <classRef expand="sequenceOptional"
    key="model.placeNamePart"/>
   <elementRef key="institution"
    minOccurs="0"/>
   <elementRef key="repository"
    minOccurs="0"/>
   <elementRef key="collection"
    maxOccurs="unbounded" minOccurs="0"/>
   <elementRef key="idno" minOccurs="0"/>
  </sequence>
  <alternate maxOccurs="unbounded"
   minOccurs="0">
   <elementRef key="msName"/>
   <elementRef key="altIdentifier"/>
  </alternate>
 </sequence>
</content>
    
\end{Verbatim}

    \item[{Schéma Declaration}]
  \mbox{}\hfill\\[-10pt]\begin{Verbatim}[fontsize=\small]
element msIdentifier
{
   tei_att.global.attributes,
   (
      (
         tei_placeName?,
         tei_country?,
         tei_region?,
         tei_settlement?,
         tei_geogName?,
         tei_institution?,
         tei_repository?,
         tei_collection*,
         tei_idno?
      ),
      ( tei_msName | tei_altIdentifier )*
   )
}
\end{Verbatim}

\end{reflist}  \index{msItem=<msItem>|oddindex}
\begin{reflist}
\item[]\begin{specHead}{TEI.msItem}{<msItem> }(item de manuscrit) décrit une œuvre ou un item individualisés dans le contenu intellectuel d'un manuscrit ou d'une partie de manuscrit. [\xref{http://www.tei-c.org/release/doc/tei-p5-doc/en/html/MS.html\#mscoit}{10.6.1. The msItem and msItemStruct Elements}]\end{specHead} 
    \item[{Module}]
  msdescription
    \item[{Attributs}]
  Attributs \hyperref[TEI.att.global]{att.global} (\textit{@xml:id}, \textit{@n}, \textit{@xml:lang}, \textit{@xml:base}, \textit{@xml:space})  (\hyperref[TEI.att.global.rendition]{att.global.rendition} (\textit{@rend}, \textit{@style}, \textit{@rendition})) (\hyperref[TEI.att.global.linking]{att.global.linking} (\textit{@corresp}, \textit{@synch}, \textit{@sameAs}, \textit{@copyOf}, \textit{@next}, \textit{@prev}, \textit{@exclude}, \textit{@select})) (\hyperref[TEI.att.global.analytic]{att.global.analytic} (\textit{@ana})) (\hyperref[TEI.att.global.facs]{att.global.facs} (\textit{@facs})) (\hyperref[TEI.att.global.change]{att.global.change} (\textit{@change})) (\hyperref[TEI.att.global.responsibility]{att.global.responsibility} (\textit{@cert}, \textit{@resp})) (\hyperref[TEI.att.global.source]{att.global.source} (\textit{@source})) \hyperref[TEI.att.msExcerpt]{att.msExcerpt} (\textit{@defective}) \hyperref[TEI.att.msClass]{att.msClass} (\textit{@class}) 
    \item[{Membre du}]
  \hyperref[TEI.model.msItemPart]{model.msItemPart} 
    \item[{Contenu dans}]
  
    \item[msdescription: ]
   \hyperref[TEI.msContents]{msContents} \hyperref[TEI.msItem]{msItem}
    \item[{Peut contenir}]
  
    \item[analysis: ]
   \hyperref[TEI.interp]{interp} \hyperref[TEI.interpGrp]{interpGrp} \hyperref[TEI.span]{span} \hyperref[TEI.spanGrp]{spanGrp}\par 
    \item[core: ]
   \hyperref[TEI.author]{author} \hyperref[TEI.bibl]{bibl} \hyperref[TEI.biblStruct]{biblStruct} \hyperref[TEI.binaryObject]{binaryObject} \hyperref[TEI.cb]{cb} \hyperref[TEI.cit]{cit} \hyperref[TEI.editor]{editor} \hyperref[TEI.gap]{gap} \hyperref[TEI.gb]{gb} \hyperref[TEI.graphic]{graphic} \hyperref[TEI.index]{index} \hyperref[TEI.lb]{lb} \hyperref[TEI.listBibl]{listBibl} \hyperref[TEI.meeting]{meeting} \hyperref[TEI.milestone]{milestone} \hyperref[TEI.note]{note} \hyperref[TEI.p]{p} \hyperref[TEI.pb]{pb} \hyperref[TEI.quote]{quote} \hyperref[TEI.respStmt]{respStmt} \hyperref[TEI.textLang]{textLang} \hyperref[TEI.title]{title}\par 
    \item[figures: ]
   \hyperref[TEI.figure]{figure} \hyperref[TEI.notatedMusic]{notatedMusic}\par 
    \item[header: ]
   \hyperref[TEI.biblFull]{biblFull} \hyperref[TEI.funder]{funder} \hyperref[TEI.idno]{idno}\par 
    \item[iso-fs: ]
   \hyperref[TEI.fLib]{fLib} \hyperref[TEI.fs]{fs} \hyperref[TEI.fvLib]{fvLib}\par 
    \item[linking: ]
   \hyperref[TEI.ab]{ab} \hyperref[TEI.alt]{alt} \hyperref[TEI.altGrp]{altGrp} \hyperref[TEI.anchor]{anchor} \hyperref[TEI.join]{join} \hyperref[TEI.joinGrp]{joinGrp} \hyperref[TEI.link]{link} \hyperref[TEI.linkGrp]{linkGrp} \hyperref[TEI.timeline]{timeline}\par 
    \item[msdescription: ]
   \hyperref[TEI.colophon]{colophon} \hyperref[TEI.decoNote]{decoNote} \hyperref[TEI.explicit]{explicit} \hyperref[TEI.filiation]{filiation} \hyperref[TEI.finalRubric]{finalRubric} \hyperref[TEI.incipit]{incipit} \hyperref[TEI.locus]{locus} \hyperref[TEI.locusGrp]{locusGrp} \hyperref[TEI.msDesc]{msDesc} \hyperref[TEI.msItem]{msItem} \hyperref[TEI.msItemStruct]{msItemStruct} \hyperref[TEI.rubric]{rubric} \hyperref[TEI.source]{source}\par 
    \item[textstructure: ]
   \hyperref[TEI.docAuthor]{docAuthor} \hyperref[TEI.docDate]{docDate} \hyperref[TEI.docEdition]{docEdition} \hyperref[TEI.docTitle]{docTitle} \hyperref[TEI.titlePart]{titlePart}\par 
    \item[transcr: ]
   \hyperref[TEI.addSpan]{addSpan} \hyperref[TEI.damageSpan]{damageSpan} \hyperref[TEI.delSpan]{delSpan} \hyperref[TEI.fw]{fw} \hyperref[TEI.listTranspose]{listTranspose} \hyperref[TEI.metamark]{metamark} \hyperref[TEI.space]{space} \hyperref[TEI.substJoin]{substJoin}
    \item[{Exemple}]
  \leavevmode\bgroup\exampleFont \begin{shaded}\noindent\mbox{}{<\textbf{msItem}\hspace*{6pt}{class}="{\#saga}">}\mbox{}\newline 
\hspace*{6pt}{<\textbf{locus}>}ff. 1r-24v{</\textbf{locus}>}\mbox{}\newline 
\hspace*{6pt}{<\textbf{title}>}Agrip af Noregs konunga sögum{</\textbf{title}>}\mbox{}\newline 
\hspace*{6pt}{<\textbf{incipit}>}regi oc h{<\textbf{ex}>}ann{</\textbf{ex}>} setiho\mbox{}\newline 
\hspace*{6pt}{<\textbf{gap}\hspace*{6pt}{extent}="{7}"\hspace*{6pt}{reason}="{illegible}"/>}sc\mbox{}\newline 
\hspace*{6pt}\hspace*{6pt} heim se{<\textbf{ex}>}m{</\textbf{ex}>} þio{</\textbf{incipit}>}\mbox{}\newline 
\hspace*{6pt}{<\textbf{explicit}>}h{<\textbf{ex}>}on{</\textbf{ex}>} hev{<\textbf{ex}>}er{</\textbf{ex}>}\mbox{}\newline 
\hspace*{6pt}\hspace*{6pt}{<\textbf{ex}>}oc{</\textbf{ex}>}þa buit hesta .ij. aNan viþ\mbox{}\newline 
\hspace*{6pt}\hspace*{6pt} fé enh{<\textbf{ex}>}on{</\textbf{ex}>}o{<\textbf{ex}>}m{</\textbf{ex}>} aNan til\mbox{}\newline 
\hspace*{6pt}\hspace*{6pt} reiþ{<\textbf{ex}>}ar{</\textbf{ex}>}\mbox{}\newline 
\hspace*{6pt}{</\textbf{explicit}>}\mbox{}\newline 
\hspace*{6pt}{<\textbf{textLang}\hspace*{6pt}{mainLang}="{non}">}Old Norse/Icelandic{</\textbf{textLang}>}\mbox{}\newline 
{</\textbf{msItem}>}\end{shaded}\egroup 


    \item[{Exemple}]
  \leavevmode\bgroup\exampleFont \begin{shaded}\noindent\mbox{}{<\textbf{msContents}>}\mbox{}\newline 
\hspace*{6pt}{<\textbf{msItem}>}\mbox{}\newline 
\textit{<!-- pour le traitement des recueils la solution possible est de répéter l'élément <msItem>  -->}\mbox{}\newline 
\hspace*{6pt}\hspace*{6pt}{<\textbf{docAuthor}>}\mbox{}\newline 
\hspace*{6pt}\hspace*{6pt}\hspace*{6pt}{<\textbf{surname}>}Longus{</\textbf{surname}>}\mbox{}\newline 
\hspace*{6pt}\hspace*{6pt}{</\textbf{docAuthor}>}\mbox{}\newline 
\hspace*{6pt}\hspace*{6pt}{<\textbf{docTitle}>}\mbox{}\newline 
\hspace*{6pt}\hspace*{6pt}\hspace*{6pt}{<\textbf{titlePart}\hspace*{6pt}{type}="{main}">}Les amours pastorales de Daphnis et Chloé{</\textbf{titlePart}>}\mbox{}\newline 
\hspace*{6pt}\hspace*{6pt}{</\textbf{docTitle}>}\mbox{}\newline 
\hspace*{6pt}\hspace*{6pt}{<\textbf{docImprint}>}\mbox{}\newline 
\hspace*{6pt}\hspace*{6pt}\hspace*{6pt}{<\textbf{pubPlace}>}Paris{</\textbf{pubPlace}>}\mbox{}\newline 
\hspace*{6pt}\hspace*{6pt}\hspace*{6pt}{<\textbf{publisher}>}[Jacques Quillau]{</\textbf{publisher}>}\mbox{}\newline 
\hspace*{6pt}\hspace*{6pt}{</\textbf{docImprint}>}\mbox{}\newline 
\hspace*{6pt}\hspace*{6pt}{<\textbf{docDate}\hspace*{6pt}{when}="{1718}">}1718{</\textbf{docDate}>}\mbox{}\newline 
\hspace*{6pt}\hspace*{6pt}{<\textbf{note}>}in-8°.{</\textbf{note}>}\mbox{}\newline 
\textit{<!-- in-32°; in-24°; in-16°; in-8°; in-4°; in-folio; gr. folio -->}\mbox{}\newline 
\hspace*{6pt}\hspace*{6pt}{<\textbf{note}>}Exemplaire réglé.{</\textbf{note}>}\mbox{}\newline 
\hspace*{6pt}\hspace*{6pt}{<\textbf{note}>}\mbox{}\newline 
\hspace*{6pt}\hspace*{6pt}\hspace*{6pt}{<\textbf{ref}\hspace*{6pt}{target}="{http://catalogue.bnf.fr/ark:/12148/cb30831232s}">}Notice bibliographique\mbox{}\newline 
\hspace*{6pt}\hspace*{6pt}\hspace*{6pt}\hspace*{6pt}\hspace*{6pt}\hspace*{6pt} dans le Catalogue général{</\textbf{ref}>}\mbox{}\newline 
\hspace*{6pt}\hspace*{6pt}{</\textbf{note}>}\mbox{}\newline 
\hspace*{6pt}{</\textbf{msItem}>}\mbox{}\newline 
{</\textbf{msContents}>}\end{shaded}\egroup 


    \item[{Modèle de contenu}]
  \mbox{}\hfill\\[-10pt]\begin{Verbatim}[fontsize=\small]
<content>
 <sequence maxOccurs="1" minOccurs="1">
  <alternate maxOccurs="unbounded"
   minOccurs="0">
   <elementRef key="locus"/>
   <elementRef key="locusGrp"/>
  </alternate>
  <alternate maxOccurs="1" minOccurs="1">
   <classRef key="model.pLike"
    maxOccurs="unbounded" minOccurs="1"/>
   <alternate maxOccurs="unbounded"
    minOccurs="1">
    <classRef key="model.titlepagePart"/>
    <classRef key="model.msItemPart"/>
    <classRef key="model.global"/>
   </alternate>
  </alternate>
 </sequence>
</content>
    
\end{Verbatim}

    \item[{Schéma Declaration}]
  \mbox{}\hfill\\[-10pt]\begin{Verbatim}[fontsize=\small]
element msItem
{
   tei_att.global.attributes,
   tei_att.msExcerpt.attributes,
   tei_att.msClass.attributes,
   (
      ( tei_locus | tei_locusGrp )*,
      (
         tei_model.pLike+
       | ( tei_model.titlepagePart | tei_model.msItemPart | tei_model.global )+
      )
   )
}
\end{Verbatim}

\end{reflist}  \index{msItemStruct=<msItemStruct>|oddindex}
\begin{reflist}
\item[]\begin{specHead}{TEI.msItemStruct}{<msItemStruct> }(description structurée d'un item de manuscrit) contient la description structurée d'un item ou d'une œuvre, dans le contenu intellectuel d'un manuscrit ou d'une partie d'un manuscrit. [\xref{http://www.tei-c.org/release/doc/tei-p5-doc/en/html/MS.html\#mscoit}{10.6.1. The msItem and msItemStruct Elements}]\end{specHead} 
    \item[{Module}]
  msdescription
    \item[{Attributs}]
  Attributs \hyperref[TEI.att.global]{att.global} (\textit{@xml:id}, \textit{@n}, \textit{@xml:lang}, \textit{@xml:base}, \textit{@xml:space})  (\hyperref[TEI.att.global.rendition]{att.global.rendition} (\textit{@rend}, \textit{@style}, \textit{@rendition})) (\hyperref[TEI.att.global.linking]{att.global.linking} (\textit{@corresp}, \textit{@synch}, \textit{@sameAs}, \textit{@copyOf}, \textit{@next}, \textit{@prev}, \textit{@exclude}, \textit{@select})) (\hyperref[TEI.att.global.analytic]{att.global.analytic} (\textit{@ana})) (\hyperref[TEI.att.global.facs]{att.global.facs} (\textit{@facs})) (\hyperref[TEI.att.global.change]{att.global.change} (\textit{@change})) (\hyperref[TEI.att.global.responsibility]{att.global.responsibility} (\textit{@cert}, \textit{@resp})) (\hyperref[TEI.att.global.source]{att.global.source} (\textit{@source})) \hyperref[TEI.att.msExcerpt]{att.msExcerpt} (\textit{@defective}) \hyperref[TEI.att.msClass]{att.msClass} (\textit{@class}) 
    \item[{Membre du}]
  \hyperref[TEI.model.msItemPart]{model.msItemPart} 
    \item[{Contenu dans}]
  
    \item[msdescription: ]
   \hyperref[TEI.msContents]{msContents} \hyperref[TEI.msItem]{msItem} \hyperref[TEI.msItemStruct]{msItemStruct}
    \item[{Peut contenir}]
  
    \item[core: ]
   \hyperref[TEI.author]{author} \hyperref[TEI.bibl]{bibl} \hyperref[TEI.biblStruct]{biblStruct} \hyperref[TEI.listBibl]{listBibl} \hyperref[TEI.note]{note} \hyperref[TEI.p]{p} \hyperref[TEI.respStmt]{respStmt} \hyperref[TEI.textLang]{textLang} \hyperref[TEI.title]{title}\par 
    \item[linking: ]
   \hyperref[TEI.ab]{ab}\par 
    \item[msdescription: ]
   \hyperref[TEI.colophon]{colophon} \hyperref[TEI.decoNote]{decoNote} \hyperref[TEI.explicit]{explicit} \hyperref[TEI.filiation]{filiation} \hyperref[TEI.finalRubric]{finalRubric} \hyperref[TEI.incipit]{incipit} \hyperref[TEI.locus]{locus} \hyperref[TEI.locusGrp]{locusGrp} \hyperref[TEI.msItemStruct]{msItemStruct} \hyperref[TEI.rubric]{rubric}
    \item[{Exemple}]
  \leavevmode\bgroup\exampleFont \begin{shaded}\noindent\mbox{}{<\textbf{msItemStruct}\hspace*{6pt}{class}="{\#biblComm}"\mbox{}\newline 
\hspace*{6pt}{defective}="{false}"\hspace*{6pt}{n}="{2}">}\mbox{}\newline 
\hspace*{6pt}{<\textbf{locus}\hspace*{6pt}{from}="{24v}"\hspace*{6pt}{to}="{97v}">}24v-97v{</\textbf{locus}>}\mbox{}\newline 
\hspace*{6pt}{<\textbf{author}>}Apringius de Beja{</\textbf{author}>}\mbox{}\newline 
\hspace*{6pt}{<\textbf{title}\hspace*{6pt}{type}="{uniform}"\hspace*{6pt}{xml:lang}="{la}">}Tractatus in Apocalypsin{</\textbf{title}>}\mbox{}\newline 
\hspace*{6pt}{<\textbf{rubric}>}Incipit Trac{<\textbf{supplied}\hspace*{6pt}{reason}="{omitted}">}ta{</\textbf{supplied}>}tus\mbox{}\newline 
\hspace*{6pt}\hspace*{6pt} in apoka{<\textbf{lb}/>}lipsin eruditissimi uiri {<\textbf{lb}/>} Apringi ep{<\textbf{ex}>}iscop{</\textbf{ex}>}i\mbox{}\newline 
\hspace*{6pt}\hspace*{6pt} Pacensis eccl{<\textbf{ex}>}esi{</\textbf{ex}>}e{</\textbf{rubric}>}\mbox{}\newline 
\hspace*{6pt}{<\textbf{finalRubric}>}EXPLIC{<\textbf{ex}>}IT{</\textbf{ex}>} EXPO{<\textbf{lb}/>}SITIO APOCALIPSIS\mbox{}\newline 
\hspace*{6pt}\hspace*{6pt} QVA{<\textbf{ex}>}M{</\textbf{ex}>} EXPOSVIT DOM{<\textbf{lb}/>}NVS APRINGIUS EP{<\textbf{ex}>}ISCOPU{</\textbf{ex}>}S.\mbox{}\newline 
\hspace*{6pt}\hspace*{6pt} DEO GR{<\textbf{ex}>}ACI{</\textbf{ex}>}AS AGO. FI{<\textbf{lb}/>}NITO LABORE ISTO.{</\textbf{finalRubric}>}\mbox{}\newline 
\hspace*{6pt}{<\textbf{bibl}>}\mbox{}\newline 
\hspace*{6pt}\hspace*{6pt}{<\textbf{ref}\hspace*{6pt}{target}="{http://amiBibl.xml\#Apringius1900}">}Apringius{</\textbf{ref}>}, ed. Férotin{</\textbf{bibl}>}\mbox{}\newline 
\hspace*{6pt}{<\textbf{textLang}\hspace*{6pt}{mainLang}="{la}">}Latin{</\textbf{textLang}>}\mbox{}\newline 
{</\textbf{msItemStruct}>}\end{shaded}\egroup 


    \item[{Exemple}]
  \leavevmode\bgroup\exampleFont \begin{shaded}\noindent\mbox{}{<\textbf{msItemStruct}\hspace*{6pt}{class}="{\#biblComm}"\mbox{}\newline 
\hspace*{6pt}{defective}="{false}"\hspace*{6pt}{n}="{2}">}\mbox{}\newline 
\hspace*{6pt}{<\textbf{locus}\hspace*{6pt}{from}="{24v}"\hspace*{6pt}{to}="{97v}">}24v-97v{</\textbf{locus}>}\mbox{}\newline 
\hspace*{6pt}{<\textbf{author}>}Apringius de Beja{</\textbf{author}>}\mbox{}\newline 
\hspace*{6pt}{<\textbf{title}\hspace*{6pt}{type}="{uniform}"\hspace*{6pt}{xml:lang}="{la}">}Tractatus in Apocalypsin{</\textbf{title}>}\mbox{}\newline 
\hspace*{6pt}{<\textbf{rubric}>}Incipit Trac{<\textbf{supplied}\hspace*{6pt}{reason}="{omitted}">}ta{</\textbf{supplied}>}tus in apoka{<\textbf{lb}/>}lipsin\mbox{}\newline 
\hspace*{6pt}\hspace*{6pt} eruditissimi uiri {<\textbf{lb}/>} Apringi ep{<\textbf{ex}>}iscop{</\textbf{ex}>}i Pacensis eccl{<\textbf{ex}>}esi{</\textbf{ex}>}e{</\textbf{rubric}>}\mbox{}\newline 
\hspace*{6pt}{<\textbf{finalRubric}>}EXPLIC{<\textbf{ex}>}IT{</\textbf{ex}>} EXPO{<\textbf{lb}/>}SITIO APOCALIPSIS QVA{<\textbf{ex}>}M{</\textbf{ex}>}\mbox{}\newline 
\hspace*{6pt}\hspace*{6pt} EXPOSVIT DOM{<\textbf{lb}/>}NVS APRINGIUS EP{<\textbf{ex}>}ISCOPU{</\textbf{ex}>}S. DEO GR{<\textbf{ex}>}ACI{</\textbf{ex}>}AS AGO.\mbox{}\newline 
\hspace*{6pt}\hspace*{6pt} FI{<\textbf{lb}/>}NITO LABORE ISTO.{</\textbf{finalRubric}>}\mbox{}\newline 
\hspace*{6pt}{<\textbf{bibl}>}\mbox{}\newline 
\hspace*{6pt}\hspace*{6pt}{<\textbf{ref}\hspace*{6pt}{target}="{http://amiBibl.xml\#Apringius1900}">}Apringius{</\textbf{ref}>}, ed. Férotin{</\textbf{bibl}>}\mbox{}\newline 
\hspace*{6pt}{<\textbf{textLang}\hspace*{6pt}{mainLang}="{la}">}Latin{</\textbf{textLang}>}\mbox{}\newline 
{</\textbf{msItemStruct}>}\end{shaded}\egroup 


    \item[{Modèle de contenu}]
  \mbox{}\hfill\\[-10pt]\begin{Verbatim}[fontsize=\small]
<content>
 <sequence maxOccurs="1" minOccurs="1">
  <alternate maxOccurs="1" minOccurs="0">
   <elementRef key="locus"/>
   <elementRef key="locusGrp"/>
  </alternate>
  <alternate maxOccurs="1" minOccurs="1">
   <classRef key="model.pLike"
    maxOccurs="unbounded" minOccurs="1"/>
   <sequence maxOccurs="1" minOccurs="1">
    <elementRef key="author"
     maxOccurs="unbounded" minOccurs="0"/>
    <elementRef key="respStmt"
     maxOccurs="unbounded" minOccurs="0"/>
    <elementRef key="title"
     maxOccurs="unbounded" minOccurs="0"/>
    <elementRef key="rubric" minOccurs="0"/>
    <elementRef key="incipit" minOccurs="0"/>
    <elementRef key="msItemStruct"
     maxOccurs="unbounded" minOccurs="0"/>
    <elementRef key="explicit"
     minOccurs="0"/>
    <elementRef key="finalRubric"
     minOccurs="0"/>
    <elementRef key="colophon"
     maxOccurs="unbounded" minOccurs="0"/>
    <elementRef key="decoNote"
     maxOccurs="unbounded" minOccurs="0"/>
    <elementRef key="listBibl"
     maxOccurs="unbounded" minOccurs="0"/>
    <alternate maxOccurs="unbounded"
     minOccurs="0">
     <elementRef key="bibl"/>
     <elementRef key="biblStruct"/>
    </alternate>
    <elementRef key="filiation"
     minOccurs="0"/>
    <classRef key="model.noteLike"
     maxOccurs="unbounded" minOccurs="0"/>
    <elementRef key="textLang"
     minOccurs="0"/>
   </sequence>
  </alternate>
 </sequence>
</content>
    
\end{Verbatim}

    \item[{Schéma Declaration}]
  \mbox{}\hfill\\[-10pt]\begin{Verbatim}[fontsize=\small]
element msItemStruct
{
   tei_att.global.attributes,
   tei_att.msExcerpt.attributes,
   tei_att.msClass.attributes,
   (
      ( tei_locus | tei_locusGrp )?,
      (
         tei_model.pLike+
       | (
            tei_author*,
            tei_respStmt*,
            tei_title*,
            tei_rubric?,
            tei_incipit?,
            tei_msItemStruct*,
            tei_explicit?,
            tei_finalRubric?,
            tei_colophon*,
            tei_decoNote*,
            tei_listBibl*,
            ( tei_bibl | tei_biblStruct )*,
            tei_filiation?,
            tei_model.noteLike*,
            tei_textLang?
         )
      )
   )
}
\end{Verbatim}

\end{reflist}  \index{msName=<msName>|oddindex}
\begin{reflist}
\item[]\begin{specHead}{TEI.msName}{<msName> }(autre nom) contient un autre nom, dans une forme libre, utilisé pour désigner le manuscrit, tel qu'un surnom ou un ‘ocellus nominum’. [\xref{http://www.tei-c.org/release/doc/tei-p5-doc/en/html/MS.html\#msid}{10.4. The Manuscript Identifier}]\end{specHead} 
    \item[{Module}]
  msdescription
    \item[{Attributs}]
  Attributs \hyperref[TEI.att.global]{att.global} (\textit{@xml:id}, \textit{@n}, \textit{@xml:lang}, \textit{@xml:base}, \textit{@xml:space})  (\hyperref[TEI.att.global.rendition]{att.global.rendition} (\textit{@rend}, \textit{@style}, \textit{@rendition})) (\hyperref[TEI.att.global.linking]{att.global.linking} (\textit{@corresp}, \textit{@synch}, \textit{@sameAs}, \textit{@copyOf}, \textit{@next}, \textit{@prev}, \textit{@exclude}, \textit{@select})) (\hyperref[TEI.att.global.analytic]{att.global.analytic} (\textit{@ana})) (\hyperref[TEI.att.global.facs]{att.global.facs} (\textit{@facs})) (\hyperref[TEI.att.global.change]{att.global.change} (\textit{@change})) (\hyperref[TEI.att.global.responsibility]{att.global.responsibility} (\textit{@cert}, \textit{@resp})) (\hyperref[TEI.att.global.source]{att.global.source} (\textit{@source})) \hyperref[TEI.att.typed]{att.typed} (\textit{@type}, \textit{@subtype}) 
    \item[{Contenu dans}]
  
    \item[msdescription: ]
   \hyperref[TEI.msIdentifier]{msIdentifier}
    \item[{Peut contenir}]
  
    \item[core: ]
   \hyperref[TEI.name]{name} \hyperref[TEI.rs]{rs}\par des données textuelles
    \item[{Exemple}]
  \leavevmode\bgroup\exampleFont \begin{shaded}\noindent\mbox{}{<\textbf{msName}>}The Vercelli Book{</\textbf{msName}>}\mbox{}\newline 
\textit{<!--NOTE : LA TRADUCTION DE MSNAME EST A REPRENDRE-->}\end{shaded}\egroup 


    \item[{Modèle de contenu}]
  \mbox{}\hfill\\[-10pt]\begin{Verbatim}[fontsize=\small]
<content>
 <alternate maxOccurs="unbounded"
  minOccurs="0">
  <textNode/>
  <classRef key="model.gLike"/>
  <elementRef key="rs"/>
  <elementRef key="name"/>
 </alternate>
</content>
    
\end{Verbatim}

    \item[{Schéma Declaration}]
  \mbox{}\hfill\\[-10pt]\begin{Verbatim}[fontsize=\small]
element msName
{
   tei_att.global.attributes,
   tei_att.typed.attributes,
   ( text | tei_model.gLike | tei_rs | tei_name )*
}
\end{Verbatim}

\end{reflist}  \index{msPart=<msPart>|oddindex}
\begin{reflist}
\item[]\begin{specHead}{TEI.msPart}{<msPart> }(unité codicologique d'un manuscrit) contient des informations sur un manuscrit ou sur une partie d'un manuscrit, distinct à l'origine, qui fait aujourd'hui partie d'un manuscrit composite. [\xref{http://www.tei-c.org/release/doc/tei-p5-doc/en/html/MS.html\#mspt}{10.10. Manuscript Parts}]\end{specHead} 
    \item[{Module}]
  msdescription
    \item[{Attributs}]
  Attributs \hyperref[TEI.att.global]{att.global} (\textit{@xml:id}, \textit{@n}, \textit{@xml:lang}, \textit{@xml:base}, \textit{@xml:space})  (\hyperref[TEI.att.global.rendition]{att.global.rendition} (\textit{@rend}, \textit{@style}, \textit{@rendition})) (\hyperref[TEI.att.global.linking]{att.global.linking} (\textit{@corresp}, \textit{@synch}, \textit{@sameAs}, \textit{@copyOf}, \textit{@next}, \textit{@prev}, \textit{@exclude}, \textit{@select})) (\hyperref[TEI.att.global.analytic]{att.global.analytic} (\textit{@ana})) (\hyperref[TEI.att.global.facs]{att.global.facs} (\textit{@facs})) (\hyperref[TEI.att.global.change]{att.global.change} (\textit{@change})) (\hyperref[TEI.att.global.responsibility]{att.global.responsibility} (\textit{@cert}, \textit{@resp})) (\hyperref[TEI.att.global.source]{att.global.source} (\textit{@source})) \hyperref[TEI.att.typed]{att.typed} (\textit{@type}, \textit{@subtype}) 
    \item[{Contenu dans}]
  
    \item[msdescription: ]
   \hyperref[TEI.msDesc]{msDesc} \hyperref[TEI.msPart]{msPart}
    \item[{Peut contenir}]
  
    \item[core: ]
   \hyperref[TEI.head]{head} \hyperref[TEI.p]{p}\par 
    \item[linking: ]
   \hyperref[TEI.ab]{ab}\par 
    \item[msdescription: ]
   \hyperref[TEI.additional]{additional} \hyperref[TEI.history]{history} \hyperref[TEI.msContents]{msContents} \hyperref[TEI.msIdentifier]{msIdentifier} \hyperref[TEI.msPart]{msPart} \hyperref[TEI.physDesc]{physDesc}
    \item[{Note}]
  \par
As this last example shows, for compatibility reasons the identifier of a manuscript part may be supplied as a simple \hyperref[TEI.altIdentifier]{<altIdentifier>} rather than using the more structured \hyperref[TEI.msIdentifier]{<msIdentifier>} element. This usage is however deprecated.
    \item[{Exemple}]
  \leavevmode\bgroup\exampleFont \begin{shaded}\noindent\mbox{}{<\textbf{msPart}>}\mbox{}\newline 
\hspace*{6pt}{<\textbf{msIdentifier}>}\mbox{}\newline 
\hspace*{6pt}\hspace*{6pt}{<\textbf{idno}>}A{</\textbf{idno}>}\mbox{}\newline 
\hspace*{6pt}\hspace*{6pt}{<\textbf{altIdentifier}\hspace*{6pt}{type}="{catalog}">}\mbox{}\newline 
\hspace*{6pt}\hspace*{6pt}\hspace*{6pt}{<\textbf{collection}>}Becker{</\textbf{collection}>}\mbox{}\newline 
\hspace*{6pt}\hspace*{6pt}\hspace*{6pt}{<\textbf{idno}>}48, Nr. 145{</\textbf{idno}>}\mbox{}\newline 
\hspace*{6pt}\hspace*{6pt}{</\textbf{altIdentifier}>}\mbox{}\newline 
\hspace*{6pt}\hspace*{6pt}{<\textbf{altIdentifier}\hspace*{6pt}{type}="{catalog}">}\mbox{}\newline 
\hspace*{6pt}\hspace*{6pt}\hspace*{6pt}{<\textbf{collection}>}Wiener Liste{</\textbf{collection}>}\mbox{}\newline 
\hspace*{6pt}\hspace*{6pt}\hspace*{6pt}{<\textbf{idno}>}4°5{</\textbf{idno}>}\mbox{}\newline 
\hspace*{6pt}\hspace*{6pt}{</\textbf{altIdentifier}>}\mbox{}\newline 
\hspace*{6pt}{</\textbf{msIdentifier}>}\mbox{}\newline 
\hspace*{6pt}{<\textbf{head}>}\mbox{}\newline 
\hspace*{6pt}\hspace*{6pt}{<\textbf{title}\hspace*{6pt}{xml:lang}="{la}">}Gregorius: Homiliae in Ezechielem{</\textbf{title}>}\mbox{}\newline 
\hspace*{6pt}\hspace*{6pt}{<\textbf{origPlace}\hspace*{6pt}{key}="{tgn\textunderscore 7008085}">}Weissenburg (?){</\textbf{origPlace}>}\mbox{}\newline 
\hspace*{6pt}\hspace*{6pt}{<\textbf{origDate}\hspace*{6pt}{notAfter}="{0815}"\mbox{}\newline 
\hspace*{6pt}\hspace*{6pt}\hspace*{6pt}{notBefore}="{0801}">}IX. Jh., Anfang{</\textbf{origDate}>}\mbox{}\newline 
\hspace*{6pt}{</\textbf{head}>}\mbox{}\newline 
{</\textbf{msPart}>}\end{shaded}\egroup 


    \item[{Exemple}]
  \leavevmode\bgroup\exampleFont \begin{shaded}\noindent\mbox{}{<\textbf{msDesc}>}\mbox{}\newline 
\hspace*{6pt}{<\textbf{msIdentifier}>}\mbox{}\newline 
\hspace*{6pt}\hspace*{6pt}{<\textbf{settlement}>}Amiens{</\textbf{settlement}>}\mbox{}\newline 
\hspace*{6pt}\hspace*{6pt}{<\textbf{repository}>}Bibliothèque Municipale{</\textbf{repository}>}\mbox{}\newline 
\hspace*{6pt}\hspace*{6pt}{<\textbf{idno}>}MS 3{</\textbf{idno}>}\mbox{}\newline 
\hspace*{6pt}\hspace*{6pt}{<\textbf{msName}>}Maurdramnus Bible{</\textbf{msName}>}\mbox{}\newline 
\hspace*{6pt}{</\textbf{msIdentifier}>}\mbox{}\newline 
\hspace*{6pt}{<\textbf{msContents}>}\mbox{}\newline 
\hspace*{6pt}\hspace*{6pt}{<\textbf{summary}\hspace*{6pt}{xml:lang}="{lat}">}Miscellany of various texts; Prudentius, Psychomachia; Physiologus de natura animantium{</\textbf{summary}>}\mbox{}\newline 
\hspace*{6pt}\hspace*{6pt}{<\textbf{textLang}\hspace*{6pt}{mainLang}="{lat}">}Latin{</\textbf{textLang}>}\mbox{}\newline 
\hspace*{6pt}{</\textbf{msContents}>}\mbox{}\newline 
\hspace*{6pt}{<\textbf{physDesc}>}\mbox{}\newline 
\hspace*{6pt}\hspace*{6pt}{<\textbf{objectDesc}\hspace*{6pt}{form}="{composite\textunderscore manuscript}"/>}\mbox{}\newline 
\hspace*{6pt}{</\textbf{physDesc}>}\mbox{}\newline 
\hspace*{6pt}{<\textbf{msPart}>}\mbox{}\newline 
\hspace*{6pt}\hspace*{6pt}{<\textbf{msIdentifier}>}\mbox{}\newline 
\hspace*{6pt}\hspace*{6pt}\hspace*{6pt}{<\textbf{idno}>}ms. 10066-77 ff. 140r-156v{</\textbf{idno}>}\mbox{}\newline 
\hspace*{6pt}\hspace*{6pt}{</\textbf{msIdentifier}>}\mbox{}\newline 
\hspace*{6pt}\hspace*{6pt}{<\textbf{msContents}>}\mbox{}\newline 
\hspace*{6pt}\hspace*{6pt}\hspace*{6pt}{<\textbf{summary}\hspace*{6pt}{xml:lang}="{lat}">}Physiologus{</\textbf{summary}>}\mbox{}\newline 
\hspace*{6pt}\hspace*{6pt}\hspace*{6pt}{<\textbf{textLang}\hspace*{6pt}{mainLang}="{lat}">}Latin{</\textbf{textLang}>}\mbox{}\newline 
\hspace*{6pt}\hspace*{6pt}{</\textbf{msContents}>}\mbox{}\newline 
\hspace*{6pt}{</\textbf{msPart}>}\mbox{}\newline 
\hspace*{6pt}{<\textbf{msPart}>}\mbox{}\newline 
\hspace*{6pt}\hspace*{6pt}{<\textbf{msIdentifier}>}\mbox{}\newline 
\hspace*{6pt}\hspace*{6pt}\hspace*{6pt}{<\textbf{altIdentifier}>}\mbox{}\newline 
\hspace*{6pt}\hspace*{6pt}\hspace*{6pt}\hspace*{6pt}{<\textbf{idno}>}MS 6{</\textbf{idno}>}\mbox{}\newline 
\hspace*{6pt}\hspace*{6pt}\hspace*{6pt}{</\textbf{altIdentifier}>}\mbox{}\newline 
\hspace*{6pt}\hspace*{6pt}{</\textbf{msIdentifier}>}\mbox{}\newline 
\textit{<!-- other information specific to this part here -->}\mbox{}\newline 
\hspace*{6pt}{</\textbf{msPart}>}\mbox{}\newline 
\textit{<!-- more parts here -->}\mbox{}\newline 
{</\textbf{msDesc}>}\end{shaded}\egroup 


    \item[{Modèle de contenu}]
  \mbox{}\hfill\\[-10pt]\begin{Verbatim}[fontsize=\small]
<content>
 <sequence maxOccurs="1" minOccurs="1">
  <elementRef key="msIdentifier"/>
  <classRef key="model.headLike"
   maxOccurs="unbounded" minOccurs="0"/>
  <alternate maxOccurs="1" minOccurs="1">
   <classRef key="model.pLike"
    maxOccurs="unbounded" minOccurs="1"/>
   <sequence maxOccurs="1" minOccurs="1">
    <elementRef key="msContents"
     minOccurs="0"/>
    <elementRef key="physDesc"
     minOccurs="0"/>
    <elementRef key="history" minOccurs="0"/>
    <elementRef key="additional"
     minOccurs="0"/>
    <elementRef key="msPart"
     maxOccurs="unbounded" minOccurs="0"/>
   </sequence>
  </alternate>
 </sequence>
</content>
    
\end{Verbatim}

    \item[{Schéma Declaration}]
  \mbox{}\hfill\\[-10pt]\begin{Verbatim}[fontsize=\small]
element msPart
{
   tei_att.global.attributes,
   tei_att.typed.attributes,
   (
      tei_msIdentifier,
      tei_model.headLike*,
      (
         tei_model.pLike+
       | (
            tei_msContents?,
            tei_physDesc?,
            tei_history?,
            tei_additional?,
            tei_msPart*
         )
      )
   )
}
\end{Verbatim}

\end{reflist}  \index{mspace=<mspace>|oddindex}\index{width=@width!<mspace>|oddindex}\index{class=@class!<mspace>|oddindex}\index{linebreak=@linebreak!<mspace>|oddindex}
\begin{reflist}
\item[]\begin{specHead}{TEI.mspace}{<mspace> }\end{specHead} 
    \item[{Namespace}]
  http://www.w3.org/1998/Math/MathML
    \item[{Module}]
  derived-module-tei.istex
    \item[{Attributs}]
  Attributs\hfil\\[-10pt]\begin{sansreflist}
    \item[@width]
  
\begin{reflist}
    \item[{Statut}]
  Optionel
    \item[{Type de données}]
  \xref{https://www.w3.org/TR/xmlschema-2/\#}{}
\end{reflist}  
    \item[@class]
  
\begin{reflist}
    \item[{Statut}]
  Optionel
    \item[{Type de données}]
  \xref{https://www.w3.org/TR/xmlschema-2/\#}{}
\end{reflist}  
    \item[@linebreak]
  
\begin{reflist}
    \item[{Statut}]
  Optionel
    \item[{Type de données}]
  \xref{https://www.w3.org/TR/xmlschema-2/\#}{}
\end{reflist}  
\end{sansreflist}  
    \item[{Contenu dans}]
  
    \item[derived-module-tei.istex: ]
   \hyperref[TEI.math]{math} \hyperref[TEI.menclose]{menclose} \hyperref[TEI.mfenced]{mfenced} \hyperref[TEI.mfrac]{mfrac} \hyperref[TEI.mmultiscripts]{mmultiscripts} \hyperref[TEI.mover]{mover} \hyperref[TEI.mpadded]{mpadded} \hyperref[TEI.mphantom]{mphantom} \hyperref[TEI.mprescripts]{mprescripts} \hyperref[TEI.mrow]{mrow} \hyperref[TEI.msqrt]{msqrt} \hyperref[TEI.mstyle]{mstyle} \hyperref[TEI.msub]{msub} \hyperref[TEI.msubsup]{msubsup} \hyperref[TEI.msup]{msup} \hyperref[TEI.msupsub]{msupsub} \hyperref[TEI.mtable]{mtable} \hyperref[TEI.mtd]{mtd} \hyperref[TEI.mtr]{mtr} \hyperref[TEI.munder]{munder} \hyperref[TEI.munderover]{munderover} \hyperref[TEI.semantics]{semantics}
    \item[{Peut contenir}]
  Des données textuelles uniquement
    \item[{Modèle de contenu}]
  \fbox{\ttfamily <content>\newline
 <textNode/>\newline
</content>\newline
    } 
    \item[{Schéma Declaration}]
  \mbox{}\hfill\\[-10pt]\begin{Verbatim}[fontsize=\small]
element mspace
{
   attribute width { width }?,
   attribute class { class }?,
   attribute linebreak { linebreak }?,
   text
}
\end{Verbatim}

\end{reflist}  \index{msqrt=<msqrt>|oddindex}\index{open=@open!<msqrt>|oddindex}\index{close=@close!<msqrt>|oddindex}
\begin{reflist}
\item[]\begin{specHead}{TEI.msqrt}{<msqrt> }\end{specHead} 
    \item[{Namespace}]
  http://www.w3.org/1998/Math/MathML
    \item[{Module}]
  derived-module-tei.istex
    \item[{Attributs}]
  Attributs\hfil\\[-10pt]\begin{sansreflist}
    \item[@open]
  
\begin{reflist}
    \item[{Statut}]
  Optionel
    \item[{Type de données}]
  \xref{https://www.w3.org/TR/xmlschema-2/\#}{}
\end{reflist}  
    \item[@close]
  
\begin{reflist}
    \item[{Statut}]
  Optionel
    \item[{Type de données}]
  \xref{https://www.w3.org/TR/xmlschema-2/\#}{}
\end{reflist}  
\end{sansreflist}  
    \item[{Contenu dans}]
  
    \item[derived-module-tei.istex: ]
   \hyperref[TEI.math]{math} \hyperref[TEI.menclose]{menclose} \hyperref[TEI.mfenced]{mfenced} \hyperref[TEI.mfrac]{mfrac} \hyperref[TEI.mmultiscripts]{mmultiscripts} \hyperref[TEI.mover]{mover} \hyperref[TEI.mpadded]{mpadded} \hyperref[TEI.mphantom]{mphantom} \hyperref[TEI.mprescripts]{mprescripts} \hyperref[TEI.mrow]{mrow} \hyperref[TEI.msqrt]{msqrt} \hyperref[TEI.mstyle]{mstyle} \hyperref[TEI.msub]{msub} \hyperref[TEI.msubsup]{msubsup} \hyperref[TEI.msup]{msup} \hyperref[TEI.msupsub]{msupsub} \hyperref[TEI.mtable]{mtable} \hyperref[TEI.mtd]{mtd} \hyperref[TEI.mtr]{mtr} \hyperref[TEI.munder]{munder} \hyperref[TEI.munderover]{munderover} \hyperref[TEI.semantics]{semantics}
    \item[{Peut contenir}]
  
    \item[derived-module-tei.istex: ]
   \hyperref[TEI.menclose]{menclose} \hyperref[TEI.mfenced]{mfenced} \hyperref[TEI.mfrac]{mfrac} \hyperref[TEI.mi]{mi} \hyperref[TEI.mmultiscripts]{mmultiscripts} \hyperref[TEI.mn]{mn} \hyperref[TEI.mo]{mo} \hyperref[TEI.mover]{mover} \hyperref[TEI.mpadded]{mpadded} \hyperref[TEI.mphantom]{mphantom} \hyperref[TEI.mprescripts]{mprescripts} \hyperref[TEI.mrow]{mrow} \hyperref[TEI.mspace]{mspace} \hyperref[TEI.msqrt]{msqrt} \hyperref[TEI.mstyle]{mstyle} \hyperref[TEI.msub]{msub} \hyperref[TEI.msubsup]{msubsup} \hyperref[TEI.msup]{msup} \hyperref[TEI.msupsub]{msupsub} \hyperref[TEI.mtable]{mtable} \hyperref[TEI.mtd]{mtd} \hyperref[TEI.mtext]{mtext} \hyperref[TEI.mtr]{mtr} \hyperref[TEI.munder]{munder} \hyperref[TEI.munderover]{munderover} \hyperref[TEI.none]{none}\par des données textuelles
    \item[{Modèle de contenu}]
  \mbox{}\hfill\\[-10pt]\begin{Verbatim}[fontsize=\small]
<content>
 <alternate maxOccurs="unbounded"
  minOccurs="0">
  <textNode/>
  <elementRef key="mstyle"/>
  <elementRef key="mtr"/>
  <elementRef key="mtd"/>
  <elementRef key="mrow"/>
  <elementRef key="mi"/>
  <elementRef key="mn"/>
  <elementRef key="mtext"/>
  <elementRef key="mfrac"/>
  <elementRef key="mspace"/>
  <elementRef key="msqrt"/>
  <elementRef key="msub"/>
  <elementRef key="msup"/>
  <elementRef key="mo"/>
  <elementRef key="mover"/>
  <elementRef key="mfenced"/>
  <elementRef key="mtable"/>
  <elementRef key="msubsup"/>
  <elementRef key="msupsub"/>
  <elementRef key="mmultiscripts"/>
  <elementRef key="munderover"/>
  <elementRef key="mprescripts"/>
  <elementRef key="none"/>
  <elementRef key="munder"/>
  <elementRef key="mphantom"/>
  <elementRef key="mpadded"/>
  <elementRef key="menclose"/>
 </alternate>
</content>
    
\end{Verbatim}

    \item[{Schéma Declaration}]
  \mbox{}\hfill\\[-10pt]\begin{Verbatim}[fontsize=\small]
element msqrt
{
   attribute open { open }?,
   attribute close { close }?,
   (
      text
    | tei_mstyle    | tei_mtr    | tei_mtd    | tei_mrow    | tei_mi    | tei_mn    | tei_mtext    | tei_mfrac    | tei_mspace    | tei_msqrt    | tei_msub    | tei_msup    | tei_mo    | tei_mover    | tei_mfenced    | tei_mtable    | tei_msubsup    | tei_msupsub    | tei_mmultiscripts    | tei_munderover    | tei_mprescripts    | tei_none    | tei_munder    | tei_mphantom    | tei_mpadded    | tei_menclose   )*
}
\end{Verbatim}

\end{reflist}  \index{mstyle=<mstyle>|oddindex}\index{scriptlevel=@scriptlevel!<mstyle>|oddindex}\index{mathvariant=@mathvariant!<mstyle>|oddindex}\index{displaystyle=@displaystyle!<mstyle>|oddindex}\index{mathsize=@mathsize!<mstyle>|oddindex}\index{fontsize=@fontsize!<mstyle>|oddindex}
\begin{reflist}
\item[]\begin{specHead}{TEI.mstyle}{<mstyle> }\end{specHead} 
    \item[{Namespace}]
  http://www.w3.org/1998/Math/MathML
    \item[{Module}]
  derived-module-tei.istex
    \item[{Attributs}]
  Attributs\hfil\\[-10pt]\begin{sansreflist}
    \item[@scriptlevel]
  
\begin{reflist}
    \item[{Statut}]
  Optionel
    \item[{Type de données}]
  \xref{https://www.w3.org/TR/xmlschema-2/\#}{}
\end{reflist}  
    \item[@mathvariant]
  
\begin{reflist}
    \item[{Statut}]
  Optionel
    \item[{Type de données}]
  \xref{https://www.w3.org/TR/xmlschema-2/\#}{}
\end{reflist}  
    \item[@displaystyle]
  
\begin{reflist}
    \item[{Statut}]
  Optionel
    \item[{Type de données}]
  \xref{https://www.w3.org/TR/xmlschema-2/\#}{}
\end{reflist}  
    \item[@mathsize]
  
\begin{reflist}
    \item[{Statut}]
  Optionel
    \item[{Type de données}]
  \xref{https://www.w3.org/TR/xmlschema-2/\#}{}
\end{reflist}  
    \item[@fontsize]
  
\begin{reflist}
    \item[{Statut}]
  Optionel
    \item[{Type de données}]
  \xref{https://www.w3.org/TR/xmlschema-2/\#}{}
\end{reflist}  
\end{sansreflist}  
    \item[{Contenu dans}]
  
    \item[derived-module-tei.istex: ]
   \hyperref[TEI.math]{math} \hyperref[TEI.menclose]{menclose} \hyperref[TEI.mfenced]{mfenced} \hyperref[TEI.mfrac]{mfrac} \hyperref[TEI.mmultiscripts]{mmultiscripts} \hyperref[TEI.mover]{mover} \hyperref[TEI.mpadded]{mpadded} \hyperref[TEI.mphantom]{mphantom} \hyperref[TEI.mprescripts]{mprescripts} \hyperref[TEI.mrow]{mrow} \hyperref[TEI.msqrt]{msqrt} \hyperref[TEI.mstyle]{mstyle} \hyperref[TEI.msub]{msub} \hyperref[TEI.msubsup]{msubsup} \hyperref[TEI.msup]{msup} \hyperref[TEI.msupsub]{msupsub} \hyperref[TEI.mtable]{mtable} \hyperref[TEI.mtd]{mtd} \hyperref[TEI.mtr]{mtr} \hyperref[TEI.munder]{munder} \hyperref[TEI.munderover]{munderover} \hyperref[TEI.semantics]{semantics}
    \item[{Peut contenir}]
  
    \item[derived-module-tei.istex: ]
   \hyperref[TEI.menclose]{menclose} \hyperref[TEI.mfenced]{mfenced} \hyperref[TEI.mfrac]{mfrac} \hyperref[TEI.mi]{mi} \hyperref[TEI.mmultiscripts]{mmultiscripts} \hyperref[TEI.mn]{mn} \hyperref[TEI.mo]{mo} \hyperref[TEI.mover]{mover} \hyperref[TEI.mpadded]{mpadded} \hyperref[TEI.mphantom]{mphantom} \hyperref[TEI.mprescripts]{mprescripts} \hyperref[TEI.mrow]{mrow} \hyperref[TEI.mspace]{mspace} \hyperref[TEI.msqrt]{msqrt} \hyperref[TEI.mstyle]{mstyle} \hyperref[TEI.msub]{msub} \hyperref[TEI.msubsup]{msubsup} \hyperref[TEI.msup]{msup} \hyperref[TEI.msupsub]{msupsub} \hyperref[TEI.mtable]{mtable} \hyperref[TEI.mtd]{mtd} \hyperref[TEI.mtext]{mtext} \hyperref[TEI.mtr]{mtr} \hyperref[TEI.munder]{munder} \hyperref[TEI.munderover]{munderover} \hyperref[TEI.none]{none}\par des données textuelles
    \item[{Modèle de contenu}]
  \mbox{}\hfill\\[-10pt]\begin{Verbatim}[fontsize=\small]
<content>
 <alternate maxOccurs="unbounded"
  minOccurs="0">
  <textNode/>
  <elementRef key="mstyle"/>
  <elementRef key="mtr"/>
  <elementRef key="mtd"/>
  <elementRef key="mrow"/>
  <elementRef key="mi"/>
  <elementRef key="mn"/>
  <elementRef key="mtext"/>
  <elementRef key="mfrac"/>
  <elementRef key="mspace"/>
  <elementRef key="msqrt"/>
  <elementRef key="msub"/>
  <elementRef key="msup"/>
  <elementRef key="mo"/>
  <elementRef key="mover"/>
  <elementRef key="mfenced"/>
  <elementRef key="mtable"/>
  <elementRef key="msubsup"/>
  <elementRef key="msupsub"/>
  <elementRef key="mmultiscripts"/>
  <elementRef key="munderover"/>
  <elementRef key="mprescripts"/>
  <elementRef key="none"/>
  <elementRef key="munder"/>
  <elementRef key="mphantom"/>
  <elementRef key="mpadded"/>
  <elementRef key="menclose"/>
 </alternate>
</content>
    
\end{Verbatim}

    \item[{Schéma Declaration}]
  \mbox{}\hfill\\[-10pt]\begin{Verbatim}[fontsize=\small]
element mstyle
{
   attribute scriptlevel { scriptlevel }?,
   attribute mathvariant { mathvariant }?,
   attribute displaystyle { displaystyle }?,
   attribute mathsize { mathsize }?,
   attribute fontsize { fontsize }?,
   (
      text
    | tei_mstyle    | tei_mtr    | tei_mtd    | tei_mrow    | tei_mi    | tei_mn    | tei_mtext    | tei_mfrac    | tei_mspace    | tei_msqrt    | tei_msub    | tei_msup    | tei_mo    | tei_mover    | tei_mfenced    | tei_mtable    | tei_msubsup    | tei_msupsub    | tei_mmultiscripts    | tei_munderover    | tei_mprescripts    | tei_none    | tei_munder    | tei_mphantom    | tei_mpadded    | tei_menclose   )*
}
\end{Verbatim}

\end{reflist}  \index{msub=<msub>|oddindex}
\begin{reflist}
\item[]\begin{specHead}{TEI.msub}{<msub> }\end{specHead} 
    \item[{Namespace}]
  http://www.w3.org/1998/Math/MathML
    \item[{Module}]
  derived-module-tei.istex
    \item[{Contenu dans}]
  
    \item[derived-module-tei.istex: ]
   \hyperref[TEI.math]{math} \hyperref[TEI.menclose]{menclose} \hyperref[TEI.mfenced]{mfenced} \hyperref[TEI.mfrac]{mfrac} \hyperref[TEI.mmultiscripts]{mmultiscripts} \hyperref[TEI.mover]{mover} \hyperref[TEI.mpadded]{mpadded} \hyperref[TEI.mphantom]{mphantom} \hyperref[TEI.mprescripts]{mprescripts} \hyperref[TEI.mrow]{mrow} \hyperref[TEI.msqrt]{msqrt} \hyperref[TEI.mstyle]{mstyle} \hyperref[TEI.msub]{msub} \hyperref[TEI.msubsup]{msubsup} \hyperref[TEI.msup]{msup} \hyperref[TEI.msupsub]{msupsub} \hyperref[TEI.mtable]{mtable} \hyperref[TEI.mtd]{mtd} \hyperref[TEI.mtr]{mtr} \hyperref[TEI.munder]{munder} \hyperref[TEI.munderover]{munderover} \hyperref[TEI.semantics]{semantics}
    \item[{Peut contenir}]
  
    \item[derived-module-tei.istex: ]
   \hyperref[TEI.menclose]{menclose} \hyperref[TEI.mfenced]{mfenced} \hyperref[TEI.mfrac]{mfrac} \hyperref[TEI.mi]{mi} \hyperref[TEI.mmultiscripts]{mmultiscripts} \hyperref[TEI.mn]{mn} \hyperref[TEI.mo]{mo} \hyperref[TEI.mover]{mover} \hyperref[TEI.mpadded]{mpadded} \hyperref[TEI.mphantom]{mphantom} \hyperref[TEI.mprescripts]{mprescripts} \hyperref[TEI.mrow]{mrow} \hyperref[TEI.mspace]{mspace} \hyperref[TEI.msqrt]{msqrt} \hyperref[TEI.mstyle]{mstyle} \hyperref[TEI.msub]{msub} \hyperref[TEI.msubsup]{msubsup} \hyperref[TEI.msup]{msup} \hyperref[TEI.msupsub]{msupsub} \hyperref[TEI.mtable]{mtable} \hyperref[TEI.mtd]{mtd} \hyperref[TEI.mtext]{mtext} \hyperref[TEI.mtr]{mtr} \hyperref[TEI.munder]{munder} \hyperref[TEI.munderover]{munderover} \hyperref[TEI.none]{none}\par des données textuelles
    \item[{Modèle de contenu}]
  \mbox{}\hfill\\[-10pt]\begin{Verbatim}[fontsize=\small]
<content>
 <alternate maxOccurs="unbounded"
  minOccurs="0">
  <textNode/>
  <elementRef key="mstyle"/>
  <elementRef key="mtr"/>
  <elementRef key="mtd"/>
  <elementRef key="mrow"/>
  <elementRef key="mi"/>
  <elementRef key="mn"/>
  <elementRef key="mtext"/>
  <elementRef key="mfrac"/>
  <elementRef key="mspace"/>
  <elementRef key="msqrt"/>
  <elementRef key="msub"/>
  <elementRef key="msup"/>
  <elementRef key="mo"/>
  <elementRef key="mover"/>
  <elementRef key="mfenced"/>
  <elementRef key="mtable"/>
  <elementRef key="msubsup"/>
  <elementRef key="msupsub"/>
  <elementRef key="mmultiscripts"/>
  <elementRef key="munderover"/>
  <elementRef key="mprescripts"/>
  <elementRef key="none"/>
  <elementRef key="munder"/>
  <elementRef key="mphantom"/>
  <elementRef key="mpadded"/>
  <elementRef key="menclose"/>
 </alternate>
</content>
    
\end{Verbatim}

    \item[{Schéma Declaration}]
  \mbox{}\hfill\\[-10pt]\begin{Verbatim}[fontsize=\small]
element msub
{
   (
      text
    | tei_mstyle    | tei_mtr    | tei_mtd    | tei_mrow    | tei_mi    | tei_mn    | tei_mtext    | tei_mfrac    | tei_mspace    | tei_msqrt    | tei_msub    | tei_msup    | tei_mo    | tei_mover    | tei_mfenced    | tei_mtable    | tei_msubsup    | tei_msupsub    | tei_mmultiscripts    | tei_munderover    | tei_mprescripts    | tei_none    | tei_munder    | tei_mphantom    | tei_mpadded    | tei_menclose   )*
}
\end{Verbatim}

\end{reflist}  \index{msubsup=<msubsup>|oddindex}
\begin{reflist}
\item[]\begin{specHead}{TEI.msubsup}{<msubsup> }\end{specHead} 
    \item[{Namespace}]
  http://www.w3.org/1998/Math/MathML
    \item[{Module}]
  derived-module-tei.istex
    \item[{Contenu dans}]
  
    \item[derived-module-tei.istex: ]
   \hyperref[TEI.math]{math} \hyperref[TEI.menclose]{menclose} \hyperref[TEI.mfenced]{mfenced} \hyperref[TEI.mfrac]{mfrac} \hyperref[TEI.mmultiscripts]{mmultiscripts} \hyperref[TEI.mover]{mover} \hyperref[TEI.mpadded]{mpadded} \hyperref[TEI.mphantom]{mphantom} \hyperref[TEI.mprescripts]{mprescripts} \hyperref[TEI.mrow]{mrow} \hyperref[TEI.msqrt]{msqrt} \hyperref[TEI.mstyle]{mstyle} \hyperref[TEI.msub]{msub} \hyperref[TEI.msubsup]{msubsup} \hyperref[TEI.msup]{msup} \hyperref[TEI.msupsub]{msupsub} \hyperref[TEI.mtable]{mtable} \hyperref[TEI.mtd]{mtd} \hyperref[TEI.mtr]{mtr} \hyperref[TEI.munder]{munder} \hyperref[TEI.munderover]{munderover} \hyperref[TEI.semantics]{semantics}
    \item[{Peut contenir}]
  
    \item[derived-module-tei.istex: ]
   \hyperref[TEI.menclose]{menclose} \hyperref[TEI.mfenced]{mfenced} \hyperref[TEI.mfrac]{mfrac} \hyperref[TEI.mi]{mi} \hyperref[TEI.mmultiscripts]{mmultiscripts} \hyperref[TEI.mn]{mn} \hyperref[TEI.mo]{mo} \hyperref[TEI.mover]{mover} \hyperref[TEI.mpadded]{mpadded} \hyperref[TEI.mphantom]{mphantom} \hyperref[TEI.mprescripts]{mprescripts} \hyperref[TEI.mrow]{mrow} \hyperref[TEI.mspace]{mspace} \hyperref[TEI.msqrt]{msqrt} \hyperref[TEI.mstyle]{mstyle} \hyperref[TEI.msub]{msub} \hyperref[TEI.msubsup]{msubsup} \hyperref[TEI.msup]{msup} \hyperref[TEI.msupsub]{msupsub} \hyperref[TEI.mtable]{mtable} \hyperref[TEI.mtd]{mtd} \hyperref[TEI.mtext]{mtext} \hyperref[TEI.mtr]{mtr} \hyperref[TEI.munder]{munder} \hyperref[TEI.munderover]{munderover} \hyperref[TEI.none]{none}\par des données textuelles
    \item[{Modèle de contenu}]
  \mbox{}\hfill\\[-10pt]\begin{Verbatim}[fontsize=\small]
<content>
 <alternate maxOccurs="unbounded"
  minOccurs="0">
  <textNode/>
  <elementRef key="mstyle"/>
  <elementRef key="mtr"/>
  <elementRef key="mtd"/>
  <elementRef key="mrow"/>
  <elementRef key="mi"/>
  <elementRef key="mn"/>
  <elementRef key="mtext"/>
  <elementRef key="mfrac"/>
  <elementRef key="mspace"/>
  <elementRef key="msqrt"/>
  <elementRef key="msub"/>
  <elementRef key="msup"/>
  <elementRef key="mo"/>
  <elementRef key="mover"/>
  <elementRef key="mfenced"/>
  <elementRef key="mtable"/>
  <elementRef key="msubsup"/>
  <elementRef key="msupsub"/>
  <elementRef key="mmultiscripts"/>
  <elementRef key="munderover"/>
  <elementRef key="mprescripts"/>
  <elementRef key="none"/>
  <elementRef key="munder"/>
  <elementRef key="mphantom"/>
  <elementRef key="mpadded"/>
  <elementRef key="menclose"/>
 </alternate>
</content>
    
\end{Verbatim}

    \item[{Schéma Declaration}]
  \mbox{}\hfill\\[-10pt]\begin{Verbatim}[fontsize=\small]
element msubsup
{
   (
      text
    | tei_mstyle    | tei_mtr    | tei_mtd    | tei_mrow    | tei_mi    | tei_mn    | tei_mtext    | tei_mfrac    | tei_mspace    | tei_msqrt    | tei_msub    | tei_msup    | tei_mo    | tei_mover    | tei_mfenced    | tei_mtable    | tei_msubsup    | tei_msupsub    | tei_mmultiscripts    | tei_munderover    | tei_mprescripts    | tei_none    | tei_munder    | tei_mphantom    | tei_mpadded    | tei_menclose   )*
}
\end{Verbatim}

\end{reflist}  \index{msup=<msup>|oddindex}
\begin{reflist}
\item[]\begin{specHead}{TEI.msup}{<msup> }\end{specHead} 
    \item[{Namespace}]
  http://www.w3.org/1998/Math/MathML
    \item[{Module}]
  derived-module-tei.istex
    \item[{Contenu dans}]
  
    \item[derived-module-tei.istex: ]
   \hyperref[TEI.math]{math} \hyperref[TEI.menclose]{menclose} \hyperref[TEI.mfenced]{mfenced} \hyperref[TEI.mfrac]{mfrac} \hyperref[TEI.mmultiscripts]{mmultiscripts} \hyperref[TEI.mover]{mover} \hyperref[TEI.mpadded]{mpadded} \hyperref[TEI.mphantom]{mphantom} \hyperref[TEI.mprescripts]{mprescripts} \hyperref[TEI.mrow]{mrow} \hyperref[TEI.msqrt]{msqrt} \hyperref[TEI.mstyle]{mstyle} \hyperref[TEI.msub]{msub} \hyperref[TEI.msubsup]{msubsup} \hyperref[TEI.msup]{msup} \hyperref[TEI.msupsub]{msupsub} \hyperref[TEI.mtable]{mtable} \hyperref[TEI.mtd]{mtd} \hyperref[TEI.mtr]{mtr} \hyperref[TEI.munder]{munder} \hyperref[TEI.munderover]{munderover} \hyperref[TEI.semantics]{semantics}
    \item[{Peut contenir}]
  
    \item[derived-module-tei.istex: ]
   \hyperref[TEI.menclose]{menclose} \hyperref[TEI.mfenced]{mfenced} \hyperref[TEI.mfrac]{mfrac} \hyperref[TEI.mi]{mi} \hyperref[TEI.mmultiscripts]{mmultiscripts} \hyperref[TEI.mn]{mn} \hyperref[TEI.mo]{mo} \hyperref[TEI.mover]{mover} \hyperref[TEI.mpadded]{mpadded} \hyperref[TEI.mphantom]{mphantom} \hyperref[TEI.mprescripts]{mprescripts} \hyperref[TEI.mrow]{mrow} \hyperref[TEI.mspace]{mspace} \hyperref[TEI.msqrt]{msqrt} \hyperref[TEI.mstyle]{mstyle} \hyperref[TEI.msub]{msub} \hyperref[TEI.msubsup]{msubsup} \hyperref[TEI.msup]{msup} \hyperref[TEI.msupsub]{msupsub} \hyperref[TEI.mtable]{mtable} \hyperref[TEI.mtd]{mtd} \hyperref[TEI.mtext]{mtext} \hyperref[TEI.mtr]{mtr} \hyperref[TEI.munder]{munder} \hyperref[TEI.munderover]{munderover} \hyperref[TEI.none]{none}\par des données textuelles
    \item[{Modèle de contenu}]
  \mbox{}\hfill\\[-10pt]\begin{Verbatim}[fontsize=\small]
<content>
 <alternate maxOccurs="unbounded"
  minOccurs="0">
  <textNode/>
  <elementRef key="mstyle"/>
  <elementRef key="mtr"/>
  <elementRef key="mtd"/>
  <elementRef key="mrow"/>
  <elementRef key="mi"/>
  <elementRef key="mn"/>
  <elementRef key="mtext"/>
  <elementRef key="mfrac"/>
  <elementRef key="mspace"/>
  <elementRef key="msqrt"/>
  <elementRef key="msub"/>
  <elementRef key="msup"/>
  <elementRef key="mo"/>
  <elementRef key="mover"/>
  <elementRef key="mfenced"/>
  <elementRef key="mtable"/>
  <elementRef key="msubsup"/>
  <elementRef key="msupsub"/>
  <elementRef key="mmultiscripts"/>
  <elementRef key="munderover"/>
  <elementRef key="mprescripts"/>
  <elementRef key="none"/>
  <elementRef key="munder"/>
  <elementRef key="mphantom"/>
  <elementRef key="mpadded"/>
  <elementRef key="menclose"/>
 </alternate>
</content>
    
\end{Verbatim}

    \item[{Schéma Declaration}]
  \mbox{}\hfill\\[-10pt]\begin{Verbatim}[fontsize=\small]
element msup
{
   (
      text
    | tei_mstyle    | tei_mtr    | tei_mtd    | tei_mrow    | tei_mi    | tei_mn    | tei_mtext    | tei_mfrac    | tei_mspace    | tei_msqrt    | tei_msub    | tei_msup    | tei_mo    | tei_mover    | tei_mfenced    | tei_mtable    | tei_msubsup    | tei_msupsub    | tei_mmultiscripts    | tei_munderover    | tei_mprescripts    | tei_none    | tei_munder    | tei_mphantom    | tei_mpadded    | tei_menclose   )*
}
\end{Verbatim}

\end{reflist}  \index{msupsub=<msupsub>|oddindex}
\begin{reflist}
\item[]\begin{specHead}{TEI.msupsub}{<msupsub> }\end{specHead} 
    \item[{Namespace}]
  http://www.w3.org/1998/Math/MathML
    \item[{Module}]
  derived-module-tei.istex
    \item[{Contenu dans}]
  
    \item[derived-module-tei.istex: ]
   \hyperref[TEI.math]{math} \hyperref[TEI.menclose]{menclose} \hyperref[TEI.mfenced]{mfenced} \hyperref[TEI.mfrac]{mfrac} \hyperref[TEI.mmultiscripts]{mmultiscripts} \hyperref[TEI.mover]{mover} \hyperref[TEI.mpadded]{mpadded} \hyperref[TEI.mphantom]{mphantom} \hyperref[TEI.mprescripts]{mprescripts} \hyperref[TEI.mrow]{mrow} \hyperref[TEI.msqrt]{msqrt} \hyperref[TEI.mstyle]{mstyle} \hyperref[TEI.msub]{msub} \hyperref[TEI.msubsup]{msubsup} \hyperref[TEI.msup]{msup} \hyperref[TEI.msupsub]{msupsub} \hyperref[TEI.mtable]{mtable} \hyperref[TEI.mtd]{mtd} \hyperref[TEI.mtr]{mtr} \hyperref[TEI.munder]{munder} \hyperref[TEI.munderover]{munderover} \hyperref[TEI.semantics]{semantics}
    \item[{Peut contenir}]
  
    \item[derived-module-tei.istex: ]
   \hyperref[TEI.menclose]{menclose} \hyperref[TEI.mfenced]{mfenced} \hyperref[TEI.mfrac]{mfrac} \hyperref[TEI.mi]{mi} \hyperref[TEI.mmultiscripts]{mmultiscripts} \hyperref[TEI.mn]{mn} \hyperref[TEI.mo]{mo} \hyperref[TEI.mover]{mover} \hyperref[TEI.mpadded]{mpadded} \hyperref[TEI.mphantom]{mphantom} \hyperref[TEI.mprescripts]{mprescripts} \hyperref[TEI.mrow]{mrow} \hyperref[TEI.mspace]{mspace} \hyperref[TEI.msqrt]{msqrt} \hyperref[TEI.mstyle]{mstyle} \hyperref[TEI.msub]{msub} \hyperref[TEI.msubsup]{msubsup} \hyperref[TEI.msup]{msup} \hyperref[TEI.msupsub]{msupsub} \hyperref[TEI.mtable]{mtable} \hyperref[TEI.mtd]{mtd} \hyperref[TEI.mtext]{mtext} \hyperref[TEI.mtr]{mtr} \hyperref[TEI.munder]{munder} \hyperref[TEI.munderover]{munderover} \hyperref[TEI.none]{none}\par des données textuelles
    \item[{Modèle de contenu}]
  \mbox{}\hfill\\[-10pt]\begin{Verbatim}[fontsize=\small]
<content>
 <alternate maxOccurs="unbounded"
  minOccurs="0">
  <textNode/>
  <elementRef key="mstyle"/>
  <elementRef key="mtr"/>
  <elementRef key="mtd"/>
  <elementRef key="mrow"/>
  <elementRef key="mi"/>
  <elementRef key="mn"/>
  <elementRef key="mtext"/>
  <elementRef key="mfrac"/>
  <elementRef key="mspace"/>
  <elementRef key="msqrt"/>
  <elementRef key="msub"/>
  <elementRef key="msup"/>
  <elementRef key="mo"/>
  <elementRef key="mover"/>
  <elementRef key="mfenced"/>
  <elementRef key="mtable"/>
  <elementRef key="msubsup"/>
  <elementRef key="msupsub"/>
  <elementRef key="mmultiscripts"/>
  <elementRef key="munderover"/>
  <elementRef key="mprescripts"/>
  <elementRef key="none"/>
  <elementRef key="munder"/>
  <elementRef key="mphantom"/>
  <elementRef key="mpadded"/>
  <elementRef key="menclose"/>
 </alternate>
</content>
    
\end{Verbatim}

    \item[{Schéma Declaration}]
  \mbox{}\hfill\\[-10pt]\begin{Verbatim}[fontsize=\small]
element msupsub
{
   (
      text
    | tei_mstyle    | tei_mtr    | tei_mtd    | tei_mrow    | tei_mi    | tei_mn    | tei_mtext    | tei_mfrac    | tei_mspace    | tei_msqrt    | tei_msub    | tei_msup    | tei_mo    | tei_mover    | tei_mfenced    | tei_mtable    | tei_msubsup    | tei_msupsub    | tei_mmultiscripts    | tei_munderover    | tei_mprescripts    | tei_none    | tei_munder    | tei_mphantom    | tei_mpadded    | tei_menclose   )*
}
\end{Verbatim}

\end{reflist}  \index{mtable=<mtable>|oddindex}\index{columnalign=@columnalign!<mtable>|oddindex}\index{columnspan=@columnspan!<mtable>|oddindex}\index{columnlines=@columnlines!<mtable>|oddindex}\index{equalrows=@equalrows!<mtable>|oddindex}\index{equalcolumns=@equalcolumns!<mtable>|oddindex}\index{class=@class!<mtable>|oddindex}
\begin{reflist}
\item[]\begin{specHead}{TEI.mtable}{<mtable> }\end{specHead} 
    \item[{Namespace}]
  http://www.w3.org/1998/Math/MathML
    \item[{Module}]
  derived-module-tei.istex
    \item[{Attributs}]
  Attributs\hfil\\[-10pt]\begin{sansreflist}
    \item[@columnalign]
  
\begin{reflist}
    \item[{Statut}]
  Optionel
    \item[{Type de données}]
  \xref{https://www.w3.org/TR/xmlschema-2/\#}{}
\end{reflist}  
    \item[@columnspan]
  
\begin{reflist}
    \item[{Statut}]
  Optionel
    \item[{Type de données}]
  \xref{https://www.w3.org/TR/xmlschema-2/\#}{}
\end{reflist}  
    \item[@columnlines]
  
\begin{reflist}
    \item[{Statut}]
  Optionel
    \item[{Type de données}]
  \xref{https://www.w3.org/TR/xmlschema-2/\#}{}
\end{reflist}  
    \item[@equalrows]
  
\begin{reflist}
    \item[{Statut}]
  Optionel
    \item[{Type de données}]
  \xref{https://www.w3.org/TR/xmlschema-2/\#}{}
\end{reflist}  
    \item[@equalcolumns]
  
\begin{reflist}
    \item[{Statut}]
  Optionel
    \item[{Type de données}]
  \xref{https://www.w3.org/TR/xmlschema-2/\#}{}
\end{reflist}  
    \item[@class]
  
\begin{reflist}
    \item[{Statut}]
  Optionel
    \item[{Type de données}]
  \xref{https://www.w3.org/TR/xmlschema-2/\#}{}
\end{reflist}  
\end{sansreflist}  
    \item[{Contenu dans}]
  
    \item[derived-module-tei.istex: ]
   \hyperref[TEI.math]{math} \hyperref[TEI.menclose]{menclose} \hyperref[TEI.mfenced]{mfenced} \hyperref[TEI.mfrac]{mfrac} \hyperref[TEI.mmultiscripts]{mmultiscripts} \hyperref[TEI.mover]{mover} \hyperref[TEI.mpadded]{mpadded} \hyperref[TEI.mphantom]{mphantom} \hyperref[TEI.mprescripts]{mprescripts} \hyperref[TEI.mrow]{mrow} \hyperref[TEI.msqrt]{msqrt} \hyperref[TEI.mstyle]{mstyle} \hyperref[TEI.msub]{msub} \hyperref[TEI.msubsup]{msubsup} \hyperref[TEI.msup]{msup} \hyperref[TEI.msupsub]{msupsub} \hyperref[TEI.mtable]{mtable} \hyperref[TEI.mtd]{mtd} \hyperref[TEI.mtr]{mtr} \hyperref[TEI.munder]{munder} \hyperref[TEI.munderover]{munderover} \hyperref[TEI.semantics]{semantics}
    \item[{Peut contenir}]
  
    \item[derived-module-tei.istex: ]
   \hyperref[TEI.menclose]{menclose} \hyperref[TEI.mfenced]{mfenced} \hyperref[TEI.mfrac]{mfrac} \hyperref[TEI.mi]{mi} \hyperref[TEI.mmultiscripts]{mmultiscripts} \hyperref[TEI.mn]{mn} \hyperref[TEI.mo]{mo} \hyperref[TEI.mover]{mover} \hyperref[TEI.mpadded]{mpadded} \hyperref[TEI.mphantom]{mphantom} \hyperref[TEI.mprescripts]{mprescripts} \hyperref[TEI.mrow]{mrow} \hyperref[TEI.mspace]{mspace} \hyperref[TEI.msqrt]{msqrt} \hyperref[TEI.mstyle]{mstyle} \hyperref[TEI.msub]{msub} \hyperref[TEI.msubsup]{msubsup} \hyperref[TEI.msup]{msup} \hyperref[TEI.msupsub]{msupsub} \hyperref[TEI.mtable]{mtable} \hyperref[TEI.mtd]{mtd} \hyperref[TEI.mtext]{mtext} \hyperref[TEI.mtr]{mtr} \hyperref[TEI.munder]{munder} \hyperref[TEI.munderover]{munderover} \hyperref[TEI.none]{none}\par des données textuelles
    \item[{Modèle de contenu}]
  \mbox{}\hfill\\[-10pt]\begin{Verbatim}[fontsize=\small]
<content>
 <alternate maxOccurs="unbounded"
  minOccurs="0">
  <textNode/>
  <elementRef key="mstyle"/>
  <elementRef key="mtr"/>
  <elementRef key="mtd"/>
  <elementRef key="mrow"/>
  <elementRef key="mi"/>
  <elementRef key="mn"/>
  <elementRef key="mtext"/>
  <elementRef key="mfrac"/>
  <elementRef key="mspace"/>
  <elementRef key="msqrt"/>
  <elementRef key="msub"/>
  <elementRef key="msup"/>
  <elementRef key="mo"/>
  <elementRef key="mover"/>
  <elementRef key="mfenced"/>
  <elementRef key="mtable"/>
  <elementRef key="msubsup"/>
  <elementRef key="msupsub"/>
  <elementRef key="mmultiscripts"/>
  <elementRef key="munderover"/>
  <elementRef key="mprescripts"/>
  <elementRef key="none"/>
  <elementRef key="munder"/>
  <elementRef key="mphantom"/>
  <elementRef key="mpadded"/>
  <elementRef key="menclose"/>
 </alternate>
</content>
    
\end{Verbatim}

    \item[{Schéma Declaration}]
  \mbox{}\hfill\\[-10pt]\begin{Verbatim}[fontsize=\small]
element mtable
{
   attribute columnalign { columnalign }?,
   attribute columnspan { columnspan }?,
   attribute columnlines { columnlines }?,
   attribute equalrows { equalrows }?,
   attribute equalcolumns { equalcolumns }?,
   attribute class { class }?,
   (
      text
    | tei_mstyle    | tei_mtr    | tei_mtd    | tei_mrow    | tei_mi    | tei_mn    | tei_mtext    | tei_mfrac    | tei_mspace    | tei_msqrt    | tei_msub    | tei_msup    | tei_mo    | tei_mover    | tei_mfenced    | tei_mtable    | tei_msubsup    | tei_msupsub    | tei_mmultiscripts    | tei_munderover    | tei_mprescripts    | tei_none    | tei_munder    | tei_mphantom    | tei_mpadded    | tei_menclose   )*
}
\end{Verbatim}

\end{reflist}  \index{mtd=<mtd>|oddindex}\index{columnalign=@columnalign!<mtd>|oddindex}\index{columnspan=@columnspan!<mtd>|oddindex}\index{rowalign=@rowalign!<mtd>|oddindex}\index{class=@class!<mtd>|oddindex}
\begin{reflist}
\item[]\begin{specHead}{TEI.mtd}{<mtd> }\end{specHead} 
    \item[{Namespace}]
  http://www.w3.org/1998/Math/MathML
    \item[{Module}]
  derived-module-tei.istex
    \item[{Attributs}]
  Attributs\hfil\\[-10pt]\begin{sansreflist}
    \item[@columnalign]
  
\begin{reflist}
    \item[{Statut}]
  Optionel
    \item[{Type de données}]
  \xref{https://www.w3.org/TR/xmlschema-2/\#}{}
\end{reflist}  
    \item[@columnspan]
  
\begin{reflist}
    \item[{Statut}]
  Optionel
    \item[{Type de données}]
  \xref{https://www.w3.org/TR/xmlschema-2/\#}{}
\end{reflist}  
    \item[@rowalign]
  
\begin{reflist}
    \item[{Statut}]
  Optionel
    \item[{Type de données}]
  \xref{https://www.w3.org/TR/xmlschema-2/\#}{}
\end{reflist}  
    \item[@class]
  
\begin{reflist}
    \item[{Statut}]
  Optionel
    \item[{Type de données}]
  \xref{https://www.w3.org/TR/xmlschema-2/\#}{}
\end{reflist}  
\end{sansreflist}  
    \item[{Contenu dans}]
  
    \item[derived-module-tei.istex: ]
   \hyperref[TEI.menclose]{menclose} \hyperref[TEI.mfenced]{mfenced} \hyperref[TEI.mfrac]{mfrac} \hyperref[TEI.mmultiscripts]{mmultiscripts} \hyperref[TEI.mover]{mover} \hyperref[TEI.mpadded]{mpadded} \hyperref[TEI.mphantom]{mphantom} \hyperref[TEI.mprescripts]{mprescripts} \hyperref[TEI.mrow]{mrow} \hyperref[TEI.msqrt]{msqrt} \hyperref[TEI.mstyle]{mstyle} \hyperref[TEI.msub]{msub} \hyperref[TEI.msubsup]{msubsup} \hyperref[TEI.msup]{msup} \hyperref[TEI.msupsub]{msupsub} \hyperref[TEI.mtable]{mtable} \hyperref[TEI.mtd]{mtd} \hyperref[TEI.mtr]{mtr} \hyperref[TEI.munder]{munder} \hyperref[TEI.munderover]{munderover} \hyperref[TEI.semantics]{semantics}
    \item[{Peut contenir}]
  
    \item[derived-module-tei.istex: ]
   \hyperref[TEI.menclose]{menclose} \hyperref[TEI.mfenced]{mfenced} \hyperref[TEI.mfrac]{mfrac} \hyperref[TEI.mi]{mi} \hyperref[TEI.mmultiscripts]{mmultiscripts} \hyperref[TEI.mn]{mn} \hyperref[TEI.mo]{mo} \hyperref[TEI.mover]{mover} \hyperref[TEI.mpadded]{mpadded} \hyperref[TEI.mphantom]{mphantom} \hyperref[TEI.mprescripts]{mprescripts} \hyperref[TEI.mrow]{mrow} \hyperref[TEI.mspace]{mspace} \hyperref[TEI.msqrt]{msqrt} \hyperref[TEI.mstyle]{mstyle} \hyperref[TEI.msub]{msub} \hyperref[TEI.msubsup]{msubsup} \hyperref[TEI.msup]{msup} \hyperref[TEI.msupsub]{msupsub} \hyperref[TEI.mtable]{mtable} \hyperref[TEI.mtd]{mtd} \hyperref[TEI.mtext]{mtext} \hyperref[TEI.mtr]{mtr} \hyperref[TEI.munder]{munder} \hyperref[TEI.munderover]{munderover} \hyperref[TEI.none]{none}\par des données textuelles
    \item[{Modèle de contenu}]
  \mbox{}\hfill\\[-10pt]\begin{Verbatim}[fontsize=\small]
<content>
 <alternate maxOccurs="unbounded"
  minOccurs="0">
  <textNode/>
  <elementRef key="mstyle"/>
  <elementRef key="mtr"/>
  <elementRef key="mtd"/>
  <elementRef key="mrow"/>
  <elementRef key="mi"/>
  <elementRef key="mn"/>
  <elementRef key="mtext"/>
  <elementRef key="mfrac"/>
  <elementRef key="mspace"/>
  <elementRef key="msqrt"/>
  <elementRef key="msub"/>
  <elementRef key="msup"/>
  <elementRef key="mo"/>
  <elementRef key="mover"/>
  <elementRef key="mfenced"/>
  <elementRef key="mtable"/>
  <elementRef key="msubsup"/>
  <elementRef key="msupsub"/>
  <elementRef key="mmultiscripts"/>
  <elementRef key="munderover"/>
  <elementRef key="mprescripts"/>
  <elementRef key="none"/>
  <elementRef key="munder"/>
  <elementRef key="mphantom"/>
  <elementRef key="mpadded"/>
  <elementRef key="menclose"/>
 </alternate>
</content>
    
\end{Verbatim}

    \item[{Schéma Declaration}]
  \mbox{}\hfill\\[-10pt]\begin{Verbatim}[fontsize=\small]
element mtd
{
   attribute columnalign { columnalign }?,
   attribute columnspan { columnspan }?,
   attribute rowalign { rowalign }?,
   attribute class { class }?,
   (
      text
    | tei_mstyle    | tei_mtr    | tei_mtd    | tei_mrow    | tei_mi    | tei_mn    | tei_mtext    | tei_mfrac    | tei_mspace    | tei_msqrt    | tei_msub    | tei_msup    | tei_mo    | tei_mover    | tei_mfenced    | tei_mtable    | tei_msubsup    | tei_msupsub    | tei_mmultiscripts    | tei_munderover    | tei_mprescripts    | tei_none    | tei_munder    | tei_mphantom    | tei_mpadded    | tei_menclose   )*
}
\end{Verbatim}

\end{reflist}  \index{mtext=<mtext>|oddindex}\index{fontstyle=@fontstyle!<mtext>|oddindex}\index{fontweight=@fontweight!<mtext>|oddindex}\index{mathvariant=@mathvariant!<mtext>|oddindex}
\begin{reflist}
\item[]\begin{specHead}{TEI.mtext}{<mtext> }\end{specHead} 
    \item[{Namespace}]
  http://www.w3.org/1998/Math/MathML
    \item[{Module}]
  derived-module-tei.istex
    \item[{Attributs}]
  Attributs\hfil\\[-10pt]\begin{sansreflist}
    \item[@fontstyle]
  
\begin{reflist}
    \item[{Statut}]
  Optionel
    \item[{Type de données}]
  \xref{https://www.w3.org/TR/xmlschema-2/\#}{}
\end{reflist}  
    \item[@fontweight]
  
\begin{reflist}
    \item[{Statut}]
  Optionel
    \item[{Type de données}]
  \xref{https://www.w3.org/TR/xmlschema-2/\#}{}
\end{reflist}  
    \item[@mathvariant]
  
\begin{reflist}
    \item[{Statut}]
  Optionel
    \item[{Type de données}]
  \xref{https://www.w3.org/TR/xmlschema-2/\#}{}
\end{reflist}  
\end{sansreflist}  
    \item[{Contenu dans}]
  
    \item[derived-module-tei.istex: ]
   \hyperref[TEI.math]{math} \hyperref[TEI.menclose]{menclose} \hyperref[TEI.mfenced]{mfenced} \hyperref[TEI.mfrac]{mfrac} \hyperref[TEI.mmultiscripts]{mmultiscripts} \hyperref[TEI.mover]{mover} \hyperref[TEI.mpadded]{mpadded} \hyperref[TEI.mphantom]{mphantom} \hyperref[TEI.mprescripts]{mprescripts} \hyperref[TEI.mrow]{mrow} \hyperref[TEI.msqrt]{msqrt} \hyperref[TEI.mstyle]{mstyle} \hyperref[TEI.msub]{msub} \hyperref[TEI.msubsup]{msubsup} \hyperref[TEI.msup]{msup} \hyperref[TEI.msupsub]{msupsub} \hyperref[TEI.mtable]{mtable} \hyperref[TEI.mtd]{mtd} \hyperref[TEI.mtr]{mtr} \hyperref[TEI.munder]{munder} \hyperref[TEI.munderover]{munderover} \hyperref[TEI.semantics]{semantics}
    \item[{Peut contenir}]
  Des données textuelles uniquement
    \item[{Modèle de contenu}]
  \fbox{\ttfamily <content>\newline
 <textNode/>\newline
</content>\newline
    } 
    \item[{Schéma Declaration}]
  \mbox{}\hfill\\[-10pt]\begin{Verbatim}[fontsize=\small]
element mtext
{
   attribute fontstyle { fontstyle }?,
   attribute fontweight { fontweight }?,
   attribute mathvariant { mathvariant }?,
   text
}
\end{Verbatim}

\end{reflist}  \index{mtr=<mtr>|oddindex}\index{columnalign=@columnalign!<mtr>|oddindex}\index{columnspan=@columnspan!<mtr>|oddindex}
\begin{reflist}
\item[]\begin{specHead}{TEI.mtr}{<mtr> }\end{specHead} 
    \item[{Namespace}]
  http://www.w3.org/1998/Math/MathML
    \item[{Module}]
  derived-module-tei.istex
    \item[{Attributs}]
  Attributs\hfil\\[-10pt]\begin{sansreflist}
    \item[@columnalign]
  
\begin{reflist}
    \item[{Statut}]
  Optionel
    \item[{Type de données}]
  \xref{https://www.w3.org/TR/xmlschema-2/\#}{}
\end{reflist}  
    \item[@columnspan]
  
\begin{reflist}
    \item[{Statut}]
  Optionel
    \item[{Type de données}]
  \xref{https://www.w3.org/TR/xmlschema-2/\#}{}
\end{reflist}  
\end{sansreflist}  
    \item[{Contenu dans}]
  
    \item[derived-module-tei.istex: ]
   \hyperref[TEI.menclose]{menclose} \hyperref[TEI.mfenced]{mfenced} \hyperref[TEI.mfrac]{mfrac} \hyperref[TEI.mmultiscripts]{mmultiscripts} \hyperref[TEI.mover]{mover} \hyperref[TEI.mpadded]{mpadded} \hyperref[TEI.mphantom]{mphantom} \hyperref[TEI.mprescripts]{mprescripts} \hyperref[TEI.mrow]{mrow} \hyperref[TEI.msqrt]{msqrt} \hyperref[TEI.mstyle]{mstyle} \hyperref[TEI.msub]{msub} \hyperref[TEI.msubsup]{msubsup} \hyperref[TEI.msup]{msup} \hyperref[TEI.msupsub]{msupsub} \hyperref[TEI.mtable]{mtable} \hyperref[TEI.mtd]{mtd} \hyperref[TEI.mtr]{mtr} \hyperref[TEI.munder]{munder} \hyperref[TEI.munderover]{munderover} \hyperref[TEI.semantics]{semantics}
    \item[{Peut contenir}]
  
    \item[derived-module-tei.istex: ]
   \hyperref[TEI.menclose]{menclose} \hyperref[TEI.mfenced]{mfenced} \hyperref[TEI.mfrac]{mfrac} \hyperref[TEI.mi]{mi} \hyperref[TEI.mmultiscripts]{mmultiscripts} \hyperref[TEI.mn]{mn} \hyperref[TEI.mo]{mo} \hyperref[TEI.mover]{mover} \hyperref[TEI.mpadded]{mpadded} \hyperref[TEI.mphantom]{mphantom} \hyperref[TEI.mprescripts]{mprescripts} \hyperref[TEI.mrow]{mrow} \hyperref[TEI.mspace]{mspace} \hyperref[TEI.msqrt]{msqrt} \hyperref[TEI.mstyle]{mstyle} \hyperref[TEI.msub]{msub} \hyperref[TEI.msubsup]{msubsup} \hyperref[TEI.msup]{msup} \hyperref[TEI.msupsub]{msupsub} \hyperref[TEI.mtable]{mtable} \hyperref[TEI.mtd]{mtd} \hyperref[TEI.mtext]{mtext} \hyperref[TEI.mtr]{mtr} \hyperref[TEI.munder]{munder} \hyperref[TEI.munderover]{munderover} \hyperref[TEI.none]{none}\par des données textuelles
    \item[{Modèle de contenu}]
  \mbox{}\hfill\\[-10pt]\begin{Verbatim}[fontsize=\small]
<content>
 <alternate maxOccurs="unbounded"
  minOccurs="0">
  <textNode/>
  <elementRef key="mstyle"/>
  <elementRef key="mtr"/>
  <elementRef key="mtd"/>
  <elementRef key="mrow"/>
  <elementRef key="mi"/>
  <elementRef key="mn"/>
  <elementRef key="mtext"/>
  <elementRef key="mfrac"/>
  <elementRef key="mspace"/>
  <elementRef key="msqrt"/>
  <elementRef key="msub"/>
  <elementRef key="msup"/>
  <elementRef key="mo"/>
  <elementRef key="mover"/>
  <elementRef key="mfenced"/>
  <elementRef key="mtable"/>
  <elementRef key="msubsup"/>
  <elementRef key="msupsub"/>
  <elementRef key="mmultiscripts"/>
  <elementRef key="munderover"/>
  <elementRef key="mprescripts"/>
  <elementRef key="none"/>
  <elementRef key="munder"/>
  <elementRef key="mphantom"/>
  <elementRef key="mpadded"/>
  <elementRef key="menclose"/>
 </alternate>
</content>
    
\end{Verbatim}

    \item[{Schéma Declaration}]
  \mbox{}\hfill\\[-10pt]\begin{Verbatim}[fontsize=\small]
element mtr
{
   attribute columnalign { columnalign }?,
   attribute columnspan { columnspan }?,
   (
      text
    | tei_mstyle    | tei_mtr    | tei_mtd    | tei_mrow    | tei_mi    | tei_mn    | tei_mtext    | tei_mfrac    | tei_mspace    | tei_msqrt    | tei_msub    | tei_msup    | tei_mo    | tei_mover    | tei_mfenced    | tei_mtable    | tei_msubsup    | tei_msupsub    | tei_mmultiscripts    | tei_munderover    | tei_mprescripts    | tei_none    | tei_munder    | tei_mphantom    | tei_mpadded    | tei_menclose   )*
}
\end{Verbatim}

\end{reflist}  \index{munder=<munder>|oddindex}\index{accent=@accent!<munder>|oddindex}\index{accentunder=@accentunder!<munder>|oddindex}\index{class=@class!<munder>|oddindex}
\begin{reflist}
\item[]\begin{specHead}{TEI.munder}{<munder> }\end{specHead} 
    \item[{Namespace}]
  http://www.w3.org/1998/Math/MathML
    \item[{Module}]
  derived-module-tei.istex
    \item[{Attributs}]
  Attributs\hfil\\[-10pt]\begin{sansreflist}
    \item[@accent]
  
\begin{reflist}
    \item[{Statut}]
  Optionel
    \item[{Type de données}]
  \xref{https://www.w3.org/TR/xmlschema-2/\#}{}
\end{reflist}  
    \item[@accentunder]
  
\begin{reflist}
    \item[{Statut}]
  Optionel
    \item[{Type de données}]
  \xref{https://www.w3.org/TR/xmlschema-2/\#}{}
\end{reflist}  
    \item[@class]
  
\begin{reflist}
    \item[{Statut}]
  Optionel
    \item[{Type de données}]
  \xref{https://www.w3.org/TR/xmlschema-2/\#}{}
\end{reflist}  
\end{sansreflist}  
    \item[{Contenu dans}]
  
    \item[derived-module-tei.istex: ]
   \hyperref[TEI.math]{math} \hyperref[TEI.menclose]{menclose} \hyperref[TEI.mfenced]{mfenced} \hyperref[TEI.mfrac]{mfrac} \hyperref[TEI.mmultiscripts]{mmultiscripts} \hyperref[TEI.mover]{mover} \hyperref[TEI.mpadded]{mpadded} \hyperref[TEI.mphantom]{mphantom} \hyperref[TEI.mprescripts]{mprescripts} \hyperref[TEI.mrow]{mrow} \hyperref[TEI.msqrt]{msqrt} \hyperref[TEI.mstyle]{mstyle} \hyperref[TEI.msub]{msub} \hyperref[TEI.msubsup]{msubsup} \hyperref[TEI.msup]{msup} \hyperref[TEI.msupsub]{msupsub} \hyperref[TEI.mtable]{mtable} \hyperref[TEI.mtd]{mtd} \hyperref[TEI.mtr]{mtr} \hyperref[TEI.munder]{munder} \hyperref[TEI.munderover]{munderover} \hyperref[TEI.semantics]{semantics}
    \item[{Peut contenir}]
  
    \item[derived-module-tei.istex: ]
   \hyperref[TEI.menclose]{menclose} \hyperref[TEI.mfenced]{mfenced} \hyperref[TEI.mfrac]{mfrac} \hyperref[TEI.mi]{mi} \hyperref[TEI.mmultiscripts]{mmultiscripts} \hyperref[TEI.mn]{mn} \hyperref[TEI.mo]{mo} \hyperref[TEI.mover]{mover} \hyperref[TEI.mpadded]{mpadded} \hyperref[TEI.mphantom]{mphantom} \hyperref[TEI.mprescripts]{mprescripts} \hyperref[TEI.mrow]{mrow} \hyperref[TEI.mspace]{mspace} \hyperref[TEI.msqrt]{msqrt} \hyperref[TEI.mstyle]{mstyle} \hyperref[TEI.msub]{msub} \hyperref[TEI.msubsup]{msubsup} \hyperref[TEI.msup]{msup} \hyperref[TEI.msupsub]{msupsub} \hyperref[TEI.mtable]{mtable} \hyperref[TEI.mtd]{mtd} \hyperref[TEI.mtext]{mtext} \hyperref[TEI.mtr]{mtr} \hyperref[TEI.munder]{munder} \hyperref[TEI.munderover]{munderover} \hyperref[TEI.none]{none}\par des données textuelles
    \item[{Modèle de contenu}]
  \mbox{}\hfill\\[-10pt]\begin{Verbatim}[fontsize=\small]
<content>
 <alternate maxOccurs="unbounded"
  minOccurs="0">
  <textNode/>
  <elementRef key="mstyle"/>
  <elementRef key="mtr"/>
  <elementRef key="mtd"/>
  <elementRef key="mrow"/>
  <elementRef key="mi"/>
  <elementRef key="mn"/>
  <elementRef key="mtext"/>
  <elementRef key="mfrac"/>
  <elementRef key="mspace"/>
  <elementRef key="msqrt"/>
  <elementRef key="msub"/>
  <elementRef key="msup"/>
  <elementRef key="mo"/>
  <elementRef key="mover"/>
  <elementRef key="mfenced"/>
  <elementRef key="mtable"/>
  <elementRef key="msubsup"/>
  <elementRef key="msupsub"/>
  <elementRef key="mmultiscripts"/>
  <elementRef key="munderover"/>
  <elementRef key="mprescripts"/>
  <elementRef key="none"/>
  <elementRef key="munder"/>
  <elementRef key="mphantom"/>
  <elementRef key="mpadded"/>
  <elementRef key="menclose"/>
 </alternate>
</content>
    
\end{Verbatim}

    \item[{Schéma Declaration}]
  \mbox{}\hfill\\[-10pt]\begin{Verbatim}[fontsize=\small]
element munder
{
   attribute accent { accent }?,
   attribute accentunder { accentunder }?,
   attribute class { class }?,
   (
      text
    | tei_mstyle    | tei_mtr    | tei_mtd    | tei_mrow    | tei_mi    | tei_mn    | tei_mtext    | tei_mfrac    | tei_mspace    | tei_msqrt    | tei_msub    | tei_msup    | tei_mo    | tei_mover    | tei_mfenced    | tei_mtable    | tei_msubsup    | tei_msupsub    | tei_mmultiscripts    | tei_munderover    | tei_mprescripts    | tei_none    | tei_munder    | tei_mphantom    | tei_mpadded    | tei_menclose   )*
}
\end{Verbatim}

\end{reflist}  \index{munderover=<munderover>|oddindex}\index{accentunder=@accentunder!<munderover>|oddindex}\index{accent=@accent!<munderover>|oddindex}
\begin{reflist}
\item[]\begin{specHead}{TEI.munderover}{<munderover> }\end{specHead} 
    \item[{Namespace}]
  http://www.w3.org/1998/Math/MathML
    \item[{Module}]
  derived-module-tei.istex
    \item[{Attributs}]
  Attributs\hfil\\[-10pt]\begin{sansreflist}
    \item[@accentunder]
  
\begin{reflist}
    \item[{Statut}]
  Optionel
    \item[{Type de données}]
  \xref{https://www.w3.org/TR/xmlschema-2/\#}{}
\end{reflist}  
    \item[@accent]
  
\begin{reflist}
    \item[{Statut}]
  Optionel
    \item[{Type de données}]
  \xref{https://www.w3.org/TR/xmlschema-2/\#}{}
\end{reflist}  
\end{sansreflist}  
    \item[{Contenu dans}]
  
    \item[derived-module-tei.istex: ]
   \hyperref[TEI.math]{math} \hyperref[TEI.menclose]{menclose} \hyperref[TEI.mfenced]{mfenced} \hyperref[TEI.mfrac]{mfrac} \hyperref[TEI.mmultiscripts]{mmultiscripts} \hyperref[TEI.mover]{mover} \hyperref[TEI.mpadded]{mpadded} \hyperref[TEI.mphantom]{mphantom} \hyperref[TEI.mprescripts]{mprescripts} \hyperref[TEI.mrow]{mrow} \hyperref[TEI.msqrt]{msqrt} \hyperref[TEI.mstyle]{mstyle} \hyperref[TEI.msub]{msub} \hyperref[TEI.msubsup]{msubsup} \hyperref[TEI.msup]{msup} \hyperref[TEI.msupsub]{msupsub} \hyperref[TEI.mtable]{mtable} \hyperref[TEI.mtd]{mtd} \hyperref[TEI.mtr]{mtr} \hyperref[TEI.munder]{munder} \hyperref[TEI.munderover]{munderover} \hyperref[TEI.semantics]{semantics}
    \item[{Peut contenir}]
  
    \item[derived-module-tei.istex: ]
   \hyperref[TEI.menclose]{menclose} \hyperref[TEI.mfenced]{mfenced} \hyperref[TEI.mfrac]{mfrac} \hyperref[TEI.mi]{mi} \hyperref[TEI.mmultiscripts]{mmultiscripts} \hyperref[TEI.mn]{mn} \hyperref[TEI.mo]{mo} \hyperref[TEI.mover]{mover} \hyperref[TEI.mpadded]{mpadded} \hyperref[TEI.mphantom]{mphantom} \hyperref[TEI.mprescripts]{mprescripts} \hyperref[TEI.mrow]{mrow} \hyperref[TEI.mspace]{mspace} \hyperref[TEI.msqrt]{msqrt} \hyperref[TEI.mstyle]{mstyle} \hyperref[TEI.msub]{msub} \hyperref[TEI.msubsup]{msubsup} \hyperref[TEI.msup]{msup} \hyperref[TEI.msupsub]{msupsub} \hyperref[TEI.mtable]{mtable} \hyperref[TEI.mtd]{mtd} \hyperref[TEI.mtext]{mtext} \hyperref[TEI.mtr]{mtr} \hyperref[TEI.munder]{munder} \hyperref[TEI.munderover]{munderover} \hyperref[TEI.none]{none}\par des données textuelles
    \item[{Modèle de contenu}]
  \mbox{}\hfill\\[-10pt]\begin{Verbatim}[fontsize=\small]
<content>
 <alternate maxOccurs="unbounded"
  minOccurs="0">
  <textNode/>
  <elementRef key="mstyle"/>
  <elementRef key="mtr"/>
  <elementRef key="mtd"/>
  <elementRef key="mrow"/>
  <elementRef key="mi"/>
  <elementRef key="mn"/>
  <elementRef key="mtext"/>
  <elementRef key="mfrac"/>
  <elementRef key="mspace"/>
  <elementRef key="msqrt"/>
  <elementRef key="msub"/>
  <elementRef key="msup"/>
  <elementRef key="mo"/>
  <elementRef key="mover"/>
  <elementRef key="mfenced"/>
  <elementRef key="mtable"/>
  <elementRef key="msubsup"/>
  <elementRef key="msupsub"/>
  <elementRef key="mmultiscripts"/>
  <elementRef key="munderover"/>
  <elementRef key="mprescripts"/>
  <elementRef key="none"/>
  <elementRef key="munder"/>
  <elementRef key="mphantom"/>
  <elementRef key="mpadded"/>
  <elementRef key="menclose"/>
 </alternate>
</content>
    
\end{Verbatim}

    \item[{Schéma Declaration}]
  \mbox{}\hfill\\[-10pt]\begin{Verbatim}[fontsize=\small]
element munderover
{
   attribute accentunder { accentunder }?,
   attribute accent { accent }?,
   (
      text
    | tei_mstyle    | tei_mtr    | tei_mtd    | tei_mrow    | tei_mi    | tei_mn    | tei_mtext    | tei_mfrac    | tei_mspace    | tei_msqrt    | tei_msub    | tei_msup    | tei_mo    | tei_mover    | tei_mfenced    | tei_mtable    | tei_msubsup    | tei_msupsub    | tei_mmultiscripts    | tei_munderover    | tei_mprescripts    | tei_none    | tei_munder    | tei_mphantom    | tei_mpadded    | tei_menclose   )*
}
\end{Verbatim}

\end{reflist}  \index{musicNotation=<musicNotation>|oddindex}
\begin{reflist}
\item[]\begin{specHead}{TEI.musicNotation}{<musicNotation> }(notation musicale) contient la description d'un type de notation musicale. [\xref{http://www.tei-c.org/release/doc/tei-p5-doc/en/html/MS.html\#msph2}{10.7.2. Writing, Decoration, and Other Notations}]\end{specHead} 
    \item[{Module}]
  msdescription
    \item[{Attributs}]
  Attributs \hyperref[TEI.att.global]{att.global} (\textit{@xml:id}, \textit{@n}, \textit{@xml:lang}, \textit{@xml:base}, \textit{@xml:space})  (\hyperref[TEI.att.global.rendition]{att.global.rendition} (\textit{@rend}, \textit{@style}, \textit{@rendition})) (\hyperref[TEI.att.global.linking]{att.global.linking} (\textit{@corresp}, \textit{@synch}, \textit{@sameAs}, \textit{@copyOf}, \textit{@next}, \textit{@prev}, \textit{@exclude}, \textit{@select})) (\hyperref[TEI.att.global.analytic]{att.global.analytic} (\textit{@ana})) (\hyperref[TEI.att.global.facs]{att.global.facs} (\textit{@facs})) (\hyperref[TEI.att.global.change]{att.global.change} (\textit{@change})) (\hyperref[TEI.att.global.responsibility]{att.global.responsibility} (\textit{@cert}, \textit{@resp})) (\hyperref[TEI.att.global.source]{att.global.source} (\textit{@source}))
    \item[{Membre du}]
  \hyperref[TEI.model.physDescPart]{model.physDescPart}
    \item[{Contenu dans}]
  
    \item[msdescription: ]
   \hyperref[TEI.physDesc]{physDesc}
    \item[{Peut contenir}]
  
    \item[analysis: ]
   \hyperref[TEI.c]{c} \hyperref[TEI.cl]{cl} \hyperref[TEI.interp]{interp} \hyperref[TEI.interpGrp]{interpGrp} \hyperref[TEI.m]{m} \hyperref[TEI.pc]{pc} \hyperref[TEI.phr]{phr} \hyperref[TEI.s]{s} \hyperref[TEI.span]{span} \hyperref[TEI.spanGrp]{spanGrp} \hyperref[TEI.w]{w}\par 
    \item[core: ]
   \hyperref[TEI.abbr]{abbr} \hyperref[TEI.add]{add} \hyperref[TEI.address]{address} \hyperref[TEI.bibl]{bibl} \hyperref[TEI.biblStruct]{biblStruct} \hyperref[TEI.binaryObject]{binaryObject} \hyperref[TEI.cb]{cb} \hyperref[TEI.choice]{choice} \hyperref[TEI.cit]{cit} \hyperref[TEI.corr]{corr} \hyperref[TEI.date]{date} \hyperref[TEI.del]{del} \hyperref[TEI.desc]{desc} \hyperref[TEI.distinct]{distinct} \hyperref[TEI.email]{email} \hyperref[TEI.emph]{emph} \hyperref[TEI.expan]{expan} \hyperref[TEI.foreign]{foreign} \hyperref[TEI.gap]{gap} \hyperref[TEI.gb]{gb} \hyperref[TEI.gloss]{gloss} \hyperref[TEI.graphic]{graphic} \hyperref[TEI.hi]{hi} \hyperref[TEI.index]{index} \hyperref[TEI.l]{l} \hyperref[TEI.label]{label} \hyperref[TEI.lb]{lb} \hyperref[TEI.lg]{lg} \hyperref[TEI.list]{list} \hyperref[TEI.listBibl]{listBibl} \hyperref[TEI.measure]{measure} \hyperref[TEI.measureGrp]{measureGrp} \hyperref[TEI.media]{media} \hyperref[TEI.mentioned]{mentioned} \hyperref[TEI.milestone]{milestone} \hyperref[TEI.name]{name} \hyperref[TEI.note]{note} \hyperref[TEI.num]{num} \hyperref[TEI.orig]{orig} \hyperref[TEI.p]{p} \hyperref[TEI.pb]{pb} \hyperref[TEI.ptr]{ptr} \hyperref[TEI.q]{q} \hyperref[TEI.quote]{quote} \hyperref[TEI.ref]{ref} \hyperref[TEI.reg]{reg} \hyperref[TEI.rs]{rs} \hyperref[TEI.said]{said} \hyperref[TEI.sic]{sic} \hyperref[TEI.soCalled]{soCalled} \hyperref[TEI.sp]{sp} \hyperref[TEI.stage]{stage} \hyperref[TEI.term]{term} \hyperref[TEI.time]{time} \hyperref[TEI.title]{title} \hyperref[TEI.unclear]{unclear}\par 
    \item[derived-module-tei.istex: ]
   \hyperref[TEI.math]{math} \hyperref[TEI.mrow]{mrow}\par 
    \item[figures: ]
   \hyperref[TEI.figure]{figure} \hyperref[TEI.formula]{formula} \hyperref[TEI.notatedMusic]{notatedMusic} \hyperref[TEI.table]{table}\par 
    \item[header: ]
   \hyperref[TEI.biblFull]{biblFull} \hyperref[TEI.idno]{idno}\par 
    \item[iso-fs: ]
   \hyperref[TEI.fLib]{fLib} \hyperref[TEI.fs]{fs} \hyperref[TEI.fvLib]{fvLib}\par 
    \item[linking: ]
   \hyperref[TEI.ab]{ab} \hyperref[TEI.alt]{alt} \hyperref[TEI.altGrp]{altGrp} \hyperref[TEI.anchor]{anchor} \hyperref[TEI.join]{join} \hyperref[TEI.joinGrp]{joinGrp} \hyperref[TEI.link]{link} \hyperref[TEI.linkGrp]{linkGrp} \hyperref[TEI.seg]{seg} \hyperref[TEI.timeline]{timeline}\par 
    \item[msdescription: ]
   \hyperref[TEI.catchwords]{catchwords} \hyperref[TEI.depth]{depth} \hyperref[TEI.dim]{dim} \hyperref[TEI.dimensions]{dimensions} \hyperref[TEI.height]{height} \hyperref[TEI.heraldry]{heraldry} \hyperref[TEI.locus]{locus} \hyperref[TEI.locusGrp]{locusGrp} \hyperref[TEI.material]{material} \hyperref[TEI.msDesc]{msDesc} \hyperref[TEI.objectType]{objectType} \hyperref[TEI.origDate]{origDate} \hyperref[TEI.origPlace]{origPlace} \hyperref[TEI.secFol]{secFol} \hyperref[TEI.signatures]{signatures} \hyperref[TEI.source]{source} \hyperref[TEI.stamp]{stamp} \hyperref[TEI.watermark]{watermark} \hyperref[TEI.width]{width}\par 
    \item[namesdates: ]
   \hyperref[TEI.addName]{addName} \hyperref[TEI.affiliation]{affiliation} \hyperref[TEI.country]{country} \hyperref[TEI.forename]{forename} \hyperref[TEI.genName]{genName} \hyperref[TEI.geogName]{geogName} \hyperref[TEI.listOrg]{listOrg} \hyperref[TEI.listPlace]{listPlace} \hyperref[TEI.location]{location} \hyperref[TEI.nameLink]{nameLink} \hyperref[TEI.orgName]{orgName} \hyperref[TEI.persName]{persName} \hyperref[TEI.placeName]{placeName} \hyperref[TEI.region]{region} \hyperref[TEI.roleName]{roleName} \hyperref[TEI.settlement]{settlement} \hyperref[TEI.state]{state} \hyperref[TEI.surname]{surname}\par 
    \item[spoken: ]
   \hyperref[TEI.annotationBlock]{annotationBlock}\par 
    \item[textstructure: ]
   \hyperref[TEI.floatingText]{floatingText}\par 
    \item[transcr: ]
   \hyperref[TEI.addSpan]{addSpan} \hyperref[TEI.am]{am} \hyperref[TEI.damage]{damage} \hyperref[TEI.damageSpan]{damageSpan} \hyperref[TEI.delSpan]{delSpan} \hyperref[TEI.ex]{ex} \hyperref[TEI.fw]{fw} \hyperref[TEI.handShift]{handShift} \hyperref[TEI.listTranspose]{listTranspose} \hyperref[TEI.metamark]{metamark} \hyperref[TEI.mod]{mod} \hyperref[TEI.redo]{redo} \hyperref[TEI.restore]{restore} \hyperref[TEI.retrace]{retrace} \hyperref[TEI.secl]{secl} \hyperref[TEI.space]{space} \hyperref[TEI.subst]{subst} \hyperref[TEI.substJoin]{substJoin} \hyperref[TEI.supplied]{supplied} \hyperref[TEI.surplus]{surplus} \hyperref[TEI.undo]{undo}\par des données textuelles
    \item[{Exemple}]
  \leavevmode\bgroup\exampleFont \begin{shaded}\noindent\mbox{}{<\textbf{musicNotation}>}\mbox{}\newline 
\hspace*{6pt}{<\textbf{p}>}Les clés se placent au commencement de la portée. Elles servent à fixer le nom des\mbox{}\newline 
\hspace*{6pt}\hspace*{6pt} notes et à indiquer en même temps la place que celles-ci occupent dans l'échelle\mbox{}\newline 
\hspace*{6pt}\hspace*{6pt} musicale.{</\textbf{p}>}\mbox{}\newline 
{</\textbf{musicNotation}>}\end{shaded}\egroup 


    \item[{Exemple}]
  \leavevmode\bgroup\exampleFont \begin{shaded}\noindent\mbox{}{<\textbf{musicNotation}>}Même, si l'on voulait démontrer que les livres de chants ont été\mbox{}\newline 
{<\textbf{term}>}neumés{</\textbf{term}>} dés le IXe siècle, il ne faudrait pas oublier que des livres de chants\mbox{}\newline 
 sans {<\textbf{term}>}neumes{</\textbf{term}>} ont été écrits jusqu'au Xe siècle. {</\textbf{musicNotation}>}\end{shaded}\egroup 


    \item[{Modèle de contenu}]
  \mbox{}\hfill\\[-10pt]\begin{Verbatim}[fontsize=\small]
<content>
 <macroRef key="macro.specialPara"/>
</content>
    
\end{Verbatim}

    \item[{Schéma Declaration}]
  \mbox{}\hfill\\[-10pt]\begin{Verbatim}[fontsize=\small]
element musicNotation { tei_att.global.attributes, tei_macro.specialPara }
\end{Verbatim}

\end{reflist}  \index{name=<name>|oddindex}
\begin{reflist}
\item[]\begin{specHead}{TEI.name}{<name> }(nom, nom propre) contient un nom propre ou un syntagme nominal. [\xref{http://www.tei-c.org/release/doc/tei-p5-doc/en/html/CO.html\#CONARS}{3.5.1. Referring Strings}]\end{specHead} 
    \item[{Module}]
  core
    \item[{Attributs}]
  Attributs \hyperref[TEI.att.global]{att.global} (\textit{@xml:id}, \textit{@n}, \textit{@xml:lang}, \textit{@xml:base}, \textit{@xml:space})  (\hyperref[TEI.att.global.rendition]{att.global.rendition} (\textit{@rend}, \textit{@style}, \textit{@rendition})) (\hyperref[TEI.att.global.linking]{att.global.linking} (\textit{@corresp}, \textit{@synch}, \textit{@sameAs}, \textit{@copyOf}, \textit{@next}, \textit{@prev}, \textit{@exclude}, \textit{@select})) (\hyperref[TEI.att.global.analytic]{att.global.analytic} (\textit{@ana})) (\hyperref[TEI.att.global.facs]{att.global.facs} (\textit{@facs})) (\hyperref[TEI.att.global.change]{att.global.change} (\textit{@change})) (\hyperref[TEI.att.global.responsibility]{att.global.responsibility} (\textit{@cert}, \textit{@resp})) (\hyperref[TEI.att.global.source]{att.global.source} (\textit{@source})) \hyperref[TEI.att.personal]{att.personal} (\textit{@full}, \textit{@sort})  (\hyperref[TEI.att.naming]{att.naming} (\textit{@role}, \textit{@nymRef}) (\hyperref[TEI.att.canonical]{att.canonical} (\textit{@key}, \textit{@ref})) ) \hyperref[TEI.att.datable]{att.datable} (\textit{@calendar}, \textit{@period})  (\hyperref[TEI.att.datable.w3c]{att.datable.w3c} (\textit{@when}, \textit{@notBefore}, \textit{@notAfter}, \textit{@from}, \textit{@to})) (\hyperref[TEI.att.datable.iso]{att.datable.iso} (\textit{@when-iso}, \textit{@notBefore-iso}, \textit{@notAfter-iso}, \textit{@from-iso}, \textit{@to-iso})) (\hyperref[TEI.att.datable.custom]{att.datable.custom} (\textit{@when-custom}, \textit{@notBefore-custom}, \textit{@notAfter-custom}, \textit{@from-custom}, \textit{@to-custom}, \textit{@datingPoint}, \textit{@datingMethod})) \hyperref[TEI.att.editLike]{att.editLike} (\textit{@evidence}, \textit{@instant})  (\hyperref[TEI.att.dimensions]{att.dimensions} (\textit{@unit}, \textit{@quantity}, \textit{@extent}, \textit{@precision}, \textit{@scope}) (\hyperref[TEI.att.ranging]{att.ranging} (\textit{@atLeast}, \textit{@atMost}, \textit{@min}, \textit{@max}, \textit{@confidence})) ) \hyperref[TEI.att.typed]{att.typed} (\textit{@type}, \textit{@subtype}) 
    \item[{Membre du}]
  \hyperref[TEI.model.nameLike.agent]{model.nameLike.agent} \hyperref[TEI.model.personPart]{model.personPart} 
    \item[{Contenu dans}]
  
    \item[analysis: ]
   \hyperref[TEI.cl]{cl} \hyperref[TEI.phr]{phr} \hyperref[TEI.s]{s} \hyperref[TEI.span]{span}\par 
    \item[core: ]
   \hyperref[TEI.abbr]{abbr} \hyperref[TEI.add]{add} \hyperref[TEI.addrLine]{addrLine} \hyperref[TEI.address]{address} \hyperref[TEI.author]{author} \hyperref[TEI.bibl]{bibl} \hyperref[TEI.biblScope]{biblScope} \hyperref[TEI.citedRange]{citedRange} \hyperref[TEI.corr]{corr} \hyperref[TEI.date]{date} \hyperref[TEI.del]{del} \hyperref[TEI.desc]{desc} \hyperref[TEI.distinct]{distinct} \hyperref[TEI.editor]{editor} \hyperref[TEI.email]{email} \hyperref[TEI.emph]{emph} \hyperref[TEI.expan]{expan} \hyperref[TEI.foreign]{foreign} \hyperref[TEI.gloss]{gloss} \hyperref[TEI.head]{head} \hyperref[TEI.headItem]{headItem} \hyperref[TEI.headLabel]{headLabel} \hyperref[TEI.hi]{hi} \hyperref[TEI.item]{item} \hyperref[TEI.l]{l} \hyperref[TEI.label]{label} \hyperref[TEI.measure]{measure} \hyperref[TEI.meeting]{meeting} \hyperref[TEI.mentioned]{mentioned} \hyperref[TEI.name]{name} \hyperref[TEI.note]{note} \hyperref[TEI.num]{num} \hyperref[TEI.orig]{orig} \hyperref[TEI.p]{p} \hyperref[TEI.pubPlace]{pubPlace} \hyperref[TEI.publisher]{publisher} \hyperref[TEI.q]{q} \hyperref[TEI.quote]{quote} \hyperref[TEI.ref]{ref} \hyperref[TEI.reg]{reg} \hyperref[TEI.resp]{resp} \hyperref[TEI.respStmt]{respStmt} \hyperref[TEI.rs]{rs} \hyperref[TEI.said]{said} \hyperref[TEI.sic]{sic} \hyperref[TEI.soCalled]{soCalled} \hyperref[TEI.speaker]{speaker} \hyperref[TEI.stage]{stage} \hyperref[TEI.street]{street} \hyperref[TEI.term]{term} \hyperref[TEI.textLang]{textLang} \hyperref[TEI.time]{time} \hyperref[TEI.title]{title} \hyperref[TEI.unclear]{unclear}\par 
    \item[figures: ]
   \hyperref[TEI.cell]{cell} \hyperref[TEI.figDesc]{figDesc}\par 
    \item[header: ]
   \hyperref[TEI.authority]{authority} \hyperref[TEI.change]{change} \hyperref[TEI.classCode]{classCode} \hyperref[TEI.creation]{creation} \hyperref[TEI.distributor]{distributor} \hyperref[TEI.edition]{edition} \hyperref[TEI.extent]{extent} \hyperref[TEI.funder]{funder} \hyperref[TEI.language]{language} \hyperref[TEI.licence]{licence} \hyperref[TEI.rendition]{rendition}\par 
    \item[iso-fs: ]
   \hyperref[TEI.fDescr]{fDescr} \hyperref[TEI.fsDescr]{fsDescr}\par 
    \item[linking: ]
   \hyperref[TEI.ab]{ab} \hyperref[TEI.seg]{seg}\par 
    \item[msdescription: ]
   \hyperref[TEI.accMat]{accMat} \hyperref[TEI.acquisition]{acquisition} \hyperref[TEI.additions]{additions} \hyperref[TEI.catchwords]{catchwords} \hyperref[TEI.collation]{collation} \hyperref[TEI.colophon]{colophon} \hyperref[TEI.condition]{condition} \hyperref[TEI.custEvent]{custEvent} \hyperref[TEI.decoNote]{decoNote} \hyperref[TEI.explicit]{explicit} \hyperref[TEI.filiation]{filiation} \hyperref[TEI.finalRubric]{finalRubric} \hyperref[TEI.foliation]{foliation} \hyperref[TEI.heraldry]{heraldry} \hyperref[TEI.incipit]{incipit} \hyperref[TEI.layout]{layout} \hyperref[TEI.material]{material} \hyperref[TEI.msName]{msName} \hyperref[TEI.musicNotation]{musicNotation} \hyperref[TEI.objectType]{objectType} \hyperref[TEI.origDate]{origDate} \hyperref[TEI.origPlace]{origPlace} \hyperref[TEI.origin]{origin} \hyperref[TEI.provenance]{provenance} \hyperref[TEI.rubric]{rubric} \hyperref[TEI.secFol]{secFol} \hyperref[TEI.signatures]{signatures} \hyperref[TEI.source]{source} \hyperref[TEI.stamp]{stamp} \hyperref[TEI.summary]{summary} \hyperref[TEI.support]{support} \hyperref[TEI.surrogates]{surrogates} \hyperref[TEI.typeNote]{typeNote} \hyperref[TEI.watermark]{watermark}\par 
    \item[namesdates: ]
   \hyperref[TEI.addName]{addName} \hyperref[TEI.affiliation]{affiliation} \hyperref[TEI.country]{country} \hyperref[TEI.forename]{forename} \hyperref[TEI.genName]{genName} \hyperref[TEI.geogName]{geogName} \hyperref[TEI.nameLink]{nameLink} \hyperref[TEI.org]{org} \hyperref[TEI.orgName]{orgName} \hyperref[TEI.persName]{persName} \hyperref[TEI.person]{person} \hyperref[TEI.personGrp]{personGrp} \hyperref[TEI.persona]{persona} \hyperref[TEI.placeName]{placeName} \hyperref[TEI.region]{region} \hyperref[TEI.roleName]{roleName} \hyperref[TEI.settlement]{settlement} \hyperref[TEI.surname]{surname}\par 
    \item[spoken: ]
   \hyperref[TEI.annotationBlock]{annotationBlock}\par 
    \item[standOff: ]
   \hyperref[TEI.listAnnotation]{listAnnotation}\par 
    \item[textstructure: ]
   \hyperref[TEI.docAuthor]{docAuthor} \hyperref[TEI.docDate]{docDate} \hyperref[TEI.docEdition]{docEdition} \hyperref[TEI.titlePart]{titlePart}\par 
    \item[transcr: ]
   \hyperref[TEI.damage]{damage} \hyperref[TEI.fw]{fw} \hyperref[TEI.metamark]{metamark} \hyperref[TEI.mod]{mod} \hyperref[TEI.restore]{restore} \hyperref[TEI.retrace]{retrace} \hyperref[TEI.secl]{secl} \hyperref[TEI.supplied]{supplied} \hyperref[TEI.surplus]{surplus}
    \item[{Peut contenir}]
  
    \item[analysis: ]
   \hyperref[TEI.c]{c} \hyperref[TEI.cl]{cl} \hyperref[TEI.interp]{interp} \hyperref[TEI.interpGrp]{interpGrp} \hyperref[TEI.m]{m} \hyperref[TEI.pc]{pc} \hyperref[TEI.phr]{phr} \hyperref[TEI.s]{s} \hyperref[TEI.span]{span} \hyperref[TEI.spanGrp]{spanGrp} \hyperref[TEI.w]{w}\par 
    \item[core: ]
   \hyperref[TEI.abbr]{abbr} \hyperref[TEI.add]{add} \hyperref[TEI.address]{address} \hyperref[TEI.binaryObject]{binaryObject} \hyperref[TEI.cb]{cb} \hyperref[TEI.choice]{choice} \hyperref[TEI.corr]{corr} \hyperref[TEI.date]{date} \hyperref[TEI.del]{del} \hyperref[TEI.distinct]{distinct} \hyperref[TEI.email]{email} \hyperref[TEI.emph]{emph} \hyperref[TEI.expan]{expan} \hyperref[TEI.foreign]{foreign} \hyperref[TEI.gap]{gap} \hyperref[TEI.gb]{gb} \hyperref[TEI.gloss]{gloss} \hyperref[TEI.graphic]{graphic} \hyperref[TEI.hi]{hi} \hyperref[TEI.index]{index} \hyperref[TEI.lb]{lb} \hyperref[TEI.measure]{measure} \hyperref[TEI.measureGrp]{measureGrp} \hyperref[TEI.media]{media} \hyperref[TEI.mentioned]{mentioned} \hyperref[TEI.milestone]{milestone} \hyperref[TEI.name]{name} \hyperref[TEI.note]{note} \hyperref[TEI.num]{num} \hyperref[TEI.orig]{orig} \hyperref[TEI.pb]{pb} \hyperref[TEI.ptr]{ptr} \hyperref[TEI.ref]{ref} \hyperref[TEI.reg]{reg} \hyperref[TEI.rs]{rs} \hyperref[TEI.sic]{sic} \hyperref[TEI.soCalled]{soCalled} \hyperref[TEI.term]{term} \hyperref[TEI.time]{time} \hyperref[TEI.title]{title} \hyperref[TEI.unclear]{unclear}\par 
    \item[derived-module-tei.istex: ]
   \hyperref[TEI.math]{math} \hyperref[TEI.mrow]{mrow}\par 
    \item[figures: ]
   \hyperref[TEI.figure]{figure} \hyperref[TEI.formula]{formula} \hyperref[TEI.notatedMusic]{notatedMusic}\par 
    \item[header: ]
   \hyperref[TEI.idno]{idno}\par 
    \item[iso-fs: ]
   \hyperref[TEI.fLib]{fLib} \hyperref[TEI.fs]{fs} \hyperref[TEI.fvLib]{fvLib}\par 
    \item[linking: ]
   \hyperref[TEI.alt]{alt} \hyperref[TEI.altGrp]{altGrp} \hyperref[TEI.anchor]{anchor} \hyperref[TEI.join]{join} \hyperref[TEI.joinGrp]{joinGrp} \hyperref[TEI.link]{link} \hyperref[TEI.linkGrp]{linkGrp} \hyperref[TEI.seg]{seg} \hyperref[TEI.timeline]{timeline}\par 
    \item[msdescription: ]
   \hyperref[TEI.catchwords]{catchwords} \hyperref[TEI.depth]{depth} \hyperref[TEI.dim]{dim} \hyperref[TEI.dimensions]{dimensions} \hyperref[TEI.height]{height} \hyperref[TEI.heraldry]{heraldry} \hyperref[TEI.locus]{locus} \hyperref[TEI.locusGrp]{locusGrp} \hyperref[TEI.material]{material} \hyperref[TEI.objectType]{objectType} \hyperref[TEI.origDate]{origDate} \hyperref[TEI.origPlace]{origPlace} \hyperref[TEI.secFol]{secFol} \hyperref[TEI.signatures]{signatures} \hyperref[TEI.source]{source} \hyperref[TEI.stamp]{stamp} \hyperref[TEI.watermark]{watermark} \hyperref[TEI.width]{width}\par 
    \item[namesdates: ]
   \hyperref[TEI.addName]{addName} \hyperref[TEI.affiliation]{affiliation} \hyperref[TEI.country]{country} \hyperref[TEI.forename]{forename} \hyperref[TEI.genName]{genName} \hyperref[TEI.geogName]{geogName} \hyperref[TEI.location]{location} \hyperref[TEI.nameLink]{nameLink} \hyperref[TEI.orgName]{orgName} \hyperref[TEI.persName]{persName} \hyperref[TEI.placeName]{placeName} \hyperref[TEI.region]{region} \hyperref[TEI.roleName]{roleName} \hyperref[TEI.settlement]{settlement} \hyperref[TEI.state]{state} \hyperref[TEI.surname]{surname}\par 
    \item[spoken: ]
   \hyperref[TEI.annotationBlock]{annotationBlock}\par 
    \item[transcr: ]
   \hyperref[TEI.addSpan]{addSpan} \hyperref[TEI.am]{am} \hyperref[TEI.damage]{damage} \hyperref[TEI.damageSpan]{damageSpan} \hyperref[TEI.delSpan]{delSpan} \hyperref[TEI.ex]{ex} \hyperref[TEI.fw]{fw} \hyperref[TEI.handShift]{handShift} \hyperref[TEI.listTranspose]{listTranspose} \hyperref[TEI.metamark]{metamark} \hyperref[TEI.mod]{mod} \hyperref[TEI.redo]{redo} \hyperref[TEI.restore]{restore} \hyperref[TEI.retrace]{retrace} \hyperref[TEI.secl]{secl} \hyperref[TEI.space]{space} \hyperref[TEI.subst]{subst} \hyperref[TEI.substJoin]{substJoin} \hyperref[TEI.supplied]{supplied} \hyperref[TEI.surplus]{surplus} \hyperref[TEI.undo]{undo}\par des données textuelles
    \item[{Note}]
  \par
Les noms propres relatifs aux personnes, aux lieux et aux organismes peuvent également être balisés à l'aide de \hyperref[TEI.persName]{<persName>}, \hyperref[TEI.placeName]{<placeName>}, ou \hyperref[TEI.orgName]{<orgName>}, lorsque le module TEI concernant les noms et dates est inclus.
    \item[{Exemple}]
  \leavevmode\bgroup\exampleFont \begin{shaded}\noindent\mbox{}{<\textbf{name}\hspace*{6pt}{type}="{person}">}Thomas Hoccleve{</\textbf{name}>}\mbox{}\newline 
{<\textbf{name}\hspace*{6pt}{type}="{place}">}Villingaholt{</\textbf{name}>}\mbox{}\newline 
{<\textbf{name}\hspace*{6pt}{type}="{org}">}Vetus Latina Institut{</\textbf{name}>}\mbox{}\newline 
{<\textbf{name}\hspace*{6pt}{ref}="{\#HOC001}"\hspace*{6pt}{type}="{person}">}Occleve{</\textbf{name}>}\end{shaded}\egroup 


    \item[{Modèle de contenu}]
  \mbox{}\hfill\\[-10pt]\begin{Verbatim}[fontsize=\small]
<content>
 <macroRef key="macro.phraseSeq"/>
</content>
    
\end{Verbatim}

    \item[{Schéma Declaration}]
  \mbox{}\hfill\\[-10pt]\begin{Verbatim}[fontsize=\small]
element name
{
   tei_att.global.attributes,
   tei_att.personal.attributes,
   tei_att.datable.attributes,
   tei_att.editLike.attributes,
   tei_att.typed.attributes,
   tei_macro.phraseSeq}
\end{Verbatim}

\end{reflist}  \index{nameLink=<nameLink>|oddindex}
\begin{reflist}
\item[]\begin{specHead}{TEI.nameLink}{<nameLink> }(lien entre les composants d'un nom) contient une particule ou une expression exprimant un lien, utilisés dans un nom mais considérés comme n'en faisant pas partie, comme "van der" ou "de". [\xref{http://www.tei-c.org/release/doc/tei-p5-doc/en/html/ND.html\#NDPER}{13.2.1. Personal Names}]\end{specHead} 
    \item[{Module}]
  namesdates
    \item[{Attributs}]
  Attributs \hyperref[TEI.att.global]{att.global} (\textit{@xml:id}, \textit{@n}, \textit{@xml:lang}, \textit{@xml:base}, \textit{@xml:space})  (\hyperref[TEI.att.global.rendition]{att.global.rendition} (\textit{@rend}, \textit{@style}, \textit{@rendition})) (\hyperref[TEI.att.global.linking]{att.global.linking} (\textit{@corresp}, \textit{@synch}, \textit{@sameAs}, \textit{@copyOf}, \textit{@next}, \textit{@prev}, \textit{@exclude}, \textit{@select})) (\hyperref[TEI.att.global.analytic]{att.global.analytic} (\textit{@ana})) (\hyperref[TEI.att.global.facs]{att.global.facs} (\textit{@facs})) (\hyperref[TEI.att.global.change]{att.global.change} (\textit{@change})) (\hyperref[TEI.att.global.responsibility]{att.global.responsibility} (\textit{@cert}, \textit{@resp})) (\hyperref[TEI.att.global.source]{att.global.source} (\textit{@source})) \hyperref[TEI.att.typed]{att.typed} (\textit{@type}, \textit{@subtype}) 
    \item[{Membre du}]
  \hyperref[TEI.model.persNamePart]{model.persNamePart}
    \item[{Contenu dans}]
  
    \item[analysis: ]
   \hyperref[TEI.cl]{cl} \hyperref[TEI.phr]{phr} \hyperref[TEI.s]{s} \hyperref[TEI.span]{span}\par 
    \item[core: ]
   \hyperref[TEI.abbr]{abbr} \hyperref[TEI.add]{add} \hyperref[TEI.addrLine]{addrLine} \hyperref[TEI.address]{address} \hyperref[TEI.author]{author} \hyperref[TEI.bibl]{bibl} \hyperref[TEI.biblScope]{biblScope} \hyperref[TEI.citedRange]{citedRange} \hyperref[TEI.corr]{corr} \hyperref[TEI.date]{date} \hyperref[TEI.del]{del} \hyperref[TEI.desc]{desc} \hyperref[TEI.distinct]{distinct} \hyperref[TEI.editor]{editor} \hyperref[TEI.email]{email} \hyperref[TEI.emph]{emph} \hyperref[TEI.expan]{expan} \hyperref[TEI.foreign]{foreign} \hyperref[TEI.gloss]{gloss} \hyperref[TEI.head]{head} \hyperref[TEI.headItem]{headItem} \hyperref[TEI.headLabel]{headLabel} \hyperref[TEI.hi]{hi} \hyperref[TEI.item]{item} \hyperref[TEI.l]{l} \hyperref[TEI.label]{label} \hyperref[TEI.measure]{measure} \hyperref[TEI.meeting]{meeting} \hyperref[TEI.mentioned]{mentioned} \hyperref[TEI.name]{name} \hyperref[TEI.note]{note} \hyperref[TEI.num]{num} \hyperref[TEI.orig]{orig} \hyperref[TEI.p]{p} \hyperref[TEI.pubPlace]{pubPlace} \hyperref[TEI.publisher]{publisher} \hyperref[TEI.q]{q} \hyperref[TEI.quote]{quote} \hyperref[TEI.ref]{ref} \hyperref[TEI.reg]{reg} \hyperref[TEI.resp]{resp} \hyperref[TEI.rs]{rs} \hyperref[TEI.said]{said} \hyperref[TEI.sic]{sic} \hyperref[TEI.soCalled]{soCalled} \hyperref[TEI.speaker]{speaker} \hyperref[TEI.stage]{stage} \hyperref[TEI.street]{street} \hyperref[TEI.term]{term} \hyperref[TEI.textLang]{textLang} \hyperref[TEI.time]{time} \hyperref[TEI.title]{title} \hyperref[TEI.unclear]{unclear}\par 
    \item[figures: ]
   \hyperref[TEI.cell]{cell} \hyperref[TEI.figDesc]{figDesc}\par 
    \item[header: ]
   \hyperref[TEI.authority]{authority} \hyperref[TEI.change]{change} \hyperref[TEI.classCode]{classCode} \hyperref[TEI.creation]{creation} \hyperref[TEI.distributor]{distributor} \hyperref[TEI.edition]{edition} \hyperref[TEI.extent]{extent} \hyperref[TEI.funder]{funder} \hyperref[TEI.language]{language} \hyperref[TEI.licence]{licence} \hyperref[TEI.rendition]{rendition}\par 
    \item[iso-fs: ]
   \hyperref[TEI.fDescr]{fDescr} \hyperref[TEI.fsDescr]{fsDescr}\par 
    \item[linking: ]
   \hyperref[TEI.ab]{ab} \hyperref[TEI.seg]{seg}\par 
    \item[msdescription: ]
   \hyperref[TEI.accMat]{accMat} \hyperref[TEI.acquisition]{acquisition} \hyperref[TEI.additions]{additions} \hyperref[TEI.catchwords]{catchwords} \hyperref[TEI.collation]{collation} \hyperref[TEI.colophon]{colophon} \hyperref[TEI.condition]{condition} \hyperref[TEI.custEvent]{custEvent} \hyperref[TEI.decoNote]{decoNote} \hyperref[TEI.explicit]{explicit} \hyperref[TEI.filiation]{filiation} \hyperref[TEI.finalRubric]{finalRubric} \hyperref[TEI.foliation]{foliation} \hyperref[TEI.heraldry]{heraldry} \hyperref[TEI.incipit]{incipit} \hyperref[TEI.layout]{layout} \hyperref[TEI.material]{material} \hyperref[TEI.musicNotation]{musicNotation} \hyperref[TEI.objectType]{objectType} \hyperref[TEI.origDate]{origDate} \hyperref[TEI.origPlace]{origPlace} \hyperref[TEI.origin]{origin} \hyperref[TEI.provenance]{provenance} \hyperref[TEI.rubric]{rubric} \hyperref[TEI.secFol]{secFol} \hyperref[TEI.signatures]{signatures} \hyperref[TEI.source]{source} \hyperref[TEI.stamp]{stamp} \hyperref[TEI.summary]{summary} \hyperref[TEI.support]{support} \hyperref[TEI.surrogates]{surrogates} \hyperref[TEI.typeNote]{typeNote} \hyperref[TEI.watermark]{watermark}\par 
    \item[namesdates: ]
   \hyperref[TEI.addName]{addName} \hyperref[TEI.affiliation]{affiliation} \hyperref[TEI.country]{country} \hyperref[TEI.forename]{forename} \hyperref[TEI.genName]{genName} \hyperref[TEI.geogName]{geogName} \hyperref[TEI.nameLink]{nameLink} \hyperref[TEI.org]{org} \hyperref[TEI.orgName]{orgName} \hyperref[TEI.persName]{persName} \hyperref[TEI.placeName]{placeName} \hyperref[TEI.region]{region} \hyperref[TEI.roleName]{roleName} \hyperref[TEI.settlement]{settlement} \hyperref[TEI.surname]{surname}\par 
    \item[spoken: ]
   \hyperref[TEI.annotationBlock]{annotationBlock}\par 
    \item[standOff: ]
   \hyperref[TEI.listAnnotation]{listAnnotation}\par 
    \item[textstructure: ]
   \hyperref[TEI.docAuthor]{docAuthor} \hyperref[TEI.docDate]{docDate} \hyperref[TEI.docEdition]{docEdition} \hyperref[TEI.titlePart]{titlePart}\par 
    \item[transcr: ]
   \hyperref[TEI.damage]{damage} \hyperref[TEI.fw]{fw} \hyperref[TEI.metamark]{metamark} \hyperref[TEI.mod]{mod} \hyperref[TEI.restore]{restore} \hyperref[TEI.retrace]{retrace} \hyperref[TEI.secl]{secl} \hyperref[TEI.supplied]{supplied} \hyperref[TEI.surplus]{surplus}
    \item[{Peut contenir}]
  
    \item[analysis: ]
   \hyperref[TEI.c]{c} \hyperref[TEI.cl]{cl} \hyperref[TEI.interp]{interp} \hyperref[TEI.interpGrp]{interpGrp} \hyperref[TEI.m]{m} \hyperref[TEI.pc]{pc} \hyperref[TEI.phr]{phr} \hyperref[TEI.s]{s} \hyperref[TEI.span]{span} \hyperref[TEI.spanGrp]{spanGrp} \hyperref[TEI.w]{w}\par 
    \item[core: ]
   \hyperref[TEI.abbr]{abbr} \hyperref[TEI.add]{add} \hyperref[TEI.address]{address} \hyperref[TEI.binaryObject]{binaryObject} \hyperref[TEI.cb]{cb} \hyperref[TEI.choice]{choice} \hyperref[TEI.corr]{corr} \hyperref[TEI.date]{date} \hyperref[TEI.del]{del} \hyperref[TEI.distinct]{distinct} \hyperref[TEI.email]{email} \hyperref[TEI.emph]{emph} \hyperref[TEI.expan]{expan} \hyperref[TEI.foreign]{foreign} \hyperref[TEI.gap]{gap} \hyperref[TEI.gb]{gb} \hyperref[TEI.gloss]{gloss} \hyperref[TEI.graphic]{graphic} \hyperref[TEI.hi]{hi} \hyperref[TEI.index]{index} \hyperref[TEI.lb]{lb} \hyperref[TEI.measure]{measure} \hyperref[TEI.measureGrp]{measureGrp} \hyperref[TEI.media]{media} \hyperref[TEI.mentioned]{mentioned} \hyperref[TEI.milestone]{milestone} \hyperref[TEI.name]{name} \hyperref[TEI.note]{note} \hyperref[TEI.num]{num} \hyperref[TEI.orig]{orig} \hyperref[TEI.pb]{pb} \hyperref[TEI.ptr]{ptr} \hyperref[TEI.ref]{ref} \hyperref[TEI.reg]{reg} \hyperref[TEI.rs]{rs} \hyperref[TEI.sic]{sic} \hyperref[TEI.soCalled]{soCalled} \hyperref[TEI.term]{term} \hyperref[TEI.time]{time} \hyperref[TEI.title]{title} \hyperref[TEI.unclear]{unclear}\par 
    \item[derived-module-tei.istex: ]
   \hyperref[TEI.math]{math} \hyperref[TEI.mrow]{mrow}\par 
    \item[figures: ]
   \hyperref[TEI.figure]{figure} \hyperref[TEI.formula]{formula} \hyperref[TEI.notatedMusic]{notatedMusic}\par 
    \item[header: ]
   \hyperref[TEI.idno]{idno}\par 
    \item[iso-fs: ]
   \hyperref[TEI.fLib]{fLib} \hyperref[TEI.fs]{fs} \hyperref[TEI.fvLib]{fvLib}\par 
    \item[linking: ]
   \hyperref[TEI.alt]{alt} \hyperref[TEI.altGrp]{altGrp} \hyperref[TEI.anchor]{anchor} \hyperref[TEI.join]{join} \hyperref[TEI.joinGrp]{joinGrp} \hyperref[TEI.link]{link} \hyperref[TEI.linkGrp]{linkGrp} \hyperref[TEI.seg]{seg} \hyperref[TEI.timeline]{timeline}\par 
    \item[msdescription: ]
   \hyperref[TEI.catchwords]{catchwords} \hyperref[TEI.depth]{depth} \hyperref[TEI.dim]{dim} \hyperref[TEI.dimensions]{dimensions} \hyperref[TEI.height]{height} \hyperref[TEI.heraldry]{heraldry} \hyperref[TEI.locus]{locus} \hyperref[TEI.locusGrp]{locusGrp} \hyperref[TEI.material]{material} \hyperref[TEI.objectType]{objectType} \hyperref[TEI.origDate]{origDate} \hyperref[TEI.origPlace]{origPlace} \hyperref[TEI.secFol]{secFol} \hyperref[TEI.signatures]{signatures} \hyperref[TEI.source]{source} \hyperref[TEI.stamp]{stamp} \hyperref[TEI.watermark]{watermark} \hyperref[TEI.width]{width}\par 
    \item[namesdates: ]
   \hyperref[TEI.addName]{addName} \hyperref[TEI.affiliation]{affiliation} \hyperref[TEI.country]{country} \hyperref[TEI.forename]{forename} \hyperref[TEI.genName]{genName} \hyperref[TEI.geogName]{geogName} \hyperref[TEI.location]{location} \hyperref[TEI.nameLink]{nameLink} \hyperref[TEI.orgName]{orgName} \hyperref[TEI.persName]{persName} \hyperref[TEI.placeName]{placeName} \hyperref[TEI.region]{region} \hyperref[TEI.roleName]{roleName} \hyperref[TEI.settlement]{settlement} \hyperref[TEI.state]{state} \hyperref[TEI.surname]{surname}\par 
    \item[spoken: ]
   \hyperref[TEI.annotationBlock]{annotationBlock}\par 
    \item[transcr: ]
   \hyperref[TEI.addSpan]{addSpan} \hyperref[TEI.am]{am} \hyperref[TEI.damage]{damage} \hyperref[TEI.damageSpan]{damageSpan} \hyperref[TEI.delSpan]{delSpan} \hyperref[TEI.ex]{ex} \hyperref[TEI.fw]{fw} \hyperref[TEI.handShift]{handShift} \hyperref[TEI.listTranspose]{listTranspose} \hyperref[TEI.metamark]{metamark} \hyperref[TEI.mod]{mod} \hyperref[TEI.redo]{redo} \hyperref[TEI.restore]{restore} \hyperref[TEI.retrace]{retrace} \hyperref[TEI.secl]{secl} \hyperref[TEI.space]{space} \hyperref[TEI.subst]{subst} \hyperref[TEI.substJoin]{substJoin} \hyperref[TEI.supplied]{supplied} \hyperref[TEI.surplus]{surplus} \hyperref[TEI.undo]{undo}\par des données textuelles
    \item[{Exemple}]
  \leavevmode\bgroup\exampleFont \begin{shaded}\noindent\mbox{}{<\textbf{persName}>}\mbox{}\newline 
\hspace*{6pt}{<\textbf{forename}>}Frederick{</\textbf{forename}>}\mbox{}\newline 
\hspace*{6pt}{<\textbf{nameLink}>}van der{</\textbf{nameLink}>}\mbox{}\newline 
\hspace*{6pt}{<\textbf{surname}>}Tronck{</\textbf{surname}>}\mbox{}\newline 
{</\textbf{persName}>}\end{shaded}\egroup 


    \item[{Exemple}]
  \leavevmode\bgroup\exampleFont \begin{shaded}\noindent\mbox{}{<\textbf{persName}>}\mbox{}\newline 
\hspace*{6pt}{<\textbf{forename}>}Frederick{</\textbf{forename}>}\mbox{}\newline 
\hspace*{6pt}{<\textbf{nameLink}>}van der{</\textbf{nameLink}>}\mbox{}\newline 
\hspace*{6pt}{<\textbf{surname}>}Tronck{</\textbf{surname}>}\mbox{}\newline 
{</\textbf{persName}>}\end{shaded}\egroup 


    \item[{Exemple}]
  \leavevmode\bgroup\exampleFont \begin{shaded}\noindent\mbox{}{<\textbf{persName}>}\mbox{}\newline 
\hspace*{6pt}{<\textbf{forename}>}Alfred{</\textbf{forename}>}\mbox{}\newline 
\hspace*{6pt}{<\textbf{nameLink}>}de{</\textbf{nameLink}>}\mbox{}\newline 
\hspace*{6pt}{<\textbf{surname}>}Musset{</\textbf{surname}>}\mbox{}\newline 
{</\textbf{persName}>}\end{shaded}\egroup 


    \item[{Exemple}]
  \leavevmode\bgroup\exampleFont \begin{shaded}\noindent\mbox{}{<\textbf{persName}>}\mbox{}\newline 
\hspace*{6pt}{<\textbf{forename}>}Alfred{</\textbf{forename}>}\mbox{}\newline 
\hspace*{6pt}{<\textbf{nameLink}>}de{</\textbf{nameLink}>}\mbox{}\newline 
\hspace*{6pt}{<\textbf{surname}>}Musset{</\textbf{surname}>}\mbox{}\newline 
{</\textbf{persName}>}\end{shaded}\egroup 


    \item[{Modèle de contenu}]
  \mbox{}\hfill\\[-10pt]\begin{Verbatim}[fontsize=\small]
<content>
 <macroRef key="macro.phraseSeq"/>
</content>
    
\end{Verbatim}

    \item[{Schéma Declaration}]
  \mbox{}\hfill\\[-10pt]\begin{Verbatim}[fontsize=\small]
element nameLink
{
   tei_att.global.attributes,
   tei_att.typed.attributes,
   tei_macro.phraseSeq}
\end{Verbatim}

\end{reflist}  \index{namespace=<namespace>|oddindex}\index{name=@name!<namespace>|oddindex}
\begin{reflist}
\item[]\begin{specHead}{TEI.namespace}{<namespace> }(espace de noms) fournit le nom formel de l'espace de noms auquel appartiennent les éléments documentés par ses éléments fils. [\xref{http://www.tei-c.org/release/doc/tei-p5-doc/en/html/HD.html\#HD57}{2.3.4. The Tagging Declaration}]\end{specHead} 
    \item[{Module}]
  header
    \item[{Attributs}]
  Attributs \hyperref[TEI.att.global]{att.global} (\textit{@xml:id}, \textit{@n}, \textit{@xml:lang}, \textit{@xml:base}, \textit{@xml:space})  (\hyperref[TEI.att.global.rendition]{att.global.rendition} (\textit{@rend}, \textit{@style}, \textit{@rendition})) (\hyperref[TEI.att.global.linking]{att.global.linking} (\textit{@corresp}, \textit{@synch}, \textit{@sameAs}, \textit{@copyOf}, \textit{@next}, \textit{@prev}, \textit{@exclude}, \textit{@select})) (\hyperref[TEI.att.global.analytic]{att.global.analytic} (\textit{@ana})) (\hyperref[TEI.att.global.facs]{att.global.facs} (\textit{@facs})) (\hyperref[TEI.att.global.change]{att.global.change} (\textit{@change})) (\hyperref[TEI.att.global.responsibility]{att.global.responsibility} (\textit{@cert}, \textit{@resp})) (\hyperref[TEI.att.global.source]{att.global.source} (\textit{@source})) \hfil\\[-10pt]\begin{sansreflist}
    \item[@name]
  le nom formel complet de l'espace de noms concerné.
\begin{reflist}
    \item[{Statut}]
  Requis
    \item[{Type de données}]
  \hyperref[TEI.teidata.namespace]{teidata.namespace}
\end{reflist}  
\end{sansreflist}  
    \item[{Contenu dans}]
  —
    \item[{Peut contenir}]
  Elément vide
    \item[{Exemple}]
  \leavevmode\bgroup\exampleFont \begin{shaded}\noindent\mbox{}{<\textbf{namespace}\hspace*{6pt}{name}="{http://www.tei-c.org/ns/1.0}">}\mbox{}\newline 
\hspace*{6pt}{<\textbf{tagUsage}\hspace*{6pt}{gi}="{foreign}">}Employé pour marquer des mots non-français dans le\mbox{}\newline 
\hspace*{6pt}\hspace*{6pt} texte.{</\textbf{tagUsage}>}\mbox{}\newline 
{</\textbf{namespace}>}\end{shaded}\egroup 


    \item[{Modèle de contenu}]
  \mbox{}\hfill\\[-10pt]\begin{Verbatim}[fontsize=\small]
<content>
 <elementRef key="tagUsage"
  maxOccurs="unbounded" minOccurs="1"/>
</content>
    
\end{Verbatim}

    \item[{Schéma Declaration}]
  \mbox{}\hfill\\[-10pt]\begin{Verbatim}[fontsize=\small]
element namespace
{
   tei_att.global.attributes,
   attribute name { text },
   tagUsage+
}
\end{Verbatim}

\end{reflist}  \index{none=<none>|oddindex}
\begin{reflist}
\item[]\begin{specHead}{TEI.none}{<none> }\end{specHead} 
    \item[{Namespace}]
  http://www.w3.org/1998/Math/MathML
    \item[{Module}]
  derived-module-tei.istex
    \item[{Contenu dans}]
  
    \item[derived-module-tei.istex: ]
   \hyperref[TEI.math]{math} \hyperref[TEI.menclose]{menclose} \hyperref[TEI.mfenced]{mfenced} \hyperref[TEI.mfrac]{mfrac} \hyperref[TEI.mmultiscripts]{mmultiscripts} \hyperref[TEI.mover]{mover} \hyperref[TEI.mpadded]{mpadded} \hyperref[TEI.mphantom]{mphantom} \hyperref[TEI.mprescripts]{mprescripts} \hyperref[TEI.mrow]{mrow} \hyperref[TEI.msqrt]{msqrt} \hyperref[TEI.mstyle]{mstyle} \hyperref[TEI.msub]{msub} \hyperref[TEI.msubsup]{msubsup} \hyperref[TEI.msup]{msup} \hyperref[TEI.msupsub]{msupsub} \hyperref[TEI.mtable]{mtable} \hyperref[TEI.mtd]{mtd} \hyperref[TEI.mtr]{mtr} \hyperref[TEI.munder]{munder} \hyperref[TEI.munderover]{munderover} \hyperref[TEI.semantics]{semantics}
    \item[{Peut contenir}]
  Des données textuelles uniquement
    \item[{Modèle de contenu}]
  \fbox{\ttfamily <content>\newline
 <textNode/>\newline
</content>\newline
    } 
    \item[{Schéma Declaration}]
  \fbox{\ttfamily element none ❴ text ❵} 
\end{reflist}  \index{notatedMusic=<notatedMusic>|oddindex}
\begin{reflist}
\item[]\begin{specHead}{TEI.notatedMusic}{<notatedMusic> }encodes the presence of music notation in a text [\xref{http://www.tei-c.org/release/doc/tei-p5-doc/en/html/FT.html\#FTNM}{14.3. Notated Music in Written Text}]\end{specHead} 
    \item[{Module}]
  figures
    \item[{Attributs}]
  Attributs \hyperref[TEI.att.global]{att.global} (\textit{@xml:id}, \textit{@n}, \textit{@xml:lang}, \textit{@xml:base}, \textit{@xml:space})  (\hyperref[TEI.att.global.rendition]{att.global.rendition} (\textit{@rend}, \textit{@style}, \textit{@rendition})) (\hyperref[TEI.att.global.linking]{att.global.linking} (\textit{@corresp}, \textit{@synch}, \textit{@sameAs}, \textit{@copyOf}, \textit{@next}, \textit{@prev}, \textit{@exclude}, \textit{@select})) (\hyperref[TEI.att.global.analytic]{att.global.analytic} (\textit{@ana})) (\hyperref[TEI.att.global.facs]{att.global.facs} (\textit{@facs})) (\hyperref[TEI.att.global.change]{att.global.change} (\textit{@change})) (\hyperref[TEI.att.global.responsibility]{att.global.responsibility} (\textit{@cert}, \textit{@resp})) (\hyperref[TEI.att.global.source]{att.global.source} (\textit{@source})) \hyperref[TEI.att.placement]{att.placement} (\textit{@place}) \hyperref[TEI.att.typed]{att.typed} (\textit{@type}, \textit{@subtype}) 
    \item[{Membre du}]
  \hyperref[TEI.model.global]{model.global}
    \item[{Contenu dans}]
  
    \item[analysis: ]
   \hyperref[TEI.cl]{cl} \hyperref[TEI.m]{m} \hyperref[TEI.phr]{phr} \hyperref[TEI.s]{s} \hyperref[TEI.span]{span} \hyperref[TEI.w]{w}\par 
    \item[core: ]
   \hyperref[TEI.abbr]{abbr} \hyperref[TEI.add]{add} \hyperref[TEI.addrLine]{addrLine} \hyperref[TEI.address]{address} \hyperref[TEI.author]{author} \hyperref[TEI.bibl]{bibl} \hyperref[TEI.biblScope]{biblScope} \hyperref[TEI.cit]{cit} \hyperref[TEI.citedRange]{citedRange} \hyperref[TEI.corr]{corr} \hyperref[TEI.date]{date} \hyperref[TEI.del]{del} \hyperref[TEI.distinct]{distinct} \hyperref[TEI.editor]{editor} \hyperref[TEI.email]{email} \hyperref[TEI.emph]{emph} \hyperref[TEI.expan]{expan} \hyperref[TEI.foreign]{foreign} \hyperref[TEI.gloss]{gloss} \hyperref[TEI.head]{head} \hyperref[TEI.headItem]{headItem} \hyperref[TEI.headLabel]{headLabel} \hyperref[TEI.hi]{hi} \hyperref[TEI.imprint]{imprint} \hyperref[TEI.item]{item} \hyperref[TEI.l]{l} \hyperref[TEI.label]{label} \hyperref[TEI.lg]{lg} \hyperref[TEI.list]{list} \hyperref[TEI.measure]{measure} \hyperref[TEI.mentioned]{mentioned} \hyperref[TEI.name]{name} \hyperref[TEI.note]{note} \hyperref[TEI.num]{num} \hyperref[TEI.orig]{orig} \hyperref[TEI.p]{p} \hyperref[TEI.pubPlace]{pubPlace} \hyperref[TEI.publisher]{publisher} \hyperref[TEI.q]{q} \hyperref[TEI.quote]{quote} \hyperref[TEI.ref]{ref} \hyperref[TEI.reg]{reg} \hyperref[TEI.resp]{resp} \hyperref[TEI.rs]{rs} \hyperref[TEI.said]{said} \hyperref[TEI.series]{series} \hyperref[TEI.sic]{sic} \hyperref[TEI.soCalled]{soCalled} \hyperref[TEI.sp]{sp} \hyperref[TEI.speaker]{speaker} \hyperref[TEI.stage]{stage} \hyperref[TEI.street]{street} \hyperref[TEI.term]{term} \hyperref[TEI.textLang]{textLang} \hyperref[TEI.time]{time} \hyperref[TEI.title]{title} \hyperref[TEI.unclear]{unclear}\par 
    \item[figures: ]
   \hyperref[TEI.cell]{cell} \hyperref[TEI.figure]{figure} \hyperref[TEI.table]{table}\par 
    \item[header: ]
   \hyperref[TEI.authority]{authority} \hyperref[TEI.change]{change} \hyperref[TEI.classCode]{classCode} \hyperref[TEI.distributor]{distributor} \hyperref[TEI.edition]{edition} \hyperref[TEI.extent]{extent} \hyperref[TEI.funder]{funder} \hyperref[TEI.language]{language} \hyperref[TEI.licence]{licence}\par 
    \item[linking: ]
   \hyperref[TEI.ab]{ab} \hyperref[TEI.seg]{seg}\par 
    \item[msdescription: ]
   \hyperref[TEI.accMat]{accMat} \hyperref[TEI.acquisition]{acquisition} \hyperref[TEI.additions]{additions} \hyperref[TEI.catchwords]{catchwords} \hyperref[TEI.collation]{collation} \hyperref[TEI.colophon]{colophon} \hyperref[TEI.condition]{condition} \hyperref[TEI.custEvent]{custEvent} \hyperref[TEI.decoNote]{decoNote} \hyperref[TEI.explicit]{explicit} \hyperref[TEI.filiation]{filiation} \hyperref[TEI.finalRubric]{finalRubric} \hyperref[TEI.foliation]{foliation} \hyperref[TEI.heraldry]{heraldry} \hyperref[TEI.incipit]{incipit} \hyperref[TEI.layout]{layout} \hyperref[TEI.material]{material} \hyperref[TEI.msItem]{msItem} \hyperref[TEI.musicNotation]{musicNotation} \hyperref[TEI.objectType]{objectType} \hyperref[TEI.origDate]{origDate} \hyperref[TEI.origPlace]{origPlace} \hyperref[TEI.origin]{origin} \hyperref[TEI.provenance]{provenance} \hyperref[TEI.rubric]{rubric} \hyperref[TEI.secFol]{secFol} \hyperref[TEI.signatures]{signatures} \hyperref[TEI.source]{source} \hyperref[TEI.stamp]{stamp} \hyperref[TEI.summary]{summary} \hyperref[TEI.support]{support} \hyperref[TEI.surrogates]{surrogates} \hyperref[TEI.typeNote]{typeNote} \hyperref[TEI.watermark]{watermark}\par 
    \item[namesdates: ]
   \hyperref[TEI.addName]{addName} \hyperref[TEI.affiliation]{affiliation} \hyperref[TEI.country]{country} \hyperref[TEI.forename]{forename} \hyperref[TEI.genName]{genName} \hyperref[TEI.geogName]{geogName} \hyperref[TEI.nameLink]{nameLink} \hyperref[TEI.orgName]{orgName} \hyperref[TEI.persName]{persName} \hyperref[TEI.person]{person} \hyperref[TEI.personGrp]{personGrp} \hyperref[TEI.persona]{persona} \hyperref[TEI.placeName]{placeName} \hyperref[TEI.region]{region} \hyperref[TEI.roleName]{roleName} \hyperref[TEI.settlement]{settlement} \hyperref[TEI.surname]{surname}\par 
    \item[textstructure: ]
   \hyperref[TEI.back]{back} \hyperref[TEI.body]{body} \hyperref[TEI.div]{div} \hyperref[TEI.docAuthor]{docAuthor} \hyperref[TEI.docDate]{docDate} \hyperref[TEI.docEdition]{docEdition} \hyperref[TEI.docTitle]{docTitle} \hyperref[TEI.floatingText]{floatingText} \hyperref[TEI.front]{front} \hyperref[TEI.group]{group} \hyperref[TEI.text]{text} \hyperref[TEI.titlePage]{titlePage} \hyperref[TEI.titlePart]{titlePart}\par 
    \item[transcr: ]
   \hyperref[TEI.damage]{damage} \hyperref[TEI.fw]{fw} \hyperref[TEI.line]{line} \hyperref[TEI.metamark]{metamark} \hyperref[TEI.mod]{mod} \hyperref[TEI.restore]{restore} \hyperref[TEI.retrace]{retrace} \hyperref[TEI.secl]{secl} \hyperref[TEI.sourceDoc]{sourceDoc} \hyperref[TEI.supplied]{supplied} \hyperref[TEI.surface]{surface} \hyperref[TEI.surfaceGrp]{surfaceGrp} \hyperref[TEI.surplus]{surplus} \hyperref[TEI.zone]{zone}
    \item[{Peut contenir}]
  
    \item[core: ]
   \hyperref[TEI.binaryObject]{binaryObject} \hyperref[TEI.desc]{desc} \hyperref[TEI.graphic]{graphic} \hyperref[TEI.label]{label} \hyperref[TEI.ptr]{ptr} \hyperref[TEI.ref]{ref}\par 
    \item[linking: ]
   \hyperref[TEI.seg]{seg}
    \item[{Note}]
  \par
It is possible to describe the content of the notation using elements from the \textsf{model.labelLike} class and it is possible to point to an external representation using elements from \textsf{model.ptrLike}. It is possible to specify the location of digital objects representing the notated music in other media such as images or audio-visual files. The encoder's interpretation of the correspondence between the notated music and these digital objects is not encoded explicitly. We recommend the use of graphic and binaryObject mainly as a fallback mechanism when the notated music format is not displayable by the application using the encoding. The alignment of encoded notated music, images carrying the notation, and audio files is a complex matter for which we refer the encoder to other formats and specifications such as MPEG-SMR.\par
It is also recommended, when useful, to embed XML-based music notation formats, such as the \xref{http://music-encoding.org}{Music Encoding Initiative} format as content of \hyperref[TEI.notatedMusic]{<notatedMusic>}. This must be done by means of customization.
    \item[{Exemple}]
  \leavevmode\bgroup\exampleFont \begin{shaded}\noindent\mbox{}{<\textbf{notatedMusic}>}\mbox{}\newline 
\hspace*{6pt}{<\textbf{ptr}\hspace*{6pt}{target}="{bar1.xml}"/>}\mbox{}\newline 
\hspace*{6pt}{<\textbf{graphic}\hspace*{6pt}{url}="{bar1.jpg}"/>}\mbox{}\newline 
\hspace*{6pt}{<\textbf{desc}>}First bar of Chopin's Scherzo No.3 Op.39{</\textbf{desc}>}\mbox{}\newline 
{</\textbf{notatedMusic}>}\end{shaded}\egroup 


    \item[{Modèle de contenu}]
  \mbox{}\hfill\\[-10pt]\begin{Verbatim}[fontsize=\small]
<content>
 <alternate maxOccurs="unbounded"
  minOccurs="0">
  <classRef key="model.labelLike"/>
  <classRef key="model.ptrLike"/>
  <elementRef key="graphic"/>
  <elementRef key="binaryObject"/>
  <elementRef key="seg"/>
 </alternate>
</content>
    
\end{Verbatim}

    \item[{Schéma Declaration}]
  \mbox{}\hfill\\[-10pt]\begin{Verbatim}[fontsize=\small]
element notatedMusic
{
   tei_att.global.attributes,
   tei_att.placement.attributes,
   tei_att.typed.attributes,
   (
      tei_model.labelLike    | tei_model.ptrLike    | tei_graphic    | tei_binaryObject    | tei_seg   )*
}
\end{Verbatim}

\end{reflist}  \index{note=<note>|oddindex}\index{scheme=@scheme!<note>|oddindex}\index{anchored=@anchored!<note>|oddindex}\index{targetEnd=@targetEnd!<note>|oddindex}
\begin{reflist}
\item[]\begin{specHead}{TEI.note}{<note> }contient une note ou une annotation [\xref{http://www.tei-c.org/release/doc/tei-p5-doc/en/html/CO.html\#CONONO}{3.8.1. Notes and Simple Annotation} \xref{http://www.tei-c.org/release/doc/tei-p5-doc/en/html/HD.html\#HD27}{2.2.6. The Notes Statement} \xref{http://www.tei-c.org/release/doc/tei-p5-doc/en/html/CO.html\#COBICON}{3.11.2.8. Notes and Statement of Language} \xref{http://www.tei-c.org/release/doc/tei-p5-doc/en/html/DI.html\#DITPNO}{9.3.5.4. Notes within Entries}]\end{specHead} 
    \item[{Module}]
  core
    \item[{Attributs}]
  Attributs \hyperref[TEI.att.global]{att.global} (\textit{@xml:id}, \textit{@n}, \textit{@xml:lang}, \textit{@xml:base}, \textit{@xml:space})  (\hyperref[TEI.att.global.rendition]{att.global.rendition} (\textit{@rend}, \textit{@style}, \textit{@rendition})) (\hyperref[TEI.att.global.linking]{att.global.linking} (\textit{@corresp}, \textit{@synch}, \textit{@sameAs}, \textit{@copyOf}, \textit{@next}, \textit{@prev}, \textit{@exclude}, \textit{@select})) (\hyperref[TEI.att.global.analytic]{att.global.analytic} (\textit{@ana})) (\hyperref[TEI.att.global.facs]{att.global.facs} (\textit{@facs})) (\hyperref[TEI.att.global.change]{att.global.change} (\textit{@change})) (\hyperref[TEI.att.global.responsibility]{att.global.responsibility} (\textit{@cert}, \textit{@resp})) (\hyperref[TEI.att.global.source]{att.global.source} (\textit{@source})) \hyperref[TEI.att.placement]{att.placement} (\textit{@place}) \hyperref[TEI.att.pointing]{att.pointing} (\textit{@targetLang}, \textit{@target}, \textit{@evaluate}) \hyperref[TEI.att.typed]{att.typed} (\textit{@type}, \textit{@subtype}) \hyperref[TEI.att.written]{att.written} (\textit{@hand}) \hfil\\[-10pt]\begin{sansreflist}
    \item[@scheme]
  désigne la liste des ontologies dans lequel l'ensemble des termes concernés sont définis.
\begin{reflist}
    \item[{Statut}]
  Optionel
    \item[{Type de données}]
  \hyperref[TEI.teidata.pointer]{teidata.pointer}
\end{reflist}  
    \item[@anchored]
  indique si l'exemplaire du texte montre l'emplacement de référence exact pour la note
\begin{reflist}
    \item[{Statut}]
  Optionel
    \item[{Type de données}]
  \hyperref[TEI.teidata.truthValue]{teidata.truthValue}
    \item[{Valeur par défaut}]
  true
    \item[{Note}]
  \par
Dans des textes modernes, les notes sont habituellement ancrées au moyen d’appels de notes explicites (pour des notes de bas de page ou des notes de fin de texte). A la place, une indication explicite de l'expression ou de la ligne annotée peut cependant être employée (par exemple ‘page 218, lignes 3–4’). L'attribut {\itshape anchored} indique si un emplacement est donné explicitement ou s'il est exprimé par un symbole ou par un renvoi. La valeur true indique qu'un endroit explicite est indiqué dans le texte ; la valeur false indique que le texte n'indique pas un endroit spécifique d'attachement pour la note. Si des symboles spécifiques utilisés dans le texte à l'endroit où la note est ancrée doivent être enregistrés, l'attribut {\itshape n} sera utilisé.
\end{reflist}  
    \item[@targetEnd]
  pointe vers la fin d'un passage auquel la note est attachée, si la note n'est pas enchâssée dans le texte à cet endroit
\begin{reflist}
    \item[{Statut}]
  Optionel
    \item[{Type de données}]
  1–∞ occurrences de \hyperref[TEI.teidata.pointer]{teidata.pointer} séparé par un espace
    \item[{Note}]
  \par
Cet attribut est conservé pour assurer un arrière-plan compatible ; il sera supprimé dans la prochaine mise à jour des Recommandations. La procédure recommandée pour pointer en direction d'une expansion des éléments est de le faire au moyen de la fonction \textsf{range} de XPointer, telle que la description en est faite à \xref{http://www.tei-c.org/release/doc/tei-p5-doc/en/html/SA.html\#SATSRN}{16.2.4.6. range()}.
\end{reflist}  
\end{sansreflist}  
    \item[{Membre du}]
  \hyperref[TEI.model.OABody]{model.OABody} \hyperref[TEI.model.noteLike]{model.noteLike} 
    \item[{Contenu dans}]
  
    \item[analysis: ]
   \hyperref[TEI.cl]{cl} \hyperref[TEI.m]{m} \hyperref[TEI.phr]{phr} \hyperref[TEI.s]{s} \hyperref[TEI.span]{span} \hyperref[TEI.w]{w}\par 
    \item[core: ]
   \hyperref[TEI.abbr]{abbr} \hyperref[TEI.add]{add} \hyperref[TEI.addrLine]{addrLine} \hyperref[TEI.address]{address} \hyperref[TEI.author]{author} \hyperref[TEI.bibl]{bibl} \hyperref[TEI.biblScope]{biblScope} \hyperref[TEI.biblStruct]{biblStruct} \hyperref[TEI.cit]{cit} \hyperref[TEI.citedRange]{citedRange} \hyperref[TEI.corr]{corr} \hyperref[TEI.date]{date} \hyperref[TEI.del]{del} \hyperref[TEI.distinct]{distinct} \hyperref[TEI.editor]{editor} \hyperref[TEI.email]{email} \hyperref[TEI.emph]{emph} \hyperref[TEI.expan]{expan} \hyperref[TEI.foreign]{foreign} \hyperref[TEI.gloss]{gloss} \hyperref[TEI.head]{head} \hyperref[TEI.headItem]{headItem} \hyperref[TEI.headLabel]{headLabel} \hyperref[TEI.hi]{hi} \hyperref[TEI.imprint]{imprint} \hyperref[TEI.item]{item} \hyperref[TEI.l]{l} \hyperref[TEI.label]{label} \hyperref[TEI.lg]{lg} \hyperref[TEI.list]{list} \hyperref[TEI.measure]{measure} \hyperref[TEI.mentioned]{mentioned} \hyperref[TEI.monogr]{monogr} \hyperref[TEI.name]{name} \hyperref[TEI.note]{note} \hyperref[TEI.num]{num} \hyperref[TEI.orig]{orig} \hyperref[TEI.p]{p} \hyperref[TEI.pubPlace]{pubPlace} \hyperref[TEI.publisher]{publisher} \hyperref[TEI.q]{q} \hyperref[TEI.quote]{quote} \hyperref[TEI.ref]{ref} \hyperref[TEI.reg]{reg} \hyperref[TEI.resp]{resp} \hyperref[TEI.respStmt]{respStmt} \hyperref[TEI.rs]{rs} \hyperref[TEI.said]{said} \hyperref[TEI.series]{series} \hyperref[TEI.sic]{sic} \hyperref[TEI.soCalled]{soCalled} \hyperref[TEI.sp]{sp} \hyperref[TEI.speaker]{speaker} \hyperref[TEI.stage]{stage} \hyperref[TEI.street]{street} \hyperref[TEI.term]{term} \hyperref[TEI.textLang]{textLang} \hyperref[TEI.time]{time} \hyperref[TEI.title]{title} \hyperref[TEI.unclear]{unclear}\par 
    \item[figures: ]
   \hyperref[TEI.cell]{cell} \hyperref[TEI.figure]{figure} \hyperref[TEI.table]{table}\par 
    \item[header: ]
   \hyperref[TEI.authority]{authority} \hyperref[TEI.change]{change} \hyperref[TEI.classCode]{classCode} \hyperref[TEI.distributor]{distributor} \hyperref[TEI.edition]{edition} \hyperref[TEI.extent]{extent} \hyperref[TEI.funder]{funder} \hyperref[TEI.language]{language} \hyperref[TEI.licence]{licence} \hyperref[TEI.notesStmt]{notesStmt}\par 
    \item[linking: ]
   \hyperref[TEI.ab]{ab} \hyperref[TEI.seg]{seg}\par 
    \item[msdescription: ]
   \hyperref[TEI.accMat]{accMat} \hyperref[TEI.acquisition]{acquisition} \hyperref[TEI.additions]{additions} \hyperref[TEI.adminInfo]{adminInfo} \hyperref[TEI.altIdentifier]{altIdentifier} \hyperref[TEI.catchwords]{catchwords} \hyperref[TEI.collation]{collation} \hyperref[TEI.colophon]{colophon} \hyperref[TEI.condition]{condition} \hyperref[TEI.custEvent]{custEvent} \hyperref[TEI.decoNote]{decoNote} \hyperref[TEI.explicit]{explicit} \hyperref[TEI.filiation]{filiation} \hyperref[TEI.finalRubric]{finalRubric} \hyperref[TEI.foliation]{foliation} \hyperref[TEI.heraldry]{heraldry} \hyperref[TEI.incipit]{incipit} \hyperref[TEI.layout]{layout} \hyperref[TEI.material]{material} \hyperref[TEI.msItem]{msItem} \hyperref[TEI.msItemStruct]{msItemStruct} \hyperref[TEI.musicNotation]{musicNotation} \hyperref[TEI.objectType]{objectType} \hyperref[TEI.origDate]{origDate} \hyperref[TEI.origPlace]{origPlace} \hyperref[TEI.origin]{origin} \hyperref[TEI.provenance]{provenance} \hyperref[TEI.rubric]{rubric} \hyperref[TEI.secFol]{secFol} \hyperref[TEI.signatures]{signatures} \hyperref[TEI.source]{source} \hyperref[TEI.stamp]{stamp} \hyperref[TEI.summary]{summary} \hyperref[TEI.support]{support} \hyperref[TEI.surrogates]{surrogates} \hyperref[TEI.typeNote]{typeNote} \hyperref[TEI.watermark]{watermark}\par 
    \item[namesdates: ]
   \hyperref[TEI.addName]{addName} \hyperref[TEI.affiliation]{affiliation} \hyperref[TEI.country]{country} \hyperref[TEI.event]{event} \hyperref[TEI.forename]{forename} \hyperref[TEI.genName]{genName} \hyperref[TEI.geogName]{geogName} \hyperref[TEI.location]{location} \hyperref[TEI.nameLink]{nameLink} \hyperref[TEI.org]{org} \hyperref[TEI.orgName]{orgName} \hyperref[TEI.persName]{persName} \hyperref[TEI.person]{person} \hyperref[TEI.personGrp]{personGrp} \hyperref[TEI.persona]{persona} \hyperref[TEI.place]{place} \hyperref[TEI.placeName]{placeName} \hyperref[TEI.region]{region} \hyperref[TEI.roleName]{roleName} \hyperref[TEI.settlement]{settlement} \hyperref[TEI.state]{state} \hyperref[TEI.surname]{surname}\par 
    \item[spoken: ]
   \hyperref[TEI.annotationBlock]{annotationBlock}\par 
    \item[textstructure: ]
   \hyperref[TEI.back]{back} \hyperref[TEI.body]{body} \hyperref[TEI.div]{div} \hyperref[TEI.docAuthor]{docAuthor} \hyperref[TEI.docDate]{docDate} \hyperref[TEI.docEdition]{docEdition} \hyperref[TEI.docTitle]{docTitle} \hyperref[TEI.floatingText]{floatingText} \hyperref[TEI.front]{front} \hyperref[TEI.group]{group} \hyperref[TEI.text]{text} \hyperref[TEI.titlePage]{titlePage} \hyperref[TEI.titlePart]{titlePart}\par 
    \item[transcr: ]
   \hyperref[TEI.damage]{damage} \hyperref[TEI.fw]{fw} \hyperref[TEI.line]{line} \hyperref[TEI.metamark]{metamark} \hyperref[TEI.mod]{mod} \hyperref[TEI.restore]{restore} \hyperref[TEI.retrace]{retrace} \hyperref[TEI.secl]{secl} \hyperref[TEI.sourceDoc]{sourceDoc} \hyperref[TEI.supplied]{supplied} \hyperref[TEI.surface]{surface} \hyperref[TEI.surfaceGrp]{surfaceGrp} \hyperref[TEI.surplus]{surplus} \hyperref[TEI.zone]{zone}
    \item[{Peut contenir}]
  
    \item[analysis: ]
   \hyperref[TEI.c]{c} \hyperref[TEI.cl]{cl} \hyperref[TEI.interp]{interp} \hyperref[TEI.interpGrp]{interpGrp} \hyperref[TEI.m]{m} \hyperref[TEI.pc]{pc} \hyperref[TEI.phr]{phr} \hyperref[TEI.s]{s} \hyperref[TEI.span]{span} \hyperref[TEI.spanGrp]{spanGrp} \hyperref[TEI.w]{w}\par 
    \item[core: ]
   \hyperref[TEI.abbr]{abbr} \hyperref[TEI.add]{add} \hyperref[TEI.address]{address} \hyperref[TEI.bibl]{bibl} \hyperref[TEI.biblStruct]{biblStruct} \hyperref[TEI.binaryObject]{binaryObject} \hyperref[TEI.cb]{cb} \hyperref[TEI.choice]{choice} \hyperref[TEI.cit]{cit} \hyperref[TEI.corr]{corr} \hyperref[TEI.date]{date} \hyperref[TEI.del]{del} \hyperref[TEI.desc]{desc} \hyperref[TEI.distinct]{distinct} \hyperref[TEI.email]{email} \hyperref[TEI.emph]{emph} \hyperref[TEI.expan]{expan} \hyperref[TEI.foreign]{foreign} \hyperref[TEI.gap]{gap} \hyperref[TEI.gb]{gb} \hyperref[TEI.gloss]{gloss} \hyperref[TEI.graphic]{graphic} \hyperref[TEI.hi]{hi} \hyperref[TEI.index]{index} \hyperref[TEI.l]{l} \hyperref[TEI.label]{label} \hyperref[TEI.lb]{lb} \hyperref[TEI.lg]{lg} \hyperref[TEI.list]{list} \hyperref[TEI.listBibl]{listBibl} \hyperref[TEI.measure]{measure} \hyperref[TEI.measureGrp]{measureGrp} \hyperref[TEI.media]{media} \hyperref[TEI.mentioned]{mentioned} \hyperref[TEI.milestone]{milestone} \hyperref[TEI.name]{name} \hyperref[TEI.note]{note} \hyperref[TEI.num]{num} \hyperref[TEI.orig]{orig} \hyperref[TEI.p]{p} \hyperref[TEI.pb]{pb} \hyperref[TEI.ptr]{ptr} \hyperref[TEI.q]{q} \hyperref[TEI.quote]{quote} \hyperref[TEI.ref]{ref} \hyperref[TEI.reg]{reg} \hyperref[TEI.rs]{rs} \hyperref[TEI.said]{said} \hyperref[TEI.sic]{sic} \hyperref[TEI.soCalled]{soCalled} \hyperref[TEI.sp]{sp} \hyperref[TEI.stage]{stage} \hyperref[TEI.term]{term} \hyperref[TEI.time]{time} \hyperref[TEI.title]{title} \hyperref[TEI.unclear]{unclear}\par 
    \item[derived-module-tei.istex: ]
   \hyperref[TEI.math]{math} \hyperref[TEI.mrow]{mrow}\par 
    \item[figures: ]
   \hyperref[TEI.figure]{figure} \hyperref[TEI.formula]{formula} \hyperref[TEI.notatedMusic]{notatedMusic} \hyperref[TEI.table]{table}\par 
    \item[header: ]
   \hyperref[TEI.biblFull]{biblFull} \hyperref[TEI.idno]{idno}\par 
    \item[iso-fs: ]
   \hyperref[TEI.fLib]{fLib} \hyperref[TEI.fs]{fs} \hyperref[TEI.fvLib]{fvLib}\par 
    \item[linking: ]
   \hyperref[TEI.ab]{ab} \hyperref[TEI.alt]{alt} \hyperref[TEI.altGrp]{altGrp} \hyperref[TEI.anchor]{anchor} \hyperref[TEI.join]{join} \hyperref[TEI.joinGrp]{joinGrp} \hyperref[TEI.link]{link} \hyperref[TEI.linkGrp]{linkGrp} \hyperref[TEI.seg]{seg} \hyperref[TEI.timeline]{timeline}\par 
    \item[msdescription: ]
   \hyperref[TEI.catchwords]{catchwords} \hyperref[TEI.depth]{depth} \hyperref[TEI.dim]{dim} \hyperref[TEI.dimensions]{dimensions} \hyperref[TEI.height]{height} \hyperref[TEI.heraldry]{heraldry} \hyperref[TEI.locus]{locus} \hyperref[TEI.locusGrp]{locusGrp} \hyperref[TEI.material]{material} \hyperref[TEI.msDesc]{msDesc} \hyperref[TEI.objectType]{objectType} \hyperref[TEI.origDate]{origDate} \hyperref[TEI.origPlace]{origPlace} \hyperref[TEI.secFol]{secFol} \hyperref[TEI.signatures]{signatures} \hyperref[TEI.source]{source} \hyperref[TEI.stamp]{stamp} \hyperref[TEI.watermark]{watermark} \hyperref[TEI.width]{width}\par 
    \item[namesdates: ]
   \hyperref[TEI.addName]{addName} \hyperref[TEI.affiliation]{affiliation} \hyperref[TEI.country]{country} \hyperref[TEI.forename]{forename} \hyperref[TEI.genName]{genName} \hyperref[TEI.geogName]{geogName} \hyperref[TEI.listOrg]{listOrg} \hyperref[TEI.listPlace]{listPlace} \hyperref[TEI.location]{location} \hyperref[TEI.nameLink]{nameLink} \hyperref[TEI.orgName]{orgName} \hyperref[TEI.persName]{persName} \hyperref[TEI.placeName]{placeName} \hyperref[TEI.region]{region} \hyperref[TEI.roleName]{roleName} \hyperref[TEI.settlement]{settlement} \hyperref[TEI.state]{state} \hyperref[TEI.surname]{surname}\par 
    \item[spoken: ]
   \hyperref[TEI.annotationBlock]{annotationBlock}\par 
    \item[textstructure: ]
   \hyperref[TEI.floatingText]{floatingText}\par 
    \item[transcr: ]
   \hyperref[TEI.addSpan]{addSpan} \hyperref[TEI.am]{am} \hyperref[TEI.damage]{damage} \hyperref[TEI.damageSpan]{damageSpan} \hyperref[TEI.delSpan]{delSpan} \hyperref[TEI.ex]{ex} \hyperref[TEI.fw]{fw} \hyperref[TEI.handShift]{handShift} \hyperref[TEI.listTranspose]{listTranspose} \hyperref[TEI.metamark]{metamark} \hyperref[TEI.mod]{mod} \hyperref[TEI.redo]{redo} \hyperref[TEI.restore]{restore} \hyperref[TEI.retrace]{retrace} \hyperref[TEI.secl]{secl} \hyperref[TEI.space]{space} \hyperref[TEI.subst]{subst} \hyperref[TEI.substJoin]{substJoin} \hyperref[TEI.supplied]{supplied} \hyperref[TEI.surplus]{surplus} \hyperref[TEI.undo]{undo}\par des données textuelles
    \item[{Note}]
  \par
L'attribut global{\itshape n} indique le symbole ou le nombre utilisé pour marquer le point d'insertion dans le texte source, comme dans l'exemple suivant : \par\bgroup\exampleFont \begin{shaded}\noindent\mbox{}Mevorakh b. Saadya's\mbox{}\newline 
 mother, the matriarch of the family during the second half of the\mbox{}\newline 
 eleventh century, {<\textbf{note}\hspace*{6pt}{anchored}="{true}"\hspace*{6pt}{n}="{126}">} The alleged\mbox{}\newline 
 mention of Judah Nagid's mother in a letter from 1071 is, in fact,\mbox{}\newline 
 a reference to Judah's children; cf. above, nn. 111 and\mbox{}\newline 
 54. {</\textbf{note}>} is well known from Geniza documents published by Jacob\mbox{}\newline 
 Mann.\end{shaded}\egroup\par \noindent  Cependant, si les notes sont ordonnées et numérotées et qu’on veuille reconstruire automatiquement leur numérotation par un traitement informatique, il est inutile d’enregistrer le numéro des notes.
    \item[{Exemple}]
  \leavevmode\bgroup\exampleFont \begin{shaded}\noindent\mbox{}{<\textbf{p}>}J'écris dans la{<\textbf{lb}/>} marge...{<\textbf{lb}/>} Je vais{<\textbf{lb}/>} à la ligne.{<\textbf{lb}/>} Je renvoie à une\mbox{}\newline 
 note{<\textbf{note}\hspace*{6pt}{place}="{foot}"\hspace*{6pt}{type}="{gloss}">} J'aime beaucoup les renvois en bas de page, même si\mbox{}\newline 
\hspace*{6pt}\hspace*{6pt} je n'ai rien de particulier à y préciser.{</\textbf{note}>}en bas de page.{</\textbf{p}>}\end{shaded}\egroup 


    \item[{Modèle de contenu}]
  \mbox{}\hfill\\[-10pt]\begin{Verbatim}[fontsize=\small]
<content>
 <macroRef key="macro.specialPara"/>
</content>
    
\end{Verbatim}

    \item[{Schéma Declaration}]
  \mbox{}\hfill\\[-10pt]\begin{Verbatim}[fontsize=\small]
element note
{
   tei_att.global.attributes,
   tei_att.placement.attributes,
   tei_att.pointing.attributes,
   tei_att.typed.attributes,
   tei_att.written.attributes,
   attribute scheme { text }?,
   attribute anchored { text }?,
   attribute targetEnd { list { + } }?,
   tei_macro.specialPara}
\end{Verbatim}

\end{reflist}  \index{notesStmt=<notesStmt>|oddindex}
\begin{reflist}
\item[]\begin{specHead}{TEI.notesStmt}{<notesStmt> }(mention de notes) rassemble toutes les notes fournissant des informations sur un texte, en plus des informations mentionnées dans d'autres parties de la description bibliographique. [\xref{http://www.tei-c.org/release/doc/tei-p5-doc/en/html/HD.html\#HD27}{2.2.6. The Notes Statement} \xref{http://www.tei-c.org/release/doc/tei-p5-doc/en/html/HD.html\#HD2}{2.2. The File Description}]\end{specHead} 
    \item[{Module}]
  header
    \item[{Attributs}]
  Attributs \hyperref[TEI.att.global]{att.global} (\textit{@xml:id}, \textit{@n}, \textit{@xml:lang}, \textit{@xml:base}, \textit{@xml:space})  (\hyperref[TEI.att.global.rendition]{att.global.rendition} (\textit{@rend}, \textit{@style}, \textit{@rendition})) (\hyperref[TEI.att.global.linking]{att.global.linking} (\textit{@corresp}, \textit{@synch}, \textit{@sameAs}, \textit{@copyOf}, \textit{@next}, \textit{@prev}, \textit{@exclude}, \textit{@select})) (\hyperref[TEI.att.global.analytic]{att.global.analytic} (\textit{@ana})) (\hyperref[TEI.att.global.facs]{att.global.facs} (\textit{@facs})) (\hyperref[TEI.att.global.change]{att.global.change} (\textit{@change})) (\hyperref[TEI.att.global.responsibility]{att.global.responsibility} (\textit{@cert}, \textit{@resp})) (\hyperref[TEI.att.global.source]{att.global.source} (\textit{@source}))
    \item[{Contenu dans}]
  
    \item[header: ]
   \hyperref[TEI.biblFull]{biblFull} \hyperref[TEI.fileDesc]{fileDesc}
    \item[{Peut contenir}]
  
    \item[core: ]
   \hyperref[TEI.note]{note} \hyperref[TEI.relatedItem]{relatedItem}
    \item[{Note}]
  \par
des informations hétérogènes ne doivent pas être regroupées dans une même note.
    \item[{Exemple}]
  \leavevmode\bgroup\exampleFont \begin{shaded}\noindent\mbox{}{<\textbf{notesStmt}>}\mbox{}\newline 
\hspace*{6pt}{<\textbf{note}>}Les photographies, héliogravures et cartes postales colorisées signées Lehnert\mbox{}\newline 
\hspace*{6pt}\hspace*{6pt} \& Landrock, de techniques novatrices, sont réellement des œuvres d'art ; elles\mbox{}\newline 
\hspace*{6pt}\hspace*{6pt} apportent en outre une documentation considérable sur la Tunisie du début du XXe\mbox{}\newline 
\hspace*{6pt}\hspace*{6pt} siècle.{</\textbf{note}>}\mbox{}\newline 
{</\textbf{notesStmt}>}\end{shaded}\egroup 


    \item[{Modèle de contenu}]
  \mbox{}\hfill\\[-10pt]\begin{Verbatim}[fontsize=\small]
<content>
 <alternate maxOccurs="unbounded"
  minOccurs="1">
  <classRef key="model.noteLike"/>
  <elementRef key="relatedItem"/>
 </alternate>
</content>
    
\end{Verbatim}

    \item[{Schéma Declaration}]
  \mbox{}\hfill\\[-10pt]\begin{Verbatim}[fontsize=\small]
element notesStmt
{
   tei_att.global.attributes,
   ( tei_model.noteLike | tei_relatedItem )+
}
\end{Verbatim}

\end{reflist}  \index{num=<num>|oddindex}\index{type=@type!<num>|oddindex}\index{value=@value!<num>|oddindex}
\begin{reflist}
\item[]\begin{specHead}{TEI.num}{<num> }(numéral) contient un nombre écrit sous une forme quelconque. [\xref{http://www.tei-c.org/release/doc/tei-p5-doc/en/html/CO.html\#CONANU}{3.5.3. Numbers and Measures}]\end{specHead} 
    \item[{Module}]
  core
    \item[{Attributs}]
  Attributs \hyperref[TEI.att.global]{att.global} (\textit{@xml:id}, \textit{@n}, \textit{@xml:lang}, \textit{@xml:base}, \textit{@xml:space})  (\hyperref[TEI.att.global.rendition]{att.global.rendition} (\textit{@rend}, \textit{@style}, \textit{@rendition})) (\hyperref[TEI.att.global.linking]{att.global.linking} (\textit{@corresp}, \textit{@synch}, \textit{@sameAs}, \textit{@copyOf}, \textit{@next}, \textit{@prev}, \textit{@exclude}, \textit{@select})) (\hyperref[TEI.att.global.analytic]{att.global.analytic} (\textit{@ana})) (\hyperref[TEI.att.global.facs]{att.global.facs} (\textit{@facs})) (\hyperref[TEI.att.global.change]{att.global.change} (\textit{@change})) (\hyperref[TEI.att.global.responsibility]{att.global.responsibility} (\textit{@cert}, \textit{@resp})) (\hyperref[TEI.att.global.source]{att.global.source} (\textit{@source})) \hyperref[TEI.att.ranging]{att.ranging} (\textit{@atLeast}, \textit{@atMost}, \textit{@min}, \textit{@max}, \textit{@confidence}) \hfil\\[-10pt]\begin{sansreflist}
    \item[@type]
  indique le type de valeur numérique
\begin{reflist}
    \item[{Statut}]
  Optionel
    \item[{Type de données}]
  \hyperref[TEI.teidata.enumerated]{teidata.enumerated}
    \item[{Les valeurs suggérées comprennent:}]
  \begin{description}

\item[{cardinal}]nombre entier ou décimal, par exemple 21, 21.5
\item[{ordinal}]nombre ordinal, par exemple 21ème
\item[{fraction}]fraction, par exemple une moitié ou trois-quarts
\item[{percentage}]un pourcentage
\end{description} 
    \item[{Note}]
  \par
Si une autre typologie est souhaitée, d'autres valeurs peuvent être utilisées pour cet attribut.
\end{reflist}  
    \item[@value]
  fournit la valeur d'un nombre sous une forme normalisée.
\begin{reflist}
    \item[{Statut}]
  Optionel
    \item[{Type de données}]
  \hyperref[TEI.teidata.numeric]{teidata.numeric}
    \item[{Valeurs}]
  une valeur numérique.
    \item[{Note}]
  \par
La forme normalisée utilisée est définie par le type de données TEI qui concerne les données numériques.
\end{reflist}  
\end{sansreflist}  
    \item[{Membre du}]
  \hyperref[TEI.model.measureLike]{model.measureLike}
    \item[{Contenu dans}]
  
    \item[analysis: ]
   \hyperref[TEI.cl]{cl} \hyperref[TEI.phr]{phr} \hyperref[TEI.s]{s} \hyperref[TEI.span]{span}\par 
    \item[core: ]
   \hyperref[TEI.abbr]{abbr} \hyperref[TEI.add]{add} \hyperref[TEI.addrLine]{addrLine} \hyperref[TEI.author]{author} \hyperref[TEI.bibl]{bibl} \hyperref[TEI.biblScope]{biblScope} \hyperref[TEI.citedRange]{citedRange} \hyperref[TEI.corr]{corr} \hyperref[TEI.date]{date} \hyperref[TEI.del]{del} \hyperref[TEI.desc]{desc} \hyperref[TEI.distinct]{distinct} \hyperref[TEI.editor]{editor} \hyperref[TEI.email]{email} \hyperref[TEI.emph]{emph} \hyperref[TEI.expan]{expan} \hyperref[TEI.foreign]{foreign} \hyperref[TEI.gloss]{gloss} \hyperref[TEI.head]{head} \hyperref[TEI.headItem]{headItem} \hyperref[TEI.headLabel]{headLabel} \hyperref[TEI.hi]{hi} \hyperref[TEI.item]{item} \hyperref[TEI.l]{l} \hyperref[TEI.label]{label} \hyperref[TEI.measure]{measure} \hyperref[TEI.measureGrp]{measureGrp} \hyperref[TEI.meeting]{meeting} \hyperref[TEI.mentioned]{mentioned} \hyperref[TEI.name]{name} \hyperref[TEI.note]{note} \hyperref[TEI.num]{num} \hyperref[TEI.orig]{orig} \hyperref[TEI.p]{p} \hyperref[TEI.pubPlace]{pubPlace} \hyperref[TEI.publisher]{publisher} \hyperref[TEI.q]{q} \hyperref[TEI.quote]{quote} \hyperref[TEI.ref]{ref} \hyperref[TEI.reg]{reg} \hyperref[TEI.resp]{resp} \hyperref[TEI.rs]{rs} \hyperref[TEI.said]{said} \hyperref[TEI.sic]{sic} \hyperref[TEI.soCalled]{soCalled} \hyperref[TEI.speaker]{speaker} \hyperref[TEI.stage]{stage} \hyperref[TEI.street]{street} \hyperref[TEI.term]{term} \hyperref[TEI.textLang]{textLang} \hyperref[TEI.time]{time} \hyperref[TEI.title]{title} \hyperref[TEI.unclear]{unclear}\par 
    \item[figures: ]
   \hyperref[TEI.cell]{cell} \hyperref[TEI.figDesc]{figDesc}\par 
    \item[header: ]
   \hyperref[TEI.authority]{authority} \hyperref[TEI.change]{change} \hyperref[TEI.classCode]{classCode} \hyperref[TEI.creation]{creation} \hyperref[TEI.distributor]{distributor} \hyperref[TEI.edition]{edition} \hyperref[TEI.extent]{extent} \hyperref[TEI.funder]{funder} \hyperref[TEI.language]{language} \hyperref[TEI.licence]{licence} \hyperref[TEI.rendition]{rendition}\par 
    \item[iso-fs: ]
   \hyperref[TEI.fDescr]{fDescr} \hyperref[TEI.fsDescr]{fsDescr}\par 
    \item[linking: ]
   \hyperref[TEI.ab]{ab} \hyperref[TEI.seg]{seg}\par 
    \item[msdescription: ]
   \hyperref[TEI.accMat]{accMat} \hyperref[TEI.acquisition]{acquisition} \hyperref[TEI.additions]{additions} \hyperref[TEI.catchwords]{catchwords} \hyperref[TEI.collation]{collation} \hyperref[TEI.colophon]{colophon} \hyperref[TEI.condition]{condition} \hyperref[TEI.custEvent]{custEvent} \hyperref[TEI.decoNote]{decoNote} \hyperref[TEI.explicit]{explicit} \hyperref[TEI.filiation]{filiation} \hyperref[TEI.finalRubric]{finalRubric} \hyperref[TEI.foliation]{foliation} \hyperref[TEI.heraldry]{heraldry} \hyperref[TEI.incipit]{incipit} \hyperref[TEI.layout]{layout} \hyperref[TEI.material]{material} \hyperref[TEI.musicNotation]{musicNotation} \hyperref[TEI.objectType]{objectType} \hyperref[TEI.origDate]{origDate} \hyperref[TEI.origPlace]{origPlace} \hyperref[TEI.origin]{origin} \hyperref[TEI.provenance]{provenance} \hyperref[TEI.rubric]{rubric} \hyperref[TEI.secFol]{secFol} \hyperref[TEI.signatures]{signatures} \hyperref[TEI.source]{source} \hyperref[TEI.stamp]{stamp} \hyperref[TEI.summary]{summary} \hyperref[TEI.support]{support} \hyperref[TEI.surrogates]{surrogates} \hyperref[TEI.typeNote]{typeNote} \hyperref[TEI.watermark]{watermark}\par 
    \item[namesdates: ]
   \hyperref[TEI.addName]{addName} \hyperref[TEI.affiliation]{affiliation} \hyperref[TEI.country]{country} \hyperref[TEI.forename]{forename} \hyperref[TEI.genName]{genName} \hyperref[TEI.geogName]{geogName} \hyperref[TEI.location]{location} \hyperref[TEI.nameLink]{nameLink} \hyperref[TEI.orgName]{orgName} \hyperref[TEI.persName]{persName} \hyperref[TEI.placeName]{placeName} \hyperref[TEI.region]{region} \hyperref[TEI.roleName]{roleName} \hyperref[TEI.settlement]{settlement} \hyperref[TEI.surname]{surname}\par 
    \item[textstructure: ]
   \hyperref[TEI.docAuthor]{docAuthor} \hyperref[TEI.docDate]{docDate} \hyperref[TEI.docEdition]{docEdition} \hyperref[TEI.titlePart]{titlePart}\par 
    \item[transcr: ]
   \hyperref[TEI.damage]{damage} \hyperref[TEI.fw]{fw} \hyperref[TEI.metamark]{metamark} \hyperref[TEI.mod]{mod} \hyperref[TEI.restore]{restore} \hyperref[TEI.retrace]{retrace} \hyperref[TEI.secl]{secl} \hyperref[TEI.supplied]{supplied} \hyperref[TEI.surplus]{surplus}
    \item[{Peut contenir}]
  
    \item[analysis: ]
   \hyperref[TEI.c]{c} \hyperref[TEI.cl]{cl} \hyperref[TEI.interp]{interp} \hyperref[TEI.interpGrp]{interpGrp} \hyperref[TEI.m]{m} \hyperref[TEI.pc]{pc} \hyperref[TEI.phr]{phr} \hyperref[TEI.s]{s} \hyperref[TEI.span]{span} \hyperref[TEI.spanGrp]{spanGrp} \hyperref[TEI.w]{w}\par 
    \item[core: ]
   \hyperref[TEI.abbr]{abbr} \hyperref[TEI.add]{add} \hyperref[TEI.address]{address} \hyperref[TEI.binaryObject]{binaryObject} \hyperref[TEI.cb]{cb} \hyperref[TEI.choice]{choice} \hyperref[TEI.corr]{corr} \hyperref[TEI.date]{date} \hyperref[TEI.del]{del} \hyperref[TEI.distinct]{distinct} \hyperref[TEI.email]{email} \hyperref[TEI.emph]{emph} \hyperref[TEI.expan]{expan} \hyperref[TEI.foreign]{foreign} \hyperref[TEI.gap]{gap} \hyperref[TEI.gb]{gb} \hyperref[TEI.gloss]{gloss} \hyperref[TEI.graphic]{graphic} \hyperref[TEI.hi]{hi} \hyperref[TEI.index]{index} \hyperref[TEI.lb]{lb} \hyperref[TEI.measure]{measure} \hyperref[TEI.measureGrp]{measureGrp} \hyperref[TEI.media]{media} \hyperref[TEI.mentioned]{mentioned} \hyperref[TEI.milestone]{milestone} \hyperref[TEI.name]{name} \hyperref[TEI.note]{note} \hyperref[TEI.num]{num} \hyperref[TEI.orig]{orig} \hyperref[TEI.pb]{pb} \hyperref[TEI.ptr]{ptr} \hyperref[TEI.ref]{ref} \hyperref[TEI.reg]{reg} \hyperref[TEI.rs]{rs} \hyperref[TEI.sic]{sic} \hyperref[TEI.soCalled]{soCalled} \hyperref[TEI.term]{term} \hyperref[TEI.time]{time} \hyperref[TEI.title]{title} \hyperref[TEI.unclear]{unclear}\par 
    \item[derived-module-tei.istex: ]
   \hyperref[TEI.math]{math} \hyperref[TEI.mrow]{mrow}\par 
    \item[figures: ]
   \hyperref[TEI.figure]{figure} \hyperref[TEI.formula]{formula} \hyperref[TEI.notatedMusic]{notatedMusic}\par 
    \item[header: ]
   \hyperref[TEI.idno]{idno}\par 
    \item[iso-fs: ]
   \hyperref[TEI.fLib]{fLib} \hyperref[TEI.fs]{fs} \hyperref[TEI.fvLib]{fvLib}\par 
    \item[linking: ]
   \hyperref[TEI.alt]{alt} \hyperref[TEI.altGrp]{altGrp} \hyperref[TEI.anchor]{anchor} \hyperref[TEI.join]{join} \hyperref[TEI.joinGrp]{joinGrp} \hyperref[TEI.link]{link} \hyperref[TEI.linkGrp]{linkGrp} \hyperref[TEI.seg]{seg} \hyperref[TEI.timeline]{timeline}\par 
    \item[msdescription: ]
   \hyperref[TEI.catchwords]{catchwords} \hyperref[TEI.depth]{depth} \hyperref[TEI.dim]{dim} \hyperref[TEI.dimensions]{dimensions} \hyperref[TEI.height]{height} \hyperref[TEI.heraldry]{heraldry} \hyperref[TEI.locus]{locus} \hyperref[TEI.locusGrp]{locusGrp} \hyperref[TEI.material]{material} \hyperref[TEI.objectType]{objectType} \hyperref[TEI.origDate]{origDate} \hyperref[TEI.origPlace]{origPlace} \hyperref[TEI.secFol]{secFol} \hyperref[TEI.signatures]{signatures} \hyperref[TEI.source]{source} \hyperref[TEI.stamp]{stamp} \hyperref[TEI.watermark]{watermark} \hyperref[TEI.width]{width}\par 
    \item[namesdates: ]
   \hyperref[TEI.addName]{addName} \hyperref[TEI.affiliation]{affiliation} \hyperref[TEI.country]{country} \hyperref[TEI.forename]{forename} \hyperref[TEI.genName]{genName} \hyperref[TEI.geogName]{geogName} \hyperref[TEI.location]{location} \hyperref[TEI.nameLink]{nameLink} \hyperref[TEI.orgName]{orgName} \hyperref[TEI.persName]{persName} \hyperref[TEI.placeName]{placeName} \hyperref[TEI.region]{region} \hyperref[TEI.roleName]{roleName} \hyperref[TEI.settlement]{settlement} \hyperref[TEI.state]{state} \hyperref[TEI.surname]{surname}\par 
    \item[spoken: ]
   \hyperref[TEI.annotationBlock]{annotationBlock}\par 
    \item[transcr: ]
   \hyperref[TEI.addSpan]{addSpan} \hyperref[TEI.am]{am} \hyperref[TEI.damage]{damage} \hyperref[TEI.damageSpan]{damageSpan} \hyperref[TEI.delSpan]{delSpan} \hyperref[TEI.ex]{ex} \hyperref[TEI.fw]{fw} \hyperref[TEI.handShift]{handShift} \hyperref[TEI.listTranspose]{listTranspose} \hyperref[TEI.metamark]{metamark} \hyperref[TEI.mod]{mod} \hyperref[TEI.redo]{redo} \hyperref[TEI.restore]{restore} \hyperref[TEI.retrace]{retrace} \hyperref[TEI.secl]{secl} \hyperref[TEI.space]{space} \hyperref[TEI.subst]{subst} \hyperref[TEI.substJoin]{substJoin} \hyperref[TEI.supplied]{supplied} \hyperref[TEI.surplus]{surplus} \hyperref[TEI.undo]{undo}\par des données textuelles
    \item[{Note}]
  \par
Les analyses détaillées des quantités et unités de mesure dans les textes historiques peuvent aussi utiliser le mécanisme de structure de traits décrit au chapitre\xref{http://www.tei-c.org/release/doc/tei-p5-doc/en/html/FS.html\#FS}{18. Feature Structures}. L'élément \hyperref[TEI.num]{<num>} est conçu pour un usage dans des applications simples.
    \item[{Exemple}]
  \leavevmode\bgroup\exampleFont \begin{shaded}\noindent\mbox{}{<\textbf{p}>}Pierre eut {<\textbf{num}\hspace*{6pt}{type}="{cardinal}"\hspace*{6pt}{value}="{10}">}dix{</\textbf{num}>}ans le jour de mon{<\textbf{num}\hspace*{6pt}{type}="{ordinal}"\hspace*{6pt}{value}="{21}">}vingtième {</\textbf{num}>} anniversaire.{</\textbf{p}>}\end{shaded}\egroup 


    \item[{Modèle de contenu}]
  \mbox{}\hfill\\[-10pt]\begin{Verbatim}[fontsize=\small]
<content>
 <macroRef key="macro.phraseSeq"/>
</content>
    
\end{Verbatim}

    \item[{Schéma Declaration}]
  \mbox{}\hfill\\[-10pt]\begin{Verbatim}[fontsize=\small]
element num
{
   tei_att.global.attributes,
   tei_att.ranging.attributes,
   attribute type { "cardinal" | "ordinal" | "fraction" | "percentage" }?,
   attribute value { text }?,
   tei_macro.phraseSeq}
\end{Verbatim}

\end{reflist}  \index{numeric=<numeric>|oddindex}\index{value=@value!<numeric>|oddindex}\index{max=@max!<numeric>|oddindex}\index{trunc=@trunc!<numeric>|oddindex}
\begin{reflist}
\item[]\begin{specHead}{TEI.numeric}{<numeric> }(valeur numérique) représente la partie valeur d'une spécification trait-valeur qui contient une valeur ou une série numériques. [\xref{http://www.tei-c.org/release/doc/tei-p5-doc/en/html/FS.html\#FSSY}{18.3. Other Atomic Feature Values}]\end{specHead} 
    \item[{Module}]
  iso-fs
    \item[{Attributs}]
  Attributs \hyperref[TEI.att.global]{att.global} (\textit{@xml:id}, \textit{@n}, \textit{@xml:lang}, \textit{@xml:base}, \textit{@xml:space})  (\hyperref[TEI.att.global.rendition]{att.global.rendition} (\textit{@rend}, \textit{@style}, \textit{@rendition})) (\hyperref[TEI.att.global.linking]{att.global.linking} (\textit{@corresp}, \textit{@synch}, \textit{@sameAs}, \textit{@copyOf}, \textit{@next}, \textit{@prev}, \textit{@exclude}, \textit{@select})) (\hyperref[TEI.att.global.analytic]{att.global.analytic} (\textit{@ana})) (\hyperref[TEI.att.global.facs]{att.global.facs} (\textit{@facs})) (\hyperref[TEI.att.global.change]{att.global.change} (\textit{@change})) (\hyperref[TEI.att.global.responsibility]{att.global.responsibility} (\textit{@cert}, \textit{@resp})) (\hyperref[TEI.att.global.source]{att.global.source} (\textit{@source})) \hyperref[TEI.att.datcat]{att.datcat} (\textit{@datcat}, \textit{@valueDatcat}) \hfil\\[-10pt]\begin{sansreflist}
    \item[@value]
  donne une limite inférieure pour la valeur numérique représentée et aussi (si{\itshape max} n'est pas donné) sa limite supérieure.
\begin{reflist}
    \item[{Statut}]
  Requis
    \item[{Type de données}]
  \hyperref[TEI.teidata.numeric]{teidata.numeric}
\end{reflist}  
    \item[@max]
  donne une limite supérieure pour la valeur numérique représentée.
\begin{reflist}
    \item[{Statut}]
  Optionel
    \item[{Type de données}]
  \hyperref[TEI.teidata.numeric]{teidata.numeric}
\end{reflist}  
    \item[@trunc]
  spécifie si la valeur représentée doit être tronquée pour donner un nombre entier.
\begin{reflist}
    \item[{Statut}]
  Optionel
    \item[{Type de données}]
  \hyperref[TEI.teidata.truthValue]{teidata.truthValue}
\end{reflist}  
\end{sansreflist}  
    \item[{Membre du}]
  \hyperref[TEI.model.featureVal.single]{model.featureVal.single}
    \item[{Contenu dans}]
  
    \item[iso-fs: ]
   \hyperref[TEI.f]{f} \hyperref[TEI.fvLib]{fvLib} \hyperref[TEI.if]{if} \hyperref[TEI.vAlt]{vAlt} \hyperref[TEI.vColl]{vColl} \hyperref[TEI.vDefault]{vDefault} \hyperref[TEI.vLabel]{vLabel} \hyperref[TEI.vMerge]{vMerge} \hyperref[TEI.vNot]{vNot} \hyperref[TEI.vRange]{vRange}
    \item[{Peut contenir}]
  Des données textuelles uniquement
    \item[{Note}]
  \par
C'est une erreur d'utiliser l'attribut {\itshape max} s'il n'y a pas de valeur pour l'attribut {\itshape value}.
    \item[{Exemple}]
  l'élément \hyperref[TEI.numeric]{<numeric>}contient la valeur de la fréquence\leavevmode\bgroup\exampleFont \begin{shaded}\noindent\mbox{}{<\textbf{f}\hspace*{6pt}{name}="{frequency}">}\mbox{}\newline 
\hspace*{6pt}{<\textbf{numeric}\hspace*{6pt}{value}="{2}"/>}\mbox{}\newline 
{</\textbf{f}>}\end{shaded}\egroup 


    \item[{Modèle de contenu}]
  \fbox{\ttfamily <content>\newline
 <macroRef key="macro.xtext"/>\newline
</content>\newline
    } 
    \item[{Schéma Declaration}]
  \mbox{}\hfill\\[-10pt]\begin{Verbatim}[fontsize=\small]
element numeric
{
   tei_att.global.attributes,
   tei_att.datcat.attributes,
   attribute value { text },
   attribute max { text }?,
   attribute trunc { text }?,
   tei_macro.xtext}
\end{Verbatim}

\end{reflist}  \index{objectDesc=<objectDesc>|oddindex}\index{form=@form!<objectDesc>|oddindex}
\begin{reflist}
\item[]\begin{specHead}{TEI.objectDesc}{<objectDesc> }(description d'objet) contient la description des composants matériels de l'objet en cours de traitement [\xref{http://www.tei-c.org/release/doc/tei-p5-doc/en/html/MS.html\#msph1}{10.7.1. Object Description}]\end{specHead} 
    \item[{Module}]
  msdescription
    \item[{Attributs}]
  Attributs \hyperref[TEI.att.global]{att.global} (\textit{@xml:id}, \textit{@n}, \textit{@xml:lang}, \textit{@xml:base}, \textit{@xml:space})  (\hyperref[TEI.att.global.rendition]{att.global.rendition} (\textit{@rend}, \textit{@style}, \textit{@rendition})) (\hyperref[TEI.att.global.linking]{att.global.linking} (\textit{@corresp}, \textit{@synch}, \textit{@sameAs}, \textit{@copyOf}, \textit{@next}, \textit{@prev}, \textit{@exclude}, \textit{@select})) (\hyperref[TEI.att.global.analytic]{att.global.analytic} (\textit{@ana})) (\hyperref[TEI.att.global.facs]{att.global.facs} (\textit{@facs})) (\hyperref[TEI.att.global.change]{att.global.change} (\textit{@change})) (\hyperref[TEI.att.global.responsibility]{att.global.responsibility} (\textit{@cert}, \textit{@resp})) (\hyperref[TEI.att.global.source]{att.global.source} (\textit{@source})) \hfil\\[-10pt]\begin{sansreflist}
    \item[@form]
  (forme) contient un nom abrégé spécifique au projet, désignant la forme physique du support, par exemple : codex, rouleau, fragment, fragment de feuillet, découpe, etc.
\begin{reflist}
    \item[{Statut}]
  Optionel
    \item[{Type de données}]
  \hyperref[TEI.teidata.enumerated]{teidata.enumerated}
\end{reflist}  
\end{sansreflist}  
    \item[{Membre du}]
  \hyperref[TEI.model.physDescPart]{model.physDescPart}
    \item[{Contenu dans}]
  
    \item[msdescription: ]
   \hyperref[TEI.physDesc]{physDesc}
    \item[{Peut contenir}]
  
    \item[core: ]
   \hyperref[TEI.p]{p}\par 
    \item[linking: ]
   \hyperref[TEI.ab]{ab}\par 
    \item[msdescription: ]
   \hyperref[TEI.layoutDesc]{layoutDesc} \hyperref[TEI.supportDesc]{supportDesc}
    \item[{Exemple}]
  \leavevmode\bgroup\exampleFont \begin{shaded}\noindent\mbox{}{<\textbf{objectDesc}>}\mbox{}\newline 
\hspace*{6pt}{<\textbf{supportDesc}>}\mbox{}\newline 
\hspace*{6pt}\hspace*{6pt}{<\textbf{extent}>}\mbox{}\newline 
\hspace*{6pt}\hspace*{6pt}\hspace*{6pt}{<\textbf{dimensions}\hspace*{6pt}{type}="{binding}">}\mbox{}\newline 
\hspace*{6pt}\hspace*{6pt}\hspace*{6pt}\hspace*{6pt}{<\textbf{height}\hspace*{6pt}{unit}="{mm}">}168{</\textbf{height}>}\mbox{}\newline 
\hspace*{6pt}\hspace*{6pt}\hspace*{6pt}\hspace*{6pt}{<\textbf{width}\hspace*{6pt}{unit}="{mm}">}106{</\textbf{width}>}\mbox{}\newline 
\hspace*{6pt}\hspace*{6pt}\hspace*{6pt}\hspace*{6pt}{<\textbf{depth}\hspace*{6pt}{unit}="{mm}">}22{</\textbf{depth}>}\mbox{}\newline 
\hspace*{6pt}\hspace*{6pt}\hspace*{6pt}{</\textbf{dimensions}>}\mbox{}\newline 
\hspace*{6pt}\hspace*{6pt}{</\textbf{extent}>}\mbox{}\newline 
\hspace*{6pt}{</\textbf{supportDesc}>}\mbox{}\newline 
{</\textbf{objectDesc}>}\end{shaded}\egroup 


    \item[{Modèle de contenu}]
  \mbox{}\hfill\\[-10pt]\begin{Verbatim}[fontsize=\small]
<content>
 <alternate maxOccurs="1" minOccurs="1">
  <classRef key="model.pLike"
   maxOccurs="unbounded" minOccurs="1"/>
  <sequence maxOccurs="1" minOccurs="1">
   <elementRef key="supportDesc"
    minOccurs="0"/>
   <elementRef key="layoutDesc"
    minOccurs="0"/>
  </sequence>
 </alternate>
</content>
    
\end{Verbatim}

    \item[{Schéma Declaration}]
  \mbox{}\hfill\\[-10pt]\begin{Verbatim}[fontsize=\small]
element objectDesc
{
   tei_att.global.attributes,
   attribute form { text }?,
   ( tei_model.pLike+ | ( tei_supportDesc?, tei_layoutDesc? ) )
}
\end{Verbatim}

\end{reflist}  \index{objectType=<objectType>|oddindex}
\begin{reflist}
\item[]\begin{specHead}{TEI.objectType}{<objectType> }(type d'objet) contient un mot ou une expression qui décrit le type de l'objet consideré. [\xref{http://www.tei-c.org/release/doc/tei-p5-doc/en/html/MS.html\#msmat}{10.3.2. Material and Object Type}]\end{specHead} 
    \item[{Module}]
  msdescription
    \item[{Attributs}]
  Attributs \hyperref[TEI.att.global]{att.global} (\textit{@xml:id}, \textit{@n}, \textit{@xml:lang}, \textit{@xml:base}, \textit{@xml:space})  (\hyperref[TEI.att.global.rendition]{att.global.rendition} (\textit{@rend}, \textit{@style}, \textit{@rendition})) (\hyperref[TEI.att.global.linking]{att.global.linking} (\textit{@corresp}, \textit{@synch}, \textit{@sameAs}, \textit{@copyOf}, \textit{@next}, \textit{@prev}, \textit{@exclude}, \textit{@select})) (\hyperref[TEI.att.global.analytic]{att.global.analytic} (\textit{@ana})) (\hyperref[TEI.att.global.facs]{att.global.facs} (\textit{@facs})) (\hyperref[TEI.att.global.change]{att.global.change} (\textit{@change})) (\hyperref[TEI.att.global.responsibility]{att.global.responsibility} (\textit{@cert}, \textit{@resp})) (\hyperref[TEI.att.global.source]{att.global.source} (\textit{@source})) \hyperref[TEI.att.canonical]{att.canonical} (\textit{@key}, \textit{@ref}) 
    \item[{Membre du}]
  \hyperref[TEI.model.pPart.msdesc]{model.pPart.msdesc}
    \item[{Contenu dans}]
  
    \item[analysis: ]
   \hyperref[TEI.cl]{cl} \hyperref[TEI.phr]{phr} \hyperref[TEI.s]{s} \hyperref[TEI.span]{span}\par 
    \item[core: ]
   \hyperref[TEI.abbr]{abbr} \hyperref[TEI.add]{add} \hyperref[TEI.addrLine]{addrLine} \hyperref[TEI.author]{author} \hyperref[TEI.biblScope]{biblScope} \hyperref[TEI.citedRange]{citedRange} \hyperref[TEI.corr]{corr} \hyperref[TEI.date]{date} \hyperref[TEI.del]{del} \hyperref[TEI.desc]{desc} \hyperref[TEI.distinct]{distinct} \hyperref[TEI.editor]{editor} \hyperref[TEI.email]{email} \hyperref[TEI.emph]{emph} \hyperref[TEI.expan]{expan} \hyperref[TEI.foreign]{foreign} \hyperref[TEI.gloss]{gloss} \hyperref[TEI.head]{head} \hyperref[TEI.headItem]{headItem} \hyperref[TEI.headLabel]{headLabel} \hyperref[TEI.hi]{hi} \hyperref[TEI.item]{item} \hyperref[TEI.l]{l} \hyperref[TEI.label]{label} \hyperref[TEI.measure]{measure} \hyperref[TEI.meeting]{meeting} \hyperref[TEI.mentioned]{mentioned} \hyperref[TEI.name]{name} \hyperref[TEI.note]{note} \hyperref[TEI.num]{num} \hyperref[TEI.orig]{orig} \hyperref[TEI.p]{p} \hyperref[TEI.pubPlace]{pubPlace} \hyperref[TEI.publisher]{publisher} \hyperref[TEI.q]{q} \hyperref[TEI.quote]{quote} \hyperref[TEI.ref]{ref} \hyperref[TEI.reg]{reg} \hyperref[TEI.resp]{resp} \hyperref[TEI.rs]{rs} \hyperref[TEI.said]{said} \hyperref[TEI.sic]{sic} \hyperref[TEI.soCalled]{soCalled} \hyperref[TEI.speaker]{speaker} \hyperref[TEI.stage]{stage} \hyperref[TEI.street]{street} \hyperref[TEI.term]{term} \hyperref[TEI.textLang]{textLang} \hyperref[TEI.time]{time} \hyperref[TEI.title]{title} \hyperref[TEI.unclear]{unclear}\par 
    \item[figures: ]
   \hyperref[TEI.cell]{cell} \hyperref[TEI.figDesc]{figDesc}\par 
    \item[header: ]
   \hyperref[TEI.authority]{authority} \hyperref[TEI.change]{change} \hyperref[TEI.classCode]{classCode} \hyperref[TEI.creation]{creation} \hyperref[TEI.distributor]{distributor} \hyperref[TEI.edition]{edition} \hyperref[TEI.extent]{extent} \hyperref[TEI.funder]{funder} \hyperref[TEI.language]{language} \hyperref[TEI.licence]{licence} \hyperref[TEI.rendition]{rendition}\par 
    \item[iso-fs: ]
   \hyperref[TEI.fDescr]{fDescr} \hyperref[TEI.fsDescr]{fsDescr}\par 
    \item[linking: ]
   \hyperref[TEI.ab]{ab} \hyperref[TEI.seg]{seg}\par 
    \item[msdescription: ]
   \hyperref[TEI.accMat]{accMat} \hyperref[TEI.acquisition]{acquisition} \hyperref[TEI.additions]{additions} \hyperref[TEI.catchwords]{catchwords} \hyperref[TEI.collation]{collation} \hyperref[TEI.colophon]{colophon} \hyperref[TEI.condition]{condition} \hyperref[TEI.custEvent]{custEvent} \hyperref[TEI.decoNote]{decoNote} \hyperref[TEI.explicit]{explicit} \hyperref[TEI.filiation]{filiation} \hyperref[TEI.finalRubric]{finalRubric} \hyperref[TEI.foliation]{foliation} \hyperref[TEI.heraldry]{heraldry} \hyperref[TEI.incipit]{incipit} \hyperref[TEI.layout]{layout} \hyperref[TEI.material]{material} \hyperref[TEI.musicNotation]{musicNotation} \hyperref[TEI.objectType]{objectType} \hyperref[TEI.origDate]{origDate} \hyperref[TEI.origPlace]{origPlace} \hyperref[TEI.origin]{origin} \hyperref[TEI.provenance]{provenance} \hyperref[TEI.rubric]{rubric} \hyperref[TEI.secFol]{secFol} \hyperref[TEI.signatures]{signatures} \hyperref[TEI.source]{source} \hyperref[TEI.stamp]{stamp} \hyperref[TEI.summary]{summary} \hyperref[TEI.support]{support} \hyperref[TEI.surrogates]{surrogates} \hyperref[TEI.typeNote]{typeNote} \hyperref[TEI.watermark]{watermark}\par 
    \item[namesdates: ]
   \hyperref[TEI.addName]{addName} \hyperref[TEI.affiliation]{affiliation} \hyperref[TEI.country]{country} \hyperref[TEI.forename]{forename} \hyperref[TEI.genName]{genName} \hyperref[TEI.geogName]{geogName} \hyperref[TEI.nameLink]{nameLink} \hyperref[TEI.orgName]{orgName} \hyperref[TEI.persName]{persName} \hyperref[TEI.placeName]{placeName} \hyperref[TEI.region]{region} \hyperref[TEI.roleName]{roleName} \hyperref[TEI.settlement]{settlement} \hyperref[TEI.surname]{surname}\par 
    \item[textstructure: ]
   \hyperref[TEI.docAuthor]{docAuthor} \hyperref[TEI.docDate]{docDate} \hyperref[TEI.docEdition]{docEdition} \hyperref[TEI.titlePart]{titlePart}\par 
    \item[transcr: ]
   \hyperref[TEI.damage]{damage} \hyperref[TEI.fw]{fw} \hyperref[TEI.metamark]{metamark} \hyperref[TEI.mod]{mod} \hyperref[TEI.restore]{restore} \hyperref[TEI.retrace]{retrace} \hyperref[TEI.secl]{secl} \hyperref[TEI.supplied]{supplied} \hyperref[TEI.surplus]{surplus}
    \item[{Peut contenir}]
  
    \item[analysis: ]
   \hyperref[TEI.c]{c} \hyperref[TEI.cl]{cl} \hyperref[TEI.interp]{interp} \hyperref[TEI.interpGrp]{interpGrp} \hyperref[TEI.m]{m} \hyperref[TEI.pc]{pc} \hyperref[TEI.phr]{phr} \hyperref[TEI.s]{s} \hyperref[TEI.span]{span} \hyperref[TEI.spanGrp]{spanGrp} \hyperref[TEI.w]{w}\par 
    \item[core: ]
   \hyperref[TEI.abbr]{abbr} \hyperref[TEI.add]{add} \hyperref[TEI.address]{address} \hyperref[TEI.binaryObject]{binaryObject} \hyperref[TEI.cb]{cb} \hyperref[TEI.choice]{choice} \hyperref[TEI.corr]{corr} \hyperref[TEI.date]{date} \hyperref[TEI.del]{del} \hyperref[TEI.distinct]{distinct} \hyperref[TEI.email]{email} \hyperref[TEI.emph]{emph} \hyperref[TEI.expan]{expan} \hyperref[TEI.foreign]{foreign} \hyperref[TEI.gap]{gap} \hyperref[TEI.gb]{gb} \hyperref[TEI.gloss]{gloss} \hyperref[TEI.graphic]{graphic} \hyperref[TEI.hi]{hi} \hyperref[TEI.index]{index} \hyperref[TEI.lb]{lb} \hyperref[TEI.measure]{measure} \hyperref[TEI.measureGrp]{measureGrp} \hyperref[TEI.media]{media} \hyperref[TEI.mentioned]{mentioned} \hyperref[TEI.milestone]{milestone} \hyperref[TEI.name]{name} \hyperref[TEI.note]{note} \hyperref[TEI.num]{num} \hyperref[TEI.orig]{orig} \hyperref[TEI.pb]{pb} \hyperref[TEI.ptr]{ptr} \hyperref[TEI.ref]{ref} \hyperref[TEI.reg]{reg} \hyperref[TEI.rs]{rs} \hyperref[TEI.sic]{sic} \hyperref[TEI.soCalled]{soCalled} \hyperref[TEI.term]{term} \hyperref[TEI.time]{time} \hyperref[TEI.title]{title} \hyperref[TEI.unclear]{unclear}\par 
    \item[derived-module-tei.istex: ]
   \hyperref[TEI.math]{math} \hyperref[TEI.mrow]{mrow}\par 
    \item[figures: ]
   \hyperref[TEI.figure]{figure} \hyperref[TEI.formula]{formula} \hyperref[TEI.notatedMusic]{notatedMusic}\par 
    \item[header: ]
   \hyperref[TEI.idno]{idno}\par 
    \item[iso-fs: ]
   \hyperref[TEI.fLib]{fLib} \hyperref[TEI.fs]{fs} \hyperref[TEI.fvLib]{fvLib}\par 
    \item[linking: ]
   \hyperref[TEI.alt]{alt} \hyperref[TEI.altGrp]{altGrp} \hyperref[TEI.anchor]{anchor} \hyperref[TEI.join]{join} \hyperref[TEI.joinGrp]{joinGrp} \hyperref[TEI.link]{link} \hyperref[TEI.linkGrp]{linkGrp} \hyperref[TEI.seg]{seg} \hyperref[TEI.timeline]{timeline}\par 
    \item[msdescription: ]
   \hyperref[TEI.catchwords]{catchwords} \hyperref[TEI.depth]{depth} \hyperref[TEI.dim]{dim} \hyperref[TEI.dimensions]{dimensions} \hyperref[TEI.height]{height} \hyperref[TEI.heraldry]{heraldry} \hyperref[TEI.locus]{locus} \hyperref[TEI.locusGrp]{locusGrp} \hyperref[TEI.material]{material} \hyperref[TEI.objectType]{objectType} \hyperref[TEI.origDate]{origDate} \hyperref[TEI.origPlace]{origPlace} \hyperref[TEI.secFol]{secFol} \hyperref[TEI.signatures]{signatures} \hyperref[TEI.source]{source} \hyperref[TEI.stamp]{stamp} \hyperref[TEI.watermark]{watermark} \hyperref[TEI.width]{width}\par 
    \item[namesdates: ]
   \hyperref[TEI.addName]{addName} \hyperref[TEI.affiliation]{affiliation} \hyperref[TEI.country]{country} \hyperref[TEI.forename]{forename} \hyperref[TEI.genName]{genName} \hyperref[TEI.geogName]{geogName} \hyperref[TEI.location]{location} \hyperref[TEI.nameLink]{nameLink} \hyperref[TEI.orgName]{orgName} \hyperref[TEI.persName]{persName} \hyperref[TEI.placeName]{placeName} \hyperref[TEI.region]{region} \hyperref[TEI.roleName]{roleName} \hyperref[TEI.settlement]{settlement} \hyperref[TEI.state]{state} \hyperref[TEI.surname]{surname}\par 
    \item[spoken: ]
   \hyperref[TEI.annotationBlock]{annotationBlock}\par 
    \item[transcr: ]
   \hyperref[TEI.addSpan]{addSpan} \hyperref[TEI.am]{am} \hyperref[TEI.damage]{damage} \hyperref[TEI.damageSpan]{damageSpan} \hyperref[TEI.delSpan]{delSpan} \hyperref[TEI.ex]{ex} \hyperref[TEI.fw]{fw} \hyperref[TEI.handShift]{handShift} \hyperref[TEI.listTranspose]{listTranspose} \hyperref[TEI.metamark]{metamark} \hyperref[TEI.mod]{mod} \hyperref[TEI.redo]{redo} \hyperref[TEI.restore]{restore} \hyperref[TEI.retrace]{retrace} \hyperref[TEI.secl]{secl} \hyperref[TEI.space]{space} \hyperref[TEI.subst]{subst} \hyperref[TEI.substJoin]{substJoin} \hyperref[TEI.supplied]{supplied} \hyperref[TEI.surplus]{surplus} \hyperref[TEI.undo]{undo}\par des données textuelles
    \item[{Note}]
  \par
The {\itshape ref} attribute may be used to point to one or more items within a taxonomy of types of object, defined either internally or externally.
    \item[{Exemple}]
  \leavevmode\bgroup\exampleFont \begin{shaded}\noindent\mbox{}{<\textbf{physDesc}>}\mbox{}\newline 
\hspace*{6pt}{<\textbf{p}>}\mbox{}\newline 
\hspace*{6pt}\hspace*{6pt}{<\textbf{objectType}>}Codex{</\textbf{objectType}>} avec feuilles de parchemin colorées avec la pourpre du murex.\mbox{}\newline 
\hspace*{6pt}{</\textbf{p}>}\mbox{}\newline 
{</\textbf{physDesc}>}\end{shaded}\egroup 


    \item[{Exemple}]
  \leavevmode\bgroup\exampleFont \begin{shaded}\noindent\mbox{}{<\textbf{physDesc}>}\mbox{}\newline 
\hspace*{6pt}{<\textbf{p}>}\mbox{}\newline 
\hspace*{6pt}\hspace*{6pt}{<\textbf{objectType}>}Socle{</\textbf{objectType}>} fragmentaire d'Aphrodite Anadyomène en terre cuite.{</\textbf{p}>}\mbox{}\newline 
{</\textbf{physDesc}>}\end{shaded}\egroup 


    \item[{Modèle de contenu}]
  \mbox{}\hfill\\[-10pt]\begin{Verbatim}[fontsize=\small]
<content>
 <macroRef key="macro.phraseSeq"/>
</content>
    
\end{Verbatim}

    \item[{Schéma Declaration}]
  \mbox{}\hfill\\[-10pt]\begin{Verbatim}[fontsize=\small]
element objectType
{
   tei_att.global.attributes,
   tei_att.canonical.attributes,
   tei_macro.phraseSeq}
\end{Verbatim}

\end{reflist}  \index{org=<org>|oddindex}\index{role=@role!<org>|oddindex}
\begin{reflist}
\item[]\begin{specHead}{TEI.org}{<org> }(organisation) fournit des informations sur une organisation identifiable, telle qu'une entreprise, une tribu ou tout autre groupement de personnes. [\xref{http://www.tei-c.org/release/doc/tei-p5-doc/en/html/ND.html\#NDORG}{13.2.2. Organizational Names}]\end{specHead} 
    \item[{Module}]
  namesdates
    \item[{Attributs}]
  Attributs \hyperref[TEI.att.global]{att.global} (\textit{@xml:id}, \textit{@n}, \textit{@xml:lang}, \textit{@xml:base}, \textit{@xml:space})  (\hyperref[TEI.att.global.rendition]{att.global.rendition} (\textit{@rend}, \textit{@style}, \textit{@rendition})) (\hyperref[TEI.att.global.linking]{att.global.linking} (\textit{@corresp}, \textit{@synch}, \textit{@sameAs}, \textit{@copyOf}, \textit{@next}, \textit{@prev}, \textit{@exclude}, \textit{@select})) (\hyperref[TEI.att.global.analytic]{att.global.analytic} (\textit{@ana})) (\hyperref[TEI.att.global.facs]{att.global.facs} (\textit{@facs})) (\hyperref[TEI.att.global.change]{att.global.change} (\textit{@change})) (\hyperref[TEI.att.global.responsibility]{att.global.responsibility} (\textit{@cert}, \textit{@resp})) (\hyperref[TEI.att.global.source]{att.global.source} (\textit{@source})) \hyperref[TEI.att.typed]{att.typed} (\textit{@type}, \textit{@subtype}) \hyperref[TEI.att.editLike]{att.editLike} (\textit{@evidence}, \textit{@instant})  (\hyperref[TEI.att.dimensions]{att.dimensions} (\textit{@unit}, \textit{@quantity}, \textit{@extent}, \textit{@precision}, \textit{@scope}) (\hyperref[TEI.att.ranging]{att.ranging} (\textit{@atLeast}, \textit{@atMost}, \textit{@min}, \textit{@max}, \textit{@confidence})) ) \hyperref[TEI.att.sortable]{att.sortable} (\textit{@sortKey}) \hfil\\[-10pt]\begin{sansreflist}
    \item[@role]
  spécifie le rôle principal ou la catégorie d'une organisation.
\begin{reflist}
    \item[{Statut}]
  Optionel
    \item[{Type de données}]
  1–∞ occurrences de \hyperref[TEI.teidata.word]{teidata.word} séparé par un espace
\end{reflist}  
\end{sansreflist}  
    \item[{Membre du}]
  \hyperref[TEI.model.OABody]{model.OABody} \hyperref[TEI.model.personLike]{model.personLike}
    \item[{Contenu dans}]
  
    \item[namesdates: ]
   \hyperref[TEI.listOrg]{listOrg} \hyperref[TEI.org]{org}\par 
    \item[spoken: ]
   \hyperref[TEI.annotationBlock]{annotationBlock}
    \item[{Peut contenir}]
  
    \item[core: ]
   \hyperref[TEI.bibl]{bibl} \hyperref[TEI.biblStruct]{biblStruct} \hyperref[TEI.cb]{cb} \hyperref[TEI.desc]{desc} \hyperref[TEI.gb]{gb} \hyperref[TEI.head]{head} \hyperref[TEI.label]{label} \hyperref[TEI.lb]{lb} \hyperref[TEI.listBibl]{listBibl} \hyperref[TEI.milestone]{milestone} \hyperref[TEI.name]{name} \hyperref[TEI.note]{note} \hyperref[TEI.p]{p} \hyperref[TEI.pb]{pb} \hyperref[TEI.rs]{rs}\par 
    \item[header: ]
   \hyperref[TEI.biblFull]{biblFull} \hyperref[TEI.idno]{idno}\par 
    \item[linking: ]
   \hyperref[TEI.ab]{ab} \hyperref[TEI.anchor]{anchor} \hyperref[TEI.link]{link} \hyperref[TEI.linkGrp]{linkGrp}\par 
    \item[msdescription: ]
   \hyperref[TEI.msDesc]{msDesc}\par 
    \item[namesdates: ]
   \hyperref[TEI.addName]{addName} \hyperref[TEI.country]{country} \hyperref[TEI.event]{event} \hyperref[TEI.forename]{forename} \hyperref[TEI.genName]{genName} \hyperref[TEI.geogName]{geogName} \hyperref[TEI.listOrg]{listOrg} \hyperref[TEI.listPlace]{listPlace} \hyperref[TEI.location]{location} \hyperref[TEI.nameLink]{nameLink} \hyperref[TEI.org]{org} \hyperref[TEI.orgName]{orgName} \hyperref[TEI.persName]{persName} \hyperref[TEI.person]{person} \hyperref[TEI.personGrp]{personGrp} \hyperref[TEI.place]{place} \hyperref[TEI.placeName]{placeName} \hyperref[TEI.region]{region} \hyperref[TEI.roleName]{roleName} \hyperref[TEI.settlement]{settlement} \hyperref[TEI.state]{state} \hyperref[TEI.surname]{surname}\par 
    \item[transcr: ]
   \hyperref[TEI.fw]{fw}
    \item[{Exemple}]
  \leavevmode\bgroup\exampleFont \begin{shaded}\noindent\mbox{}{<\textbf{org}\hspace*{6pt}{xml:id}="{JAMs}">}\mbox{}\newline 
\hspace*{6pt}{<\textbf{orgName}>}Justified Ancients of Mummu{</\textbf{orgName}>}\mbox{}\newline 
\hspace*{6pt}{<\textbf{desc}>}An underground anarchist collective spearheaded by {<\textbf{persName}>}Hagbard\mbox{}\newline 
\hspace*{6pt}\hspace*{6pt}\hspace*{6pt}\hspace*{6pt} Celine{</\textbf{persName}>}, who fight the Illuminati from a golden submarine, the\mbox{}\newline 
\hspace*{6pt}{<\textbf{name}>}Leif Ericson{</\textbf{name}>}\mbox{}\newline 
\hspace*{6pt}{</\textbf{desc}>}\mbox{}\newline 
\hspace*{6pt}{<\textbf{bibl}>}\mbox{}\newline 
\hspace*{6pt}\hspace*{6pt}{<\textbf{author}>}Robert Shea{</\textbf{author}>}\mbox{}\newline 
\hspace*{6pt}\hspace*{6pt}{<\textbf{author}>}Robert Anton Wilson{</\textbf{author}>}\mbox{}\newline 
\hspace*{6pt}\hspace*{6pt}{<\textbf{title}>}The Illuminatus! Trilogy{</\textbf{title}>}\mbox{}\newline 
\hspace*{6pt}{</\textbf{bibl}>}\mbox{}\newline 
{</\textbf{org}>}\end{shaded}\egroup 


    \item[{Modèle de contenu}]
  \mbox{}\hfill\\[-10pt]\begin{Verbatim}[fontsize=\small]
<content>
 <sequence maxOccurs="1" minOccurs="1">
  <classRef key="model.headLike"
   maxOccurs="unbounded" minOccurs="0"/>
  <alternate maxOccurs="1" minOccurs="1">
   <classRef key="model.pLike"
    maxOccurs="unbounded" minOccurs="0"/>
   <alternate maxOccurs="unbounded"
    minOccurs="0">
    <classRef key="model.labelLike"/>
    <classRef key="model.nameLike"/>
    <classRef key="model.placeLike"/>
    <classRef key="model.orgPart"/>
    <classRef key="model.milestoneLike"/>
   </alternate>
  </alternate>
  <alternate maxOccurs="unbounded"
   minOccurs="0">
   <classRef key="model.noteLike"/>
   <classRef key="model.biblLike"/>
   <elementRef key="linkGrp"/>
   <elementRef key="link"/>
  </alternate>
  <classRef key="model.personLike"
   maxOccurs="unbounded" minOccurs="0"/>
 </sequence>
</content>
    
\end{Verbatim}

    \item[{Schéma Declaration}]
  \mbox{}\hfill\\[-10pt]\begin{Verbatim}[fontsize=\small]
element org
{
   tei_att.global.attributes,
   tei_att.typed.attributes,
   tei_att.editLike.attributes,
   tei_att.sortable.attributes,
   attribute role { list { + } }?,
   (
      tei_model.headLike*,
      (
         tei_model.pLike*
       | (
            tei_model.labelLike          | tei_model.nameLike          | tei_model.placeLike          | tei_model.orgPart          | tei_model.milestoneLike         )*
      ),
      ( tei_model.noteLike | tei_model.biblLike | tei_linkGrp | tei_link )*,
      tei_model.personLike*
   )
}
\end{Verbatim}

\end{reflist}  \index{orgName=<orgName>|oddindex}\index{scheme=@scheme!<orgName>|oddindex}
\begin{reflist}
\item[]\begin{specHead}{TEI.orgName}{<orgName> }(nom d'organisation) contient le nom d'une organisation. [\xref{http://www.tei-c.org/release/doc/tei-p5-doc/en/html/ND.html\#NDORG}{13.2.2. Organizational Names}]\end{specHead} 
    \item[{Module}]
  namesdates
    \item[{Attributs}]
  Attributs \hyperref[TEI.att.global]{att.global} (\textit{@xml:id}, \textit{@n}, \textit{@xml:lang}, \textit{@xml:base}, \textit{@xml:space})  (\hyperref[TEI.att.global.rendition]{att.global.rendition} (\textit{@rend}, \textit{@style}, \textit{@rendition})) (\hyperref[TEI.att.global.linking]{att.global.linking} (\textit{@corresp}, \textit{@synch}, \textit{@sameAs}, \textit{@copyOf}, \textit{@next}, \textit{@prev}, \textit{@exclude}, \textit{@select})) (\hyperref[TEI.att.global.analytic]{att.global.analytic} (\textit{@ana})) (\hyperref[TEI.att.global.facs]{att.global.facs} (\textit{@facs})) (\hyperref[TEI.att.global.change]{att.global.change} (\textit{@change})) (\hyperref[TEI.att.global.responsibility]{att.global.responsibility} (\textit{@cert}, \textit{@resp})) (\hyperref[TEI.att.global.source]{att.global.source} (\textit{@source})) \hyperref[TEI.att.datable]{att.datable} (\textit{@calendar}, \textit{@period})  (\hyperref[TEI.att.datable.w3c]{att.datable.w3c} (\textit{@when}, \textit{@notBefore}, \textit{@notAfter}, \textit{@from}, \textit{@to})) (\hyperref[TEI.att.datable.iso]{att.datable.iso} (\textit{@when-iso}, \textit{@notBefore-iso}, \textit{@notAfter-iso}, \textit{@from-iso}, \textit{@to-iso})) (\hyperref[TEI.att.datable.custom]{att.datable.custom} (\textit{@when-custom}, \textit{@notBefore-custom}, \textit{@notAfter-custom}, \textit{@from-custom}, \textit{@to-custom}, \textit{@datingPoint}, \textit{@datingMethod})) \hyperref[TEI.att.editLike]{att.editLike} (\textit{@evidence}, \textit{@instant})  (\hyperref[TEI.att.dimensions]{att.dimensions} (\textit{@unit}, \textit{@quantity}, \textit{@extent}, \textit{@precision}, \textit{@scope}) (\hyperref[TEI.att.ranging]{att.ranging} (\textit{@atLeast}, \textit{@atMost}, \textit{@min}, \textit{@max}, \textit{@confidence})) ) \hyperref[TEI.att.personal]{att.personal} (\textit{@full}, \textit{@sort})  (\hyperref[TEI.att.naming]{att.naming} (\textit{@role}, \textit{@nymRef}) (\hyperref[TEI.att.canonical]{att.canonical} (\textit{@key}, \textit{@ref})) ) \hyperref[TEI.att.typed]{att.typed} (\textit{@type}, \textit{@subtype}) \hfil\\[-10pt]\begin{sansreflist}
    \item[@scheme]
  désigne la liste des ontologies dans lequel l'ensemble des termes concernés sont définis.
\begin{reflist}
    \item[{Statut}]
  Optionel
    \item[{Type de données}]
  \hyperref[TEI.teidata.pointer]{teidata.pointer}
\end{reflist}  
\end{sansreflist}  
    \item[{Membre du}]
  \hyperref[TEI.model.nameLike.agent]{model.nameLike.agent}
    \item[{Contenu dans}]
  
    \item[analysis: ]
   \hyperref[TEI.cl]{cl} \hyperref[TEI.phr]{phr} \hyperref[TEI.s]{s} \hyperref[TEI.span]{span}\par 
    \item[core: ]
   \hyperref[TEI.abbr]{abbr} \hyperref[TEI.add]{add} \hyperref[TEI.addrLine]{addrLine} \hyperref[TEI.address]{address} \hyperref[TEI.author]{author} \hyperref[TEI.bibl]{bibl} \hyperref[TEI.biblScope]{biblScope} \hyperref[TEI.citedRange]{citedRange} \hyperref[TEI.corr]{corr} \hyperref[TEI.date]{date} \hyperref[TEI.del]{del} \hyperref[TEI.desc]{desc} \hyperref[TEI.distinct]{distinct} \hyperref[TEI.editor]{editor} \hyperref[TEI.email]{email} \hyperref[TEI.emph]{emph} \hyperref[TEI.expan]{expan} \hyperref[TEI.foreign]{foreign} \hyperref[TEI.gloss]{gloss} \hyperref[TEI.head]{head} \hyperref[TEI.headItem]{headItem} \hyperref[TEI.headLabel]{headLabel} \hyperref[TEI.hi]{hi} \hyperref[TEI.item]{item} \hyperref[TEI.l]{l} \hyperref[TEI.label]{label} \hyperref[TEI.measure]{measure} \hyperref[TEI.meeting]{meeting} \hyperref[TEI.mentioned]{mentioned} \hyperref[TEI.name]{name} \hyperref[TEI.note]{note} \hyperref[TEI.num]{num} \hyperref[TEI.orig]{orig} \hyperref[TEI.p]{p} \hyperref[TEI.pubPlace]{pubPlace} \hyperref[TEI.publisher]{publisher} \hyperref[TEI.q]{q} \hyperref[TEI.quote]{quote} \hyperref[TEI.ref]{ref} \hyperref[TEI.reg]{reg} \hyperref[TEI.resp]{resp} \hyperref[TEI.respStmt]{respStmt} \hyperref[TEI.rs]{rs} \hyperref[TEI.said]{said} \hyperref[TEI.sic]{sic} \hyperref[TEI.soCalled]{soCalled} \hyperref[TEI.speaker]{speaker} \hyperref[TEI.stage]{stage} \hyperref[TEI.street]{street} \hyperref[TEI.term]{term} \hyperref[TEI.textLang]{textLang} \hyperref[TEI.time]{time} \hyperref[TEI.title]{title} \hyperref[TEI.unclear]{unclear}\par 
    \item[figures: ]
   \hyperref[TEI.cell]{cell} \hyperref[TEI.figDesc]{figDesc}\par 
    \item[header: ]
   \hyperref[TEI.authority]{authority} \hyperref[TEI.change]{change} \hyperref[TEI.classCode]{classCode} \hyperref[TEI.creation]{creation} \hyperref[TEI.distributor]{distributor} \hyperref[TEI.edition]{edition} \hyperref[TEI.extent]{extent} \hyperref[TEI.funder]{funder} \hyperref[TEI.language]{language} \hyperref[TEI.licence]{licence} \hyperref[TEI.rendition]{rendition}\par 
    \item[iso-fs: ]
   \hyperref[TEI.fDescr]{fDescr} \hyperref[TEI.fsDescr]{fsDescr}\par 
    \item[linking: ]
   \hyperref[TEI.ab]{ab} \hyperref[TEI.seg]{seg}\par 
    \item[msdescription: ]
   \hyperref[TEI.accMat]{accMat} \hyperref[TEI.acquisition]{acquisition} \hyperref[TEI.additions]{additions} \hyperref[TEI.catchwords]{catchwords} \hyperref[TEI.collation]{collation} \hyperref[TEI.colophon]{colophon} \hyperref[TEI.condition]{condition} \hyperref[TEI.custEvent]{custEvent} \hyperref[TEI.decoNote]{decoNote} \hyperref[TEI.explicit]{explicit} \hyperref[TEI.filiation]{filiation} \hyperref[TEI.finalRubric]{finalRubric} \hyperref[TEI.foliation]{foliation} \hyperref[TEI.heraldry]{heraldry} \hyperref[TEI.incipit]{incipit} \hyperref[TEI.layout]{layout} \hyperref[TEI.material]{material} \hyperref[TEI.musicNotation]{musicNotation} \hyperref[TEI.objectType]{objectType} \hyperref[TEI.origDate]{origDate} \hyperref[TEI.origPlace]{origPlace} \hyperref[TEI.origin]{origin} \hyperref[TEI.provenance]{provenance} \hyperref[TEI.rubric]{rubric} \hyperref[TEI.secFol]{secFol} \hyperref[TEI.signatures]{signatures} \hyperref[TEI.source]{source} \hyperref[TEI.stamp]{stamp} \hyperref[TEI.summary]{summary} \hyperref[TEI.support]{support} \hyperref[TEI.surrogates]{surrogates} \hyperref[TEI.typeNote]{typeNote} \hyperref[TEI.watermark]{watermark}\par 
    \item[namesdates: ]
   \hyperref[TEI.addName]{addName} \hyperref[TEI.affiliation]{affiliation} \hyperref[TEI.country]{country} \hyperref[TEI.forename]{forename} \hyperref[TEI.genName]{genName} \hyperref[TEI.geogName]{geogName} \hyperref[TEI.nameLink]{nameLink} \hyperref[TEI.org]{org} \hyperref[TEI.orgName]{orgName} \hyperref[TEI.persName]{persName} \hyperref[TEI.placeName]{placeName} \hyperref[TEI.region]{region} \hyperref[TEI.roleName]{roleName} \hyperref[TEI.settlement]{settlement} \hyperref[TEI.surname]{surname}\par 
    \item[spoken: ]
   \hyperref[TEI.annotationBlock]{annotationBlock}\par 
    \item[standOff: ]
   \hyperref[TEI.listAnnotation]{listAnnotation}\par 
    \item[textstructure: ]
   \hyperref[TEI.docAuthor]{docAuthor} \hyperref[TEI.docDate]{docDate} \hyperref[TEI.docEdition]{docEdition} \hyperref[TEI.titlePart]{titlePart}\par 
    \item[transcr: ]
   \hyperref[TEI.damage]{damage} \hyperref[TEI.fw]{fw} \hyperref[TEI.metamark]{metamark} \hyperref[TEI.mod]{mod} \hyperref[TEI.restore]{restore} \hyperref[TEI.retrace]{retrace} \hyperref[TEI.secl]{secl} \hyperref[TEI.supplied]{supplied} \hyperref[TEI.surplus]{surplus}
    \item[{Peut contenir}]
  
    \item[analysis: ]
   \hyperref[TEI.c]{c} \hyperref[TEI.cl]{cl} \hyperref[TEI.interp]{interp} \hyperref[TEI.interpGrp]{interpGrp} \hyperref[TEI.m]{m} \hyperref[TEI.pc]{pc} \hyperref[TEI.phr]{phr} \hyperref[TEI.s]{s} \hyperref[TEI.span]{span} \hyperref[TEI.spanGrp]{spanGrp} \hyperref[TEI.w]{w}\par 
    \item[core: ]
   \hyperref[TEI.abbr]{abbr} \hyperref[TEI.add]{add} \hyperref[TEI.address]{address} \hyperref[TEI.binaryObject]{binaryObject} \hyperref[TEI.cb]{cb} \hyperref[TEI.choice]{choice} \hyperref[TEI.corr]{corr} \hyperref[TEI.date]{date} \hyperref[TEI.del]{del} \hyperref[TEI.distinct]{distinct} \hyperref[TEI.email]{email} \hyperref[TEI.emph]{emph} \hyperref[TEI.expan]{expan} \hyperref[TEI.foreign]{foreign} \hyperref[TEI.gap]{gap} \hyperref[TEI.gb]{gb} \hyperref[TEI.gloss]{gloss} \hyperref[TEI.graphic]{graphic} \hyperref[TEI.hi]{hi} \hyperref[TEI.index]{index} \hyperref[TEI.lb]{lb} \hyperref[TEI.measure]{measure} \hyperref[TEI.measureGrp]{measureGrp} \hyperref[TEI.media]{media} \hyperref[TEI.mentioned]{mentioned} \hyperref[TEI.milestone]{milestone} \hyperref[TEI.name]{name} \hyperref[TEI.note]{note} \hyperref[TEI.num]{num} \hyperref[TEI.orig]{orig} \hyperref[TEI.pb]{pb} \hyperref[TEI.ptr]{ptr} \hyperref[TEI.ref]{ref} \hyperref[TEI.reg]{reg} \hyperref[TEI.rs]{rs} \hyperref[TEI.sic]{sic} \hyperref[TEI.soCalled]{soCalled} \hyperref[TEI.term]{term} \hyperref[TEI.time]{time} \hyperref[TEI.title]{title} \hyperref[TEI.unclear]{unclear}\par 
    \item[derived-module-tei.istex: ]
   \hyperref[TEI.math]{math} \hyperref[TEI.mrow]{mrow}\par 
    \item[figures: ]
   \hyperref[TEI.figure]{figure} \hyperref[TEI.formula]{formula} \hyperref[TEI.notatedMusic]{notatedMusic}\par 
    \item[header: ]
   \hyperref[TEI.idno]{idno}\par 
    \item[iso-fs: ]
   \hyperref[TEI.fLib]{fLib} \hyperref[TEI.fs]{fs} \hyperref[TEI.fvLib]{fvLib}\par 
    \item[linking: ]
   \hyperref[TEI.alt]{alt} \hyperref[TEI.altGrp]{altGrp} \hyperref[TEI.anchor]{anchor} \hyperref[TEI.join]{join} \hyperref[TEI.joinGrp]{joinGrp} \hyperref[TEI.link]{link} \hyperref[TEI.linkGrp]{linkGrp} \hyperref[TEI.seg]{seg} \hyperref[TEI.timeline]{timeline}\par 
    \item[msdescription: ]
   \hyperref[TEI.catchwords]{catchwords} \hyperref[TEI.depth]{depth} \hyperref[TEI.dim]{dim} \hyperref[TEI.dimensions]{dimensions} \hyperref[TEI.height]{height} \hyperref[TEI.heraldry]{heraldry} \hyperref[TEI.locus]{locus} \hyperref[TEI.locusGrp]{locusGrp} \hyperref[TEI.material]{material} \hyperref[TEI.objectType]{objectType} \hyperref[TEI.origDate]{origDate} \hyperref[TEI.origPlace]{origPlace} \hyperref[TEI.secFol]{secFol} \hyperref[TEI.signatures]{signatures} \hyperref[TEI.source]{source} \hyperref[TEI.stamp]{stamp} \hyperref[TEI.watermark]{watermark} \hyperref[TEI.width]{width}\par 
    \item[namesdates: ]
   \hyperref[TEI.addName]{addName} \hyperref[TEI.affiliation]{affiliation} \hyperref[TEI.country]{country} \hyperref[TEI.forename]{forename} \hyperref[TEI.genName]{genName} \hyperref[TEI.geogName]{geogName} \hyperref[TEI.location]{location} \hyperref[TEI.nameLink]{nameLink} \hyperref[TEI.orgName]{orgName} \hyperref[TEI.persName]{persName} \hyperref[TEI.placeName]{placeName} \hyperref[TEI.region]{region} \hyperref[TEI.roleName]{roleName} \hyperref[TEI.settlement]{settlement} \hyperref[TEI.state]{state} \hyperref[TEI.surname]{surname}\par 
    \item[spoken: ]
   \hyperref[TEI.annotationBlock]{annotationBlock}\par 
    \item[transcr: ]
   \hyperref[TEI.addSpan]{addSpan} \hyperref[TEI.am]{am} \hyperref[TEI.damage]{damage} \hyperref[TEI.damageSpan]{damageSpan} \hyperref[TEI.delSpan]{delSpan} \hyperref[TEI.ex]{ex} \hyperref[TEI.fw]{fw} \hyperref[TEI.handShift]{handShift} \hyperref[TEI.listTranspose]{listTranspose} \hyperref[TEI.metamark]{metamark} \hyperref[TEI.mod]{mod} \hyperref[TEI.redo]{redo} \hyperref[TEI.restore]{restore} \hyperref[TEI.retrace]{retrace} \hyperref[TEI.secl]{secl} \hyperref[TEI.space]{space} \hyperref[TEI.subst]{subst} \hyperref[TEI.substJoin]{substJoin} \hyperref[TEI.supplied]{supplied} \hyperref[TEI.surplus]{surplus} \hyperref[TEI.undo]{undo}\par des données textuelles
    \item[{Exemple}]
  StandOff entité nommée orgName\leavevmode\bgroup\exampleFont \begin{shaded}\noindent\mbox{}{<\textbf{annotationBlock}\hspace*{6pt}{corresp}="{text}">}\mbox{}\newline 
\hspace*{6pt}{<\textbf{orgName}\hspace*{6pt}{change}="{\#Unitex-3.2.0-alpha}"\mbox{}\newline 
\hspace*{6pt}\hspace*{6pt}{resp}="{istex}"\mbox{}\newline 
\hspace*{6pt}\hspace*{6pt}{scheme}="{https://orgname-entity.data.istex.fr}">}\mbox{}\newline 
\hspace*{6pt}\hspace*{6pt}{<\textbf{term}>}Working Party{</\textbf{term}>}\mbox{}\newline 
\hspace*{6pt}\hspace*{6pt}{<\textbf{fs}\hspace*{6pt}{type}="{statistics}">}\mbox{}\newline 
\hspace*{6pt}\hspace*{6pt}\hspace*{6pt}{<\textbf{f}\hspace*{6pt}{name}="{frequency}">}\mbox{}\newline 
\hspace*{6pt}\hspace*{6pt}\hspace*{6pt}\hspace*{6pt}{<\textbf{numeric}\hspace*{6pt}{value}="{2}"/>}\mbox{}\newline 
\hspace*{6pt}\hspace*{6pt}\hspace*{6pt}{</\textbf{f}>}\mbox{}\newline 
\hspace*{6pt}\hspace*{6pt}{</\textbf{fs}>}\mbox{}\newline 
\hspace*{6pt}{</\textbf{orgName}>}\mbox{}\newline 
{</\textbf{annotationBlock}>}\end{shaded}\egroup 


    \item[{Exemple}]
  StandOff enrichissement entité nommée orgName type="funder"\leavevmode\bgroup\exampleFont \begin{shaded}\noindent\mbox{}{<\textbf{annotationBlock}\hspace*{6pt}{corresp}="{text}">}\mbox{}\newline 
\hspace*{6pt}{<\textbf{orgName}\hspace*{6pt}{change}="{\#Unitex-3.2.0-alpha}"\mbox{}\newline 
\hspace*{6pt}\hspace*{6pt}{resp}="{istex}"\mbox{}\newline 
\hspace*{6pt}\hspace*{6pt}{scheme}="{https://orgnamefunder-entity.data.istex.fr}">}\mbox{}\newline 
\hspace*{6pt}\hspace*{6pt}{<\textbf{term}>}Organization for Pharmaceutical Safety and Research{</\textbf{term}>}\mbox{}\newline 
\hspace*{6pt}\hspace*{6pt}{<\textbf{fs}\hspace*{6pt}{type}="{statistics}">}\mbox{}\newline 
\hspace*{6pt}\hspace*{6pt}\hspace*{6pt}{<\textbf{f}\hspace*{6pt}{name}="{frequency}">}\mbox{}\newline 
\hspace*{6pt}\hspace*{6pt}\hspace*{6pt}\hspace*{6pt}{<\textbf{numeric}\hspace*{6pt}{value}="{1}"/>}\mbox{}\newline 
\hspace*{6pt}\hspace*{6pt}\hspace*{6pt}{</\textbf{f}>}\mbox{}\newline 
\hspace*{6pt}\hspace*{6pt}{</\textbf{fs}>}\mbox{}\newline 
\hspace*{6pt}{</\textbf{orgName}>}\mbox{}\newline 
{</\textbf{annotationBlock}>}\end{shaded}\egroup 


    \item[{Exemple}]
  StandOff enrichissement entité nommée orgName type="provider"\leavevmode\bgroup\exampleFont \begin{shaded}\noindent\mbox{}{<\textbf{annotationBlock}\hspace*{6pt}{corresp}="{text}">}\mbox{}\newline 
\hspace*{6pt}{<\textbf{orgName}\hspace*{6pt}{change}="{\#Unitex-3.2.0-alpha}"\mbox{}\newline 
\hspace*{6pt}\hspace*{6pt}{resp}="{istex}"\mbox{}\newline 
\hspace*{6pt}\hspace*{6pt}{scheme}="{https://orgnameprovider-entity.data.istex.fr}">}\mbox{}\newline 
\hspace*{6pt}\hspace*{6pt}{<\textbf{term}>}TOMCAT at the SLS in Villigen, Switzerland{</\textbf{term}>}\mbox{}\newline 
\hspace*{6pt}\hspace*{6pt}{<\textbf{fs}\hspace*{6pt}{type}="{statistics}">}\mbox{}\newline 
\hspace*{6pt}\hspace*{6pt}\hspace*{6pt}{<\textbf{f}\hspace*{6pt}{name}="{frequency}">}\mbox{}\newline 
\hspace*{6pt}\hspace*{6pt}\hspace*{6pt}\hspace*{6pt}{<\textbf{numeric}\hspace*{6pt}{value}="{1}"/>}\mbox{}\newline 
\hspace*{6pt}\hspace*{6pt}\hspace*{6pt}{</\textbf{f}>}\mbox{}\newline 
\hspace*{6pt}\hspace*{6pt}{</\textbf{fs}>}\mbox{}\newline 
\hspace*{6pt}{</\textbf{orgName}>}\mbox{}\newline 
{</\textbf{annotationBlock}>}\end{shaded}\egroup 


    \item[{Modèle de contenu}]
  \mbox{}\hfill\\[-10pt]\begin{Verbatim}[fontsize=\small]
<content>
 <macroRef key="macro.phraseSeq"/>
</content>
    
\end{Verbatim}

    \item[{Schéma Declaration}]
  \mbox{}\hfill\\[-10pt]\begin{Verbatim}[fontsize=\small]
element orgName
{
   tei_att.global.attributes,
   tei_att.datable.attributes,
   tei_att.editLike.attributes,
   tei_att.personal.attributes,
   tei_att.typed.attributes,
   attribute scheme { text }?,
   tei_macro.phraseSeq}
\end{Verbatim}

\end{reflist}  \index{orig=<orig>|oddindex}
\begin{reflist}
\item[]\begin{specHead}{TEI.orig}{<orig> }(forme originale) contient une partie notée comme étant fidèle à l'original et non pas normalisée ou corrigée. [\xref{http://www.tei-c.org/release/doc/tei-p5-doc/en/html/CO.html\#COEDREG}{3.4.2. Regularization and Normalization} \xref{http://www.tei-c.org/release/doc/tei-p5-doc/en/html/TC.html\#TC}{12. Critical Apparatus}]\end{specHead} 
    \item[{Module}]
  core
    \item[{Attributs}]
  Attributs \hyperref[TEI.att.global]{att.global} (\textit{@xml:id}, \textit{@n}, \textit{@xml:lang}, \textit{@xml:base}, \textit{@xml:space})  (\hyperref[TEI.att.global.rendition]{att.global.rendition} (\textit{@rend}, \textit{@style}, \textit{@rendition})) (\hyperref[TEI.att.global.linking]{att.global.linking} (\textit{@corresp}, \textit{@synch}, \textit{@sameAs}, \textit{@copyOf}, \textit{@next}, \textit{@prev}, \textit{@exclude}, \textit{@select})) (\hyperref[TEI.att.global.analytic]{att.global.analytic} (\textit{@ana})) (\hyperref[TEI.att.global.facs]{att.global.facs} (\textit{@facs})) (\hyperref[TEI.att.global.change]{att.global.change} (\textit{@change})) (\hyperref[TEI.att.global.responsibility]{att.global.responsibility} (\textit{@cert}, \textit{@resp})) (\hyperref[TEI.att.global.source]{att.global.source} (\textit{@source}))
    \item[{Membre du}]
  \hyperref[TEI.model.choicePart]{model.choicePart} \hyperref[TEI.model.pPart.transcriptional]{model.pPart.transcriptional}
    \item[{Contenu dans}]
  
    \item[analysis: ]
   \hyperref[TEI.cl]{cl} \hyperref[TEI.pc]{pc} \hyperref[TEI.phr]{phr} \hyperref[TEI.s]{s} \hyperref[TEI.w]{w}\par 
    \item[core: ]
   \hyperref[TEI.abbr]{abbr} \hyperref[TEI.add]{add} \hyperref[TEI.addrLine]{addrLine} \hyperref[TEI.author]{author} \hyperref[TEI.bibl]{bibl} \hyperref[TEI.biblScope]{biblScope} \hyperref[TEI.choice]{choice} \hyperref[TEI.citedRange]{citedRange} \hyperref[TEI.corr]{corr} \hyperref[TEI.date]{date} \hyperref[TEI.del]{del} \hyperref[TEI.distinct]{distinct} \hyperref[TEI.editor]{editor} \hyperref[TEI.email]{email} \hyperref[TEI.emph]{emph} \hyperref[TEI.expan]{expan} \hyperref[TEI.foreign]{foreign} \hyperref[TEI.gloss]{gloss} \hyperref[TEI.head]{head} \hyperref[TEI.headItem]{headItem} \hyperref[TEI.headLabel]{headLabel} \hyperref[TEI.hi]{hi} \hyperref[TEI.item]{item} \hyperref[TEI.l]{l} \hyperref[TEI.label]{label} \hyperref[TEI.measure]{measure} \hyperref[TEI.mentioned]{mentioned} \hyperref[TEI.name]{name} \hyperref[TEI.note]{note} \hyperref[TEI.num]{num} \hyperref[TEI.orig]{orig} \hyperref[TEI.p]{p} \hyperref[TEI.pubPlace]{pubPlace} \hyperref[TEI.publisher]{publisher} \hyperref[TEI.q]{q} \hyperref[TEI.quote]{quote} \hyperref[TEI.ref]{ref} \hyperref[TEI.reg]{reg} \hyperref[TEI.rs]{rs} \hyperref[TEI.said]{said} \hyperref[TEI.sic]{sic} \hyperref[TEI.soCalled]{soCalled} \hyperref[TEI.speaker]{speaker} \hyperref[TEI.stage]{stage} \hyperref[TEI.street]{street} \hyperref[TEI.term]{term} \hyperref[TEI.textLang]{textLang} \hyperref[TEI.time]{time} \hyperref[TEI.title]{title} \hyperref[TEI.unclear]{unclear}\par 
    \item[figures: ]
   \hyperref[TEI.cell]{cell}\par 
    \item[header: ]
   \hyperref[TEI.change]{change} \hyperref[TEI.distributor]{distributor} \hyperref[TEI.edition]{edition} \hyperref[TEI.extent]{extent} \hyperref[TEI.licence]{licence}\par 
    \item[linking: ]
   \hyperref[TEI.ab]{ab} \hyperref[TEI.seg]{seg}\par 
    \item[msdescription: ]
   \hyperref[TEI.accMat]{accMat} \hyperref[TEI.acquisition]{acquisition} \hyperref[TEI.additions]{additions} \hyperref[TEI.catchwords]{catchwords} \hyperref[TEI.collation]{collation} \hyperref[TEI.colophon]{colophon} \hyperref[TEI.condition]{condition} \hyperref[TEI.custEvent]{custEvent} \hyperref[TEI.decoNote]{decoNote} \hyperref[TEI.explicit]{explicit} \hyperref[TEI.filiation]{filiation} \hyperref[TEI.finalRubric]{finalRubric} \hyperref[TEI.foliation]{foliation} \hyperref[TEI.heraldry]{heraldry} \hyperref[TEI.incipit]{incipit} \hyperref[TEI.layout]{layout} \hyperref[TEI.material]{material} \hyperref[TEI.musicNotation]{musicNotation} \hyperref[TEI.objectType]{objectType} \hyperref[TEI.origDate]{origDate} \hyperref[TEI.origPlace]{origPlace} \hyperref[TEI.origin]{origin} \hyperref[TEI.provenance]{provenance} \hyperref[TEI.rubric]{rubric} \hyperref[TEI.secFol]{secFol} \hyperref[TEI.signatures]{signatures} \hyperref[TEI.source]{source} \hyperref[TEI.stamp]{stamp} \hyperref[TEI.summary]{summary} \hyperref[TEI.support]{support} \hyperref[TEI.surrogates]{surrogates} \hyperref[TEI.typeNote]{typeNote} \hyperref[TEI.watermark]{watermark}\par 
    \item[namesdates: ]
   \hyperref[TEI.addName]{addName} \hyperref[TEI.affiliation]{affiliation} \hyperref[TEI.country]{country} \hyperref[TEI.forename]{forename} \hyperref[TEI.genName]{genName} \hyperref[TEI.geogName]{geogName} \hyperref[TEI.nameLink]{nameLink} \hyperref[TEI.orgName]{orgName} \hyperref[TEI.persName]{persName} \hyperref[TEI.placeName]{placeName} \hyperref[TEI.region]{region} \hyperref[TEI.roleName]{roleName} \hyperref[TEI.settlement]{settlement} \hyperref[TEI.surname]{surname}\par 
    \item[textstructure: ]
   \hyperref[TEI.docAuthor]{docAuthor} \hyperref[TEI.docDate]{docDate} \hyperref[TEI.docEdition]{docEdition} \hyperref[TEI.titlePart]{titlePart}\par 
    \item[transcr: ]
   \hyperref[TEI.am]{am} \hyperref[TEI.damage]{damage} \hyperref[TEI.fw]{fw} \hyperref[TEI.metamark]{metamark} \hyperref[TEI.mod]{mod} \hyperref[TEI.restore]{restore} \hyperref[TEI.retrace]{retrace} \hyperref[TEI.secl]{secl} \hyperref[TEI.supplied]{supplied} \hyperref[TEI.surplus]{surplus}
    \item[{Peut contenir}]
  
    \item[analysis: ]
   \hyperref[TEI.c]{c} \hyperref[TEI.cl]{cl} \hyperref[TEI.interp]{interp} \hyperref[TEI.interpGrp]{interpGrp} \hyperref[TEI.m]{m} \hyperref[TEI.pc]{pc} \hyperref[TEI.phr]{phr} \hyperref[TEI.s]{s} \hyperref[TEI.span]{span} \hyperref[TEI.spanGrp]{spanGrp} \hyperref[TEI.w]{w}\par 
    \item[core: ]
   \hyperref[TEI.abbr]{abbr} \hyperref[TEI.add]{add} \hyperref[TEI.address]{address} \hyperref[TEI.bibl]{bibl} \hyperref[TEI.biblStruct]{biblStruct} \hyperref[TEI.binaryObject]{binaryObject} \hyperref[TEI.cb]{cb} \hyperref[TEI.choice]{choice} \hyperref[TEI.cit]{cit} \hyperref[TEI.corr]{corr} \hyperref[TEI.date]{date} \hyperref[TEI.del]{del} \hyperref[TEI.desc]{desc} \hyperref[TEI.distinct]{distinct} \hyperref[TEI.email]{email} \hyperref[TEI.emph]{emph} \hyperref[TEI.expan]{expan} \hyperref[TEI.foreign]{foreign} \hyperref[TEI.gap]{gap} \hyperref[TEI.gb]{gb} \hyperref[TEI.gloss]{gloss} \hyperref[TEI.graphic]{graphic} \hyperref[TEI.hi]{hi} \hyperref[TEI.index]{index} \hyperref[TEI.l]{l} \hyperref[TEI.label]{label} \hyperref[TEI.lb]{lb} \hyperref[TEI.lg]{lg} \hyperref[TEI.list]{list} \hyperref[TEI.listBibl]{listBibl} \hyperref[TEI.measure]{measure} \hyperref[TEI.measureGrp]{measureGrp} \hyperref[TEI.media]{media} \hyperref[TEI.mentioned]{mentioned} \hyperref[TEI.milestone]{milestone} \hyperref[TEI.name]{name} \hyperref[TEI.note]{note} \hyperref[TEI.num]{num} \hyperref[TEI.orig]{orig} \hyperref[TEI.pb]{pb} \hyperref[TEI.ptr]{ptr} \hyperref[TEI.q]{q} \hyperref[TEI.quote]{quote} \hyperref[TEI.ref]{ref} \hyperref[TEI.reg]{reg} \hyperref[TEI.rs]{rs} \hyperref[TEI.said]{said} \hyperref[TEI.sic]{sic} \hyperref[TEI.soCalled]{soCalled} \hyperref[TEI.stage]{stage} \hyperref[TEI.term]{term} \hyperref[TEI.time]{time} \hyperref[TEI.title]{title} \hyperref[TEI.unclear]{unclear}\par 
    \item[derived-module-tei.istex: ]
   \hyperref[TEI.math]{math} \hyperref[TEI.mrow]{mrow}\par 
    \item[figures: ]
   \hyperref[TEI.figure]{figure} \hyperref[TEI.formula]{formula} \hyperref[TEI.notatedMusic]{notatedMusic} \hyperref[TEI.table]{table}\par 
    \item[header: ]
   \hyperref[TEI.biblFull]{biblFull} \hyperref[TEI.idno]{idno}\par 
    \item[iso-fs: ]
   \hyperref[TEI.fLib]{fLib} \hyperref[TEI.fs]{fs} \hyperref[TEI.fvLib]{fvLib}\par 
    \item[linking: ]
   \hyperref[TEI.alt]{alt} \hyperref[TEI.altGrp]{altGrp} \hyperref[TEI.anchor]{anchor} \hyperref[TEI.join]{join} \hyperref[TEI.joinGrp]{joinGrp} \hyperref[TEI.link]{link} \hyperref[TEI.linkGrp]{linkGrp} \hyperref[TEI.seg]{seg} \hyperref[TEI.timeline]{timeline}\par 
    \item[msdescription: ]
   \hyperref[TEI.catchwords]{catchwords} \hyperref[TEI.depth]{depth} \hyperref[TEI.dim]{dim} \hyperref[TEI.dimensions]{dimensions} \hyperref[TEI.height]{height} \hyperref[TEI.heraldry]{heraldry} \hyperref[TEI.locus]{locus} \hyperref[TEI.locusGrp]{locusGrp} \hyperref[TEI.material]{material} \hyperref[TEI.msDesc]{msDesc} \hyperref[TEI.objectType]{objectType} \hyperref[TEI.origDate]{origDate} \hyperref[TEI.origPlace]{origPlace} \hyperref[TEI.secFol]{secFol} \hyperref[TEI.signatures]{signatures} \hyperref[TEI.source]{source} \hyperref[TEI.stamp]{stamp} \hyperref[TEI.watermark]{watermark} \hyperref[TEI.width]{width}\par 
    \item[namesdates: ]
   \hyperref[TEI.addName]{addName} \hyperref[TEI.affiliation]{affiliation} \hyperref[TEI.country]{country} \hyperref[TEI.forename]{forename} \hyperref[TEI.genName]{genName} \hyperref[TEI.geogName]{geogName} \hyperref[TEI.listOrg]{listOrg} \hyperref[TEI.listPlace]{listPlace} \hyperref[TEI.location]{location} \hyperref[TEI.nameLink]{nameLink} \hyperref[TEI.orgName]{orgName} \hyperref[TEI.persName]{persName} \hyperref[TEI.placeName]{placeName} \hyperref[TEI.region]{region} \hyperref[TEI.roleName]{roleName} \hyperref[TEI.settlement]{settlement} \hyperref[TEI.state]{state} \hyperref[TEI.surname]{surname}\par 
    \item[spoken: ]
   \hyperref[TEI.annotationBlock]{annotationBlock}\par 
    \item[textstructure: ]
   \hyperref[TEI.floatingText]{floatingText}\par 
    \item[transcr: ]
   \hyperref[TEI.addSpan]{addSpan} \hyperref[TEI.am]{am} \hyperref[TEI.damage]{damage} \hyperref[TEI.damageSpan]{damageSpan} \hyperref[TEI.delSpan]{delSpan} \hyperref[TEI.ex]{ex} \hyperref[TEI.fw]{fw} \hyperref[TEI.handShift]{handShift} \hyperref[TEI.listTranspose]{listTranspose} \hyperref[TEI.metamark]{metamark} \hyperref[TEI.mod]{mod} \hyperref[TEI.redo]{redo} \hyperref[TEI.restore]{restore} \hyperref[TEI.retrace]{retrace} \hyperref[TEI.secl]{secl} \hyperref[TEI.space]{space} \hyperref[TEI.subst]{subst} \hyperref[TEI.substJoin]{substJoin} \hyperref[TEI.supplied]{supplied} \hyperref[TEI.surplus]{surplus} \hyperref[TEI.undo]{undo}\par des données textuelles
    \item[{Exemple}]
  Si on veut privilégier la version originale du texte, \hyperref[TEI.orig]{<orig>} sera utilisé seul:\leavevmode\bgroup\exampleFont \begin{shaded}\noindent\mbox{}{<\textbf{p}>}si mes pensées se sont entretenues des occurences {<\textbf{orig}>}estrangieres{</\textbf{orig}>} quelque\mbox{}\newline 
 partie du temps, quelque autre partie je les {<\textbf{orig}>}rameine{</\textbf{orig}>} à la promenade, au{<\textbf{orig}>} vergier{</\textbf{orig}>}, à la douceur de cette solitude et à moy. {</\textbf{p}>}\end{shaded}\egroup 


    \item[{Exemple}]
  Généralement, \hyperref[TEI.orig]{<orig>} sera associé à la forme corrigée dans un élément \hyperref[TEI.choice]{<choice>}.\leavevmode\bgroup\exampleFont \begin{shaded}\noindent\mbox{}{<\textbf{l}>}Un bienfait perd sa grâce à le trop {<\textbf{choice}>}\mbox{}\newline 
\hspace*{6pt}\hspace*{6pt}{<\textbf{orig}>}oublier{</\textbf{orig}>}\mbox{}\newline 
\hspace*{6pt}\hspace*{6pt}{<\textbf{corr}>}publier{</\textbf{corr}>}\mbox{}\newline 
\hspace*{6pt}{</\textbf{choice}>} ; {</\textbf{l}>}\mbox{}\newline 
{<\textbf{l}>}Qui veut qu'on s'en souvienne, il le faut {<\textbf{choice}>}\mbox{}\newline 
\hspace*{6pt}\hspace*{6pt}{<\textbf{orig}>}publier{</\textbf{orig}>}\mbox{}\newline 
\hspace*{6pt}\hspace*{6pt}{<\textbf{corr}>}oublier{</\textbf{corr}>}\mbox{}\newline 
\hspace*{6pt}{</\textbf{choice}>}.{</\textbf{l}>}\end{shaded}\egroup 


    \item[{Modèle de contenu}]
  \mbox{}\hfill\\[-10pt]\begin{Verbatim}[fontsize=\small]
<content>
 <macroRef key="macro.paraContent"/>
</content>
    
\end{Verbatim}

    \item[{Schéma Declaration}]
  \mbox{}\hfill\\[-10pt]\begin{Verbatim}[fontsize=\small]
element orig { tei_att.global.attributes, tei_macro.paraContent }
\end{Verbatim}

\end{reflist}  \index{origDate=<origDate>|oddindex}
\begin{reflist}
\item[]\begin{specHead}{TEI.origDate}{<origDate> }(date de la création) Contient une date, dans une forme libre, utilisée pour dater la création d'un manuscrit ou d'une partie d'un manuscrit. [\xref{http://www.tei-c.org/release/doc/tei-p5-doc/en/html/MS.html\#msdates}{10.3.1. Origination}]\end{specHead} 
    \item[{Module}]
  msdescription
    \item[{Attributs}]
  Attributs \hyperref[TEI.att.global]{att.global} (\textit{@xml:id}, \textit{@n}, \textit{@xml:lang}, \textit{@xml:base}, \textit{@xml:space})  (\hyperref[TEI.att.global.rendition]{att.global.rendition} (\textit{@rend}, \textit{@style}, \textit{@rendition})) (\hyperref[TEI.att.global.linking]{att.global.linking} (\textit{@corresp}, \textit{@synch}, \textit{@sameAs}, \textit{@copyOf}, \textit{@next}, \textit{@prev}, \textit{@exclude}, \textit{@select})) (\hyperref[TEI.att.global.analytic]{att.global.analytic} (\textit{@ana})) (\hyperref[TEI.att.global.facs]{att.global.facs} (\textit{@facs})) (\hyperref[TEI.att.global.change]{att.global.change} (\textit{@change})) (\hyperref[TEI.att.global.responsibility]{att.global.responsibility} (\textit{@cert}, \textit{@resp})) (\hyperref[TEI.att.global.source]{att.global.source} (\textit{@source})) \hyperref[TEI.att.datable]{att.datable} (\textit{@calendar}, \textit{@period})  (\hyperref[TEI.att.datable.w3c]{att.datable.w3c} (\textit{@when}, \textit{@notBefore}, \textit{@notAfter}, \textit{@from}, \textit{@to})) (\hyperref[TEI.att.datable.iso]{att.datable.iso} (\textit{@when-iso}, \textit{@notBefore-iso}, \textit{@notAfter-iso}, \textit{@from-iso}, \textit{@to-iso})) (\hyperref[TEI.att.datable.custom]{att.datable.custom} (\textit{@when-custom}, \textit{@notBefore-custom}, \textit{@notAfter-custom}, \textit{@from-custom}, \textit{@to-custom}, \textit{@datingPoint}, \textit{@datingMethod})) \hyperref[TEI.att.editLike]{att.editLike} (\textit{@evidence}, \textit{@instant})  (\hyperref[TEI.att.dimensions]{att.dimensions} (\textit{@unit}, \textit{@quantity}, \textit{@extent}, \textit{@precision}, \textit{@scope}) (\hyperref[TEI.att.ranging]{att.ranging} (\textit{@atLeast}, \textit{@atMost}, \textit{@min}, \textit{@max}, \textit{@confidence})) ) \hyperref[TEI.att.typed]{att.typed} (\textit{@type}, \textit{@subtype}) 
    \item[{Membre du}]
  \hyperref[TEI.model.pPart.msdesc]{model.pPart.msdesc}
    \item[{Contenu dans}]
  
    \item[analysis: ]
   \hyperref[TEI.cl]{cl} \hyperref[TEI.phr]{phr} \hyperref[TEI.s]{s} \hyperref[TEI.span]{span}\par 
    \item[core: ]
   \hyperref[TEI.abbr]{abbr} \hyperref[TEI.add]{add} \hyperref[TEI.addrLine]{addrLine} \hyperref[TEI.author]{author} \hyperref[TEI.biblScope]{biblScope} \hyperref[TEI.citedRange]{citedRange} \hyperref[TEI.corr]{corr} \hyperref[TEI.date]{date} \hyperref[TEI.del]{del} \hyperref[TEI.desc]{desc} \hyperref[TEI.distinct]{distinct} \hyperref[TEI.editor]{editor} \hyperref[TEI.email]{email} \hyperref[TEI.emph]{emph} \hyperref[TEI.expan]{expan} \hyperref[TEI.foreign]{foreign} \hyperref[TEI.gloss]{gloss} \hyperref[TEI.head]{head} \hyperref[TEI.headItem]{headItem} \hyperref[TEI.headLabel]{headLabel} \hyperref[TEI.hi]{hi} \hyperref[TEI.item]{item} \hyperref[TEI.l]{l} \hyperref[TEI.label]{label} \hyperref[TEI.measure]{measure} \hyperref[TEI.meeting]{meeting} \hyperref[TEI.mentioned]{mentioned} \hyperref[TEI.name]{name} \hyperref[TEI.note]{note} \hyperref[TEI.num]{num} \hyperref[TEI.orig]{orig} \hyperref[TEI.p]{p} \hyperref[TEI.pubPlace]{pubPlace} \hyperref[TEI.publisher]{publisher} \hyperref[TEI.q]{q} \hyperref[TEI.quote]{quote} \hyperref[TEI.ref]{ref} \hyperref[TEI.reg]{reg} \hyperref[TEI.resp]{resp} \hyperref[TEI.rs]{rs} \hyperref[TEI.said]{said} \hyperref[TEI.sic]{sic} \hyperref[TEI.soCalled]{soCalled} \hyperref[TEI.speaker]{speaker} \hyperref[TEI.stage]{stage} \hyperref[TEI.street]{street} \hyperref[TEI.term]{term} \hyperref[TEI.textLang]{textLang} \hyperref[TEI.time]{time} \hyperref[TEI.title]{title} \hyperref[TEI.unclear]{unclear}\par 
    \item[figures: ]
   \hyperref[TEI.cell]{cell} \hyperref[TEI.figDesc]{figDesc}\par 
    \item[header: ]
   \hyperref[TEI.authority]{authority} \hyperref[TEI.change]{change} \hyperref[TEI.classCode]{classCode} \hyperref[TEI.creation]{creation} \hyperref[TEI.distributor]{distributor} \hyperref[TEI.edition]{edition} \hyperref[TEI.extent]{extent} \hyperref[TEI.funder]{funder} \hyperref[TEI.language]{language} \hyperref[TEI.licence]{licence} \hyperref[TEI.rendition]{rendition}\par 
    \item[iso-fs: ]
   \hyperref[TEI.fDescr]{fDescr} \hyperref[TEI.fsDescr]{fsDescr}\par 
    \item[linking: ]
   \hyperref[TEI.ab]{ab} \hyperref[TEI.seg]{seg}\par 
    \item[msdescription: ]
   \hyperref[TEI.accMat]{accMat} \hyperref[TEI.acquisition]{acquisition} \hyperref[TEI.additions]{additions} \hyperref[TEI.catchwords]{catchwords} \hyperref[TEI.collation]{collation} \hyperref[TEI.colophon]{colophon} \hyperref[TEI.condition]{condition} \hyperref[TEI.custEvent]{custEvent} \hyperref[TEI.decoNote]{decoNote} \hyperref[TEI.explicit]{explicit} \hyperref[TEI.filiation]{filiation} \hyperref[TEI.finalRubric]{finalRubric} \hyperref[TEI.foliation]{foliation} \hyperref[TEI.heraldry]{heraldry} \hyperref[TEI.incipit]{incipit} \hyperref[TEI.layout]{layout} \hyperref[TEI.material]{material} \hyperref[TEI.musicNotation]{musicNotation} \hyperref[TEI.objectType]{objectType} \hyperref[TEI.origDate]{origDate} \hyperref[TEI.origPlace]{origPlace} \hyperref[TEI.origin]{origin} \hyperref[TEI.provenance]{provenance} \hyperref[TEI.rubric]{rubric} \hyperref[TEI.secFol]{secFol} \hyperref[TEI.signatures]{signatures} \hyperref[TEI.source]{source} \hyperref[TEI.stamp]{stamp} \hyperref[TEI.summary]{summary} \hyperref[TEI.support]{support} \hyperref[TEI.surrogates]{surrogates} \hyperref[TEI.typeNote]{typeNote} \hyperref[TEI.watermark]{watermark}\par 
    \item[namesdates: ]
   \hyperref[TEI.addName]{addName} \hyperref[TEI.affiliation]{affiliation} \hyperref[TEI.country]{country} \hyperref[TEI.forename]{forename} \hyperref[TEI.genName]{genName} \hyperref[TEI.geogName]{geogName} \hyperref[TEI.nameLink]{nameLink} \hyperref[TEI.orgName]{orgName} \hyperref[TEI.persName]{persName} \hyperref[TEI.placeName]{placeName} \hyperref[TEI.region]{region} \hyperref[TEI.roleName]{roleName} \hyperref[TEI.settlement]{settlement} \hyperref[TEI.surname]{surname}\par 
    \item[textstructure: ]
   \hyperref[TEI.docAuthor]{docAuthor} \hyperref[TEI.docDate]{docDate} \hyperref[TEI.docEdition]{docEdition} \hyperref[TEI.titlePart]{titlePart}\par 
    \item[transcr: ]
   \hyperref[TEI.damage]{damage} \hyperref[TEI.fw]{fw} \hyperref[TEI.metamark]{metamark} \hyperref[TEI.mod]{mod} \hyperref[TEI.restore]{restore} \hyperref[TEI.retrace]{retrace} \hyperref[TEI.secl]{secl} \hyperref[TEI.supplied]{supplied} \hyperref[TEI.surplus]{surplus}
    \item[{Peut contenir}]
  
    \item[analysis: ]
   \hyperref[TEI.c]{c} \hyperref[TEI.cl]{cl} \hyperref[TEI.interp]{interp} \hyperref[TEI.interpGrp]{interpGrp} \hyperref[TEI.m]{m} \hyperref[TEI.pc]{pc} \hyperref[TEI.phr]{phr} \hyperref[TEI.s]{s} \hyperref[TEI.span]{span} \hyperref[TEI.spanGrp]{spanGrp} \hyperref[TEI.w]{w}\par 
    \item[core: ]
   \hyperref[TEI.abbr]{abbr} \hyperref[TEI.add]{add} \hyperref[TEI.address]{address} \hyperref[TEI.binaryObject]{binaryObject} \hyperref[TEI.cb]{cb} \hyperref[TEI.choice]{choice} \hyperref[TEI.corr]{corr} \hyperref[TEI.date]{date} \hyperref[TEI.del]{del} \hyperref[TEI.distinct]{distinct} \hyperref[TEI.email]{email} \hyperref[TEI.emph]{emph} \hyperref[TEI.expan]{expan} \hyperref[TEI.foreign]{foreign} \hyperref[TEI.gap]{gap} \hyperref[TEI.gb]{gb} \hyperref[TEI.gloss]{gloss} \hyperref[TEI.graphic]{graphic} \hyperref[TEI.hi]{hi} \hyperref[TEI.index]{index} \hyperref[TEI.lb]{lb} \hyperref[TEI.measure]{measure} \hyperref[TEI.measureGrp]{measureGrp} \hyperref[TEI.media]{media} \hyperref[TEI.mentioned]{mentioned} \hyperref[TEI.milestone]{milestone} \hyperref[TEI.name]{name} \hyperref[TEI.note]{note} \hyperref[TEI.num]{num} \hyperref[TEI.orig]{orig} \hyperref[TEI.pb]{pb} \hyperref[TEI.ptr]{ptr} \hyperref[TEI.ref]{ref} \hyperref[TEI.reg]{reg} \hyperref[TEI.rs]{rs} \hyperref[TEI.sic]{sic} \hyperref[TEI.soCalled]{soCalled} \hyperref[TEI.term]{term} \hyperref[TEI.time]{time} \hyperref[TEI.title]{title} \hyperref[TEI.unclear]{unclear}\par 
    \item[derived-module-tei.istex: ]
   \hyperref[TEI.math]{math} \hyperref[TEI.mrow]{mrow}\par 
    \item[figures: ]
   \hyperref[TEI.figure]{figure} \hyperref[TEI.formula]{formula} \hyperref[TEI.notatedMusic]{notatedMusic}\par 
    \item[header: ]
   \hyperref[TEI.idno]{idno}\par 
    \item[iso-fs: ]
   \hyperref[TEI.fLib]{fLib} \hyperref[TEI.fs]{fs} \hyperref[TEI.fvLib]{fvLib}\par 
    \item[linking: ]
   \hyperref[TEI.alt]{alt} \hyperref[TEI.altGrp]{altGrp} \hyperref[TEI.anchor]{anchor} \hyperref[TEI.join]{join} \hyperref[TEI.joinGrp]{joinGrp} \hyperref[TEI.link]{link} \hyperref[TEI.linkGrp]{linkGrp} \hyperref[TEI.seg]{seg} \hyperref[TEI.timeline]{timeline}\par 
    \item[msdescription: ]
   \hyperref[TEI.catchwords]{catchwords} \hyperref[TEI.depth]{depth} \hyperref[TEI.dim]{dim} \hyperref[TEI.dimensions]{dimensions} \hyperref[TEI.height]{height} \hyperref[TEI.heraldry]{heraldry} \hyperref[TEI.locus]{locus} \hyperref[TEI.locusGrp]{locusGrp} \hyperref[TEI.material]{material} \hyperref[TEI.objectType]{objectType} \hyperref[TEI.origDate]{origDate} \hyperref[TEI.origPlace]{origPlace} \hyperref[TEI.secFol]{secFol} \hyperref[TEI.signatures]{signatures} \hyperref[TEI.source]{source} \hyperref[TEI.stamp]{stamp} \hyperref[TEI.watermark]{watermark} \hyperref[TEI.width]{width}\par 
    \item[namesdates: ]
   \hyperref[TEI.addName]{addName} \hyperref[TEI.affiliation]{affiliation} \hyperref[TEI.country]{country} \hyperref[TEI.forename]{forename} \hyperref[TEI.genName]{genName} \hyperref[TEI.geogName]{geogName} \hyperref[TEI.location]{location} \hyperref[TEI.nameLink]{nameLink} \hyperref[TEI.orgName]{orgName} \hyperref[TEI.persName]{persName} \hyperref[TEI.placeName]{placeName} \hyperref[TEI.region]{region} \hyperref[TEI.roleName]{roleName} \hyperref[TEI.settlement]{settlement} \hyperref[TEI.state]{state} \hyperref[TEI.surname]{surname}\par 
    \item[spoken: ]
   \hyperref[TEI.annotationBlock]{annotationBlock}\par 
    \item[transcr: ]
   \hyperref[TEI.addSpan]{addSpan} \hyperref[TEI.am]{am} \hyperref[TEI.damage]{damage} \hyperref[TEI.damageSpan]{damageSpan} \hyperref[TEI.delSpan]{delSpan} \hyperref[TEI.ex]{ex} \hyperref[TEI.fw]{fw} \hyperref[TEI.handShift]{handShift} \hyperref[TEI.listTranspose]{listTranspose} \hyperref[TEI.metamark]{metamark} \hyperref[TEI.mod]{mod} \hyperref[TEI.redo]{redo} \hyperref[TEI.restore]{restore} \hyperref[TEI.retrace]{retrace} \hyperref[TEI.secl]{secl} \hyperref[TEI.space]{space} \hyperref[TEI.subst]{subst} \hyperref[TEI.substJoin]{substJoin} \hyperref[TEI.supplied]{supplied} \hyperref[TEI.surplus]{surplus} \hyperref[TEI.undo]{undo}\par des données textuelles
    \item[{Exemple}]
  \leavevmode\bgroup\exampleFont \begin{shaded}\noindent\mbox{}{<\textbf{origDate}\hspace*{6pt}{notAfter}="{-0200}"\mbox{}\newline 
\hspace*{6pt}{notBefore}="{-0300}">}3rd century BCE{</\textbf{origDate}>}\end{shaded}\egroup 


    \item[{Modèle de contenu}]
  \mbox{}\hfill\\[-10pt]\begin{Verbatim}[fontsize=\small]
<content>
 <alternate maxOccurs="unbounded"
  minOccurs="0">
  <textNode/>
  <classRef key="model.gLike"/>
  <classRef key="model.phrase"/>
  <classRef key="model.global"/>
 </alternate>
</content>
    
\end{Verbatim}

    \item[{Schéma Declaration}]
  \mbox{}\hfill\\[-10pt]\begin{Verbatim}[fontsize=\small]
element origDate
{
   tei_att.global.attributes,
   tei_att.datable.attributes,
   tei_att.editLike.attributes,
   tei_att.typed.attributes,
   ( text | tei_model.gLike | tei_model.phrase | tei_model.global )*
}
\end{Verbatim}

\end{reflist}  \index{origPlace=<origPlace>|oddindex}
\begin{reflist}
\item[]\begin{specHead}{TEI.origPlace}{<origPlace> }(lieu de création) Contient un nom de lieu, dans une forme libre, utilisé pour désigner l'endroit où a été produit un manuscrit ou une partie d'un manuscrit. [\xref{http://www.tei-c.org/release/doc/tei-p5-doc/en/html/MS.html\#msdates}{10.3.1. Origination}]\end{specHead} 
    \item[{Module}]
  msdescription
    \item[{Attributs}]
  Attributs \hyperref[TEI.att.global]{att.global} (\textit{@xml:id}, \textit{@n}, \textit{@xml:lang}, \textit{@xml:base}, \textit{@xml:space})  (\hyperref[TEI.att.global.rendition]{att.global.rendition} (\textit{@rend}, \textit{@style}, \textit{@rendition})) (\hyperref[TEI.att.global.linking]{att.global.linking} (\textit{@corresp}, \textit{@synch}, \textit{@sameAs}, \textit{@copyOf}, \textit{@next}, \textit{@prev}, \textit{@exclude}, \textit{@select})) (\hyperref[TEI.att.global.analytic]{att.global.analytic} (\textit{@ana})) (\hyperref[TEI.att.global.facs]{att.global.facs} (\textit{@facs})) (\hyperref[TEI.att.global.change]{att.global.change} (\textit{@change})) (\hyperref[TEI.att.global.responsibility]{att.global.responsibility} (\textit{@cert}, \textit{@resp})) (\hyperref[TEI.att.global.source]{att.global.source} (\textit{@source})) \hyperref[TEI.att.naming]{att.naming} (\textit{@role}, \textit{@nymRef})  (\hyperref[TEI.att.canonical]{att.canonical} (\textit{@key}, \textit{@ref})) \hyperref[TEI.att.datable]{att.datable} (\textit{@calendar}, \textit{@period})  (\hyperref[TEI.att.datable.w3c]{att.datable.w3c} (\textit{@when}, \textit{@notBefore}, \textit{@notAfter}, \textit{@from}, \textit{@to})) (\hyperref[TEI.att.datable.iso]{att.datable.iso} (\textit{@when-iso}, \textit{@notBefore-iso}, \textit{@notAfter-iso}, \textit{@from-iso}, \textit{@to-iso})) (\hyperref[TEI.att.datable.custom]{att.datable.custom} (\textit{@when-custom}, \textit{@notBefore-custom}, \textit{@notAfter-custom}, \textit{@from-custom}, \textit{@to-custom}, \textit{@datingPoint}, \textit{@datingMethod})) \hyperref[TEI.att.editLike]{att.editLike} (\textit{@evidence}, \textit{@instant})  (\hyperref[TEI.att.dimensions]{att.dimensions} (\textit{@unit}, \textit{@quantity}, \textit{@extent}, \textit{@precision}, \textit{@scope}) (\hyperref[TEI.att.ranging]{att.ranging} (\textit{@atLeast}, \textit{@atMost}, \textit{@min}, \textit{@max}, \textit{@confidence})) ) \hyperref[TEI.att.typed]{att.typed} (\textit{@type}, \textit{@subtype}) 
    \item[{Membre du}]
  \hyperref[TEI.model.pPart.msdesc]{model.pPart.msdesc}
    \item[{Contenu dans}]
  
    \item[analysis: ]
   \hyperref[TEI.cl]{cl} \hyperref[TEI.phr]{phr} \hyperref[TEI.s]{s} \hyperref[TEI.span]{span}\par 
    \item[core: ]
   \hyperref[TEI.abbr]{abbr} \hyperref[TEI.add]{add} \hyperref[TEI.addrLine]{addrLine} \hyperref[TEI.author]{author} \hyperref[TEI.biblScope]{biblScope} \hyperref[TEI.citedRange]{citedRange} \hyperref[TEI.corr]{corr} \hyperref[TEI.date]{date} \hyperref[TEI.del]{del} \hyperref[TEI.desc]{desc} \hyperref[TEI.distinct]{distinct} \hyperref[TEI.editor]{editor} \hyperref[TEI.email]{email} \hyperref[TEI.emph]{emph} \hyperref[TEI.expan]{expan} \hyperref[TEI.foreign]{foreign} \hyperref[TEI.gloss]{gloss} \hyperref[TEI.head]{head} \hyperref[TEI.headItem]{headItem} \hyperref[TEI.headLabel]{headLabel} \hyperref[TEI.hi]{hi} \hyperref[TEI.item]{item} \hyperref[TEI.l]{l} \hyperref[TEI.label]{label} \hyperref[TEI.measure]{measure} \hyperref[TEI.meeting]{meeting} \hyperref[TEI.mentioned]{mentioned} \hyperref[TEI.name]{name} \hyperref[TEI.note]{note} \hyperref[TEI.num]{num} \hyperref[TEI.orig]{orig} \hyperref[TEI.p]{p} \hyperref[TEI.pubPlace]{pubPlace} \hyperref[TEI.publisher]{publisher} \hyperref[TEI.q]{q} \hyperref[TEI.quote]{quote} \hyperref[TEI.ref]{ref} \hyperref[TEI.reg]{reg} \hyperref[TEI.resp]{resp} \hyperref[TEI.rs]{rs} \hyperref[TEI.said]{said} \hyperref[TEI.sic]{sic} \hyperref[TEI.soCalled]{soCalled} \hyperref[TEI.speaker]{speaker} \hyperref[TEI.stage]{stage} \hyperref[TEI.street]{street} \hyperref[TEI.term]{term} \hyperref[TEI.textLang]{textLang} \hyperref[TEI.time]{time} \hyperref[TEI.title]{title} \hyperref[TEI.unclear]{unclear}\par 
    \item[figures: ]
   \hyperref[TEI.cell]{cell} \hyperref[TEI.figDesc]{figDesc}\par 
    \item[header: ]
   \hyperref[TEI.authority]{authority} \hyperref[TEI.change]{change} \hyperref[TEI.classCode]{classCode} \hyperref[TEI.creation]{creation} \hyperref[TEI.distributor]{distributor} \hyperref[TEI.edition]{edition} \hyperref[TEI.extent]{extent} \hyperref[TEI.funder]{funder} \hyperref[TEI.language]{language} \hyperref[TEI.licence]{licence} \hyperref[TEI.rendition]{rendition}\par 
    \item[iso-fs: ]
   \hyperref[TEI.fDescr]{fDescr} \hyperref[TEI.fsDescr]{fsDescr}\par 
    \item[linking: ]
   \hyperref[TEI.ab]{ab} \hyperref[TEI.seg]{seg}\par 
    \item[msdescription: ]
   \hyperref[TEI.accMat]{accMat} \hyperref[TEI.acquisition]{acquisition} \hyperref[TEI.additions]{additions} \hyperref[TEI.catchwords]{catchwords} \hyperref[TEI.collation]{collation} \hyperref[TEI.colophon]{colophon} \hyperref[TEI.condition]{condition} \hyperref[TEI.custEvent]{custEvent} \hyperref[TEI.decoNote]{decoNote} \hyperref[TEI.explicit]{explicit} \hyperref[TEI.filiation]{filiation} \hyperref[TEI.finalRubric]{finalRubric} \hyperref[TEI.foliation]{foliation} \hyperref[TEI.heraldry]{heraldry} \hyperref[TEI.incipit]{incipit} \hyperref[TEI.layout]{layout} \hyperref[TEI.material]{material} \hyperref[TEI.musicNotation]{musicNotation} \hyperref[TEI.objectType]{objectType} \hyperref[TEI.origDate]{origDate} \hyperref[TEI.origPlace]{origPlace} \hyperref[TEI.origin]{origin} \hyperref[TEI.provenance]{provenance} \hyperref[TEI.rubric]{rubric} \hyperref[TEI.secFol]{secFol} \hyperref[TEI.signatures]{signatures} \hyperref[TEI.source]{source} \hyperref[TEI.stamp]{stamp} \hyperref[TEI.summary]{summary} \hyperref[TEI.support]{support} \hyperref[TEI.surrogates]{surrogates} \hyperref[TEI.typeNote]{typeNote} \hyperref[TEI.watermark]{watermark}\par 
    \item[namesdates: ]
   \hyperref[TEI.addName]{addName} \hyperref[TEI.affiliation]{affiliation} \hyperref[TEI.country]{country} \hyperref[TEI.forename]{forename} \hyperref[TEI.genName]{genName} \hyperref[TEI.geogName]{geogName} \hyperref[TEI.nameLink]{nameLink} \hyperref[TEI.orgName]{orgName} \hyperref[TEI.persName]{persName} \hyperref[TEI.placeName]{placeName} \hyperref[TEI.region]{region} \hyperref[TEI.roleName]{roleName} \hyperref[TEI.settlement]{settlement} \hyperref[TEI.surname]{surname}\par 
    \item[textstructure: ]
   \hyperref[TEI.docAuthor]{docAuthor} \hyperref[TEI.docDate]{docDate} \hyperref[TEI.docEdition]{docEdition} \hyperref[TEI.titlePart]{titlePart}\par 
    \item[transcr: ]
   \hyperref[TEI.damage]{damage} \hyperref[TEI.fw]{fw} \hyperref[TEI.metamark]{metamark} \hyperref[TEI.mod]{mod} \hyperref[TEI.restore]{restore} \hyperref[TEI.retrace]{retrace} \hyperref[TEI.secl]{secl} \hyperref[TEI.supplied]{supplied} \hyperref[TEI.surplus]{surplus}
    \item[{Peut contenir}]
  
    \item[analysis: ]
   \hyperref[TEI.c]{c} \hyperref[TEI.cl]{cl} \hyperref[TEI.interp]{interp} \hyperref[TEI.interpGrp]{interpGrp} \hyperref[TEI.m]{m} \hyperref[TEI.pc]{pc} \hyperref[TEI.phr]{phr} \hyperref[TEI.s]{s} \hyperref[TEI.span]{span} \hyperref[TEI.spanGrp]{spanGrp} \hyperref[TEI.w]{w}\par 
    \item[core: ]
   \hyperref[TEI.abbr]{abbr} \hyperref[TEI.add]{add} \hyperref[TEI.address]{address} \hyperref[TEI.binaryObject]{binaryObject} \hyperref[TEI.cb]{cb} \hyperref[TEI.choice]{choice} \hyperref[TEI.corr]{corr} \hyperref[TEI.date]{date} \hyperref[TEI.del]{del} \hyperref[TEI.distinct]{distinct} \hyperref[TEI.email]{email} \hyperref[TEI.emph]{emph} \hyperref[TEI.expan]{expan} \hyperref[TEI.foreign]{foreign} \hyperref[TEI.gap]{gap} \hyperref[TEI.gb]{gb} \hyperref[TEI.gloss]{gloss} \hyperref[TEI.graphic]{graphic} \hyperref[TEI.hi]{hi} \hyperref[TEI.index]{index} \hyperref[TEI.lb]{lb} \hyperref[TEI.measure]{measure} \hyperref[TEI.measureGrp]{measureGrp} \hyperref[TEI.media]{media} \hyperref[TEI.mentioned]{mentioned} \hyperref[TEI.milestone]{milestone} \hyperref[TEI.name]{name} \hyperref[TEI.note]{note} \hyperref[TEI.num]{num} \hyperref[TEI.orig]{orig} \hyperref[TEI.pb]{pb} \hyperref[TEI.ptr]{ptr} \hyperref[TEI.ref]{ref} \hyperref[TEI.reg]{reg} \hyperref[TEI.rs]{rs} \hyperref[TEI.sic]{sic} \hyperref[TEI.soCalled]{soCalled} \hyperref[TEI.term]{term} \hyperref[TEI.time]{time} \hyperref[TEI.title]{title} \hyperref[TEI.unclear]{unclear}\par 
    \item[derived-module-tei.istex: ]
   \hyperref[TEI.math]{math} \hyperref[TEI.mrow]{mrow}\par 
    \item[figures: ]
   \hyperref[TEI.figure]{figure} \hyperref[TEI.formula]{formula} \hyperref[TEI.notatedMusic]{notatedMusic}\par 
    \item[header: ]
   \hyperref[TEI.idno]{idno}\par 
    \item[iso-fs: ]
   \hyperref[TEI.fLib]{fLib} \hyperref[TEI.fs]{fs} \hyperref[TEI.fvLib]{fvLib}\par 
    \item[linking: ]
   \hyperref[TEI.alt]{alt} \hyperref[TEI.altGrp]{altGrp} \hyperref[TEI.anchor]{anchor} \hyperref[TEI.join]{join} \hyperref[TEI.joinGrp]{joinGrp} \hyperref[TEI.link]{link} \hyperref[TEI.linkGrp]{linkGrp} \hyperref[TEI.seg]{seg} \hyperref[TEI.timeline]{timeline}\par 
    \item[msdescription: ]
   \hyperref[TEI.catchwords]{catchwords} \hyperref[TEI.depth]{depth} \hyperref[TEI.dim]{dim} \hyperref[TEI.dimensions]{dimensions} \hyperref[TEI.height]{height} \hyperref[TEI.heraldry]{heraldry} \hyperref[TEI.locus]{locus} \hyperref[TEI.locusGrp]{locusGrp} \hyperref[TEI.material]{material} \hyperref[TEI.objectType]{objectType} \hyperref[TEI.origDate]{origDate} \hyperref[TEI.origPlace]{origPlace} \hyperref[TEI.secFol]{secFol} \hyperref[TEI.signatures]{signatures} \hyperref[TEI.source]{source} \hyperref[TEI.stamp]{stamp} \hyperref[TEI.watermark]{watermark} \hyperref[TEI.width]{width}\par 
    \item[namesdates: ]
   \hyperref[TEI.addName]{addName} \hyperref[TEI.affiliation]{affiliation} \hyperref[TEI.country]{country} \hyperref[TEI.forename]{forename} \hyperref[TEI.genName]{genName} \hyperref[TEI.geogName]{geogName} \hyperref[TEI.location]{location} \hyperref[TEI.nameLink]{nameLink} \hyperref[TEI.orgName]{orgName} \hyperref[TEI.persName]{persName} \hyperref[TEI.placeName]{placeName} \hyperref[TEI.region]{region} \hyperref[TEI.roleName]{roleName} \hyperref[TEI.settlement]{settlement} \hyperref[TEI.state]{state} \hyperref[TEI.surname]{surname}\par 
    \item[spoken: ]
   \hyperref[TEI.annotationBlock]{annotationBlock}\par 
    \item[transcr: ]
   \hyperref[TEI.addSpan]{addSpan} \hyperref[TEI.am]{am} \hyperref[TEI.damage]{damage} \hyperref[TEI.damageSpan]{damageSpan} \hyperref[TEI.delSpan]{delSpan} \hyperref[TEI.ex]{ex} \hyperref[TEI.fw]{fw} \hyperref[TEI.handShift]{handShift} \hyperref[TEI.listTranspose]{listTranspose} \hyperref[TEI.metamark]{metamark} \hyperref[TEI.mod]{mod} \hyperref[TEI.redo]{redo} \hyperref[TEI.restore]{restore} \hyperref[TEI.retrace]{retrace} \hyperref[TEI.secl]{secl} \hyperref[TEI.space]{space} \hyperref[TEI.subst]{subst} \hyperref[TEI.substJoin]{substJoin} \hyperref[TEI.supplied]{supplied} \hyperref[TEI.surplus]{surplus} \hyperref[TEI.undo]{undo}\par des données textuelles
    \item[{Note}]
  \par
The {\itshape type} attribute may be used to distinguish different kinds of ‘origin’, for example original place of publication, as opposed to original place of printing.
    \item[{Exemple}]
  \leavevmode\bgroup\exampleFont \begin{shaded}\noindent\mbox{}{<\textbf{origPlace}>}Birmingham{</\textbf{origPlace}>}\end{shaded}\egroup 


    \item[{Modèle de contenu}]
  \mbox{}\hfill\\[-10pt]\begin{Verbatim}[fontsize=\small]
<content>
 <macroRef key="macro.phraseSeq"/>
</content>
    
\end{Verbatim}

    \item[{Schéma Declaration}]
  \mbox{}\hfill\\[-10pt]\begin{Verbatim}[fontsize=\small]
element origPlace
{
   tei_att.global.attributes,
   tei_att.naming.attributes,
   tei_att.datable.attributes,
   tei_att.editLike.attributes,
   tei_att.typed.attributes,
   tei_macro.phraseSeq}
\end{Verbatim}

\end{reflist}  \index{origin=<origin>|oddindex}
\begin{reflist}
\item[]\begin{specHead}{TEI.origin}{<origin> }(origine) contient des informations sur l'origine du manuscrit ou de la partie de manuscrit. [\xref{http://www.tei-c.org/release/doc/tei-p5-doc/en/html/MS.html\#mshy}{10.8. History}]\end{specHead} 
    \item[{Module}]
  msdescription
    \item[{Attributs}]
  Attributs \hyperref[TEI.att.global]{att.global} (\textit{@xml:id}, \textit{@n}, \textit{@xml:lang}, \textit{@xml:base}, \textit{@xml:space})  (\hyperref[TEI.att.global.rendition]{att.global.rendition} (\textit{@rend}, \textit{@style}, \textit{@rendition})) (\hyperref[TEI.att.global.linking]{att.global.linking} (\textit{@corresp}, \textit{@synch}, \textit{@sameAs}, \textit{@copyOf}, \textit{@next}, \textit{@prev}, \textit{@exclude}, \textit{@select})) (\hyperref[TEI.att.global.analytic]{att.global.analytic} (\textit{@ana})) (\hyperref[TEI.att.global.facs]{att.global.facs} (\textit{@facs})) (\hyperref[TEI.att.global.change]{att.global.change} (\textit{@change})) (\hyperref[TEI.att.global.responsibility]{att.global.responsibility} (\textit{@cert}, \textit{@resp})) (\hyperref[TEI.att.global.source]{att.global.source} (\textit{@source})) \hyperref[TEI.att.editLike]{att.editLike} (\textit{@evidence}, \textit{@instant})  (\hyperref[TEI.att.dimensions]{att.dimensions} (\textit{@unit}, \textit{@quantity}, \textit{@extent}, \textit{@precision}, \textit{@scope}) (\hyperref[TEI.att.ranging]{att.ranging} (\textit{@atLeast}, \textit{@atMost}, \textit{@min}, \textit{@max}, \textit{@confidence})) ) \hyperref[TEI.att.datable]{att.datable} (\textit{@calendar}, \textit{@period})  (\hyperref[TEI.att.datable.w3c]{att.datable.w3c} (\textit{@when}, \textit{@notBefore}, \textit{@notAfter}, \textit{@from}, \textit{@to})) (\hyperref[TEI.att.datable.iso]{att.datable.iso} (\textit{@when-iso}, \textit{@notBefore-iso}, \textit{@notAfter-iso}, \textit{@from-iso}, \textit{@to-iso})) (\hyperref[TEI.att.datable.custom]{att.datable.custom} (\textit{@when-custom}, \textit{@notBefore-custom}, \textit{@notAfter-custom}, \textit{@from-custom}, \textit{@to-custom}, \textit{@datingPoint}, \textit{@datingMethod}))
    \item[{Contenu dans}]
  
    \item[msdescription: ]
   \hyperref[TEI.history]{history}
    \item[{Peut contenir}]
  
    \item[analysis: ]
   \hyperref[TEI.c]{c} \hyperref[TEI.cl]{cl} \hyperref[TEI.interp]{interp} \hyperref[TEI.interpGrp]{interpGrp} \hyperref[TEI.m]{m} \hyperref[TEI.pc]{pc} \hyperref[TEI.phr]{phr} \hyperref[TEI.s]{s} \hyperref[TEI.span]{span} \hyperref[TEI.spanGrp]{spanGrp} \hyperref[TEI.w]{w}\par 
    \item[core: ]
   \hyperref[TEI.abbr]{abbr} \hyperref[TEI.add]{add} \hyperref[TEI.address]{address} \hyperref[TEI.bibl]{bibl} \hyperref[TEI.biblStruct]{biblStruct} \hyperref[TEI.binaryObject]{binaryObject} \hyperref[TEI.cb]{cb} \hyperref[TEI.choice]{choice} \hyperref[TEI.cit]{cit} \hyperref[TEI.corr]{corr} \hyperref[TEI.date]{date} \hyperref[TEI.del]{del} \hyperref[TEI.desc]{desc} \hyperref[TEI.distinct]{distinct} \hyperref[TEI.email]{email} \hyperref[TEI.emph]{emph} \hyperref[TEI.expan]{expan} \hyperref[TEI.foreign]{foreign} \hyperref[TEI.gap]{gap} \hyperref[TEI.gb]{gb} \hyperref[TEI.gloss]{gloss} \hyperref[TEI.graphic]{graphic} \hyperref[TEI.hi]{hi} \hyperref[TEI.index]{index} \hyperref[TEI.l]{l} \hyperref[TEI.label]{label} \hyperref[TEI.lb]{lb} \hyperref[TEI.lg]{lg} \hyperref[TEI.list]{list} \hyperref[TEI.listBibl]{listBibl} \hyperref[TEI.measure]{measure} \hyperref[TEI.measureGrp]{measureGrp} \hyperref[TEI.media]{media} \hyperref[TEI.mentioned]{mentioned} \hyperref[TEI.milestone]{milestone} \hyperref[TEI.name]{name} \hyperref[TEI.note]{note} \hyperref[TEI.num]{num} \hyperref[TEI.orig]{orig} \hyperref[TEI.p]{p} \hyperref[TEI.pb]{pb} \hyperref[TEI.ptr]{ptr} \hyperref[TEI.q]{q} \hyperref[TEI.quote]{quote} \hyperref[TEI.ref]{ref} \hyperref[TEI.reg]{reg} \hyperref[TEI.rs]{rs} \hyperref[TEI.said]{said} \hyperref[TEI.sic]{sic} \hyperref[TEI.soCalled]{soCalled} \hyperref[TEI.sp]{sp} \hyperref[TEI.stage]{stage} \hyperref[TEI.term]{term} \hyperref[TEI.time]{time} \hyperref[TEI.title]{title} \hyperref[TEI.unclear]{unclear}\par 
    \item[derived-module-tei.istex: ]
   \hyperref[TEI.math]{math} \hyperref[TEI.mrow]{mrow}\par 
    \item[figures: ]
   \hyperref[TEI.figure]{figure} \hyperref[TEI.formula]{formula} \hyperref[TEI.notatedMusic]{notatedMusic} \hyperref[TEI.table]{table}\par 
    \item[header: ]
   \hyperref[TEI.biblFull]{biblFull} \hyperref[TEI.idno]{idno}\par 
    \item[iso-fs: ]
   \hyperref[TEI.fLib]{fLib} \hyperref[TEI.fs]{fs} \hyperref[TEI.fvLib]{fvLib}\par 
    \item[linking: ]
   \hyperref[TEI.ab]{ab} \hyperref[TEI.alt]{alt} \hyperref[TEI.altGrp]{altGrp} \hyperref[TEI.anchor]{anchor} \hyperref[TEI.join]{join} \hyperref[TEI.joinGrp]{joinGrp} \hyperref[TEI.link]{link} \hyperref[TEI.linkGrp]{linkGrp} \hyperref[TEI.seg]{seg} \hyperref[TEI.timeline]{timeline}\par 
    \item[msdescription: ]
   \hyperref[TEI.catchwords]{catchwords} \hyperref[TEI.depth]{depth} \hyperref[TEI.dim]{dim} \hyperref[TEI.dimensions]{dimensions} \hyperref[TEI.height]{height} \hyperref[TEI.heraldry]{heraldry} \hyperref[TEI.locus]{locus} \hyperref[TEI.locusGrp]{locusGrp} \hyperref[TEI.material]{material} \hyperref[TEI.msDesc]{msDesc} \hyperref[TEI.objectType]{objectType} \hyperref[TEI.origDate]{origDate} \hyperref[TEI.origPlace]{origPlace} \hyperref[TEI.secFol]{secFol} \hyperref[TEI.signatures]{signatures} \hyperref[TEI.source]{source} \hyperref[TEI.stamp]{stamp} \hyperref[TEI.watermark]{watermark} \hyperref[TEI.width]{width}\par 
    \item[namesdates: ]
   \hyperref[TEI.addName]{addName} \hyperref[TEI.affiliation]{affiliation} \hyperref[TEI.country]{country} \hyperref[TEI.forename]{forename} \hyperref[TEI.genName]{genName} \hyperref[TEI.geogName]{geogName} \hyperref[TEI.listOrg]{listOrg} \hyperref[TEI.listPlace]{listPlace} \hyperref[TEI.location]{location} \hyperref[TEI.nameLink]{nameLink} \hyperref[TEI.orgName]{orgName} \hyperref[TEI.persName]{persName} \hyperref[TEI.placeName]{placeName} \hyperref[TEI.region]{region} \hyperref[TEI.roleName]{roleName} \hyperref[TEI.settlement]{settlement} \hyperref[TEI.state]{state} \hyperref[TEI.surname]{surname}\par 
    \item[spoken: ]
   \hyperref[TEI.annotationBlock]{annotationBlock}\par 
    \item[textstructure: ]
   \hyperref[TEI.floatingText]{floatingText}\par 
    \item[transcr: ]
   \hyperref[TEI.addSpan]{addSpan} \hyperref[TEI.am]{am} \hyperref[TEI.damage]{damage} \hyperref[TEI.damageSpan]{damageSpan} \hyperref[TEI.delSpan]{delSpan} \hyperref[TEI.ex]{ex} \hyperref[TEI.fw]{fw} \hyperref[TEI.handShift]{handShift} \hyperref[TEI.listTranspose]{listTranspose} \hyperref[TEI.metamark]{metamark} \hyperref[TEI.mod]{mod} \hyperref[TEI.redo]{redo} \hyperref[TEI.restore]{restore} \hyperref[TEI.retrace]{retrace} \hyperref[TEI.secl]{secl} \hyperref[TEI.space]{space} \hyperref[TEI.subst]{subst} \hyperref[TEI.substJoin]{substJoin} \hyperref[TEI.supplied]{supplied} \hyperref[TEI.surplus]{surplus} \hyperref[TEI.undo]{undo}\par des données textuelles
    \item[{Exemple}]
  \leavevmode\bgroup\exampleFont \begin{shaded}\noindent\mbox{}{<\textbf{origin}\hspace*{6pt}{evidence}="{internal}"\hspace*{6pt}{notAfter}="{1845}"\mbox{}\newline 
\hspace*{6pt}{notBefore}="{1802}"\hspace*{6pt}{resp}="{\#fr\textunderscore AMH}">} Copied in {<\textbf{name}\hspace*{6pt}{type}="{origPlace}">}Derby{</\textbf{name}>}, probably from an old Flemish original, between 1802 and\mbox{}\newline 
 1845, according to {<\textbf{persName}\hspace*{6pt}{xml:id}="{fr\textunderscore AMH}">}Anne-Mette Hansen{</\textbf{persName}>}.{</\textbf{origin}>}\end{shaded}\egroup 


    \item[{Modèle de contenu}]
  \mbox{}\hfill\\[-10pt]\begin{Verbatim}[fontsize=\small]
<content>
 <macroRef key="macro.specialPara"/>
</content>
    
\end{Verbatim}

    \item[{Schéma Declaration}]
  \mbox{}\hfill\\[-10pt]\begin{Verbatim}[fontsize=\small]
element origin
{
   tei_att.global.attributes,
   tei_att.editLike.attributes,
   tei_att.datable.attributes,
   tei_macro.specialPara}
\end{Verbatim}

\end{reflist}  \index{p=<p>|oddindex}\index{scheme=@scheme!<p>|oddindex}
\begin{reflist}
\item[]\begin{specHead}{TEI.p}{<p> }(paragraphe) marque les paragraphes dans un texte en prose. [\xref{http://www.tei-c.org/release/doc/tei-p5-doc/en/html/CO.html\#COPA}{3.1. Paragraphs} \xref{http://www.tei-c.org/release/doc/tei-p5-doc/en/html/DR.html\#DRPAL}{7.2.5. Speech Contents}]\end{specHead} 
    \item[{Module}]
  core
    \item[{Attributs}]
  Attributs \hyperref[TEI.att.global]{att.global} (\textit{@xml:id}, \textit{@n}, \textit{@xml:lang}, \textit{@xml:base}, \textit{@xml:space})  (\hyperref[TEI.att.global.rendition]{att.global.rendition} (\textit{@rend}, \textit{@style}, \textit{@rendition})) (\hyperref[TEI.att.global.linking]{att.global.linking} (\textit{@corresp}, \textit{@synch}, \textit{@sameAs}, \textit{@copyOf}, \textit{@next}, \textit{@prev}, \textit{@exclude}, \textit{@select})) (\hyperref[TEI.att.global.analytic]{att.global.analytic} (\textit{@ana})) (\hyperref[TEI.att.global.facs]{att.global.facs} (\textit{@facs})) (\hyperref[TEI.att.global.change]{att.global.change} (\textit{@change})) (\hyperref[TEI.att.global.responsibility]{att.global.responsibility} (\textit{@cert}, \textit{@resp})) (\hyperref[TEI.att.global.source]{att.global.source} (\textit{@source})) \hyperref[TEI.att.declaring]{att.declaring} (\textit{@decls}) \hyperref[TEI.att.fragmentable]{att.fragmentable} (\textit{@part}) \hyperref[TEI.att.written]{att.written} (\textit{@hand}) \hfil\\[-10pt]\begin{sansreflist}
    \item[@scheme]
  désigne la liste des ontologies dans lequel l'ensemble des termes concernés sont définis.
\begin{reflist}
    \item[{Statut}]
  Optionel
    \item[{Type de données}]
  \hyperref[TEI.teidata.pointer]{teidata.pointer}
\end{reflist}  
\end{sansreflist}  
    \item[{Membre du}]
  \hyperref[TEI.model.pLike]{model.pLike}
    \item[{Contenu dans}]
  
    \item[core: ]
   \hyperref[TEI.item]{item} \hyperref[TEI.note]{note} \hyperref[TEI.q]{q} \hyperref[TEI.quote]{quote} \hyperref[TEI.said]{said} \hyperref[TEI.sp]{sp} \hyperref[TEI.stage]{stage}\par 
    \item[figures: ]
   \hyperref[TEI.cell]{cell} \hyperref[TEI.figure]{figure}\par 
    \item[header: ]
   \hyperref[TEI.abstract]{abstract} \hyperref[TEI.application]{application} \hyperref[TEI.availability]{availability} \hyperref[TEI.change]{change} \hyperref[TEI.correction]{correction} \hyperref[TEI.editionStmt]{editionStmt} \hyperref[TEI.encodingDesc]{encodingDesc} \hyperref[TEI.langUsage]{langUsage} \hyperref[TEI.licence]{licence} \hyperref[TEI.publicationStmt]{publicationStmt} \hyperref[TEI.seriesStmt]{seriesStmt} \hyperref[TEI.sourceDesc]{sourceDesc}\par 
    \item[msdescription: ]
   \hyperref[TEI.accMat]{accMat} \hyperref[TEI.acquisition]{acquisition} \hyperref[TEI.additions]{additions} \hyperref[TEI.binding]{binding} \hyperref[TEI.bindingDesc]{bindingDesc} \hyperref[TEI.collation]{collation} \hyperref[TEI.condition]{condition} \hyperref[TEI.custEvent]{custEvent} \hyperref[TEI.custodialHist]{custodialHist} \hyperref[TEI.decoDesc]{decoDesc} \hyperref[TEI.decoNote]{decoNote} \hyperref[TEI.filiation]{filiation} \hyperref[TEI.foliation]{foliation} \hyperref[TEI.handDesc]{handDesc} \hyperref[TEI.history]{history} \hyperref[TEI.layout]{layout} \hyperref[TEI.layoutDesc]{layoutDesc} \hyperref[TEI.msContents]{msContents} \hyperref[TEI.msDesc]{msDesc} \hyperref[TEI.msFrag]{msFrag} \hyperref[TEI.msItem]{msItem} \hyperref[TEI.msItemStruct]{msItemStruct} \hyperref[TEI.msPart]{msPart} \hyperref[TEI.musicNotation]{musicNotation} \hyperref[TEI.objectDesc]{objectDesc} \hyperref[TEI.origin]{origin} \hyperref[TEI.physDesc]{physDesc} \hyperref[TEI.provenance]{provenance} \hyperref[TEI.recordHist]{recordHist} \hyperref[TEI.scriptDesc]{scriptDesc} \hyperref[TEI.seal]{seal} \hyperref[TEI.sealDesc]{sealDesc} \hyperref[TEI.signatures]{signatures} \hyperref[TEI.source]{source} \hyperref[TEI.summary]{summary} \hyperref[TEI.support]{support} \hyperref[TEI.supportDesc]{supportDesc} \hyperref[TEI.surrogates]{surrogates} \hyperref[TEI.typeDesc]{typeDesc} \hyperref[TEI.typeNote]{typeNote}\par 
    \item[namesdates: ]
   \hyperref[TEI.event]{event} \hyperref[TEI.org]{org} \hyperref[TEI.person]{person} \hyperref[TEI.personGrp]{personGrp} \hyperref[TEI.persona]{persona} \hyperref[TEI.place]{place} \hyperref[TEI.state]{state}\par 
    \item[textstructure: ]
   \hyperref[TEI.back]{back} \hyperref[TEI.body]{body} \hyperref[TEI.div]{div} \hyperref[TEI.front]{front}\par 
    \item[transcr: ]
   \hyperref[TEI.metamark]{metamark}
    \item[{Peut contenir}]
  
    \item[analysis: ]
   \hyperref[TEI.c]{c} \hyperref[TEI.cl]{cl} \hyperref[TEI.interp]{interp} \hyperref[TEI.interpGrp]{interpGrp} \hyperref[TEI.m]{m} \hyperref[TEI.pc]{pc} \hyperref[TEI.phr]{phr} \hyperref[TEI.s]{s} \hyperref[TEI.span]{span} \hyperref[TEI.spanGrp]{spanGrp} \hyperref[TEI.w]{w}\par 
    \item[core: ]
   \hyperref[TEI.abbr]{abbr} \hyperref[TEI.add]{add} \hyperref[TEI.address]{address} \hyperref[TEI.bibl]{bibl} \hyperref[TEI.biblStruct]{biblStruct} \hyperref[TEI.binaryObject]{binaryObject} \hyperref[TEI.cb]{cb} \hyperref[TEI.choice]{choice} \hyperref[TEI.cit]{cit} \hyperref[TEI.corr]{corr} \hyperref[TEI.date]{date} \hyperref[TEI.del]{del} \hyperref[TEI.desc]{desc} \hyperref[TEI.distinct]{distinct} \hyperref[TEI.email]{email} \hyperref[TEI.emph]{emph} \hyperref[TEI.expan]{expan} \hyperref[TEI.foreign]{foreign} \hyperref[TEI.gap]{gap} \hyperref[TEI.gb]{gb} \hyperref[TEI.gloss]{gloss} \hyperref[TEI.graphic]{graphic} \hyperref[TEI.hi]{hi} \hyperref[TEI.index]{index} \hyperref[TEI.l]{l} \hyperref[TEI.label]{label} \hyperref[TEI.lb]{lb} \hyperref[TEI.lg]{lg} \hyperref[TEI.list]{list} \hyperref[TEI.listBibl]{listBibl} \hyperref[TEI.measure]{measure} \hyperref[TEI.measureGrp]{measureGrp} \hyperref[TEI.media]{media} \hyperref[TEI.mentioned]{mentioned} \hyperref[TEI.milestone]{milestone} \hyperref[TEI.name]{name} \hyperref[TEI.note]{note} \hyperref[TEI.num]{num} \hyperref[TEI.orig]{orig} \hyperref[TEI.pb]{pb} \hyperref[TEI.ptr]{ptr} \hyperref[TEI.q]{q} \hyperref[TEI.quote]{quote} \hyperref[TEI.ref]{ref} \hyperref[TEI.reg]{reg} \hyperref[TEI.rs]{rs} \hyperref[TEI.said]{said} \hyperref[TEI.sic]{sic} \hyperref[TEI.soCalled]{soCalled} \hyperref[TEI.stage]{stage} \hyperref[TEI.term]{term} \hyperref[TEI.time]{time} \hyperref[TEI.title]{title} \hyperref[TEI.unclear]{unclear}\par 
    \item[derived-module-tei.istex: ]
   \hyperref[TEI.math]{math} \hyperref[TEI.mrow]{mrow}\par 
    \item[figures: ]
   \hyperref[TEI.figure]{figure} \hyperref[TEI.formula]{formula} \hyperref[TEI.notatedMusic]{notatedMusic} \hyperref[TEI.table]{table}\par 
    \item[header: ]
   \hyperref[TEI.biblFull]{biblFull} \hyperref[TEI.idno]{idno}\par 
    \item[iso-fs: ]
   \hyperref[TEI.fLib]{fLib} \hyperref[TEI.fs]{fs} \hyperref[TEI.fvLib]{fvLib}\par 
    \item[linking: ]
   \hyperref[TEI.alt]{alt} \hyperref[TEI.altGrp]{altGrp} \hyperref[TEI.anchor]{anchor} \hyperref[TEI.join]{join} \hyperref[TEI.joinGrp]{joinGrp} \hyperref[TEI.link]{link} \hyperref[TEI.linkGrp]{linkGrp} \hyperref[TEI.seg]{seg} \hyperref[TEI.timeline]{timeline}\par 
    \item[msdescription: ]
   \hyperref[TEI.catchwords]{catchwords} \hyperref[TEI.depth]{depth} \hyperref[TEI.dim]{dim} \hyperref[TEI.dimensions]{dimensions} \hyperref[TEI.height]{height} \hyperref[TEI.heraldry]{heraldry} \hyperref[TEI.locus]{locus} \hyperref[TEI.locusGrp]{locusGrp} \hyperref[TEI.material]{material} \hyperref[TEI.msDesc]{msDesc} \hyperref[TEI.objectType]{objectType} \hyperref[TEI.origDate]{origDate} \hyperref[TEI.origPlace]{origPlace} \hyperref[TEI.secFol]{secFol} \hyperref[TEI.signatures]{signatures} \hyperref[TEI.source]{source} \hyperref[TEI.stamp]{stamp} \hyperref[TEI.watermark]{watermark} \hyperref[TEI.width]{width}\par 
    \item[namesdates: ]
   \hyperref[TEI.addName]{addName} \hyperref[TEI.affiliation]{affiliation} \hyperref[TEI.country]{country} \hyperref[TEI.forename]{forename} \hyperref[TEI.genName]{genName} \hyperref[TEI.geogName]{geogName} \hyperref[TEI.listOrg]{listOrg} \hyperref[TEI.listPlace]{listPlace} \hyperref[TEI.location]{location} \hyperref[TEI.nameLink]{nameLink} \hyperref[TEI.orgName]{orgName} \hyperref[TEI.persName]{persName} \hyperref[TEI.placeName]{placeName} \hyperref[TEI.region]{region} \hyperref[TEI.roleName]{roleName} \hyperref[TEI.settlement]{settlement} \hyperref[TEI.state]{state} \hyperref[TEI.surname]{surname}\par 
    \item[spoken: ]
   \hyperref[TEI.annotationBlock]{annotationBlock}\par 
    \item[textstructure: ]
   \hyperref[TEI.floatingText]{floatingText}\par 
    \item[transcr: ]
   \hyperref[TEI.addSpan]{addSpan} \hyperref[TEI.am]{am} \hyperref[TEI.damage]{damage} \hyperref[TEI.damageSpan]{damageSpan} \hyperref[TEI.delSpan]{delSpan} \hyperref[TEI.ex]{ex} \hyperref[TEI.fw]{fw} \hyperref[TEI.handShift]{handShift} \hyperref[TEI.listTranspose]{listTranspose} \hyperref[TEI.metamark]{metamark} \hyperref[TEI.mod]{mod} \hyperref[TEI.redo]{redo} \hyperref[TEI.restore]{restore} \hyperref[TEI.retrace]{retrace} \hyperref[TEI.secl]{secl} \hyperref[TEI.space]{space} \hyperref[TEI.subst]{subst} \hyperref[TEI.substJoin]{substJoin} \hyperref[TEI.supplied]{supplied} \hyperref[TEI.surplus]{surplus} \hyperref[TEI.undo]{undo}\par des données textuelles
    \item[{Exemple}]
  l'élément \hyperref[TEI.p]{<p>} peut contenir un élément \hyperref[TEI.source]{<source>}\leavevmode\bgroup\exampleFont \begin{shaded}\noindent\mbox{}{<\textbf{listAnnotation}\hspace*{6pt}{type}="{subject}">}\mbox{}\newline 
\hspace*{6pt}{<\textbf{figure}\hspace*{6pt}{type}="{box}">}\mbox{}\newline 
\hspace*{6pt}\hspace*{6pt}{<\textbf{p}>} Workers never organised because they were intimidated. There were so many temporary workers that even the slightest hint of organisation and you were gone … even the people who were permanent were afraid to lose their jobs\mbox{}\newline 
\hspace*{6pt}\hspace*{6pt}{<\textbf{source}>}(Leader B){</\textbf{source}>}\mbox{}\newline 
\hspace*{6pt}\hspace*{6pt}\hspace*{6pt}\hspace*{6pt} .\mbox{}\newline 
\hspace*{6pt}\hspace*{6pt}{</\textbf{p}>}\mbox{}\newline 
\hspace*{6pt}{</\textbf{figure}>}\mbox{}\newline 
{</\textbf{listAnnotation}>}\end{shaded}\egroup 


    \item[{Schematron}]
   <s:report test="not(ancestor::floatingText) and (ancestor::tei:p or ancestor::tei:ab)   and not(parent::tei:exemplum |parent::tei:item |parent::tei:note |parent::tei:q   |parent::tei:quote |parent::tei:remarks |parent::tei:said |parent::tei:sp   |parent::tei:stage |parent::tei:cell |parent::tei:figure )"> Abstract model violation: Paragraphs may not contain other paragraphs or ab elements. </s:report>
    \item[{Schematron}]
   <s:report test="ancestor::tei:l[not(.//tei:note//tei:p[. = current()])]"> Abstract model violation: Lines may not contain higher-level structural elements such as div, p, or ab. </s:report>
    \item[{Modèle de contenu}]
  \mbox{}\hfill\\[-10pt]\begin{Verbatim}[fontsize=\small]
<content>
 <macroRef key="macro.paraContent"/>
</content>
    
\end{Verbatim}

    \item[{Schéma Declaration}]
  \mbox{}\hfill\\[-10pt]\begin{Verbatim}[fontsize=\small]
element p
{
   tei_att.global.attributes,
   tei_att.declaring.attributes,
   tei_att.fragmentable.attributes,
   tei_att.written.attributes,
   attribute scheme { text }?,
   tei_macro.paraContent}
\end{Verbatim}

\end{reflist}  \index{pb=<pb>|oddindex}
\begin{reflist}
\item[]\begin{specHead}{TEI.pb}{<pb> }(saut de page) marque le début d'une page de texte dans un document paginé. [\xref{http://www.tei-c.org/release/doc/tei-p5-doc/en/html/CO.html\#CORS5}{3.10.3. Milestone Elements}]\end{specHead} 
    \item[{Module}]
  core
    \item[{Attributs}]
  Attributs \hyperref[TEI.att.global]{att.global} (\textit{@xml:id}, \textit{@n}, \textit{@xml:lang}, \textit{@xml:base}, \textit{@xml:space})  (\hyperref[TEI.att.global.rendition]{att.global.rendition} (\textit{@rend}, \textit{@style}, \textit{@rendition})) (\hyperref[TEI.att.global.linking]{att.global.linking} (\textit{@corresp}, \textit{@synch}, \textit{@sameAs}, \textit{@copyOf}, \textit{@next}, \textit{@prev}, \textit{@exclude}, \textit{@select})) (\hyperref[TEI.att.global.analytic]{att.global.analytic} (\textit{@ana})) (\hyperref[TEI.att.global.facs]{att.global.facs} (\textit{@facs})) (\hyperref[TEI.att.global.change]{att.global.change} (\textit{@change})) (\hyperref[TEI.att.global.responsibility]{att.global.responsibility} (\textit{@cert}, \textit{@resp})) (\hyperref[TEI.att.global.source]{att.global.source} (\textit{@source})) \hyperref[TEI.att.typed]{att.typed} (\textit{@type}, \textit{@subtype}) \hyperref[TEI.att.edition]{att.edition} (\textit{@ed}, \textit{@edRef}) \hyperref[TEI.att.spanning]{att.spanning} (\textit{@spanTo}) \hyperref[TEI.att.breaking]{att.breaking} (\textit{@break}) 
    \item[{Membre du}]
  \hyperref[TEI.model.milestoneLike]{model.milestoneLike}
    \item[{Contenu dans}]
  
    \item[analysis: ]
   \hyperref[TEI.cl]{cl} \hyperref[TEI.m]{m} \hyperref[TEI.phr]{phr} \hyperref[TEI.s]{s} \hyperref[TEI.span]{span} \hyperref[TEI.w]{w}\par 
    \item[core: ]
   \hyperref[TEI.abbr]{abbr} \hyperref[TEI.add]{add} \hyperref[TEI.addrLine]{addrLine} \hyperref[TEI.address]{address} \hyperref[TEI.author]{author} \hyperref[TEI.bibl]{bibl} \hyperref[TEI.biblScope]{biblScope} \hyperref[TEI.cit]{cit} \hyperref[TEI.citedRange]{citedRange} \hyperref[TEI.corr]{corr} \hyperref[TEI.date]{date} \hyperref[TEI.del]{del} \hyperref[TEI.distinct]{distinct} \hyperref[TEI.editor]{editor} \hyperref[TEI.email]{email} \hyperref[TEI.emph]{emph} \hyperref[TEI.expan]{expan} \hyperref[TEI.foreign]{foreign} \hyperref[TEI.gloss]{gloss} \hyperref[TEI.head]{head} \hyperref[TEI.headItem]{headItem} \hyperref[TEI.headLabel]{headLabel} \hyperref[TEI.hi]{hi} \hyperref[TEI.imprint]{imprint} \hyperref[TEI.item]{item} \hyperref[TEI.l]{l} \hyperref[TEI.label]{label} \hyperref[TEI.lg]{lg} \hyperref[TEI.list]{list} \hyperref[TEI.listBibl]{listBibl} \hyperref[TEI.measure]{measure} \hyperref[TEI.mentioned]{mentioned} \hyperref[TEI.name]{name} \hyperref[TEI.note]{note} \hyperref[TEI.num]{num} \hyperref[TEI.orig]{orig} \hyperref[TEI.p]{p} \hyperref[TEI.pubPlace]{pubPlace} \hyperref[TEI.publisher]{publisher} \hyperref[TEI.q]{q} \hyperref[TEI.quote]{quote} \hyperref[TEI.ref]{ref} \hyperref[TEI.reg]{reg} \hyperref[TEI.resp]{resp} \hyperref[TEI.rs]{rs} \hyperref[TEI.said]{said} \hyperref[TEI.series]{series} \hyperref[TEI.sic]{sic} \hyperref[TEI.soCalled]{soCalled} \hyperref[TEI.sp]{sp} \hyperref[TEI.speaker]{speaker} \hyperref[TEI.stage]{stage} \hyperref[TEI.street]{street} \hyperref[TEI.term]{term} \hyperref[TEI.textLang]{textLang} \hyperref[TEI.time]{time} \hyperref[TEI.title]{title} \hyperref[TEI.unclear]{unclear}\par 
    \item[figures: ]
   \hyperref[TEI.cell]{cell} \hyperref[TEI.figure]{figure} \hyperref[TEI.table]{table}\par 
    \item[header: ]
   \hyperref[TEI.authority]{authority} \hyperref[TEI.change]{change} \hyperref[TEI.classCode]{classCode} \hyperref[TEI.distributor]{distributor} \hyperref[TEI.edition]{edition} \hyperref[TEI.extent]{extent} \hyperref[TEI.funder]{funder} \hyperref[TEI.language]{language} \hyperref[TEI.licence]{licence}\par 
    \item[linking: ]
   \hyperref[TEI.ab]{ab} \hyperref[TEI.seg]{seg}\par 
    \item[msdescription: ]
   \hyperref[TEI.accMat]{accMat} \hyperref[TEI.acquisition]{acquisition} \hyperref[TEI.additions]{additions} \hyperref[TEI.catchwords]{catchwords} \hyperref[TEI.collation]{collation} \hyperref[TEI.colophon]{colophon} \hyperref[TEI.condition]{condition} \hyperref[TEI.custEvent]{custEvent} \hyperref[TEI.decoNote]{decoNote} \hyperref[TEI.explicit]{explicit} \hyperref[TEI.filiation]{filiation} \hyperref[TEI.finalRubric]{finalRubric} \hyperref[TEI.foliation]{foliation} \hyperref[TEI.heraldry]{heraldry} \hyperref[TEI.incipit]{incipit} \hyperref[TEI.layout]{layout} \hyperref[TEI.material]{material} \hyperref[TEI.msItem]{msItem} \hyperref[TEI.musicNotation]{musicNotation} \hyperref[TEI.objectType]{objectType} \hyperref[TEI.origDate]{origDate} \hyperref[TEI.origPlace]{origPlace} \hyperref[TEI.origin]{origin} \hyperref[TEI.provenance]{provenance} \hyperref[TEI.rubric]{rubric} \hyperref[TEI.secFol]{secFol} \hyperref[TEI.signatures]{signatures} \hyperref[TEI.source]{source} \hyperref[TEI.stamp]{stamp} \hyperref[TEI.summary]{summary} \hyperref[TEI.support]{support} \hyperref[TEI.surrogates]{surrogates} \hyperref[TEI.typeNote]{typeNote} \hyperref[TEI.watermark]{watermark}\par 
    \item[namesdates: ]
   \hyperref[TEI.addName]{addName} \hyperref[TEI.affiliation]{affiliation} \hyperref[TEI.country]{country} \hyperref[TEI.forename]{forename} \hyperref[TEI.genName]{genName} \hyperref[TEI.geogName]{geogName} \hyperref[TEI.nameLink]{nameLink} \hyperref[TEI.org]{org} \hyperref[TEI.orgName]{orgName} \hyperref[TEI.persName]{persName} \hyperref[TEI.person]{person} \hyperref[TEI.personGrp]{personGrp} \hyperref[TEI.persona]{persona} \hyperref[TEI.placeName]{placeName} \hyperref[TEI.region]{region} \hyperref[TEI.roleName]{roleName} \hyperref[TEI.settlement]{settlement} \hyperref[TEI.surname]{surname}\par 
    \item[textstructure: ]
   \hyperref[TEI.back]{back} \hyperref[TEI.body]{body} \hyperref[TEI.div]{div} \hyperref[TEI.docAuthor]{docAuthor} \hyperref[TEI.docDate]{docDate} \hyperref[TEI.docEdition]{docEdition} \hyperref[TEI.docTitle]{docTitle} \hyperref[TEI.floatingText]{floatingText} \hyperref[TEI.front]{front} \hyperref[TEI.group]{group} \hyperref[TEI.text]{text} \hyperref[TEI.titlePage]{titlePage} \hyperref[TEI.titlePart]{titlePart}\par 
    \item[transcr: ]
   \hyperref[TEI.damage]{damage} \hyperref[TEI.fw]{fw} \hyperref[TEI.line]{line} \hyperref[TEI.metamark]{metamark} \hyperref[TEI.mod]{mod} \hyperref[TEI.restore]{restore} \hyperref[TEI.retrace]{retrace} \hyperref[TEI.secl]{secl} \hyperref[TEI.sourceDoc]{sourceDoc} \hyperref[TEI.subst]{subst} \hyperref[TEI.supplied]{supplied} \hyperref[TEI.surface]{surface} \hyperref[TEI.surfaceGrp]{surfaceGrp} \hyperref[TEI.surplus]{surplus} \hyperref[TEI.zone]{zone}
    \item[{Peut contenir}]
  Elément vide
    \item[{Note}]
  \par
Un élément \hyperref[TEI.pb]{<pb>} apparaît au début de la page à laquelle il se rapporte. L'attribut global {\itshape n} donne un numéro ou une autre valeur associée à cette page. Ce sera normalement le numéro de page ou la signature qui y est imprimée, puisque le numéro d'ordre matériel est implicite avec l'élément \hyperref[TEI.pb]{<pb>} lui-même.\par
L' attribut {\itshape type} sera employé pour indiquer toutes ses caractéristiques du saut de page, par exemple comme coupure de mot ou non.
    \item[{Exemple}]
  Page numbers may vary in different editions of a text.\leavevmode\bgroup\exampleFont \begin{shaded}\noindent\mbox{}{<\textbf{p}>} ... {<\textbf{pb}\hspace*{6pt}{ed}="{ed2}"\hspace*{6pt}{n}="{145}"/>}\mbox{}\newline 
\textit{<!-- Page 145 in edition "ed2" starts here -->} ... {<\textbf{pb}\hspace*{6pt}{ed}="{ed1}"\hspace*{6pt}{n}="{283}"/>}\mbox{}\newline 
\textit{<!-- Page 283 in edition "ed1" starts here-->} ... {</\textbf{p}>}\end{shaded}\egroup 


    \item[{Exemple}]
  A page break may be associated with a facsimile image of the page it introduces by means of the {\itshape facs} attribute\leavevmode\bgroup\exampleFont \begin{shaded}\noindent\mbox{}{<\textbf{body}>}\mbox{}\newline 
\hspace*{6pt}{<\textbf{pb}\hspace*{6pt}{facs}="{page1.png}"\hspace*{6pt}{n}="{1}"/>}\mbox{}\newline 
\textit{<!-- page1.png contains an image of the page;\newline
                        the text it contains is encoded here -->}\mbox{}\newline 
\hspace*{6pt}{<\textbf{p}>}\mbox{}\newline 
\textit{<!-- ... -->}\mbox{}\newline 
\hspace*{6pt}{</\textbf{p}>}\mbox{}\newline 
\hspace*{6pt}{<\textbf{pb}\hspace*{6pt}{facs}="{page2.png}"\hspace*{6pt}{n}="{2}"/>}\mbox{}\newline 
\textit{<!-- similarly, for page 2 -->}\mbox{}\newline 
\hspace*{6pt}{<\textbf{p}>}\mbox{}\newline 
\textit{<!-- ... -->}\mbox{}\newline 
\hspace*{6pt}{</\textbf{p}>}\mbox{}\newline 
{</\textbf{body}>}\end{shaded}\egroup 


    \item[{Modèle de contenu}]
  \fbox{\ttfamily <content>\newline
</content>\newline
    } 
    \item[{Schéma Declaration}]
  \mbox{}\hfill\\[-10pt]\begin{Verbatim}[fontsize=\small]
element pb
{
   tei_att.global.attributes,
   tei_att.typed.attributes,
   tei_att.edition.attributes,
   tei_att.spanning.attributes,
   tei_att.breaking.attributes,
   empty
}
\end{Verbatim}

\end{reflist}  \index{pc=<pc>|oddindex}\index{force=@force!<pc>|oddindex}\index{unit=@unit!<pc>|oddindex}\index{pre=@pre!<pc>|oddindex}
\begin{reflist}
\item[]\begin{specHead}{TEI.pc}{<pc> }(punctuation character) contient un caractère ou une chaîne de caractères considérés comme un signe de ponctuation unique. [\xref{http://www.tei-c.org/release/doc/tei-p5-doc/en/html/AI.html\#AIPC}{17.1.2. Below the Word Level}]\end{specHead} 
    \item[{Module}]
  analysis
    \item[{Attributs}]
  Attributs \hyperref[TEI.att.global]{att.global} (\textit{@xml:id}, \textit{@n}, \textit{@xml:lang}, \textit{@xml:base}, \textit{@xml:space})  (\hyperref[TEI.att.global.rendition]{att.global.rendition} (\textit{@rend}, \textit{@style}, \textit{@rendition})) (\hyperref[TEI.att.global.linking]{att.global.linking} (\textit{@corresp}, \textit{@synch}, \textit{@sameAs}, \textit{@copyOf}, \textit{@next}, \textit{@prev}, \textit{@exclude}, \textit{@select})) (\hyperref[TEI.att.global.analytic]{att.global.analytic} (\textit{@ana})) (\hyperref[TEI.att.global.facs]{att.global.facs} (\textit{@facs})) (\hyperref[TEI.att.global.change]{att.global.change} (\textit{@change})) (\hyperref[TEI.att.global.responsibility]{att.global.responsibility} (\textit{@cert}, \textit{@resp})) (\hyperref[TEI.att.global.source]{att.global.source} (\textit{@source})) \hyperref[TEI.att.segLike]{att.segLike} (\textit{@function})  (\hyperref[TEI.att.datcat]{att.datcat} (\textit{@datcat}, \textit{@valueDatcat})) (\hyperref[TEI.att.fragmentable]{att.fragmentable} (\textit{@part})) \hyperref[TEI.att.typed]{att.typed} (\textit{@type}, \textit{@subtype}) \hfil\\[-10pt]\begin{sansreflist}
    \item[@force]
  indicates the extent to which this punctuation mark conventionally separates words or phrases
\begin{reflist}
    \item[{Statut}]
  Optionel
    \item[{Type de données}]
  \hyperref[TEI.teidata.enumerated]{teidata.enumerated}
    \item[{Les valeurs autorisées sont:}]
  \begin{description}

\item[{strong}]the punctuation mark is a word separator
\item[{weak}]the punctuation mark is not a word separator
\item[{inter}]the punctuation mark may or may not be a word separator
\end{description} 
\end{reflist}  
    \item[@unit]
  provides a name for the kind of unit delimited by this punctuation mark.
\begin{reflist}
    \item[{Statut}]
  Optionel
    \item[{Type de données}]
  \hyperref[TEI.teidata.enumerated]{teidata.enumerated}
\end{reflist}  
    \item[@pre]
  indicates whether this punctuation mark precedes or follows the unit it delimits.
\begin{reflist}
    \item[{Statut}]
  Optionel
    \item[{Type de données}]
  \hyperref[TEI.teidata.truthValue]{teidata.truthValue}
\end{reflist}  
\end{sansreflist}  
    \item[{Membre du}]
  \hyperref[TEI.model.linePart]{model.linePart} \hyperref[TEI.model.segLike]{model.segLike} 
    \item[{Contenu dans}]
  
    \item[analysis: ]
   \hyperref[TEI.cl]{cl} \hyperref[TEI.phr]{phr} \hyperref[TEI.s]{s} \hyperref[TEI.w]{w}\par 
    \item[core: ]
   \hyperref[TEI.abbr]{abbr} \hyperref[TEI.add]{add} \hyperref[TEI.addrLine]{addrLine} \hyperref[TEI.author]{author} \hyperref[TEI.bibl]{bibl} \hyperref[TEI.biblScope]{biblScope} \hyperref[TEI.citedRange]{citedRange} \hyperref[TEI.corr]{corr} \hyperref[TEI.date]{date} \hyperref[TEI.del]{del} \hyperref[TEI.distinct]{distinct} \hyperref[TEI.editor]{editor} \hyperref[TEI.email]{email} \hyperref[TEI.emph]{emph} \hyperref[TEI.expan]{expan} \hyperref[TEI.foreign]{foreign} \hyperref[TEI.gloss]{gloss} \hyperref[TEI.head]{head} \hyperref[TEI.headItem]{headItem} \hyperref[TEI.headLabel]{headLabel} \hyperref[TEI.hi]{hi} \hyperref[TEI.item]{item} \hyperref[TEI.l]{l} \hyperref[TEI.label]{label} \hyperref[TEI.measure]{measure} \hyperref[TEI.mentioned]{mentioned} \hyperref[TEI.name]{name} \hyperref[TEI.note]{note} \hyperref[TEI.num]{num} \hyperref[TEI.orig]{orig} \hyperref[TEI.p]{p} \hyperref[TEI.pubPlace]{pubPlace} \hyperref[TEI.publisher]{publisher} \hyperref[TEI.q]{q} \hyperref[TEI.quote]{quote} \hyperref[TEI.ref]{ref} \hyperref[TEI.reg]{reg} \hyperref[TEI.rs]{rs} \hyperref[TEI.said]{said} \hyperref[TEI.sic]{sic} \hyperref[TEI.soCalled]{soCalled} \hyperref[TEI.speaker]{speaker} \hyperref[TEI.stage]{stage} \hyperref[TEI.street]{street} \hyperref[TEI.term]{term} \hyperref[TEI.textLang]{textLang} \hyperref[TEI.time]{time} \hyperref[TEI.title]{title} \hyperref[TEI.unclear]{unclear}\par 
    \item[figures: ]
   \hyperref[TEI.cell]{cell}\par 
    \item[header: ]
   \hyperref[TEI.change]{change} \hyperref[TEI.distributor]{distributor} \hyperref[TEI.edition]{edition} \hyperref[TEI.extent]{extent} \hyperref[TEI.licence]{licence}\par 
    \item[linking: ]
   \hyperref[TEI.ab]{ab} \hyperref[TEI.seg]{seg}\par 
    \item[msdescription: ]
   \hyperref[TEI.accMat]{accMat} \hyperref[TEI.acquisition]{acquisition} \hyperref[TEI.additions]{additions} \hyperref[TEI.catchwords]{catchwords} \hyperref[TEI.collation]{collation} \hyperref[TEI.colophon]{colophon} \hyperref[TEI.condition]{condition} \hyperref[TEI.custEvent]{custEvent} \hyperref[TEI.decoNote]{decoNote} \hyperref[TEI.explicit]{explicit} \hyperref[TEI.filiation]{filiation} \hyperref[TEI.finalRubric]{finalRubric} \hyperref[TEI.foliation]{foliation} \hyperref[TEI.heraldry]{heraldry} \hyperref[TEI.incipit]{incipit} \hyperref[TEI.layout]{layout} \hyperref[TEI.material]{material} \hyperref[TEI.musicNotation]{musicNotation} \hyperref[TEI.objectType]{objectType} \hyperref[TEI.origDate]{origDate} \hyperref[TEI.origPlace]{origPlace} \hyperref[TEI.origin]{origin} \hyperref[TEI.provenance]{provenance} \hyperref[TEI.rubric]{rubric} \hyperref[TEI.secFol]{secFol} \hyperref[TEI.signatures]{signatures} \hyperref[TEI.source]{source} \hyperref[TEI.stamp]{stamp} \hyperref[TEI.summary]{summary} \hyperref[TEI.support]{support} \hyperref[TEI.surrogates]{surrogates} \hyperref[TEI.typeNote]{typeNote} \hyperref[TEI.watermark]{watermark}\par 
    \item[namesdates: ]
   \hyperref[TEI.addName]{addName} \hyperref[TEI.affiliation]{affiliation} \hyperref[TEI.country]{country} \hyperref[TEI.forename]{forename} \hyperref[TEI.genName]{genName} \hyperref[TEI.geogName]{geogName} \hyperref[TEI.nameLink]{nameLink} \hyperref[TEI.orgName]{orgName} \hyperref[TEI.persName]{persName} \hyperref[TEI.placeName]{placeName} \hyperref[TEI.region]{region} \hyperref[TEI.roleName]{roleName} \hyperref[TEI.settlement]{settlement} \hyperref[TEI.surname]{surname}\par 
    \item[textstructure: ]
   \hyperref[TEI.docAuthor]{docAuthor} \hyperref[TEI.docDate]{docDate} \hyperref[TEI.docEdition]{docEdition} \hyperref[TEI.titlePart]{titlePart}\par 
    \item[transcr: ]
   \hyperref[TEI.damage]{damage} \hyperref[TEI.fw]{fw} \hyperref[TEI.line]{line} \hyperref[TEI.metamark]{metamark} \hyperref[TEI.mod]{mod} \hyperref[TEI.restore]{restore} \hyperref[TEI.retrace]{retrace} \hyperref[TEI.secl]{secl} \hyperref[TEI.supplied]{supplied} \hyperref[TEI.surplus]{surplus} \hyperref[TEI.zone]{zone}
    \item[{Peut contenir}]
  
    \item[analysis: ]
   \hyperref[TEI.c]{c}\par 
    \item[core: ]
   \hyperref[TEI.abbr]{abbr} \hyperref[TEI.add]{add} \hyperref[TEI.choice]{choice} \hyperref[TEI.corr]{corr} \hyperref[TEI.del]{del} \hyperref[TEI.expan]{expan} \hyperref[TEI.orig]{orig} \hyperref[TEI.reg]{reg} \hyperref[TEI.sic]{sic} \hyperref[TEI.unclear]{unclear}\par 
    \item[transcr: ]
   \hyperref[TEI.am]{am} \hyperref[TEI.damage]{damage} \hyperref[TEI.ex]{ex} \hyperref[TEI.handShift]{handShift} \hyperref[TEI.mod]{mod} \hyperref[TEI.redo]{redo} \hyperref[TEI.restore]{restore} \hyperref[TEI.retrace]{retrace} \hyperref[TEI.secl]{secl} \hyperref[TEI.subst]{subst} \hyperref[TEI.supplied]{supplied} \hyperref[TEI.surplus]{surplus} \hyperref[TEI.undo]{undo}\par des données textuelles
    \item[{Exemple}]
  \leavevmode\bgroup\exampleFont \begin{shaded}\noindent\mbox{}{<\textbf{phr}>}\mbox{}\newline 
\hspace*{6pt}{<\textbf{w}>}do{</\textbf{w}>}\mbox{}\newline 
\hspace*{6pt}{<\textbf{w}>}you{</\textbf{w}>}\mbox{}\newline 
\hspace*{6pt}{<\textbf{w}>}understand{</\textbf{w}>}\mbox{}\newline 
\hspace*{6pt}{<\textbf{pc}\hspace*{6pt}{type}="{interrogative}">}?{</\textbf{pc}>}\mbox{}\newline 
{</\textbf{phr}>}\end{shaded}\egroup 


    \item[{Modèle de contenu}]
  \mbox{}\hfill\\[-10pt]\begin{Verbatim}[fontsize=\small]
<content>
 <alternate maxOccurs="unbounded"
  minOccurs="0">
  <textNode/>
  <classRef key="model.gLike"/>
  <elementRef key="c"/>
  <classRef key="model.pPart.edit"/>
 </alternate>
</content>
    
\end{Verbatim}

    \item[{Schéma Declaration}]
  \mbox{}\hfill\\[-10pt]\begin{Verbatim}[fontsize=\small]
element pc
{
   tei_att.global.attributes,
   tei_att.segLike.attributes,
   tei_att.typed.attributes,
   attribute force { "strong" | "weak" | "inter" }?,
   attribute unit { text }?,
   attribute pre { text }?,
   ( text | tei_model.gLike | tei_c | tei_model.pPart.edit )*
}
\end{Verbatim}

\end{reflist}  \index{persName=<persName>|oddindex}\index{scheme=@scheme!<persName>|oddindex}
\begin{reflist}
\item[]\begin{specHead}{TEI.persName}{<persName> }(nom de personne) contient un nom propre ou une expression nominale se référant à une personne, pouvant inclure tout ou partie de ses prénoms, noms de famille, titres honorifiques, noms ajoutés, etc. [\xref{http://www.tei-c.org/release/doc/tei-p5-doc/en/html/ND.html\#NDPER}{13.2.1. Personal Names}]\end{specHead} 
    \item[{Module}]
  namesdates
    \item[{Attributs}]
  Attributs \hyperref[TEI.att.global]{att.global} (\textit{@xml:id}, \textit{@n}, \textit{@xml:lang}, \textit{@xml:base}, \textit{@xml:space})  (\hyperref[TEI.att.global.rendition]{att.global.rendition} (\textit{@rend}, \textit{@style}, \textit{@rendition})) (\hyperref[TEI.att.global.linking]{att.global.linking} (\textit{@corresp}, \textit{@synch}, \textit{@sameAs}, \textit{@copyOf}, \textit{@next}, \textit{@prev}, \textit{@exclude}, \textit{@select})) (\hyperref[TEI.att.global.analytic]{att.global.analytic} (\textit{@ana})) (\hyperref[TEI.att.global.facs]{att.global.facs} (\textit{@facs})) (\hyperref[TEI.att.global.change]{att.global.change} (\textit{@change})) (\hyperref[TEI.att.global.responsibility]{att.global.responsibility} (\textit{@cert}, \textit{@resp})) (\hyperref[TEI.att.global.source]{att.global.source} (\textit{@source})) \hyperref[TEI.att.datable]{att.datable} (\textit{@calendar}, \textit{@period})  (\hyperref[TEI.att.datable.w3c]{att.datable.w3c} (\textit{@when}, \textit{@notBefore}, \textit{@notAfter}, \textit{@from}, \textit{@to})) (\hyperref[TEI.att.datable.iso]{att.datable.iso} (\textit{@when-iso}, \textit{@notBefore-iso}, \textit{@notAfter-iso}, \textit{@from-iso}, \textit{@to-iso})) (\hyperref[TEI.att.datable.custom]{att.datable.custom} (\textit{@when-custom}, \textit{@notBefore-custom}, \textit{@notAfter-custom}, \textit{@from-custom}, \textit{@to-custom}, \textit{@datingPoint}, \textit{@datingMethod})) \hyperref[TEI.att.editLike]{att.editLike} (\textit{@evidence}, \textit{@instant})  (\hyperref[TEI.att.dimensions]{att.dimensions} (\textit{@unit}, \textit{@quantity}, \textit{@extent}, \textit{@precision}, \textit{@scope}) (\hyperref[TEI.att.ranging]{att.ranging} (\textit{@atLeast}, \textit{@atMost}, \textit{@min}, \textit{@max}, \textit{@confidence})) ) \hyperref[TEI.att.personal]{att.personal} (\textit{@full}, \textit{@sort})  (\hyperref[TEI.att.naming]{att.naming} (\textit{@role}, \textit{@nymRef}) (\hyperref[TEI.att.canonical]{att.canonical} (\textit{@key}, \textit{@ref})) ) \hyperref[TEI.att.typed]{att.typed} (\textit{@type}, \textit{@subtype}) \hfil\\[-10pt]\begin{sansreflist}
    \item[@scheme]
  désigne la liste des ontologies dans lequel l'ensemble des termes concernés sont définis.
\begin{reflist}
    \item[{Statut}]
  Optionel
    \item[{Type de données}]
  \hyperref[TEI.teidata.pointer]{teidata.pointer}
\end{reflist}  
\end{sansreflist}  
    \item[{Membre du}]
  \hyperref[TEI.model.nameLike.agent]{model.nameLike.agent} \hyperref[TEI.model.persStateLike]{model.persStateLike}
    \item[{Contenu dans}]
  
    \item[analysis: ]
   \hyperref[TEI.cl]{cl} \hyperref[TEI.phr]{phr} \hyperref[TEI.s]{s} \hyperref[TEI.span]{span}\par 
    \item[core: ]
   \hyperref[TEI.abbr]{abbr} \hyperref[TEI.add]{add} \hyperref[TEI.addrLine]{addrLine} \hyperref[TEI.address]{address} \hyperref[TEI.author]{author} \hyperref[TEI.bibl]{bibl} \hyperref[TEI.biblScope]{biblScope} \hyperref[TEI.citedRange]{citedRange} \hyperref[TEI.corr]{corr} \hyperref[TEI.date]{date} \hyperref[TEI.del]{del} \hyperref[TEI.desc]{desc} \hyperref[TEI.distinct]{distinct} \hyperref[TEI.editor]{editor} \hyperref[TEI.email]{email} \hyperref[TEI.emph]{emph} \hyperref[TEI.expan]{expan} \hyperref[TEI.foreign]{foreign} \hyperref[TEI.gloss]{gloss} \hyperref[TEI.head]{head} \hyperref[TEI.headItem]{headItem} \hyperref[TEI.headLabel]{headLabel} \hyperref[TEI.hi]{hi} \hyperref[TEI.item]{item} \hyperref[TEI.l]{l} \hyperref[TEI.label]{label} \hyperref[TEI.measure]{measure} \hyperref[TEI.meeting]{meeting} \hyperref[TEI.mentioned]{mentioned} \hyperref[TEI.name]{name} \hyperref[TEI.note]{note} \hyperref[TEI.num]{num} \hyperref[TEI.orig]{orig} \hyperref[TEI.p]{p} \hyperref[TEI.pubPlace]{pubPlace} \hyperref[TEI.publisher]{publisher} \hyperref[TEI.q]{q} \hyperref[TEI.quote]{quote} \hyperref[TEI.ref]{ref} \hyperref[TEI.reg]{reg} \hyperref[TEI.resp]{resp} \hyperref[TEI.respStmt]{respStmt} \hyperref[TEI.rs]{rs} \hyperref[TEI.said]{said} \hyperref[TEI.sic]{sic} \hyperref[TEI.soCalled]{soCalled} \hyperref[TEI.speaker]{speaker} \hyperref[TEI.stage]{stage} \hyperref[TEI.street]{street} \hyperref[TEI.term]{term} \hyperref[TEI.textLang]{textLang} \hyperref[TEI.time]{time} \hyperref[TEI.title]{title} \hyperref[TEI.unclear]{unclear}\par 
    \item[figures: ]
   \hyperref[TEI.cell]{cell} \hyperref[TEI.figDesc]{figDesc}\par 
    \item[header: ]
   \hyperref[TEI.authority]{authority} \hyperref[TEI.change]{change} \hyperref[TEI.classCode]{classCode} \hyperref[TEI.creation]{creation} \hyperref[TEI.distributor]{distributor} \hyperref[TEI.edition]{edition} \hyperref[TEI.extent]{extent} \hyperref[TEI.funder]{funder} \hyperref[TEI.language]{language} \hyperref[TEI.licence]{licence} \hyperref[TEI.rendition]{rendition}\par 
    \item[iso-fs: ]
   \hyperref[TEI.fDescr]{fDescr} \hyperref[TEI.fsDescr]{fsDescr}\par 
    \item[linking: ]
   \hyperref[TEI.ab]{ab} \hyperref[TEI.seg]{seg}\par 
    \item[msdescription: ]
   \hyperref[TEI.accMat]{accMat} \hyperref[TEI.acquisition]{acquisition} \hyperref[TEI.additions]{additions} \hyperref[TEI.catchwords]{catchwords} \hyperref[TEI.collation]{collation} \hyperref[TEI.colophon]{colophon} \hyperref[TEI.condition]{condition} \hyperref[TEI.custEvent]{custEvent} \hyperref[TEI.decoNote]{decoNote} \hyperref[TEI.explicit]{explicit} \hyperref[TEI.filiation]{filiation} \hyperref[TEI.finalRubric]{finalRubric} \hyperref[TEI.foliation]{foliation} \hyperref[TEI.heraldry]{heraldry} \hyperref[TEI.incipit]{incipit} \hyperref[TEI.layout]{layout} \hyperref[TEI.material]{material} \hyperref[TEI.musicNotation]{musicNotation} \hyperref[TEI.objectType]{objectType} \hyperref[TEI.origDate]{origDate} \hyperref[TEI.origPlace]{origPlace} \hyperref[TEI.origin]{origin} \hyperref[TEI.provenance]{provenance} \hyperref[TEI.rubric]{rubric} \hyperref[TEI.secFol]{secFol} \hyperref[TEI.signatures]{signatures} \hyperref[TEI.source]{source} \hyperref[TEI.stamp]{stamp} \hyperref[TEI.summary]{summary} \hyperref[TEI.support]{support} \hyperref[TEI.surrogates]{surrogates} \hyperref[TEI.typeNote]{typeNote} \hyperref[TEI.watermark]{watermark}\par 
    \item[namesdates: ]
   \hyperref[TEI.addName]{addName} \hyperref[TEI.affiliation]{affiliation} \hyperref[TEI.country]{country} \hyperref[TEI.forename]{forename} \hyperref[TEI.genName]{genName} \hyperref[TEI.geogName]{geogName} \hyperref[TEI.nameLink]{nameLink} \hyperref[TEI.org]{org} \hyperref[TEI.orgName]{orgName} \hyperref[TEI.persName]{persName} \hyperref[TEI.person]{person} \hyperref[TEI.personGrp]{personGrp} \hyperref[TEI.persona]{persona} \hyperref[TEI.placeName]{placeName} \hyperref[TEI.region]{region} \hyperref[TEI.roleName]{roleName} \hyperref[TEI.settlement]{settlement} \hyperref[TEI.surname]{surname}\par 
    \item[spoken: ]
   \hyperref[TEI.annotationBlock]{annotationBlock}\par 
    \item[standOff: ]
   \hyperref[TEI.listAnnotation]{listAnnotation}\par 
    \item[textstructure: ]
   \hyperref[TEI.docAuthor]{docAuthor} \hyperref[TEI.docDate]{docDate} \hyperref[TEI.docEdition]{docEdition} \hyperref[TEI.titlePart]{titlePart}\par 
    \item[transcr: ]
   \hyperref[TEI.damage]{damage} \hyperref[TEI.fw]{fw} \hyperref[TEI.metamark]{metamark} \hyperref[TEI.mod]{mod} \hyperref[TEI.restore]{restore} \hyperref[TEI.retrace]{retrace} \hyperref[TEI.secl]{secl} \hyperref[TEI.supplied]{supplied} \hyperref[TEI.surplus]{surplus}
    \item[{Peut contenir}]
  
    \item[analysis: ]
   \hyperref[TEI.c]{c} \hyperref[TEI.cl]{cl} \hyperref[TEI.interp]{interp} \hyperref[TEI.interpGrp]{interpGrp} \hyperref[TEI.m]{m} \hyperref[TEI.pc]{pc} \hyperref[TEI.phr]{phr} \hyperref[TEI.s]{s} \hyperref[TEI.span]{span} \hyperref[TEI.spanGrp]{spanGrp} \hyperref[TEI.w]{w}\par 
    \item[core: ]
   \hyperref[TEI.abbr]{abbr} \hyperref[TEI.add]{add} \hyperref[TEI.address]{address} \hyperref[TEI.binaryObject]{binaryObject} \hyperref[TEI.cb]{cb} \hyperref[TEI.choice]{choice} \hyperref[TEI.corr]{corr} \hyperref[TEI.date]{date} \hyperref[TEI.del]{del} \hyperref[TEI.distinct]{distinct} \hyperref[TEI.email]{email} \hyperref[TEI.emph]{emph} \hyperref[TEI.expan]{expan} \hyperref[TEI.foreign]{foreign} \hyperref[TEI.gap]{gap} \hyperref[TEI.gb]{gb} \hyperref[TEI.gloss]{gloss} \hyperref[TEI.graphic]{graphic} \hyperref[TEI.hi]{hi} \hyperref[TEI.index]{index} \hyperref[TEI.lb]{lb} \hyperref[TEI.measure]{measure} \hyperref[TEI.measureGrp]{measureGrp} \hyperref[TEI.media]{media} \hyperref[TEI.mentioned]{mentioned} \hyperref[TEI.milestone]{milestone} \hyperref[TEI.name]{name} \hyperref[TEI.note]{note} \hyperref[TEI.num]{num} \hyperref[TEI.orig]{orig} \hyperref[TEI.pb]{pb} \hyperref[TEI.ptr]{ptr} \hyperref[TEI.ref]{ref} \hyperref[TEI.reg]{reg} \hyperref[TEI.rs]{rs} \hyperref[TEI.sic]{sic} \hyperref[TEI.soCalled]{soCalled} \hyperref[TEI.term]{term} \hyperref[TEI.time]{time} \hyperref[TEI.title]{title} \hyperref[TEI.unclear]{unclear}\par 
    \item[derived-module-tei.istex: ]
   \hyperref[TEI.math]{math} \hyperref[TEI.mrow]{mrow}\par 
    \item[figures: ]
   \hyperref[TEI.figure]{figure} \hyperref[TEI.formula]{formula} \hyperref[TEI.notatedMusic]{notatedMusic}\par 
    \item[header: ]
   \hyperref[TEI.idno]{idno}\par 
    \item[iso-fs: ]
   \hyperref[TEI.fLib]{fLib} \hyperref[TEI.fs]{fs} \hyperref[TEI.fvLib]{fvLib}\par 
    \item[linking: ]
   \hyperref[TEI.alt]{alt} \hyperref[TEI.altGrp]{altGrp} \hyperref[TEI.anchor]{anchor} \hyperref[TEI.join]{join} \hyperref[TEI.joinGrp]{joinGrp} \hyperref[TEI.link]{link} \hyperref[TEI.linkGrp]{linkGrp} \hyperref[TEI.seg]{seg} \hyperref[TEI.timeline]{timeline}\par 
    \item[msdescription: ]
   \hyperref[TEI.catchwords]{catchwords} \hyperref[TEI.depth]{depth} \hyperref[TEI.dim]{dim} \hyperref[TEI.dimensions]{dimensions} \hyperref[TEI.height]{height} \hyperref[TEI.heraldry]{heraldry} \hyperref[TEI.locus]{locus} \hyperref[TEI.locusGrp]{locusGrp} \hyperref[TEI.material]{material} \hyperref[TEI.objectType]{objectType} \hyperref[TEI.origDate]{origDate} \hyperref[TEI.origPlace]{origPlace} \hyperref[TEI.secFol]{secFol} \hyperref[TEI.signatures]{signatures} \hyperref[TEI.source]{source} \hyperref[TEI.stamp]{stamp} \hyperref[TEI.watermark]{watermark} \hyperref[TEI.width]{width}\par 
    \item[namesdates: ]
   \hyperref[TEI.addName]{addName} \hyperref[TEI.affiliation]{affiliation} \hyperref[TEI.country]{country} \hyperref[TEI.forename]{forename} \hyperref[TEI.genName]{genName} \hyperref[TEI.geogName]{geogName} \hyperref[TEI.location]{location} \hyperref[TEI.nameLink]{nameLink} \hyperref[TEI.orgName]{orgName} \hyperref[TEI.persName]{persName} \hyperref[TEI.placeName]{placeName} \hyperref[TEI.region]{region} \hyperref[TEI.roleName]{roleName} \hyperref[TEI.settlement]{settlement} \hyperref[TEI.state]{state} \hyperref[TEI.surname]{surname}\par 
    \item[spoken: ]
   \hyperref[TEI.annotationBlock]{annotationBlock}\par 
    \item[transcr: ]
   \hyperref[TEI.addSpan]{addSpan} \hyperref[TEI.am]{am} \hyperref[TEI.damage]{damage} \hyperref[TEI.damageSpan]{damageSpan} \hyperref[TEI.delSpan]{delSpan} \hyperref[TEI.ex]{ex} \hyperref[TEI.fw]{fw} \hyperref[TEI.handShift]{handShift} \hyperref[TEI.listTranspose]{listTranspose} \hyperref[TEI.metamark]{metamark} \hyperref[TEI.mod]{mod} \hyperref[TEI.redo]{redo} \hyperref[TEI.restore]{restore} \hyperref[TEI.retrace]{retrace} \hyperref[TEI.secl]{secl} \hyperref[TEI.space]{space} \hyperref[TEI.subst]{subst} \hyperref[TEI.substJoin]{substJoin} \hyperref[TEI.supplied]{supplied} \hyperref[TEI.surplus]{surplus} \hyperref[TEI.undo]{undo}\par des données textuelles
    \item[{Exemple}]
  StandOff entité nommée persName\leavevmode\bgroup\exampleFont \begin{shaded}\noindent\mbox{}{<\textbf{annotationBlock}\hspace*{6pt}{corresp}="{text}">}\mbox{}\newline 
\hspace*{6pt}{<\textbf{persName}\hspace*{6pt}{change}="{\#Unitex-3.2.0-alpha}"\mbox{}\newline 
\hspace*{6pt}\hspace*{6pt}{resp}="{istex}"\mbox{}\newline 
\hspace*{6pt}\hspace*{6pt}{scheme}="{https://persname-entity.data.istex.fr}">}\mbox{}\newline 
\hspace*{6pt}\hspace*{6pt}{<\textbf{term}>}Timothy Adams{</\textbf{term}>}\mbox{}\newline 
\hspace*{6pt}\hspace*{6pt}{<\textbf{fs}\hspace*{6pt}{type}="{statistics}">}\mbox{}\newline 
\hspace*{6pt}\hspace*{6pt}\hspace*{6pt}{<\textbf{f}\hspace*{6pt}{name}="{frequency}">}\mbox{}\newline 
\hspace*{6pt}\hspace*{6pt}\hspace*{6pt}\hspace*{6pt}{<\textbf{numeric}\hspace*{6pt}{value}="{1}"/>}\mbox{}\newline 
\hspace*{6pt}\hspace*{6pt}\hspace*{6pt}{</\textbf{f}>}\mbox{}\newline 
\hspace*{6pt}\hspace*{6pt}{</\textbf{fs}>}\mbox{}\newline 
\hspace*{6pt}{</\textbf{persName}>}\mbox{}\newline 
{</\textbf{annotationBlock}>}\end{shaded}\egroup 


    \item[{Modèle de contenu}]
  \mbox{}\hfill\\[-10pt]\begin{Verbatim}[fontsize=\small]
<content>
 <macroRef key="macro.phraseSeq"/>
</content>
    
\end{Verbatim}

    \item[{Schéma Declaration}]
  \mbox{}\hfill\\[-10pt]\begin{Verbatim}[fontsize=\small]
element persName
{
   tei_att.global.attributes,
   tei_att.datable.attributes,
   tei_att.editLike.attributes,
   tei_att.personal.attributes,
   tei_att.typed.attributes,
   attribute scheme { text }?,
   tei_macro.phraseSeq}
\end{Verbatim}

\end{reflist}  \index{person=<person>|oddindex}\index{role=@role!<person>|oddindex}\index{sex=@sex!<person>|oddindex}\index{age=@age!<person>|oddindex}
\begin{reflist}
\item[]\begin{specHead}{TEI.person}{<person> }(personne) fournit des informations sur un individu identifiable, par exemple un participant à une interaction linguistique, ou une personne citée dans une source historique. [\xref{http://www.tei-c.org/release/doc/tei-p5-doc/en/html/ND.html\#NDPERSE}{13.3.2. The Person Element} \xref{http://www.tei-c.org/release/doc/tei-p5-doc/en/html/CC.html\#CCAHPA}{15.2.2. The Participant Description}]\end{specHead} 
    \item[{Module}]
  namesdates
    \item[{Attributs}]
  Attributs \hyperref[TEI.att.global]{att.global} (\textit{@xml:id}, \textit{@n}, \textit{@xml:lang}, \textit{@xml:base}, \textit{@xml:space})  (\hyperref[TEI.att.global.rendition]{att.global.rendition} (\textit{@rend}, \textit{@style}, \textit{@rendition})) (\hyperref[TEI.att.global.linking]{att.global.linking} (\textit{@corresp}, \textit{@synch}, \textit{@sameAs}, \textit{@copyOf}, \textit{@next}, \textit{@prev}, \textit{@exclude}, \textit{@select})) (\hyperref[TEI.att.global.analytic]{att.global.analytic} (\textit{@ana})) (\hyperref[TEI.att.global.facs]{att.global.facs} (\textit{@facs})) (\hyperref[TEI.att.global.change]{att.global.change} (\textit{@change})) (\hyperref[TEI.att.global.responsibility]{att.global.responsibility} (\textit{@cert}, \textit{@resp})) (\hyperref[TEI.att.global.source]{att.global.source} (\textit{@source})) \hyperref[TEI.att.editLike]{att.editLike} (\textit{@evidence}, \textit{@instant})  (\hyperref[TEI.att.dimensions]{att.dimensions} (\textit{@unit}, \textit{@quantity}, \textit{@extent}, \textit{@precision}, \textit{@scope}) (\hyperref[TEI.att.ranging]{att.ranging} (\textit{@atLeast}, \textit{@atMost}, \textit{@min}, \textit{@max}, \textit{@confidence})) ) \hyperref[TEI.att.sortable]{att.sortable} (\textit{@sortKey}) \hfil\\[-10pt]\begin{sansreflist}
    \item[@role]
  précise un rôle principal ou une classification principale pour cette personne.
\begin{reflist}
    \item[{Statut}]
  Optionel
    \item[{Type de données}]
  1–∞ occurrences de \hyperref[TEI.teidata.enumerated]{teidata.enumerated} séparé par un espace
\end{reflist}  
    \item[@sex]
  précise le sexe de la personne.
\begin{reflist}
    \item[{Statut}]
  Optionel
    \item[{Type de données}]
  1–∞ occurrences de \hyperref[TEI.teidata.sex]{teidata.sex} séparé par un espace
\end{reflist}  
    \item[@age]
  précise une tranche d'âge pour la personne.
\begin{reflist}
    \item[{Statut}]
  Optionel
    \item[{Type de données}]
  \hyperref[TEI.teidata.enumerated]{teidata.enumerated}
\end{reflist}  
\end{sansreflist}  
    \item[{Membre du}]
  \hyperref[TEI.model.OABody]{model.OABody} \hyperref[TEI.model.personLike]{model.personLike}
    \item[{Contenu dans}]
  
    \item[namesdates: ]
   \hyperref[TEI.org]{org}\par 
    \item[spoken: ]
   \hyperref[TEI.annotationBlock]{annotationBlock}
    \item[{Peut contenir}]
  
    \item[analysis: ]
   \hyperref[TEI.interp]{interp} \hyperref[TEI.interpGrp]{interpGrp} \hyperref[TEI.span]{span} \hyperref[TEI.spanGrp]{spanGrp}\par 
    \item[core: ]
   \hyperref[TEI.bibl]{bibl} \hyperref[TEI.biblStruct]{biblStruct} \hyperref[TEI.cb]{cb} \hyperref[TEI.gap]{gap} \hyperref[TEI.gb]{gb} \hyperref[TEI.index]{index} \hyperref[TEI.lb]{lb} \hyperref[TEI.listBibl]{listBibl} \hyperref[TEI.milestone]{milestone} \hyperref[TEI.name]{name} \hyperref[TEI.note]{note} \hyperref[TEI.p]{p} \hyperref[TEI.pb]{pb}\par 
    \item[figures: ]
   \hyperref[TEI.figure]{figure} \hyperref[TEI.notatedMusic]{notatedMusic}\par 
    \item[header: ]
   \hyperref[TEI.biblFull]{biblFull} \hyperref[TEI.idno]{idno}\par 
    \item[iso-fs: ]
   \hyperref[TEI.fLib]{fLib} \hyperref[TEI.fs]{fs} \hyperref[TEI.fvLib]{fvLib}\par 
    \item[linking: ]
   \hyperref[TEI.ab]{ab} \hyperref[TEI.alt]{alt} \hyperref[TEI.altGrp]{altGrp} \hyperref[TEI.anchor]{anchor} \hyperref[TEI.join]{join} \hyperref[TEI.joinGrp]{joinGrp} \hyperref[TEI.link]{link} \hyperref[TEI.linkGrp]{linkGrp} \hyperref[TEI.timeline]{timeline}\par 
    \item[msdescription: ]
   \hyperref[TEI.msDesc]{msDesc} \hyperref[TEI.source]{source}\par 
    \item[namesdates: ]
   \hyperref[TEI.affiliation]{affiliation} \hyperref[TEI.event]{event} \hyperref[TEI.persName]{persName} \hyperref[TEI.persona]{persona} \hyperref[TEI.state]{state}\par 
    \item[transcr: ]
   \hyperref[TEI.addSpan]{addSpan} \hyperref[TEI.damageSpan]{damageSpan} \hyperref[TEI.delSpan]{delSpan} \hyperref[TEI.fw]{fw} \hyperref[TEI.listTranspose]{listTranspose} \hyperref[TEI.metamark]{metamark} \hyperref[TEI.space]{space} \hyperref[TEI.substJoin]{substJoin}
    \item[{Note}]
  \par
Peut contenir soit une description en prose organisée en paragraphes, soit une suite d'éléments spécifiques relatifs à la démographie extraits de la classe \textsf{model.personPart}.
    \item[{Exemple}]
  \leavevmode\bgroup\exampleFont \begin{shaded}\noindent\mbox{}{<\textbf{person}\hspace*{6pt}{age}="{adult}"\hspace*{6pt}{sex}="{2}">}\mbox{}\newline 
\hspace*{6pt}{<\textbf{p}>}Personne interrogée, de sexe féminin, instruite, née à Shropshire, au ROYAUME-UNI le 12\mbox{}\newline 
\hspace*{6pt}\hspace*{6pt} Janvier 1950, d'occupation inconnue. Parle le français couramment. Statut\mbox{}\newline 
\hspace*{6pt}\hspace*{6pt} socio-économique B2.{</\textbf{p}>}\mbox{}\newline 
{</\textbf{person}>}\end{shaded}\egroup 


    \item[{Exemple}]
  \leavevmode\bgroup\exampleFont \begin{shaded}\noindent\mbox{}{<\textbf{person}\hspace*{6pt}{role}="{poet}"\hspace*{6pt}{sex}="{1}"\mbox{}\newline 
\hspace*{6pt}{xml:id}="{fr\textunderscore Ovi01}">}\mbox{}\newline 
\hspace*{6pt}{<\textbf{persName}\hspace*{6pt}{xml:lang}="{en}">}Ovid{</\textbf{persName}>}\mbox{}\newline 
\hspace*{6pt}{<\textbf{persName}\hspace*{6pt}{xml:lang}="{la}">}Publius Ovidius Naso{</\textbf{persName}>}\mbox{}\newline 
\hspace*{6pt}{<\textbf{birth}\hspace*{6pt}{when}="{-0044-03-20}">} 20 March 43 BC{<\textbf{placeName}>}\mbox{}\newline 
\hspace*{6pt}\hspace*{6pt}\hspace*{6pt}{<\textbf{settlement}\hspace*{6pt}{type}="{city}">}Sulmona{</\textbf{settlement}>}\mbox{}\newline 
\hspace*{6pt}\hspace*{6pt}\hspace*{6pt}{<\textbf{country}\hspace*{6pt}{key}="{IT}">}Italie{</\textbf{country}>}\mbox{}\newline 
\hspace*{6pt}\hspace*{6pt}{</\textbf{placeName}>}\mbox{}\newline 
\hspace*{6pt}{</\textbf{birth}>}\mbox{}\newline 
\hspace*{6pt}{<\textbf{death}\hspace*{6pt}{notAfter}="{0018}"\hspace*{6pt}{notBefore}="{0017}">}17 or 18 AD{<\textbf{placeName}>}\mbox{}\newline 
\hspace*{6pt}\hspace*{6pt}\hspace*{6pt}{<\textbf{settlement}\hspace*{6pt}{type}="{city}">}Tomis (Constanta){</\textbf{settlement}>}\mbox{}\newline 
\hspace*{6pt}\hspace*{6pt}\hspace*{6pt}{<\textbf{country}\hspace*{6pt}{key}="{RO}">}Roumanie{</\textbf{country}>}\mbox{}\newline 
\hspace*{6pt}\hspace*{6pt}{</\textbf{placeName}>}\mbox{}\newline 
\hspace*{6pt}{</\textbf{death}>}\mbox{}\newline 
{</\textbf{person}>}\end{shaded}\egroup 


    \item[{Modèle de contenu}]
  \mbox{}\hfill\\[-10pt]\begin{Verbatim}[fontsize=\small]
<content>
 <alternate maxOccurs="1" minOccurs="1">
  <classRef key="model.pLike"
   maxOccurs="unbounded" minOccurs="1"/>
  <alternate maxOccurs="unbounded"
   minOccurs="0">
   <classRef key="model.personPart"/>
   <classRef key="model.global"/>
  </alternate>
 </alternate>
</content>
    
\end{Verbatim}

    \item[{Schéma Declaration}]
  \mbox{}\hfill\\[-10pt]\begin{Verbatim}[fontsize=\small]
element person
{
   tei_att.global.attributes,
   tei_att.editLike.attributes,
   tei_att.sortable.attributes,
   attribute role { list { + } }?,
   attribute sex { list { + } }?,
   attribute age { text }?,
   ( tei_model.pLike+ | ( tei_model.personPart | tei_model.global )* )
}
\end{Verbatim}

\end{reflist}  \index{personGrp=<personGrp>|oddindex}\index{role=@role!<personGrp>|oddindex}\index{sex=@sex!<personGrp>|oddindex}\index{age=@age!<personGrp>|oddindex}\index{size=@size!<personGrp>|oddindex}
\begin{reflist}
\item[]\begin{specHead}{TEI.personGrp}{<personGrp> }(groupe de personnes) décrit un groupe d'individus traité comme une personne unique à des fins d'analyse. [\xref{http://www.tei-c.org/release/doc/tei-p5-doc/en/html/CC.html\#CCAHPA}{15.2.2. The Participant Description}]\end{specHead} 
    \item[{Module}]
  namesdates
    \item[{Attributs}]
  Attributs \hyperref[TEI.att.global]{att.global} (\textit{@xml:id}, \textit{@n}, \textit{@xml:lang}, \textit{@xml:base}, \textit{@xml:space})  (\hyperref[TEI.att.global.rendition]{att.global.rendition} (\textit{@rend}, \textit{@style}, \textit{@rendition})) (\hyperref[TEI.att.global.linking]{att.global.linking} (\textit{@corresp}, \textit{@synch}, \textit{@sameAs}, \textit{@copyOf}, \textit{@next}, \textit{@prev}, \textit{@exclude}, \textit{@select})) (\hyperref[TEI.att.global.analytic]{att.global.analytic} (\textit{@ana})) (\hyperref[TEI.att.global.facs]{att.global.facs} (\textit{@facs})) (\hyperref[TEI.att.global.change]{att.global.change} (\textit{@change})) (\hyperref[TEI.att.global.responsibility]{att.global.responsibility} (\textit{@cert}, \textit{@resp})) (\hyperref[TEI.att.global.source]{att.global.source} (\textit{@source})) \hyperref[TEI.att.sortable]{att.sortable} (\textit{@sortKey}) \hfil\\[-10pt]\begin{sansreflist}
    \item[@role]
  précise le rôle joué par ce groupe de personnes dans l'interaction.
\begin{reflist}
    \item[{Statut}]
  Optionel
    \item[{Type de données}]
  \hyperref[TEI.teidata.enumerated]{teidata.enumerated}
\end{reflist}  
    \item[@sex]
  précise le sexe du groupe participant.
\begin{reflist}
    \item[{Statut}]
  Optionel
    \item[{Type de données}]
  1–∞ occurrences de \hyperref[TEI.teidata.sex]{teidata.sex} séparé par un espace
\end{reflist}  
    \item[@age]
  précise la tranche d'âge des participants.
\begin{reflist}
    \item[{Statut}]
  Optionel
    \item[{Type de données}]
  \hyperref[TEI.teidata.enumerated]{teidata.enumerated}
\end{reflist}  
    \item[@size]
  précise la taille exacte ou approximative du groupe.
\begin{reflist}
    \item[{Statut}]
  Optionel
    \item[{Type de données}]
  1–∞ occurrences de \hyperref[TEI.teidata.word]{teidata.word} séparé par un espace
\end{reflist}  
\end{sansreflist}  
    \item[{Membre du}]
  \hyperref[TEI.model.personLike]{model.personLike}
    \item[{Contenu dans}]
  
    \item[namesdates: ]
   \hyperref[TEI.org]{org}
    \item[{Peut contenir}]
  
    \item[analysis: ]
   \hyperref[TEI.interp]{interp} \hyperref[TEI.interpGrp]{interpGrp} \hyperref[TEI.span]{span} \hyperref[TEI.spanGrp]{spanGrp}\par 
    \item[core: ]
   \hyperref[TEI.bibl]{bibl} \hyperref[TEI.biblStruct]{biblStruct} \hyperref[TEI.cb]{cb} \hyperref[TEI.gap]{gap} \hyperref[TEI.gb]{gb} \hyperref[TEI.index]{index} \hyperref[TEI.lb]{lb} \hyperref[TEI.listBibl]{listBibl} \hyperref[TEI.milestone]{milestone} \hyperref[TEI.name]{name} \hyperref[TEI.note]{note} \hyperref[TEI.p]{p} \hyperref[TEI.pb]{pb}\par 
    \item[figures: ]
   \hyperref[TEI.figure]{figure} \hyperref[TEI.notatedMusic]{notatedMusic}\par 
    \item[header: ]
   \hyperref[TEI.biblFull]{biblFull} \hyperref[TEI.idno]{idno}\par 
    \item[iso-fs: ]
   \hyperref[TEI.fLib]{fLib} \hyperref[TEI.fs]{fs} \hyperref[TEI.fvLib]{fvLib}\par 
    \item[linking: ]
   \hyperref[TEI.ab]{ab} \hyperref[TEI.alt]{alt} \hyperref[TEI.altGrp]{altGrp} \hyperref[TEI.anchor]{anchor} \hyperref[TEI.join]{join} \hyperref[TEI.joinGrp]{joinGrp} \hyperref[TEI.link]{link} \hyperref[TEI.linkGrp]{linkGrp} \hyperref[TEI.timeline]{timeline}\par 
    \item[msdescription: ]
   \hyperref[TEI.msDesc]{msDesc} \hyperref[TEI.source]{source}\par 
    \item[namesdates: ]
   \hyperref[TEI.affiliation]{affiliation} \hyperref[TEI.event]{event} \hyperref[TEI.persName]{persName} \hyperref[TEI.persona]{persona} \hyperref[TEI.state]{state}\par 
    \item[transcr: ]
   \hyperref[TEI.addSpan]{addSpan} \hyperref[TEI.damageSpan]{damageSpan} \hyperref[TEI.delSpan]{delSpan} \hyperref[TEI.fw]{fw} \hyperref[TEI.listTranspose]{listTranspose} \hyperref[TEI.metamark]{metamark} \hyperref[TEI.space]{space} \hyperref[TEI.substJoin]{substJoin}
    \item[{Note}]
  \par
Peut contenir une description en texte libre organisée en paragraphes, ou une suite quelconque d'éléments relatifs à la démographie.\par
Il faut utiliser l'attribut global {\itshape xml:id} pour identifier chaque locuteur dans une transcription de paroles si l'attribut {\itshape who} est présent pour chaque prise de parole.
    \item[{Exemple}]
  \leavevmode\bgroup\exampleFont \begin{shaded}\noindent\mbox{}{<\textbf{personGrp}\hspace*{6pt}{age}="{teen}"\hspace*{6pt}{role}="{audience}"\mbox{}\newline 
\hspace*{6pt}{sex}="{mixed}"\hspace*{6pt}{size}="{approx 50}"\hspace*{6pt}{xml:id}="{fr\textunderscore pg1}"/>}\end{shaded}\egroup 


    \item[{Modèle de contenu}]
  \mbox{}\hfill\\[-10pt]\begin{Verbatim}[fontsize=\small]
<content>
 <alternate maxOccurs="1" minOccurs="1">
  <classRef key="model.pLike"
   maxOccurs="unbounded" minOccurs="1"/>
  <alternate maxOccurs="unbounded"
   minOccurs="0">
   <classRef key="model.personPart"/>
   <classRef key="model.global"/>
  </alternate>
 </alternate>
</content>
    
\end{Verbatim}

    \item[{Schéma Declaration}]
  \mbox{}\hfill\\[-10pt]\begin{Verbatim}[fontsize=\small]
element personGrp
{
   tei_att.global.attributes,
   tei_att.sortable.attributes,
   attribute role { text }?,
   attribute sex { list { + } }?,
   attribute age { text }?,
   attribute size { list { + } }?,
   ( tei_model.pLike+ | ( tei_model.personPart | tei_model.global )* )
}
\end{Verbatim}

\end{reflist}  \index{persona=<persona>|oddindex}\index{role=@role!<persona>|oddindex}\index{sex=@sex!<persona>|oddindex}\index{age=@age!<persona>|oddindex}
\begin{reflist}
\item[]\begin{specHead}{TEI.persona}{<persona> }provides information about one of the personalities identified for a given individual, where an individual has multiple personalities. [\xref{http://www.tei-c.org/release/doc/tei-p5-doc/en/html/ND.html\#NDPERSE}{13.3.2. The Person Element}]\end{specHead} 
    \item[{Module}]
  namesdates
    \item[{Attributs}]
  Attributs \hyperref[TEI.att.global]{att.global} (\textit{@xml:id}, \textit{@n}, \textit{@xml:lang}, \textit{@xml:base}, \textit{@xml:space})  (\hyperref[TEI.att.global.rendition]{att.global.rendition} (\textit{@rend}, \textit{@style}, \textit{@rendition})) (\hyperref[TEI.att.global.linking]{att.global.linking} (\textit{@corresp}, \textit{@synch}, \textit{@sameAs}, \textit{@copyOf}, \textit{@next}, \textit{@prev}, \textit{@exclude}, \textit{@select})) (\hyperref[TEI.att.global.analytic]{att.global.analytic} (\textit{@ana})) (\hyperref[TEI.att.global.facs]{att.global.facs} (\textit{@facs})) (\hyperref[TEI.att.global.change]{att.global.change} (\textit{@change})) (\hyperref[TEI.att.global.responsibility]{att.global.responsibility} (\textit{@cert}, \textit{@resp})) (\hyperref[TEI.att.global.source]{att.global.source} (\textit{@source})) \hyperref[TEI.att.editLike]{att.editLike} (\textit{@evidence}, \textit{@instant})  (\hyperref[TEI.att.dimensions]{att.dimensions} (\textit{@unit}, \textit{@quantity}, \textit{@extent}, \textit{@precision}, \textit{@scope}) (\hyperref[TEI.att.ranging]{att.ranging} (\textit{@atLeast}, \textit{@atMost}, \textit{@min}, \textit{@max}, \textit{@confidence})) ) \hyperref[TEI.att.sortable]{att.sortable} (\textit{@sortKey}) \hfil\\[-10pt]\begin{sansreflist}
    \item[@role]
  précise un rôle principal ou une classification principale pour cette personne.
\begin{reflist}
    \item[{Statut}]
  Optionel
    \item[{Type de données}]
  1–∞ occurrences de \hyperref[TEI.teidata.enumerated]{teidata.enumerated} séparé par un espace
\end{reflist}  
    \item[@sex]
  précise le sexe de la personne.
\begin{reflist}
    \item[{Statut}]
  Optionel
    \item[{Type de données}]
  1–∞ occurrences de \hyperref[TEI.teidata.sex]{teidata.sex} séparé par un espace
\end{reflist}  
    \item[@age]
  précise une tranche d'âge pour la personne.
\begin{reflist}
    \item[{Statut}]
  Optionel
    \item[{Type de données}]
  \hyperref[TEI.teidata.enumerated]{teidata.enumerated}
\end{reflist}  
\end{sansreflist}  
    \item[{Membre du}]
  \hyperref[TEI.model.persStateLike]{model.persStateLike}
    \item[{Contenu dans}]
  
    \item[namesdates: ]
   \hyperref[TEI.person]{person} \hyperref[TEI.personGrp]{personGrp} \hyperref[TEI.persona]{persona}
    \item[{Peut contenir}]
  
    \item[analysis: ]
   \hyperref[TEI.interp]{interp} \hyperref[TEI.interpGrp]{interpGrp} \hyperref[TEI.span]{span} \hyperref[TEI.spanGrp]{spanGrp}\par 
    \item[core: ]
   \hyperref[TEI.bibl]{bibl} \hyperref[TEI.biblStruct]{biblStruct} \hyperref[TEI.cb]{cb} \hyperref[TEI.gap]{gap} \hyperref[TEI.gb]{gb} \hyperref[TEI.index]{index} \hyperref[TEI.lb]{lb} \hyperref[TEI.listBibl]{listBibl} \hyperref[TEI.milestone]{milestone} \hyperref[TEI.name]{name} \hyperref[TEI.note]{note} \hyperref[TEI.p]{p} \hyperref[TEI.pb]{pb}\par 
    \item[figures: ]
   \hyperref[TEI.figure]{figure} \hyperref[TEI.notatedMusic]{notatedMusic}\par 
    \item[header: ]
   \hyperref[TEI.biblFull]{biblFull} \hyperref[TEI.idno]{idno}\par 
    \item[iso-fs: ]
   \hyperref[TEI.fLib]{fLib} \hyperref[TEI.fs]{fs} \hyperref[TEI.fvLib]{fvLib}\par 
    \item[linking: ]
   \hyperref[TEI.ab]{ab} \hyperref[TEI.alt]{alt} \hyperref[TEI.altGrp]{altGrp} \hyperref[TEI.anchor]{anchor} \hyperref[TEI.join]{join} \hyperref[TEI.joinGrp]{joinGrp} \hyperref[TEI.link]{link} \hyperref[TEI.linkGrp]{linkGrp} \hyperref[TEI.timeline]{timeline}\par 
    \item[msdescription: ]
   \hyperref[TEI.msDesc]{msDesc} \hyperref[TEI.source]{source}\par 
    \item[namesdates: ]
   \hyperref[TEI.affiliation]{affiliation} \hyperref[TEI.event]{event} \hyperref[TEI.persName]{persName} \hyperref[TEI.persona]{persona} \hyperref[TEI.state]{state}\par 
    \item[transcr: ]
   \hyperref[TEI.addSpan]{addSpan} \hyperref[TEI.damageSpan]{damageSpan} \hyperref[TEI.delSpan]{delSpan} \hyperref[TEI.fw]{fw} \hyperref[TEI.listTranspose]{listTranspose} \hyperref[TEI.metamark]{metamark} \hyperref[TEI.space]{space} \hyperref[TEI.substJoin]{substJoin}
    \item[{Note}]
  \par
Note that a persona is not the same as a role. A role may be assumed by different people on different occasions, whereas a persona is unique to a particular person, even though it may resemble others. Similarly, when an actor takes on or enacts the role of a historical person, they do not thereby acquire a new persona. 
    \item[{Exemple}]
  \leavevmode\bgroup\exampleFont \begin{shaded}\noindent\mbox{}{<\textbf{person}\hspace*{6pt}{age}="{adult}"\hspace*{6pt}{sex}="{M}">}\mbox{}\newline 
\hspace*{6pt}{<\textbf{persona}\hspace*{6pt}{sex}="{M}">}\mbox{}\newline 
\hspace*{6pt}\hspace*{6pt}{<\textbf{persName}>}Dr Henry Jekyll{</\textbf{persName}>}\mbox{}\newline 
\hspace*{6pt}{</\textbf{persona}>}\mbox{}\newline 
\hspace*{6pt}{<\textbf{persona}\hspace*{6pt}{age}="{youth}"\hspace*{6pt}{sex}="{M}">}\mbox{}\newline 
\hspace*{6pt}\hspace*{6pt}{<\textbf{persName}>}Edward Hyde{</\textbf{persName}>}\mbox{}\newline 
\hspace*{6pt}{</\textbf{persona}>}\mbox{}\newline 
{</\textbf{person}>}\end{shaded}\egroup 


    \item[{Modèle de contenu}]
  \mbox{}\hfill\\[-10pt]\begin{Verbatim}[fontsize=\small]
<content>
 <alternate maxOccurs="1" minOccurs="1">
  <classRef key="model.pLike"
   maxOccurs="unbounded" minOccurs="1"/>
  <alternate maxOccurs="unbounded"
   minOccurs="0">
   <classRef key="model.personPart"/>
   <classRef key="model.global"/>
  </alternate>
 </alternate>
</content>
    
\end{Verbatim}

    \item[{Schéma Declaration}]
  \mbox{}\hfill\\[-10pt]\begin{Verbatim}[fontsize=\small]
element persona
{
   tei_att.global.attributes,
   tei_att.editLike.attributes,
   tei_att.sortable.attributes,
   attribute role { list { + } }?,
   attribute sex { list { + } }?,
   attribute age { text }?,
   ( tei_model.pLike+ | ( tei_model.personPart | tei_model.global )* )
}
\end{Verbatim}

\end{reflist}  \index{phr=<phr>|oddindex}
\begin{reflist}
\item[]\begin{specHead}{TEI.phr}{<phr> }(syntagme) représente un syntagme grammatical [\xref{http://www.tei-c.org/release/doc/tei-p5-doc/en/html/AI.html\#AILC}{17.1. Linguistic Segment Categories}]\end{specHead} 
    \item[{Module}]
  analysis
    \item[{Attributs}]
  Attributs \hyperref[TEI.att.global]{att.global} (\textit{@xml:id}, \textit{@n}, \textit{@xml:lang}, \textit{@xml:base}, \textit{@xml:space})  (\hyperref[TEI.att.global.rendition]{att.global.rendition} (\textit{@rend}, \textit{@style}, \textit{@rendition})) (\hyperref[TEI.att.global.linking]{att.global.linking} (\textit{@corresp}, \textit{@synch}, \textit{@sameAs}, \textit{@copyOf}, \textit{@next}, \textit{@prev}, \textit{@exclude}, \textit{@select})) (\hyperref[TEI.att.global.analytic]{att.global.analytic} (\textit{@ana})) (\hyperref[TEI.att.global.facs]{att.global.facs} (\textit{@facs})) (\hyperref[TEI.att.global.change]{att.global.change} (\textit{@change})) (\hyperref[TEI.att.global.responsibility]{att.global.responsibility} (\textit{@cert}, \textit{@resp})) (\hyperref[TEI.att.global.source]{att.global.source} (\textit{@source})) \hyperref[TEI.att.segLike]{att.segLike} (\textit{@function})  (\hyperref[TEI.att.datcat]{att.datcat} (\textit{@datcat}, \textit{@valueDatcat})) (\hyperref[TEI.att.fragmentable]{att.fragmentable} (\textit{@part})) \hyperref[TEI.att.typed]{att.typed} (\textit{@type}, \textit{@subtype}) 
    \item[{Membre du}]
  \hyperref[TEI.model.segLike]{model.segLike}
    \item[{Contenu dans}]
  
    \item[analysis: ]
   \hyperref[TEI.cl]{cl} \hyperref[TEI.phr]{phr} \hyperref[TEI.s]{s}\par 
    \item[core: ]
   \hyperref[TEI.abbr]{abbr} \hyperref[TEI.add]{add} \hyperref[TEI.addrLine]{addrLine} \hyperref[TEI.author]{author} \hyperref[TEI.bibl]{bibl} \hyperref[TEI.biblScope]{biblScope} \hyperref[TEI.citedRange]{citedRange} \hyperref[TEI.corr]{corr} \hyperref[TEI.date]{date} \hyperref[TEI.del]{del} \hyperref[TEI.distinct]{distinct} \hyperref[TEI.editor]{editor} \hyperref[TEI.email]{email} \hyperref[TEI.emph]{emph} \hyperref[TEI.expan]{expan} \hyperref[TEI.foreign]{foreign} \hyperref[TEI.gloss]{gloss} \hyperref[TEI.head]{head} \hyperref[TEI.headItem]{headItem} \hyperref[TEI.headLabel]{headLabel} \hyperref[TEI.hi]{hi} \hyperref[TEI.item]{item} \hyperref[TEI.l]{l} \hyperref[TEI.label]{label} \hyperref[TEI.measure]{measure} \hyperref[TEI.mentioned]{mentioned} \hyperref[TEI.name]{name} \hyperref[TEI.note]{note} \hyperref[TEI.num]{num} \hyperref[TEI.orig]{orig} \hyperref[TEI.p]{p} \hyperref[TEI.pubPlace]{pubPlace} \hyperref[TEI.publisher]{publisher} \hyperref[TEI.q]{q} \hyperref[TEI.quote]{quote} \hyperref[TEI.ref]{ref} \hyperref[TEI.reg]{reg} \hyperref[TEI.rs]{rs} \hyperref[TEI.said]{said} \hyperref[TEI.sic]{sic} \hyperref[TEI.soCalled]{soCalled} \hyperref[TEI.speaker]{speaker} \hyperref[TEI.stage]{stage} \hyperref[TEI.street]{street} \hyperref[TEI.term]{term} \hyperref[TEI.textLang]{textLang} \hyperref[TEI.time]{time} \hyperref[TEI.title]{title} \hyperref[TEI.unclear]{unclear}\par 
    \item[figures: ]
   \hyperref[TEI.cell]{cell}\par 
    \item[header: ]
   \hyperref[TEI.change]{change} \hyperref[TEI.distributor]{distributor} \hyperref[TEI.edition]{edition} \hyperref[TEI.extent]{extent} \hyperref[TEI.licence]{licence}\par 
    \item[linking: ]
   \hyperref[TEI.ab]{ab} \hyperref[TEI.seg]{seg}\par 
    \item[msdescription: ]
   \hyperref[TEI.accMat]{accMat} \hyperref[TEI.acquisition]{acquisition} \hyperref[TEI.additions]{additions} \hyperref[TEI.catchwords]{catchwords} \hyperref[TEI.collation]{collation} \hyperref[TEI.colophon]{colophon} \hyperref[TEI.condition]{condition} \hyperref[TEI.custEvent]{custEvent} \hyperref[TEI.decoNote]{decoNote} \hyperref[TEI.explicit]{explicit} \hyperref[TEI.filiation]{filiation} \hyperref[TEI.finalRubric]{finalRubric} \hyperref[TEI.foliation]{foliation} \hyperref[TEI.heraldry]{heraldry} \hyperref[TEI.incipit]{incipit} \hyperref[TEI.layout]{layout} \hyperref[TEI.material]{material} \hyperref[TEI.musicNotation]{musicNotation} \hyperref[TEI.objectType]{objectType} \hyperref[TEI.origDate]{origDate} \hyperref[TEI.origPlace]{origPlace} \hyperref[TEI.origin]{origin} \hyperref[TEI.provenance]{provenance} \hyperref[TEI.rubric]{rubric} \hyperref[TEI.secFol]{secFol} \hyperref[TEI.signatures]{signatures} \hyperref[TEI.source]{source} \hyperref[TEI.stamp]{stamp} \hyperref[TEI.summary]{summary} \hyperref[TEI.support]{support} \hyperref[TEI.surrogates]{surrogates} \hyperref[TEI.typeNote]{typeNote} \hyperref[TEI.watermark]{watermark}\par 
    \item[namesdates: ]
   \hyperref[TEI.addName]{addName} \hyperref[TEI.affiliation]{affiliation} \hyperref[TEI.country]{country} \hyperref[TEI.forename]{forename} \hyperref[TEI.genName]{genName} \hyperref[TEI.geogName]{geogName} \hyperref[TEI.nameLink]{nameLink} \hyperref[TEI.orgName]{orgName} \hyperref[TEI.persName]{persName} \hyperref[TEI.placeName]{placeName} \hyperref[TEI.region]{region} \hyperref[TEI.roleName]{roleName} \hyperref[TEI.settlement]{settlement} \hyperref[TEI.surname]{surname}\par 
    \item[textstructure: ]
   \hyperref[TEI.docAuthor]{docAuthor} \hyperref[TEI.docDate]{docDate} \hyperref[TEI.docEdition]{docEdition} \hyperref[TEI.titlePart]{titlePart}\par 
    \item[transcr: ]
   \hyperref[TEI.damage]{damage} \hyperref[TEI.fw]{fw} \hyperref[TEI.metamark]{metamark} \hyperref[TEI.mod]{mod} \hyperref[TEI.restore]{restore} \hyperref[TEI.retrace]{retrace} \hyperref[TEI.secl]{secl} \hyperref[TEI.supplied]{supplied} \hyperref[TEI.surplus]{surplus}
    \item[{Peut contenir}]
  
    \item[analysis: ]
   \hyperref[TEI.c]{c} \hyperref[TEI.cl]{cl} \hyperref[TEI.interp]{interp} \hyperref[TEI.interpGrp]{interpGrp} \hyperref[TEI.m]{m} \hyperref[TEI.pc]{pc} \hyperref[TEI.phr]{phr} \hyperref[TEI.s]{s} \hyperref[TEI.span]{span} \hyperref[TEI.spanGrp]{spanGrp} \hyperref[TEI.w]{w}\par 
    \item[core: ]
   \hyperref[TEI.abbr]{abbr} \hyperref[TEI.add]{add} \hyperref[TEI.address]{address} \hyperref[TEI.binaryObject]{binaryObject} \hyperref[TEI.cb]{cb} \hyperref[TEI.choice]{choice} \hyperref[TEI.corr]{corr} \hyperref[TEI.date]{date} \hyperref[TEI.del]{del} \hyperref[TEI.distinct]{distinct} \hyperref[TEI.email]{email} \hyperref[TEI.emph]{emph} \hyperref[TEI.expan]{expan} \hyperref[TEI.foreign]{foreign} \hyperref[TEI.gap]{gap} \hyperref[TEI.gb]{gb} \hyperref[TEI.gloss]{gloss} \hyperref[TEI.graphic]{graphic} \hyperref[TEI.hi]{hi} \hyperref[TEI.index]{index} \hyperref[TEI.lb]{lb} \hyperref[TEI.measure]{measure} \hyperref[TEI.measureGrp]{measureGrp} \hyperref[TEI.media]{media} \hyperref[TEI.mentioned]{mentioned} \hyperref[TEI.milestone]{milestone} \hyperref[TEI.name]{name} \hyperref[TEI.note]{note} \hyperref[TEI.num]{num} \hyperref[TEI.orig]{orig} \hyperref[TEI.pb]{pb} \hyperref[TEI.ptr]{ptr} \hyperref[TEI.ref]{ref} \hyperref[TEI.reg]{reg} \hyperref[TEI.rs]{rs} \hyperref[TEI.sic]{sic} \hyperref[TEI.soCalled]{soCalled} \hyperref[TEI.term]{term} \hyperref[TEI.time]{time} \hyperref[TEI.title]{title} \hyperref[TEI.unclear]{unclear}\par 
    \item[derived-module-tei.istex: ]
   \hyperref[TEI.math]{math} \hyperref[TEI.mrow]{mrow}\par 
    \item[figures: ]
   \hyperref[TEI.figure]{figure} \hyperref[TEI.formula]{formula} \hyperref[TEI.notatedMusic]{notatedMusic}\par 
    \item[header: ]
   \hyperref[TEI.idno]{idno}\par 
    \item[iso-fs: ]
   \hyperref[TEI.fLib]{fLib} \hyperref[TEI.fs]{fs} \hyperref[TEI.fvLib]{fvLib}\par 
    \item[linking: ]
   \hyperref[TEI.alt]{alt} \hyperref[TEI.altGrp]{altGrp} \hyperref[TEI.anchor]{anchor} \hyperref[TEI.join]{join} \hyperref[TEI.joinGrp]{joinGrp} \hyperref[TEI.link]{link} \hyperref[TEI.linkGrp]{linkGrp} \hyperref[TEI.seg]{seg} \hyperref[TEI.timeline]{timeline}\par 
    \item[msdescription: ]
   \hyperref[TEI.catchwords]{catchwords} \hyperref[TEI.depth]{depth} \hyperref[TEI.dim]{dim} \hyperref[TEI.dimensions]{dimensions} \hyperref[TEI.height]{height} \hyperref[TEI.heraldry]{heraldry} \hyperref[TEI.locus]{locus} \hyperref[TEI.locusGrp]{locusGrp} \hyperref[TEI.material]{material} \hyperref[TEI.objectType]{objectType} \hyperref[TEI.origDate]{origDate} \hyperref[TEI.origPlace]{origPlace} \hyperref[TEI.secFol]{secFol} \hyperref[TEI.signatures]{signatures} \hyperref[TEI.source]{source} \hyperref[TEI.stamp]{stamp} \hyperref[TEI.watermark]{watermark} \hyperref[TEI.width]{width}\par 
    \item[namesdates: ]
   \hyperref[TEI.addName]{addName} \hyperref[TEI.affiliation]{affiliation} \hyperref[TEI.country]{country} \hyperref[TEI.forename]{forename} \hyperref[TEI.genName]{genName} \hyperref[TEI.geogName]{geogName} \hyperref[TEI.location]{location} \hyperref[TEI.nameLink]{nameLink} \hyperref[TEI.orgName]{orgName} \hyperref[TEI.persName]{persName} \hyperref[TEI.placeName]{placeName} \hyperref[TEI.region]{region} \hyperref[TEI.roleName]{roleName} \hyperref[TEI.settlement]{settlement} \hyperref[TEI.state]{state} \hyperref[TEI.surname]{surname}\par 
    \item[spoken: ]
   \hyperref[TEI.annotationBlock]{annotationBlock}\par 
    \item[transcr: ]
   \hyperref[TEI.addSpan]{addSpan} \hyperref[TEI.am]{am} \hyperref[TEI.damage]{damage} \hyperref[TEI.damageSpan]{damageSpan} \hyperref[TEI.delSpan]{delSpan} \hyperref[TEI.ex]{ex} \hyperref[TEI.fw]{fw} \hyperref[TEI.handShift]{handShift} \hyperref[TEI.listTranspose]{listTranspose} \hyperref[TEI.metamark]{metamark} \hyperref[TEI.mod]{mod} \hyperref[TEI.redo]{redo} \hyperref[TEI.restore]{restore} \hyperref[TEI.retrace]{retrace} \hyperref[TEI.secl]{secl} \hyperref[TEI.space]{space} \hyperref[TEI.subst]{subst} \hyperref[TEI.substJoin]{substJoin} \hyperref[TEI.supplied]{supplied} \hyperref[TEI.surplus]{surplus} \hyperref[TEI.undo]{undo}\par des données textuelles
    \item[{Note}]
  \par
L'attribut {\itshape type} peut être utilisé pour indiquer le type de syntagme grammatical, avec des valeurs telles que nom, verbe, préposition, etc. selon le cas.
    \item[{Exemple}]
  \leavevmode\bgroup\exampleFont \begin{shaded}\noindent\mbox{}{<\textbf{phr}\hspace*{6pt}{function}="{extraposted\textunderscore modifier}"\mbox{}\newline 
\hspace*{6pt}{type}="{verb}">}To talk\mbox{}\newline 
{<\textbf{phr}\hspace*{6pt}{function}="{complement}"\mbox{}\newline 
\hspace*{6pt}\hspace*{6pt}{type}="{preposition}">}of\mbox{}\newline 
\hspace*{6pt}{<\textbf{phr}\hspace*{6pt}{function}="{object}"\hspace*{6pt}{type}="{noun}">}many things{</\textbf{phr}>}\mbox{}\newline 
\hspace*{6pt}{</\textbf{phr}>}\mbox{}\newline 
{</\textbf{phr}>}\end{shaded}\egroup 


    \item[{Modèle de contenu}]
  \mbox{}\hfill\\[-10pt]\begin{Verbatim}[fontsize=\small]
<content>
 <macroRef key="macro.phraseSeq"/>
</content>
    
\end{Verbatim}

    \item[{Schéma Declaration}]
  \mbox{}\hfill\\[-10pt]\begin{Verbatim}[fontsize=\small]
element phr
{
   tei_att.global.attributes,
   tei_att.segLike.attributes,
   tei_att.typed.attributes,
   tei_macro.phraseSeq}
\end{Verbatim}

\end{reflist}  \index{physDesc=<physDesc>|oddindex}
\begin{reflist}
\item[]\begin{specHead}{TEI.physDesc}{<physDesc> }(description physique) contient la description physique complète d'un manuscrit ou d'une partie d'un manuscrit, éventuellement structurée en utilisant les éléments plus spécialisés appartenant à la classe \textsf{model.physDescPart}. [\xref{http://www.tei-c.org/release/doc/tei-p5-doc/en/html/MS.html\#msph}{10.7. Physical Description}]\end{specHead} 
    \item[{Module}]
  msdescription
    \item[{Attributs}]
  Attributs \hyperref[TEI.att.global]{att.global} (\textit{@xml:id}, \textit{@n}, \textit{@xml:lang}, \textit{@xml:base}, \textit{@xml:space})  (\hyperref[TEI.att.global.rendition]{att.global.rendition} (\textit{@rend}, \textit{@style}, \textit{@rendition})) (\hyperref[TEI.att.global.linking]{att.global.linking} (\textit{@corresp}, \textit{@synch}, \textit{@sameAs}, \textit{@copyOf}, \textit{@next}, \textit{@prev}, \textit{@exclude}, \textit{@select})) (\hyperref[TEI.att.global.analytic]{att.global.analytic} (\textit{@ana})) (\hyperref[TEI.att.global.facs]{att.global.facs} (\textit{@facs})) (\hyperref[TEI.att.global.change]{att.global.change} (\textit{@change})) (\hyperref[TEI.att.global.responsibility]{att.global.responsibility} (\textit{@cert}, \textit{@resp})) (\hyperref[TEI.att.global.source]{att.global.source} (\textit{@source}))
    \item[{Contenu dans}]
  
    \item[msdescription: ]
   \hyperref[TEI.msDesc]{msDesc} \hyperref[TEI.msFrag]{msFrag} \hyperref[TEI.msPart]{msPart}
    \item[{Peut contenir}]
  
    \item[core: ]
   \hyperref[TEI.p]{p}\par 
    \item[linking: ]
   \hyperref[TEI.ab]{ab}\par 
    \item[msdescription: ]
   \hyperref[TEI.accMat]{accMat} \hyperref[TEI.additions]{additions} \hyperref[TEI.bindingDesc]{bindingDesc} \hyperref[TEI.decoDesc]{decoDesc} \hyperref[TEI.handDesc]{handDesc} \hyperref[TEI.musicNotation]{musicNotation} \hyperref[TEI.objectDesc]{objectDesc} \hyperref[TEI.scriptDesc]{scriptDesc} \hyperref[TEI.sealDesc]{sealDesc} \hyperref[TEI.typeDesc]{typeDesc}
    \item[{Exemple}]
  \leavevmode\bgroup\exampleFont \begin{shaded}\noindent\mbox{}{<\textbf{physDesc}>}\mbox{}\newline 
\hspace*{6pt}{<\textbf{objectDesc}\hspace*{6pt}{form}="{codex}">}\mbox{}\newline 
\hspace*{6pt}\hspace*{6pt}{<\textbf{supportDesc}\hspace*{6pt}{material}="{perg}">}\mbox{}\newline 
\hspace*{6pt}\hspace*{6pt}\hspace*{6pt}{<\textbf{support}>}Parchment.{</\textbf{support}>}\mbox{}\newline 
\hspace*{6pt}\hspace*{6pt}\hspace*{6pt}{<\textbf{extent}>}i + 55 leaves\mbox{}\newline 
\hspace*{6pt}\hspace*{6pt}\hspace*{6pt}{<\textbf{dimensions}\hspace*{6pt}{scope}="{all}"\hspace*{6pt}{type}="{leaf}"\mbox{}\newline 
\hspace*{6pt}\hspace*{6pt}\hspace*{6pt}\hspace*{6pt}\hspace*{6pt}{unit}="{inch}">}\mbox{}\newline 
\hspace*{6pt}\hspace*{6pt}\hspace*{6pt}\hspace*{6pt}\hspace*{6pt}{<\textbf{height}>}7¼{</\textbf{height}>}\mbox{}\newline 
\hspace*{6pt}\hspace*{6pt}\hspace*{6pt}\hspace*{6pt}\hspace*{6pt}{<\textbf{width}>}5⅜{</\textbf{width}>}\mbox{}\newline 
\hspace*{6pt}\hspace*{6pt}\hspace*{6pt}\hspace*{6pt}{</\textbf{dimensions}>}\mbox{}\newline 
\hspace*{6pt}\hspace*{6pt}\hspace*{6pt}{</\textbf{extent}>}\mbox{}\newline 
\hspace*{6pt}\hspace*{6pt}{</\textbf{supportDesc}>}\mbox{}\newline 
\hspace*{6pt}\hspace*{6pt}{<\textbf{layoutDesc}>}\mbox{}\newline 
\hspace*{6pt}\hspace*{6pt}\hspace*{6pt}{<\textbf{layout}\hspace*{6pt}{columns}="{2}">}In double columns.{</\textbf{layout}>}\mbox{}\newline 
\hspace*{6pt}\hspace*{6pt}{</\textbf{layoutDesc}>}\mbox{}\newline 
\hspace*{6pt}{</\textbf{objectDesc}>}\mbox{}\newline 
\hspace*{6pt}{<\textbf{handDesc}>}\mbox{}\newline 
\hspace*{6pt}\hspace*{6pt}{<\textbf{p}>}Written in more than one hand.{</\textbf{p}>}\mbox{}\newline 
\hspace*{6pt}{</\textbf{handDesc}>}\mbox{}\newline 
\hspace*{6pt}{<\textbf{decoDesc}>}\mbox{}\newline 
\hspace*{6pt}\hspace*{6pt}{<\textbf{p}>}With a few coloured capitals.{</\textbf{p}>}\mbox{}\newline 
\hspace*{6pt}{</\textbf{decoDesc}>}\mbox{}\newline 
{</\textbf{physDesc}>}\end{shaded}\egroup 


    \item[{Modèle de contenu}]
  \mbox{}\hfill\\[-10pt]\begin{Verbatim}[fontsize=\small]
<content>
 <sequence maxOccurs="1" minOccurs="1">
  <classRef key="model.pLike"
   maxOccurs="unbounded" minOccurs="0"/>
  <classRef expand="sequenceOptional"
   key="model.physDescPart"/>
 </sequence>
</content>
    
\end{Verbatim}

    \item[{Schéma Declaration}]
  \mbox{}\hfill\\[-10pt]\begin{Verbatim}[fontsize=\small]
element physDesc
{
   tei_att.global.attributes,
   (
      tei_model.pLike*,
      tei_objectDesc?,
      tei_handDesc?,
      tei_typeDesc?,
      tei_scriptDesc?,
      tei_musicNotation?,
      tei_decoDesc?,
      tei_additions?,
      tei_bindingDesc?,
      tei_sealDesc?,
      tei_accMat?
   )
}
\end{Verbatim}

\end{reflist}  \index{place=<place>|oddindex}
\begin{reflist}
\item[]\begin{specHead}{TEI.place}{<place> }(lieu) contient des informations sur un lieu géographique. [\xref{http://www.tei-c.org/release/doc/tei-p5-doc/en/html/ND.html\#NDGEOG}{13.3.4. Places}]\end{specHead} 
    \item[{Module}]
  namesdates
    \item[{Attributs}]
  Attributs \hyperref[TEI.att.global]{att.global} (\textit{@xml:id}, \textit{@n}, \textit{@xml:lang}, \textit{@xml:base}, \textit{@xml:space})  (\hyperref[TEI.att.global.rendition]{att.global.rendition} (\textit{@rend}, \textit{@style}, \textit{@rendition})) (\hyperref[TEI.att.global.linking]{att.global.linking} (\textit{@corresp}, \textit{@synch}, \textit{@sameAs}, \textit{@copyOf}, \textit{@next}, \textit{@prev}, \textit{@exclude}, \textit{@select})) (\hyperref[TEI.att.global.analytic]{att.global.analytic} (\textit{@ana})) (\hyperref[TEI.att.global.facs]{att.global.facs} (\textit{@facs})) (\hyperref[TEI.att.global.change]{att.global.change} (\textit{@change})) (\hyperref[TEI.att.global.responsibility]{att.global.responsibility} (\textit{@cert}, \textit{@resp})) (\hyperref[TEI.att.global.source]{att.global.source} (\textit{@source})) \hyperref[TEI.att.typed]{att.typed} (\textit{@type}, \textit{@subtype}) \hyperref[TEI.att.editLike]{att.editLike} (\textit{@evidence}, \textit{@instant})  (\hyperref[TEI.att.dimensions]{att.dimensions} (\textit{@unit}, \textit{@quantity}, \textit{@extent}, \textit{@precision}, \textit{@scope}) (\hyperref[TEI.att.ranging]{att.ranging} (\textit{@atLeast}, \textit{@atMost}, \textit{@min}, \textit{@max}, \textit{@confidence})) ) \hyperref[TEI.att.sortable]{att.sortable} (\textit{@sortKey}) 
    \item[{Membre du}]
  \hyperref[TEI.model.OABody]{model.OABody} \hyperref[TEI.model.placeLike]{model.placeLike}
    \item[{Contenu dans}]
  
    \item[namesdates: ]
   \hyperref[TEI.listPlace]{listPlace} \hyperref[TEI.org]{org} \hyperref[TEI.place]{place}\par 
    \item[spoken: ]
   \hyperref[TEI.annotationBlock]{annotationBlock}
    \item[{Peut contenir}]
  
    \item[core: ]
   \hyperref[TEI.bibl]{bibl} \hyperref[TEI.biblStruct]{biblStruct} \hyperref[TEI.desc]{desc} \hyperref[TEI.head]{head} \hyperref[TEI.label]{label} \hyperref[TEI.listBibl]{listBibl} \hyperref[TEI.note]{note} \hyperref[TEI.p]{p}\par 
    \item[header: ]
   \hyperref[TEI.biblFull]{biblFull} \hyperref[TEI.idno]{idno}\par 
    \item[linking: ]
   \hyperref[TEI.ab]{ab} \hyperref[TEI.link]{link} \hyperref[TEI.linkGrp]{linkGrp}\par 
    \item[msdescription: ]
   \hyperref[TEI.msDesc]{msDesc}\par 
    \item[namesdates: ]
   \hyperref[TEI.country]{country} \hyperref[TEI.event]{event} \hyperref[TEI.geogName]{geogName} \hyperref[TEI.listPlace]{listPlace} \hyperref[TEI.location]{location} \hyperref[TEI.place]{place} \hyperref[TEI.placeName]{placeName} \hyperref[TEI.region]{region} \hyperref[TEI.settlement]{settlement} \hyperref[TEI.state]{state}
    \item[{Exemple}]
  \leavevmode\bgroup\exampleFont \begin{shaded}\noindent\mbox{}{<\textbf{place}>}\mbox{}\newline 
\hspace*{6pt}{<\textbf{country}>}Lithuania{</\textbf{country}>}\mbox{}\newline 
\hspace*{6pt}{<\textbf{country}\hspace*{6pt}{xml:lang}="{lt}">}Lietuva{</\textbf{country}>}\mbox{}\newline 
\hspace*{6pt}{<\textbf{place}>}\mbox{}\newline 
\hspace*{6pt}\hspace*{6pt}{<\textbf{settlement}>}Vilnius{</\textbf{settlement}>}\mbox{}\newline 
\hspace*{6pt}{</\textbf{place}>}\mbox{}\newline 
\hspace*{6pt}{<\textbf{place}>}\mbox{}\newline 
\hspace*{6pt}\hspace*{6pt}{<\textbf{settlement}>}Kaunas{</\textbf{settlement}>}\mbox{}\newline 
\hspace*{6pt}{</\textbf{place}>}\mbox{}\newline 
{</\textbf{place}>}\end{shaded}\egroup 


    \item[{Modèle de contenu}]
  \mbox{}\hfill\\[-10pt]\begin{Verbatim}[fontsize=\small]
<content>
 <sequence maxOccurs="1" minOccurs="1">
  <classRef key="model.headLike"
   maxOccurs="unbounded" minOccurs="0"/>
  <alternate maxOccurs="1" minOccurs="1">
   <classRef key="model.pLike"
    maxOccurs="unbounded" minOccurs="0"/>
   <alternate maxOccurs="unbounded"
    minOccurs="0">
    <classRef key="model.labelLike"/>
    <classRef key="model.placeStateLike"/>
    <classRef key="model.eventLike"/>
   </alternate>
  </alternate>
  <alternate maxOccurs="unbounded"
   minOccurs="0">
   <classRef key="model.noteLike"/>
   <classRef key="model.biblLike"/>
   <elementRef key="idno"/>
   <elementRef key="linkGrp"/>
   <elementRef key="link"/>
  </alternate>
  <alternate maxOccurs="unbounded"
   minOccurs="0">
   <classRef key="model.placeLike"/>
   <elementRef key="listPlace"/>
  </alternate>
 </sequence>
</content>
    
\end{Verbatim}

    \item[{Schéma Declaration}]
  \mbox{}\hfill\\[-10pt]\begin{Verbatim}[fontsize=\small]
element place
{
   tei_att.global.attributes,
   tei_att.typed.attributes,
   tei_att.editLike.attributes,
   tei_att.sortable.attributes,
   (
      tei_model.headLike*,
      (
         tei_model.pLike*
       | (
            tei_model.labelLike          | tei_model.placeStateLike          | tei_model.eventLike         )*
      ),
      (
         tei_model.noteLike       | tei_model.biblLike       | tei_idno       | tei_linkGrp       | tei_link      )*,
      ( tei_model.placeLike | tei_listPlace )*
   )
}
\end{Verbatim}

\end{reflist}  \index{placeName=<placeName>|oddindex}\index{scheme=@scheme!<placeName>|oddindex}
\begin{reflist}
\item[]\begin{specHead}{TEI.placeName}{<placeName> }(nom de lieu) contient un nom de lieu absolu ou relatif. [\xref{http://www.tei-c.org/release/doc/tei-p5-doc/en/html/ND.html\#NDPLAC}{13.2.3. Place Names}]\end{specHead} 
    \item[{Module}]
  namesdates
    \item[{Attributs}]
  Attributs \hyperref[TEI.att.datable]{att.datable} (\textit{@calendar}, \textit{@period})  (\hyperref[TEI.att.datable.w3c]{att.datable.w3c} (\textit{@when}, \textit{@notBefore}, \textit{@notAfter}, \textit{@from}, \textit{@to})) (\hyperref[TEI.att.datable.iso]{att.datable.iso} (\textit{@when-iso}, \textit{@notBefore-iso}, \textit{@notAfter-iso}, \textit{@from-iso}, \textit{@to-iso})) (\hyperref[TEI.att.datable.custom]{att.datable.custom} (\textit{@when-custom}, \textit{@notBefore-custom}, \textit{@notAfter-custom}, \textit{@from-custom}, \textit{@to-custom}, \textit{@datingPoint}, \textit{@datingMethod})) \hyperref[TEI.att.editLike]{att.editLike} (\textit{@evidence}, \textit{@instant})  (\hyperref[TEI.att.dimensions]{att.dimensions} (\textit{@unit}, \textit{@quantity}, \textit{@extent}, \textit{@precision}, \textit{@scope}) (\hyperref[TEI.att.ranging]{att.ranging} (\textit{@atLeast}, \textit{@atMost}, \textit{@min}, \textit{@max}, \textit{@confidence})) ) \hyperref[TEI.att.global]{att.global} (\textit{@xml:id}, \textit{@n}, \textit{@xml:lang}, \textit{@xml:base}, \textit{@xml:space})  (\hyperref[TEI.att.global.rendition]{att.global.rendition} (\textit{@rend}, \textit{@style}, \textit{@rendition})) (\hyperref[TEI.att.global.linking]{att.global.linking} (\textit{@corresp}, \textit{@synch}, \textit{@sameAs}, \textit{@copyOf}, \textit{@next}, \textit{@prev}, \textit{@exclude}, \textit{@select})) (\hyperref[TEI.att.global.analytic]{att.global.analytic} (\textit{@ana})) (\hyperref[TEI.att.global.facs]{att.global.facs} (\textit{@facs})) (\hyperref[TEI.att.global.change]{att.global.change} (\textit{@change})) (\hyperref[TEI.att.global.responsibility]{att.global.responsibility} (\textit{@cert}, \textit{@resp})) (\hyperref[TEI.att.global.source]{att.global.source} (\textit{@source})) \hyperref[TEI.att.personal]{att.personal} (\textit{@full}, \textit{@sort})  (\hyperref[TEI.att.naming]{att.naming} (\textit{@role}, \textit{@nymRef}) (\hyperref[TEI.att.canonical]{att.canonical} (\textit{@key}, \textit{@ref})) ) \hyperref[TEI.att.typed]{att.typed} (\textit{@type}, \textit{@subtype}) \hfil\\[-10pt]\begin{sansreflist}
    \item[@scheme]
  désigne la liste des ontologies dans lequel l'ensemble des termes concernés sont définis.
\begin{reflist}
    \item[{Statut}]
  Optionel
    \item[{Type de données}]
  \hyperref[TEI.teidata.pointer]{teidata.pointer}
\end{reflist}  
\end{sansreflist}  
    \item[{Membre du}]
  \hyperref[TEI.model.placeNamePart]{model.placeNamePart}
    \item[{Contenu dans}]
  
    \item[analysis: ]
   \hyperref[TEI.cl]{cl} \hyperref[TEI.phr]{phr} \hyperref[TEI.s]{s} \hyperref[TEI.span]{span}\par 
    \item[core: ]
   \hyperref[TEI.abbr]{abbr} \hyperref[TEI.add]{add} \hyperref[TEI.addrLine]{addrLine} \hyperref[TEI.address]{address} \hyperref[TEI.author]{author} \hyperref[TEI.bibl]{bibl} \hyperref[TEI.biblScope]{biblScope} \hyperref[TEI.citedRange]{citedRange} \hyperref[TEI.corr]{corr} \hyperref[TEI.date]{date} \hyperref[TEI.del]{del} \hyperref[TEI.desc]{desc} \hyperref[TEI.distinct]{distinct} \hyperref[TEI.editor]{editor} \hyperref[TEI.email]{email} \hyperref[TEI.emph]{emph} \hyperref[TEI.expan]{expan} \hyperref[TEI.foreign]{foreign} \hyperref[TEI.gloss]{gloss} \hyperref[TEI.head]{head} \hyperref[TEI.headItem]{headItem} \hyperref[TEI.headLabel]{headLabel} \hyperref[TEI.hi]{hi} \hyperref[TEI.item]{item} \hyperref[TEI.l]{l} \hyperref[TEI.label]{label} \hyperref[TEI.measure]{measure} \hyperref[TEI.meeting]{meeting} \hyperref[TEI.mentioned]{mentioned} \hyperref[TEI.name]{name} \hyperref[TEI.note]{note} \hyperref[TEI.num]{num} \hyperref[TEI.orig]{orig} \hyperref[TEI.p]{p} \hyperref[TEI.pubPlace]{pubPlace} \hyperref[TEI.publisher]{publisher} \hyperref[TEI.q]{q} \hyperref[TEI.quote]{quote} \hyperref[TEI.ref]{ref} \hyperref[TEI.reg]{reg} \hyperref[TEI.resp]{resp} \hyperref[TEI.rs]{rs} \hyperref[TEI.said]{said} \hyperref[TEI.sic]{sic} \hyperref[TEI.soCalled]{soCalled} \hyperref[TEI.speaker]{speaker} \hyperref[TEI.stage]{stage} \hyperref[TEI.street]{street} \hyperref[TEI.term]{term} \hyperref[TEI.textLang]{textLang} \hyperref[TEI.time]{time} \hyperref[TEI.title]{title} \hyperref[TEI.unclear]{unclear}\par 
    \item[figures: ]
   \hyperref[TEI.cell]{cell} \hyperref[TEI.figDesc]{figDesc}\par 
    \item[header: ]
   \hyperref[TEI.authority]{authority} \hyperref[TEI.change]{change} \hyperref[TEI.classCode]{classCode} \hyperref[TEI.creation]{creation} \hyperref[TEI.distributor]{distributor} \hyperref[TEI.edition]{edition} \hyperref[TEI.extent]{extent} \hyperref[TEI.funder]{funder} \hyperref[TEI.language]{language} \hyperref[TEI.licence]{licence} \hyperref[TEI.rendition]{rendition}\par 
    \item[iso-fs: ]
   \hyperref[TEI.fDescr]{fDescr} \hyperref[TEI.fsDescr]{fsDescr}\par 
    \item[linking: ]
   \hyperref[TEI.ab]{ab} \hyperref[TEI.seg]{seg}\par 
    \item[msdescription: ]
   \hyperref[TEI.accMat]{accMat} \hyperref[TEI.acquisition]{acquisition} \hyperref[TEI.additions]{additions} \hyperref[TEI.altIdentifier]{altIdentifier} \hyperref[TEI.catchwords]{catchwords} \hyperref[TEI.collation]{collation} \hyperref[TEI.colophon]{colophon} \hyperref[TEI.condition]{condition} \hyperref[TEI.custEvent]{custEvent} \hyperref[TEI.decoNote]{decoNote} \hyperref[TEI.explicit]{explicit} \hyperref[TEI.filiation]{filiation} \hyperref[TEI.finalRubric]{finalRubric} \hyperref[TEI.foliation]{foliation} \hyperref[TEI.heraldry]{heraldry} \hyperref[TEI.incipit]{incipit} \hyperref[TEI.layout]{layout} \hyperref[TEI.material]{material} \hyperref[TEI.msIdentifier]{msIdentifier} \hyperref[TEI.musicNotation]{musicNotation} \hyperref[TEI.objectType]{objectType} \hyperref[TEI.origDate]{origDate} \hyperref[TEI.origPlace]{origPlace} \hyperref[TEI.origin]{origin} \hyperref[TEI.provenance]{provenance} \hyperref[TEI.rubric]{rubric} \hyperref[TEI.secFol]{secFol} \hyperref[TEI.signatures]{signatures} \hyperref[TEI.source]{source} \hyperref[TEI.stamp]{stamp} \hyperref[TEI.summary]{summary} \hyperref[TEI.support]{support} \hyperref[TEI.surrogates]{surrogates} \hyperref[TEI.typeNote]{typeNote} \hyperref[TEI.watermark]{watermark}\par 
    \item[namesdates: ]
   \hyperref[TEI.addName]{addName} \hyperref[TEI.affiliation]{affiliation} \hyperref[TEI.country]{country} \hyperref[TEI.forename]{forename} \hyperref[TEI.genName]{genName} \hyperref[TEI.geogName]{geogName} \hyperref[TEI.location]{location} \hyperref[TEI.nameLink]{nameLink} \hyperref[TEI.org]{org} \hyperref[TEI.orgName]{orgName} \hyperref[TEI.persName]{persName} \hyperref[TEI.place]{place} \hyperref[TEI.placeName]{placeName} \hyperref[TEI.region]{region} \hyperref[TEI.roleName]{roleName} \hyperref[TEI.settlement]{settlement} \hyperref[TEI.surname]{surname}\par 
    \item[spoken: ]
   \hyperref[TEI.annotationBlock]{annotationBlock}\par 
    \item[standOff: ]
   \hyperref[TEI.listAnnotation]{listAnnotation}\par 
    \item[textstructure: ]
   \hyperref[TEI.docAuthor]{docAuthor} \hyperref[TEI.docDate]{docDate} \hyperref[TEI.docEdition]{docEdition} \hyperref[TEI.titlePart]{titlePart}\par 
    \item[transcr: ]
   \hyperref[TEI.damage]{damage} \hyperref[TEI.fw]{fw} \hyperref[TEI.metamark]{metamark} \hyperref[TEI.mod]{mod} \hyperref[TEI.restore]{restore} \hyperref[TEI.retrace]{retrace} \hyperref[TEI.secl]{secl} \hyperref[TEI.supplied]{supplied} \hyperref[TEI.surplus]{surplus}
    \item[{Peut contenir}]
  
    \item[analysis: ]
   \hyperref[TEI.c]{c} \hyperref[TEI.cl]{cl} \hyperref[TEI.interp]{interp} \hyperref[TEI.interpGrp]{interpGrp} \hyperref[TEI.m]{m} \hyperref[TEI.pc]{pc} \hyperref[TEI.phr]{phr} \hyperref[TEI.s]{s} \hyperref[TEI.span]{span} \hyperref[TEI.spanGrp]{spanGrp} \hyperref[TEI.w]{w}\par 
    \item[core: ]
   \hyperref[TEI.abbr]{abbr} \hyperref[TEI.add]{add} \hyperref[TEI.address]{address} \hyperref[TEI.binaryObject]{binaryObject} \hyperref[TEI.cb]{cb} \hyperref[TEI.choice]{choice} \hyperref[TEI.corr]{corr} \hyperref[TEI.date]{date} \hyperref[TEI.del]{del} \hyperref[TEI.distinct]{distinct} \hyperref[TEI.email]{email} \hyperref[TEI.emph]{emph} \hyperref[TEI.expan]{expan} \hyperref[TEI.foreign]{foreign} \hyperref[TEI.gap]{gap} \hyperref[TEI.gb]{gb} \hyperref[TEI.gloss]{gloss} \hyperref[TEI.graphic]{graphic} \hyperref[TEI.hi]{hi} \hyperref[TEI.index]{index} \hyperref[TEI.lb]{lb} \hyperref[TEI.measure]{measure} \hyperref[TEI.measureGrp]{measureGrp} \hyperref[TEI.media]{media} \hyperref[TEI.mentioned]{mentioned} \hyperref[TEI.milestone]{milestone} \hyperref[TEI.name]{name} \hyperref[TEI.note]{note} \hyperref[TEI.num]{num} \hyperref[TEI.orig]{orig} \hyperref[TEI.pb]{pb} \hyperref[TEI.ptr]{ptr} \hyperref[TEI.ref]{ref} \hyperref[TEI.reg]{reg} \hyperref[TEI.rs]{rs} \hyperref[TEI.sic]{sic} \hyperref[TEI.soCalled]{soCalled} \hyperref[TEI.term]{term} \hyperref[TEI.time]{time} \hyperref[TEI.title]{title} \hyperref[TEI.unclear]{unclear}\par 
    \item[derived-module-tei.istex: ]
   \hyperref[TEI.math]{math} \hyperref[TEI.mrow]{mrow}\par 
    \item[figures: ]
   \hyperref[TEI.figure]{figure} \hyperref[TEI.formula]{formula} \hyperref[TEI.notatedMusic]{notatedMusic}\par 
    \item[header: ]
   \hyperref[TEI.idno]{idno}\par 
    \item[iso-fs: ]
   \hyperref[TEI.fLib]{fLib} \hyperref[TEI.fs]{fs} \hyperref[TEI.fvLib]{fvLib}\par 
    \item[linking: ]
   \hyperref[TEI.alt]{alt} \hyperref[TEI.altGrp]{altGrp} \hyperref[TEI.anchor]{anchor} \hyperref[TEI.join]{join} \hyperref[TEI.joinGrp]{joinGrp} \hyperref[TEI.link]{link} \hyperref[TEI.linkGrp]{linkGrp} \hyperref[TEI.seg]{seg} \hyperref[TEI.timeline]{timeline}\par 
    \item[msdescription: ]
   \hyperref[TEI.catchwords]{catchwords} \hyperref[TEI.depth]{depth} \hyperref[TEI.dim]{dim} \hyperref[TEI.dimensions]{dimensions} \hyperref[TEI.height]{height} \hyperref[TEI.heraldry]{heraldry} \hyperref[TEI.locus]{locus} \hyperref[TEI.locusGrp]{locusGrp} \hyperref[TEI.material]{material} \hyperref[TEI.objectType]{objectType} \hyperref[TEI.origDate]{origDate} \hyperref[TEI.origPlace]{origPlace} \hyperref[TEI.secFol]{secFol} \hyperref[TEI.signatures]{signatures} \hyperref[TEI.source]{source} \hyperref[TEI.stamp]{stamp} \hyperref[TEI.watermark]{watermark} \hyperref[TEI.width]{width}\par 
    \item[namesdates: ]
   \hyperref[TEI.addName]{addName} \hyperref[TEI.affiliation]{affiliation} \hyperref[TEI.country]{country} \hyperref[TEI.forename]{forename} \hyperref[TEI.genName]{genName} \hyperref[TEI.geogName]{geogName} \hyperref[TEI.location]{location} \hyperref[TEI.nameLink]{nameLink} \hyperref[TEI.orgName]{orgName} \hyperref[TEI.persName]{persName} \hyperref[TEI.placeName]{placeName} \hyperref[TEI.region]{region} \hyperref[TEI.roleName]{roleName} \hyperref[TEI.settlement]{settlement} \hyperref[TEI.state]{state} \hyperref[TEI.surname]{surname}\par 
    \item[spoken: ]
   \hyperref[TEI.annotationBlock]{annotationBlock}\par 
    \item[transcr: ]
   \hyperref[TEI.addSpan]{addSpan} \hyperref[TEI.am]{am} \hyperref[TEI.damage]{damage} \hyperref[TEI.damageSpan]{damageSpan} \hyperref[TEI.delSpan]{delSpan} \hyperref[TEI.ex]{ex} \hyperref[TEI.fw]{fw} \hyperref[TEI.handShift]{handShift} \hyperref[TEI.listTranspose]{listTranspose} \hyperref[TEI.metamark]{metamark} \hyperref[TEI.mod]{mod} \hyperref[TEI.redo]{redo} \hyperref[TEI.restore]{restore} \hyperref[TEI.retrace]{retrace} \hyperref[TEI.secl]{secl} \hyperref[TEI.space]{space} \hyperref[TEI.subst]{subst} \hyperref[TEI.substJoin]{substJoin} \hyperref[TEI.supplied]{supplied} \hyperref[TEI.surplus]{surplus} \hyperref[TEI.undo]{undo}\par des données textuelles
    \item[{Exemple}]
  StandOff enrichissement entité nommée placeName\leavevmode\bgroup\exampleFont \begin{shaded}\noindent\mbox{}{<\textbf{annotationBlock}\hspace*{6pt}{corresp}="{text}">}\mbox{}\newline 
\hspace*{6pt}{<\textbf{placeName}\hspace*{6pt}{change}="{\#Unitex-3.2.0-alpha}"\mbox{}\newline 
\hspace*{6pt}\hspace*{6pt}{resp}="{istex}"\mbox{}\newline 
\hspace*{6pt}\hspace*{6pt}{scheme}="{https://placename-entity.data.istex.fr}">}\mbox{}\newline 
\hspace*{6pt}\hspace*{6pt}{<\textbf{term}>}Geneva{</\textbf{term}>}\mbox{}\newline 
\hspace*{6pt}\hspace*{6pt}{<\textbf{fs}\hspace*{6pt}{type}="{statistics}">}\mbox{}\newline 
\hspace*{6pt}\hspace*{6pt}\hspace*{6pt}{<\textbf{f}\hspace*{6pt}{name}="{frequency}">}\mbox{}\newline 
\hspace*{6pt}\hspace*{6pt}\hspace*{6pt}\hspace*{6pt}{<\textbf{numeric}\hspace*{6pt}{value}="{9}"/>}\mbox{}\newline 
\hspace*{6pt}\hspace*{6pt}\hspace*{6pt}{</\textbf{f}>}\mbox{}\newline 
\hspace*{6pt}\hspace*{6pt}{</\textbf{fs}>}\mbox{}\newline 
\hspace*{6pt}{</\textbf{placeName}>}\mbox{}\newline 
{</\textbf{annotationBlock}>}\end{shaded}\egroup 


    \item[{Modèle de contenu}]
  \mbox{}\hfill\\[-10pt]\begin{Verbatim}[fontsize=\small]
<content>
 <macroRef key="macro.phraseSeq"/>
</content>
    
\end{Verbatim}

    \item[{Schéma Declaration}]
  \mbox{}\hfill\\[-10pt]\begin{Verbatim}[fontsize=\small]
element placeName
{
   tei_att.datable.attributes,
   tei_att.editLike.attributes,
   tei_att.global.attributes,
   tei_att.personal.attributes,
   tei_att.typed.attributes,
   attribute scheme { text }?,
   tei_macro.phraseSeq}
\end{Verbatim}

\end{reflist}  \index{postBox=<postBox>|oddindex}
\begin{reflist}
\item[]\begin{specHead}{TEI.postBox}{<postBox> }(boîte postale) contient un numéro ou un autre identifiant d'un lieu de distribution du courrier autre qu'un nom de rue. [\xref{http://www.tei-c.org/release/doc/tei-p5-doc/en/html/CO.html\#CONAAD}{3.5.2. Addresses}]\end{specHead} 
    \item[{Module}]
  core
    \item[{Attributs}]
  Attributs \hyperref[TEI.att.global]{att.global} (\textit{@xml:id}, \textit{@n}, \textit{@xml:lang}, \textit{@xml:base}, \textit{@xml:space})  (\hyperref[TEI.att.global.rendition]{att.global.rendition} (\textit{@rend}, \textit{@style}, \textit{@rendition})) (\hyperref[TEI.att.global.linking]{att.global.linking} (\textit{@corresp}, \textit{@synch}, \textit{@sameAs}, \textit{@copyOf}, \textit{@next}, \textit{@prev}, \textit{@exclude}, \textit{@select})) (\hyperref[TEI.att.global.analytic]{att.global.analytic} (\textit{@ana})) (\hyperref[TEI.att.global.facs]{att.global.facs} (\textit{@facs})) (\hyperref[TEI.att.global.change]{att.global.change} (\textit{@change})) (\hyperref[TEI.att.global.responsibility]{att.global.responsibility} (\textit{@cert}, \textit{@resp})) (\hyperref[TEI.att.global.source]{att.global.source} (\textit{@source}))
    \item[{Membre du}]
  \hyperref[TEI.model.addrPart]{model.addrPart}
    \item[{Contenu dans}]
  
    \item[core: ]
   \hyperref[TEI.address]{address}
    \item[{Peut contenir}]
  Des données textuelles uniquement
    \item[{Note}]
  \par
La disposition et la nature des codes postaux est spécifique à chaque pays ; on utilise les conventions qui leur sont propres .
    \item[{Exemple}]
  \leavevmode\bgroup\exampleFont \begin{shaded}\noindent\mbox{}{<\textbf{postBox}>}B.P. 4232 {</\textbf{postBox}>}\end{shaded}\egroup 


    \item[{Exemple}]
  \leavevmode\bgroup\exampleFont \begin{shaded}\noindent\mbox{}{<\textbf{postBox}>}BP 3317{</\textbf{postBox}>}\end{shaded}\egroup 


    \item[{Exemple}]
  \leavevmode\bgroup\exampleFont \begin{shaded}\noindent\mbox{}{<\textbf{postBox}>}Postbus 532{</\textbf{postBox}>}\end{shaded}\egroup 


    \item[{Modèle de contenu}]
  \fbox{\ttfamily <content>\newline
 <textNode/>\newline
</content>\newline
    } 
    \item[{Schéma Declaration}]
  \fbox{\ttfamily element postBox ❴ tei\textunderscore att.global.attributes, text ❵} 
\end{reflist}  \index{postCode=<postCode>|oddindex}
\begin{reflist}
\item[]\begin{specHead}{TEI.postCode}{<postCode> }(code postal) contient un code numérique ou alphanumérique qui fait partie de l'adresse postale et sert à simplifier le tri ou la distribution du courrier. [\xref{http://www.tei-c.org/release/doc/tei-p5-doc/en/html/CO.html\#CONAAD}{3.5.2. Addresses}]\end{specHead} 
    \item[{Module}]
  core
    \item[{Attributs}]
  Attributs \hyperref[TEI.att.global]{att.global} (\textit{@xml:id}, \textit{@n}, \textit{@xml:lang}, \textit{@xml:base}, \textit{@xml:space})  (\hyperref[TEI.att.global.rendition]{att.global.rendition} (\textit{@rend}, \textit{@style}, \textit{@rendition})) (\hyperref[TEI.att.global.linking]{att.global.linking} (\textit{@corresp}, \textit{@synch}, \textit{@sameAs}, \textit{@copyOf}, \textit{@next}, \textit{@prev}, \textit{@exclude}, \textit{@select})) (\hyperref[TEI.att.global.analytic]{att.global.analytic} (\textit{@ana})) (\hyperref[TEI.att.global.facs]{att.global.facs} (\textit{@facs})) (\hyperref[TEI.att.global.change]{att.global.change} (\textit{@change})) (\hyperref[TEI.att.global.responsibility]{att.global.responsibility} (\textit{@cert}, \textit{@resp})) (\hyperref[TEI.att.global.source]{att.global.source} (\textit{@source}))
    \item[{Membre du}]
  \hyperref[TEI.model.addrPart]{model.addrPart}
    \item[{Contenu dans}]
  
    \item[core: ]
   \hyperref[TEI.address]{address}
    \item[{Peut contenir}]
  Des données textuelles uniquement
    \item[{Note}]
  \par
La disposition et la nature des codes postaux est spécifique à chaque pays ; on utilise les conventions qui leur sont propres .
    \item[{Exemple}]
  \leavevmode\bgroup\exampleFont \begin{shaded}\noindent\mbox{}{<\textbf{postCode}>}84000{</\textbf{postCode}>}\end{shaded}\egroup 


    \item[{Exemple}]
  \leavevmode\bgroup\exampleFont \begin{shaded}\noindent\mbox{}{<\textbf{postCode}>}60142-7{</\textbf{postCode}>}\end{shaded}\egroup 


    \item[{Exemple}]
  \leavevmode\bgroup\exampleFont \begin{shaded}\noindent\mbox{}{<\textbf{postCode}>}60142-7{</\textbf{postCode}>}\end{shaded}\egroup 


    \item[{Modèle de contenu}]
  \fbox{\ttfamily <content>\newline
 <textNode/>\newline
</content>\newline
    } 
    \item[{Schéma Declaration}]
  \fbox{\ttfamily element postCode ❴ tei\textunderscore att.global.attributes, text ❵} 
\end{reflist}  \index{profileDesc=<profileDesc>|oddindex}
\begin{reflist}
\item[]\begin{specHead}{TEI.profileDesc}{<profileDesc> }(description du profil) fournit une description détaillée des aspects non bibliographiques du texte, notamment les langues utilisées et leurs variantes, les circonstances de sa production, les collaborateurs et leur statut. [\xref{http://www.tei-c.org/release/doc/tei-p5-doc/en/html/HD.html\#HD4}{2.4. The Profile Description} \xref{http://www.tei-c.org/release/doc/tei-p5-doc/en/html/HD.html\#HD11}{2.1.1. The TEI Header and Its Components}]\end{specHead} 
    \item[{Module}]
  header
    \item[{Attributs}]
  Attributs \hyperref[TEI.att.global]{att.global} (\textit{@xml:id}, \textit{@n}, \textit{@xml:lang}, \textit{@xml:base}, \textit{@xml:space})  (\hyperref[TEI.att.global.rendition]{att.global.rendition} (\textit{@rend}, \textit{@style}, \textit{@rendition})) (\hyperref[TEI.att.global.linking]{att.global.linking} (\textit{@corresp}, \textit{@synch}, \textit{@sameAs}, \textit{@copyOf}, \textit{@next}, \textit{@prev}, \textit{@exclude}, \textit{@select})) (\hyperref[TEI.att.global.analytic]{att.global.analytic} (\textit{@ana})) (\hyperref[TEI.att.global.facs]{att.global.facs} (\textit{@facs})) (\hyperref[TEI.att.global.change]{att.global.change} (\textit{@change})) (\hyperref[TEI.att.global.responsibility]{att.global.responsibility} (\textit{@cert}, \textit{@resp})) (\hyperref[TEI.att.global.source]{att.global.source} (\textit{@source}))
    \item[{Membre du}]
  \hyperref[TEI.model.teiHeaderPart]{model.teiHeaderPart}
    \item[{Contenu dans}]
  
    \item[header: ]
   \hyperref[TEI.biblFull]{biblFull} \hyperref[TEI.teiHeader]{teiHeader}
    \item[{Peut contenir}]
  
    \item[header: ]
   \hyperref[TEI.abstract]{abstract} \hyperref[TEI.creation]{creation} \hyperref[TEI.langUsage]{langUsage} \hyperref[TEI.textClass]{textClass}\par 
    \item[transcr: ]
   \hyperref[TEI.handNotes]{handNotes} \hyperref[TEI.listTranspose]{listTranspose}
    \item[{Note}]
  \par
Although the content model permits it, it is rarely meaningful to supply multiple occurrences for any of the child elements of \hyperref[TEI.profileDesc]{<profileDesc>} unless these are documenting multiple texts.
    \item[{Exemple}]
  \leavevmode\bgroup\exampleFont \begin{shaded}\noindent\mbox{}{<\textbf{profileDesc}>}\mbox{}\newline 
\hspace*{6pt}{<\textbf{langUsage}>}\mbox{}\newline 
\hspace*{6pt}\hspace*{6pt}{<\textbf{language}\hspace*{6pt}{ident}="{fr}">}français{</\textbf{language}>}\mbox{}\newline 
\hspace*{6pt}{</\textbf{langUsage}>}\mbox{}\newline 
\hspace*{6pt}{<\textbf{textDesc}\hspace*{6pt}{n}="{roman}">}\mbox{}\newline 
\hspace*{6pt}\hspace*{6pt}{<\textbf{channel}\hspace*{6pt}{mode}="{w}">}copie; extraits {</\textbf{channel}>}\mbox{}\newline 
\hspace*{6pt}\hspace*{6pt}{<\textbf{constitution}\hspace*{6pt}{type}="{single}"/>}\mbox{}\newline 
\hspace*{6pt}\hspace*{6pt}{<\textbf{derivation}\hspace*{6pt}{type}="{original}"/>}\mbox{}\newline 
\hspace*{6pt}\hspace*{6pt}{<\textbf{domain}\hspace*{6pt}{type}="{art}"/>}\mbox{}\newline 
\hspace*{6pt}\hspace*{6pt}{<\textbf{factuality}\hspace*{6pt}{type}="{fiction}"/>}\mbox{}\newline 
\hspace*{6pt}\hspace*{6pt}{<\textbf{interaction}\hspace*{6pt}{type}="{none}"/>}\mbox{}\newline 
\hspace*{6pt}\hspace*{6pt}{<\textbf{preparedness}\hspace*{6pt}{type}="{prepare}"/>}\mbox{}\newline 
\hspace*{6pt}\hspace*{6pt}{<\textbf{purpose}\hspace*{6pt}{degree}="{high}"\hspace*{6pt}{type}="{distraction}"/>}\mbox{}\newline 
\hspace*{6pt}\hspace*{6pt}{<\textbf{purpose}\hspace*{6pt}{degree}="{medium}"\mbox{}\newline 
\hspace*{6pt}\hspace*{6pt}\hspace*{6pt}{type}="{information}"/>}\mbox{}\newline 
\hspace*{6pt}{</\textbf{textDesc}>}\mbox{}\newline 
\hspace*{6pt}{<\textbf{settingDesc}>}\mbox{}\newline 
\hspace*{6pt}\hspace*{6pt}{<\textbf{setting}>}\mbox{}\newline 
\hspace*{6pt}\hspace*{6pt}\hspace*{6pt}{<\textbf{name}>}Paris, France{</\textbf{name}>}\mbox{}\newline 
\hspace*{6pt}\hspace*{6pt}\hspace*{6pt}{<\textbf{time}>}Fin 19e{</\textbf{time}>}\mbox{}\newline 
\hspace*{6pt}\hspace*{6pt}{</\textbf{setting}>}\mbox{}\newline 
\hspace*{6pt}{</\textbf{settingDesc}>}\mbox{}\newline 
{</\textbf{profileDesc}>}\end{shaded}\egroup 


    \item[{Modèle de contenu}]
  \mbox{}\hfill\\[-10pt]\begin{Verbatim}[fontsize=\small]
<content>
 <classRef key="model.profileDescPart"
  maxOccurs="unbounded" minOccurs="0"/>
</content>
    
\end{Verbatim}

    \item[{Schéma Declaration}]
  \mbox{}\hfill\\[-10pt]\begin{Verbatim}[fontsize=\small]
element profileDesc { tei_att.global.attributes, tei_model.profileDescPart* }
\end{Verbatim}

\end{reflist}  \index{provenance=<provenance>|oddindex}
\begin{reflist}
\item[]\begin{specHead}{TEI.provenance}{<provenance> }(provenance) contient des informations sur un épisode précis de l'histoire du manuscrit ou de la partie du manuscrit, après sa création et avant son acquisition [\xref{http://www.tei-c.org/release/doc/tei-p5-doc/en/html/MS.html\#mshy}{10.8. History}]\end{specHead} 
    \item[{Module}]
  msdescription
    \item[{Attributs}]
  Attributs \hyperref[TEI.att.global]{att.global} (\textit{@xml:id}, \textit{@n}, \textit{@xml:lang}, \textit{@xml:base}, \textit{@xml:space})  (\hyperref[TEI.att.global.rendition]{att.global.rendition} (\textit{@rend}, \textit{@style}, \textit{@rendition})) (\hyperref[TEI.att.global.linking]{att.global.linking} (\textit{@corresp}, \textit{@synch}, \textit{@sameAs}, \textit{@copyOf}, \textit{@next}, \textit{@prev}, \textit{@exclude}, \textit{@select})) (\hyperref[TEI.att.global.analytic]{att.global.analytic} (\textit{@ana})) (\hyperref[TEI.att.global.facs]{att.global.facs} (\textit{@facs})) (\hyperref[TEI.att.global.change]{att.global.change} (\textit{@change})) (\hyperref[TEI.att.global.responsibility]{att.global.responsibility} (\textit{@cert}, \textit{@resp})) (\hyperref[TEI.att.global.source]{att.global.source} (\textit{@source})) \hyperref[TEI.att.datable]{att.datable} (\textit{@calendar}, \textit{@period})  (\hyperref[TEI.att.datable.w3c]{att.datable.w3c} (\textit{@when}, \textit{@notBefore}, \textit{@notAfter}, \textit{@from}, \textit{@to})) (\hyperref[TEI.att.datable.iso]{att.datable.iso} (\textit{@when-iso}, \textit{@notBefore-iso}, \textit{@notAfter-iso}, \textit{@from-iso}, \textit{@to-iso})) (\hyperref[TEI.att.datable.custom]{att.datable.custom} (\textit{@when-custom}, \textit{@notBefore-custom}, \textit{@notAfter-custom}, \textit{@from-custom}, \textit{@to-custom}, \textit{@datingPoint}, \textit{@datingMethod})) \hyperref[TEI.att.typed]{att.typed} (\textit{@type}, \textit{@subtype}) 
    \item[{Contenu dans}]
  
    \item[msdescription: ]
   \hyperref[TEI.history]{history}
    \item[{Peut contenir}]
  
    \item[analysis: ]
   \hyperref[TEI.c]{c} \hyperref[TEI.cl]{cl} \hyperref[TEI.interp]{interp} \hyperref[TEI.interpGrp]{interpGrp} \hyperref[TEI.m]{m} \hyperref[TEI.pc]{pc} \hyperref[TEI.phr]{phr} \hyperref[TEI.s]{s} \hyperref[TEI.span]{span} \hyperref[TEI.spanGrp]{spanGrp} \hyperref[TEI.w]{w}\par 
    \item[core: ]
   \hyperref[TEI.abbr]{abbr} \hyperref[TEI.add]{add} \hyperref[TEI.address]{address} \hyperref[TEI.bibl]{bibl} \hyperref[TEI.biblStruct]{biblStruct} \hyperref[TEI.binaryObject]{binaryObject} \hyperref[TEI.cb]{cb} \hyperref[TEI.choice]{choice} \hyperref[TEI.cit]{cit} \hyperref[TEI.corr]{corr} \hyperref[TEI.date]{date} \hyperref[TEI.del]{del} \hyperref[TEI.desc]{desc} \hyperref[TEI.distinct]{distinct} \hyperref[TEI.email]{email} \hyperref[TEI.emph]{emph} \hyperref[TEI.expan]{expan} \hyperref[TEI.foreign]{foreign} \hyperref[TEI.gap]{gap} \hyperref[TEI.gb]{gb} \hyperref[TEI.gloss]{gloss} \hyperref[TEI.graphic]{graphic} \hyperref[TEI.hi]{hi} \hyperref[TEI.index]{index} \hyperref[TEI.l]{l} \hyperref[TEI.label]{label} \hyperref[TEI.lb]{lb} \hyperref[TEI.lg]{lg} \hyperref[TEI.list]{list} \hyperref[TEI.listBibl]{listBibl} \hyperref[TEI.measure]{measure} \hyperref[TEI.measureGrp]{measureGrp} \hyperref[TEI.media]{media} \hyperref[TEI.mentioned]{mentioned} \hyperref[TEI.milestone]{milestone} \hyperref[TEI.name]{name} \hyperref[TEI.note]{note} \hyperref[TEI.num]{num} \hyperref[TEI.orig]{orig} \hyperref[TEI.p]{p} \hyperref[TEI.pb]{pb} \hyperref[TEI.ptr]{ptr} \hyperref[TEI.q]{q} \hyperref[TEI.quote]{quote} \hyperref[TEI.ref]{ref} \hyperref[TEI.reg]{reg} \hyperref[TEI.rs]{rs} \hyperref[TEI.said]{said} \hyperref[TEI.sic]{sic} \hyperref[TEI.soCalled]{soCalled} \hyperref[TEI.sp]{sp} \hyperref[TEI.stage]{stage} \hyperref[TEI.term]{term} \hyperref[TEI.time]{time} \hyperref[TEI.title]{title} \hyperref[TEI.unclear]{unclear}\par 
    \item[derived-module-tei.istex: ]
   \hyperref[TEI.math]{math} \hyperref[TEI.mrow]{mrow}\par 
    \item[figures: ]
   \hyperref[TEI.figure]{figure} \hyperref[TEI.formula]{formula} \hyperref[TEI.notatedMusic]{notatedMusic} \hyperref[TEI.table]{table}\par 
    \item[header: ]
   \hyperref[TEI.biblFull]{biblFull} \hyperref[TEI.idno]{idno}\par 
    \item[iso-fs: ]
   \hyperref[TEI.fLib]{fLib} \hyperref[TEI.fs]{fs} \hyperref[TEI.fvLib]{fvLib}\par 
    \item[linking: ]
   \hyperref[TEI.ab]{ab} \hyperref[TEI.alt]{alt} \hyperref[TEI.altGrp]{altGrp} \hyperref[TEI.anchor]{anchor} \hyperref[TEI.join]{join} \hyperref[TEI.joinGrp]{joinGrp} \hyperref[TEI.link]{link} \hyperref[TEI.linkGrp]{linkGrp} \hyperref[TEI.seg]{seg} \hyperref[TEI.timeline]{timeline}\par 
    \item[msdescription: ]
   \hyperref[TEI.catchwords]{catchwords} \hyperref[TEI.depth]{depth} \hyperref[TEI.dim]{dim} \hyperref[TEI.dimensions]{dimensions} \hyperref[TEI.height]{height} \hyperref[TEI.heraldry]{heraldry} \hyperref[TEI.locus]{locus} \hyperref[TEI.locusGrp]{locusGrp} \hyperref[TEI.material]{material} \hyperref[TEI.msDesc]{msDesc} \hyperref[TEI.objectType]{objectType} \hyperref[TEI.origDate]{origDate} \hyperref[TEI.origPlace]{origPlace} \hyperref[TEI.secFol]{secFol} \hyperref[TEI.signatures]{signatures} \hyperref[TEI.source]{source} \hyperref[TEI.stamp]{stamp} \hyperref[TEI.watermark]{watermark} \hyperref[TEI.width]{width}\par 
    \item[namesdates: ]
   \hyperref[TEI.addName]{addName} \hyperref[TEI.affiliation]{affiliation} \hyperref[TEI.country]{country} \hyperref[TEI.forename]{forename} \hyperref[TEI.genName]{genName} \hyperref[TEI.geogName]{geogName} \hyperref[TEI.listOrg]{listOrg} \hyperref[TEI.listPlace]{listPlace} \hyperref[TEI.location]{location} \hyperref[TEI.nameLink]{nameLink} \hyperref[TEI.orgName]{orgName} \hyperref[TEI.persName]{persName} \hyperref[TEI.placeName]{placeName} \hyperref[TEI.region]{region} \hyperref[TEI.roleName]{roleName} \hyperref[TEI.settlement]{settlement} \hyperref[TEI.state]{state} \hyperref[TEI.surname]{surname}\par 
    \item[spoken: ]
   \hyperref[TEI.annotationBlock]{annotationBlock}\par 
    \item[textstructure: ]
   \hyperref[TEI.floatingText]{floatingText}\par 
    \item[transcr: ]
   \hyperref[TEI.addSpan]{addSpan} \hyperref[TEI.am]{am} \hyperref[TEI.damage]{damage} \hyperref[TEI.damageSpan]{damageSpan} \hyperref[TEI.delSpan]{delSpan} \hyperref[TEI.ex]{ex} \hyperref[TEI.fw]{fw} \hyperref[TEI.handShift]{handShift} \hyperref[TEI.listTranspose]{listTranspose} \hyperref[TEI.metamark]{metamark} \hyperref[TEI.mod]{mod} \hyperref[TEI.redo]{redo} \hyperref[TEI.restore]{restore} \hyperref[TEI.retrace]{retrace} \hyperref[TEI.secl]{secl} \hyperref[TEI.space]{space} \hyperref[TEI.subst]{subst} \hyperref[TEI.substJoin]{substJoin} \hyperref[TEI.supplied]{supplied} \hyperref[TEI.surplus]{surplus} \hyperref[TEI.undo]{undo}\par des données textuelles
    \item[{Exemple}]
  \leavevmode\bgroup\exampleFont \begin{shaded}\noindent\mbox{}{<\textbf{provenance}>}Listed as the property of Lawrence Sterne in 1788.{</\textbf{provenance}>}\mbox{}\newline 
{<\textbf{provenance}>}Sold at Sothebys in 1899.{</\textbf{provenance}>}\end{shaded}\egroup 


    \item[{Modèle de contenu}]
  \mbox{}\hfill\\[-10pt]\begin{Verbatim}[fontsize=\small]
<content>
 <macroRef key="macro.specialPara"/>
</content>
    
\end{Verbatim}

    \item[{Schéma Declaration}]
  \mbox{}\hfill\\[-10pt]\begin{Verbatim}[fontsize=\small]
element provenance
{
   tei_att.global.attributes,
   tei_att.datable.attributes,
   tei_att.typed.attributes,
   tei_macro.specialPara}
\end{Verbatim}

\end{reflist}  \index{ptr=<ptr>|oddindex}
\begin{reflist}
\item[]\begin{specHead}{TEI.ptr}{<ptr> }(pointeur) définit un pointeur vers un autre emplacement. [\xref{http://www.tei-c.org/release/doc/tei-p5-doc/en/html/CO.html\#COXR}{3.6. Simple Links and Cross-References} \xref{http://www.tei-c.org/release/doc/tei-p5-doc/en/html/SA.html\#SAPT}{16.1. Links}]\end{specHead} 
    \item[{Module}]
  core
    \item[{Attributs}]
  Attributs \hyperref[TEI.att.global]{att.global} (\textit{@xml:id}, \textit{@n}, \textit{@xml:lang}, \textit{@xml:base}, \textit{@xml:space})  (\hyperref[TEI.att.global.rendition]{att.global.rendition} (\textit{@rend}, \textit{@style}, \textit{@rendition})) (\hyperref[TEI.att.global.linking]{att.global.linking} (\textit{@corresp}, \textit{@synch}, \textit{@sameAs}, \textit{@copyOf}, \textit{@next}, \textit{@prev}, \textit{@exclude}, \textit{@select})) (\hyperref[TEI.att.global.analytic]{att.global.analytic} (\textit{@ana})) (\hyperref[TEI.att.global.facs]{att.global.facs} (\textit{@facs})) (\hyperref[TEI.att.global.change]{att.global.change} (\textit{@change})) (\hyperref[TEI.att.global.responsibility]{att.global.responsibility} (\textit{@cert}, \textit{@resp})) (\hyperref[TEI.att.global.source]{att.global.source} (\textit{@source})) \hyperref[TEI.att.pointing]{att.pointing} (\textit{@targetLang}, \textit{@target}, \textit{@evaluate}) \hyperref[TEI.att.internetMedia]{att.internetMedia} (\textit{@mimeType}) \hyperref[TEI.att.typed]{att.typed} (\textit{@type}, \textit{@subtype}) \hyperref[TEI.att.declaring]{att.declaring} (\textit{@decls}) \hyperref[TEI.att.cReferencing]{att.cReferencing} (\textit{@cRef}) 
    \item[{Membre du}]
  \hyperref[TEI.model.ptrLike]{model.ptrLike} 
    \item[{Contenu dans}]
  
    \item[analysis: ]
   \hyperref[TEI.cl]{cl} \hyperref[TEI.phr]{phr} \hyperref[TEI.s]{s} \hyperref[TEI.span]{span}\par 
    \item[core: ]
   \hyperref[TEI.abbr]{abbr} \hyperref[TEI.add]{add} \hyperref[TEI.addrLine]{addrLine} \hyperref[TEI.analytic]{analytic} \hyperref[TEI.author]{author} \hyperref[TEI.bibl]{bibl} \hyperref[TEI.biblScope]{biblScope} \hyperref[TEI.biblStruct]{biblStruct} \hyperref[TEI.cit]{cit} \hyperref[TEI.citedRange]{citedRange} \hyperref[TEI.corr]{corr} \hyperref[TEI.date]{date} \hyperref[TEI.del]{del} \hyperref[TEI.desc]{desc} \hyperref[TEI.distinct]{distinct} \hyperref[TEI.editor]{editor} \hyperref[TEI.email]{email} \hyperref[TEI.emph]{emph} \hyperref[TEI.expan]{expan} \hyperref[TEI.foreign]{foreign} \hyperref[TEI.gloss]{gloss} \hyperref[TEI.head]{head} \hyperref[TEI.headItem]{headItem} \hyperref[TEI.headLabel]{headLabel} \hyperref[TEI.hi]{hi} \hyperref[TEI.item]{item} \hyperref[TEI.l]{l} \hyperref[TEI.label]{label} \hyperref[TEI.measure]{measure} \hyperref[TEI.meeting]{meeting} \hyperref[TEI.mentioned]{mentioned} \hyperref[TEI.monogr]{monogr} \hyperref[TEI.name]{name} \hyperref[TEI.note]{note} \hyperref[TEI.num]{num} \hyperref[TEI.orig]{orig} \hyperref[TEI.p]{p} \hyperref[TEI.pubPlace]{pubPlace} \hyperref[TEI.publisher]{publisher} \hyperref[TEI.q]{q} \hyperref[TEI.quote]{quote} \hyperref[TEI.ref]{ref} \hyperref[TEI.reg]{reg} \hyperref[TEI.relatedItem]{relatedItem} \hyperref[TEI.resp]{resp} \hyperref[TEI.rs]{rs} \hyperref[TEI.said]{said} \hyperref[TEI.series]{series} \hyperref[TEI.sic]{sic} \hyperref[TEI.soCalled]{soCalled} \hyperref[TEI.speaker]{speaker} \hyperref[TEI.stage]{stage} \hyperref[TEI.street]{street} \hyperref[TEI.term]{term} \hyperref[TEI.textLang]{textLang} \hyperref[TEI.time]{time} \hyperref[TEI.title]{title} \hyperref[TEI.unclear]{unclear}\par 
    \item[figures: ]
   \hyperref[TEI.cell]{cell} \hyperref[TEI.figDesc]{figDesc} \hyperref[TEI.notatedMusic]{notatedMusic}\par 
    \item[header: ]
   \hyperref[TEI.application]{application} \hyperref[TEI.authority]{authority} \hyperref[TEI.change]{change} \hyperref[TEI.classCode]{classCode} \hyperref[TEI.creation]{creation} \hyperref[TEI.distributor]{distributor} \hyperref[TEI.edition]{edition} \hyperref[TEI.extent]{extent} \hyperref[TEI.funder]{funder} \hyperref[TEI.language]{language} \hyperref[TEI.licence]{licence} \hyperref[TEI.publicationStmt]{publicationStmt} \hyperref[TEI.rendition]{rendition}\par 
    \item[iso-fs: ]
   \hyperref[TEI.fDescr]{fDescr} \hyperref[TEI.fsDescr]{fsDescr}\par 
    \item[linking: ]
   \hyperref[TEI.ab]{ab} \hyperref[TEI.altGrp]{altGrp} \hyperref[TEI.joinGrp]{joinGrp} \hyperref[TEI.linkGrp]{linkGrp} \hyperref[TEI.seg]{seg}\par 
    \item[msdescription: ]
   \hyperref[TEI.accMat]{accMat} \hyperref[TEI.acquisition]{acquisition} \hyperref[TEI.additions]{additions} \hyperref[TEI.catchwords]{catchwords} \hyperref[TEI.collation]{collation} \hyperref[TEI.colophon]{colophon} \hyperref[TEI.condition]{condition} \hyperref[TEI.custEvent]{custEvent} \hyperref[TEI.decoNote]{decoNote} \hyperref[TEI.explicit]{explicit} \hyperref[TEI.filiation]{filiation} \hyperref[TEI.finalRubric]{finalRubric} \hyperref[TEI.foliation]{foliation} \hyperref[TEI.heraldry]{heraldry} \hyperref[TEI.incipit]{incipit} \hyperref[TEI.layout]{layout} \hyperref[TEI.material]{material} \hyperref[TEI.musicNotation]{musicNotation} \hyperref[TEI.objectType]{objectType} \hyperref[TEI.origDate]{origDate} \hyperref[TEI.origPlace]{origPlace} \hyperref[TEI.origin]{origin} \hyperref[TEI.provenance]{provenance} \hyperref[TEI.rubric]{rubric} \hyperref[TEI.secFol]{secFol} \hyperref[TEI.signatures]{signatures} \hyperref[TEI.source]{source} \hyperref[TEI.stamp]{stamp} \hyperref[TEI.summary]{summary} \hyperref[TEI.support]{support} \hyperref[TEI.surrogates]{surrogates} \hyperref[TEI.typeNote]{typeNote} \hyperref[TEI.watermark]{watermark}\par 
    \item[namesdates: ]
   \hyperref[TEI.addName]{addName} \hyperref[TEI.affiliation]{affiliation} \hyperref[TEI.country]{country} \hyperref[TEI.forename]{forename} \hyperref[TEI.genName]{genName} \hyperref[TEI.geogName]{geogName} \hyperref[TEI.nameLink]{nameLink} \hyperref[TEI.orgName]{orgName} \hyperref[TEI.persName]{persName} \hyperref[TEI.placeName]{placeName} \hyperref[TEI.region]{region} \hyperref[TEI.roleName]{roleName} \hyperref[TEI.settlement]{settlement} \hyperref[TEI.surname]{surname}\par 
    \item[spoken: ]
   \hyperref[TEI.annotationBlock]{annotationBlock}\par 
    \item[standOff: ]
   \hyperref[TEI.listAnnotation]{listAnnotation}\par 
    \item[textstructure: ]
   \hyperref[TEI.docAuthor]{docAuthor} \hyperref[TEI.docDate]{docDate} \hyperref[TEI.docEdition]{docEdition} \hyperref[TEI.titlePart]{titlePart}\par 
    \item[transcr: ]
   \hyperref[TEI.damage]{damage} \hyperref[TEI.fw]{fw} \hyperref[TEI.metamark]{metamark} \hyperref[TEI.mod]{mod} \hyperref[TEI.restore]{restore} \hyperref[TEI.retrace]{retrace} \hyperref[TEI.secl]{secl} \hyperref[TEI.supplied]{supplied} \hyperref[TEI.surplus]{surplus} \hyperref[TEI.transpose]{transpose}
    \item[{Peut contenir}]
  Elément vide
    \item[{Exemple}]
  \leavevmode\bgroup\exampleFont \begin{shaded}\noindent\mbox{}{<\textbf{ptr}\hspace*{6pt}{target}="{\#p143 \#p144}"/>}\mbox{}\newline 
{<\textbf{ptr}\hspace*{6pt}{target}="{http://www.tei-c.org}"/>}\mbox{}\newline 
{<\textbf{ptr}\hspace*{6pt}{cRef}="{1.3.4}"/>}\end{shaded}\egroup 


    \item[{Schematron}]
   <s:report test="@target and @cRef">Only one of the  attributes @target and @cRef may be supplied on <s:name/>.</s:report>
    \item[{Modèle de contenu}]
  \fbox{\ttfamily <content>\newline
</content>\newline
    } 
    \item[{Schéma Declaration}]
  \mbox{}\hfill\\[-10pt]\begin{Verbatim}[fontsize=\small]
element ptr
{
   tei_att.global.attributes,
   tei_att.pointing.attributes,
   tei_att.internetMedia.attributes,
   tei_att.typed.attributes,
   tei_att.declaring.attributes,
   tei_att.cReferencing.attributes,
   empty
}
\end{Verbatim}

\end{reflist}  \index{pubPlace=<pubPlace>|oddindex}
\begin{reflist}
\item[]\begin{specHead}{TEI.pubPlace}{<pubPlace> }(lieu de publication) contient le nom du lieu d'une publication. [\xref{http://www.tei-c.org/release/doc/tei-p5-doc/en/html/CO.html\#COBICOI}{3.11.2.4. Imprint, Size of a Document, and Reprint Information}]\end{specHead} 
    \item[{Module}]
  core
    \item[{Attributs}]
  Attributs \hyperref[TEI.att.global]{att.global} (\textit{@xml:id}, \textit{@n}, \textit{@xml:lang}, \textit{@xml:base}, \textit{@xml:space})  (\hyperref[TEI.att.global.rendition]{att.global.rendition} (\textit{@rend}, \textit{@style}, \textit{@rendition})) (\hyperref[TEI.att.global.linking]{att.global.linking} (\textit{@corresp}, \textit{@synch}, \textit{@sameAs}, \textit{@copyOf}, \textit{@next}, \textit{@prev}, \textit{@exclude}, \textit{@select})) (\hyperref[TEI.att.global.analytic]{att.global.analytic} (\textit{@ana})) (\hyperref[TEI.att.global.facs]{att.global.facs} (\textit{@facs})) (\hyperref[TEI.att.global.change]{att.global.change} (\textit{@change})) (\hyperref[TEI.att.global.responsibility]{att.global.responsibility} (\textit{@cert}, \textit{@resp})) (\hyperref[TEI.att.global.source]{att.global.source} (\textit{@source})) \hyperref[TEI.att.naming]{att.naming} (\textit{@role}, \textit{@nymRef})  (\hyperref[TEI.att.canonical]{att.canonical} (\textit{@key}, \textit{@ref}))
    \item[{Membre du}]
  \hyperref[TEI.model.imprintPart]{model.imprintPart} \hyperref[TEI.model.publicationStmtPart.detail]{model.publicationStmtPart.detail}
    \item[{Contenu dans}]
  
    \item[core: ]
   \hyperref[TEI.bibl]{bibl} \hyperref[TEI.imprint]{imprint}\par 
    \item[header: ]
   \hyperref[TEI.publicationStmt]{publicationStmt}
    \item[{Peut contenir}]
  
    \item[analysis: ]
   \hyperref[TEI.c]{c} \hyperref[TEI.cl]{cl} \hyperref[TEI.interp]{interp} \hyperref[TEI.interpGrp]{interpGrp} \hyperref[TEI.m]{m} \hyperref[TEI.pc]{pc} \hyperref[TEI.phr]{phr} \hyperref[TEI.s]{s} \hyperref[TEI.span]{span} \hyperref[TEI.spanGrp]{spanGrp} \hyperref[TEI.w]{w}\par 
    \item[core: ]
   \hyperref[TEI.abbr]{abbr} \hyperref[TEI.add]{add} \hyperref[TEI.address]{address} \hyperref[TEI.binaryObject]{binaryObject} \hyperref[TEI.cb]{cb} \hyperref[TEI.choice]{choice} \hyperref[TEI.corr]{corr} \hyperref[TEI.date]{date} \hyperref[TEI.del]{del} \hyperref[TEI.distinct]{distinct} \hyperref[TEI.email]{email} \hyperref[TEI.emph]{emph} \hyperref[TEI.expan]{expan} \hyperref[TEI.foreign]{foreign} \hyperref[TEI.gap]{gap} \hyperref[TEI.gb]{gb} \hyperref[TEI.gloss]{gloss} \hyperref[TEI.graphic]{graphic} \hyperref[TEI.hi]{hi} \hyperref[TEI.index]{index} \hyperref[TEI.lb]{lb} \hyperref[TEI.measure]{measure} \hyperref[TEI.measureGrp]{measureGrp} \hyperref[TEI.media]{media} \hyperref[TEI.mentioned]{mentioned} \hyperref[TEI.milestone]{milestone} \hyperref[TEI.name]{name} \hyperref[TEI.note]{note} \hyperref[TEI.num]{num} \hyperref[TEI.orig]{orig} \hyperref[TEI.pb]{pb} \hyperref[TEI.ptr]{ptr} \hyperref[TEI.ref]{ref} \hyperref[TEI.reg]{reg} \hyperref[TEI.rs]{rs} \hyperref[TEI.sic]{sic} \hyperref[TEI.soCalled]{soCalled} \hyperref[TEI.term]{term} \hyperref[TEI.time]{time} \hyperref[TEI.title]{title} \hyperref[TEI.unclear]{unclear}\par 
    \item[derived-module-tei.istex: ]
   \hyperref[TEI.math]{math} \hyperref[TEI.mrow]{mrow}\par 
    \item[figures: ]
   \hyperref[TEI.figure]{figure} \hyperref[TEI.formula]{formula} \hyperref[TEI.notatedMusic]{notatedMusic}\par 
    \item[header: ]
   \hyperref[TEI.idno]{idno}\par 
    \item[iso-fs: ]
   \hyperref[TEI.fLib]{fLib} \hyperref[TEI.fs]{fs} \hyperref[TEI.fvLib]{fvLib}\par 
    \item[linking: ]
   \hyperref[TEI.alt]{alt} \hyperref[TEI.altGrp]{altGrp} \hyperref[TEI.anchor]{anchor} \hyperref[TEI.join]{join} \hyperref[TEI.joinGrp]{joinGrp} \hyperref[TEI.link]{link} \hyperref[TEI.linkGrp]{linkGrp} \hyperref[TEI.seg]{seg} \hyperref[TEI.timeline]{timeline}\par 
    \item[msdescription: ]
   \hyperref[TEI.catchwords]{catchwords} \hyperref[TEI.depth]{depth} \hyperref[TEI.dim]{dim} \hyperref[TEI.dimensions]{dimensions} \hyperref[TEI.height]{height} \hyperref[TEI.heraldry]{heraldry} \hyperref[TEI.locus]{locus} \hyperref[TEI.locusGrp]{locusGrp} \hyperref[TEI.material]{material} \hyperref[TEI.objectType]{objectType} \hyperref[TEI.origDate]{origDate} \hyperref[TEI.origPlace]{origPlace} \hyperref[TEI.secFol]{secFol} \hyperref[TEI.signatures]{signatures} \hyperref[TEI.source]{source} \hyperref[TEI.stamp]{stamp} \hyperref[TEI.watermark]{watermark} \hyperref[TEI.width]{width}\par 
    \item[namesdates: ]
   \hyperref[TEI.addName]{addName} \hyperref[TEI.affiliation]{affiliation} \hyperref[TEI.country]{country} \hyperref[TEI.forename]{forename} \hyperref[TEI.genName]{genName} \hyperref[TEI.geogName]{geogName} \hyperref[TEI.location]{location} \hyperref[TEI.nameLink]{nameLink} \hyperref[TEI.orgName]{orgName} \hyperref[TEI.persName]{persName} \hyperref[TEI.placeName]{placeName} \hyperref[TEI.region]{region} \hyperref[TEI.roleName]{roleName} \hyperref[TEI.settlement]{settlement} \hyperref[TEI.state]{state} \hyperref[TEI.surname]{surname}\par 
    \item[spoken: ]
   \hyperref[TEI.annotationBlock]{annotationBlock}\par 
    \item[transcr: ]
   \hyperref[TEI.addSpan]{addSpan} \hyperref[TEI.am]{am} \hyperref[TEI.damage]{damage} \hyperref[TEI.damageSpan]{damageSpan} \hyperref[TEI.delSpan]{delSpan} \hyperref[TEI.ex]{ex} \hyperref[TEI.fw]{fw} \hyperref[TEI.handShift]{handShift} \hyperref[TEI.listTranspose]{listTranspose} \hyperref[TEI.metamark]{metamark} \hyperref[TEI.mod]{mod} \hyperref[TEI.redo]{redo} \hyperref[TEI.restore]{restore} \hyperref[TEI.retrace]{retrace} \hyperref[TEI.secl]{secl} \hyperref[TEI.space]{space} \hyperref[TEI.subst]{subst} \hyperref[TEI.substJoin]{substJoin} \hyperref[TEI.supplied]{supplied} \hyperref[TEI.surplus]{surplus} \hyperref[TEI.undo]{undo}\par des données textuelles
    \item[{Exemple}]
  \leavevmode\bgroup\exampleFont \begin{shaded}\noindent\mbox{}{<\textbf{publicationStmt}>}\mbox{}\newline 
\hspace*{6pt}{<\textbf{publisher}>}Editions Denoëll{</\textbf{publisher}>}\mbox{}\newline 
\hspace*{6pt}{<\textbf{pubPlace}>}Paris{</\textbf{pubPlace}>}\mbox{}\newline 
\hspace*{6pt}{<\textbf{date}>}1975{</\textbf{date}>}\mbox{}\newline 
{</\textbf{publicationStmt}>}\end{shaded}\egroup 


    \item[{Modèle de contenu}]
  \mbox{}\hfill\\[-10pt]\begin{Verbatim}[fontsize=\small]
<content>
 <macroRef key="macro.phraseSeq"/>
</content>
    
\end{Verbatim}

    \item[{Schéma Declaration}]
  \mbox{}\hfill\\[-10pt]\begin{Verbatim}[fontsize=\small]
element pubPlace
{
   tei_att.global.attributes,
   tei_att.naming.attributes,
   tei_macro.phraseSeq}
\end{Verbatim}

\end{reflist}  \index{publicationStmt=<publicationStmt>|oddindex}
\begin{reflist}
\item[]\begin{specHead}{TEI.publicationStmt}{<publicationStmt> }(mention de publication) regroupe des informations concernant la publication ou la diffusion d’un texte électronique ou d’un autre type de texte. [\xref{http://www.tei-c.org/release/doc/tei-p5-doc/en/html/HD.html\#HD24}{2.2.4. Publication, Distribution, Licensing, etc.} \xref{http://www.tei-c.org/release/doc/tei-p5-doc/en/html/HD.html\#HD2}{2.2. The File Description}]\end{specHead} 
    \item[{Module}]
  header
    \item[{Attributs}]
  Attributs \hyperref[TEI.att.global]{att.global} (\textit{@xml:id}, \textit{@n}, \textit{@xml:lang}, \textit{@xml:base}, \textit{@xml:space})  (\hyperref[TEI.att.global.rendition]{att.global.rendition} (\textit{@rend}, \textit{@style}, \textit{@rendition})) (\hyperref[TEI.att.global.linking]{att.global.linking} (\textit{@corresp}, \textit{@synch}, \textit{@sameAs}, \textit{@copyOf}, \textit{@next}, \textit{@prev}, \textit{@exclude}, \textit{@select})) (\hyperref[TEI.att.global.analytic]{att.global.analytic} (\textit{@ana})) (\hyperref[TEI.att.global.facs]{att.global.facs} (\textit{@facs})) (\hyperref[TEI.att.global.change]{att.global.change} (\textit{@change})) (\hyperref[TEI.att.global.responsibility]{att.global.responsibility} (\textit{@cert}, \textit{@resp})) (\hyperref[TEI.att.global.source]{att.global.source} (\textit{@source}))
    \item[{Contenu dans}]
  
    \item[header: ]
   \hyperref[TEI.biblFull]{biblFull} \hyperref[TEI.fileDesc]{fileDesc}
    \item[{Peut contenir}]
  
    \item[core: ]
   \hyperref[TEI.address]{address} \hyperref[TEI.date]{date} \hyperref[TEI.p]{p} \hyperref[TEI.ptr]{ptr} \hyperref[TEI.pubPlace]{pubPlace} \hyperref[TEI.publisher]{publisher} \hyperref[TEI.ref]{ref}\par 
    \item[header: ]
   \hyperref[TEI.authority]{authority} \hyperref[TEI.availability]{availability} \hyperref[TEI.distributor]{distributor} \hyperref[TEI.idno]{idno}\par 
    \item[linking: ]
   \hyperref[TEI.ab]{ab}
    \item[{Note}]
  \par
Bien que non imposé par les schémas, un document conforme à la TEI doit donner des informations sur le lieu de publication, l'adresse, l'identifiant, les droits de diffusion et la date dans cet ordre, après le nom de l'éditeur, du distributeur, ou de l'autorité concernée.
    \item[{Exemple}]
  \leavevmode\bgroup\exampleFont \begin{shaded}\noindent\mbox{}{<\textbf{publicationStmt}>}\mbox{}\newline 
\hspace*{6pt}{<\textbf{publisher}>}C. Muquardt {</\textbf{publisher}>}\mbox{}\newline 
\hspace*{6pt}{<\textbf{pubPlace}>}Bruxelles \& Leipzig{</\textbf{pubPlace}>}\mbox{}\newline 
\hspace*{6pt}{<\textbf{date}\hspace*{6pt}{when}="{1846}"/>}\mbox{}\newline 
{</\textbf{publicationStmt}>}\end{shaded}\egroup 


    \item[{Exemple}]
  \leavevmode\bgroup\exampleFont \begin{shaded}\noindent\mbox{}{<\textbf{publicationStmt}>}\mbox{}\newline 
\hspace*{6pt}{<\textbf{distributor}>}ATILF (Analyse et Traitement Informatique de la Langue Française){</\textbf{distributor}>}\mbox{}\newline 
\hspace*{6pt}{<\textbf{idno}\hspace*{6pt}{type}="{FRANTEXT}">}L434{</\textbf{idno}>}\mbox{}\newline 
\hspace*{6pt}{<\textbf{address}>}\mbox{}\newline 
\hspace*{6pt}\hspace*{6pt}{<\textbf{addrLine}>}44, avenue de la Libération{</\textbf{addrLine}>}\mbox{}\newline 
\hspace*{6pt}\hspace*{6pt}{<\textbf{addrLine}>}BP 30687{</\textbf{addrLine}>}\mbox{}\newline 
\hspace*{6pt}\hspace*{6pt}{<\textbf{addrLine}>}54063 Nancy Cedex{</\textbf{addrLine}>}\mbox{}\newline 
\hspace*{6pt}\hspace*{6pt}{<\textbf{addrLine}>}FRANCE{</\textbf{addrLine}>}\mbox{}\newline 
\hspace*{6pt}{</\textbf{address}>}\mbox{}\newline 
\hspace*{6pt}{<\textbf{availability}\hspace*{6pt}{status}="{free}">}\mbox{}\newline 
\hspace*{6pt}\hspace*{6pt}{<\textbf{p}>}Dans un cadre de recherche ou d'enseignement{</\textbf{p}>}\mbox{}\newline 
\hspace*{6pt}{</\textbf{availability}>}\mbox{}\newline 
{</\textbf{publicationStmt}>}\end{shaded}\egroup 


    \item[{Modèle de contenu}]
  \mbox{}\hfill\\[-10pt]\begin{Verbatim}[fontsize=\small]
<content>
 <alternate maxOccurs="1" minOccurs="1">
  <sequence maxOccurs="unbounded"
   minOccurs="1">
   <classRef key="model.publicationStmtPart.agency"/>
   <classRef key="model.publicationStmtPart.detail"
    maxOccurs="unbounded" minOccurs="0"/>
  </sequence>
  <classRef key="model.pLike"
   maxOccurs="unbounded" minOccurs="1"/>
 </alternate>
</content>
    
\end{Verbatim}

    \item[{Schéma Declaration}]
  \mbox{}\hfill\\[-10pt]\begin{Verbatim}[fontsize=\small]
element publicationStmt
{
   tei_att.global.attributes,
   (
      (
         tei_model.publicationStmtPart.agency,
         tei_model.publicationStmtPart.detail*
      )+
    | tei_model.pLike+
   )
}
\end{Verbatim}

\end{reflist}  \index{publisher=<publisher>|oddindex}\index{scheme=@scheme!<publisher>|oddindex}
\begin{reflist}
\item[]\begin{specHead}{TEI.publisher}{<publisher> }(éditeur) donne le nom de l'organisme responsable de la publication ou de la distribution d'un élément de la bibliographie. [\xref{http://www.tei-c.org/release/doc/tei-p5-doc/en/html/CO.html\#COBICOI}{3.11.2.4. Imprint, Size of a Document, and Reprint Information} \xref{http://www.tei-c.org/release/doc/tei-p5-doc/en/html/HD.html\#HD24}{2.2.4. Publication, Distribution, Licensing, etc.}]\end{specHead} 
    \item[{Module}]
  core
    \item[{Attributs}]
  Attributs \hyperref[TEI.att.global]{att.global} (\textit{@xml:id}, \textit{@n}, \textit{@xml:lang}, \textit{@xml:base}, \textit{@xml:space})  (\hyperref[TEI.att.global.rendition]{att.global.rendition} (\textit{@rend}, \textit{@style}, \textit{@rendition})) (\hyperref[TEI.att.global.linking]{att.global.linking} (\textit{@corresp}, \textit{@synch}, \textit{@sameAs}, \textit{@copyOf}, \textit{@next}, \textit{@prev}, \textit{@exclude}, \textit{@select})) (\hyperref[TEI.att.global.analytic]{att.global.analytic} (\textit{@ana})) (\hyperref[TEI.att.global.facs]{att.global.facs} (\textit{@facs})) (\hyperref[TEI.att.global.change]{att.global.change} (\textit{@change})) (\hyperref[TEI.att.global.responsibility]{att.global.responsibility} (\textit{@cert}, \textit{@resp})) (\hyperref[TEI.att.global.source]{att.global.source} (\textit{@source})) \hfil\\[-10pt]\begin{sansreflist}
    \item[@scheme]
  désigne la liste des ontologies dans lequel l'ensemble des termes concernés sont définis.
\begin{reflist}
    \item[{Statut}]
  Optionel
    \item[{Type de données}]
  \hyperref[TEI.teidata.pointer]{teidata.pointer}
\end{reflist}  
\end{sansreflist}  
    \item[{Membre du}]
  \hyperref[TEI.model.imprintPart]{model.imprintPart} \hyperref[TEI.model.publicationStmtPart.agency]{model.publicationStmtPart.agency}
    \item[{Contenu dans}]
  
    \item[core: ]
   \hyperref[TEI.bibl]{bibl} \hyperref[TEI.imprint]{imprint}\par 
    \item[header: ]
   \hyperref[TEI.publicationStmt]{publicationStmt}
    \item[{Peut contenir}]
  
    \item[analysis: ]
   \hyperref[TEI.c]{c} \hyperref[TEI.cl]{cl} \hyperref[TEI.interp]{interp} \hyperref[TEI.interpGrp]{interpGrp} \hyperref[TEI.m]{m} \hyperref[TEI.pc]{pc} \hyperref[TEI.phr]{phr} \hyperref[TEI.s]{s} \hyperref[TEI.span]{span} \hyperref[TEI.spanGrp]{spanGrp} \hyperref[TEI.w]{w}\par 
    \item[core: ]
   \hyperref[TEI.abbr]{abbr} \hyperref[TEI.add]{add} \hyperref[TEI.address]{address} \hyperref[TEI.binaryObject]{binaryObject} \hyperref[TEI.cb]{cb} \hyperref[TEI.choice]{choice} \hyperref[TEI.corr]{corr} \hyperref[TEI.date]{date} \hyperref[TEI.del]{del} \hyperref[TEI.distinct]{distinct} \hyperref[TEI.email]{email} \hyperref[TEI.emph]{emph} \hyperref[TEI.expan]{expan} \hyperref[TEI.foreign]{foreign} \hyperref[TEI.gap]{gap} \hyperref[TEI.gb]{gb} \hyperref[TEI.gloss]{gloss} \hyperref[TEI.graphic]{graphic} \hyperref[TEI.hi]{hi} \hyperref[TEI.index]{index} \hyperref[TEI.lb]{lb} \hyperref[TEI.measure]{measure} \hyperref[TEI.measureGrp]{measureGrp} \hyperref[TEI.media]{media} \hyperref[TEI.mentioned]{mentioned} \hyperref[TEI.milestone]{milestone} \hyperref[TEI.name]{name} \hyperref[TEI.note]{note} \hyperref[TEI.num]{num} \hyperref[TEI.orig]{orig} \hyperref[TEI.pb]{pb} \hyperref[TEI.ptr]{ptr} \hyperref[TEI.ref]{ref} \hyperref[TEI.reg]{reg} \hyperref[TEI.rs]{rs} \hyperref[TEI.sic]{sic} \hyperref[TEI.soCalled]{soCalled} \hyperref[TEI.term]{term} \hyperref[TEI.time]{time} \hyperref[TEI.title]{title} \hyperref[TEI.unclear]{unclear}\par 
    \item[derived-module-tei.istex: ]
   \hyperref[TEI.math]{math} \hyperref[TEI.mrow]{mrow}\par 
    \item[figures: ]
   \hyperref[TEI.figure]{figure} \hyperref[TEI.formula]{formula} \hyperref[TEI.notatedMusic]{notatedMusic}\par 
    \item[header: ]
   \hyperref[TEI.idno]{idno}\par 
    \item[iso-fs: ]
   \hyperref[TEI.fLib]{fLib} \hyperref[TEI.fs]{fs} \hyperref[TEI.fvLib]{fvLib}\par 
    \item[linking: ]
   \hyperref[TEI.alt]{alt} \hyperref[TEI.altGrp]{altGrp} \hyperref[TEI.anchor]{anchor} \hyperref[TEI.join]{join} \hyperref[TEI.joinGrp]{joinGrp} \hyperref[TEI.link]{link} \hyperref[TEI.linkGrp]{linkGrp} \hyperref[TEI.seg]{seg} \hyperref[TEI.timeline]{timeline}\par 
    \item[msdescription: ]
   \hyperref[TEI.catchwords]{catchwords} \hyperref[TEI.depth]{depth} \hyperref[TEI.dim]{dim} \hyperref[TEI.dimensions]{dimensions} \hyperref[TEI.height]{height} \hyperref[TEI.heraldry]{heraldry} \hyperref[TEI.locus]{locus} \hyperref[TEI.locusGrp]{locusGrp} \hyperref[TEI.material]{material} \hyperref[TEI.objectType]{objectType} \hyperref[TEI.origDate]{origDate} \hyperref[TEI.origPlace]{origPlace} \hyperref[TEI.secFol]{secFol} \hyperref[TEI.signatures]{signatures} \hyperref[TEI.source]{source} \hyperref[TEI.stamp]{stamp} \hyperref[TEI.watermark]{watermark} \hyperref[TEI.width]{width}\par 
    \item[namesdates: ]
   \hyperref[TEI.addName]{addName} \hyperref[TEI.affiliation]{affiliation} \hyperref[TEI.country]{country} \hyperref[TEI.forename]{forename} \hyperref[TEI.genName]{genName} \hyperref[TEI.geogName]{geogName} \hyperref[TEI.location]{location} \hyperref[TEI.nameLink]{nameLink} \hyperref[TEI.orgName]{orgName} \hyperref[TEI.persName]{persName} \hyperref[TEI.placeName]{placeName} \hyperref[TEI.region]{region} \hyperref[TEI.roleName]{roleName} \hyperref[TEI.settlement]{settlement} \hyperref[TEI.state]{state} \hyperref[TEI.surname]{surname}\par 
    \item[spoken: ]
   \hyperref[TEI.annotationBlock]{annotationBlock}\par 
    \item[transcr: ]
   \hyperref[TEI.addSpan]{addSpan} \hyperref[TEI.am]{am} \hyperref[TEI.damage]{damage} \hyperref[TEI.damageSpan]{damageSpan} \hyperref[TEI.delSpan]{delSpan} \hyperref[TEI.ex]{ex} \hyperref[TEI.fw]{fw} \hyperref[TEI.handShift]{handShift} \hyperref[TEI.listTranspose]{listTranspose} \hyperref[TEI.metamark]{metamark} \hyperref[TEI.mod]{mod} \hyperref[TEI.redo]{redo} \hyperref[TEI.restore]{restore} \hyperref[TEI.retrace]{retrace} \hyperref[TEI.secl]{secl} \hyperref[TEI.space]{space} \hyperref[TEI.subst]{subst} \hyperref[TEI.substJoin]{substJoin} \hyperref[TEI.supplied]{supplied} \hyperref[TEI.surplus]{surplus} \hyperref[TEI.undo]{undo}\par des données textuelles
    \item[{Note}]
  \par
Utiliser la forme développée du nom au moyen duquel l'organisme est habituellement cité, plutôt qu'une abréviation, cette dernière pouvant apparaître sur une page de titre.
    \item[{Exemple}]
  \leavevmode\bgroup\exampleFont \begin{shaded}\noindent\mbox{}{<\textbf{imprint}>}\mbox{}\newline 
\hspace*{6pt}{<\textbf{pubPlace}>}Paris{</\textbf{pubPlace}>}\mbox{}\newline 
\hspace*{6pt}{<\textbf{publisher}>}Les Éditions de Minuit{</\textbf{publisher}>}\mbox{}\newline 
\hspace*{6pt}{<\textbf{date}>}2001{</\textbf{date}>}\mbox{}\newline 
{</\textbf{imprint}>}\end{shaded}\egroup 


    \item[{Modèle de contenu}]
  \mbox{}\hfill\\[-10pt]\begin{Verbatim}[fontsize=\small]
<content>
 <macroRef key="macro.phraseSeq"/>
</content>
    
\end{Verbatim}

    \item[{Schéma Declaration}]
  \mbox{}\hfill\\[-10pt]\begin{Verbatim}[fontsize=\small]
element publisher
{
   tei_att.global.attributes,
   attribute scheme { text }?,
   tei_macro.phraseSeq}
\end{Verbatim}

\end{reflist}  \index{q=<q>|oddindex}\index{type=@type!<q>|oddindex}
\begin{reflist}
\item[]\begin{specHead}{TEI.q}{<q> }(séparé du texte environnant par des guillemets) contient un fragment qui est marqué (visiblement) comme étant d’une manière ou d'une autre différent du texte environnant, pour diverses raisons telles que, par exemple, un discours direct ou une pensée, des termes techniques ou du jargon, une mise à distance par rapport à l’auteur, des citations empruntées et des passages qui sont mentionnés mais non employés. [\xref{http://www.tei-c.org/release/doc/tei-p5-doc/en/html/CO.html\#COHQQ}{3.3.3. Quotation}]\end{specHead} 
    \item[{Module}]
  core
    \item[{Attributs}]
  Attributs \hyperref[TEI.att.global]{att.global} (\textit{@xml:id}, \textit{@n}, \textit{@xml:lang}, \textit{@xml:base}, \textit{@xml:space})  (\hyperref[TEI.att.global.rendition]{att.global.rendition} (\textit{@rend}, \textit{@style}, \textit{@rendition})) (\hyperref[TEI.att.global.linking]{att.global.linking} (\textit{@corresp}, \textit{@synch}, \textit{@sameAs}, \textit{@copyOf}, \textit{@next}, \textit{@prev}, \textit{@exclude}, \textit{@select})) (\hyperref[TEI.att.global.analytic]{att.global.analytic} (\textit{@ana})) (\hyperref[TEI.att.global.facs]{att.global.facs} (\textit{@facs})) (\hyperref[TEI.att.global.change]{att.global.change} (\textit{@change})) (\hyperref[TEI.att.global.responsibility]{att.global.responsibility} (\textit{@cert}, \textit{@resp})) (\hyperref[TEI.att.global.source]{att.global.source} (\textit{@source})) \hyperref[TEI.att.ascribed]{att.ascribed} (\textit{@who}) \hfil\\[-10pt]\begin{sansreflist}
    \item[@type]
  peut être utilisé pour indiquer si le passage cité correspond à une parole ou à une pensée ou encore pour le caractériser plus finement.
\begin{reflist}
    \item[{Statut}]
  Optionel
    \item[{Type de données}]
  \hyperref[TEI.teidata.enumerated]{teidata.enumerated}
    \item[{Les valeurs suggérées comprennent:}]
  \begin{description}

\item[{spoken}]notation du discours direct
\item[{thought}]représentation de la pensée, par exemple un monologue intérieur.
\item[{written}]citation d'une source écrite
\item[{soCalled}]distance prise par rapport à l'auteur
\item[{foreign}]mots étrangers
\item[{distinct}]linguistiquement distinct
\item[{term}]terme technique
\item[{emph}]mis en valeur par un procédé rhétorique.
\item[{mentioned}]métalinguistic, i.e. faisant référence à lui-même et non à son référent habituel.
\end{description} 
\end{reflist}  
\end{sansreflist}  
    \item[{Membre du}]
  \hyperref[TEI.model.qLike]{model.qLike}
    \item[{Contenu dans}]
  
    \item[core: ]
   \hyperref[TEI.add]{add} \hyperref[TEI.cit]{cit} \hyperref[TEI.corr]{corr} \hyperref[TEI.del]{del} \hyperref[TEI.desc]{desc} \hyperref[TEI.emph]{emph} \hyperref[TEI.head]{head} \hyperref[TEI.hi]{hi} \hyperref[TEI.item]{item} \hyperref[TEI.l]{l} \hyperref[TEI.meeting]{meeting} \hyperref[TEI.note]{note} \hyperref[TEI.orig]{orig} \hyperref[TEI.p]{p} \hyperref[TEI.q]{q} \hyperref[TEI.quote]{quote} \hyperref[TEI.ref]{ref} \hyperref[TEI.reg]{reg} \hyperref[TEI.said]{said} \hyperref[TEI.sic]{sic} \hyperref[TEI.sp]{sp} \hyperref[TEI.stage]{stage} \hyperref[TEI.title]{title} \hyperref[TEI.unclear]{unclear}\par 
    \item[figures: ]
   \hyperref[TEI.cell]{cell} \hyperref[TEI.figDesc]{figDesc} \hyperref[TEI.figure]{figure}\par 
    \item[header: ]
   \hyperref[TEI.change]{change} \hyperref[TEI.licence]{licence} \hyperref[TEI.rendition]{rendition}\par 
    \item[iso-fs: ]
   \hyperref[TEI.fDescr]{fDescr} \hyperref[TEI.fsDescr]{fsDescr}\par 
    \item[linking: ]
   \hyperref[TEI.ab]{ab} \hyperref[TEI.seg]{seg}\par 
    \item[msdescription: ]
   \hyperref[TEI.accMat]{accMat} \hyperref[TEI.acquisition]{acquisition} \hyperref[TEI.additions]{additions} \hyperref[TEI.collation]{collation} \hyperref[TEI.condition]{condition} \hyperref[TEI.custEvent]{custEvent} \hyperref[TEI.decoNote]{decoNote} \hyperref[TEI.filiation]{filiation} \hyperref[TEI.foliation]{foliation} \hyperref[TEI.layout]{layout} \hyperref[TEI.musicNotation]{musicNotation} \hyperref[TEI.origin]{origin} \hyperref[TEI.provenance]{provenance} \hyperref[TEI.signatures]{signatures} \hyperref[TEI.source]{source} \hyperref[TEI.summary]{summary} \hyperref[TEI.support]{support} \hyperref[TEI.surrogates]{surrogates} \hyperref[TEI.typeNote]{typeNote}\par 
    \item[textstructure: ]
   \hyperref[TEI.body]{body} \hyperref[TEI.div]{div} \hyperref[TEI.docEdition]{docEdition} \hyperref[TEI.titlePart]{titlePart}\par 
    \item[transcr: ]
   \hyperref[TEI.damage]{damage} \hyperref[TEI.metamark]{metamark} \hyperref[TEI.mod]{mod} \hyperref[TEI.restore]{restore} \hyperref[TEI.retrace]{retrace} \hyperref[TEI.secl]{secl} \hyperref[TEI.supplied]{supplied} \hyperref[TEI.surplus]{surplus}
    \item[{Peut contenir}]
  
    \item[analysis: ]
   \hyperref[TEI.c]{c} \hyperref[TEI.cl]{cl} \hyperref[TEI.interp]{interp} \hyperref[TEI.interpGrp]{interpGrp} \hyperref[TEI.m]{m} \hyperref[TEI.pc]{pc} \hyperref[TEI.phr]{phr} \hyperref[TEI.s]{s} \hyperref[TEI.span]{span} \hyperref[TEI.spanGrp]{spanGrp} \hyperref[TEI.w]{w}\par 
    \item[core: ]
   \hyperref[TEI.abbr]{abbr} \hyperref[TEI.add]{add} \hyperref[TEI.address]{address} \hyperref[TEI.bibl]{bibl} \hyperref[TEI.biblStruct]{biblStruct} \hyperref[TEI.binaryObject]{binaryObject} \hyperref[TEI.cb]{cb} \hyperref[TEI.choice]{choice} \hyperref[TEI.cit]{cit} \hyperref[TEI.corr]{corr} \hyperref[TEI.date]{date} \hyperref[TEI.del]{del} \hyperref[TEI.desc]{desc} \hyperref[TEI.distinct]{distinct} \hyperref[TEI.email]{email} \hyperref[TEI.emph]{emph} \hyperref[TEI.expan]{expan} \hyperref[TEI.foreign]{foreign} \hyperref[TEI.gap]{gap} \hyperref[TEI.gb]{gb} \hyperref[TEI.gloss]{gloss} \hyperref[TEI.graphic]{graphic} \hyperref[TEI.hi]{hi} \hyperref[TEI.index]{index} \hyperref[TEI.l]{l} \hyperref[TEI.label]{label} \hyperref[TEI.lb]{lb} \hyperref[TEI.lg]{lg} \hyperref[TEI.list]{list} \hyperref[TEI.listBibl]{listBibl} \hyperref[TEI.measure]{measure} \hyperref[TEI.measureGrp]{measureGrp} \hyperref[TEI.media]{media} \hyperref[TEI.mentioned]{mentioned} \hyperref[TEI.milestone]{milestone} \hyperref[TEI.name]{name} \hyperref[TEI.note]{note} \hyperref[TEI.num]{num} \hyperref[TEI.orig]{orig} \hyperref[TEI.p]{p} \hyperref[TEI.pb]{pb} \hyperref[TEI.ptr]{ptr} \hyperref[TEI.q]{q} \hyperref[TEI.quote]{quote} \hyperref[TEI.ref]{ref} \hyperref[TEI.reg]{reg} \hyperref[TEI.rs]{rs} \hyperref[TEI.said]{said} \hyperref[TEI.sic]{sic} \hyperref[TEI.soCalled]{soCalled} \hyperref[TEI.sp]{sp} \hyperref[TEI.stage]{stage} \hyperref[TEI.term]{term} \hyperref[TEI.time]{time} \hyperref[TEI.title]{title} \hyperref[TEI.unclear]{unclear}\par 
    \item[derived-module-tei.istex: ]
   \hyperref[TEI.math]{math} \hyperref[TEI.mrow]{mrow}\par 
    \item[figures: ]
   \hyperref[TEI.figure]{figure} \hyperref[TEI.formula]{formula} \hyperref[TEI.notatedMusic]{notatedMusic} \hyperref[TEI.table]{table}\par 
    \item[header: ]
   \hyperref[TEI.biblFull]{biblFull} \hyperref[TEI.idno]{idno}\par 
    \item[iso-fs: ]
   \hyperref[TEI.fLib]{fLib} \hyperref[TEI.fs]{fs} \hyperref[TEI.fvLib]{fvLib}\par 
    \item[linking: ]
   \hyperref[TEI.ab]{ab} \hyperref[TEI.alt]{alt} \hyperref[TEI.altGrp]{altGrp} \hyperref[TEI.anchor]{anchor} \hyperref[TEI.join]{join} \hyperref[TEI.joinGrp]{joinGrp} \hyperref[TEI.link]{link} \hyperref[TEI.linkGrp]{linkGrp} \hyperref[TEI.seg]{seg} \hyperref[TEI.timeline]{timeline}\par 
    \item[msdescription: ]
   \hyperref[TEI.catchwords]{catchwords} \hyperref[TEI.depth]{depth} \hyperref[TEI.dim]{dim} \hyperref[TEI.dimensions]{dimensions} \hyperref[TEI.height]{height} \hyperref[TEI.heraldry]{heraldry} \hyperref[TEI.locus]{locus} \hyperref[TEI.locusGrp]{locusGrp} \hyperref[TEI.material]{material} \hyperref[TEI.msDesc]{msDesc} \hyperref[TEI.objectType]{objectType} \hyperref[TEI.origDate]{origDate} \hyperref[TEI.origPlace]{origPlace} \hyperref[TEI.secFol]{secFol} \hyperref[TEI.signatures]{signatures} \hyperref[TEI.source]{source} \hyperref[TEI.stamp]{stamp} \hyperref[TEI.watermark]{watermark} \hyperref[TEI.width]{width}\par 
    \item[namesdates: ]
   \hyperref[TEI.addName]{addName} \hyperref[TEI.affiliation]{affiliation} \hyperref[TEI.country]{country} \hyperref[TEI.forename]{forename} \hyperref[TEI.genName]{genName} \hyperref[TEI.geogName]{geogName} \hyperref[TEI.listOrg]{listOrg} \hyperref[TEI.listPlace]{listPlace} \hyperref[TEI.location]{location} \hyperref[TEI.nameLink]{nameLink} \hyperref[TEI.orgName]{orgName} \hyperref[TEI.persName]{persName} \hyperref[TEI.placeName]{placeName} \hyperref[TEI.region]{region} \hyperref[TEI.roleName]{roleName} \hyperref[TEI.settlement]{settlement} \hyperref[TEI.state]{state} \hyperref[TEI.surname]{surname}\par 
    \item[spoken: ]
   \hyperref[TEI.annotationBlock]{annotationBlock}\par 
    \item[textstructure: ]
   \hyperref[TEI.floatingText]{floatingText}\par 
    \item[transcr: ]
   \hyperref[TEI.addSpan]{addSpan} \hyperref[TEI.am]{am} \hyperref[TEI.damage]{damage} \hyperref[TEI.damageSpan]{damageSpan} \hyperref[TEI.delSpan]{delSpan} \hyperref[TEI.ex]{ex} \hyperref[TEI.fw]{fw} \hyperref[TEI.handShift]{handShift} \hyperref[TEI.listTranspose]{listTranspose} \hyperref[TEI.metamark]{metamark} \hyperref[TEI.mod]{mod} \hyperref[TEI.redo]{redo} \hyperref[TEI.restore]{restore} \hyperref[TEI.retrace]{retrace} \hyperref[TEI.secl]{secl} \hyperref[TEI.space]{space} \hyperref[TEI.subst]{subst} \hyperref[TEI.substJoin]{substJoin} \hyperref[TEI.supplied]{supplied} \hyperref[TEI.surplus]{surplus} \hyperref[TEI.undo]{undo}\par des données textuelles
    \item[{Note}]
  \par
Peut être utilisé pour indiquer qu'un passage est distingué du texte environnant par des guillemets, pour des raisons non explicitées. Lorsqu'il est utilisé ainsi, \hyperref[TEI.q]{<q>} peut être considéré comme un encodage plus lisible (sucre syntaxique) pour \hyperref[TEI.hi]{<hi>} avec une valeur de {\itshape rend} indiquant la fonction des guillemets.
    \item[{Exemple}]
  \leavevmode\bgroup\exampleFont \begin{shaded}\noindent\mbox{}{<\textbf{p}>}Si quelque serrure allait mal, il l'avait bientôt démontée, rafistolée, huilée, limée,\mbox{}\newline 
 remontée, en disant :{<\textbf{q}>}ça me connaît{</\textbf{q}>}.{</\textbf{p}>}\end{shaded}\egroup 


    \item[{Exemple}]
  \leavevmode\bgroup\exampleFont \begin{shaded}\noindent\mbox{}{<\textbf{p}>}Enfin je me rappelai le pis-aller d’une grande princesse à\mbox{}\newline 
 qui l’on disait que les paysans n’avaient pas de pain, et qui\mbox{}\newline 
 répondit : {<\textbf{q}>}Qu’ils mangent de la brioche.{</\textbf{q}>}\mbox{}\newline 
{</\textbf{p}>}\end{shaded}\egroup 


    \item[{Modèle de contenu}]
  \mbox{}\hfill\\[-10pt]\begin{Verbatim}[fontsize=\small]
<content>
 <macroRef key="macro.specialPara"/>
</content>
    
\end{Verbatim}

    \item[{Schéma Declaration}]
  \mbox{}\hfill\\[-10pt]\begin{Verbatim}[fontsize=\small]
element q
{
   tei_att.global.attributes,
   tei_att.ascribed.attributes,
   attribute type
   {
      "spoken"
    | "thought"
    | "written"
    | "soCalled"
    | "foreign"
    | "distinct"
    | "term"
    | "emph"
    | "mentioned"
   }?,
   tei_macro.specialPara}
\end{Verbatim}

\end{reflist}  \index{quote=<quote>|oddindex}
\begin{reflist}
\item[]\begin{specHead}{TEI.quote}{<quote> }(citation) contient une expression ou un passage que le narrateur ou l'auteur attribue à une origine extérieure au texte. [\xref{http://www.tei-c.org/release/doc/tei-p5-doc/en/html/CO.html\#COHQQ}{3.3.3. Quotation} \xref{http://www.tei-c.org/release/doc/tei-p5-doc/en/html/DS.html\#DSGRP}{4.3.1. Grouped Texts}]\end{specHead} 
    \item[{Module}]
  core
    \item[{Attributs}]
  Attributs \hyperref[TEI.att.global]{att.global} (\textit{@xml:id}, \textit{@n}, \textit{@xml:lang}, \textit{@xml:base}, \textit{@xml:space})  (\hyperref[TEI.att.global.rendition]{att.global.rendition} (\textit{@rend}, \textit{@style}, \textit{@rendition})) (\hyperref[TEI.att.global.linking]{att.global.linking} (\textit{@corresp}, \textit{@synch}, \textit{@sameAs}, \textit{@copyOf}, \textit{@next}, \textit{@prev}, \textit{@exclude}, \textit{@select})) (\hyperref[TEI.att.global.analytic]{att.global.analytic} (\textit{@ana})) (\hyperref[TEI.att.global.facs]{att.global.facs} (\textit{@facs})) (\hyperref[TEI.att.global.change]{att.global.change} (\textit{@change})) (\hyperref[TEI.att.global.responsibility]{att.global.responsibility} (\textit{@cert}, \textit{@resp})) (\hyperref[TEI.att.global.source]{att.global.source} (\textit{@source})) \hyperref[TEI.att.typed]{att.typed} (\textit{@type}, \textit{@subtype}) \hyperref[TEI.att.msExcerpt]{att.msExcerpt} (\textit{@defective}) 
    \item[{Membre du}]
  \hyperref[TEI.model.quoteLike]{model.quoteLike}
    \item[{Contenu dans}]
  
    \item[core: ]
   \hyperref[TEI.add]{add} \hyperref[TEI.cit]{cit} \hyperref[TEI.corr]{corr} \hyperref[TEI.del]{del} \hyperref[TEI.desc]{desc} \hyperref[TEI.emph]{emph} \hyperref[TEI.head]{head} \hyperref[TEI.hi]{hi} \hyperref[TEI.item]{item} \hyperref[TEI.l]{l} \hyperref[TEI.meeting]{meeting} \hyperref[TEI.note]{note} \hyperref[TEI.orig]{orig} \hyperref[TEI.p]{p} \hyperref[TEI.q]{q} \hyperref[TEI.quote]{quote} \hyperref[TEI.ref]{ref} \hyperref[TEI.reg]{reg} \hyperref[TEI.said]{said} \hyperref[TEI.sic]{sic} \hyperref[TEI.sp]{sp} \hyperref[TEI.stage]{stage} \hyperref[TEI.title]{title} \hyperref[TEI.unclear]{unclear}\par 
    \item[figures: ]
   \hyperref[TEI.cell]{cell} \hyperref[TEI.figDesc]{figDesc} \hyperref[TEI.figure]{figure}\par 
    \item[header: ]
   \hyperref[TEI.change]{change} \hyperref[TEI.licence]{licence} \hyperref[TEI.rendition]{rendition}\par 
    \item[iso-fs: ]
   \hyperref[TEI.fDescr]{fDescr} \hyperref[TEI.fsDescr]{fsDescr}\par 
    \item[linking: ]
   \hyperref[TEI.ab]{ab} \hyperref[TEI.seg]{seg}\par 
    \item[msdescription: ]
   \hyperref[TEI.accMat]{accMat} \hyperref[TEI.acquisition]{acquisition} \hyperref[TEI.additions]{additions} \hyperref[TEI.collation]{collation} \hyperref[TEI.condition]{condition} \hyperref[TEI.custEvent]{custEvent} \hyperref[TEI.decoNote]{decoNote} \hyperref[TEI.filiation]{filiation} \hyperref[TEI.foliation]{foliation} \hyperref[TEI.layout]{layout} \hyperref[TEI.msItem]{msItem} \hyperref[TEI.musicNotation]{musicNotation} \hyperref[TEI.origin]{origin} \hyperref[TEI.provenance]{provenance} \hyperref[TEI.signatures]{signatures} \hyperref[TEI.source]{source} \hyperref[TEI.summary]{summary} \hyperref[TEI.support]{support} \hyperref[TEI.surrogates]{surrogates} \hyperref[TEI.typeNote]{typeNote}\par 
    \item[textstructure: ]
   \hyperref[TEI.body]{body} \hyperref[TEI.div]{div} \hyperref[TEI.docEdition]{docEdition} \hyperref[TEI.titlePart]{titlePart}\par 
    \item[transcr: ]
   \hyperref[TEI.damage]{damage} \hyperref[TEI.metamark]{metamark} \hyperref[TEI.mod]{mod} \hyperref[TEI.restore]{restore} \hyperref[TEI.retrace]{retrace} \hyperref[TEI.secl]{secl} \hyperref[TEI.supplied]{supplied} \hyperref[TEI.surplus]{surplus}
    \item[{Peut contenir}]
  
    \item[analysis: ]
   \hyperref[TEI.c]{c} \hyperref[TEI.cl]{cl} \hyperref[TEI.interp]{interp} \hyperref[TEI.interpGrp]{interpGrp} \hyperref[TEI.m]{m} \hyperref[TEI.pc]{pc} \hyperref[TEI.phr]{phr} \hyperref[TEI.s]{s} \hyperref[TEI.span]{span} \hyperref[TEI.spanGrp]{spanGrp} \hyperref[TEI.w]{w}\par 
    \item[core: ]
   \hyperref[TEI.abbr]{abbr} \hyperref[TEI.add]{add} \hyperref[TEI.address]{address} \hyperref[TEI.bibl]{bibl} \hyperref[TEI.biblStruct]{biblStruct} \hyperref[TEI.binaryObject]{binaryObject} \hyperref[TEI.cb]{cb} \hyperref[TEI.choice]{choice} \hyperref[TEI.cit]{cit} \hyperref[TEI.corr]{corr} \hyperref[TEI.date]{date} \hyperref[TEI.del]{del} \hyperref[TEI.desc]{desc} \hyperref[TEI.distinct]{distinct} \hyperref[TEI.email]{email} \hyperref[TEI.emph]{emph} \hyperref[TEI.expan]{expan} \hyperref[TEI.foreign]{foreign} \hyperref[TEI.gap]{gap} \hyperref[TEI.gb]{gb} \hyperref[TEI.gloss]{gloss} \hyperref[TEI.graphic]{graphic} \hyperref[TEI.hi]{hi} \hyperref[TEI.index]{index} \hyperref[TEI.l]{l} \hyperref[TEI.label]{label} \hyperref[TEI.lb]{lb} \hyperref[TEI.lg]{lg} \hyperref[TEI.list]{list} \hyperref[TEI.listBibl]{listBibl} \hyperref[TEI.measure]{measure} \hyperref[TEI.measureGrp]{measureGrp} \hyperref[TEI.media]{media} \hyperref[TEI.mentioned]{mentioned} \hyperref[TEI.milestone]{milestone} \hyperref[TEI.name]{name} \hyperref[TEI.note]{note} \hyperref[TEI.num]{num} \hyperref[TEI.orig]{orig} \hyperref[TEI.p]{p} \hyperref[TEI.pb]{pb} \hyperref[TEI.ptr]{ptr} \hyperref[TEI.q]{q} \hyperref[TEI.quote]{quote} \hyperref[TEI.ref]{ref} \hyperref[TEI.reg]{reg} \hyperref[TEI.rs]{rs} \hyperref[TEI.said]{said} \hyperref[TEI.sic]{sic} \hyperref[TEI.soCalled]{soCalled} \hyperref[TEI.sp]{sp} \hyperref[TEI.stage]{stage} \hyperref[TEI.term]{term} \hyperref[TEI.time]{time} \hyperref[TEI.title]{title} \hyperref[TEI.unclear]{unclear}\par 
    \item[derived-module-tei.istex: ]
   \hyperref[TEI.math]{math} \hyperref[TEI.mrow]{mrow}\par 
    \item[figures: ]
   \hyperref[TEI.figure]{figure} \hyperref[TEI.formula]{formula} \hyperref[TEI.notatedMusic]{notatedMusic} \hyperref[TEI.table]{table}\par 
    \item[header: ]
   \hyperref[TEI.biblFull]{biblFull} \hyperref[TEI.idno]{idno}\par 
    \item[iso-fs: ]
   \hyperref[TEI.fLib]{fLib} \hyperref[TEI.fs]{fs} \hyperref[TEI.fvLib]{fvLib}\par 
    \item[linking: ]
   \hyperref[TEI.ab]{ab} \hyperref[TEI.alt]{alt} \hyperref[TEI.altGrp]{altGrp} \hyperref[TEI.anchor]{anchor} \hyperref[TEI.join]{join} \hyperref[TEI.joinGrp]{joinGrp} \hyperref[TEI.link]{link} \hyperref[TEI.linkGrp]{linkGrp} \hyperref[TEI.seg]{seg} \hyperref[TEI.timeline]{timeline}\par 
    \item[msdescription: ]
   \hyperref[TEI.catchwords]{catchwords} \hyperref[TEI.depth]{depth} \hyperref[TEI.dim]{dim} \hyperref[TEI.dimensions]{dimensions} \hyperref[TEI.height]{height} \hyperref[TEI.heraldry]{heraldry} \hyperref[TEI.locus]{locus} \hyperref[TEI.locusGrp]{locusGrp} \hyperref[TEI.material]{material} \hyperref[TEI.msDesc]{msDesc} \hyperref[TEI.objectType]{objectType} \hyperref[TEI.origDate]{origDate} \hyperref[TEI.origPlace]{origPlace} \hyperref[TEI.secFol]{secFol} \hyperref[TEI.signatures]{signatures} \hyperref[TEI.source]{source} \hyperref[TEI.stamp]{stamp} \hyperref[TEI.watermark]{watermark} \hyperref[TEI.width]{width}\par 
    \item[namesdates: ]
   \hyperref[TEI.addName]{addName} \hyperref[TEI.affiliation]{affiliation} \hyperref[TEI.country]{country} \hyperref[TEI.forename]{forename} \hyperref[TEI.genName]{genName} \hyperref[TEI.geogName]{geogName} \hyperref[TEI.listOrg]{listOrg} \hyperref[TEI.listPlace]{listPlace} \hyperref[TEI.location]{location} \hyperref[TEI.nameLink]{nameLink} \hyperref[TEI.orgName]{orgName} \hyperref[TEI.persName]{persName} \hyperref[TEI.placeName]{placeName} \hyperref[TEI.region]{region} \hyperref[TEI.roleName]{roleName} \hyperref[TEI.settlement]{settlement} \hyperref[TEI.state]{state} \hyperref[TEI.surname]{surname}\par 
    \item[spoken: ]
   \hyperref[TEI.annotationBlock]{annotationBlock}\par 
    \item[textstructure: ]
   \hyperref[TEI.floatingText]{floatingText}\par 
    \item[transcr: ]
   \hyperref[TEI.addSpan]{addSpan} \hyperref[TEI.am]{am} \hyperref[TEI.damage]{damage} \hyperref[TEI.damageSpan]{damageSpan} \hyperref[TEI.delSpan]{delSpan} \hyperref[TEI.ex]{ex} \hyperref[TEI.fw]{fw} \hyperref[TEI.handShift]{handShift} \hyperref[TEI.listTranspose]{listTranspose} \hyperref[TEI.metamark]{metamark} \hyperref[TEI.mod]{mod} \hyperref[TEI.redo]{redo} \hyperref[TEI.restore]{restore} \hyperref[TEI.retrace]{retrace} \hyperref[TEI.secl]{secl} \hyperref[TEI.space]{space} \hyperref[TEI.subst]{subst} \hyperref[TEI.substJoin]{substJoin} \hyperref[TEI.supplied]{supplied} \hyperref[TEI.surplus]{surplus} \hyperref[TEI.undo]{undo}\par des données textuelles
    \item[{Note}]
  \par
Si une référence bibliographique est donnée comme source de la citation, on peut les regrouper dans l'élément \hyperref[TEI.cit]{<cit>}.
    \item[{Exemple}]
  \leavevmode\bgroup\exampleFont \begin{shaded}\noindent\mbox{}C'est sûrement ça\mbox{}\newline 
 qu'on appelle la glorieuse liberté des enfants de Dieu. {<\textbf{quote}>}Aime et fais tout ce que tu\mbox{}\newline 
 voudras.{</\textbf{quote}>}Mais moi, ça me démolit. \end{shaded}\egroup 


    \item[{Modèle de contenu}]
  \mbox{}\hfill\\[-10pt]\begin{Verbatim}[fontsize=\small]
<content>
 <macroRef key="macro.specialPara"/>
</content>
    
\end{Verbatim}

    \item[{Schéma Declaration}]
  \mbox{}\hfill\\[-10pt]\begin{Verbatim}[fontsize=\small]
element quote
{
   tei_att.global.attributes,
   tei_att.typed.attributes,
   tei_att.msExcerpt.attributes,
   tei_macro.specialPara}
\end{Verbatim}

\end{reflist}  \index{recordHist=<recordHist>|oddindex}
\begin{reflist}
\item[]\begin{specHead}{TEI.recordHist}{<recordHist> }(histoire de la description) donne des informations sur la source de la description et sur les modifications apportées à la description précédente. [\xref{http://www.tei-c.org/release/doc/tei-p5-doc/en/html/MS.html\#msadad}{10.9.1. Administrative Information}]\end{specHead} 
    \item[{Module}]
  msdescription
    \item[{Attributs}]
  Attributs \hyperref[TEI.att.global]{att.global} (\textit{@xml:id}, \textit{@n}, \textit{@xml:lang}, \textit{@xml:base}, \textit{@xml:space})  (\hyperref[TEI.att.global.rendition]{att.global.rendition} (\textit{@rend}, \textit{@style}, \textit{@rendition})) (\hyperref[TEI.att.global.linking]{att.global.linking} (\textit{@corresp}, \textit{@synch}, \textit{@sameAs}, \textit{@copyOf}, \textit{@next}, \textit{@prev}, \textit{@exclude}, \textit{@select})) (\hyperref[TEI.att.global.analytic]{att.global.analytic} (\textit{@ana})) (\hyperref[TEI.att.global.facs]{att.global.facs} (\textit{@facs})) (\hyperref[TEI.att.global.change]{att.global.change} (\textit{@change})) (\hyperref[TEI.att.global.responsibility]{att.global.responsibility} (\textit{@cert}, \textit{@resp})) (\hyperref[TEI.att.global.source]{att.global.source} (\textit{@source}))
    \item[{Contenu dans}]
  
    \item[msdescription: ]
   \hyperref[TEI.adminInfo]{adminInfo}
    \item[{Peut contenir}]
  
    \item[core: ]
   \hyperref[TEI.p]{p}\par 
    \item[header: ]
   \hyperref[TEI.change]{change}\par 
    \item[linking: ]
   \hyperref[TEI.ab]{ab}\par 
    \item[msdescription: ]
   \hyperref[TEI.source]{source}
    \item[{Exemple}]
  \leavevmode\bgroup\exampleFont \begin{shaded}\noindent\mbox{}{<\textbf{recordHist}>}\mbox{}\newline 
\hspace*{6pt}{<\textbf{source}>}\mbox{}\newline 
\hspace*{6pt}\hspace*{6pt}{<\textbf{p}>}Derived from {<\textbf{ref}\hspace*{6pt}{target}="{\#fr\textunderscore IMEV}">}IMEV 123{</\textbf{ref}>} with additional research by\mbox{}\newline 
\hspace*{6pt}\hspace*{6pt}\hspace*{6pt}\hspace*{6pt} P.M.W.Robinson{</\textbf{p}>}\mbox{}\newline 
\hspace*{6pt}{</\textbf{source}>}\mbox{}\newline 
\hspace*{6pt}{<\textbf{change}\hspace*{6pt}{when}="{1999-06-23}">}\mbox{}\newline 
\hspace*{6pt}\hspace*{6pt}{<\textbf{name}>}LDB{</\textbf{name}>} (editor) checked examples against DTD version\mbox{}\newline 
\hspace*{6pt}\hspace*{6pt} 3.6 {</\textbf{change}>}\mbox{}\newline 
{</\textbf{recordHist}>}\end{shaded}\egroup 


    \item[{Modèle de contenu}]
  \mbox{}\hfill\\[-10pt]\begin{Verbatim}[fontsize=\small]
<content>
 <alternate maxOccurs="1" minOccurs="1">
  <classRef key="model.pLike"
   maxOccurs="unbounded" minOccurs="1"/>
  <sequence maxOccurs="1" minOccurs="1">
   <elementRef key="source"/>
   <elementRef key="change"
    maxOccurs="unbounded" minOccurs="0"/>
  </sequence>
 </alternate>
</content>
    
\end{Verbatim}

    \item[{Schéma Declaration}]
  \mbox{}\hfill\\[-10pt]\begin{Verbatim}[fontsize=\small]
element recordHist
{
   tei_att.global.attributes,
   ( tei_model.pLike+ | ( tei_source, tei_change* ) )
}
\end{Verbatim}

\end{reflist}  \index{redo=<redo>|oddindex}\index{target=@target!<redo>|oddindex}
\begin{reflist}
\item[]\begin{specHead}{TEI.redo}{<redo> }indicates one or more cancelled interventions in a document which have subsequently been marked as reaffirmed or repeated. [\xref{http://www.tei-c.org/release/doc/tei-p5-doc/en/html/PH.html\#undo}{11.3.4.4. Confirmation, Cancellation, and Reinstatement of Modifications}]\end{specHead} 
    \item[{Module}]
  transcr
    \item[{Attributs}]
  Attributs \hyperref[TEI.att.global]{att.global} (\textit{@xml:id}, \textit{@n}, \textit{@xml:lang}, \textit{@xml:base}, \textit{@xml:space})  (\hyperref[TEI.att.global.rendition]{att.global.rendition} (\textit{@rend}, \textit{@style}, \textit{@rendition})) (\hyperref[TEI.att.global.linking]{att.global.linking} (\textit{@corresp}, \textit{@synch}, \textit{@sameAs}, \textit{@copyOf}, \textit{@next}, \textit{@prev}, \textit{@exclude}, \textit{@select})) (\hyperref[TEI.att.global.analytic]{att.global.analytic} (\textit{@ana})) (\hyperref[TEI.att.global.facs]{att.global.facs} (\textit{@facs})) (\hyperref[TEI.att.global.change]{att.global.change} (\textit{@change})) (\hyperref[TEI.att.global.responsibility]{att.global.responsibility} (\textit{@cert}, \textit{@resp})) (\hyperref[TEI.att.global.source]{att.global.source} (\textit{@source})) \hyperref[TEI.att.spanning]{att.spanning} (\textit{@spanTo}) \hyperref[TEI.att.transcriptional]{att.transcriptional} (\textit{@status}, \textit{@cause}, \textit{@seq})  (\hyperref[TEI.att.editLike]{att.editLike} (\textit{@evidence}, \textit{@instant}) (\hyperref[TEI.att.dimensions]{att.dimensions} (\textit{@unit}, \textit{@quantity}, \textit{@extent}, \textit{@precision}, \textit{@scope}) (\hyperref[TEI.att.ranging]{att.ranging} (\textit{@atLeast}, \textit{@atMost}, \textit{@min}, \textit{@max}, \textit{@confidence})) ) ) (\hyperref[TEI.att.written]{att.written} (\textit{@hand})) \hfil\\[-10pt]\begin{sansreflist}
    \item[@target]
  points to one or more elements representing the interventions which are being reasserted.
\begin{reflist}
    \item[{Statut}]
  Optionel
    \item[{Type de données}]
  1–∞ occurrences de \hyperref[TEI.teidata.pointer]{teidata.pointer} séparé par un espace
\end{reflist}  
\end{sansreflist}  
    \item[{Membre du}]
  \hyperref[TEI.model.linePart]{model.linePart} \hyperref[TEI.model.pPart.transcriptional]{model.pPart.transcriptional}
    \item[{Contenu dans}]
  
    \item[analysis: ]
   \hyperref[TEI.cl]{cl} \hyperref[TEI.pc]{pc} \hyperref[TEI.phr]{phr} \hyperref[TEI.s]{s} \hyperref[TEI.w]{w}\par 
    \item[core: ]
   \hyperref[TEI.abbr]{abbr} \hyperref[TEI.add]{add} \hyperref[TEI.addrLine]{addrLine} \hyperref[TEI.author]{author} \hyperref[TEI.bibl]{bibl} \hyperref[TEI.biblScope]{biblScope} \hyperref[TEI.citedRange]{citedRange} \hyperref[TEI.corr]{corr} \hyperref[TEI.date]{date} \hyperref[TEI.del]{del} \hyperref[TEI.distinct]{distinct} \hyperref[TEI.editor]{editor} \hyperref[TEI.email]{email} \hyperref[TEI.emph]{emph} \hyperref[TEI.expan]{expan} \hyperref[TEI.foreign]{foreign} \hyperref[TEI.gloss]{gloss} \hyperref[TEI.head]{head} \hyperref[TEI.headItem]{headItem} \hyperref[TEI.headLabel]{headLabel} \hyperref[TEI.hi]{hi} \hyperref[TEI.item]{item} \hyperref[TEI.l]{l} \hyperref[TEI.label]{label} \hyperref[TEI.measure]{measure} \hyperref[TEI.mentioned]{mentioned} \hyperref[TEI.name]{name} \hyperref[TEI.note]{note} \hyperref[TEI.num]{num} \hyperref[TEI.orig]{orig} \hyperref[TEI.p]{p} \hyperref[TEI.pubPlace]{pubPlace} \hyperref[TEI.publisher]{publisher} \hyperref[TEI.q]{q} \hyperref[TEI.quote]{quote} \hyperref[TEI.ref]{ref} \hyperref[TEI.reg]{reg} \hyperref[TEI.rs]{rs} \hyperref[TEI.said]{said} \hyperref[TEI.sic]{sic} \hyperref[TEI.soCalled]{soCalled} \hyperref[TEI.speaker]{speaker} \hyperref[TEI.stage]{stage} \hyperref[TEI.street]{street} \hyperref[TEI.term]{term} \hyperref[TEI.textLang]{textLang} \hyperref[TEI.time]{time} \hyperref[TEI.title]{title} \hyperref[TEI.unclear]{unclear}\par 
    \item[figures: ]
   \hyperref[TEI.cell]{cell}\par 
    \item[header: ]
   \hyperref[TEI.change]{change} \hyperref[TEI.distributor]{distributor} \hyperref[TEI.edition]{edition} \hyperref[TEI.extent]{extent} \hyperref[TEI.licence]{licence}\par 
    \item[linking: ]
   \hyperref[TEI.ab]{ab} \hyperref[TEI.seg]{seg}\par 
    \item[msdescription: ]
   \hyperref[TEI.accMat]{accMat} \hyperref[TEI.acquisition]{acquisition} \hyperref[TEI.additions]{additions} \hyperref[TEI.catchwords]{catchwords} \hyperref[TEI.collation]{collation} \hyperref[TEI.colophon]{colophon} \hyperref[TEI.condition]{condition} \hyperref[TEI.custEvent]{custEvent} \hyperref[TEI.decoNote]{decoNote} \hyperref[TEI.explicit]{explicit} \hyperref[TEI.filiation]{filiation} \hyperref[TEI.finalRubric]{finalRubric} \hyperref[TEI.foliation]{foliation} \hyperref[TEI.heraldry]{heraldry} \hyperref[TEI.incipit]{incipit} \hyperref[TEI.layout]{layout} \hyperref[TEI.material]{material} \hyperref[TEI.musicNotation]{musicNotation} \hyperref[TEI.objectType]{objectType} \hyperref[TEI.origDate]{origDate} \hyperref[TEI.origPlace]{origPlace} \hyperref[TEI.origin]{origin} \hyperref[TEI.provenance]{provenance} \hyperref[TEI.rubric]{rubric} \hyperref[TEI.secFol]{secFol} \hyperref[TEI.signatures]{signatures} \hyperref[TEI.source]{source} \hyperref[TEI.stamp]{stamp} \hyperref[TEI.summary]{summary} \hyperref[TEI.support]{support} \hyperref[TEI.surrogates]{surrogates} \hyperref[TEI.typeNote]{typeNote} \hyperref[TEI.watermark]{watermark}\par 
    \item[namesdates: ]
   \hyperref[TEI.addName]{addName} \hyperref[TEI.affiliation]{affiliation} \hyperref[TEI.country]{country} \hyperref[TEI.forename]{forename} \hyperref[TEI.genName]{genName} \hyperref[TEI.geogName]{geogName} \hyperref[TEI.nameLink]{nameLink} \hyperref[TEI.orgName]{orgName} \hyperref[TEI.persName]{persName} \hyperref[TEI.placeName]{placeName} \hyperref[TEI.region]{region} \hyperref[TEI.roleName]{roleName} \hyperref[TEI.settlement]{settlement} \hyperref[TEI.surname]{surname}\par 
    \item[textstructure: ]
   \hyperref[TEI.docAuthor]{docAuthor} \hyperref[TEI.docDate]{docDate} \hyperref[TEI.docEdition]{docEdition} \hyperref[TEI.titlePart]{titlePart}\par 
    \item[transcr: ]
   \hyperref[TEI.am]{am} \hyperref[TEI.damage]{damage} \hyperref[TEI.fw]{fw} \hyperref[TEI.line]{line} \hyperref[TEI.metamark]{metamark} \hyperref[TEI.mod]{mod} \hyperref[TEI.restore]{restore} \hyperref[TEI.retrace]{retrace} \hyperref[TEI.secl]{secl} \hyperref[TEI.supplied]{supplied} \hyperref[TEI.surplus]{surplus} \hyperref[TEI.zone]{zone}
    \item[{Peut contenir}]
  Elément vide
    \item[{Exemple}]
  \leavevmode\bgroup\exampleFont \begin{shaded}\noindent\mbox{}{<\textbf{line}>}\mbox{}\newline 
\hspace*{6pt}{<\textbf{redo}\hspace*{6pt}{cause}="{fix}"\hspace*{6pt}{hand}="{\#g\textunderscore t}"\mbox{}\newline 
\hspace*{6pt}\hspace*{6pt}{target}="{\#redo-1}"/>}\mbox{}\newline 
\hspace*{6pt}{<\textbf{mod}\hspace*{6pt}{hand}="{\#g\textunderscore bl}"\hspace*{6pt}{rend}="{strikethrough}"\mbox{}\newline 
\hspace*{6pt}\hspace*{6pt}{spanTo}="{\#anchor-1}"\hspace*{6pt}{xml:id}="{redo-1}"/>}Ihr hagren, triſten, krummgezog{<\textbf{mod}\hspace*{6pt}{rend}="{strikethrough}">}nen{</\textbf{mod}>}ener Nacken\mbox{}\newline 
{</\textbf{line}>}\mbox{}\newline 
{<\textbf{line}>}Wenn ihr nur piepſet iſt die Welt ſchon matt.{<\textbf{anchor}\hspace*{6pt}{xml:id}="{anchor-1}"/>}\mbox{}\newline 
{</\textbf{line}>}\end{shaded}\egroup 

This encoding represents the following sequence of events: \begin{itemize}
\item "Ihr hagren, tristen, krummgezog nenener Nacken/ Wenn ihr nur piepset ist die Welt schon matt." is written 
\item the redundant letters "nen" in "nenener" are deleted
\item the whole passage is deleted by hand \texttt{g\textunderscore bl} using strikethrough
\item the deletion is reasserted by another hand (identified here as \texttt{g\textunderscore t})
\end{itemize} 
    \item[{Modèle de contenu}]
  \fbox{\ttfamily <content>\newline
</content>\newline
    } 
    \item[{Schéma Declaration}]
  \mbox{}\hfill\\[-10pt]\begin{Verbatim}[fontsize=\small]
element redo
{
   tei_att.global.attributes,
   tei_att.spanning.attributes,
   tei_att.transcriptional.attributes,
   attribute target { list { + } }?,
   empty
}
\end{Verbatim}

\end{reflist}  \index{ref=<ref>|oddindex}\index{scheme=@scheme!<ref>|oddindex}
\begin{reflist}
\item[]\begin{specHead}{TEI.ref}{<ref> }(référence) définit une référence vers un autre emplacement, la référence étant éventuellement modifiée ou complétée par un texte ou un commentaire. [\xref{http://www.tei-c.org/release/doc/tei-p5-doc/en/html/CO.html\#COXR}{3.6. Simple Links and Cross-References} \xref{http://www.tei-c.org/release/doc/tei-p5-doc/en/html/SA.html\#SAPT}{16.1. Links}]\end{specHead} 
    \item[{Module}]
  core
    \item[{Attributs}]
  Attributs \hyperref[TEI.att.global]{att.global} (\textit{@xml:id}, \textit{@n}, \textit{@xml:lang}, \textit{@xml:base}, \textit{@xml:space})  (\hyperref[TEI.att.global.rendition]{att.global.rendition} (\textit{@rend}, \textit{@style}, \textit{@rendition})) (\hyperref[TEI.att.global.linking]{att.global.linking} (\textit{@corresp}, \textit{@synch}, \textit{@sameAs}, \textit{@copyOf}, \textit{@next}, \textit{@prev}, \textit{@exclude}, \textit{@select})) (\hyperref[TEI.att.global.analytic]{att.global.analytic} (\textit{@ana})) (\hyperref[TEI.att.global.facs]{att.global.facs} (\textit{@facs})) (\hyperref[TEI.att.global.change]{att.global.change} (\textit{@change})) (\hyperref[TEI.att.global.responsibility]{att.global.responsibility} (\textit{@cert}, \textit{@resp})) (\hyperref[TEI.att.global.source]{att.global.source} (\textit{@source})) \hyperref[TEI.att.pointing]{att.pointing} (\textit{@targetLang}, \textit{@target}, \textit{@evaluate}) \hyperref[TEI.att.internetMedia]{att.internetMedia} (\textit{@mimeType}) \hyperref[TEI.att.typed]{att.typed} (\textit{@type}, \textit{@subtype}) \hyperref[TEI.att.declaring]{att.declaring} (\textit{@decls}) \hyperref[TEI.att.cReferencing]{att.cReferencing} (\textit{@cRef}) \hfil\\[-10pt]\begin{sansreflist}
    \item[@scheme]
  désigne la liste des ontologies dans lequel l'ensemble des termes concernés sont définis.
\begin{reflist}
    \item[{Statut}]
  Optionel
    \item[{Type de données}]
  \hyperref[TEI.teidata.pointer]{teidata.pointer}
\end{reflist}  
\end{sansreflist}  
    \item[{Membre du}]
  \hyperref[TEI.model.ptrLike]{model.ptrLike}
    \item[{Contenu dans}]
  
    \item[analysis: ]
   \hyperref[TEI.cl]{cl} \hyperref[TEI.phr]{phr} \hyperref[TEI.s]{s} \hyperref[TEI.span]{span}\par 
    \item[core: ]
   \hyperref[TEI.abbr]{abbr} \hyperref[TEI.add]{add} \hyperref[TEI.addrLine]{addrLine} \hyperref[TEI.analytic]{analytic} \hyperref[TEI.author]{author} \hyperref[TEI.bibl]{bibl} \hyperref[TEI.biblScope]{biblScope} \hyperref[TEI.biblStruct]{biblStruct} \hyperref[TEI.cit]{cit} \hyperref[TEI.citedRange]{citedRange} \hyperref[TEI.corr]{corr} \hyperref[TEI.date]{date} \hyperref[TEI.del]{del} \hyperref[TEI.desc]{desc} \hyperref[TEI.distinct]{distinct} \hyperref[TEI.editor]{editor} \hyperref[TEI.email]{email} \hyperref[TEI.emph]{emph} \hyperref[TEI.expan]{expan} \hyperref[TEI.foreign]{foreign} \hyperref[TEI.gloss]{gloss} \hyperref[TEI.head]{head} \hyperref[TEI.headItem]{headItem} \hyperref[TEI.headLabel]{headLabel} \hyperref[TEI.hi]{hi} \hyperref[TEI.item]{item} \hyperref[TEI.l]{l} \hyperref[TEI.label]{label} \hyperref[TEI.measure]{measure} \hyperref[TEI.meeting]{meeting} \hyperref[TEI.mentioned]{mentioned} \hyperref[TEI.monogr]{monogr} \hyperref[TEI.name]{name} \hyperref[TEI.note]{note} \hyperref[TEI.num]{num} \hyperref[TEI.orig]{orig} \hyperref[TEI.p]{p} \hyperref[TEI.pubPlace]{pubPlace} \hyperref[TEI.publisher]{publisher} \hyperref[TEI.q]{q} \hyperref[TEI.quote]{quote} \hyperref[TEI.ref]{ref} \hyperref[TEI.reg]{reg} \hyperref[TEI.relatedItem]{relatedItem} \hyperref[TEI.resp]{resp} \hyperref[TEI.rs]{rs} \hyperref[TEI.said]{said} \hyperref[TEI.series]{series} \hyperref[TEI.sic]{sic} \hyperref[TEI.soCalled]{soCalled} \hyperref[TEI.speaker]{speaker} \hyperref[TEI.stage]{stage} \hyperref[TEI.street]{street} \hyperref[TEI.term]{term} \hyperref[TEI.textLang]{textLang} \hyperref[TEI.time]{time} \hyperref[TEI.title]{title} \hyperref[TEI.unclear]{unclear}\par 
    \item[figures: ]
   \hyperref[TEI.cell]{cell} \hyperref[TEI.figDesc]{figDesc} \hyperref[TEI.notatedMusic]{notatedMusic}\par 
    \item[header: ]
   \hyperref[TEI.application]{application} \hyperref[TEI.authority]{authority} \hyperref[TEI.change]{change} \hyperref[TEI.classCode]{classCode} \hyperref[TEI.creation]{creation} \hyperref[TEI.distributor]{distributor} \hyperref[TEI.edition]{edition} \hyperref[TEI.extent]{extent} \hyperref[TEI.funder]{funder} \hyperref[TEI.language]{language} \hyperref[TEI.licence]{licence} \hyperref[TEI.publicationStmt]{publicationStmt} \hyperref[TEI.rendition]{rendition}\par 
    \item[iso-fs: ]
   \hyperref[TEI.fDescr]{fDescr} \hyperref[TEI.fsDescr]{fsDescr}\par 
    \item[linking: ]
   \hyperref[TEI.ab]{ab} \hyperref[TEI.seg]{seg}\par 
    \item[msdescription: ]
   \hyperref[TEI.accMat]{accMat} \hyperref[TEI.acquisition]{acquisition} \hyperref[TEI.additions]{additions} \hyperref[TEI.catchwords]{catchwords} \hyperref[TEI.collation]{collation} \hyperref[TEI.colophon]{colophon} \hyperref[TEI.condition]{condition} \hyperref[TEI.custEvent]{custEvent} \hyperref[TEI.decoNote]{decoNote} \hyperref[TEI.explicit]{explicit} \hyperref[TEI.filiation]{filiation} \hyperref[TEI.finalRubric]{finalRubric} \hyperref[TEI.foliation]{foliation} \hyperref[TEI.heraldry]{heraldry} \hyperref[TEI.incipit]{incipit} \hyperref[TEI.layout]{layout} \hyperref[TEI.material]{material} \hyperref[TEI.musicNotation]{musicNotation} \hyperref[TEI.objectType]{objectType} \hyperref[TEI.origDate]{origDate} \hyperref[TEI.origPlace]{origPlace} \hyperref[TEI.origin]{origin} \hyperref[TEI.provenance]{provenance} \hyperref[TEI.rubric]{rubric} \hyperref[TEI.secFol]{secFol} \hyperref[TEI.signatures]{signatures} \hyperref[TEI.source]{source} \hyperref[TEI.stamp]{stamp} \hyperref[TEI.summary]{summary} \hyperref[TEI.support]{support} \hyperref[TEI.surrogates]{surrogates} \hyperref[TEI.typeNote]{typeNote} \hyperref[TEI.watermark]{watermark}\par 
    \item[namesdates: ]
   \hyperref[TEI.addName]{addName} \hyperref[TEI.affiliation]{affiliation} \hyperref[TEI.country]{country} \hyperref[TEI.forename]{forename} \hyperref[TEI.genName]{genName} \hyperref[TEI.geogName]{geogName} \hyperref[TEI.nameLink]{nameLink} \hyperref[TEI.orgName]{orgName} \hyperref[TEI.persName]{persName} \hyperref[TEI.placeName]{placeName} \hyperref[TEI.region]{region} \hyperref[TEI.roleName]{roleName} \hyperref[TEI.settlement]{settlement} \hyperref[TEI.surname]{surname}\par 
    \item[spoken: ]
   \hyperref[TEI.annotationBlock]{annotationBlock}\par 
    \item[standOff: ]
   \hyperref[TEI.listAnnotation]{listAnnotation}\par 
    \item[textstructure: ]
   \hyperref[TEI.docAuthor]{docAuthor} \hyperref[TEI.docDate]{docDate} \hyperref[TEI.docEdition]{docEdition} \hyperref[TEI.titlePart]{titlePart}\par 
    \item[transcr: ]
   \hyperref[TEI.damage]{damage} \hyperref[TEI.fw]{fw} \hyperref[TEI.metamark]{metamark} \hyperref[TEI.mod]{mod} \hyperref[TEI.restore]{restore} \hyperref[TEI.retrace]{retrace} \hyperref[TEI.secl]{secl} \hyperref[TEI.supplied]{supplied} \hyperref[TEI.surplus]{surplus}
    \item[{Peut contenir}]
  
    \item[analysis: ]
   \hyperref[TEI.c]{c} \hyperref[TEI.cl]{cl} \hyperref[TEI.interp]{interp} \hyperref[TEI.interpGrp]{interpGrp} \hyperref[TEI.m]{m} \hyperref[TEI.pc]{pc} \hyperref[TEI.phr]{phr} \hyperref[TEI.s]{s} \hyperref[TEI.span]{span} \hyperref[TEI.spanGrp]{spanGrp} \hyperref[TEI.w]{w}\par 
    \item[core: ]
   \hyperref[TEI.abbr]{abbr} \hyperref[TEI.add]{add} \hyperref[TEI.address]{address} \hyperref[TEI.bibl]{bibl} \hyperref[TEI.biblStruct]{biblStruct} \hyperref[TEI.binaryObject]{binaryObject} \hyperref[TEI.cb]{cb} \hyperref[TEI.choice]{choice} \hyperref[TEI.cit]{cit} \hyperref[TEI.corr]{corr} \hyperref[TEI.date]{date} \hyperref[TEI.del]{del} \hyperref[TEI.desc]{desc} \hyperref[TEI.distinct]{distinct} \hyperref[TEI.email]{email} \hyperref[TEI.emph]{emph} \hyperref[TEI.expan]{expan} \hyperref[TEI.foreign]{foreign} \hyperref[TEI.gap]{gap} \hyperref[TEI.gb]{gb} \hyperref[TEI.gloss]{gloss} \hyperref[TEI.graphic]{graphic} \hyperref[TEI.hi]{hi} \hyperref[TEI.index]{index} \hyperref[TEI.l]{l} \hyperref[TEI.label]{label} \hyperref[TEI.lb]{lb} \hyperref[TEI.lg]{lg} \hyperref[TEI.list]{list} \hyperref[TEI.listBibl]{listBibl} \hyperref[TEI.measure]{measure} \hyperref[TEI.measureGrp]{measureGrp} \hyperref[TEI.media]{media} \hyperref[TEI.mentioned]{mentioned} \hyperref[TEI.milestone]{milestone} \hyperref[TEI.name]{name} \hyperref[TEI.note]{note} \hyperref[TEI.num]{num} \hyperref[TEI.orig]{orig} \hyperref[TEI.pb]{pb} \hyperref[TEI.ptr]{ptr} \hyperref[TEI.q]{q} \hyperref[TEI.quote]{quote} \hyperref[TEI.ref]{ref} \hyperref[TEI.reg]{reg} \hyperref[TEI.rs]{rs} \hyperref[TEI.said]{said} \hyperref[TEI.sic]{sic} \hyperref[TEI.soCalled]{soCalled} \hyperref[TEI.stage]{stage} \hyperref[TEI.term]{term} \hyperref[TEI.time]{time} \hyperref[TEI.title]{title} \hyperref[TEI.unclear]{unclear}\par 
    \item[derived-module-tei.istex: ]
   \hyperref[TEI.math]{math} \hyperref[TEI.mrow]{mrow}\par 
    \item[figures: ]
   \hyperref[TEI.figure]{figure} \hyperref[TEI.formula]{formula} \hyperref[TEI.notatedMusic]{notatedMusic} \hyperref[TEI.table]{table}\par 
    \item[header: ]
   \hyperref[TEI.biblFull]{biblFull} \hyperref[TEI.idno]{idno}\par 
    \item[iso-fs: ]
   \hyperref[TEI.fLib]{fLib} \hyperref[TEI.fs]{fs} \hyperref[TEI.fvLib]{fvLib}\par 
    \item[linking: ]
   \hyperref[TEI.alt]{alt} \hyperref[TEI.altGrp]{altGrp} \hyperref[TEI.anchor]{anchor} \hyperref[TEI.join]{join} \hyperref[TEI.joinGrp]{joinGrp} \hyperref[TEI.link]{link} \hyperref[TEI.linkGrp]{linkGrp} \hyperref[TEI.seg]{seg} \hyperref[TEI.timeline]{timeline}\par 
    \item[msdescription: ]
   \hyperref[TEI.catchwords]{catchwords} \hyperref[TEI.depth]{depth} \hyperref[TEI.dim]{dim} \hyperref[TEI.dimensions]{dimensions} \hyperref[TEI.height]{height} \hyperref[TEI.heraldry]{heraldry} \hyperref[TEI.locus]{locus} \hyperref[TEI.locusGrp]{locusGrp} \hyperref[TEI.material]{material} \hyperref[TEI.msDesc]{msDesc} \hyperref[TEI.objectType]{objectType} \hyperref[TEI.origDate]{origDate} \hyperref[TEI.origPlace]{origPlace} \hyperref[TEI.secFol]{secFol} \hyperref[TEI.signatures]{signatures} \hyperref[TEI.source]{source} \hyperref[TEI.stamp]{stamp} \hyperref[TEI.watermark]{watermark} \hyperref[TEI.width]{width}\par 
    \item[namesdates: ]
   \hyperref[TEI.addName]{addName} \hyperref[TEI.affiliation]{affiliation} \hyperref[TEI.country]{country} \hyperref[TEI.forename]{forename} \hyperref[TEI.genName]{genName} \hyperref[TEI.geogName]{geogName} \hyperref[TEI.listOrg]{listOrg} \hyperref[TEI.listPlace]{listPlace} \hyperref[TEI.location]{location} \hyperref[TEI.nameLink]{nameLink} \hyperref[TEI.orgName]{orgName} \hyperref[TEI.persName]{persName} \hyperref[TEI.placeName]{placeName} \hyperref[TEI.region]{region} \hyperref[TEI.roleName]{roleName} \hyperref[TEI.settlement]{settlement} \hyperref[TEI.state]{state} \hyperref[TEI.surname]{surname}\par 
    \item[spoken: ]
   \hyperref[TEI.annotationBlock]{annotationBlock}\par 
    \item[textstructure: ]
   \hyperref[TEI.floatingText]{floatingText}\par 
    \item[transcr: ]
   \hyperref[TEI.addSpan]{addSpan} \hyperref[TEI.am]{am} \hyperref[TEI.damage]{damage} \hyperref[TEI.damageSpan]{damageSpan} \hyperref[TEI.delSpan]{delSpan} \hyperref[TEI.ex]{ex} \hyperref[TEI.fw]{fw} \hyperref[TEI.handShift]{handShift} \hyperref[TEI.listTranspose]{listTranspose} \hyperref[TEI.metamark]{metamark} \hyperref[TEI.mod]{mod} \hyperref[TEI.redo]{redo} \hyperref[TEI.restore]{restore} \hyperref[TEI.retrace]{retrace} \hyperref[TEI.secl]{secl} \hyperref[TEI.space]{space} \hyperref[TEI.subst]{subst} \hyperref[TEI.substJoin]{substJoin} \hyperref[TEI.supplied]{supplied} \hyperref[TEI.surplus]{surplus} \hyperref[TEI.undo]{undo}\par des données textuelles
    \item[{Exemple}]
  StandOff enrichissement entité nommée ref type="bibl"\leavevmode\bgroup\exampleFont \begin{shaded}\noindent\mbox{}{<\textbf{annotationBlock}\hspace*{6pt}{corresp}="{text}">}\mbox{}\newline 
\hspace*{6pt}{<\textbf{ref}\hspace*{6pt}{change}="{\#Unitex-3.2.0-alpha}"\mbox{}\newline 
\hspace*{6pt}\hspace*{6pt}{resp}="{istex}"\mbox{}\newline 
\hspace*{6pt}\hspace*{6pt}{scheme}="{https://refbibl-entity.data.istex.fr}"\hspace*{6pt}{type}="{bibl}">}\mbox{}\newline 
\hspace*{6pt}\hspace*{6pt}{<\textbf{term}>}Stern et al.{</\textbf{term}>}\mbox{}\newline 
\hspace*{6pt}\hspace*{6pt}{<\textbf{fs}\hspace*{6pt}{type}="{statistics}">}\mbox{}\newline 
\hspace*{6pt}\hspace*{6pt}\hspace*{6pt}{<\textbf{f}\hspace*{6pt}{name}="{frequency}">}\mbox{}\newline 
\hspace*{6pt}\hspace*{6pt}\hspace*{6pt}\hspace*{6pt}{<\textbf{numeric}\hspace*{6pt}{value}="{1}"/>}\mbox{}\newline 
\hspace*{6pt}\hspace*{6pt}\hspace*{6pt}{</\textbf{f}>}\mbox{}\newline 
\hspace*{6pt}\hspace*{6pt}{</\textbf{fs}>}\mbox{}\newline 
\hspace*{6pt}{</\textbf{ref}>}\mbox{}\newline 
{</\textbf{annotationBlock}>}\end{shaded}\egroup 


    \item[{Exemple}]
  StandOff enrichissement entité nommée ref type="url"\leavevmode\bgroup\exampleFont \begin{shaded}\noindent\mbox{}{<\textbf{annotationBlock}\hspace*{6pt}{corresp}="{text}">}\mbox{}\newline 
\hspace*{6pt}{<\textbf{ref}\hspace*{6pt}{change}="{\#Unitex-3.2.0-alpha}"\mbox{}\newline 
\hspace*{6pt}\hspace*{6pt}{resp}="{istex}"\mbox{}\newline 
\hspace*{6pt}\hspace*{6pt}{scheme}="{https://refurl-entity.data.istex.fr}"\hspace*{6pt}{type}="{url}">}\mbox{}\newline 
\hspace*{6pt}\hspace*{6pt}{<\textbf{term}>}http://www.treas.gov/press/releases/js{</\textbf{term}>}\mbox{}\newline 
\hspace*{6pt}\hspace*{6pt}{<\textbf{fs}\hspace*{6pt}{type}="{statistics}">}\mbox{}\newline 
\hspace*{6pt}\hspace*{6pt}\hspace*{6pt}{<\textbf{f}\hspace*{6pt}{name}="{frequency}">}\mbox{}\newline 
\hspace*{6pt}\hspace*{6pt}\hspace*{6pt}\hspace*{6pt}{<\textbf{numeric}\hspace*{6pt}{value}="{3}"/>}\mbox{}\newline 
\hspace*{6pt}\hspace*{6pt}\hspace*{6pt}{</\textbf{f}>}\mbox{}\newline 
\hspace*{6pt}\hspace*{6pt}{</\textbf{fs}>}\mbox{}\newline 
\hspace*{6pt}{</\textbf{ref}>}\mbox{}\newline 
{</\textbf{annotationBlock}>}\end{shaded}\egroup 


    \item[{Exemple}]
  Cas où un paragraphe contient titre de revue/livre + sa tomaison \leavevmode\bgroup\exampleFont \begin{shaded}\noindent\mbox{}{<\textbf{p}>}Originally published as {<\textbf{ref}\hspace*{6pt}{type}="{cit}"\hspace*{6pt}{xml:id}="{n1}">}\mbox{}\newline 
\hspace*{6pt}\hspace*{6pt}{<\textbf{title}\hspace*{6pt}{level}="{j}"\hspace*{6pt}{type}="{main}">}Nature{</\textbf{title}>}\mbox{}\newline 
\hspace*{6pt}\hspace*{6pt}{<\textbf{biblScope}\hspace*{6pt}{unit}="{vol}">}402{</\textbf{biblScope}>}, {<\textbf{biblScope}\hspace*{6pt}{from}="{ 255}"\hspace*{6pt}{unit}="{page}">} 255{</\textbf{biblScope}>}ndash;{<\textbf{biblScope}\hspace*{6pt}{to}="{262}"\hspace*{6pt}{unit}="{page}">}262{</\textbf{biblScope}>}; {</\textbf{ref}>}\mbox{}\newline 
{</\textbf{p}>}\end{shaded}\egroup 


    \item[{Schematron}]
   <s:report test="@target and @cRef">Only one of the  attributes @target' and @cRef' may be supplied on <s:name/> </s:report>
    \item[{Modèle de contenu}]
  \mbox{}\hfill\\[-10pt]\begin{Verbatim}[fontsize=\small]
<content>
 <macroRef key="macro.paraContent"/>
</content>
    
\end{Verbatim}

    \item[{Schéma Declaration}]
  \mbox{}\hfill\\[-10pt]\begin{Verbatim}[fontsize=\small]
element ref
{
   tei_att.global.attributes,
   tei_att.pointing.attributes,
   tei_att.internetMedia.attributes,
   tei_att.typed.attributes,
   tei_att.declaring.attributes,
   tei_att.cReferencing.attributes,
   attribute scheme { text }?,
   tei_macro.paraContent}
\end{Verbatim}

\end{reflist}  \index{reg=<reg>|oddindex}
\begin{reflist}
\item[]\begin{specHead}{TEI.reg}{<reg> }(régularisation) contient une partie qui a été régularisée ou normalisée de façon quelconque [\xref{http://www.tei-c.org/release/doc/tei-p5-doc/en/html/CO.html\#COEDREG}{3.4.2. Regularization and Normalization} \xref{http://www.tei-c.org/release/doc/tei-p5-doc/en/html/TC.html\#TC}{12. Critical Apparatus}]\end{specHead} 
    \item[{Module}]
  core
    \item[{Attributs}]
  Attributs \hyperref[TEI.att.global]{att.global} (\textit{@xml:id}, \textit{@n}, \textit{@xml:lang}, \textit{@xml:base}, \textit{@xml:space})  (\hyperref[TEI.att.global.rendition]{att.global.rendition} (\textit{@rend}, \textit{@style}, \textit{@rendition})) (\hyperref[TEI.att.global.linking]{att.global.linking} (\textit{@corresp}, \textit{@synch}, \textit{@sameAs}, \textit{@copyOf}, \textit{@next}, \textit{@prev}, \textit{@exclude}, \textit{@select})) (\hyperref[TEI.att.global.analytic]{att.global.analytic} (\textit{@ana})) (\hyperref[TEI.att.global.facs]{att.global.facs} (\textit{@facs})) (\hyperref[TEI.att.global.change]{att.global.change} (\textit{@change})) (\hyperref[TEI.att.global.responsibility]{att.global.responsibility} (\textit{@cert}, \textit{@resp})) (\hyperref[TEI.att.global.source]{att.global.source} (\textit{@source})) \hyperref[TEI.att.editLike]{att.editLike} (\textit{@evidence}, \textit{@instant})  (\hyperref[TEI.att.dimensions]{att.dimensions} (\textit{@unit}, \textit{@quantity}, \textit{@extent}, \textit{@precision}, \textit{@scope}) (\hyperref[TEI.att.ranging]{att.ranging} (\textit{@atLeast}, \textit{@atMost}, \textit{@min}, \textit{@max}, \textit{@confidence})) ) \hyperref[TEI.att.typed]{att.typed} (\textit{@type}, \textit{@subtype}) 
    \item[{Membre du}]
  \hyperref[TEI.model.choicePart]{model.choicePart} \hyperref[TEI.model.pPart.transcriptional]{model.pPart.transcriptional}
    \item[{Contenu dans}]
  
    \item[analysis: ]
   \hyperref[TEI.cl]{cl} \hyperref[TEI.pc]{pc} \hyperref[TEI.phr]{phr} \hyperref[TEI.s]{s} \hyperref[TEI.w]{w}\par 
    \item[core: ]
   \hyperref[TEI.abbr]{abbr} \hyperref[TEI.add]{add} \hyperref[TEI.addrLine]{addrLine} \hyperref[TEI.author]{author} \hyperref[TEI.bibl]{bibl} \hyperref[TEI.biblScope]{biblScope} \hyperref[TEI.choice]{choice} \hyperref[TEI.citedRange]{citedRange} \hyperref[TEI.corr]{corr} \hyperref[TEI.date]{date} \hyperref[TEI.del]{del} \hyperref[TEI.distinct]{distinct} \hyperref[TEI.editor]{editor} \hyperref[TEI.email]{email} \hyperref[TEI.emph]{emph} \hyperref[TEI.expan]{expan} \hyperref[TEI.foreign]{foreign} \hyperref[TEI.gloss]{gloss} \hyperref[TEI.head]{head} \hyperref[TEI.headItem]{headItem} \hyperref[TEI.headLabel]{headLabel} \hyperref[TEI.hi]{hi} \hyperref[TEI.item]{item} \hyperref[TEI.l]{l} \hyperref[TEI.label]{label} \hyperref[TEI.measure]{measure} \hyperref[TEI.mentioned]{mentioned} \hyperref[TEI.name]{name} \hyperref[TEI.note]{note} \hyperref[TEI.num]{num} \hyperref[TEI.orig]{orig} \hyperref[TEI.p]{p} \hyperref[TEI.pubPlace]{pubPlace} \hyperref[TEI.publisher]{publisher} \hyperref[TEI.q]{q} \hyperref[TEI.quote]{quote} \hyperref[TEI.ref]{ref} \hyperref[TEI.reg]{reg} \hyperref[TEI.rs]{rs} \hyperref[TEI.said]{said} \hyperref[TEI.sic]{sic} \hyperref[TEI.soCalled]{soCalled} \hyperref[TEI.speaker]{speaker} \hyperref[TEI.stage]{stage} \hyperref[TEI.street]{street} \hyperref[TEI.term]{term} \hyperref[TEI.textLang]{textLang} \hyperref[TEI.time]{time} \hyperref[TEI.title]{title} \hyperref[TEI.unclear]{unclear}\par 
    \item[figures: ]
   \hyperref[TEI.cell]{cell}\par 
    \item[header: ]
   \hyperref[TEI.change]{change} \hyperref[TEI.distributor]{distributor} \hyperref[TEI.edition]{edition} \hyperref[TEI.extent]{extent} \hyperref[TEI.licence]{licence}\par 
    \item[linking: ]
   \hyperref[TEI.ab]{ab} \hyperref[TEI.seg]{seg}\par 
    \item[msdescription: ]
   \hyperref[TEI.accMat]{accMat} \hyperref[TEI.acquisition]{acquisition} \hyperref[TEI.additions]{additions} \hyperref[TEI.catchwords]{catchwords} \hyperref[TEI.collation]{collation} \hyperref[TEI.colophon]{colophon} \hyperref[TEI.condition]{condition} \hyperref[TEI.custEvent]{custEvent} \hyperref[TEI.decoNote]{decoNote} \hyperref[TEI.explicit]{explicit} \hyperref[TEI.filiation]{filiation} \hyperref[TEI.finalRubric]{finalRubric} \hyperref[TEI.foliation]{foliation} \hyperref[TEI.heraldry]{heraldry} \hyperref[TEI.incipit]{incipit} \hyperref[TEI.layout]{layout} \hyperref[TEI.material]{material} \hyperref[TEI.musicNotation]{musicNotation} \hyperref[TEI.objectType]{objectType} \hyperref[TEI.origDate]{origDate} \hyperref[TEI.origPlace]{origPlace} \hyperref[TEI.origin]{origin} \hyperref[TEI.provenance]{provenance} \hyperref[TEI.rubric]{rubric} \hyperref[TEI.secFol]{secFol} \hyperref[TEI.signatures]{signatures} \hyperref[TEI.source]{source} \hyperref[TEI.stamp]{stamp} \hyperref[TEI.summary]{summary} \hyperref[TEI.support]{support} \hyperref[TEI.surrogates]{surrogates} \hyperref[TEI.typeNote]{typeNote} \hyperref[TEI.watermark]{watermark}\par 
    \item[namesdates: ]
   \hyperref[TEI.addName]{addName} \hyperref[TEI.affiliation]{affiliation} \hyperref[TEI.country]{country} \hyperref[TEI.forename]{forename} \hyperref[TEI.genName]{genName} \hyperref[TEI.geogName]{geogName} \hyperref[TEI.nameLink]{nameLink} \hyperref[TEI.orgName]{orgName} \hyperref[TEI.persName]{persName} \hyperref[TEI.placeName]{placeName} \hyperref[TEI.region]{region} \hyperref[TEI.roleName]{roleName} \hyperref[TEI.settlement]{settlement} \hyperref[TEI.surname]{surname}\par 
    \item[textstructure: ]
   \hyperref[TEI.docAuthor]{docAuthor} \hyperref[TEI.docDate]{docDate} \hyperref[TEI.docEdition]{docEdition} \hyperref[TEI.titlePart]{titlePart}\par 
    \item[transcr: ]
   \hyperref[TEI.am]{am} \hyperref[TEI.damage]{damage} \hyperref[TEI.fw]{fw} \hyperref[TEI.metamark]{metamark} \hyperref[TEI.mod]{mod} \hyperref[TEI.restore]{restore} \hyperref[TEI.retrace]{retrace} \hyperref[TEI.secl]{secl} \hyperref[TEI.supplied]{supplied} \hyperref[TEI.surplus]{surplus}
    \item[{Peut contenir}]
  
    \item[analysis: ]
   \hyperref[TEI.c]{c} \hyperref[TEI.cl]{cl} \hyperref[TEI.interp]{interp} \hyperref[TEI.interpGrp]{interpGrp} \hyperref[TEI.m]{m} \hyperref[TEI.pc]{pc} \hyperref[TEI.phr]{phr} \hyperref[TEI.s]{s} \hyperref[TEI.span]{span} \hyperref[TEI.spanGrp]{spanGrp} \hyperref[TEI.w]{w}\par 
    \item[core: ]
   \hyperref[TEI.abbr]{abbr} \hyperref[TEI.add]{add} \hyperref[TEI.address]{address} \hyperref[TEI.bibl]{bibl} \hyperref[TEI.biblStruct]{biblStruct} \hyperref[TEI.binaryObject]{binaryObject} \hyperref[TEI.cb]{cb} \hyperref[TEI.choice]{choice} \hyperref[TEI.cit]{cit} \hyperref[TEI.corr]{corr} \hyperref[TEI.date]{date} \hyperref[TEI.del]{del} \hyperref[TEI.desc]{desc} \hyperref[TEI.distinct]{distinct} \hyperref[TEI.email]{email} \hyperref[TEI.emph]{emph} \hyperref[TEI.expan]{expan} \hyperref[TEI.foreign]{foreign} \hyperref[TEI.gap]{gap} \hyperref[TEI.gb]{gb} \hyperref[TEI.gloss]{gloss} \hyperref[TEI.graphic]{graphic} \hyperref[TEI.hi]{hi} \hyperref[TEI.index]{index} \hyperref[TEI.l]{l} \hyperref[TEI.label]{label} \hyperref[TEI.lb]{lb} \hyperref[TEI.lg]{lg} \hyperref[TEI.list]{list} \hyperref[TEI.listBibl]{listBibl} \hyperref[TEI.measure]{measure} \hyperref[TEI.measureGrp]{measureGrp} \hyperref[TEI.media]{media} \hyperref[TEI.mentioned]{mentioned} \hyperref[TEI.milestone]{milestone} \hyperref[TEI.name]{name} \hyperref[TEI.note]{note} \hyperref[TEI.num]{num} \hyperref[TEI.orig]{orig} \hyperref[TEI.pb]{pb} \hyperref[TEI.ptr]{ptr} \hyperref[TEI.q]{q} \hyperref[TEI.quote]{quote} \hyperref[TEI.ref]{ref} \hyperref[TEI.reg]{reg} \hyperref[TEI.rs]{rs} \hyperref[TEI.said]{said} \hyperref[TEI.sic]{sic} \hyperref[TEI.soCalled]{soCalled} \hyperref[TEI.stage]{stage} \hyperref[TEI.term]{term} \hyperref[TEI.time]{time} \hyperref[TEI.title]{title} \hyperref[TEI.unclear]{unclear}\par 
    \item[derived-module-tei.istex: ]
   \hyperref[TEI.math]{math} \hyperref[TEI.mrow]{mrow}\par 
    \item[figures: ]
   \hyperref[TEI.figure]{figure} \hyperref[TEI.formula]{formula} \hyperref[TEI.notatedMusic]{notatedMusic} \hyperref[TEI.table]{table}\par 
    \item[header: ]
   \hyperref[TEI.biblFull]{biblFull} \hyperref[TEI.idno]{idno}\par 
    \item[iso-fs: ]
   \hyperref[TEI.fLib]{fLib} \hyperref[TEI.fs]{fs} \hyperref[TEI.fvLib]{fvLib}\par 
    \item[linking: ]
   \hyperref[TEI.alt]{alt} \hyperref[TEI.altGrp]{altGrp} \hyperref[TEI.anchor]{anchor} \hyperref[TEI.join]{join} \hyperref[TEI.joinGrp]{joinGrp} \hyperref[TEI.link]{link} \hyperref[TEI.linkGrp]{linkGrp} \hyperref[TEI.seg]{seg} \hyperref[TEI.timeline]{timeline}\par 
    \item[msdescription: ]
   \hyperref[TEI.catchwords]{catchwords} \hyperref[TEI.depth]{depth} \hyperref[TEI.dim]{dim} \hyperref[TEI.dimensions]{dimensions} \hyperref[TEI.height]{height} \hyperref[TEI.heraldry]{heraldry} \hyperref[TEI.locus]{locus} \hyperref[TEI.locusGrp]{locusGrp} \hyperref[TEI.material]{material} \hyperref[TEI.msDesc]{msDesc} \hyperref[TEI.objectType]{objectType} \hyperref[TEI.origDate]{origDate} \hyperref[TEI.origPlace]{origPlace} \hyperref[TEI.secFol]{secFol} \hyperref[TEI.signatures]{signatures} \hyperref[TEI.source]{source} \hyperref[TEI.stamp]{stamp} \hyperref[TEI.watermark]{watermark} \hyperref[TEI.width]{width}\par 
    \item[namesdates: ]
   \hyperref[TEI.addName]{addName} \hyperref[TEI.affiliation]{affiliation} \hyperref[TEI.country]{country} \hyperref[TEI.forename]{forename} \hyperref[TEI.genName]{genName} \hyperref[TEI.geogName]{geogName} \hyperref[TEI.listOrg]{listOrg} \hyperref[TEI.listPlace]{listPlace} \hyperref[TEI.location]{location} \hyperref[TEI.nameLink]{nameLink} \hyperref[TEI.orgName]{orgName} \hyperref[TEI.persName]{persName} \hyperref[TEI.placeName]{placeName} \hyperref[TEI.region]{region} \hyperref[TEI.roleName]{roleName} \hyperref[TEI.settlement]{settlement} \hyperref[TEI.state]{state} \hyperref[TEI.surname]{surname}\par 
    \item[spoken: ]
   \hyperref[TEI.annotationBlock]{annotationBlock}\par 
    \item[textstructure: ]
   \hyperref[TEI.floatingText]{floatingText}\par 
    \item[transcr: ]
   \hyperref[TEI.addSpan]{addSpan} \hyperref[TEI.am]{am} \hyperref[TEI.damage]{damage} \hyperref[TEI.damageSpan]{damageSpan} \hyperref[TEI.delSpan]{delSpan} \hyperref[TEI.ex]{ex} \hyperref[TEI.fw]{fw} \hyperref[TEI.handShift]{handShift} \hyperref[TEI.listTranspose]{listTranspose} \hyperref[TEI.metamark]{metamark} \hyperref[TEI.mod]{mod} \hyperref[TEI.redo]{redo} \hyperref[TEI.restore]{restore} \hyperref[TEI.retrace]{retrace} \hyperref[TEI.secl]{secl} \hyperref[TEI.space]{space} \hyperref[TEI.subst]{subst} \hyperref[TEI.substJoin]{substJoin} \hyperref[TEI.supplied]{supplied} \hyperref[TEI.surplus]{surplus} \hyperref[TEI.undo]{undo}\par des données textuelles
    \item[{Exemple}]
  Si on veut attirer l'attention sur le fait que le texte a été régularisé, \hyperref[TEI.reg]{<reg>} est utilisé seul :\leavevmode\bgroup\exampleFont \begin{shaded}\noindent\mbox{}{<\textbf{l}>}\mbox{}\newline 
\hspace*{6pt}{<\textbf{reg}>}Maître{</\textbf{reg}>} Corbeau sur un arbre perché,\mbox{}\newline 
{</\textbf{l}>}\mbox{}\newline 
{<\textbf{l}>}\mbox{}\newline 
\hspace*{6pt}{<\textbf{reg}>}Tenait{</\textbf{reg}>} en son bec un fromage.\mbox{}\newline 
{</\textbf{l}>}\end{shaded}\egroup 


    \item[{Exemple}]
  Il est également possible d'identifier l'auteur de la régularisation, et avec les éléments \hyperref[TEI.choice]{<choice>} et\hyperref[TEI.orig]{<orig>}, donner à la fois la lecture originale et la lecture régularisée.:\leavevmode\bgroup\exampleFont \begin{shaded}\noindent\mbox{}{<\textbf{l}>}\mbox{}\newline 
\hspace*{6pt}{<\textbf{choice}>}\mbox{}\newline 
\hspace*{6pt}\hspace*{6pt}{<\textbf{orig}>}Maistre{</\textbf{orig}>}\mbox{}\newline 
\hspace*{6pt}\hspace*{6pt}{<\textbf{reg}\hspace*{6pt}{resp}="{\#LB}">}Maître{</\textbf{reg}>}\mbox{}\newline 
\hspace*{6pt}{</\textbf{choice}>}Corbeau sur un arbre perché,\mbox{}\newline 
{</\textbf{l}>}\mbox{}\newline 
{<\textbf{l}>}\mbox{}\newline 
\hspace*{6pt}{<\textbf{choice}>}\mbox{}\newline 
\hspace*{6pt}\hspace*{6pt}{<\textbf{orig}>}Tenoit{</\textbf{orig}>}\mbox{}\newline 
\hspace*{6pt}\hspace*{6pt}{<\textbf{reg}\hspace*{6pt}{resp}="{\#LB}">}Tenait{</\textbf{reg}>}\mbox{}\newline 
\hspace*{6pt}{</\textbf{choice}>} en son bec un fromage.\mbox{}\newline 
{</\textbf{l}>}\end{shaded}\egroup 


    \item[{Modèle de contenu}]
  \mbox{}\hfill\\[-10pt]\begin{Verbatim}[fontsize=\small]
<content>
 <macroRef key="macro.paraContent"/>
</content>
    
\end{Verbatim}

    \item[{Schéma Declaration}]
  \mbox{}\hfill\\[-10pt]\begin{Verbatim}[fontsize=\small]
element reg
{
   tei_att.global.attributes,
   tei_att.editLike.attributes,
   tei_att.typed.attributes,
   tei_macro.paraContent}
\end{Verbatim}

\end{reflist}  \index{region=<region>|oddindex}
\begin{reflist}
\item[]\begin{specHead}{TEI.region}{<region> }(région) contient le nom d'une unité administrative comme un état, une province ou un comté, plus grande qu'un lieu de peuplement, mais plus petite qu'un pays. [\xref{http://www.tei-c.org/release/doc/tei-p5-doc/en/html/ND.html\#NDPLAC}{13.2.3. Place Names}]\end{specHead} 
    \item[{Module}]
  namesdates
    \item[{Attributs}]
  Attributs \hyperref[TEI.att.global]{att.global} (\textit{@xml:id}, \textit{@n}, \textit{@xml:lang}, \textit{@xml:base}, \textit{@xml:space})  (\hyperref[TEI.att.global.rendition]{att.global.rendition} (\textit{@rend}, \textit{@style}, \textit{@rendition})) (\hyperref[TEI.att.global.linking]{att.global.linking} (\textit{@corresp}, \textit{@synch}, \textit{@sameAs}, \textit{@copyOf}, \textit{@next}, \textit{@prev}, \textit{@exclude}, \textit{@select})) (\hyperref[TEI.att.global.analytic]{att.global.analytic} (\textit{@ana})) (\hyperref[TEI.att.global.facs]{att.global.facs} (\textit{@facs})) (\hyperref[TEI.att.global.change]{att.global.change} (\textit{@change})) (\hyperref[TEI.att.global.responsibility]{att.global.responsibility} (\textit{@cert}, \textit{@resp})) (\hyperref[TEI.att.global.source]{att.global.source} (\textit{@source})) \hyperref[TEI.att.naming]{att.naming} (\textit{@role}, \textit{@nymRef})  (\hyperref[TEI.att.canonical]{att.canonical} (\textit{@key}, \textit{@ref})) \hyperref[TEI.att.typed]{att.typed} (\textit{@type}, \textit{@subtype}) \hyperref[TEI.att.datable]{att.datable} (\textit{@calendar}, \textit{@period})  (\hyperref[TEI.att.datable.w3c]{att.datable.w3c} (\textit{@when}, \textit{@notBefore}, \textit{@notAfter}, \textit{@from}, \textit{@to})) (\hyperref[TEI.att.datable.iso]{att.datable.iso} (\textit{@when-iso}, \textit{@notBefore-iso}, \textit{@notAfter-iso}, \textit{@from-iso}, \textit{@to-iso})) (\hyperref[TEI.att.datable.custom]{att.datable.custom} (\textit{@when-custom}, \textit{@notBefore-custom}, \textit{@notAfter-custom}, \textit{@from-custom}, \textit{@to-custom}, \textit{@datingPoint}, \textit{@datingMethod}))
    \item[{Membre du}]
  \hyperref[TEI.model.placeNamePart]{model.placeNamePart}
    \item[{Contenu dans}]
  
    \item[analysis: ]
   \hyperref[TEI.cl]{cl} \hyperref[TEI.phr]{phr} \hyperref[TEI.s]{s} \hyperref[TEI.span]{span}\par 
    \item[core: ]
   \hyperref[TEI.abbr]{abbr} \hyperref[TEI.add]{add} \hyperref[TEI.addrLine]{addrLine} \hyperref[TEI.address]{address} \hyperref[TEI.author]{author} \hyperref[TEI.bibl]{bibl} \hyperref[TEI.biblScope]{biblScope} \hyperref[TEI.citedRange]{citedRange} \hyperref[TEI.corr]{corr} \hyperref[TEI.date]{date} \hyperref[TEI.del]{del} \hyperref[TEI.desc]{desc} \hyperref[TEI.distinct]{distinct} \hyperref[TEI.editor]{editor} \hyperref[TEI.email]{email} \hyperref[TEI.emph]{emph} \hyperref[TEI.expan]{expan} \hyperref[TEI.foreign]{foreign} \hyperref[TEI.gloss]{gloss} \hyperref[TEI.head]{head} \hyperref[TEI.headItem]{headItem} \hyperref[TEI.headLabel]{headLabel} \hyperref[TEI.hi]{hi} \hyperref[TEI.item]{item} \hyperref[TEI.l]{l} \hyperref[TEI.label]{label} \hyperref[TEI.measure]{measure} \hyperref[TEI.meeting]{meeting} \hyperref[TEI.mentioned]{mentioned} \hyperref[TEI.name]{name} \hyperref[TEI.note]{note} \hyperref[TEI.num]{num} \hyperref[TEI.orig]{orig} \hyperref[TEI.p]{p} \hyperref[TEI.pubPlace]{pubPlace} \hyperref[TEI.publisher]{publisher} \hyperref[TEI.q]{q} \hyperref[TEI.quote]{quote} \hyperref[TEI.ref]{ref} \hyperref[TEI.reg]{reg} \hyperref[TEI.resp]{resp} \hyperref[TEI.rs]{rs} \hyperref[TEI.said]{said} \hyperref[TEI.sic]{sic} \hyperref[TEI.soCalled]{soCalled} \hyperref[TEI.speaker]{speaker} \hyperref[TEI.stage]{stage} \hyperref[TEI.street]{street} \hyperref[TEI.term]{term} \hyperref[TEI.textLang]{textLang} \hyperref[TEI.time]{time} \hyperref[TEI.title]{title} \hyperref[TEI.unclear]{unclear}\par 
    \item[figures: ]
   \hyperref[TEI.cell]{cell} \hyperref[TEI.figDesc]{figDesc}\par 
    \item[header: ]
   \hyperref[TEI.authority]{authority} \hyperref[TEI.change]{change} \hyperref[TEI.classCode]{classCode} \hyperref[TEI.creation]{creation} \hyperref[TEI.distributor]{distributor} \hyperref[TEI.edition]{edition} \hyperref[TEI.extent]{extent} \hyperref[TEI.funder]{funder} \hyperref[TEI.language]{language} \hyperref[TEI.licence]{licence} \hyperref[TEI.rendition]{rendition}\par 
    \item[iso-fs: ]
   \hyperref[TEI.fDescr]{fDescr} \hyperref[TEI.fsDescr]{fsDescr}\par 
    \item[linking: ]
   \hyperref[TEI.ab]{ab} \hyperref[TEI.seg]{seg}\par 
    \item[msdescription: ]
   \hyperref[TEI.accMat]{accMat} \hyperref[TEI.acquisition]{acquisition} \hyperref[TEI.additions]{additions} \hyperref[TEI.altIdentifier]{altIdentifier} \hyperref[TEI.catchwords]{catchwords} \hyperref[TEI.collation]{collation} \hyperref[TEI.colophon]{colophon} \hyperref[TEI.condition]{condition} \hyperref[TEI.custEvent]{custEvent} \hyperref[TEI.decoNote]{decoNote} \hyperref[TEI.explicit]{explicit} \hyperref[TEI.filiation]{filiation} \hyperref[TEI.finalRubric]{finalRubric} \hyperref[TEI.foliation]{foliation} \hyperref[TEI.heraldry]{heraldry} \hyperref[TEI.incipit]{incipit} \hyperref[TEI.layout]{layout} \hyperref[TEI.material]{material} \hyperref[TEI.msIdentifier]{msIdentifier} \hyperref[TEI.musicNotation]{musicNotation} \hyperref[TEI.objectType]{objectType} \hyperref[TEI.origDate]{origDate} \hyperref[TEI.origPlace]{origPlace} \hyperref[TEI.origin]{origin} \hyperref[TEI.provenance]{provenance} \hyperref[TEI.rubric]{rubric} \hyperref[TEI.secFol]{secFol} \hyperref[TEI.signatures]{signatures} \hyperref[TEI.source]{source} \hyperref[TEI.stamp]{stamp} \hyperref[TEI.summary]{summary} \hyperref[TEI.support]{support} \hyperref[TEI.surrogates]{surrogates} \hyperref[TEI.typeNote]{typeNote} \hyperref[TEI.watermark]{watermark}\par 
    \item[namesdates: ]
   \hyperref[TEI.addName]{addName} \hyperref[TEI.affiliation]{affiliation} \hyperref[TEI.country]{country} \hyperref[TEI.forename]{forename} \hyperref[TEI.genName]{genName} \hyperref[TEI.geogName]{geogName} \hyperref[TEI.location]{location} \hyperref[TEI.nameLink]{nameLink} \hyperref[TEI.org]{org} \hyperref[TEI.orgName]{orgName} \hyperref[TEI.persName]{persName} \hyperref[TEI.place]{place} \hyperref[TEI.placeName]{placeName} \hyperref[TEI.region]{region} \hyperref[TEI.roleName]{roleName} \hyperref[TEI.settlement]{settlement} \hyperref[TEI.surname]{surname}\par 
    \item[spoken: ]
   \hyperref[TEI.annotationBlock]{annotationBlock}\par 
    \item[standOff: ]
   \hyperref[TEI.listAnnotation]{listAnnotation}\par 
    \item[textstructure: ]
   \hyperref[TEI.docAuthor]{docAuthor} \hyperref[TEI.docDate]{docDate} \hyperref[TEI.docEdition]{docEdition} \hyperref[TEI.titlePart]{titlePart}\par 
    \item[transcr: ]
   \hyperref[TEI.damage]{damage} \hyperref[TEI.fw]{fw} \hyperref[TEI.metamark]{metamark} \hyperref[TEI.mod]{mod} \hyperref[TEI.restore]{restore} \hyperref[TEI.retrace]{retrace} \hyperref[TEI.secl]{secl} \hyperref[TEI.supplied]{supplied} \hyperref[TEI.surplus]{surplus}
    \item[{Peut contenir}]
  
    \item[analysis: ]
   \hyperref[TEI.c]{c} \hyperref[TEI.cl]{cl} \hyperref[TEI.interp]{interp} \hyperref[TEI.interpGrp]{interpGrp} \hyperref[TEI.m]{m} \hyperref[TEI.pc]{pc} \hyperref[TEI.phr]{phr} \hyperref[TEI.s]{s} \hyperref[TEI.span]{span} \hyperref[TEI.spanGrp]{spanGrp} \hyperref[TEI.w]{w}\par 
    \item[core: ]
   \hyperref[TEI.abbr]{abbr} \hyperref[TEI.add]{add} \hyperref[TEI.address]{address} \hyperref[TEI.binaryObject]{binaryObject} \hyperref[TEI.cb]{cb} \hyperref[TEI.choice]{choice} \hyperref[TEI.corr]{corr} \hyperref[TEI.date]{date} \hyperref[TEI.del]{del} \hyperref[TEI.distinct]{distinct} \hyperref[TEI.email]{email} \hyperref[TEI.emph]{emph} \hyperref[TEI.expan]{expan} \hyperref[TEI.foreign]{foreign} \hyperref[TEI.gap]{gap} \hyperref[TEI.gb]{gb} \hyperref[TEI.gloss]{gloss} \hyperref[TEI.graphic]{graphic} \hyperref[TEI.hi]{hi} \hyperref[TEI.index]{index} \hyperref[TEI.lb]{lb} \hyperref[TEI.measure]{measure} \hyperref[TEI.measureGrp]{measureGrp} \hyperref[TEI.media]{media} \hyperref[TEI.mentioned]{mentioned} \hyperref[TEI.milestone]{milestone} \hyperref[TEI.name]{name} \hyperref[TEI.note]{note} \hyperref[TEI.num]{num} \hyperref[TEI.orig]{orig} \hyperref[TEI.pb]{pb} \hyperref[TEI.ptr]{ptr} \hyperref[TEI.ref]{ref} \hyperref[TEI.reg]{reg} \hyperref[TEI.rs]{rs} \hyperref[TEI.sic]{sic} \hyperref[TEI.soCalled]{soCalled} \hyperref[TEI.term]{term} \hyperref[TEI.time]{time} \hyperref[TEI.title]{title} \hyperref[TEI.unclear]{unclear}\par 
    \item[derived-module-tei.istex: ]
   \hyperref[TEI.math]{math} \hyperref[TEI.mrow]{mrow}\par 
    \item[figures: ]
   \hyperref[TEI.figure]{figure} \hyperref[TEI.formula]{formula} \hyperref[TEI.notatedMusic]{notatedMusic}\par 
    \item[header: ]
   \hyperref[TEI.idno]{idno}\par 
    \item[iso-fs: ]
   \hyperref[TEI.fLib]{fLib} \hyperref[TEI.fs]{fs} \hyperref[TEI.fvLib]{fvLib}\par 
    \item[linking: ]
   \hyperref[TEI.alt]{alt} \hyperref[TEI.altGrp]{altGrp} \hyperref[TEI.anchor]{anchor} \hyperref[TEI.join]{join} \hyperref[TEI.joinGrp]{joinGrp} \hyperref[TEI.link]{link} \hyperref[TEI.linkGrp]{linkGrp} \hyperref[TEI.seg]{seg} \hyperref[TEI.timeline]{timeline}\par 
    \item[msdescription: ]
   \hyperref[TEI.catchwords]{catchwords} \hyperref[TEI.depth]{depth} \hyperref[TEI.dim]{dim} \hyperref[TEI.dimensions]{dimensions} \hyperref[TEI.height]{height} \hyperref[TEI.heraldry]{heraldry} \hyperref[TEI.locus]{locus} \hyperref[TEI.locusGrp]{locusGrp} \hyperref[TEI.material]{material} \hyperref[TEI.objectType]{objectType} \hyperref[TEI.origDate]{origDate} \hyperref[TEI.origPlace]{origPlace} \hyperref[TEI.secFol]{secFol} \hyperref[TEI.signatures]{signatures} \hyperref[TEI.source]{source} \hyperref[TEI.stamp]{stamp} \hyperref[TEI.watermark]{watermark} \hyperref[TEI.width]{width}\par 
    \item[namesdates: ]
   \hyperref[TEI.addName]{addName} \hyperref[TEI.affiliation]{affiliation} \hyperref[TEI.country]{country} \hyperref[TEI.forename]{forename} \hyperref[TEI.genName]{genName} \hyperref[TEI.geogName]{geogName} \hyperref[TEI.location]{location} \hyperref[TEI.nameLink]{nameLink} \hyperref[TEI.orgName]{orgName} \hyperref[TEI.persName]{persName} \hyperref[TEI.placeName]{placeName} \hyperref[TEI.region]{region} \hyperref[TEI.roleName]{roleName} \hyperref[TEI.settlement]{settlement} \hyperref[TEI.state]{state} \hyperref[TEI.surname]{surname}\par 
    \item[spoken: ]
   \hyperref[TEI.annotationBlock]{annotationBlock}\par 
    \item[transcr: ]
   \hyperref[TEI.addSpan]{addSpan} \hyperref[TEI.am]{am} \hyperref[TEI.damage]{damage} \hyperref[TEI.damageSpan]{damageSpan} \hyperref[TEI.delSpan]{delSpan} \hyperref[TEI.ex]{ex} \hyperref[TEI.fw]{fw} \hyperref[TEI.handShift]{handShift} \hyperref[TEI.listTranspose]{listTranspose} \hyperref[TEI.metamark]{metamark} \hyperref[TEI.mod]{mod} \hyperref[TEI.redo]{redo} \hyperref[TEI.restore]{restore} \hyperref[TEI.retrace]{retrace} \hyperref[TEI.secl]{secl} \hyperref[TEI.space]{space} \hyperref[TEI.subst]{subst} \hyperref[TEI.substJoin]{substJoin} \hyperref[TEI.supplied]{supplied} \hyperref[TEI.surplus]{surplus} \hyperref[TEI.undo]{undo}\par des données textuelles
    \item[{Exemple}]
  \leavevmode\bgroup\exampleFont \begin{shaded}\noindent\mbox{}{<\textbf{placeName}>}\mbox{}\newline 
\hspace*{6pt}{<\textbf{region}\hspace*{6pt}{n}="{IL}"\hspace*{6pt}{type}="{state}">}Illinois{</\textbf{region}>}\mbox{}\newline 
{</\textbf{placeName}>}\end{shaded}\egroup 


    \item[{Exemple}]
  \leavevmode\bgroup\exampleFont \begin{shaded}\noindent\mbox{}{<\textbf{placeName}>}\mbox{}\newline 
\hspace*{6pt}{<\textbf{region}\hspace*{6pt}{n}="{IL}"\hspace*{6pt}{type}="{state}">}Illinois{</\textbf{region}>}\mbox{}\newline 
{</\textbf{placeName}>}\end{shaded}\egroup 


    \item[{Modèle de contenu}]
  \mbox{}\hfill\\[-10pt]\begin{Verbatim}[fontsize=\small]
<content>
 <macroRef key="macro.phraseSeq"/>
</content>
    
\end{Verbatim}

    \item[{Schéma Declaration}]
  \mbox{}\hfill\\[-10pt]\begin{Verbatim}[fontsize=\small]
element region
{
   tei_att.global.attributes,
   tei_att.naming.attributes,
   tei_att.typed.attributes,
   tei_att.datable.attributes,
   tei_macro.phraseSeq}
\end{Verbatim}

\end{reflist}  \index{relatedItem=<relatedItem>|oddindex}\index{target=@target!<relatedItem>|oddindex}
\begin{reflist}
\item[]\begin{specHead}{TEI.relatedItem}{<relatedItem> }contient ou référe à un autre élément bibliographique ayant une relation quelconque avec l'objet décrit, par exemple comme faisant partie d'une version alternative de celui-ci, ou bien en étant une version alternative. [\xref{http://www.tei-c.org/release/doc/tei-p5-doc/en/html/CO.html\#COBIRI}{3.11.2.7. Related Items}]\end{specHead} 
    \item[{Module}]
  core
    \item[{Attributs}]
  Attributs \hyperref[TEI.att.global]{att.global} (\textit{@xml:id}, \textit{@n}, \textit{@xml:lang}, \textit{@xml:base}, \textit{@xml:space})  (\hyperref[TEI.att.global.rendition]{att.global.rendition} (\textit{@rend}, \textit{@style}, \textit{@rendition})) (\hyperref[TEI.att.global.linking]{att.global.linking} (\textit{@corresp}, \textit{@synch}, \textit{@sameAs}, \textit{@copyOf}, \textit{@next}, \textit{@prev}, \textit{@exclude}, \textit{@select})) (\hyperref[TEI.att.global.analytic]{att.global.analytic} (\textit{@ana})) (\hyperref[TEI.att.global.facs]{att.global.facs} (\textit{@facs})) (\hyperref[TEI.att.global.change]{att.global.change} (\textit{@change})) (\hyperref[TEI.att.global.responsibility]{att.global.responsibility} (\textit{@cert}, \textit{@resp})) (\hyperref[TEI.att.global.source]{att.global.source} (\textit{@source})) \hyperref[TEI.att.typed]{att.typed} (\textit{@type}, \textit{@subtype}) \hfil\\[-10pt]\begin{sansreflist}
    \item[@target]
  points to the related bibliographic element by means of an absolute or relative URI reference
\begin{reflist}
    \item[{Statut}]
  Optionel
    \item[{Type de données}]
  \hyperref[TEI.teidata.pointer]{teidata.pointer}
\end{reflist}  
\end{sansreflist}  
    \item[{Membre du}]
  \hyperref[TEI.model.biblPart]{model.biblPart} 
    \item[{Contenu dans}]
  
    \item[core: ]
   \hyperref[TEI.bibl]{bibl} \hyperref[TEI.biblStruct]{biblStruct}\par 
    \item[header: ]
   \hyperref[TEI.notesStmt]{notesStmt}
    \item[{Peut contenir}]
  
    \item[core: ]
   \hyperref[TEI.bibl]{bibl} \hyperref[TEI.biblStruct]{biblStruct} \hyperref[TEI.listBibl]{listBibl} \hyperref[TEI.ptr]{ptr} \hyperref[TEI.ref]{ref}\par 
    \item[header: ]
   \hyperref[TEI.biblFull]{biblFull}\par 
    \item[msdescription: ]
   \hyperref[TEI.msDesc]{msDesc}
    \item[{Note}]
  \par
If the {\itshape target} attribute is used to reference the related bibliographic item, the element must be empty.
    \item[{Exemple}]
  \leavevmode\bgroup\exampleFont \begin{shaded}\noindent\mbox{}{<\textbf{biblStruct}>}\mbox{}\newline 
\hspace*{6pt}{<\textbf{monogr}>}\mbox{}\newline 
\hspace*{6pt}\hspace*{6pt}{<\textbf{author}>}Shirley, James{</\textbf{author}>}\mbox{}\newline 
\hspace*{6pt}\hspace*{6pt}{<\textbf{title}\hspace*{6pt}{type}="{main}">}The gentlemen of Venice{</\textbf{title}>}\mbox{}\newline 
\hspace*{6pt}\hspace*{6pt}{<\textbf{imprint}>}\mbox{}\newline 
\hspace*{6pt}\hspace*{6pt}\hspace*{6pt}{<\textbf{pubPlace}>}New York{</\textbf{pubPlace}>}\mbox{}\newline 
\hspace*{6pt}\hspace*{6pt}\hspace*{6pt}{<\textbf{publisher}>}Readex Microprint{</\textbf{publisher}>}\mbox{}\newline 
\hspace*{6pt}\hspace*{6pt}\hspace*{6pt}{<\textbf{date}>}1953{</\textbf{date}>}\mbox{}\newline 
\hspace*{6pt}\hspace*{6pt}{</\textbf{imprint}>}\mbox{}\newline 
\hspace*{6pt}\hspace*{6pt}{<\textbf{extent}>}1 microprint card, 23 x 15 cm.{</\textbf{extent}>}\mbox{}\newline 
\hspace*{6pt}{</\textbf{monogr}>}\mbox{}\newline 
\hspace*{6pt}{<\textbf{series}>}\mbox{}\newline 
\hspace*{6pt}\hspace*{6pt}{<\textbf{title}>}Three centuries of drama: English, 1642–1700{</\textbf{title}>}\mbox{}\newline 
\hspace*{6pt}{</\textbf{series}>}\mbox{}\newline 
\hspace*{6pt}{<\textbf{relatedItem}\hspace*{6pt}{type}="{otherForm}">}\mbox{}\newline 
\hspace*{6pt}\hspace*{6pt}{<\textbf{biblStruct}>}\mbox{}\newline 
\hspace*{6pt}\hspace*{6pt}\hspace*{6pt}{<\textbf{monogr}>}\mbox{}\newline 
\hspace*{6pt}\hspace*{6pt}\hspace*{6pt}\hspace*{6pt}{<\textbf{author}>}Shirley, James{</\textbf{author}>}\mbox{}\newline 
\hspace*{6pt}\hspace*{6pt}\hspace*{6pt}\hspace*{6pt}{<\textbf{title}\hspace*{6pt}{type}="{main}">}The gentlemen of Venice{</\textbf{title}>}\mbox{}\newline 
\hspace*{6pt}\hspace*{6pt}\hspace*{6pt}\hspace*{6pt}{<\textbf{title}\hspace*{6pt}{type}="{sub}">}a tragi-comedie presented at the private house in Salisbury\mbox{}\newline 
\hspace*{6pt}\hspace*{6pt}\hspace*{6pt}\hspace*{6pt}\hspace*{6pt}\hspace*{6pt}\hspace*{6pt}\hspace*{6pt} Court by Her Majesties servants{</\textbf{title}>}\mbox{}\newline 
\hspace*{6pt}\hspace*{6pt}\hspace*{6pt}\hspace*{6pt}{<\textbf{imprint}>}\mbox{}\newline 
\hspace*{6pt}\hspace*{6pt}\hspace*{6pt}\hspace*{6pt}\hspace*{6pt}{<\textbf{pubPlace}>}London{</\textbf{pubPlace}>}\mbox{}\newline 
\hspace*{6pt}\hspace*{6pt}\hspace*{6pt}\hspace*{6pt}\hspace*{6pt}{<\textbf{publisher}>}H. Moseley{</\textbf{publisher}>}\mbox{}\newline 
\hspace*{6pt}\hspace*{6pt}\hspace*{6pt}\hspace*{6pt}\hspace*{6pt}{<\textbf{date}>}1655{</\textbf{date}>}\mbox{}\newline 
\hspace*{6pt}\hspace*{6pt}\hspace*{6pt}\hspace*{6pt}{</\textbf{imprint}>}\mbox{}\newline 
\hspace*{6pt}\hspace*{6pt}\hspace*{6pt}\hspace*{6pt}{<\textbf{extent}>}78 p.{</\textbf{extent}>}\mbox{}\newline 
\hspace*{6pt}\hspace*{6pt}\hspace*{6pt}{</\textbf{monogr}>}\mbox{}\newline 
\hspace*{6pt}\hspace*{6pt}{</\textbf{biblStruct}>}\mbox{}\newline 
\hspace*{6pt}{</\textbf{relatedItem}>}\mbox{}\newline 
{</\textbf{biblStruct}>}\end{shaded}\egroup 


    \item[{Schematron}]
   <sch:report test="@target and count( child::* ) > 0">If the @target attribute on <sch:name/> is used, the  relatedItem element must be empty</sch:report> <sch:assert test="@target or child::*">A relatedItem element should have either a 'target' attribute  or a child element to indicate the related bibliographic item</sch:assert>
    \item[{Modèle de contenu}]
  \mbox{}\hfill\\[-10pt]\begin{Verbatim}[fontsize=\small]
<content>
 <alternate maxOccurs="1" minOccurs="0">
  <classRef key="model.biblLike"/>
  <classRef key="model.ptrLike"/>
 </alternate>
</content>
    
\end{Verbatim}

    \item[{Schéma Declaration}]
  \mbox{}\hfill\\[-10pt]\begin{Verbatim}[fontsize=\small]
element relatedItem
{
   tei_att.global.attributes,
   tei_att.typed.attributes,
   attribute target { text }?,
   ( tei_model.biblLike | tei_model.ptrLike )?
}
\end{Verbatim}

\end{reflist}  \index{rendition=<rendition>|oddindex}\index{scope=@scope!<rendition>|oddindex}\index{selector=@selector!<rendition>|oddindex}
\begin{reflist}
\item[]\begin{specHead}{TEI.rendition}{<rendition> }(rendu) donne des informations sur le rendu ou sur l'apparence d'un ou de plusieurs éléments dans le texte source. [\xref{http://www.tei-c.org/release/doc/tei-p5-doc/en/html/HD.html\#HD57}{2.3.4. The Tagging Declaration}]\end{specHead} 
    \item[{Module}]
  header
    \item[{Attributs}]
  Attributs \hyperref[TEI.att.global]{att.global} (\textit{@xml:id}, \textit{@n}, \textit{@xml:lang}, \textit{@xml:base}, \textit{@xml:space})  (\hyperref[TEI.att.global.rendition]{att.global.rendition} (\textit{@rend}, \textit{@style}, \textit{@rendition})) (\hyperref[TEI.att.global.linking]{att.global.linking} (\textit{@corresp}, \textit{@synch}, \textit{@sameAs}, \textit{@copyOf}, \textit{@next}, \textit{@prev}, \textit{@exclude}, \textit{@select})) (\hyperref[TEI.att.global.analytic]{att.global.analytic} (\textit{@ana})) (\hyperref[TEI.att.global.facs]{att.global.facs} (\textit{@facs})) (\hyperref[TEI.att.global.change]{att.global.change} (\textit{@change})) (\hyperref[TEI.att.global.responsibility]{att.global.responsibility} (\textit{@cert}, \textit{@resp})) (\hyperref[TEI.att.global.source]{att.global.source} (\textit{@source})) \hyperref[TEI.att.styleDef]{att.styleDef} (\textit{@scheme}, \textit{@schemeVersion}) \hfil\\[-10pt]\begin{sansreflist}
    \item[@scope]
  where CSS is used, provides a way of defining ‘pseudo-elements’, that is, styling rules applicable to specific sub-portions of an element.
\begin{reflist}
    \item[{Statut}]
  Optionel
    \item[{Type de données}]
  \hyperref[TEI.teidata.enumerated]{teidata.enumerated}
    \item[{Exemple de valeurs possibles:}]
  \begin{description}

\item[{first-line}]styling applies to the first line of the target element
\item[{first-letter}]styling applies to the first letter of the target element
\item[{before}]styling should be applied immediately before the content of the target element
\item[{after}]styling should be applied immediately after the content of the target element
\end{description} 
\end{reflist}  
    \item[@selector]
  contains a selector or series of selectors specifying the elements to which the contained style description applies, expressed in the language specified in the {\itshape scheme} attribute.
\begin{reflist}
    \item[{Statut}]
  Optionel
    \item[{Type de données}]
  \hyperref[TEI.teidata.text]{teidata.text}
    \item[]\exampleFont {<\textbf{rendition}\hspace*{6pt}{scheme}="{css}"\mbox{}\newline 
\hspace*{6pt}{selector}="{text, front, back, body, div, p, ab}">} \mbox{}\newline 
 display: block;\mbox{}\newline 
{</\textbf{rendition}>}
    \item[]\exampleFont {<\textbf{rendition}\hspace*{6pt}{scheme}="{css}"\mbox{}\newline 
\hspace*{6pt}{selector}="{*[rend*=italic]}">} font-style: italic;\mbox{}\newline 
{</\textbf{rendition}>}
\end{reflist}  
\end{sansreflist}  
    \item[{Contenu dans}]
  —
    \item[{Peut contenir}]
  
    \item[core: ]
   \hyperref[TEI.abbr]{abbr} \hyperref[TEI.address]{address} \hyperref[TEI.bibl]{bibl} \hyperref[TEI.biblStruct]{biblStruct} \hyperref[TEI.choice]{choice} \hyperref[TEI.cit]{cit} \hyperref[TEI.date]{date} \hyperref[TEI.desc]{desc} \hyperref[TEI.distinct]{distinct} \hyperref[TEI.email]{email} \hyperref[TEI.emph]{emph} \hyperref[TEI.expan]{expan} \hyperref[TEI.foreign]{foreign} \hyperref[TEI.gloss]{gloss} \hyperref[TEI.hi]{hi} \hyperref[TEI.label]{label} \hyperref[TEI.list]{list} \hyperref[TEI.listBibl]{listBibl} \hyperref[TEI.measure]{measure} \hyperref[TEI.measureGrp]{measureGrp} \hyperref[TEI.mentioned]{mentioned} \hyperref[TEI.name]{name} \hyperref[TEI.num]{num} \hyperref[TEI.ptr]{ptr} \hyperref[TEI.q]{q} \hyperref[TEI.quote]{quote} \hyperref[TEI.ref]{ref} \hyperref[TEI.rs]{rs} \hyperref[TEI.said]{said} \hyperref[TEI.soCalled]{soCalled} \hyperref[TEI.stage]{stage} \hyperref[TEI.term]{term} \hyperref[TEI.time]{time} \hyperref[TEI.title]{title}\par 
    \item[figures: ]
   \hyperref[TEI.table]{table}\par 
    \item[header: ]
   \hyperref[TEI.biblFull]{biblFull} \hyperref[TEI.idno]{idno}\par 
    \item[msdescription: ]
   \hyperref[TEI.catchwords]{catchwords} \hyperref[TEI.depth]{depth} \hyperref[TEI.dim]{dim} \hyperref[TEI.dimensions]{dimensions} \hyperref[TEI.height]{height} \hyperref[TEI.heraldry]{heraldry} \hyperref[TEI.locus]{locus} \hyperref[TEI.locusGrp]{locusGrp} \hyperref[TEI.material]{material} \hyperref[TEI.msDesc]{msDesc} \hyperref[TEI.objectType]{objectType} \hyperref[TEI.origDate]{origDate} \hyperref[TEI.origPlace]{origPlace} \hyperref[TEI.secFol]{secFol} \hyperref[TEI.signatures]{signatures} \hyperref[TEI.stamp]{stamp} \hyperref[TEI.watermark]{watermark} \hyperref[TEI.width]{width}\par 
    \item[namesdates: ]
   \hyperref[TEI.addName]{addName} \hyperref[TEI.affiliation]{affiliation} \hyperref[TEI.country]{country} \hyperref[TEI.forename]{forename} \hyperref[TEI.genName]{genName} \hyperref[TEI.geogName]{geogName} \hyperref[TEI.listOrg]{listOrg} \hyperref[TEI.listPlace]{listPlace} \hyperref[TEI.location]{location} \hyperref[TEI.nameLink]{nameLink} \hyperref[TEI.orgName]{orgName} \hyperref[TEI.persName]{persName} \hyperref[TEI.placeName]{placeName} \hyperref[TEI.region]{region} \hyperref[TEI.roleName]{roleName} \hyperref[TEI.settlement]{settlement} \hyperref[TEI.state]{state} \hyperref[TEI.surname]{surname}\par 
    \item[textstructure: ]
   \hyperref[TEI.floatingText]{floatingText}\par 
    \item[transcr: ]
   \hyperref[TEI.am]{am} \hyperref[TEI.ex]{ex} \hyperref[TEI.subst]{subst}\par des données textuelles
    \item[{Exemple}]
  \leavevmode\bgroup\exampleFont \begin{shaded}\noindent\mbox{}{<\textbf{tagsDecl}>}\mbox{}\newline 
\hspace*{6pt}{<\textbf{rendition}\hspace*{6pt}{scheme}="{css}"\hspace*{6pt}{xml:id}="{r-center}">}text-align: center;{</\textbf{rendition}>}\mbox{}\newline 
\hspace*{6pt}{<\textbf{rendition}\hspace*{6pt}{scheme}="{css}"\hspace*{6pt}{xml:id}="{r-small}">}font-size: small;{</\textbf{rendition}>}\mbox{}\newline 
\hspace*{6pt}{<\textbf{rendition}\hspace*{6pt}{scheme}="{css}"\hspace*{6pt}{xml:id}="{r-large}">}font-size: large;{</\textbf{rendition}>}\mbox{}\newline 
\hspace*{6pt}{<\textbf{rendition}\hspace*{6pt}{scheme}="{css}"\mbox{}\newline 
\hspace*{6pt}\hspace*{6pt}{scope}="{first-letter}"\hspace*{6pt}{xml:id}="{initcaps}">}font-size: xx-large{</\textbf{rendition}>}\mbox{}\newline 
{</\textbf{tagsDecl}>}\end{shaded}\egroup 


    \item[{Modèle de contenu}]
  \mbox{}\hfill\\[-10pt]\begin{Verbatim}[fontsize=\small]
<content>
 <macroRef key="macro.limitedContent"/>
</content>
    
\end{Verbatim}

    \item[{Schéma Declaration}]
  \mbox{}\hfill\\[-10pt]\begin{Verbatim}[fontsize=\small]
element rendition
{
   tei_att.global.attributes,
   tei_att.styleDef.attributes,
   attribute scope { text }?,
   attribute selector { text }?,
   tei_macro.limitedContent}
\end{Verbatim}

\end{reflist}  \index{repository=<repository>|oddindex}
\begin{reflist}
\item[]\begin{specHead}{TEI.repository}{<repository> }(lieu de conservation) Contient le nom d'un dépôt dans lequel des manuscrits sont entreposés, et qui peut faire partie d'une institution. [\xref{http://www.tei-c.org/release/doc/tei-p5-doc/en/html/MS.html\#msid}{10.4. The Manuscript Identifier}]\end{specHead} 
    \item[{Module}]
  msdescription
    \item[{Attributs}]
  Attributs \hyperref[TEI.att.global]{att.global} (\textit{@xml:id}, \textit{@n}, \textit{@xml:lang}, \textit{@xml:base}, \textit{@xml:space})  (\hyperref[TEI.att.global.rendition]{att.global.rendition} (\textit{@rend}, \textit{@style}, \textit{@rendition})) (\hyperref[TEI.att.global.linking]{att.global.linking} (\textit{@corresp}, \textit{@synch}, \textit{@sameAs}, \textit{@copyOf}, \textit{@next}, \textit{@prev}, \textit{@exclude}, \textit{@select})) (\hyperref[TEI.att.global.analytic]{att.global.analytic} (\textit{@ana})) (\hyperref[TEI.att.global.facs]{att.global.facs} (\textit{@facs})) (\hyperref[TEI.att.global.change]{att.global.change} (\textit{@change})) (\hyperref[TEI.att.global.responsibility]{att.global.responsibility} (\textit{@cert}, \textit{@resp})) (\hyperref[TEI.att.global.source]{att.global.source} (\textit{@source})) \hyperref[TEI.att.naming]{att.naming} (\textit{@role}, \textit{@nymRef})  (\hyperref[TEI.att.canonical]{att.canonical} (\textit{@key}, \textit{@ref}))
    \item[{Contenu dans}]
  
    \item[msdescription: ]
   \hyperref[TEI.altIdentifier]{altIdentifier} \hyperref[TEI.msIdentifier]{msIdentifier}
    \item[{Peut contenir}]
  Des données textuelles uniquement
    \item[{Exemple}]
  \leavevmode\bgroup\exampleFont \begin{shaded}\noindent\mbox{}{<\textbf{msIdentifier}>}\mbox{}\newline 
\hspace*{6pt}{<\textbf{settlement}>}Oxford{</\textbf{settlement}>}\mbox{}\newline 
\hspace*{6pt}{<\textbf{institution}>}University of Oxford{</\textbf{institution}>}\mbox{}\newline 
\hspace*{6pt}{<\textbf{repository}>}Bodleian Library{</\textbf{repository}>}\mbox{}\newline 
\hspace*{6pt}{<\textbf{idno}>}MS. Bodley 406{</\textbf{idno}>}\mbox{}\newline 
{</\textbf{msIdentifier}>}\end{shaded}\egroup 


    \item[{Modèle de contenu}]
  \fbox{\ttfamily <content>\newline
 <macroRef key="macro.xtext"/>\newline
</content>\newline
    } 
    \item[{Schéma Declaration}]
  \mbox{}\hfill\\[-10pt]\begin{Verbatim}[fontsize=\small]
element repository
{
   tei_att.global.attributes,
   tei_att.naming.attributes,
   tei_macro.xtext}
\end{Verbatim}

\end{reflist}  \index{resp=<resp>|oddindex}
\begin{reflist}
\item[]\begin{specHead}{TEI.resp}{<resp> }(responsabilité) contient une expression décrivant la nature de la responsabilité intellectuelle d'une personne. [\xref{http://www.tei-c.org/release/doc/tei-p5-doc/en/html/CO.html\#COBICOR}{3.11.2.2. Titles, Authors, and Editors} \xref{http://www.tei-c.org/release/doc/tei-p5-doc/en/html/HD.html\#HD21}{2.2.1. The Title Statement} \xref{http://www.tei-c.org/release/doc/tei-p5-doc/en/html/HD.html\#HD22}{2.2.2. The Edition Statement} \xref{http://www.tei-c.org/release/doc/tei-p5-doc/en/html/HD.html\#HD26}{2.2.5. The Series Statement}]\end{specHead} 
    \item[{Module}]
  core
    \item[{Attributs}]
  Attributs \hyperref[TEI.att.global]{att.global} (\textit{@xml:id}, \textit{@n}, \textit{@xml:lang}, \textit{@xml:base}, \textit{@xml:space})  (\hyperref[TEI.att.global.rendition]{att.global.rendition} (\textit{@rend}, \textit{@style}, \textit{@rendition})) (\hyperref[TEI.att.global.linking]{att.global.linking} (\textit{@corresp}, \textit{@synch}, \textit{@sameAs}, \textit{@copyOf}, \textit{@next}, \textit{@prev}, \textit{@exclude}, \textit{@select})) (\hyperref[TEI.att.global.analytic]{att.global.analytic} (\textit{@ana})) (\hyperref[TEI.att.global.facs]{att.global.facs} (\textit{@facs})) (\hyperref[TEI.att.global.change]{att.global.change} (\textit{@change})) (\hyperref[TEI.att.global.responsibility]{att.global.responsibility} (\textit{@cert}, \textit{@resp})) (\hyperref[TEI.att.global.source]{att.global.source} (\textit{@source})) \hyperref[TEI.att.canonical]{att.canonical} (\textit{@key}, \textit{@ref}) \hyperref[TEI.att.datable]{att.datable} (\textit{@calendar}, \textit{@period})  (\hyperref[TEI.att.datable.w3c]{att.datable.w3c} (\textit{@when}, \textit{@notBefore}, \textit{@notAfter}, \textit{@from}, \textit{@to})) (\hyperref[TEI.att.datable.iso]{att.datable.iso} (\textit{@when-iso}, \textit{@notBefore-iso}, \textit{@notAfter-iso}, \textit{@from-iso}, \textit{@to-iso})) (\hyperref[TEI.att.datable.custom]{att.datable.custom} (\textit{@when-custom}, \textit{@notBefore-custom}, \textit{@notAfter-custom}, \textit{@from-custom}, \textit{@to-custom}, \textit{@datingPoint}, \textit{@datingMethod}))
    \item[{Contenu dans}]
  
    \item[core: ]
   \hyperref[TEI.respStmt]{respStmt}
    \item[{Peut contenir}]
  
    \item[analysis: ]
   \hyperref[TEI.interp]{interp} \hyperref[TEI.interpGrp]{interpGrp} \hyperref[TEI.span]{span} \hyperref[TEI.spanGrp]{spanGrp}\par 
    \item[core: ]
   \hyperref[TEI.abbr]{abbr} \hyperref[TEI.address]{address} \hyperref[TEI.cb]{cb} \hyperref[TEI.choice]{choice} \hyperref[TEI.date]{date} \hyperref[TEI.distinct]{distinct} \hyperref[TEI.email]{email} \hyperref[TEI.emph]{emph} \hyperref[TEI.expan]{expan} \hyperref[TEI.foreign]{foreign} \hyperref[TEI.gap]{gap} \hyperref[TEI.gb]{gb} \hyperref[TEI.gloss]{gloss} \hyperref[TEI.hi]{hi} \hyperref[TEI.index]{index} \hyperref[TEI.lb]{lb} \hyperref[TEI.measure]{measure} \hyperref[TEI.measureGrp]{measureGrp} \hyperref[TEI.mentioned]{mentioned} \hyperref[TEI.milestone]{milestone} \hyperref[TEI.name]{name} \hyperref[TEI.note]{note} \hyperref[TEI.num]{num} \hyperref[TEI.pb]{pb} \hyperref[TEI.ptr]{ptr} \hyperref[TEI.ref]{ref} \hyperref[TEI.rs]{rs} \hyperref[TEI.soCalled]{soCalled} \hyperref[TEI.term]{term} \hyperref[TEI.time]{time} \hyperref[TEI.title]{title}\par 
    \item[figures: ]
   \hyperref[TEI.figure]{figure} \hyperref[TEI.notatedMusic]{notatedMusic}\par 
    \item[header: ]
   \hyperref[TEI.idno]{idno}\par 
    \item[iso-fs: ]
   \hyperref[TEI.fLib]{fLib} \hyperref[TEI.fs]{fs} \hyperref[TEI.fvLib]{fvLib}\par 
    \item[linking: ]
   \hyperref[TEI.alt]{alt} \hyperref[TEI.altGrp]{altGrp} \hyperref[TEI.anchor]{anchor} \hyperref[TEI.join]{join} \hyperref[TEI.joinGrp]{joinGrp} \hyperref[TEI.link]{link} \hyperref[TEI.linkGrp]{linkGrp} \hyperref[TEI.timeline]{timeline}\par 
    \item[msdescription: ]
   \hyperref[TEI.catchwords]{catchwords} \hyperref[TEI.depth]{depth} \hyperref[TEI.dim]{dim} \hyperref[TEI.dimensions]{dimensions} \hyperref[TEI.height]{height} \hyperref[TEI.heraldry]{heraldry} \hyperref[TEI.locus]{locus} \hyperref[TEI.locusGrp]{locusGrp} \hyperref[TEI.material]{material} \hyperref[TEI.objectType]{objectType} \hyperref[TEI.origDate]{origDate} \hyperref[TEI.origPlace]{origPlace} \hyperref[TEI.secFol]{secFol} \hyperref[TEI.signatures]{signatures} \hyperref[TEI.source]{source} \hyperref[TEI.stamp]{stamp} \hyperref[TEI.watermark]{watermark} \hyperref[TEI.width]{width}\par 
    \item[namesdates: ]
   \hyperref[TEI.addName]{addName} \hyperref[TEI.affiliation]{affiliation} \hyperref[TEI.country]{country} \hyperref[TEI.forename]{forename} \hyperref[TEI.genName]{genName} \hyperref[TEI.geogName]{geogName} \hyperref[TEI.location]{location} \hyperref[TEI.nameLink]{nameLink} \hyperref[TEI.orgName]{orgName} \hyperref[TEI.persName]{persName} \hyperref[TEI.placeName]{placeName} \hyperref[TEI.region]{region} \hyperref[TEI.roleName]{roleName} \hyperref[TEI.settlement]{settlement} \hyperref[TEI.state]{state} \hyperref[TEI.surname]{surname}\par 
    \item[transcr: ]
   \hyperref[TEI.addSpan]{addSpan} \hyperref[TEI.am]{am} \hyperref[TEI.damageSpan]{damageSpan} \hyperref[TEI.delSpan]{delSpan} \hyperref[TEI.ex]{ex} \hyperref[TEI.fw]{fw} \hyperref[TEI.listTranspose]{listTranspose} \hyperref[TEI.metamark]{metamark} \hyperref[TEI.space]{space} \hyperref[TEI.subst]{subst} \hyperref[TEI.substJoin]{substJoin}\par des données textuelles
    \item[{Note}]
  \par
Les attributs {\itshape key} or {\itshape ref}, issus de la classe \textsf{att.canonical}, peuvent être utilisés pour indiquer le type de responsabilité sous une forme normalisée, en faisant référence directement (par l'utilisation de {\itshape ref}) ou indirectement (par l'utilisation de {\itshape key}) à une liste normalisée contenant des types de responsabilité, comme celle qui est maintenue par une autorité de nommage, par exemple la liste \url{http://www.loc.gov/marc/relators/relacode.html} à usage bibliographique.
    \item[{Exemple}]
  \leavevmode\bgroup\exampleFont \begin{shaded}\noindent\mbox{}{<\textbf{respStmt}>}\mbox{}\newline 
\hspace*{6pt}{<\textbf{resp}>}compilateur{</\textbf{resp}>}\mbox{}\newline 
\hspace*{6pt}{<\textbf{name}>}Edward Child{</\textbf{name}>}\mbox{}\newline 
{</\textbf{respStmt}>}\end{shaded}\egroup 


    \item[{Modèle de contenu}]
  \mbox{}\hfill\\[-10pt]\begin{Verbatim}[fontsize=\small]
<content>
 <macroRef key="macro.phraseSeq.limited"/>
</content>
    
\end{Verbatim}

    \item[{Schéma Declaration}]
  \mbox{}\hfill\\[-10pt]\begin{Verbatim}[fontsize=\small]
element resp
{
   tei_att.global.attributes,
   tei_att.canonical.attributes,
   tei_att.datable.attributes,
   tei_macro.phraseSeq.limited}
\end{Verbatim}

\end{reflist}  \index{respStmt=<respStmt>|oddindex}
\begin{reflist}
\item[]\begin{specHead}{TEI.respStmt}{<respStmt> }(mention de responsabilité) indique la responsabilité quant au contenu intellectuel d'un texte, d'une édition, d'un enregistrement ou d'une publication en série, lorsque les éléments spécifiques relatifs aux auteurs, éditeurs, etc. ne suffisent pas ou ne s'appliquent pas. [\xref{http://www.tei-c.org/release/doc/tei-p5-doc/en/html/CO.html\#COBICOR}{3.11.2.2. Titles, Authors, and Editors} \xref{http://www.tei-c.org/release/doc/tei-p5-doc/en/html/HD.html\#HD21}{2.2.1. The Title Statement} \xref{http://www.tei-c.org/release/doc/tei-p5-doc/en/html/HD.html\#HD22}{2.2.2. The Edition Statement} \xref{http://www.tei-c.org/release/doc/tei-p5-doc/en/html/HD.html\#HD26}{2.2.5. The Series Statement}]\end{specHead} 
    \item[{Module}]
  core
    \item[{Attributs}]
  Attributs \hyperref[TEI.att.global]{att.global} (\textit{@xml:id}, \textit{@n}, \textit{@xml:lang}, \textit{@xml:base}, \textit{@xml:space})  (\hyperref[TEI.att.global.rendition]{att.global.rendition} (\textit{@rend}, \textit{@style}, \textit{@rendition})) (\hyperref[TEI.att.global.linking]{att.global.linking} (\textit{@corresp}, \textit{@synch}, \textit{@sameAs}, \textit{@copyOf}, \textit{@next}, \textit{@prev}, \textit{@exclude}, \textit{@select})) (\hyperref[TEI.att.global.analytic]{att.global.analytic} (\textit{@ana})) (\hyperref[TEI.att.global.facs]{att.global.facs} (\textit{@facs})) (\hyperref[TEI.att.global.change]{att.global.change} (\textit{@change})) (\hyperref[TEI.att.global.responsibility]{att.global.responsibility} (\textit{@cert}, \textit{@resp})) (\hyperref[TEI.att.global.source]{att.global.source} (\textit{@source})) \hyperref[TEI.att.canonical]{att.canonical} (\textit{@key}, \textit{@ref}) 
    \item[{Membre du}]
  \hyperref[TEI.model.respLike]{model.respLike} 
    \item[{Contenu dans}]
  
    \item[core: ]
   \hyperref[TEI.analytic]{analytic} \hyperref[TEI.bibl]{bibl} \hyperref[TEI.imprint]{imprint} \hyperref[TEI.monogr]{monogr} \hyperref[TEI.series]{series}\par 
    \item[header: ]
   \hyperref[TEI.editionStmt]{editionStmt} \hyperref[TEI.seriesStmt]{seriesStmt} \hyperref[TEI.titleStmt]{titleStmt}\par 
    \item[msdescription: ]
   \hyperref[TEI.msItem]{msItem} \hyperref[TEI.msItemStruct]{msItemStruct}
    \item[{Peut contenir}]
  
    \item[core: ]
   \hyperref[TEI.name]{name} \hyperref[TEI.note]{note} \hyperref[TEI.resp]{resp}\par 
    \item[namesdates: ]
   \hyperref[TEI.orgName]{orgName} \hyperref[TEI.persName]{persName}
    \item[{Exemple}]
  \leavevmode\bgroup\exampleFont \begin{shaded}\noindent\mbox{}{<\textbf{respStmt}>}\mbox{}\newline 
\hspace*{6pt}{<\textbf{resp}>}Nouvelle édition originale{</\textbf{resp}>}\mbox{}\newline 
\hspace*{6pt}{<\textbf{persName}>}Geneviève Hasenohr{</\textbf{persName}>}\mbox{}\newline 
{</\textbf{respStmt}>}\end{shaded}\egroup 


    \item[{Exemple}]
  \leavevmode\bgroup\exampleFont \begin{shaded}\noindent\mbox{}{<\textbf{respStmt}>}\mbox{}\newline 
\hspace*{6pt}{<\textbf{resp}>}converti en langage SGML{</\textbf{resp}>}\mbox{}\newline 
\hspace*{6pt}{<\textbf{name}>}Alan Morrison{</\textbf{name}>}\mbox{}\newline 
{</\textbf{respStmt}>}\end{shaded}\egroup 


    \item[{Modèle de contenu}]
  \mbox{}\hfill\\[-10pt]\begin{Verbatim}[fontsize=\small]
<content>
 <sequence maxOccurs="1" minOccurs="1">
  <alternate maxOccurs="1" minOccurs="1">
   <sequence maxOccurs="1" minOccurs="1">
    <elementRef key="resp"
     maxOccurs="unbounded" minOccurs="1"/>
    <classRef key="model.nameLike.agent"
     maxOccurs="unbounded" minOccurs="1"/>
   </sequence>
   <sequence maxOccurs="1" minOccurs="1">
    <classRef key="model.nameLike.agent"
     maxOccurs="unbounded" minOccurs="1"/>
    <elementRef key="resp"
     maxOccurs="unbounded" minOccurs="1"/>
   </sequence>
  </alternate>
  <elementRef key="note"
   maxOccurs="unbounded" minOccurs="0"/>
 </sequence>
</content>
    
\end{Verbatim}

    \item[{Schéma Declaration}]
  \mbox{}\hfill\\[-10pt]\begin{Verbatim}[fontsize=\small]
element respStmt
{
   tei_att.global.attributes,
   tei_att.canonical.attributes,
   (
      (
         ( tei_resp+, tei_model.nameLike.agent+ )
       | ( tei_model.nameLike.agent+, tei_resp+ )
      ),
      tei_note*
   )
}
\end{Verbatim}

\end{reflist}  \index{restore=<restore>|oddindex}
\begin{reflist}
\item[]\begin{specHead}{TEI.restore}{<restore> }(rétablissement) indique le rétablissement d'un état antérieur du texte par suppression d'une marque ou d'une instruction de l'éditeur ou de l'auteur. [\xref{http://www.tei-c.org/release/doc/tei-p5-doc/en/html/PH.html\#PHCD}{11.3.1.6. Cancellation of Deletions and Other Markings}]\end{specHead} 
    \item[{Module}]
  transcr
    \item[{Attributs}]
  Attributs \hyperref[TEI.att.global]{att.global} (\textit{@xml:id}, \textit{@n}, \textit{@xml:lang}, \textit{@xml:base}, \textit{@xml:space})  (\hyperref[TEI.att.global.rendition]{att.global.rendition} (\textit{@rend}, \textit{@style}, \textit{@rendition})) (\hyperref[TEI.att.global.linking]{att.global.linking} (\textit{@corresp}, \textit{@synch}, \textit{@sameAs}, \textit{@copyOf}, \textit{@next}, \textit{@prev}, \textit{@exclude}, \textit{@select})) (\hyperref[TEI.att.global.analytic]{att.global.analytic} (\textit{@ana})) (\hyperref[TEI.att.global.facs]{att.global.facs} (\textit{@facs})) (\hyperref[TEI.att.global.change]{att.global.change} (\textit{@change})) (\hyperref[TEI.att.global.responsibility]{att.global.responsibility} (\textit{@cert}, \textit{@resp})) (\hyperref[TEI.att.global.source]{att.global.source} (\textit{@source})) \hyperref[TEI.att.transcriptional]{att.transcriptional} (\textit{@status}, \textit{@cause}, \textit{@seq})  (\hyperref[TEI.att.editLike]{att.editLike} (\textit{@evidence}, \textit{@instant}) (\hyperref[TEI.att.dimensions]{att.dimensions} (\textit{@unit}, \textit{@quantity}, \textit{@extent}, \textit{@precision}, \textit{@scope}) (\hyperref[TEI.att.ranging]{att.ranging} (\textit{@atLeast}, \textit{@atMost}, \textit{@min}, \textit{@max}, \textit{@confidence})) ) ) (\hyperref[TEI.att.written]{att.written} (\textit{@hand})) \hyperref[TEI.att.typed]{att.typed} (\textit{@type}, \textit{@subtype}) 
    \item[{Membre du}]
  \hyperref[TEI.model.linePart]{model.linePart} \hyperref[TEI.model.pPart.transcriptional]{model.pPart.transcriptional}
    \item[{Contenu dans}]
  
    \item[analysis: ]
   \hyperref[TEI.cl]{cl} \hyperref[TEI.pc]{pc} \hyperref[TEI.phr]{phr} \hyperref[TEI.s]{s} \hyperref[TEI.w]{w}\par 
    \item[core: ]
   \hyperref[TEI.abbr]{abbr} \hyperref[TEI.add]{add} \hyperref[TEI.addrLine]{addrLine} \hyperref[TEI.author]{author} \hyperref[TEI.bibl]{bibl} \hyperref[TEI.biblScope]{biblScope} \hyperref[TEI.citedRange]{citedRange} \hyperref[TEI.corr]{corr} \hyperref[TEI.date]{date} \hyperref[TEI.del]{del} \hyperref[TEI.distinct]{distinct} \hyperref[TEI.editor]{editor} \hyperref[TEI.email]{email} \hyperref[TEI.emph]{emph} \hyperref[TEI.expan]{expan} \hyperref[TEI.foreign]{foreign} \hyperref[TEI.gloss]{gloss} \hyperref[TEI.head]{head} \hyperref[TEI.headItem]{headItem} \hyperref[TEI.headLabel]{headLabel} \hyperref[TEI.hi]{hi} \hyperref[TEI.item]{item} \hyperref[TEI.l]{l} \hyperref[TEI.label]{label} \hyperref[TEI.measure]{measure} \hyperref[TEI.mentioned]{mentioned} \hyperref[TEI.name]{name} \hyperref[TEI.note]{note} \hyperref[TEI.num]{num} \hyperref[TEI.orig]{orig} \hyperref[TEI.p]{p} \hyperref[TEI.pubPlace]{pubPlace} \hyperref[TEI.publisher]{publisher} \hyperref[TEI.q]{q} \hyperref[TEI.quote]{quote} \hyperref[TEI.ref]{ref} \hyperref[TEI.reg]{reg} \hyperref[TEI.rs]{rs} \hyperref[TEI.said]{said} \hyperref[TEI.sic]{sic} \hyperref[TEI.soCalled]{soCalled} \hyperref[TEI.speaker]{speaker} \hyperref[TEI.stage]{stage} \hyperref[TEI.street]{street} \hyperref[TEI.term]{term} \hyperref[TEI.textLang]{textLang} \hyperref[TEI.time]{time} \hyperref[TEI.title]{title} \hyperref[TEI.unclear]{unclear}\par 
    \item[figures: ]
   \hyperref[TEI.cell]{cell}\par 
    \item[header: ]
   \hyperref[TEI.change]{change} \hyperref[TEI.distributor]{distributor} \hyperref[TEI.edition]{edition} \hyperref[TEI.extent]{extent} \hyperref[TEI.licence]{licence}\par 
    \item[linking: ]
   \hyperref[TEI.ab]{ab} \hyperref[TEI.seg]{seg}\par 
    \item[msdescription: ]
   \hyperref[TEI.accMat]{accMat} \hyperref[TEI.acquisition]{acquisition} \hyperref[TEI.additions]{additions} \hyperref[TEI.catchwords]{catchwords} \hyperref[TEI.collation]{collation} \hyperref[TEI.colophon]{colophon} \hyperref[TEI.condition]{condition} \hyperref[TEI.custEvent]{custEvent} \hyperref[TEI.decoNote]{decoNote} \hyperref[TEI.explicit]{explicit} \hyperref[TEI.filiation]{filiation} \hyperref[TEI.finalRubric]{finalRubric} \hyperref[TEI.foliation]{foliation} \hyperref[TEI.heraldry]{heraldry} \hyperref[TEI.incipit]{incipit} \hyperref[TEI.layout]{layout} \hyperref[TEI.material]{material} \hyperref[TEI.musicNotation]{musicNotation} \hyperref[TEI.objectType]{objectType} \hyperref[TEI.origDate]{origDate} \hyperref[TEI.origPlace]{origPlace} \hyperref[TEI.origin]{origin} \hyperref[TEI.provenance]{provenance} \hyperref[TEI.rubric]{rubric} \hyperref[TEI.secFol]{secFol} \hyperref[TEI.signatures]{signatures} \hyperref[TEI.source]{source} \hyperref[TEI.stamp]{stamp} \hyperref[TEI.summary]{summary} \hyperref[TEI.support]{support} \hyperref[TEI.surrogates]{surrogates} \hyperref[TEI.typeNote]{typeNote} \hyperref[TEI.watermark]{watermark}\par 
    \item[namesdates: ]
   \hyperref[TEI.addName]{addName} \hyperref[TEI.affiliation]{affiliation} \hyperref[TEI.country]{country} \hyperref[TEI.forename]{forename} \hyperref[TEI.genName]{genName} \hyperref[TEI.geogName]{geogName} \hyperref[TEI.nameLink]{nameLink} \hyperref[TEI.orgName]{orgName} \hyperref[TEI.persName]{persName} \hyperref[TEI.placeName]{placeName} \hyperref[TEI.region]{region} \hyperref[TEI.roleName]{roleName} \hyperref[TEI.settlement]{settlement} \hyperref[TEI.surname]{surname}\par 
    \item[textstructure: ]
   \hyperref[TEI.docAuthor]{docAuthor} \hyperref[TEI.docDate]{docDate} \hyperref[TEI.docEdition]{docEdition} \hyperref[TEI.titlePart]{titlePart}\par 
    \item[transcr: ]
   \hyperref[TEI.am]{am} \hyperref[TEI.damage]{damage} \hyperref[TEI.fw]{fw} \hyperref[TEI.line]{line} \hyperref[TEI.metamark]{metamark} \hyperref[TEI.mod]{mod} \hyperref[TEI.restore]{restore} \hyperref[TEI.retrace]{retrace} \hyperref[TEI.secl]{secl} \hyperref[TEI.supplied]{supplied} \hyperref[TEI.surplus]{surplus} \hyperref[TEI.zone]{zone}
    \item[{Peut contenir}]
  
    \item[analysis: ]
   \hyperref[TEI.c]{c} \hyperref[TEI.cl]{cl} \hyperref[TEI.interp]{interp} \hyperref[TEI.interpGrp]{interpGrp} \hyperref[TEI.m]{m} \hyperref[TEI.pc]{pc} \hyperref[TEI.phr]{phr} \hyperref[TEI.s]{s} \hyperref[TEI.span]{span} \hyperref[TEI.spanGrp]{spanGrp} \hyperref[TEI.w]{w}\par 
    \item[core: ]
   \hyperref[TEI.abbr]{abbr} \hyperref[TEI.add]{add} \hyperref[TEI.address]{address} \hyperref[TEI.bibl]{bibl} \hyperref[TEI.biblStruct]{biblStruct} \hyperref[TEI.binaryObject]{binaryObject} \hyperref[TEI.cb]{cb} \hyperref[TEI.choice]{choice} \hyperref[TEI.cit]{cit} \hyperref[TEI.corr]{corr} \hyperref[TEI.date]{date} \hyperref[TEI.del]{del} \hyperref[TEI.desc]{desc} \hyperref[TEI.distinct]{distinct} \hyperref[TEI.email]{email} \hyperref[TEI.emph]{emph} \hyperref[TEI.expan]{expan} \hyperref[TEI.foreign]{foreign} \hyperref[TEI.gap]{gap} \hyperref[TEI.gb]{gb} \hyperref[TEI.gloss]{gloss} \hyperref[TEI.graphic]{graphic} \hyperref[TEI.hi]{hi} \hyperref[TEI.index]{index} \hyperref[TEI.l]{l} \hyperref[TEI.label]{label} \hyperref[TEI.lb]{lb} \hyperref[TEI.lg]{lg} \hyperref[TEI.list]{list} \hyperref[TEI.listBibl]{listBibl} \hyperref[TEI.measure]{measure} \hyperref[TEI.measureGrp]{measureGrp} \hyperref[TEI.media]{media} \hyperref[TEI.mentioned]{mentioned} \hyperref[TEI.milestone]{milestone} \hyperref[TEI.name]{name} \hyperref[TEI.note]{note} \hyperref[TEI.num]{num} \hyperref[TEI.orig]{orig} \hyperref[TEI.pb]{pb} \hyperref[TEI.ptr]{ptr} \hyperref[TEI.q]{q} \hyperref[TEI.quote]{quote} \hyperref[TEI.ref]{ref} \hyperref[TEI.reg]{reg} \hyperref[TEI.rs]{rs} \hyperref[TEI.said]{said} \hyperref[TEI.sic]{sic} \hyperref[TEI.soCalled]{soCalled} \hyperref[TEI.stage]{stage} \hyperref[TEI.term]{term} \hyperref[TEI.time]{time} \hyperref[TEI.title]{title} \hyperref[TEI.unclear]{unclear}\par 
    \item[derived-module-tei.istex: ]
   \hyperref[TEI.math]{math} \hyperref[TEI.mrow]{mrow}\par 
    \item[figures: ]
   \hyperref[TEI.figure]{figure} \hyperref[TEI.formula]{formula} \hyperref[TEI.notatedMusic]{notatedMusic} \hyperref[TEI.table]{table}\par 
    \item[header: ]
   \hyperref[TEI.biblFull]{biblFull} \hyperref[TEI.idno]{idno}\par 
    \item[iso-fs: ]
   \hyperref[TEI.fLib]{fLib} \hyperref[TEI.fs]{fs} \hyperref[TEI.fvLib]{fvLib}\par 
    \item[linking: ]
   \hyperref[TEI.alt]{alt} \hyperref[TEI.altGrp]{altGrp} \hyperref[TEI.anchor]{anchor} \hyperref[TEI.join]{join} \hyperref[TEI.joinGrp]{joinGrp} \hyperref[TEI.link]{link} \hyperref[TEI.linkGrp]{linkGrp} \hyperref[TEI.seg]{seg} \hyperref[TEI.timeline]{timeline}\par 
    \item[msdescription: ]
   \hyperref[TEI.catchwords]{catchwords} \hyperref[TEI.depth]{depth} \hyperref[TEI.dim]{dim} \hyperref[TEI.dimensions]{dimensions} \hyperref[TEI.height]{height} \hyperref[TEI.heraldry]{heraldry} \hyperref[TEI.locus]{locus} \hyperref[TEI.locusGrp]{locusGrp} \hyperref[TEI.material]{material} \hyperref[TEI.msDesc]{msDesc} \hyperref[TEI.objectType]{objectType} \hyperref[TEI.origDate]{origDate} \hyperref[TEI.origPlace]{origPlace} \hyperref[TEI.secFol]{secFol} \hyperref[TEI.signatures]{signatures} \hyperref[TEI.source]{source} \hyperref[TEI.stamp]{stamp} \hyperref[TEI.watermark]{watermark} \hyperref[TEI.width]{width}\par 
    \item[namesdates: ]
   \hyperref[TEI.addName]{addName} \hyperref[TEI.affiliation]{affiliation} \hyperref[TEI.country]{country} \hyperref[TEI.forename]{forename} \hyperref[TEI.genName]{genName} \hyperref[TEI.geogName]{geogName} \hyperref[TEI.listOrg]{listOrg} \hyperref[TEI.listPlace]{listPlace} \hyperref[TEI.location]{location} \hyperref[TEI.nameLink]{nameLink} \hyperref[TEI.orgName]{orgName} \hyperref[TEI.persName]{persName} \hyperref[TEI.placeName]{placeName} \hyperref[TEI.region]{region} \hyperref[TEI.roleName]{roleName} \hyperref[TEI.settlement]{settlement} \hyperref[TEI.state]{state} \hyperref[TEI.surname]{surname}\par 
    \item[spoken: ]
   \hyperref[TEI.annotationBlock]{annotationBlock}\par 
    \item[textstructure: ]
   \hyperref[TEI.floatingText]{floatingText}\par 
    \item[transcr: ]
   \hyperref[TEI.addSpan]{addSpan} \hyperref[TEI.am]{am} \hyperref[TEI.damage]{damage} \hyperref[TEI.damageSpan]{damageSpan} \hyperref[TEI.delSpan]{delSpan} \hyperref[TEI.ex]{ex} \hyperref[TEI.fw]{fw} \hyperref[TEI.handShift]{handShift} \hyperref[TEI.listTranspose]{listTranspose} \hyperref[TEI.metamark]{metamark} \hyperref[TEI.mod]{mod} \hyperref[TEI.redo]{redo} \hyperref[TEI.restore]{restore} \hyperref[TEI.retrace]{retrace} \hyperref[TEI.secl]{secl} \hyperref[TEI.space]{space} \hyperref[TEI.subst]{subst} \hyperref[TEI.substJoin]{substJoin} \hyperref[TEI.supplied]{supplied} \hyperref[TEI.surplus]{surplus} \hyperref[TEI.undo]{undo}\par des données textuelles
    \item[{Note}]
  \par
L'attribut {\itshape type} de cet élément caractérise la manière dont l'intervention supprimée a été mentionnée, par exemple par une note marginale, par une surcharge de l'écriture, par un balisage additionnel, etc.
    \item[{Exemple}]
  \leavevmode\bgroup\exampleFont \begin{shaded}\noindent\mbox{}For I hate this\mbox{}\newline 
{<\textbf{restore}\hspace*{6pt}{hand}="{\#dhl}"\mbox{}\newline 
\hspace*{6pt}{type}="{marginalStetNote}">}\mbox{}\newline 
\hspace*{6pt}{<\textbf{del}>}my{</\textbf{del}>}\mbox{}\newline 
{</\textbf{restore}>} body \end{shaded}\egroup 


    \item[{Modèle de contenu}]
  \mbox{}\hfill\\[-10pt]\begin{Verbatim}[fontsize=\small]
<content>
 <macroRef key="macro.paraContent"/>
</content>
    
\end{Verbatim}

    \item[{Schéma Declaration}]
  \mbox{}\hfill\\[-10pt]\begin{Verbatim}[fontsize=\small]
element restore
{
   tei_att.global.attributes,
   tei_att.transcriptional.attributes,
   tei_att.typed.attributes,
   tei_macro.paraContent}
\end{Verbatim}

\end{reflist}  \index{retrace=<retrace>|oddindex}
\begin{reflist}
\item[]\begin{specHead}{TEI.retrace}{<retrace> }contains a sequence of writing which has been retraced, for example by over-inking, to clarify or fix it. [\xref{http://www.tei-c.org/release/doc/tei-p5-doc/en/html/PH.html\#PH-fix}{11.3.4.3. Fixation and Clarification}]\end{specHead} 
    \item[{Module}]
  transcr
    \item[{Attributs}]
  Attributs \hyperref[TEI.att.global]{att.global} (\textit{@xml:id}, \textit{@n}, \textit{@xml:lang}, \textit{@xml:base}, \textit{@xml:space})  (\hyperref[TEI.att.global.rendition]{att.global.rendition} (\textit{@rend}, \textit{@style}, \textit{@rendition})) (\hyperref[TEI.att.global.linking]{att.global.linking} (\textit{@corresp}, \textit{@synch}, \textit{@sameAs}, \textit{@copyOf}, \textit{@next}, \textit{@prev}, \textit{@exclude}, \textit{@select})) (\hyperref[TEI.att.global.analytic]{att.global.analytic} (\textit{@ana})) (\hyperref[TEI.att.global.facs]{att.global.facs} (\textit{@facs})) (\hyperref[TEI.att.global.change]{att.global.change} (\textit{@change})) (\hyperref[TEI.att.global.responsibility]{att.global.responsibility} (\textit{@cert}, \textit{@resp})) (\hyperref[TEI.att.global.source]{att.global.source} (\textit{@source})) \hyperref[TEI.att.spanning]{att.spanning} (\textit{@spanTo}) \hyperref[TEI.att.transcriptional]{att.transcriptional} (\textit{@status}, \textit{@cause}, \textit{@seq})  (\hyperref[TEI.att.editLike]{att.editLike} (\textit{@evidence}, \textit{@instant}) (\hyperref[TEI.att.dimensions]{att.dimensions} (\textit{@unit}, \textit{@quantity}, \textit{@extent}, \textit{@precision}, \textit{@scope}) (\hyperref[TEI.att.ranging]{att.ranging} (\textit{@atLeast}, \textit{@atMost}, \textit{@min}, \textit{@max}, \textit{@confidence})) ) ) (\hyperref[TEI.att.written]{att.written} (\textit{@hand}))
    \item[{Membre du}]
  \hyperref[TEI.model.linePart]{model.linePart} \hyperref[TEI.model.pPart.transcriptional]{model.pPart.transcriptional}
    \item[{Contenu dans}]
  
    \item[analysis: ]
   \hyperref[TEI.cl]{cl} \hyperref[TEI.pc]{pc} \hyperref[TEI.phr]{phr} \hyperref[TEI.s]{s} \hyperref[TEI.w]{w}\par 
    \item[core: ]
   \hyperref[TEI.abbr]{abbr} \hyperref[TEI.add]{add} \hyperref[TEI.addrLine]{addrLine} \hyperref[TEI.author]{author} \hyperref[TEI.bibl]{bibl} \hyperref[TEI.biblScope]{biblScope} \hyperref[TEI.citedRange]{citedRange} \hyperref[TEI.corr]{corr} \hyperref[TEI.date]{date} \hyperref[TEI.del]{del} \hyperref[TEI.distinct]{distinct} \hyperref[TEI.editor]{editor} \hyperref[TEI.email]{email} \hyperref[TEI.emph]{emph} \hyperref[TEI.expan]{expan} \hyperref[TEI.foreign]{foreign} \hyperref[TEI.gloss]{gloss} \hyperref[TEI.head]{head} \hyperref[TEI.headItem]{headItem} \hyperref[TEI.headLabel]{headLabel} \hyperref[TEI.hi]{hi} \hyperref[TEI.item]{item} \hyperref[TEI.l]{l} \hyperref[TEI.label]{label} \hyperref[TEI.measure]{measure} \hyperref[TEI.mentioned]{mentioned} \hyperref[TEI.name]{name} \hyperref[TEI.note]{note} \hyperref[TEI.num]{num} \hyperref[TEI.orig]{orig} \hyperref[TEI.p]{p} \hyperref[TEI.pubPlace]{pubPlace} \hyperref[TEI.publisher]{publisher} \hyperref[TEI.q]{q} \hyperref[TEI.quote]{quote} \hyperref[TEI.ref]{ref} \hyperref[TEI.reg]{reg} \hyperref[TEI.rs]{rs} \hyperref[TEI.said]{said} \hyperref[TEI.sic]{sic} \hyperref[TEI.soCalled]{soCalled} \hyperref[TEI.speaker]{speaker} \hyperref[TEI.stage]{stage} \hyperref[TEI.street]{street} \hyperref[TEI.term]{term} \hyperref[TEI.textLang]{textLang} \hyperref[TEI.time]{time} \hyperref[TEI.title]{title} \hyperref[TEI.unclear]{unclear}\par 
    \item[figures: ]
   \hyperref[TEI.cell]{cell}\par 
    \item[header: ]
   \hyperref[TEI.change]{change} \hyperref[TEI.distributor]{distributor} \hyperref[TEI.edition]{edition} \hyperref[TEI.extent]{extent} \hyperref[TEI.licence]{licence}\par 
    \item[linking: ]
   \hyperref[TEI.ab]{ab} \hyperref[TEI.seg]{seg}\par 
    \item[msdescription: ]
   \hyperref[TEI.accMat]{accMat} \hyperref[TEI.acquisition]{acquisition} \hyperref[TEI.additions]{additions} \hyperref[TEI.catchwords]{catchwords} \hyperref[TEI.collation]{collation} \hyperref[TEI.colophon]{colophon} \hyperref[TEI.condition]{condition} \hyperref[TEI.custEvent]{custEvent} \hyperref[TEI.decoNote]{decoNote} \hyperref[TEI.explicit]{explicit} \hyperref[TEI.filiation]{filiation} \hyperref[TEI.finalRubric]{finalRubric} \hyperref[TEI.foliation]{foliation} \hyperref[TEI.heraldry]{heraldry} \hyperref[TEI.incipit]{incipit} \hyperref[TEI.layout]{layout} \hyperref[TEI.material]{material} \hyperref[TEI.musicNotation]{musicNotation} \hyperref[TEI.objectType]{objectType} \hyperref[TEI.origDate]{origDate} \hyperref[TEI.origPlace]{origPlace} \hyperref[TEI.origin]{origin} \hyperref[TEI.provenance]{provenance} \hyperref[TEI.rubric]{rubric} \hyperref[TEI.secFol]{secFol} \hyperref[TEI.signatures]{signatures} \hyperref[TEI.source]{source} \hyperref[TEI.stamp]{stamp} \hyperref[TEI.summary]{summary} \hyperref[TEI.support]{support} \hyperref[TEI.surrogates]{surrogates} \hyperref[TEI.typeNote]{typeNote} \hyperref[TEI.watermark]{watermark}\par 
    \item[namesdates: ]
   \hyperref[TEI.addName]{addName} \hyperref[TEI.affiliation]{affiliation} \hyperref[TEI.country]{country} \hyperref[TEI.forename]{forename} \hyperref[TEI.genName]{genName} \hyperref[TEI.geogName]{geogName} \hyperref[TEI.nameLink]{nameLink} \hyperref[TEI.orgName]{orgName} \hyperref[TEI.persName]{persName} \hyperref[TEI.placeName]{placeName} \hyperref[TEI.region]{region} \hyperref[TEI.roleName]{roleName} \hyperref[TEI.settlement]{settlement} \hyperref[TEI.surname]{surname}\par 
    \item[textstructure: ]
   \hyperref[TEI.docAuthor]{docAuthor} \hyperref[TEI.docDate]{docDate} \hyperref[TEI.docEdition]{docEdition} \hyperref[TEI.titlePart]{titlePart}\par 
    \item[transcr: ]
   \hyperref[TEI.am]{am} \hyperref[TEI.damage]{damage} \hyperref[TEI.fw]{fw} \hyperref[TEI.line]{line} \hyperref[TEI.metamark]{metamark} \hyperref[TEI.mod]{mod} \hyperref[TEI.restore]{restore} \hyperref[TEI.retrace]{retrace} \hyperref[TEI.secl]{secl} \hyperref[TEI.supplied]{supplied} \hyperref[TEI.surplus]{surplus} \hyperref[TEI.zone]{zone}
    \item[{Peut contenir}]
  
    \item[analysis: ]
   \hyperref[TEI.c]{c} \hyperref[TEI.cl]{cl} \hyperref[TEI.interp]{interp} \hyperref[TEI.interpGrp]{interpGrp} \hyperref[TEI.m]{m} \hyperref[TEI.pc]{pc} \hyperref[TEI.phr]{phr} \hyperref[TEI.s]{s} \hyperref[TEI.span]{span} \hyperref[TEI.spanGrp]{spanGrp} \hyperref[TEI.w]{w}\par 
    \item[core: ]
   \hyperref[TEI.abbr]{abbr} \hyperref[TEI.add]{add} \hyperref[TEI.address]{address} \hyperref[TEI.bibl]{bibl} \hyperref[TEI.biblStruct]{biblStruct} \hyperref[TEI.binaryObject]{binaryObject} \hyperref[TEI.cb]{cb} \hyperref[TEI.choice]{choice} \hyperref[TEI.cit]{cit} \hyperref[TEI.corr]{corr} \hyperref[TEI.date]{date} \hyperref[TEI.del]{del} \hyperref[TEI.desc]{desc} \hyperref[TEI.distinct]{distinct} \hyperref[TEI.email]{email} \hyperref[TEI.emph]{emph} \hyperref[TEI.expan]{expan} \hyperref[TEI.foreign]{foreign} \hyperref[TEI.gap]{gap} \hyperref[TEI.gb]{gb} \hyperref[TEI.gloss]{gloss} \hyperref[TEI.graphic]{graphic} \hyperref[TEI.hi]{hi} \hyperref[TEI.index]{index} \hyperref[TEI.l]{l} \hyperref[TEI.label]{label} \hyperref[TEI.lb]{lb} \hyperref[TEI.lg]{lg} \hyperref[TEI.list]{list} \hyperref[TEI.listBibl]{listBibl} \hyperref[TEI.measure]{measure} \hyperref[TEI.measureGrp]{measureGrp} \hyperref[TEI.media]{media} \hyperref[TEI.mentioned]{mentioned} \hyperref[TEI.milestone]{milestone} \hyperref[TEI.name]{name} \hyperref[TEI.note]{note} \hyperref[TEI.num]{num} \hyperref[TEI.orig]{orig} \hyperref[TEI.pb]{pb} \hyperref[TEI.ptr]{ptr} \hyperref[TEI.q]{q} \hyperref[TEI.quote]{quote} \hyperref[TEI.ref]{ref} \hyperref[TEI.reg]{reg} \hyperref[TEI.rs]{rs} \hyperref[TEI.said]{said} \hyperref[TEI.sic]{sic} \hyperref[TEI.soCalled]{soCalled} \hyperref[TEI.stage]{stage} \hyperref[TEI.term]{term} \hyperref[TEI.time]{time} \hyperref[TEI.title]{title} \hyperref[TEI.unclear]{unclear}\par 
    \item[derived-module-tei.istex: ]
   \hyperref[TEI.math]{math} \hyperref[TEI.mrow]{mrow}\par 
    \item[figures: ]
   \hyperref[TEI.figure]{figure} \hyperref[TEI.formula]{formula} \hyperref[TEI.notatedMusic]{notatedMusic} \hyperref[TEI.table]{table}\par 
    \item[header: ]
   \hyperref[TEI.biblFull]{biblFull} \hyperref[TEI.idno]{idno}\par 
    \item[iso-fs: ]
   \hyperref[TEI.fLib]{fLib} \hyperref[TEI.fs]{fs} \hyperref[TEI.fvLib]{fvLib}\par 
    \item[linking: ]
   \hyperref[TEI.alt]{alt} \hyperref[TEI.altGrp]{altGrp} \hyperref[TEI.anchor]{anchor} \hyperref[TEI.join]{join} \hyperref[TEI.joinGrp]{joinGrp} \hyperref[TEI.link]{link} \hyperref[TEI.linkGrp]{linkGrp} \hyperref[TEI.seg]{seg} \hyperref[TEI.timeline]{timeline}\par 
    \item[msdescription: ]
   \hyperref[TEI.catchwords]{catchwords} \hyperref[TEI.depth]{depth} \hyperref[TEI.dim]{dim} \hyperref[TEI.dimensions]{dimensions} \hyperref[TEI.height]{height} \hyperref[TEI.heraldry]{heraldry} \hyperref[TEI.locus]{locus} \hyperref[TEI.locusGrp]{locusGrp} \hyperref[TEI.material]{material} \hyperref[TEI.msDesc]{msDesc} \hyperref[TEI.objectType]{objectType} \hyperref[TEI.origDate]{origDate} \hyperref[TEI.origPlace]{origPlace} \hyperref[TEI.secFol]{secFol} \hyperref[TEI.signatures]{signatures} \hyperref[TEI.source]{source} \hyperref[TEI.stamp]{stamp} \hyperref[TEI.watermark]{watermark} \hyperref[TEI.width]{width}\par 
    \item[namesdates: ]
   \hyperref[TEI.addName]{addName} \hyperref[TEI.affiliation]{affiliation} \hyperref[TEI.country]{country} \hyperref[TEI.forename]{forename} \hyperref[TEI.genName]{genName} \hyperref[TEI.geogName]{geogName} \hyperref[TEI.listOrg]{listOrg} \hyperref[TEI.listPlace]{listPlace} \hyperref[TEI.location]{location} \hyperref[TEI.nameLink]{nameLink} \hyperref[TEI.orgName]{orgName} \hyperref[TEI.persName]{persName} \hyperref[TEI.placeName]{placeName} \hyperref[TEI.region]{region} \hyperref[TEI.roleName]{roleName} \hyperref[TEI.settlement]{settlement} \hyperref[TEI.state]{state} \hyperref[TEI.surname]{surname}\par 
    \item[spoken: ]
   \hyperref[TEI.annotationBlock]{annotationBlock}\par 
    \item[textstructure: ]
   \hyperref[TEI.floatingText]{floatingText}\par 
    \item[transcr: ]
   \hyperref[TEI.addSpan]{addSpan} \hyperref[TEI.am]{am} \hyperref[TEI.damage]{damage} \hyperref[TEI.damageSpan]{damageSpan} \hyperref[TEI.delSpan]{delSpan} \hyperref[TEI.ex]{ex} \hyperref[TEI.fw]{fw} \hyperref[TEI.handShift]{handShift} \hyperref[TEI.listTranspose]{listTranspose} \hyperref[TEI.metamark]{metamark} \hyperref[TEI.mod]{mod} \hyperref[TEI.redo]{redo} \hyperref[TEI.restore]{restore} \hyperref[TEI.retrace]{retrace} \hyperref[TEI.secl]{secl} \hyperref[TEI.space]{space} \hyperref[TEI.subst]{subst} \hyperref[TEI.substJoin]{substJoin} \hyperref[TEI.supplied]{supplied} \hyperref[TEI.surplus]{surplus} \hyperref[TEI.undo]{undo}\par des données textuelles
    \item[{Note}]
  \par
Multiple retraces are indicated by nesting one \hyperref[TEI.retrace]{<retrace>} within another. In principle, a retrace differs from a substitution in that second and subsequent rewrites do not materially alter the content of an element. Where minor changes have been made during the retracing action however these may be marked up using \hyperref[TEI.del]{<del>}, \hyperref[TEI.add]{<add>}, etc. with an appropriate value for the {\itshape change} attribute.
    \item[{Modèle de contenu}]
  \mbox{}\hfill\\[-10pt]\begin{Verbatim}[fontsize=\small]
<content>
 <macroRef key="macro.paraContent"/>
</content>
    
\end{Verbatim}

    \item[{Schéma Declaration}]
  \mbox{}\hfill\\[-10pt]\begin{Verbatim}[fontsize=\small]
element retrace
{
   tei_att.global.attributes,
   tei_att.spanning.attributes,
   tei_att.transcriptional.attributes,
   tei_macro.paraContent}
\end{Verbatim}

\end{reflist}  \index{revisionDesc=<revisionDesc>|oddindex}
\begin{reflist}
\item[]\begin{specHead}{TEI.revisionDesc}{<revisionDesc> }(descriptif des révisions) fournit un résumé de l’historique des révisions d’un fichier. [\xref{http://www.tei-c.org/release/doc/tei-p5-doc/en/html/HD.html\#HD6}{2.6. The Revision Description} \xref{http://www.tei-c.org/release/doc/tei-p5-doc/en/html/HD.html\#HD11}{2.1.1. The TEI Header and Its Components}]\end{specHead} 
    \item[{Module}]
  header
    \item[{Attributs}]
  Attributs \hyperref[TEI.att.global]{att.global} (\textit{@xml:id}, \textit{@n}, \textit{@xml:lang}, \textit{@xml:base}, \textit{@xml:space})  (\hyperref[TEI.att.global.rendition]{att.global.rendition} (\textit{@rend}, \textit{@style}, \textit{@rendition})) (\hyperref[TEI.att.global.linking]{att.global.linking} (\textit{@corresp}, \textit{@synch}, \textit{@sameAs}, \textit{@copyOf}, \textit{@next}, \textit{@prev}, \textit{@exclude}, \textit{@select})) (\hyperref[TEI.att.global.analytic]{att.global.analytic} (\textit{@ana})) (\hyperref[TEI.att.global.facs]{att.global.facs} (\textit{@facs})) (\hyperref[TEI.att.global.change]{att.global.change} (\textit{@change})) (\hyperref[TEI.att.global.responsibility]{att.global.responsibility} (\textit{@cert}, \textit{@resp})) (\hyperref[TEI.att.global.source]{att.global.source} (\textit{@source})) \hyperref[TEI.att.docStatus]{att.docStatus} (\textit{@status}) 
    \item[{Contenu dans}]
  
    \item[header: ]
   \hyperref[TEI.teiHeader]{teiHeader}
    \item[{Peut contenir}]
  
    \item[core: ]
   \hyperref[TEI.list]{list}\par 
    \item[header: ]
   \hyperref[TEI.change]{change}
    \item[{Note}]
  \par
Les changements les plus récents apparaissent en début de liste
    \item[{Exemple}]
  \leavevmode\bgroup\exampleFont \begin{shaded}\noindent\mbox{}{<\textbf{revisionDesc}>}\mbox{}\newline 
\hspace*{6pt}{<\textbf{list}>}\mbox{}\newline 
\hspace*{6pt}\hspace*{6pt}{<\textbf{item}>}\mbox{}\newline 
\hspace*{6pt}\hspace*{6pt}\hspace*{6pt}{<\textbf{date}\hspace*{6pt}{when}="{2003-04-12}">}12 avril 03{</\textbf{date}>}Dernière révision par F. B.{</\textbf{item}>}\mbox{}\newline 
\hspace*{6pt}\hspace*{6pt}{<\textbf{item}>}\mbox{}\newline 
\hspace*{6pt}\hspace*{6pt}\hspace*{6pt}{<\textbf{date}\hspace*{6pt}{when}="{2003-03-01}">}1 mars 03{</\textbf{date}>} F.B a fait le nouveau fichier.{</\textbf{item}>}\mbox{}\newline 
\hspace*{6pt}{</\textbf{list}>}\mbox{}\newline 
{</\textbf{revisionDesc}>}\end{shaded}\egroup 


    \item[{Modèle de contenu}]
  \mbox{}\hfill\\[-10pt]\begin{Verbatim}[fontsize=\small]
<content>
 <alternate maxOccurs="1" minOccurs="1">
  <elementRef key="list"/>
  <elementRef key="listChange"/>
  <elementRef key="change"
   maxOccurs="unbounded" minOccurs="1"/>
 </alternate>
</content>
    
\end{Verbatim}

    \item[{Schéma Declaration}]
  \mbox{}\hfill\\[-10pt]\begin{Verbatim}[fontsize=\small]
element revisionDesc
{
   tei_att.global.attributes,
   tei_att.docStatus.attributes,
   ( tei_list | listChange | tei_change+ )
}
\end{Verbatim}

\end{reflist}  \index{roleName=<roleName>|oddindex}
\begin{reflist}
\item[]\begin{specHead}{TEI.roleName}{<roleName> }(rôle) contient un composant du nom d'une personne, indiquant que celle-ci a un rôle ou une position particulière dans la société, comme un titre ou un rang officiel. [\xref{http://www.tei-c.org/release/doc/tei-p5-doc/en/html/ND.html\#NDPER}{13.2.1. Personal Names}]\end{specHead} 
    \item[{Module}]
  namesdates
    \item[{Attributs}]
  Attributs \hyperref[TEI.att.global]{att.global} (\textit{@xml:id}, \textit{@n}, \textit{@xml:lang}, \textit{@xml:base}, \textit{@xml:space})  (\hyperref[TEI.att.global.rendition]{att.global.rendition} (\textit{@rend}, \textit{@style}, \textit{@rendition})) (\hyperref[TEI.att.global.linking]{att.global.linking} (\textit{@corresp}, \textit{@synch}, \textit{@sameAs}, \textit{@copyOf}, \textit{@next}, \textit{@prev}, \textit{@exclude}, \textit{@select})) (\hyperref[TEI.att.global.analytic]{att.global.analytic} (\textit{@ana})) (\hyperref[TEI.att.global.facs]{att.global.facs} (\textit{@facs})) (\hyperref[TEI.att.global.change]{att.global.change} (\textit{@change})) (\hyperref[TEI.att.global.responsibility]{att.global.responsibility} (\textit{@cert}, \textit{@resp})) (\hyperref[TEI.att.global.source]{att.global.source} (\textit{@source})) \hyperref[TEI.att.personal]{att.personal} (\textit{@full}, \textit{@sort})  (\hyperref[TEI.att.naming]{att.naming} (\textit{@role}, \textit{@nymRef}) (\hyperref[TEI.att.canonical]{att.canonical} (\textit{@key}, \textit{@ref})) ) \hyperref[TEI.att.typed]{att.typed} (\textit{@type}, \textit{@subtype}) 
    \item[{Membre du}]
  \hyperref[TEI.model.persNamePart]{model.persNamePart}
    \item[{Contenu dans}]
  
    \item[analysis: ]
   \hyperref[TEI.cl]{cl} \hyperref[TEI.phr]{phr} \hyperref[TEI.s]{s} \hyperref[TEI.span]{span}\par 
    \item[core: ]
   \hyperref[TEI.abbr]{abbr} \hyperref[TEI.add]{add} \hyperref[TEI.addrLine]{addrLine} \hyperref[TEI.address]{address} \hyperref[TEI.author]{author} \hyperref[TEI.bibl]{bibl} \hyperref[TEI.biblScope]{biblScope} \hyperref[TEI.citedRange]{citedRange} \hyperref[TEI.corr]{corr} \hyperref[TEI.date]{date} \hyperref[TEI.del]{del} \hyperref[TEI.desc]{desc} \hyperref[TEI.distinct]{distinct} \hyperref[TEI.editor]{editor} \hyperref[TEI.email]{email} \hyperref[TEI.emph]{emph} \hyperref[TEI.expan]{expan} \hyperref[TEI.foreign]{foreign} \hyperref[TEI.gloss]{gloss} \hyperref[TEI.head]{head} \hyperref[TEI.headItem]{headItem} \hyperref[TEI.headLabel]{headLabel} \hyperref[TEI.hi]{hi} \hyperref[TEI.item]{item} \hyperref[TEI.l]{l} \hyperref[TEI.label]{label} \hyperref[TEI.measure]{measure} \hyperref[TEI.meeting]{meeting} \hyperref[TEI.mentioned]{mentioned} \hyperref[TEI.name]{name} \hyperref[TEI.note]{note} \hyperref[TEI.num]{num} \hyperref[TEI.orig]{orig} \hyperref[TEI.p]{p} \hyperref[TEI.pubPlace]{pubPlace} \hyperref[TEI.publisher]{publisher} \hyperref[TEI.q]{q} \hyperref[TEI.quote]{quote} \hyperref[TEI.ref]{ref} \hyperref[TEI.reg]{reg} \hyperref[TEI.resp]{resp} \hyperref[TEI.rs]{rs} \hyperref[TEI.said]{said} \hyperref[TEI.sic]{sic} \hyperref[TEI.soCalled]{soCalled} \hyperref[TEI.speaker]{speaker} \hyperref[TEI.stage]{stage} \hyperref[TEI.street]{street} \hyperref[TEI.term]{term} \hyperref[TEI.textLang]{textLang} \hyperref[TEI.time]{time} \hyperref[TEI.title]{title} \hyperref[TEI.unclear]{unclear}\par 
    \item[figures: ]
   \hyperref[TEI.cell]{cell} \hyperref[TEI.figDesc]{figDesc}\par 
    \item[header: ]
   \hyperref[TEI.authority]{authority} \hyperref[TEI.change]{change} \hyperref[TEI.classCode]{classCode} \hyperref[TEI.creation]{creation} \hyperref[TEI.distributor]{distributor} \hyperref[TEI.edition]{edition} \hyperref[TEI.extent]{extent} \hyperref[TEI.funder]{funder} \hyperref[TEI.language]{language} \hyperref[TEI.licence]{licence} \hyperref[TEI.rendition]{rendition}\par 
    \item[iso-fs: ]
   \hyperref[TEI.fDescr]{fDescr} \hyperref[TEI.fsDescr]{fsDescr}\par 
    \item[linking: ]
   \hyperref[TEI.ab]{ab} \hyperref[TEI.seg]{seg}\par 
    \item[msdescription: ]
   \hyperref[TEI.accMat]{accMat} \hyperref[TEI.acquisition]{acquisition} \hyperref[TEI.additions]{additions} \hyperref[TEI.catchwords]{catchwords} \hyperref[TEI.collation]{collation} \hyperref[TEI.colophon]{colophon} \hyperref[TEI.condition]{condition} \hyperref[TEI.custEvent]{custEvent} \hyperref[TEI.decoNote]{decoNote} \hyperref[TEI.explicit]{explicit} \hyperref[TEI.filiation]{filiation} \hyperref[TEI.finalRubric]{finalRubric} \hyperref[TEI.foliation]{foliation} \hyperref[TEI.heraldry]{heraldry} \hyperref[TEI.incipit]{incipit} \hyperref[TEI.layout]{layout} \hyperref[TEI.material]{material} \hyperref[TEI.musicNotation]{musicNotation} \hyperref[TEI.objectType]{objectType} \hyperref[TEI.origDate]{origDate} \hyperref[TEI.origPlace]{origPlace} \hyperref[TEI.origin]{origin} \hyperref[TEI.provenance]{provenance} \hyperref[TEI.rubric]{rubric} \hyperref[TEI.secFol]{secFol} \hyperref[TEI.signatures]{signatures} \hyperref[TEI.source]{source} \hyperref[TEI.stamp]{stamp} \hyperref[TEI.summary]{summary} \hyperref[TEI.support]{support} \hyperref[TEI.surrogates]{surrogates} \hyperref[TEI.typeNote]{typeNote} \hyperref[TEI.watermark]{watermark}\par 
    \item[namesdates: ]
   \hyperref[TEI.addName]{addName} \hyperref[TEI.affiliation]{affiliation} \hyperref[TEI.country]{country} \hyperref[TEI.forename]{forename} \hyperref[TEI.genName]{genName} \hyperref[TEI.geogName]{geogName} \hyperref[TEI.nameLink]{nameLink} \hyperref[TEI.org]{org} \hyperref[TEI.orgName]{orgName} \hyperref[TEI.persName]{persName} \hyperref[TEI.placeName]{placeName} \hyperref[TEI.region]{region} \hyperref[TEI.roleName]{roleName} \hyperref[TEI.settlement]{settlement} \hyperref[TEI.surname]{surname}\par 
    \item[spoken: ]
   \hyperref[TEI.annotationBlock]{annotationBlock}\par 
    \item[standOff: ]
   \hyperref[TEI.listAnnotation]{listAnnotation}\par 
    \item[textstructure: ]
   \hyperref[TEI.docAuthor]{docAuthor} \hyperref[TEI.docDate]{docDate} \hyperref[TEI.docEdition]{docEdition} \hyperref[TEI.titlePart]{titlePart}\par 
    \item[transcr: ]
   \hyperref[TEI.damage]{damage} \hyperref[TEI.fw]{fw} \hyperref[TEI.metamark]{metamark} \hyperref[TEI.mod]{mod} \hyperref[TEI.restore]{restore} \hyperref[TEI.retrace]{retrace} \hyperref[TEI.secl]{secl} \hyperref[TEI.supplied]{supplied} \hyperref[TEI.surplus]{surplus}
    \item[{Peut contenir}]
  
    \item[analysis: ]
   \hyperref[TEI.c]{c} \hyperref[TEI.cl]{cl} \hyperref[TEI.interp]{interp} \hyperref[TEI.interpGrp]{interpGrp} \hyperref[TEI.m]{m} \hyperref[TEI.pc]{pc} \hyperref[TEI.phr]{phr} \hyperref[TEI.s]{s} \hyperref[TEI.span]{span} \hyperref[TEI.spanGrp]{spanGrp} \hyperref[TEI.w]{w}\par 
    \item[core: ]
   \hyperref[TEI.abbr]{abbr} \hyperref[TEI.add]{add} \hyperref[TEI.address]{address} \hyperref[TEI.binaryObject]{binaryObject} \hyperref[TEI.cb]{cb} \hyperref[TEI.choice]{choice} \hyperref[TEI.corr]{corr} \hyperref[TEI.date]{date} \hyperref[TEI.del]{del} \hyperref[TEI.distinct]{distinct} \hyperref[TEI.email]{email} \hyperref[TEI.emph]{emph} \hyperref[TEI.expan]{expan} \hyperref[TEI.foreign]{foreign} \hyperref[TEI.gap]{gap} \hyperref[TEI.gb]{gb} \hyperref[TEI.gloss]{gloss} \hyperref[TEI.graphic]{graphic} \hyperref[TEI.hi]{hi} \hyperref[TEI.index]{index} \hyperref[TEI.lb]{lb} \hyperref[TEI.measure]{measure} \hyperref[TEI.measureGrp]{measureGrp} \hyperref[TEI.media]{media} \hyperref[TEI.mentioned]{mentioned} \hyperref[TEI.milestone]{milestone} \hyperref[TEI.name]{name} \hyperref[TEI.note]{note} \hyperref[TEI.num]{num} \hyperref[TEI.orig]{orig} \hyperref[TEI.pb]{pb} \hyperref[TEI.ptr]{ptr} \hyperref[TEI.ref]{ref} \hyperref[TEI.reg]{reg} \hyperref[TEI.rs]{rs} \hyperref[TEI.sic]{sic} \hyperref[TEI.soCalled]{soCalled} \hyperref[TEI.term]{term} \hyperref[TEI.time]{time} \hyperref[TEI.title]{title} \hyperref[TEI.unclear]{unclear}\par 
    \item[derived-module-tei.istex: ]
   \hyperref[TEI.math]{math} \hyperref[TEI.mrow]{mrow}\par 
    \item[figures: ]
   \hyperref[TEI.figure]{figure} \hyperref[TEI.formula]{formula} \hyperref[TEI.notatedMusic]{notatedMusic}\par 
    \item[header: ]
   \hyperref[TEI.idno]{idno}\par 
    \item[iso-fs: ]
   \hyperref[TEI.fLib]{fLib} \hyperref[TEI.fs]{fs} \hyperref[TEI.fvLib]{fvLib}\par 
    \item[linking: ]
   \hyperref[TEI.alt]{alt} \hyperref[TEI.altGrp]{altGrp} \hyperref[TEI.anchor]{anchor} \hyperref[TEI.join]{join} \hyperref[TEI.joinGrp]{joinGrp} \hyperref[TEI.link]{link} \hyperref[TEI.linkGrp]{linkGrp} \hyperref[TEI.seg]{seg} \hyperref[TEI.timeline]{timeline}\par 
    \item[msdescription: ]
   \hyperref[TEI.catchwords]{catchwords} \hyperref[TEI.depth]{depth} \hyperref[TEI.dim]{dim} \hyperref[TEI.dimensions]{dimensions} \hyperref[TEI.height]{height} \hyperref[TEI.heraldry]{heraldry} \hyperref[TEI.locus]{locus} \hyperref[TEI.locusGrp]{locusGrp} \hyperref[TEI.material]{material} \hyperref[TEI.objectType]{objectType} \hyperref[TEI.origDate]{origDate} \hyperref[TEI.origPlace]{origPlace} \hyperref[TEI.secFol]{secFol} \hyperref[TEI.signatures]{signatures} \hyperref[TEI.source]{source} \hyperref[TEI.stamp]{stamp} \hyperref[TEI.watermark]{watermark} \hyperref[TEI.width]{width}\par 
    \item[namesdates: ]
   \hyperref[TEI.addName]{addName} \hyperref[TEI.affiliation]{affiliation} \hyperref[TEI.country]{country} \hyperref[TEI.forename]{forename} \hyperref[TEI.genName]{genName} \hyperref[TEI.geogName]{geogName} \hyperref[TEI.location]{location} \hyperref[TEI.nameLink]{nameLink} \hyperref[TEI.orgName]{orgName} \hyperref[TEI.persName]{persName} \hyperref[TEI.placeName]{placeName} \hyperref[TEI.region]{region} \hyperref[TEI.roleName]{roleName} \hyperref[TEI.settlement]{settlement} \hyperref[TEI.state]{state} \hyperref[TEI.surname]{surname}\par 
    \item[spoken: ]
   \hyperref[TEI.annotationBlock]{annotationBlock}\par 
    \item[transcr: ]
   \hyperref[TEI.addSpan]{addSpan} \hyperref[TEI.am]{am} \hyperref[TEI.damage]{damage} \hyperref[TEI.damageSpan]{damageSpan} \hyperref[TEI.delSpan]{delSpan} \hyperref[TEI.ex]{ex} \hyperref[TEI.fw]{fw} \hyperref[TEI.handShift]{handShift} \hyperref[TEI.listTranspose]{listTranspose} \hyperref[TEI.metamark]{metamark} \hyperref[TEI.mod]{mod} \hyperref[TEI.redo]{redo} \hyperref[TEI.restore]{restore} \hyperref[TEI.retrace]{retrace} \hyperref[TEI.secl]{secl} \hyperref[TEI.space]{space} \hyperref[TEI.subst]{subst} \hyperref[TEI.substJoin]{substJoin} \hyperref[TEI.supplied]{supplied} \hyperref[TEI.surplus]{surplus} \hyperref[TEI.undo]{undo}\par des données textuelles
    \item[{Note}]
  \par
Un élément \hyperref[TEI.roleName]{<roleName>} peut être distingué d'un élément \hyperref[TEI.addName]{<addName>} du fait que, à l'instar d'un titre, il existe en général indépendamment de la personne qui le porte.
    \item[{Exemple}]
  \leavevmode\bgroup\exampleFont \begin{shaded}\noindent\mbox{}{<\textbf{persName}>}\mbox{}\newline 
\hspace*{6pt}{<\textbf{forename}>}Joachim{</\textbf{forename}>}\mbox{}\newline 
\hspace*{6pt}{<\textbf{surname}>}Murat{</\textbf{surname}>}, {<\textbf{roleName}>}roi de Naples{</\textbf{roleName}>}\mbox{}\newline 
{</\textbf{persName}>}\end{shaded}\egroup 


    \item[{Modèle de contenu}]
  \mbox{}\hfill\\[-10pt]\begin{Verbatim}[fontsize=\small]
<content>
 <macroRef key="macro.phraseSeq"/>
</content>
    
\end{Verbatim}

    \item[{Schéma Declaration}]
  \mbox{}\hfill\\[-10pt]\begin{Verbatim}[fontsize=\small]
element roleName
{
   tei_att.global.attributes,
   tei_att.personal.attributes,
   tei_att.typed.attributes,
   tei_macro.phraseSeq}
\end{Verbatim}

\end{reflist}  \index{row=<row>|oddindex}
\begin{reflist}
\item[]\begin{specHead}{TEI.row}{<row> }(rangée) contient une rangée d'un tableau. [\xref{http://www.tei-c.org/release/doc/tei-p5-doc/en/html/FT.html\#FTTAB1}{14.1.1. TEI Tables}]\end{specHead} 
    \item[{Module}]
  figures
    \item[{Attributs}]
  Attributs \hyperref[TEI.att.global]{att.global} (\textit{@xml:id}, \textit{@n}, \textit{@xml:lang}, \textit{@xml:base}, \textit{@xml:space})  (\hyperref[TEI.att.global.rendition]{att.global.rendition} (\textit{@rend}, \textit{@style}, \textit{@rendition})) (\hyperref[TEI.att.global.linking]{att.global.linking} (\textit{@corresp}, \textit{@synch}, \textit{@sameAs}, \textit{@copyOf}, \textit{@next}, \textit{@prev}, \textit{@exclude}, \textit{@select})) (\hyperref[TEI.att.global.analytic]{att.global.analytic} (\textit{@ana})) (\hyperref[TEI.att.global.facs]{att.global.facs} (\textit{@facs})) (\hyperref[TEI.att.global.change]{att.global.change} (\textit{@change})) (\hyperref[TEI.att.global.responsibility]{att.global.responsibility} (\textit{@cert}, \textit{@resp})) (\hyperref[TEI.att.global.source]{att.global.source} (\textit{@source})) \hyperref[TEI.att.tableDecoration]{att.tableDecoration} (\textit{@role}, \textit{@rows}, \textit{@cols}) 
    \item[{Contenu dans}]
  
    \item[figures: ]
   \hyperref[TEI.table]{table}
    \item[{Peut contenir}]
  
    \item[figures: ]
   \hyperref[TEI.cell]{cell}
    \item[{Exemple}]
  \leavevmode\bgroup\exampleFont \begin{shaded}\noindent\mbox{}{<\textbf{row}\hspace*{6pt}{role}="{data}">}\mbox{}\newline 
\hspace*{6pt}{<\textbf{cell}\hspace*{6pt}{role}="{label}">}Etudes classiques{</\textbf{cell}>}\mbox{}\newline 
\hspace*{6pt}{<\textbf{cell}>}Inoccupé indolent et sans amélioration{</\textbf{cell}>}\mbox{}\newline 
{</\textbf{row}>}\end{shaded}\egroup 


    \item[{Modèle de contenu}]
  \mbox{}\hfill\\[-10pt]\begin{Verbatim}[fontsize=\small]
<content>
 <elementRef key="cell"
  maxOccurs="unbounded" minOccurs="1"/>
</content>
    
\end{Verbatim}

    \item[{Schéma Declaration}]
  \mbox{}\hfill\\[-10pt]\begin{Verbatim}[fontsize=\small]
element row
{
   tei_att.global.attributes,
   tei_att.tableDecoration.attributes,
   tei_cell+
}
\end{Verbatim}

\end{reflist}  \index{rs=<rs>|oddindex}
\begin{reflist}
\item[]\begin{specHead}{TEI.rs}{<rs> }(chaîne de référence) contient un nom générique ou une chaîne permettant de s'y référer. [\xref{http://www.tei-c.org/release/doc/tei-p5-doc/en/html/ND.html\#NDPER}{13.2.1. Personal Names} \xref{http://www.tei-c.org/release/doc/tei-p5-doc/en/html/CO.html\#CONARS}{3.5.1. Referring Strings}]\end{specHead} 
    \item[{Module}]
  core
    \item[{Attributs}]
  Attributs \hyperref[TEI.att.global]{att.global} (\textit{@xml:id}, \textit{@n}, \textit{@xml:lang}, \textit{@xml:base}, \textit{@xml:space})  (\hyperref[TEI.att.global.rendition]{att.global.rendition} (\textit{@rend}, \textit{@style}, \textit{@rendition})) (\hyperref[TEI.att.global.linking]{att.global.linking} (\textit{@corresp}, \textit{@synch}, \textit{@sameAs}, \textit{@copyOf}, \textit{@next}, \textit{@prev}, \textit{@exclude}, \textit{@select})) (\hyperref[TEI.att.global.analytic]{att.global.analytic} (\textit{@ana})) (\hyperref[TEI.att.global.facs]{att.global.facs} (\textit{@facs})) (\hyperref[TEI.att.global.change]{att.global.change} (\textit{@change})) (\hyperref[TEI.att.global.responsibility]{att.global.responsibility} (\textit{@cert}, \textit{@resp})) (\hyperref[TEI.att.global.source]{att.global.source} (\textit{@source})) \hyperref[TEI.att.naming]{att.naming} (\textit{@role}, \textit{@nymRef})  (\hyperref[TEI.att.canonical]{att.canonical} (\textit{@key}, \textit{@ref})) \hyperref[TEI.att.typed]{att.typed} (\textit{@type}, \textit{@subtype}) 
    \item[{Membre du}]
  \hyperref[TEI.model.nameLike]{model.nameLike} 
    \item[{Contenu dans}]
  
    \item[analysis: ]
   \hyperref[TEI.cl]{cl} \hyperref[TEI.phr]{phr} \hyperref[TEI.s]{s} \hyperref[TEI.span]{span}\par 
    \item[core: ]
   \hyperref[TEI.abbr]{abbr} \hyperref[TEI.add]{add} \hyperref[TEI.addrLine]{addrLine} \hyperref[TEI.address]{address} \hyperref[TEI.author]{author} \hyperref[TEI.bibl]{bibl} \hyperref[TEI.biblScope]{biblScope} \hyperref[TEI.citedRange]{citedRange} \hyperref[TEI.corr]{corr} \hyperref[TEI.date]{date} \hyperref[TEI.del]{del} \hyperref[TEI.desc]{desc} \hyperref[TEI.distinct]{distinct} \hyperref[TEI.editor]{editor} \hyperref[TEI.email]{email} \hyperref[TEI.emph]{emph} \hyperref[TEI.expan]{expan} \hyperref[TEI.foreign]{foreign} \hyperref[TEI.gloss]{gloss} \hyperref[TEI.head]{head} \hyperref[TEI.headItem]{headItem} \hyperref[TEI.headLabel]{headLabel} \hyperref[TEI.hi]{hi} \hyperref[TEI.item]{item} \hyperref[TEI.l]{l} \hyperref[TEI.label]{label} \hyperref[TEI.measure]{measure} \hyperref[TEI.meeting]{meeting} \hyperref[TEI.mentioned]{mentioned} \hyperref[TEI.name]{name} \hyperref[TEI.note]{note} \hyperref[TEI.num]{num} \hyperref[TEI.orig]{orig} \hyperref[TEI.p]{p} \hyperref[TEI.pubPlace]{pubPlace} \hyperref[TEI.publisher]{publisher} \hyperref[TEI.q]{q} \hyperref[TEI.quote]{quote} \hyperref[TEI.ref]{ref} \hyperref[TEI.reg]{reg} \hyperref[TEI.resp]{resp} \hyperref[TEI.rs]{rs} \hyperref[TEI.said]{said} \hyperref[TEI.sic]{sic} \hyperref[TEI.soCalled]{soCalled} \hyperref[TEI.speaker]{speaker} \hyperref[TEI.stage]{stage} \hyperref[TEI.street]{street} \hyperref[TEI.term]{term} \hyperref[TEI.textLang]{textLang} \hyperref[TEI.time]{time} \hyperref[TEI.title]{title} \hyperref[TEI.unclear]{unclear}\par 
    \item[figures: ]
   \hyperref[TEI.cell]{cell} \hyperref[TEI.figDesc]{figDesc}\par 
    \item[header: ]
   \hyperref[TEI.authority]{authority} \hyperref[TEI.change]{change} \hyperref[TEI.classCode]{classCode} \hyperref[TEI.creation]{creation} \hyperref[TEI.distributor]{distributor} \hyperref[TEI.edition]{edition} \hyperref[TEI.extent]{extent} \hyperref[TEI.funder]{funder} \hyperref[TEI.language]{language} \hyperref[TEI.licence]{licence} \hyperref[TEI.rendition]{rendition}\par 
    \item[iso-fs: ]
   \hyperref[TEI.fDescr]{fDescr} \hyperref[TEI.fsDescr]{fsDescr}\par 
    \item[linking: ]
   \hyperref[TEI.ab]{ab} \hyperref[TEI.seg]{seg}\par 
    \item[msdescription: ]
   \hyperref[TEI.accMat]{accMat} \hyperref[TEI.acquisition]{acquisition} \hyperref[TEI.additions]{additions} \hyperref[TEI.catchwords]{catchwords} \hyperref[TEI.collation]{collation} \hyperref[TEI.colophon]{colophon} \hyperref[TEI.condition]{condition} \hyperref[TEI.custEvent]{custEvent} \hyperref[TEI.decoNote]{decoNote} \hyperref[TEI.explicit]{explicit} \hyperref[TEI.filiation]{filiation} \hyperref[TEI.finalRubric]{finalRubric} \hyperref[TEI.foliation]{foliation} \hyperref[TEI.heraldry]{heraldry} \hyperref[TEI.incipit]{incipit} \hyperref[TEI.layout]{layout} \hyperref[TEI.material]{material} \hyperref[TEI.msName]{msName} \hyperref[TEI.musicNotation]{musicNotation} \hyperref[TEI.objectType]{objectType} \hyperref[TEI.origDate]{origDate} \hyperref[TEI.origPlace]{origPlace} \hyperref[TEI.origin]{origin} \hyperref[TEI.provenance]{provenance} \hyperref[TEI.rubric]{rubric} \hyperref[TEI.secFol]{secFol} \hyperref[TEI.signatures]{signatures} \hyperref[TEI.source]{source} \hyperref[TEI.stamp]{stamp} \hyperref[TEI.summary]{summary} \hyperref[TEI.support]{support} \hyperref[TEI.surrogates]{surrogates} \hyperref[TEI.typeNote]{typeNote} \hyperref[TEI.watermark]{watermark}\par 
    \item[namesdates: ]
   \hyperref[TEI.addName]{addName} \hyperref[TEI.affiliation]{affiliation} \hyperref[TEI.country]{country} \hyperref[TEI.forename]{forename} \hyperref[TEI.genName]{genName} \hyperref[TEI.geogName]{geogName} \hyperref[TEI.nameLink]{nameLink} \hyperref[TEI.org]{org} \hyperref[TEI.orgName]{orgName} \hyperref[TEI.persName]{persName} \hyperref[TEI.placeName]{placeName} \hyperref[TEI.region]{region} \hyperref[TEI.roleName]{roleName} \hyperref[TEI.settlement]{settlement} \hyperref[TEI.surname]{surname}\par 
    \item[spoken: ]
   \hyperref[TEI.annotationBlock]{annotationBlock}\par 
    \item[standOff: ]
   \hyperref[TEI.listAnnotation]{listAnnotation}\par 
    \item[textstructure: ]
   \hyperref[TEI.docAuthor]{docAuthor} \hyperref[TEI.docDate]{docDate} \hyperref[TEI.docEdition]{docEdition} \hyperref[TEI.titlePart]{titlePart}\par 
    \item[transcr: ]
   \hyperref[TEI.damage]{damage} \hyperref[TEI.fw]{fw} \hyperref[TEI.metamark]{metamark} \hyperref[TEI.mod]{mod} \hyperref[TEI.restore]{restore} \hyperref[TEI.retrace]{retrace} \hyperref[TEI.secl]{secl} \hyperref[TEI.supplied]{supplied} \hyperref[TEI.surplus]{surplus}
    \item[{Peut contenir}]
  
    \item[analysis: ]
   \hyperref[TEI.c]{c} \hyperref[TEI.cl]{cl} \hyperref[TEI.interp]{interp} \hyperref[TEI.interpGrp]{interpGrp} \hyperref[TEI.m]{m} \hyperref[TEI.pc]{pc} \hyperref[TEI.phr]{phr} \hyperref[TEI.s]{s} \hyperref[TEI.span]{span} \hyperref[TEI.spanGrp]{spanGrp} \hyperref[TEI.w]{w}\par 
    \item[core: ]
   \hyperref[TEI.abbr]{abbr} \hyperref[TEI.add]{add} \hyperref[TEI.address]{address} \hyperref[TEI.binaryObject]{binaryObject} \hyperref[TEI.cb]{cb} \hyperref[TEI.choice]{choice} \hyperref[TEI.corr]{corr} \hyperref[TEI.date]{date} \hyperref[TEI.del]{del} \hyperref[TEI.distinct]{distinct} \hyperref[TEI.email]{email} \hyperref[TEI.emph]{emph} \hyperref[TEI.expan]{expan} \hyperref[TEI.foreign]{foreign} \hyperref[TEI.gap]{gap} \hyperref[TEI.gb]{gb} \hyperref[TEI.gloss]{gloss} \hyperref[TEI.graphic]{graphic} \hyperref[TEI.hi]{hi} \hyperref[TEI.index]{index} \hyperref[TEI.lb]{lb} \hyperref[TEI.measure]{measure} \hyperref[TEI.measureGrp]{measureGrp} \hyperref[TEI.media]{media} \hyperref[TEI.mentioned]{mentioned} \hyperref[TEI.milestone]{milestone} \hyperref[TEI.name]{name} \hyperref[TEI.note]{note} \hyperref[TEI.num]{num} \hyperref[TEI.orig]{orig} \hyperref[TEI.pb]{pb} \hyperref[TEI.ptr]{ptr} \hyperref[TEI.ref]{ref} \hyperref[TEI.reg]{reg} \hyperref[TEI.rs]{rs} \hyperref[TEI.sic]{sic} \hyperref[TEI.soCalled]{soCalled} \hyperref[TEI.term]{term} \hyperref[TEI.time]{time} \hyperref[TEI.title]{title} \hyperref[TEI.unclear]{unclear}\par 
    \item[derived-module-tei.istex: ]
   \hyperref[TEI.math]{math} \hyperref[TEI.mrow]{mrow}\par 
    \item[figures: ]
   \hyperref[TEI.figure]{figure} \hyperref[TEI.formula]{formula} \hyperref[TEI.notatedMusic]{notatedMusic}\par 
    \item[header: ]
   \hyperref[TEI.idno]{idno}\par 
    \item[iso-fs: ]
   \hyperref[TEI.fLib]{fLib} \hyperref[TEI.fs]{fs} \hyperref[TEI.fvLib]{fvLib}\par 
    \item[linking: ]
   \hyperref[TEI.alt]{alt} \hyperref[TEI.altGrp]{altGrp} \hyperref[TEI.anchor]{anchor} \hyperref[TEI.join]{join} \hyperref[TEI.joinGrp]{joinGrp} \hyperref[TEI.link]{link} \hyperref[TEI.linkGrp]{linkGrp} \hyperref[TEI.seg]{seg} \hyperref[TEI.timeline]{timeline}\par 
    \item[msdescription: ]
   \hyperref[TEI.catchwords]{catchwords} \hyperref[TEI.depth]{depth} \hyperref[TEI.dim]{dim} \hyperref[TEI.dimensions]{dimensions} \hyperref[TEI.height]{height} \hyperref[TEI.heraldry]{heraldry} \hyperref[TEI.locus]{locus} \hyperref[TEI.locusGrp]{locusGrp} \hyperref[TEI.material]{material} \hyperref[TEI.objectType]{objectType} \hyperref[TEI.origDate]{origDate} \hyperref[TEI.origPlace]{origPlace} \hyperref[TEI.secFol]{secFol} \hyperref[TEI.signatures]{signatures} \hyperref[TEI.source]{source} \hyperref[TEI.stamp]{stamp} \hyperref[TEI.watermark]{watermark} \hyperref[TEI.width]{width}\par 
    \item[namesdates: ]
   \hyperref[TEI.addName]{addName} \hyperref[TEI.affiliation]{affiliation} \hyperref[TEI.country]{country} \hyperref[TEI.forename]{forename} \hyperref[TEI.genName]{genName} \hyperref[TEI.geogName]{geogName} \hyperref[TEI.location]{location} \hyperref[TEI.nameLink]{nameLink} \hyperref[TEI.orgName]{orgName} \hyperref[TEI.persName]{persName} \hyperref[TEI.placeName]{placeName} \hyperref[TEI.region]{region} \hyperref[TEI.roleName]{roleName} \hyperref[TEI.settlement]{settlement} \hyperref[TEI.state]{state} \hyperref[TEI.surname]{surname}\par 
    \item[spoken: ]
   \hyperref[TEI.annotationBlock]{annotationBlock}\par 
    \item[transcr: ]
   \hyperref[TEI.addSpan]{addSpan} \hyperref[TEI.am]{am} \hyperref[TEI.damage]{damage} \hyperref[TEI.damageSpan]{damageSpan} \hyperref[TEI.delSpan]{delSpan} \hyperref[TEI.ex]{ex} \hyperref[TEI.fw]{fw} \hyperref[TEI.handShift]{handShift} \hyperref[TEI.listTranspose]{listTranspose} \hyperref[TEI.metamark]{metamark} \hyperref[TEI.mod]{mod} \hyperref[TEI.redo]{redo} \hyperref[TEI.restore]{restore} \hyperref[TEI.retrace]{retrace} \hyperref[TEI.secl]{secl} \hyperref[TEI.space]{space} \hyperref[TEI.subst]{subst} \hyperref[TEI.substJoin]{substJoin} \hyperref[TEI.supplied]{supplied} \hyperref[TEI.surplus]{surplus} \hyperref[TEI.undo]{undo}\par des données textuelles
    \item[{Exemple}]
  \leavevmode\bgroup\exampleFont \begin{shaded}\noindent\mbox{}{<\textbf{p}>}La famille s'était alors retirée en banlieue, à {<\textbf{rs}\hspace*{6pt}{type}="{place}">}Villemomble{</\textbf{rs}>}, mais\mbox{}\newline 
{<\textbf{rs}\hspace*{6pt}{type}="{person}">}Alfred {</\textbf{rs}>}aimait se rendre à {<\textbf{rs}\hspace*{6pt}{type}="{place}">}Paris{</\textbf{rs}>} et un jour de 1917 alors qu'il sortait de {<\textbf{rs}\hspace*{6pt}{type}="{place}">}la maison des\mbox{}\newline 
\hspace*{6pt}\hspace*{6pt} Arts et Métiers{</\textbf{rs}>} il était tombé inanimé dans la rue.{</\textbf{p}>}\end{shaded}\egroup 


    \item[{Modèle de contenu}]
  \mbox{}\hfill\\[-10pt]\begin{Verbatim}[fontsize=\small]
<content>
 <macroRef key="macro.phraseSeq"/>
</content>
    
\end{Verbatim}

    \item[{Schéma Declaration}]
  \mbox{}\hfill\\[-10pt]\begin{Verbatim}[fontsize=\small]
element rs
{
   tei_att.global.attributes,
   tei_att.naming.attributes,
   tei_att.typed.attributes,
   tei_macro.phraseSeq}
\end{Verbatim}

\end{reflist}  \index{rubric=<rubric>|oddindex}
\begin{reflist}
\item[]\begin{specHead}{TEI.rubric}{<rubric> }(rubrique) contient le texte d'une \textit{rubrique} ou d'un intitulé propres à un item, c'est-à-dire des mots qui signalent le début du texte, qui incluent souvent la mention de son auteur et de son titre, et qui sont différenciés du texte lui-même, généralement à l'encre rouge, par une taille ou un style d'écriture particuliers, ou par tout autre procédé de ce genre. [\xref{http://www.tei-c.org/release/doc/tei-p5-doc/en/html/MS.html\#mscoit}{10.6.1. The msItem and msItemStruct Elements}]\end{specHead} 
    \item[{Module}]
  msdescription
    \item[{Attributs}]
  Attributs \hyperref[TEI.att.global]{att.global} (\textit{@xml:id}, \textit{@n}, \textit{@xml:lang}, \textit{@xml:base}, \textit{@xml:space})  (\hyperref[TEI.att.global.rendition]{att.global.rendition} (\textit{@rend}, \textit{@style}, \textit{@rendition})) (\hyperref[TEI.att.global.linking]{att.global.linking} (\textit{@corresp}, \textit{@synch}, \textit{@sameAs}, \textit{@copyOf}, \textit{@next}, \textit{@prev}, \textit{@exclude}, \textit{@select})) (\hyperref[TEI.att.global.analytic]{att.global.analytic} (\textit{@ana})) (\hyperref[TEI.att.global.facs]{att.global.facs} (\textit{@facs})) (\hyperref[TEI.att.global.change]{att.global.change} (\textit{@change})) (\hyperref[TEI.att.global.responsibility]{att.global.responsibility} (\textit{@cert}, \textit{@resp})) (\hyperref[TEI.att.global.source]{att.global.source} (\textit{@source})) \hyperref[TEI.att.typed]{att.typed} (\textit{@type}, \textit{@subtype}) \hyperref[TEI.att.msExcerpt]{att.msExcerpt} (\textit{@defective}) 
    \item[{Membre du}]
  \hyperref[TEI.model.msQuoteLike]{model.msQuoteLike} 
    \item[{Contenu dans}]
  
    \item[msdescription: ]
   \hyperref[TEI.msItem]{msItem} \hyperref[TEI.msItemStruct]{msItemStruct}
    \item[{Peut contenir}]
  
    \item[analysis: ]
   \hyperref[TEI.c]{c} \hyperref[TEI.cl]{cl} \hyperref[TEI.interp]{interp} \hyperref[TEI.interpGrp]{interpGrp} \hyperref[TEI.m]{m} \hyperref[TEI.pc]{pc} \hyperref[TEI.phr]{phr} \hyperref[TEI.s]{s} \hyperref[TEI.span]{span} \hyperref[TEI.spanGrp]{spanGrp} \hyperref[TEI.w]{w}\par 
    \item[core: ]
   \hyperref[TEI.abbr]{abbr} \hyperref[TEI.add]{add} \hyperref[TEI.address]{address} \hyperref[TEI.binaryObject]{binaryObject} \hyperref[TEI.cb]{cb} \hyperref[TEI.choice]{choice} \hyperref[TEI.corr]{corr} \hyperref[TEI.date]{date} \hyperref[TEI.del]{del} \hyperref[TEI.distinct]{distinct} \hyperref[TEI.email]{email} \hyperref[TEI.emph]{emph} \hyperref[TEI.expan]{expan} \hyperref[TEI.foreign]{foreign} \hyperref[TEI.gap]{gap} \hyperref[TEI.gb]{gb} \hyperref[TEI.gloss]{gloss} \hyperref[TEI.graphic]{graphic} \hyperref[TEI.hi]{hi} \hyperref[TEI.index]{index} \hyperref[TEI.lb]{lb} \hyperref[TEI.measure]{measure} \hyperref[TEI.measureGrp]{measureGrp} \hyperref[TEI.media]{media} \hyperref[TEI.mentioned]{mentioned} \hyperref[TEI.milestone]{milestone} \hyperref[TEI.name]{name} \hyperref[TEI.note]{note} \hyperref[TEI.num]{num} \hyperref[TEI.orig]{orig} \hyperref[TEI.pb]{pb} \hyperref[TEI.ptr]{ptr} \hyperref[TEI.ref]{ref} \hyperref[TEI.reg]{reg} \hyperref[TEI.rs]{rs} \hyperref[TEI.sic]{sic} \hyperref[TEI.soCalled]{soCalled} \hyperref[TEI.term]{term} \hyperref[TEI.time]{time} \hyperref[TEI.title]{title} \hyperref[TEI.unclear]{unclear}\par 
    \item[derived-module-tei.istex: ]
   \hyperref[TEI.math]{math} \hyperref[TEI.mrow]{mrow}\par 
    \item[figures: ]
   \hyperref[TEI.figure]{figure} \hyperref[TEI.formula]{formula} \hyperref[TEI.notatedMusic]{notatedMusic}\par 
    \item[header: ]
   \hyperref[TEI.idno]{idno}\par 
    \item[iso-fs: ]
   \hyperref[TEI.fLib]{fLib} \hyperref[TEI.fs]{fs} \hyperref[TEI.fvLib]{fvLib}\par 
    \item[linking: ]
   \hyperref[TEI.alt]{alt} \hyperref[TEI.altGrp]{altGrp} \hyperref[TEI.anchor]{anchor} \hyperref[TEI.join]{join} \hyperref[TEI.joinGrp]{joinGrp} \hyperref[TEI.link]{link} \hyperref[TEI.linkGrp]{linkGrp} \hyperref[TEI.seg]{seg} \hyperref[TEI.timeline]{timeline}\par 
    \item[msdescription: ]
   \hyperref[TEI.catchwords]{catchwords} \hyperref[TEI.depth]{depth} \hyperref[TEI.dim]{dim} \hyperref[TEI.dimensions]{dimensions} \hyperref[TEI.height]{height} \hyperref[TEI.heraldry]{heraldry} \hyperref[TEI.locus]{locus} \hyperref[TEI.locusGrp]{locusGrp} \hyperref[TEI.material]{material} \hyperref[TEI.objectType]{objectType} \hyperref[TEI.origDate]{origDate} \hyperref[TEI.origPlace]{origPlace} \hyperref[TEI.secFol]{secFol} \hyperref[TEI.signatures]{signatures} \hyperref[TEI.source]{source} \hyperref[TEI.stamp]{stamp} \hyperref[TEI.watermark]{watermark} \hyperref[TEI.width]{width}\par 
    \item[namesdates: ]
   \hyperref[TEI.addName]{addName} \hyperref[TEI.affiliation]{affiliation} \hyperref[TEI.country]{country} \hyperref[TEI.forename]{forename} \hyperref[TEI.genName]{genName} \hyperref[TEI.geogName]{geogName} \hyperref[TEI.location]{location} \hyperref[TEI.nameLink]{nameLink} \hyperref[TEI.orgName]{orgName} \hyperref[TEI.persName]{persName} \hyperref[TEI.placeName]{placeName} \hyperref[TEI.region]{region} \hyperref[TEI.roleName]{roleName} \hyperref[TEI.settlement]{settlement} \hyperref[TEI.state]{state} \hyperref[TEI.surname]{surname}\par 
    \item[spoken: ]
   \hyperref[TEI.annotationBlock]{annotationBlock}\par 
    \item[transcr: ]
   \hyperref[TEI.addSpan]{addSpan} \hyperref[TEI.am]{am} \hyperref[TEI.damage]{damage} \hyperref[TEI.damageSpan]{damageSpan} \hyperref[TEI.delSpan]{delSpan} \hyperref[TEI.ex]{ex} \hyperref[TEI.fw]{fw} \hyperref[TEI.handShift]{handShift} \hyperref[TEI.listTranspose]{listTranspose} \hyperref[TEI.metamark]{metamark} \hyperref[TEI.mod]{mod} \hyperref[TEI.redo]{redo} \hyperref[TEI.restore]{restore} \hyperref[TEI.retrace]{retrace} \hyperref[TEI.secl]{secl} \hyperref[TEI.space]{space} \hyperref[TEI.subst]{subst} \hyperref[TEI.substJoin]{substJoin} \hyperref[TEI.supplied]{supplied} \hyperref[TEI.surplus]{surplus} \hyperref[TEI.undo]{undo}\par des données textuelles
    \item[{Exemple}]
  \leavevmode\bgroup\exampleFont \begin{shaded}\noindent\mbox{}{<\textbf{rubric}>}Nu koma Skyckiu Rym{<\textbf{ex}>}ur{</\textbf{ex}>}.{</\textbf{rubric}>}\mbox{}\newline 
{<\textbf{rubric}>}Incipit liber de consciencia humana a beatissimo Bernardo editus.{</\textbf{rubric}>}\mbox{}\newline 
{<\textbf{rubric}>}\mbox{}\newline 
\hspace*{6pt}{<\textbf{locus}>}16. f. 28v in margin: {</\textbf{locus}>}Dicta Cassiodori\mbox{}\newline 
{</\textbf{rubric}>}\end{shaded}\egroup 


    \item[{Exemple}]
  \leavevmode\bgroup\exampleFont \begin{shaded}\noindent\mbox{}{<\textbf{rubric}>}Nu koma Skyckiu Rym{<\textbf{ex}>}ur{</\textbf{ex}>}.{</\textbf{rubric}>}\mbox{}\newline 
{<\textbf{rubric}>}Incipit liber de consciencia humana a beatissimo Bernardo editus.{</\textbf{rubric}>}\mbox{}\newline 
{<\textbf{rubric}>}\mbox{}\newline 
\hspace*{6pt}{<\textbf{locus}>}16. f. 28v in margin: {</\textbf{locus}>}Dicta Cassiodori\mbox{}\newline 
{</\textbf{rubric}>}\end{shaded}\egroup 


    \item[{Modèle de contenu}]
  \mbox{}\hfill\\[-10pt]\begin{Verbatim}[fontsize=\small]
<content>
 <macroRef key="macro.phraseSeq"/>
</content>
    
\end{Verbatim}

    \item[{Schéma Declaration}]
  \mbox{}\hfill\\[-10pt]\begin{Verbatim}[fontsize=\small]
element rubric
{
   tei_att.global.attributes,
   tei_att.typed.attributes,
   tei_att.msExcerpt.attributes,
   tei_macro.phraseSeq}
\end{Verbatim}

\end{reflist}  \index{s=<s>|oddindex}
\begin{reflist}
\item[]\begin{specHead}{TEI.s}{<s> }(phrase) contient une division textuelle de type phrase [\xref{http://www.tei-c.org/release/doc/tei-p5-doc/en/html/AI.html\#AILC}{17.1. Linguistic Segment Categories} \xref{http://www.tei-c.org/release/doc/tei-p5-doc/en/html/TS.html\#TSSASE}{8.4.1. Segmentation}]\end{specHead} 
    \item[{Module}]
  analysis
    \item[{Attributs}]
  Attributs \hyperref[TEI.att.global]{att.global} (\textit{@xml:id}, \textit{@n}, \textit{@xml:lang}, \textit{@xml:base}, \textit{@xml:space})  (\hyperref[TEI.att.global.rendition]{att.global.rendition} (\textit{@rend}, \textit{@style}, \textit{@rendition})) (\hyperref[TEI.att.global.linking]{att.global.linking} (\textit{@corresp}, \textit{@synch}, \textit{@sameAs}, \textit{@copyOf}, \textit{@next}, \textit{@prev}, \textit{@exclude}, \textit{@select})) (\hyperref[TEI.att.global.analytic]{att.global.analytic} (\textit{@ana})) (\hyperref[TEI.att.global.facs]{att.global.facs} (\textit{@facs})) (\hyperref[TEI.att.global.change]{att.global.change} (\textit{@change})) (\hyperref[TEI.att.global.responsibility]{att.global.responsibility} (\textit{@cert}, \textit{@resp})) (\hyperref[TEI.att.global.source]{att.global.source} (\textit{@source})) \hyperref[TEI.att.segLike]{att.segLike} (\textit{@function})  (\hyperref[TEI.att.datcat]{att.datcat} (\textit{@datcat}, \textit{@valueDatcat})) (\hyperref[TEI.att.fragmentable]{att.fragmentable} (\textit{@part})) \hyperref[TEI.att.typed]{att.typed} (\textit{@type}, \textit{@subtype}) 
    \item[{Membre du}]
  \hyperref[TEI.model.segLike]{model.segLike}
    \item[{Contenu dans}]
  
    \item[analysis: ]
   \hyperref[TEI.cl]{cl} \hyperref[TEI.phr]{phr} \hyperref[TEI.s]{s}\par 
    \item[core: ]
   \hyperref[TEI.abbr]{abbr} \hyperref[TEI.add]{add} \hyperref[TEI.addrLine]{addrLine} \hyperref[TEI.author]{author} \hyperref[TEI.bibl]{bibl} \hyperref[TEI.biblScope]{biblScope} \hyperref[TEI.citedRange]{citedRange} \hyperref[TEI.corr]{corr} \hyperref[TEI.date]{date} \hyperref[TEI.del]{del} \hyperref[TEI.distinct]{distinct} \hyperref[TEI.editor]{editor} \hyperref[TEI.email]{email} \hyperref[TEI.emph]{emph} \hyperref[TEI.expan]{expan} \hyperref[TEI.foreign]{foreign} \hyperref[TEI.gloss]{gloss} \hyperref[TEI.head]{head} \hyperref[TEI.headItem]{headItem} \hyperref[TEI.headLabel]{headLabel} \hyperref[TEI.hi]{hi} \hyperref[TEI.item]{item} \hyperref[TEI.l]{l} \hyperref[TEI.label]{label} \hyperref[TEI.measure]{measure} \hyperref[TEI.mentioned]{mentioned} \hyperref[TEI.name]{name} \hyperref[TEI.note]{note} \hyperref[TEI.num]{num} \hyperref[TEI.orig]{orig} \hyperref[TEI.p]{p} \hyperref[TEI.pubPlace]{pubPlace} \hyperref[TEI.publisher]{publisher} \hyperref[TEI.q]{q} \hyperref[TEI.quote]{quote} \hyperref[TEI.ref]{ref} \hyperref[TEI.reg]{reg} \hyperref[TEI.rs]{rs} \hyperref[TEI.said]{said} \hyperref[TEI.sic]{sic} \hyperref[TEI.soCalled]{soCalled} \hyperref[TEI.speaker]{speaker} \hyperref[TEI.stage]{stage} \hyperref[TEI.street]{street} \hyperref[TEI.term]{term} \hyperref[TEI.textLang]{textLang} \hyperref[TEI.time]{time} \hyperref[TEI.title]{title} \hyperref[TEI.unclear]{unclear}\par 
    \item[figures: ]
   \hyperref[TEI.cell]{cell}\par 
    \item[header: ]
   \hyperref[TEI.change]{change} \hyperref[TEI.distributor]{distributor} \hyperref[TEI.edition]{edition} \hyperref[TEI.extent]{extent} \hyperref[TEI.licence]{licence}\par 
    \item[linking: ]
   \hyperref[TEI.ab]{ab} \hyperref[TEI.seg]{seg}\par 
    \item[msdescription: ]
   \hyperref[TEI.accMat]{accMat} \hyperref[TEI.acquisition]{acquisition} \hyperref[TEI.additions]{additions} \hyperref[TEI.catchwords]{catchwords} \hyperref[TEI.collation]{collation} \hyperref[TEI.colophon]{colophon} \hyperref[TEI.condition]{condition} \hyperref[TEI.custEvent]{custEvent} \hyperref[TEI.decoNote]{decoNote} \hyperref[TEI.explicit]{explicit} \hyperref[TEI.filiation]{filiation} \hyperref[TEI.finalRubric]{finalRubric} \hyperref[TEI.foliation]{foliation} \hyperref[TEI.heraldry]{heraldry} \hyperref[TEI.incipit]{incipit} \hyperref[TEI.layout]{layout} \hyperref[TEI.material]{material} \hyperref[TEI.musicNotation]{musicNotation} \hyperref[TEI.objectType]{objectType} \hyperref[TEI.origDate]{origDate} \hyperref[TEI.origPlace]{origPlace} \hyperref[TEI.origin]{origin} \hyperref[TEI.provenance]{provenance} \hyperref[TEI.rubric]{rubric} \hyperref[TEI.secFol]{secFol} \hyperref[TEI.signatures]{signatures} \hyperref[TEI.source]{source} \hyperref[TEI.stamp]{stamp} \hyperref[TEI.summary]{summary} \hyperref[TEI.support]{support} \hyperref[TEI.surrogates]{surrogates} \hyperref[TEI.typeNote]{typeNote} \hyperref[TEI.watermark]{watermark}\par 
    \item[namesdates: ]
   \hyperref[TEI.addName]{addName} \hyperref[TEI.affiliation]{affiliation} \hyperref[TEI.country]{country} \hyperref[TEI.forename]{forename} \hyperref[TEI.genName]{genName} \hyperref[TEI.geogName]{geogName} \hyperref[TEI.nameLink]{nameLink} \hyperref[TEI.orgName]{orgName} \hyperref[TEI.persName]{persName} \hyperref[TEI.placeName]{placeName} \hyperref[TEI.region]{region} \hyperref[TEI.roleName]{roleName} \hyperref[TEI.settlement]{settlement} \hyperref[TEI.surname]{surname}\par 
    \item[textstructure: ]
   \hyperref[TEI.docAuthor]{docAuthor} \hyperref[TEI.docDate]{docDate} \hyperref[TEI.docEdition]{docEdition} \hyperref[TEI.titlePart]{titlePart}\par 
    \item[transcr: ]
   \hyperref[TEI.damage]{damage} \hyperref[TEI.fw]{fw} \hyperref[TEI.metamark]{metamark} \hyperref[TEI.mod]{mod} \hyperref[TEI.restore]{restore} \hyperref[TEI.retrace]{retrace} \hyperref[TEI.secl]{secl} \hyperref[TEI.supplied]{supplied} \hyperref[TEI.surplus]{surplus}
    \item[{Peut contenir}]
  
    \item[analysis: ]
   \hyperref[TEI.c]{c} \hyperref[TEI.cl]{cl} \hyperref[TEI.interp]{interp} \hyperref[TEI.interpGrp]{interpGrp} \hyperref[TEI.m]{m} \hyperref[TEI.pc]{pc} \hyperref[TEI.phr]{phr} \hyperref[TEI.s]{s} \hyperref[TEI.span]{span} \hyperref[TEI.spanGrp]{spanGrp} \hyperref[TEI.w]{w}\par 
    \item[core: ]
   \hyperref[TEI.abbr]{abbr} \hyperref[TEI.add]{add} \hyperref[TEI.address]{address} \hyperref[TEI.binaryObject]{binaryObject} \hyperref[TEI.cb]{cb} \hyperref[TEI.choice]{choice} \hyperref[TEI.corr]{corr} \hyperref[TEI.date]{date} \hyperref[TEI.del]{del} \hyperref[TEI.distinct]{distinct} \hyperref[TEI.email]{email} \hyperref[TEI.emph]{emph} \hyperref[TEI.expan]{expan} \hyperref[TEI.foreign]{foreign} \hyperref[TEI.gap]{gap} \hyperref[TEI.gb]{gb} \hyperref[TEI.gloss]{gloss} \hyperref[TEI.graphic]{graphic} \hyperref[TEI.hi]{hi} \hyperref[TEI.index]{index} \hyperref[TEI.lb]{lb} \hyperref[TEI.measure]{measure} \hyperref[TEI.measureGrp]{measureGrp} \hyperref[TEI.media]{media} \hyperref[TEI.mentioned]{mentioned} \hyperref[TEI.milestone]{milestone} \hyperref[TEI.name]{name} \hyperref[TEI.note]{note} \hyperref[TEI.num]{num} \hyperref[TEI.orig]{orig} \hyperref[TEI.pb]{pb} \hyperref[TEI.ptr]{ptr} \hyperref[TEI.ref]{ref} \hyperref[TEI.reg]{reg} \hyperref[TEI.rs]{rs} \hyperref[TEI.sic]{sic} \hyperref[TEI.soCalled]{soCalled} \hyperref[TEI.term]{term} \hyperref[TEI.time]{time} \hyperref[TEI.title]{title} \hyperref[TEI.unclear]{unclear}\par 
    \item[derived-module-tei.istex: ]
   \hyperref[TEI.math]{math} \hyperref[TEI.mrow]{mrow}\par 
    \item[figures: ]
   \hyperref[TEI.figure]{figure} \hyperref[TEI.formula]{formula} \hyperref[TEI.notatedMusic]{notatedMusic}\par 
    \item[header: ]
   \hyperref[TEI.idno]{idno}\par 
    \item[iso-fs: ]
   \hyperref[TEI.fLib]{fLib} \hyperref[TEI.fs]{fs} \hyperref[TEI.fvLib]{fvLib}\par 
    \item[linking: ]
   \hyperref[TEI.alt]{alt} \hyperref[TEI.altGrp]{altGrp} \hyperref[TEI.anchor]{anchor} \hyperref[TEI.join]{join} \hyperref[TEI.joinGrp]{joinGrp} \hyperref[TEI.link]{link} \hyperref[TEI.linkGrp]{linkGrp} \hyperref[TEI.seg]{seg} \hyperref[TEI.timeline]{timeline}\par 
    \item[msdescription: ]
   \hyperref[TEI.catchwords]{catchwords} \hyperref[TEI.depth]{depth} \hyperref[TEI.dim]{dim} \hyperref[TEI.dimensions]{dimensions} \hyperref[TEI.height]{height} \hyperref[TEI.heraldry]{heraldry} \hyperref[TEI.locus]{locus} \hyperref[TEI.locusGrp]{locusGrp} \hyperref[TEI.material]{material} \hyperref[TEI.objectType]{objectType} \hyperref[TEI.origDate]{origDate} \hyperref[TEI.origPlace]{origPlace} \hyperref[TEI.secFol]{secFol} \hyperref[TEI.signatures]{signatures} \hyperref[TEI.source]{source} \hyperref[TEI.stamp]{stamp} \hyperref[TEI.watermark]{watermark} \hyperref[TEI.width]{width}\par 
    \item[namesdates: ]
   \hyperref[TEI.addName]{addName} \hyperref[TEI.affiliation]{affiliation} \hyperref[TEI.country]{country} \hyperref[TEI.forename]{forename} \hyperref[TEI.genName]{genName} \hyperref[TEI.geogName]{geogName} \hyperref[TEI.location]{location} \hyperref[TEI.nameLink]{nameLink} \hyperref[TEI.orgName]{orgName} \hyperref[TEI.persName]{persName} \hyperref[TEI.placeName]{placeName} \hyperref[TEI.region]{region} \hyperref[TEI.roleName]{roleName} \hyperref[TEI.settlement]{settlement} \hyperref[TEI.state]{state} \hyperref[TEI.surname]{surname}\par 
    \item[spoken: ]
   \hyperref[TEI.annotationBlock]{annotationBlock}\par 
    \item[transcr: ]
   \hyperref[TEI.addSpan]{addSpan} \hyperref[TEI.am]{am} \hyperref[TEI.damage]{damage} \hyperref[TEI.damageSpan]{damageSpan} \hyperref[TEI.delSpan]{delSpan} \hyperref[TEI.ex]{ex} \hyperref[TEI.fw]{fw} \hyperref[TEI.handShift]{handShift} \hyperref[TEI.listTranspose]{listTranspose} \hyperref[TEI.metamark]{metamark} \hyperref[TEI.mod]{mod} \hyperref[TEI.redo]{redo} \hyperref[TEI.restore]{restore} \hyperref[TEI.retrace]{retrace} \hyperref[TEI.secl]{secl} \hyperref[TEI.space]{space} \hyperref[TEI.subst]{subst} \hyperref[TEI.substJoin]{substJoin} \hyperref[TEI.supplied]{supplied} \hyperref[TEI.surplus]{surplus} \hyperref[TEI.undo]{undo}\par des données textuelles
    \item[{Note}]
  \par
L'élément \hyperref[TEI.s]{<s>} peut être utilisé pour marquer les phrases ou toute autre segmentation existant dans un texte, pourvu que cette segmentation soit présente du début à la fin du texte, complète et sans imbrication. Dans le cas d'une segmentation partielle ou récursive, l'élément \hyperref[TEI.seg]{<seg>} doit remplacer l'élément \hyperref[TEI.s]{<s>}.\par
L'attribut {\itshape type} peut être utilisé pour indiquer le type de segmentation prévue, selon une typologie appropriée.
    \item[{Exemple}]
  \leavevmode\bgroup\exampleFont \begin{shaded}\noindent\mbox{}{<\textbf{s}>}Quand partez-vous ?{</\textbf{s}>}\mbox{}\newline 
{<\textbf{s}>}Demain.{</\textbf{s}>}\end{shaded}\egroup 


    \item[{Schematron}]
   <s:report test="tei:s">You may not nest one s element within  another: use seg instead</s:report>
    \item[{Modèle de contenu}]
  \mbox{}\hfill\\[-10pt]\begin{Verbatim}[fontsize=\small]
<content>
 <macroRef key="macro.phraseSeq"/>
</content>
    
\end{Verbatim}

    \item[{Schéma Declaration}]
  \mbox{}\hfill\\[-10pt]\begin{Verbatim}[fontsize=\small]
element s
{
   tei_att.global.attributes,
   tei_att.segLike.attributes,
   tei_att.typed.attributes,
   tei_macro.phraseSeq}
\end{Verbatim}

\end{reflist}  \index{said=<said>|oddindex}\index{aloud=@aloud!<said>|oddindex}\index{direct=@direct!<said>|oddindex}
\begin{reflist}
\item[]\begin{specHead}{TEI.said}{<said> }(Parole ou pensée.) indique que les passages sont pensés ou qu'ils sont prononcés à haute voix, que cela soit indiqué explicitement ou non dans la source , que ces passages soient directement ou indirectement rapportés par des personnages réels ou fictifs. [\xref{http://www.tei-c.org/release/doc/tei-p5-doc/en/html/CO.html\#COHQQ}{3.3.3. Quotation}]\end{specHead} 
    \item[{Module}]
  core
    \item[{Attributs}]
  Attributs \hyperref[TEI.att.global]{att.global} (\textit{@xml:id}, \textit{@n}, \textit{@xml:lang}, \textit{@xml:base}, \textit{@xml:space})  (\hyperref[TEI.att.global.rendition]{att.global.rendition} (\textit{@rend}, \textit{@style}, \textit{@rendition})) (\hyperref[TEI.att.global.linking]{att.global.linking} (\textit{@corresp}, \textit{@synch}, \textit{@sameAs}, \textit{@copyOf}, \textit{@next}, \textit{@prev}, \textit{@exclude}, \textit{@select})) (\hyperref[TEI.att.global.analytic]{att.global.analytic} (\textit{@ana})) (\hyperref[TEI.att.global.facs]{att.global.facs} (\textit{@facs})) (\hyperref[TEI.att.global.change]{att.global.change} (\textit{@change})) (\hyperref[TEI.att.global.responsibility]{att.global.responsibility} (\textit{@cert}, \textit{@resp})) (\hyperref[TEI.att.global.source]{att.global.source} (\textit{@source})) \hyperref[TEI.att.ascribed]{att.ascribed} (\textit{@who}) \hfil\\[-10pt]\begin{sansreflist}
    \item[@aloud]
  peut être utilisé pour indiquer si l'on estime que l'objet cité est dit oralement ou par signes.
\begin{reflist}
    \item[{Statut}]
  Optionel
    \item[{Type de données}]
  \hyperref[TEI.teidata.xTruthValue]{teidata.xTruthValue}
    \item[{Valeur par défaut}]
  unknown \par \begin{tabular}{P{0.4969230769230769\textwidth}P{0.35307692307692307\textwidth}}
\xref{http://www.tei-c.org/Activities/Council/Working/tcw27.xml}{Deprecated}\tabcellsep The value will no longer be a default after 2017-09-05\end{tabular}
    \item[]\exampleFont {<\textbf{p}>} Celia thought privately, {<\textbf{said}\hspace*{6pt}{aloud}="{false}">}Dorothea quite despises Sir James Chettam;\mbox{}\newline 
\hspace*{6pt}\hspace*{6pt} I believe she would not accept him.{</\textbf{said}>} Celia felt that this was a pity.\mbox{}\newline 
\mbox{}\newline 
\textit{<!-- ... -->}\mbox{}\newline 
{</\textbf{p}>}
    \item[{Note}]
  \par
La valeur true indique que le passage encodé a été prononcé de manière explicite (qu'il ait été dit, exprimé par geste, chanté, crié, déclamé etc.) ; la valeur false indique que le passage encodé a été pensé, mais pas prononcé.
\end{reflist}  
    \item[@direct]
  peut être utilisé pour indiquer si le sujet cité est à considérer comme comme étant du discours direct ou du discours indirect.
\begin{reflist}
    \item[{Statut}]
  Optionel
    \item[{Type de données}]
  \hyperref[TEI.teidata.xTruthValue]{teidata.xTruthValue}
    \item[{Valeur par défaut}]
  true
    \item[]\exampleFont \mbox{}\newline 
\textit{<!-- in the header -->}{<\textbf{editorialDecl}>}\mbox{}\newline 
\hspace*{6pt}{<\textbf{quotation}\hspace*{6pt}{marks}="{none}"/>}\mbox{}\newline 
{</\textbf{editorialDecl}>}\mbox{}\newline 
\textit{<!-- ... -->}\mbox{}\newline 
{<\textbf{p}>} Tantripp had brought a card, and said that there was a gentleman waiting in the lobby.\mbox{}\newline 
 The courier had told him that {<\textbf{said}\hspace*{6pt}{direct}="{false}">}only Mrs. Casaubon was at home{</\textbf{said}>},\mbox{}\newline 
 but he said {<\textbf{said}\hspace*{6pt}{direct}="{false}">}he was a relation of Mr. Casaubon's: would she see him?{</\textbf{said}>}\mbox{}\newline 
{</\textbf{p}>}
    \item[{Note}]
  \par
La valeur true indique que la parole ou la pensée est représentée directement ; la valeur false, qu'elle est représentée indirectement , par exemple marquée par une forme verbale.
\end{reflist}  
\end{sansreflist}  
    \item[{Membre du}]
  \hyperref[TEI.model.qLike]{model.qLike}
    \item[{Contenu dans}]
  
    \item[core: ]
   \hyperref[TEI.add]{add} \hyperref[TEI.cit]{cit} \hyperref[TEI.corr]{corr} \hyperref[TEI.del]{del} \hyperref[TEI.desc]{desc} \hyperref[TEI.emph]{emph} \hyperref[TEI.head]{head} \hyperref[TEI.hi]{hi} \hyperref[TEI.item]{item} \hyperref[TEI.l]{l} \hyperref[TEI.meeting]{meeting} \hyperref[TEI.note]{note} \hyperref[TEI.orig]{orig} \hyperref[TEI.p]{p} \hyperref[TEI.q]{q} \hyperref[TEI.quote]{quote} \hyperref[TEI.ref]{ref} \hyperref[TEI.reg]{reg} \hyperref[TEI.said]{said} \hyperref[TEI.sic]{sic} \hyperref[TEI.sp]{sp} \hyperref[TEI.stage]{stage} \hyperref[TEI.title]{title} \hyperref[TEI.unclear]{unclear}\par 
    \item[figures: ]
   \hyperref[TEI.cell]{cell} \hyperref[TEI.figDesc]{figDesc} \hyperref[TEI.figure]{figure}\par 
    \item[header: ]
   \hyperref[TEI.change]{change} \hyperref[TEI.licence]{licence} \hyperref[TEI.rendition]{rendition}\par 
    \item[iso-fs: ]
   \hyperref[TEI.fDescr]{fDescr} \hyperref[TEI.fsDescr]{fsDescr}\par 
    \item[linking: ]
   \hyperref[TEI.ab]{ab} \hyperref[TEI.seg]{seg}\par 
    \item[msdescription: ]
   \hyperref[TEI.accMat]{accMat} \hyperref[TEI.acquisition]{acquisition} \hyperref[TEI.additions]{additions} \hyperref[TEI.collation]{collation} \hyperref[TEI.condition]{condition} \hyperref[TEI.custEvent]{custEvent} \hyperref[TEI.decoNote]{decoNote} \hyperref[TEI.filiation]{filiation} \hyperref[TEI.foliation]{foliation} \hyperref[TEI.layout]{layout} \hyperref[TEI.musicNotation]{musicNotation} \hyperref[TEI.origin]{origin} \hyperref[TEI.provenance]{provenance} \hyperref[TEI.signatures]{signatures} \hyperref[TEI.source]{source} \hyperref[TEI.summary]{summary} \hyperref[TEI.support]{support} \hyperref[TEI.surrogates]{surrogates} \hyperref[TEI.typeNote]{typeNote}\par 
    \item[textstructure: ]
   \hyperref[TEI.body]{body} \hyperref[TEI.div]{div} \hyperref[TEI.docEdition]{docEdition} \hyperref[TEI.titlePart]{titlePart}\par 
    \item[transcr: ]
   \hyperref[TEI.damage]{damage} \hyperref[TEI.metamark]{metamark} \hyperref[TEI.mod]{mod} \hyperref[TEI.restore]{restore} \hyperref[TEI.retrace]{retrace} \hyperref[TEI.secl]{secl} \hyperref[TEI.supplied]{supplied} \hyperref[TEI.surplus]{surplus}
    \item[{Peut contenir}]
  
    \item[analysis: ]
   \hyperref[TEI.c]{c} \hyperref[TEI.cl]{cl} \hyperref[TEI.interp]{interp} \hyperref[TEI.interpGrp]{interpGrp} \hyperref[TEI.m]{m} \hyperref[TEI.pc]{pc} \hyperref[TEI.phr]{phr} \hyperref[TEI.s]{s} \hyperref[TEI.span]{span} \hyperref[TEI.spanGrp]{spanGrp} \hyperref[TEI.w]{w}\par 
    \item[core: ]
   \hyperref[TEI.abbr]{abbr} \hyperref[TEI.add]{add} \hyperref[TEI.address]{address} \hyperref[TEI.bibl]{bibl} \hyperref[TEI.biblStruct]{biblStruct} \hyperref[TEI.binaryObject]{binaryObject} \hyperref[TEI.cb]{cb} \hyperref[TEI.choice]{choice} \hyperref[TEI.cit]{cit} \hyperref[TEI.corr]{corr} \hyperref[TEI.date]{date} \hyperref[TEI.del]{del} \hyperref[TEI.desc]{desc} \hyperref[TEI.distinct]{distinct} \hyperref[TEI.email]{email} \hyperref[TEI.emph]{emph} \hyperref[TEI.expan]{expan} \hyperref[TEI.foreign]{foreign} \hyperref[TEI.gap]{gap} \hyperref[TEI.gb]{gb} \hyperref[TEI.gloss]{gloss} \hyperref[TEI.graphic]{graphic} \hyperref[TEI.hi]{hi} \hyperref[TEI.index]{index} \hyperref[TEI.l]{l} \hyperref[TEI.label]{label} \hyperref[TEI.lb]{lb} \hyperref[TEI.lg]{lg} \hyperref[TEI.list]{list} \hyperref[TEI.listBibl]{listBibl} \hyperref[TEI.measure]{measure} \hyperref[TEI.measureGrp]{measureGrp} \hyperref[TEI.media]{media} \hyperref[TEI.mentioned]{mentioned} \hyperref[TEI.milestone]{milestone} \hyperref[TEI.name]{name} \hyperref[TEI.note]{note} \hyperref[TEI.num]{num} \hyperref[TEI.orig]{orig} \hyperref[TEI.p]{p} \hyperref[TEI.pb]{pb} \hyperref[TEI.ptr]{ptr} \hyperref[TEI.q]{q} \hyperref[TEI.quote]{quote} \hyperref[TEI.ref]{ref} \hyperref[TEI.reg]{reg} \hyperref[TEI.rs]{rs} \hyperref[TEI.said]{said} \hyperref[TEI.sic]{sic} \hyperref[TEI.soCalled]{soCalled} \hyperref[TEI.sp]{sp} \hyperref[TEI.stage]{stage} \hyperref[TEI.term]{term} \hyperref[TEI.time]{time} \hyperref[TEI.title]{title} \hyperref[TEI.unclear]{unclear}\par 
    \item[derived-module-tei.istex: ]
   \hyperref[TEI.math]{math} \hyperref[TEI.mrow]{mrow}\par 
    \item[figures: ]
   \hyperref[TEI.figure]{figure} \hyperref[TEI.formula]{formula} \hyperref[TEI.notatedMusic]{notatedMusic} \hyperref[TEI.table]{table}\par 
    \item[header: ]
   \hyperref[TEI.biblFull]{biblFull} \hyperref[TEI.idno]{idno}\par 
    \item[iso-fs: ]
   \hyperref[TEI.fLib]{fLib} \hyperref[TEI.fs]{fs} \hyperref[TEI.fvLib]{fvLib}\par 
    \item[linking: ]
   \hyperref[TEI.ab]{ab} \hyperref[TEI.alt]{alt} \hyperref[TEI.altGrp]{altGrp} \hyperref[TEI.anchor]{anchor} \hyperref[TEI.join]{join} \hyperref[TEI.joinGrp]{joinGrp} \hyperref[TEI.link]{link} \hyperref[TEI.linkGrp]{linkGrp} \hyperref[TEI.seg]{seg} \hyperref[TEI.timeline]{timeline}\par 
    \item[msdescription: ]
   \hyperref[TEI.catchwords]{catchwords} \hyperref[TEI.depth]{depth} \hyperref[TEI.dim]{dim} \hyperref[TEI.dimensions]{dimensions} \hyperref[TEI.height]{height} \hyperref[TEI.heraldry]{heraldry} \hyperref[TEI.locus]{locus} \hyperref[TEI.locusGrp]{locusGrp} \hyperref[TEI.material]{material} \hyperref[TEI.msDesc]{msDesc} \hyperref[TEI.objectType]{objectType} \hyperref[TEI.origDate]{origDate} \hyperref[TEI.origPlace]{origPlace} \hyperref[TEI.secFol]{secFol} \hyperref[TEI.signatures]{signatures} \hyperref[TEI.source]{source} \hyperref[TEI.stamp]{stamp} \hyperref[TEI.watermark]{watermark} \hyperref[TEI.width]{width}\par 
    \item[namesdates: ]
   \hyperref[TEI.addName]{addName} \hyperref[TEI.affiliation]{affiliation} \hyperref[TEI.country]{country} \hyperref[TEI.forename]{forename} \hyperref[TEI.genName]{genName} \hyperref[TEI.geogName]{geogName} \hyperref[TEI.listOrg]{listOrg} \hyperref[TEI.listPlace]{listPlace} \hyperref[TEI.location]{location} \hyperref[TEI.nameLink]{nameLink} \hyperref[TEI.orgName]{orgName} \hyperref[TEI.persName]{persName} \hyperref[TEI.placeName]{placeName} \hyperref[TEI.region]{region} \hyperref[TEI.roleName]{roleName} \hyperref[TEI.settlement]{settlement} \hyperref[TEI.state]{state} \hyperref[TEI.surname]{surname}\par 
    \item[spoken: ]
   \hyperref[TEI.annotationBlock]{annotationBlock}\par 
    \item[textstructure: ]
   \hyperref[TEI.floatingText]{floatingText}\par 
    \item[transcr: ]
   \hyperref[TEI.addSpan]{addSpan} \hyperref[TEI.am]{am} \hyperref[TEI.damage]{damage} \hyperref[TEI.damageSpan]{damageSpan} \hyperref[TEI.delSpan]{delSpan} \hyperref[TEI.ex]{ex} \hyperref[TEI.fw]{fw} \hyperref[TEI.handShift]{handShift} \hyperref[TEI.listTranspose]{listTranspose} \hyperref[TEI.metamark]{metamark} \hyperref[TEI.mod]{mod} \hyperref[TEI.redo]{redo} \hyperref[TEI.restore]{restore} \hyperref[TEI.retrace]{retrace} \hyperref[TEI.secl]{secl} \hyperref[TEI.space]{space} \hyperref[TEI.subst]{subst} \hyperref[TEI.substJoin]{substJoin} \hyperref[TEI.supplied]{supplied} \hyperref[TEI.surplus]{surplus} \hyperref[TEI.undo]{undo}\par des données textuelles
    \item[{Exemple}]
  \leavevmode\bgroup\exampleFont \begin{shaded}\noindent\mbox{}{<\textbf{p}>}Ils l'entendaient murmurer : {<\textbf{said}>} Morts ! Tous morts ! Vous ne viendrez plus obéissant\mbox{}\newline 
\hspace*{6pt}\hspace*{6pt} à ma voix, quand, assise sur le bord du lac, je vous jetais dans la gueule des pépins de\mbox{}\newline 
\hspace*{6pt}\hspace*{6pt} pastèques ! Le mystère de Tanit roulait au fond de vos yeux, plus limpides que les\mbox{}\newline 
\hspace*{6pt}\hspace*{6pt} globules des fleuves.{</\textbf{said}>} Et elle les appelait par leurs noms, qui étaient les noms\mbox{}\newline 
 des mois.{<\textbf{said}>}Siv ! Sivan ! Tammouz, Eloul, Tischri, Schebar ! Ah ! pitié pour moi,\mbox{}\newline 
\hspace*{6pt}\hspace*{6pt} Déesse ! {</\textbf{said}>}\mbox{}\newline 
{</\textbf{p}>}\end{shaded}\egroup 


    \item[{Exemple}]
  \leavevmode\bgroup\exampleFont \begin{shaded}\noindent\mbox{}{<\textbf{p}>}\mbox{}\newline 
\hspace*{6pt}{<\textbf{said}\hspace*{6pt}{aloud}="{true}"\hspace*{6pt}{rend}="{pre(“) post(”)}">}On veut donc plaire à sa petite fille ?...\mbox{}\newline 
\hspace*{6pt}{</\textbf{said}>}, dit Caroline en mettant sa tête sur l'épaule d'Adolphe, qui la baise au front en\mbox{}\newline 
 pensant : {<\textbf{said}\hspace*{6pt}{aloud}="{false}">}Dieu merci, je la tiens! {</\textbf{said}>}.\mbox{}\newline 
{</\textbf{p}>}\end{shaded}\egroup 


    \item[{Modèle de contenu}]
  \mbox{}\hfill\\[-10pt]\begin{Verbatim}[fontsize=\small]
<content>
 <macroRef key="macro.specialPara"/>
</content>
    
\end{Verbatim}

    \item[{Schéma Declaration}]
  \mbox{}\hfill\\[-10pt]\begin{Verbatim}[fontsize=\small]
element said
{
   tei_att.global.attributes,
   tei_att.ascribed.attributes,
   attribute aloud { text }?,
   attribute direct { text }?,
   tei_macro.specialPara}
\end{Verbatim}

\end{reflist}  \index{schemaRef=<schemaRef>|oddindex}\index{key=@key!<schemaRef>|oddindex}
\begin{reflist}
\item[]\begin{specHead}{TEI.schemaRef}{<schemaRef> }(schema reference) describes or points to a related customization or schema file [\xref{http://www.tei-c.org/release/doc/tei-p5-doc/en/html/HD.html\#HDSCHSPEC}{2.3.9. The Schema Specification}]\end{specHead} 
    \item[{Module}]
  header
    \item[{Attributs}]
  Attributs \hyperref[TEI.att.global]{att.global} (\textit{@xml:id}, \textit{@n}, \textit{@xml:lang}, \textit{@xml:base}, \textit{@xml:space})  (\hyperref[TEI.att.global.rendition]{att.global.rendition} (\textit{@rend}, \textit{@style}, \textit{@rendition})) (\hyperref[TEI.att.global.linking]{att.global.linking} (\textit{@corresp}, \textit{@synch}, \textit{@sameAs}, \textit{@copyOf}, \textit{@next}, \textit{@prev}, \textit{@exclude}, \textit{@select})) (\hyperref[TEI.att.global.analytic]{att.global.analytic} (\textit{@ana})) (\hyperref[TEI.att.global.facs]{att.global.facs} (\textit{@facs})) (\hyperref[TEI.att.global.change]{att.global.change} (\textit{@change})) (\hyperref[TEI.att.global.responsibility]{att.global.responsibility} (\textit{@cert}, \textit{@resp})) (\hyperref[TEI.att.global.source]{att.global.source} (\textit{@source})) \hyperref[TEI.att.typed]{att.typed} (\textit{@type}, \textit{@subtype}) \hyperref[TEI.att.resourced]{att.resourced} (\textit{@url}) \hfil\\[-10pt]\begin{sansreflist}
    \item[@key]
  the identifier used for the customization or schema
\begin{reflist}
    \item[{Statut}]
  Optionel
    \item[{Type de données}]
  \hyperref[TEI.teidata.xmlName]{teidata.xmlName}
\end{reflist}  
\end{sansreflist}  
    \item[{Membre du}]
  \hyperref[TEI.model.encodingDescPart]{model.encodingDescPart}
    \item[{Contenu dans}]
  
    \item[header: ]
   \hyperref[TEI.encodingDesc]{encodingDesc}
    \item[{Peut contenir}]
  
    \item[core: ]
   \hyperref[TEI.desc]{desc}
    \item[{Exemple}]
  \leavevmode\bgroup\exampleFont \begin{shaded}\noindent\mbox{}{<\textbf{schemaRef}\hspace*{6pt}{type}="{interchangeODD}"\mbox{}\newline 
\hspace*{6pt}{url}="{http://www.tei-c.org/release/xml/tei/custom/odd/tei\textunderscore lite.odd}"/>}\mbox{}\newline 
{<\textbf{schemaRef}\hspace*{6pt}{type}="{interchangeRNG}"\mbox{}\newline 
\hspace*{6pt}{url}="{http://www.tei-c.org/release/xml/tei/custom/odd/tei\textunderscore lite.rng}"/>}\mbox{}\newline 
{<\textbf{schemaRef}\hspace*{6pt}{type}="{projectODD}"\mbox{}\newline 
\hspace*{6pt}{url}="{file:///schema/project.odd}"/>}\end{shaded}\egroup 


    \item[{Modèle de contenu}]
  \mbox{}\hfill\\[-10pt]\begin{Verbatim}[fontsize=\small]
<content>
 <classRef key="model.descLike"
  minOccurs="0"/>
</content>
    
\end{Verbatim}

    \item[{Schéma Declaration}]
  \mbox{}\hfill\\[-10pt]\begin{Verbatim}[fontsize=\small]
element schemaRef
{
   tei_att.global.attributes,
   tei_att.typed.attributes,
   tei_att.resourced.attributes,
   attribute key { text }?,
   tei_model.descLike?
}
\end{Verbatim}

\end{reflist}  \index{scriptDesc=<scriptDesc>|oddindex}
\begin{reflist}
\item[]\begin{specHead}{TEI.scriptDesc}{<scriptDesc> }contains a description of the scripts used in a manuscript or similar source. [\xref{http://www.tei-c.org/release/doc/tei-p5-doc/en/html/MS.html\#msphwr}{10.7.2.1. Writing}]\end{specHead} 
    \item[{Module}]
  msdescription
    \item[{Attributs}]
  Attributs \hyperref[TEI.att.global]{att.global} (\textit{@xml:id}, \textit{@n}, \textit{@xml:lang}, \textit{@xml:base}, \textit{@xml:space})  (\hyperref[TEI.att.global.rendition]{att.global.rendition} (\textit{@rend}, \textit{@style}, \textit{@rendition})) (\hyperref[TEI.att.global.linking]{att.global.linking} (\textit{@corresp}, \textit{@synch}, \textit{@sameAs}, \textit{@copyOf}, \textit{@next}, \textit{@prev}, \textit{@exclude}, \textit{@select})) (\hyperref[TEI.att.global.analytic]{att.global.analytic} (\textit{@ana})) (\hyperref[TEI.att.global.facs]{att.global.facs} (\textit{@facs})) (\hyperref[TEI.att.global.change]{att.global.change} (\textit{@change})) (\hyperref[TEI.att.global.responsibility]{att.global.responsibility} (\textit{@cert}, \textit{@resp})) (\hyperref[TEI.att.global.source]{att.global.source} (\textit{@source}))
    \item[{Membre du}]
  \hyperref[TEI.model.physDescPart]{model.physDescPart}
    \item[{Contenu dans}]
  
    \item[msdescription: ]
   \hyperref[TEI.physDesc]{physDesc}
    \item[{Peut contenir}]
  
    \item[core: ]
   \hyperref[TEI.p]{p}\par 
    \item[linking: ]
   \hyperref[TEI.ab]{ab}\par 
    \item[msdescription: ]
   \hyperref[TEI.summary]{summary}
    \item[{Exemple}]
  \leavevmode\bgroup\exampleFont \begin{shaded}\noindent\mbox{}{<\textbf{scriptDesc}>}\mbox{}\newline 
\hspace*{6pt}{<\textbf{p}/>}\mbox{}\newline 
{</\textbf{scriptDesc}>}\end{shaded}\egroup 


    \item[{Exemple}]
  \leavevmode\bgroup\exampleFont \begin{shaded}\noindent\mbox{}{<\textbf{scriptDesc}>}\mbox{}\newline 
\hspace*{6pt}{<\textbf{summary}>}Contains two distinct styles of scripts {</\textbf{summary}>}\mbox{}\newline 
\hspace*{6pt}{<\textbf{scriptNote}\hspace*{6pt}{xml:id}="{style-1}">}.{</\textbf{scriptNote}>}\mbox{}\newline 
\hspace*{6pt}{<\textbf{scriptNote}\hspace*{6pt}{xml:id}="{style-2}">}.{</\textbf{scriptNote}>}\mbox{}\newline 
{</\textbf{scriptDesc}>}\end{shaded}\egroup 


    \item[{Modèle de contenu}]
  \mbox{}\hfill\\[-10pt]\begin{Verbatim}[fontsize=\small]
<content>
 <alternate maxOccurs="1" minOccurs="1">
  <classRef key="model.pLike"
   maxOccurs="unbounded" minOccurs="1"/>
  <sequence maxOccurs="1" minOccurs="1">
   <elementRef key="summary" minOccurs="0"/>
   <elementRef key="scriptNote"
    maxOccurs="unbounded" minOccurs="1"/>
  </sequence>
 </alternate>
</content>
    
\end{Verbatim}

    \item[{Schéma Declaration}]
  \mbox{}\hfill\\[-10pt]\begin{Verbatim}[fontsize=\small]
element scriptDesc
{
   tei_att.global.attributes,
   ( tei_model.pLike+ | ( tei_summary?, scriptNote+ ) )
}
\end{Verbatim}

\end{reflist}  \index{seal=<seal>|oddindex}\index{contemporary=@contemporary!<seal>|oddindex}
\begin{reflist}
\item[]\begin{specHead}{TEI.seal}{<seal> }(sceau) contient la description d'un sceau ou d'un objet similaire, attaché au manuscrit. [\xref{http://www.tei-c.org/release/doc/tei-p5-doc/en/html/MS.html\#msphse}{10.7.3.2. Seals}]\end{specHead} 
    \item[{Module}]
  msdescription
    \item[{Attributs}]
  Attributs \hyperref[TEI.att.global]{att.global} (\textit{@xml:id}, \textit{@n}, \textit{@xml:lang}, \textit{@xml:base}, \textit{@xml:space})  (\hyperref[TEI.att.global.rendition]{att.global.rendition} (\textit{@rend}, \textit{@style}, \textit{@rendition})) (\hyperref[TEI.att.global.linking]{att.global.linking} (\textit{@corresp}, \textit{@synch}, \textit{@sameAs}, \textit{@copyOf}, \textit{@next}, \textit{@prev}, \textit{@exclude}, \textit{@select})) (\hyperref[TEI.att.global.analytic]{att.global.analytic} (\textit{@ana})) (\hyperref[TEI.att.global.facs]{att.global.facs} (\textit{@facs})) (\hyperref[TEI.att.global.change]{att.global.change} (\textit{@change})) (\hyperref[TEI.att.global.responsibility]{att.global.responsibility} (\textit{@cert}, \textit{@resp})) (\hyperref[TEI.att.global.source]{att.global.source} (\textit{@source})) \hyperref[TEI.att.typed]{att.typed} (\textit{@type}, \textit{@subtype}) \hyperref[TEI.att.datable]{att.datable} (\textit{@calendar}, \textit{@period})  (\hyperref[TEI.att.datable.w3c]{att.datable.w3c} (\textit{@when}, \textit{@notBefore}, \textit{@notAfter}, \textit{@from}, \textit{@to})) (\hyperref[TEI.att.datable.iso]{att.datable.iso} (\textit{@when-iso}, \textit{@notBefore-iso}, \textit{@notAfter-iso}, \textit{@from-iso}, \textit{@to-iso})) (\hyperref[TEI.att.datable.custom]{att.datable.custom} (\textit{@when-custom}, \textit{@notBefore-custom}, \textit{@notAfter-custom}, \textit{@from-custom}, \textit{@to-custom}, \textit{@datingPoint}, \textit{@datingMethod})) \hfil\\[-10pt]\begin{sansreflist}
    \item[@contemporary]
  (contemporain) spécifie si le sceau est ou non contemporain du manuscrit auquel il est attaché.
\begin{reflist}
    \item[{Statut}]
  Optionel
    \item[{Type de données}]
  \hyperref[TEI.teidata.xTruthValue]{teidata.xTruthValue}
\end{reflist}  
\end{sansreflist}  
    \item[{Contenu dans}]
  
    \item[msdescription: ]
   \hyperref[TEI.sealDesc]{sealDesc}
    \item[{Peut contenir}]
  
    \item[core: ]
   \hyperref[TEI.p]{p}\par 
    \item[linking: ]
   \hyperref[TEI.ab]{ab}\par 
    \item[msdescription: ]
   \hyperref[TEI.decoNote]{decoNote}
    \item[{Exemple}]
  \leavevmode\bgroup\exampleFont \begin{shaded}\noindent\mbox{}{<\textbf{seal}\hspace*{6pt}{n}="{2}"\hspace*{6pt}{subtype}="{cauda\textunderscore duplex}"\mbox{}\newline 
\hspace*{6pt}{type}="{pendant}">}\mbox{}\newline 
\hspace*{6pt}{<\textbf{p}>}The seal of {<\textbf{name}>}Jens Olufsen{</\textbf{name}>} in black wax. ({<\textbf{ref}>}DAS 1061{</\textbf{ref}>}). Legend: {<\textbf{q}>}S\mbox{}\newline 
\hspace*{6pt}\hspace*{6pt}\hspace*{6pt}\hspace*{6pt} IOHANNES OLAVI{</\textbf{q}>}. Parchment tag on which is written: {<\textbf{q}>}Woldorp Iohanne G{</\textbf{q}>}.{</\textbf{p}>}\mbox{}\newline 
{</\textbf{seal}>}\end{shaded}\egroup 


    \item[{Modèle de contenu}]
  \mbox{}\hfill\\[-10pt]\begin{Verbatim}[fontsize=\small]
<content>
 <alternate maxOccurs="unbounded"
  minOccurs="1">
  <classRef key="model.pLike"/>
  <elementRef key="decoNote"/>
 </alternate>
</content>
    
\end{Verbatim}

    \item[{Schéma Declaration}]
  \mbox{}\hfill\\[-10pt]\begin{Verbatim}[fontsize=\small]
element seal
{
   tei_att.global.attributes,
   tei_att.typed.attributes,
   tei_att.datable.attributes,
   attribute contemporary { text }?,
   ( tei_model.pLike | tei_decoNote )+
}
\end{Verbatim}

\end{reflist}  \index{sealDesc=<sealDesc>|oddindex}
\begin{reflist}
\item[]\begin{specHead}{TEI.sealDesc}{<sealDesc> }(description des sceaux) décrit les sceaux ou autres objets attachés au manuscrit, soit en une série de paragraphes \hyperref[TEI.p]{<p>}, soit sous la forme d'une série d'éléments \hyperref[TEI.seal]{<seal>}, complétés éventuellement par des éléments \hyperref[TEI.decoNote]{<decoNote>}. [\xref{http://www.tei-c.org/release/doc/tei-p5-doc/en/html/MS.html\#msphse}{10.7.3.2. Seals}]\end{specHead} 
    \item[{Module}]
  msdescription
    \item[{Attributs}]
  Attributs \hyperref[TEI.att.global]{att.global} (\textit{@xml:id}, \textit{@n}, \textit{@xml:lang}, \textit{@xml:base}, \textit{@xml:space})  (\hyperref[TEI.att.global.rendition]{att.global.rendition} (\textit{@rend}, \textit{@style}, \textit{@rendition})) (\hyperref[TEI.att.global.linking]{att.global.linking} (\textit{@corresp}, \textit{@synch}, \textit{@sameAs}, \textit{@copyOf}, \textit{@next}, \textit{@prev}, \textit{@exclude}, \textit{@select})) (\hyperref[TEI.att.global.analytic]{att.global.analytic} (\textit{@ana})) (\hyperref[TEI.att.global.facs]{att.global.facs} (\textit{@facs})) (\hyperref[TEI.att.global.change]{att.global.change} (\textit{@change})) (\hyperref[TEI.att.global.responsibility]{att.global.responsibility} (\textit{@cert}, \textit{@resp})) (\hyperref[TEI.att.global.source]{att.global.source} (\textit{@source}))
    \item[{Membre du}]
  \hyperref[TEI.model.physDescPart]{model.physDescPart}
    \item[{Contenu dans}]
  
    \item[msdescription: ]
   \hyperref[TEI.physDesc]{physDesc}
    \item[{Peut contenir}]
  
    \item[core: ]
   \hyperref[TEI.p]{p}\par 
    \item[linking: ]
   \hyperref[TEI.ab]{ab}\par 
    \item[msdescription: ]
   \hyperref[TEI.condition]{condition} \hyperref[TEI.decoNote]{decoNote} \hyperref[TEI.seal]{seal} \hyperref[TEI.summary]{summary}
    \item[{Exemple}]
  \leavevmode\bgroup\exampleFont \begin{shaded}\noindent\mbox{}{<\textbf{sealDesc}>}\mbox{}\newline 
\hspace*{6pt}{<\textbf{seal}\hspace*{6pt}{contemporary}="{true}"\hspace*{6pt}{type}="{pendant}">}\mbox{}\newline 
\hspace*{6pt}\hspace*{6pt}{<\textbf{p}>}Green wax vertical oval seal attached at base.{</\textbf{p}>}\mbox{}\newline 
\hspace*{6pt}{</\textbf{seal}>}\mbox{}\newline 
{</\textbf{sealDesc}>}\end{shaded}\egroup 


    \item[{Exemple}]
  \leavevmode\bgroup\exampleFont \begin{shaded}\noindent\mbox{}{<\textbf{sealDesc}>}\mbox{}\newline 
\hspace*{6pt}{<\textbf{p}>}Parchment strip for seal in place; seal missing.{</\textbf{p}>}\mbox{}\newline 
{</\textbf{sealDesc}>}\end{shaded}\egroup 


    \item[{Modèle de contenu}]
  \mbox{}\hfill\\[-10pt]\begin{Verbatim}[fontsize=\small]
<content>
 <alternate maxOccurs="1" minOccurs="1">
  <classRef key="model.pLike"
   maxOccurs="unbounded" minOccurs="1"/>
  <sequence maxOccurs="1" minOccurs="1">
   <elementRef key="summary" minOccurs="0"/>
   <alternate maxOccurs="unbounded"
    minOccurs="1">
    <elementRef key="decoNote"/>
    <elementRef key="seal"/>
    <elementRef key="condition"/>
   </alternate>
  </sequence>
 </alternate>
</content>
    
\end{Verbatim}

    \item[{Schéma Declaration}]
  \mbox{}\hfill\\[-10pt]\begin{Verbatim}[fontsize=\small]
element sealDesc
{
   tei_att.global.attributes,
   (
      tei_model.pLike+
    | ( tei_summary?, ( tei_decoNote | tei_seal | tei_condition )+ )
   )
}
\end{Verbatim}

\end{reflist}  \index{secFol=<secFol>|oddindex}
\begin{reflist}
\item[]\begin{specHead}{TEI.secFol}{<secFol> }(deuxième folio) Le mot ou les mots repris d'un point précisément connu d'un codex (comme le début du second feuillet) pour identifier celui-ci de façon univoque. [\xref{http://www.tei-c.org/release/doc/tei-p5-doc/en/html/MS.html\#msmisc}{10.3.7. Catchwords, Signatures, Secundo Folio}]\end{specHead} 
    \item[{Module}]
  msdescription
    \item[{Attributs}]
  Attributs \hyperref[TEI.att.global]{att.global} (\textit{@xml:id}, \textit{@n}, \textit{@xml:lang}, \textit{@xml:base}, \textit{@xml:space})  (\hyperref[TEI.att.global.rendition]{att.global.rendition} (\textit{@rend}, \textit{@style}, \textit{@rendition})) (\hyperref[TEI.att.global.linking]{att.global.linking} (\textit{@corresp}, \textit{@synch}, \textit{@sameAs}, \textit{@copyOf}, \textit{@next}, \textit{@prev}, \textit{@exclude}, \textit{@select})) (\hyperref[TEI.att.global.analytic]{att.global.analytic} (\textit{@ana})) (\hyperref[TEI.att.global.facs]{att.global.facs} (\textit{@facs})) (\hyperref[TEI.att.global.change]{att.global.change} (\textit{@change})) (\hyperref[TEI.att.global.responsibility]{att.global.responsibility} (\textit{@cert}, \textit{@resp})) (\hyperref[TEI.att.global.source]{att.global.source} (\textit{@source}))
    \item[{Membre du}]
  \hyperref[TEI.model.pPart.msdesc]{model.pPart.msdesc}
    \item[{Contenu dans}]
  
    \item[analysis: ]
   \hyperref[TEI.cl]{cl} \hyperref[TEI.phr]{phr} \hyperref[TEI.s]{s} \hyperref[TEI.span]{span}\par 
    \item[core: ]
   \hyperref[TEI.abbr]{abbr} \hyperref[TEI.add]{add} \hyperref[TEI.addrLine]{addrLine} \hyperref[TEI.author]{author} \hyperref[TEI.biblScope]{biblScope} \hyperref[TEI.citedRange]{citedRange} \hyperref[TEI.corr]{corr} \hyperref[TEI.date]{date} \hyperref[TEI.del]{del} \hyperref[TEI.desc]{desc} \hyperref[TEI.distinct]{distinct} \hyperref[TEI.editor]{editor} \hyperref[TEI.email]{email} \hyperref[TEI.emph]{emph} \hyperref[TEI.expan]{expan} \hyperref[TEI.foreign]{foreign} \hyperref[TEI.gloss]{gloss} \hyperref[TEI.head]{head} \hyperref[TEI.headItem]{headItem} \hyperref[TEI.headLabel]{headLabel} \hyperref[TEI.hi]{hi} \hyperref[TEI.item]{item} \hyperref[TEI.l]{l} \hyperref[TEI.label]{label} \hyperref[TEI.measure]{measure} \hyperref[TEI.meeting]{meeting} \hyperref[TEI.mentioned]{mentioned} \hyperref[TEI.name]{name} \hyperref[TEI.note]{note} \hyperref[TEI.num]{num} \hyperref[TEI.orig]{orig} \hyperref[TEI.p]{p} \hyperref[TEI.pubPlace]{pubPlace} \hyperref[TEI.publisher]{publisher} \hyperref[TEI.q]{q} \hyperref[TEI.quote]{quote} \hyperref[TEI.ref]{ref} \hyperref[TEI.reg]{reg} \hyperref[TEI.resp]{resp} \hyperref[TEI.rs]{rs} \hyperref[TEI.said]{said} \hyperref[TEI.sic]{sic} \hyperref[TEI.soCalled]{soCalled} \hyperref[TEI.speaker]{speaker} \hyperref[TEI.stage]{stage} \hyperref[TEI.street]{street} \hyperref[TEI.term]{term} \hyperref[TEI.textLang]{textLang} \hyperref[TEI.time]{time} \hyperref[TEI.title]{title} \hyperref[TEI.unclear]{unclear}\par 
    \item[figures: ]
   \hyperref[TEI.cell]{cell} \hyperref[TEI.figDesc]{figDesc}\par 
    \item[header: ]
   \hyperref[TEI.authority]{authority} \hyperref[TEI.change]{change} \hyperref[TEI.classCode]{classCode} \hyperref[TEI.creation]{creation} \hyperref[TEI.distributor]{distributor} \hyperref[TEI.edition]{edition} \hyperref[TEI.extent]{extent} \hyperref[TEI.funder]{funder} \hyperref[TEI.language]{language} \hyperref[TEI.licence]{licence} \hyperref[TEI.rendition]{rendition}\par 
    \item[iso-fs: ]
   \hyperref[TEI.fDescr]{fDescr} \hyperref[TEI.fsDescr]{fsDescr}\par 
    \item[linking: ]
   \hyperref[TEI.ab]{ab} \hyperref[TEI.seg]{seg}\par 
    \item[msdescription: ]
   \hyperref[TEI.accMat]{accMat} \hyperref[TEI.acquisition]{acquisition} \hyperref[TEI.additions]{additions} \hyperref[TEI.catchwords]{catchwords} \hyperref[TEI.collation]{collation} \hyperref[TEI.colophon]{colophon} \hyperref[TEI.condition]{condition} \hyperref[TEI.custEvent]{custEvent} \hyperref[TEI.decoNote]{decoNote} \hyperref[TEI.explicit]{explicit} \hyperref[TEI.filiation]{filiation} \hyperref[TEI.finalRubric]{finalRubric} \hyperref[TEI.foliation]{foliation} \hyperref[TEI.heraldry]{heraldry} \hyperref[TEI.incipit]{incipit} \hyperref[TEI.layout]{layout} \hyperref[TEI.material]{material} \hyperref[TEI.musicNotation]{musicNotation} \hyperref[TEI.objectType]{objectType} \hyperref[TEI.origDate]{origDate} \hyperref[TEI.origPlace]{origPlace} \hyperref[TEI.origin]{origin} \hyperref[TEI.provenance]{provenance} \hyperref[TEI.rubric]{rubric} \hyperref[TEI.secFol]{secFol} \hyperref[TEI.signatures]{signatures} \hyperref[TEI.source]{source} \hyperref[TEI.stamp]{stamp} \hyperref[TEI.summary]{summary} \hyperref[TEI.support]{support} \hyperref[TEI.surrogates]{surrogates} \hyperref[TEI.typeNote]{typeNote} \hyperref[TEI.watermark]{watermark}\par 
    \item[namesdates: ]
   \hyperref[TEI.addName]{addName} \hyperref[TEI.affiliation]{affiliation} \hyperref[TEI.country]{country} \hyperref[TEI.forename]{forename} \hyperref[TEI.genName]{genName} \hyperref[TEI.geogName]{geogName} \hyperref[TEI.nameLink]{nameLink} \hyperref[TEI.orgName]{orgName} \hyperref[TEI.persName]{persName} \hyperref[TEI.placeName]{placeName} \hyperref[TEI.region]{region} \hyperref[TEI.roleName]{roleName} \hyperref[TEI.settlement]{settlement} \hyperref[TEI.surname]{surname}\par 
    \item[textstructure: ]
   \hyperref[TEI.docAuthor]{docAuthor} \hyperref[TEI.docDate]{docDate} \hyperref[TEI.docEdition]{docEdition} \hyperref[TEI.titlePart]{titlePart}\par 
    \item[transcr: ]
   \hyperref[TEI.damage]{damage} \hyperref[TEI.fw]{fw} \hyperref[TEI.metamark]{metamark} \hyperref[TEI.mod]{mod} \hyperref[TEI.restore]{restore} \hyperref[TEI.retrace]{retrace} \hyperref[TEI.secl]{secl} \hyperref[TEI.supplied]{supplied} \hyperref[TEI.surplus]{surplus}
    \item[{Peut contenir}]
  
    \item[analysis: ]
   \hyperref[TEI.c]{c} \hyperref[TEI.cl]{cl} \hyperref[TEI.interp]{interp} \hyperref[TEI.interpGrp]{interpGrp} \hyperref[TEI.m]{m} \hyperref[TEI.pc]{pc} \hyperref[TEI.phr]{phr} \hyperref[TEI.s]{s} \hyperref[TEI.span]{span} \hyperref[TEI.spanGrp]{spanGrp} \hyperref[TEI.w]{w}\par 
    \item[core: ]
   \hyperref[TEI.abbr]{abbr} \hyperref[TEI.add]{add} \hyperref[TEI.address]{address} \hyperref[TEI.binaryObject]{binaryObject} \hyperref[TEI.cb]{cb} \hyperref[TEI.choice]{choice} \hyperref[TEI.corr]{corr} \hyperref[TEI.date]{date} \hyperref[TEI.del]{del} \hyperref[TEI.distinct]{distinct} \hyperref[TEI.email]{email} \hyperref[TEI.emph]{emph} \hyperref[TEI.expan]{expan} \hyperref[TEI.foreign]{foreign} \hyperref[TEI.gap]{gap} \hyperref[TEI.gb]{gb} \hyperref[TEI.gloss]{gloss} \hyperref[TEI.graphic]{graphic} \hyperref[TEI.hi]{hi} \hyperref[TEI.index]{index} \hyperref[TEI.lb]{lb} \hyperref[TEI.measure]{measure} \hyperref[TEI.measureGrp]{measureGrp} \hyperref[TEI.media]{media} \hyperref[TEI.mentioned]{mentioned} \hyperref[TEI.milestone]{milestone} \hyperref[TEI.name]{name} \hyperref[TEI.note]{note} \hyperref[TEI.num]{num} \hyperref[TEI.orig]{orig} \hyperref[TEI.pb]{pb} \hyperref[TEI.ptr]{ptr} \hyperref[TEI.ref]{ref} \hyperref[TEI.reg]{reg} \hyperref[TEI.rs]{rs} \hyperref[TEI.sic]{sic} \hyperref[TEI.soCalled]{soCalled} \hyperref[TEI.term]{term} \hyperref[TEI.time]{time} \hyperref[TEI.title]{title} \hyperref[TEI.unclear]{unclear}\par 
    \item[derived-module-tei.istex: ]
   \hyperref[TEI.math]{math} \hyperref[TEI.mrow]{mrow}\par 
    \item[figures: ]
   \hyperref[TEI.figure]{figure} \hyperref[TEI.formula]{formula} \hyperref[TEI.notatedMusic]{notatedMusic}\par 
    \item[header: ]
   \hyperref[TEI.idno]{idno}\par 
    \item[iso-fs: ]
   \hyperref[TEI.fLib]{fLib} \hyperref[TEI.fs]{fs} \hyperref[TEI.fvLib]{fvLib}\par 
    \item[linking: ]
   \hyperref[TEI.alt]{alt} \hyperref[TEI.altGrp]{altGrp} \hyperref[TEI.anchor]{anchor} \hyperref[TEI.join]{join} \hyperref[TEI.joinGrp]{joinGrp} \hyperref[TEI.link]{link} \hyperref[TEI.linkGrp]{linkGrp} \hyperref[TEI.seg]{seg} \hyperref[TEI.timeline]{timeline}\par 
    \item[msdescription: ]
   \hyperref[TEI.catchwords]{catchwords} \hyperref[TEI.depth]{depth} \hyperref[TEI.dim]{dim} \hyperref[TEI.dimensions]{dimensions} \hyperref[TEI.height]{height} \hyperref[TEI.heraldry]{heraldry} \hyperref[TEI.locus]{locus} \hyperref[TEI.locusGrp]{locusGrp} \hyperref[TEI.material]{material} \hyperref[TEI.objectType]{objectType} \hyperref[TEI.origDate]{origDate} \hyperref[TEI.origPlace]{origPlace} \hyperref[TEI.secFol]{secFol} \hyperref[TEI.signatures]{signatures} \hyperref[TEI.source]{source} \hyperref[TEI.stamp]{stamp} \hyperref[TEI.watermark]{watermark} \hyperref[TEI.width]{width}\par 
    \item[namesdates: ]
   \hyperref[TEI.addName]{addName} \hyperref[TEI.affiliation]{affiliation} \hyperref[TEI.country]{country} \hyperref[TEI.forename]{forename} \hyperref[TEI.genName]{genName} \hyperref[TEI.geogName]{geogName} \hyperref[TEI.location]{location} \hyperref[TEI.nameLink]{nameLink} \hyperref[TEI.orgName]{orgName} \hyperref[TEI.persName]{persName} \hyperref[TEI.placeName]{placeName} \hyperref[TEI.region]{region} \hyperref[TEI.roleName]{roleName} \hyperref[TEI.settlement]{settlement} \hyperref[TEI.state]{state} \hyperref[TEI.surname]{surname}\par 
    \item[spoken: ]
   \hyperref[TEI.annotationBlock]{annotationBlock}\par 
    \item[transcr: ]
   \hyperref[TEI.addSpan]{addSpan} \hyperref[TEI.am]{am} \hyperref[TEI.damage]{damage} \hyperref[TEI.damageSpan]{damageSpan} \hyperref[TEI.delSpan]{delSpan} \hyperref[TEI.ex]{ex} \hyperref[TEI.fw]{fw} \hyperref[TEI.handShift]{handShift} \hyperref[TEI.listTranspose]{listTranspose} \hyperref[TEI.metamark]{metamark} \hyperref[TEI.mod]{mod} \hyperref[TEI.redo]{redo} \hyperref[TEI.restore]{restore} \hyperref[TEI.retrace]{retrace} \hyperref[TEI.secl]{secl} \hyperref[TEI.space]{space} \hyperref[TEI.subst]{subst} \hyperref[TEI.substJoin]{substJoin} \hyperref[TEI.supplied]{supplied} \hyperref[TEI.surplus]{surplus} \hyperref[TEI.undo]{undo}\par des données textuelles
    \item[{Exemple}]
  \leavevmode\bgroup\exampleFont \begin{shaded}\noindent\mbox{}{<\textbf{secFol}>}(con-)versio morum{</\textbf{secFol}>}\end{shaded}\egroup 


    \item[{Exemple}]
  \leavevmode\bgroup\exampleFont \begin{shaded}\noindent\mbox{}{<\textbf{secFol}>}(con-)versio morum{</\textbf{secFol}>}\end{shaded}\egroup 


    \item[{Schematron}]
   <sch:assert role="nonfatal"  test="ancestor::tei:msDesc">WARNING: deprecated use of element — The <sch:name/> element will not be allowed outside of msDesc as of 2018-10-01.</sch:assert>
    \item[{Modèle de contenu}]
  \mbox{}\hfill\\[-10pt]\begin{Verbatim}[fontsize=\small]
<content>
 <macroRef key="macro.phraseSeq"/>
</content>
    
\end{Verbatim}

    \item[{Schéma Declaration}]
  \mbox{}\hfill\\[-10pt]\begin{Verbatim}[fontsize=\small]
element secFol { tei_att.global.attributes, tei_macro.phraseSeq }
\end{Verbatim}

\end{reflist}  \index{secl=<secl>|oddindex}\index{reason=@reason!<secl>|oddindex}
\begin{reflist}
\item[]\begin{specHead}{TEI.secl}{<secl> }(secluded text) Secluded. Marks text present in the source which the editor believes to be genuine but out of its original place (which is unknown). [\xref{http://www.tei-c.org/release/doc/tei-p5-doc/en/html/PH.html\#PHOM}{11.3.1.7. Text Omitted from or Supplied in the Transcription}]\end{specHead} 
    \item[{Module}]
  transcr
    \item[{Attributs}]
  Attributs \hyperref[TEI.att.global]{att.global} (\textit{@xml:id}, \textit{@n}, \textit{@xml:lang}, \textit{@xml:base}, \textit{@xml:space})  (\hyperref[TEI.att.global.rendition]{att.global.rendition} (\textit{@rend}, \textit{@style}, \textit{@rendition})) (\hyperref[TEI.att.global.linking]{att.global.linking} (\textit{@corresp}, \textit{@synch}, \textit{@sameAs}, \textit{@copyOf}, \textit{@next}, \textit{@prev}, \textit{@exclude}, \textit{@select})) (\hyperref[TEI.att.global.analytic]{att.global.analytic} (\textit{@ana})) (\hyperref[TEI.att.global.facs]{att.global.facs} (\textit{@facs})) (\hyperref[TEI.att.global.change]{att.global.change} (\textit{@change})) (\hyperref[TEI.att.global.responsibility]{att.global.responsibility} (\textit{@cert}, \textit{@resp})) (\hyperref[TEI.att.global.source]{att.global.source} (\textit{@source})) \hyperref[TEI.att.editLike]{att.editLike} (\textit{@evidence}, \textit{@instant})  (\hyperref[TEI.att.dimensions]{att.dimensions} (\textit{@unit}, \textit{@quantity}, \textit{@extent}, \textit{@precision}, \textit{@scope}) (\hyperref[TEI.att.ranging]{att.ranging} (\textit{@atLeast}, \textit{@atMost}, \textit{@min}, \textit{@max}, \textit{@confidence})) ) \hfil\\[-10pt]\begin{sansreflist}
    \item[@reason]
  one or more words indicating why this text has been secluded, e.g. \textit{interpolated} etc.
\begin{reflist}
    \item[{Statut}]
  Optionel
    \item[{Type de données}]
  1–∞ occurrences de \hyperref[TEI.teidata.word]{teidata.word} séparé par un espace
\end{reflist}  
\end{sansreflist}  
    \item[{Membre du}]
  \hyperref[TEI.model.pPart.transcriptional]{model.pPart.transcriptional}
    \item[{Contenu dans}]
  
    \item[analysis: ]
   \hyperref[TEI.cl]{cl} \hyperref[TEI.pc]{pc} \hyperref[TEI.phr]{phr} \hyperref[TEI.s]{s} \hyperref[TEI.w]{w}\par 
    \item[core: ]
   \hyperref[TEI.abbr]{abbr} \hyperref[TEI.add]{add} \hyperref[TEI.addrLine]{addrLine} \hyperref[TEI.author]{author} \hyperref[TEI.bibl]{bibl} \hyperref[TEI.biblScope]{biblScope} \hyperref[TEI.citedRange]{citedRange} \hyperref[TEI.corr]{corr} \hyperref[TEI.date]{date} \hyperref[TEI.del]{del} \hyperref[TEI.distinct]{distinct} \hyperref[TEI.editor]{editor} \hyperref[TEI.email]{email} \hyperref[TEI.emph]{emph} \hyperref[TEI.expan]{expan} \hyperref[TEI.foreign]{foreign} \hyperref[TEI.gloss]{gloss} \hyperref[TEI.head]{head} \hyperref[TEI.headItem]{headItem} \hyperref[TEI.headLabel]{headLabel} \hyperref[TEI.hi]{hi} \hyperref[TEI.item]{item} \hyperref[TEI.l]{l} \hyperref[TEI.label]{label} \hyperref[TEI.measure]{measure} \hyperref[TEI.mentioned]{mentioned} \hyperref[TEI.name]{name} \hyperref[TEI.note]{note} \hyperref[TEI.num]{num} \hyperref[TEI.orig]{orig} \hyperref[TEI.p]{p} \hyperref[TEI.pubPlace]{pubPlace} \hyperref[TEI.publisher]{publisher} \hyperref[TEI.q]{q} \hyperref[TEI.quote]{quote} \hyperref[TEI.ref]{ref} \hyperref[TEI.reg]{reg} \hyperref[TEI.rs]{rs} \hyperref[TEI.said]{said} \hyperref[TEI.sic]{sic} \hyperref[TEI.soCalled]{soCalled} \hyperref[TEI.speaker]{speaker} \hyperref[TEI.stage]{stage} \hyperref[TEI.street]{street} \hyperref[TEI.term]{term} \hyperref[TEI.textLang]{textLang} \hyperref[TEI.time]{time} \hyperref[TEI.title]{title} \hyperref[TEI.unclear]{unclear}\par 
    \item[figures: ]
   \hyperref[TEI.cell]{cell}\par 
    \item[header: ]
   \hyperref[TEI.change]{change} \hyperref[TEI.distributor]{distributor} \hyperref[TEI.edition]{edition} \hyperref[TEI.extent]{extent} \hyperref[TEI.licence]{licence}\par 
    \item[linking: ]
   \hyperref[TEI.ab]{ab} \hyperref[TEI.seg]{seg}\par 
    \item[msdescription: ]
   \hyperref[TEI.accMat]{accMat} \hyperref[TEI.acquisition]{acquisition} \hyperref[TEI.additions]{additions} \hyperref[TEI.catchwords]{catchwords} \hyperref[TEI.collation]{collation} \hyperref[TEI.colophon]{colophon} \hyperref[TEI.condition]{condition} \hyperref[TEI.custEvent]{custEvent} \hyperref[TEI.decoNote]{decoNote} \hyperref[TEI.explicit]{explicit} \hyperref[TEI.filiation]{filiation} \hyperref[TEI.finalRubric]{finalRubric} \hyperref[TEI.foliation]{foliation} \hyperref[TEI.heraldry]{heraldry} \hyperref[TEI.incipit]{incipit} \hyperref[TEI.layout]{layout} \hyperref[TEI.material]{material} \hyperref[TEI.musicNotation]{musicNotation} \hyperref[TEI.objectType]{objectType} \hyperref[TEI.origDate]{origDate} \hyperref[TEI.origPlace]{origPlace} \hyperref[TEI.origin]{origin} \hyperref[TEI.provenance]{provenance} \hyperref[TEI.rubric]{rubric} \hyperref[TEI.secFol]{secFol} \hyperref[TEI.signatures]{signatures} \hyperref[TEI.source]{source} \hyperref[TEI.stamp]{stamp} \hyperref[TEI.summary]{summary} \hyperref[TEI.support]{support} \hyperref[TEI.surrogates]{surrogates} \hyperref[TEI.typeNote]{typeNote} \hyperref[TEI.watermark]{watermark}\par 
    \item[namesdates: ]
   \hyperref[TEI.addName]{addName} \hyperref[TEI.affiliation]{affiliation} \hyperref[TEI.country]{country} \hyperref[TEI.forename]{forename} \hyperref[TEI.genName]{genName} \hyperref[TEI.geogName]{geogName} \hyperref[TEI.nameLink]{nameLink} \hyperref[TEI.orgName]{orgName} \hyperref[TEI.persName]{persName} \hyperref[TEI.placeName]{placeName} \hyperref[TEI.region]{region} \hyperref[TEI.roleName]{roleName} \hyperref[TEI.settlement]{settlement} \hyperref[TEI.surname]{surname}\par 
    \item[textstructure: ]
   \hyperref[TEI.docAuthor]{docAuthor} \hyperref[TEI.docDate]{docDate} \hyperref[TEI.docEdition]{docEdition} \hyperref[TEI.titlePart]{titlePart}\par 
    \item[transcr: ]
   \hyperref[TEI.am]{am} \hyperref[TEI.damage]{damage} \hyperref[TEI.fw]{fw} \hyperref[TEI.metamark]{metamark} \hyperref[TEI.mod]{mod} \hyperref[TEI.restore]{restore} \hyperref[TEI.retrace]{retrace} \hyperref[TEI.secl]{secl} \hyperref[TEI.supplied]{supplied} \hyperref[TEI.surplus]{surplus}
    \item[{Peut contenir}]
  
    \item[analysis: ]
   \hyperref[TEI.c]{c} \hyperref[TEI.cl]{cl} \hyperref[TEI.interp]{interp} \hyperref[TEI.interpGrp]{interpGrp} \hyperref[TEI.m]{m} \hyperref[TEI.pc]{pc} \hyperref[TEI.phr]{phr} \hyperref[TEI.s]{s} \hyperref[TEI.span]{span} \hyperref[TEI.spanGrp]{spanGrp} \hyperref[TEI.w]{w}\par 
    \item[core: ]
   \hyperref[TEI.abbr]{abbr} \hyperref[TEI.add]{add} \hyperref[TEI.address]{address} \hyperref[TEI.bibl]{bibl} \hyperref[TEI.biblStruct]{biblStruct} \hyperref[TEI.binaryObject]{binaryObject} \hyperref[TEI.cb]{cb} \hyperref[TEI.choice]{choice} \hyperref[TEI.cit]{cit} \hyperref[TEI.corr]{corr} \hyperref[TEI.date]{date} \hyperref[TEI.del]{del} \hyperref[TEI.desc]{desc} \hyperref[TEI.distinct]{distinct} \hyperref[TEI.email]{email} \hyperref[TEI.emph]{emph} \hyperref[TEI.expan]{expan} \hyperref[TEI.foreign]{foreign} \hyperref[TEI.gap]{gap} \hyperref[TEI.gb]{gb} \hyperref[TEI.gloss]{gloss} \hyperref[TEI.graphic]{graphic} \hyperref[TEI.hi]{hi} \hyperref[TEI.index]{index} \hyperref[TEI.l]{l} \hyperref[TEI.label]{label} \hyperref[TEI.lb]{lb} \hyperref[TEI.lg]{lg} \hyperref[TEI.list]{list} \hyperref[TEI.listBibl]{listBibl} \hyperref[TEI.measure]{measure} \hyperref[TEI.measureGrp]{measureGrp} \hyperref[TEI.media]{media} \hyperref[TEI.mentioned]{mentioned} \hyperref[TEI.milestone]{milestone} \hyperref[TEI.name]{name} \hyperref[TEI.note]{note} \hyperref[TEI.num]{num} \hyperref[TEI.orig]{orig} \hyperref[TEI.pb]{pb} \hyperref[TEI.ptr]{ptr} \hyperref[TEI.q]{q} \hyperref[TEI.quote]{quote} \hyperref[TEI.ref]{ref} \hyperref[TEI.reg]{reg} \hyperref[TEI.rs]{rs} \hyperref[TEI.said]{said} \hyperref[TEI.sic]{sic} \hyperref[TEI.soCalled]{soCalled} \hyperref[TEI.stage]{stage} \hyperref[TEI.term]{term} \hyperref[TEI.time]{time} \hyperref[TEI.title]{title} \hyperref[TEI.unclear]{unclear}\par 
    \item[derived-module-tei.istex: ]
   \hyperref[TEI.math]{math} \hyperref[TEI.mrow]{mrow}\par 
    \item[figures: ]
   \hyperref[TEI.figure]{figure} \hyperref[TEI.formula]{formula} \hyperref[TEI.notatedMusic]{notatedMusic} \hyperref[TEI.table]{table}\par 
    \item[header: ]
   \hyperref[TEI.biblFull]{biblFull} \hyperref[TEI.idno]{idno}\par 
    \item[iso-fs: ]
   \hyperref[TEI.fLib]{fLib} \hyperref[TEI.fs]{fs} \hyperref[TEI.fvLib]{fvLib}\par 
    \item[linking: ]
   \hyperref[TEI.alt]{alt} \hyperref[TEI.altGrp]{altGrp} \hyperref[TEI.anchor]{anchor} \hyperref[TEI.join]{join} \hyperref[TEI.joinGrp]{joinGrp} \hyperref[TEI.link]{link} \hyperref[TEI.linkGrp]{linkGrp} \hyperref[TEI.seg]{seg} \hyperref[TEI.timeline]{timeline}\par 
    \item[msdescription: ]
   \hyperref[TEI.catchwords]{catchwords} \hyperref[TEI.depth]{depth} \hyperref[TEI.dim]{dim} \hyperref[TEI.dimensions]{dimensions} \hyperref[TEI.height]{height} \hyperref[TEI.heraldry]{heraldry} \hyperref[TEI.locus]{locus} \hyperref[TEI.locusGrp]{locusGrp} \hyperref[TEI.material]{material} \hyperref[TEI.msDesc]{msDesc} \hyperref[TEI.objectType]{objectType} \hyperref[TEI.origDate]{origDate} \hyperref[TEI.origPlace]{origPlace} \hyperref[TEI.secFol]{secFol} \hyperref[TEI.signatures]{signatures} \hyperref[TEI.source]{source} \hyperref[TEI.stamp]{stamp} \hyperref[TEI.watermark]{watermark} \hyperref[TEI.width]{width}\par 
    \item[namesdates: ]
   \hyperref[TEI.addName]{addName} \hyperref[TEI.affiliation]{affiliation} \hyperref[TEI.country]{country} \hyperref[TEI.forename]{forename} \hyperref[TEI.genName]{genName} \hyperref[TEI.geogName]{geogName} \hyperref[TEI.listOrg]{listOrg} \hyperref[TEI.listPlace]{listPlace} \hyperref[TEI.location]{location} \hyperref[TEI.nameLink]{nameLink} \hyperref[TEI.orgName]{orgName} \hyperref[TEI.persName]{persName} \hyperref[TEI.placeName]{placeName} \hyperref[TEI.region]{region} \hyperref[TEI.roleName]{roleName} \hyperref[TEI.settlement]{settlement} \hyperref[TEI.state]{state} \hyperref[TEI.surname]{surname}\par 
    \item[spoken: ]
   \hyperref[TEI.annotationBlock]{annotationBlock}\par 
    \item[textstructure: ]
   \hyperref[TEI.floatingText]{floatingText}\par 
    \item[transcr: ]
   \hyperref[TEI.addSpan]{addSpan} \hyperref[TEI.am]{am} \hyperref[TEI.damage]{damage} \hyperref[TEI.damageSpan]{damageSpan} \hyperref[TEI.delSpan]{delSpan} \hyperref[TEI.ex]{ex} \hyperref[TEI.fw]{fw} \hyperref[TEI.handShift]{handShift} \hyperref[TEI.listTranspose]{listTranspose} \hyperref[TEI.metamark]{metamark} \hyperref[TEI.mod]{mod} \hyperref[TEI.redo]{redo} \hyperref[TEI.restore]{restore} \hyperref[TEI.retrace]{retrace} \hyperref[TEI.secl]{secl} \hyperref[TEI.space]{space} \hyperref[TEI.subst]{subst} \hyperref[TEI.substJoin]{substJoin} \hyperref[TEI.supplied]{supplied} \hyperref[TEI.surplus]{surplus} \hyperref[TEI.undo]{undo}\par des données textuelles
    \item[{Exemple}]
  \leavevmode\bgroup\exampleFont \begin{shaded}\noindent\mbox{}{<\textbf{rdg}\hspace*{6pt}{source}="{\#Pescani}">}\mbox{}\newline 
\hspace*{6pt}{<\textbf{secl}>}\mbox{}\newline 
\hspace*{6pt}\hspace*{6pt}{<\textbf{l}\hspace*{6pt}{n}="{15}"\hspace*{6pt}{xml:id}="{l15}">}Alphesiboea suos ulta est pro coniuge fratres,{</\textbf{l}>}\mbox{}\newline 
\hspace*{6pt}\hspace*{6pt}{<\textbf{l}\hspace*{6pt}{n}="{16}"\hspace*{6pt}{xml:id}="{l16}">}sanguinis et cari vincula rupit amor.{</\textbf{l}>}\mbox{}\newline 
\hspace*{6pt}{</\textbf{secl}>}\mbox{}\newline 
{</\textbf{rdg}>}\mbox{}\newline 
{<\textbf{wit}>}secl. Pescani{</\textbf{wit}>}\end{shaded}\egroup 


    \item[{Modèle de contenu}]
  \mbox{}\hfill\\[-10pt]\begin{Verbatim}[fontsize=\small]
<content>
 <macroRef key="macro.paraContent"/>
</content>
    
\end{Verbatim}

    \item[{Schéma Declaration}]
  \mbox{}\hfill\\[-10pt]\begin{Verbatim}[fontsize=\small]
element secl
{
   tei_att.global.attributes,
   tei_att.editLike.attributes,
   attribute reason { list { + } }?,
   tei_macro.paraContent}
\end{Verbatim}

\end{reflist}  \index{seg=<seg>|oddindex}
\begin{reflist}
\item[]\begin{specHead}{TEI.seg}{<seg> }(segment quelconque) contient une unité de texte quelconque de niveau ‘segment’. [\xref{http://www.tei-c.org/release/doc/tei-p5-doc/en/html/SA.html\#SASE}{16.3. Blocks, Segments, and Anchors} \xref{http://www.tei-c.org/release/doc/tei-p5-doc/en/html/VE.html\#VESE}{6.2. Components of the Verse Line} \xref{http://www.tei-c.org/release/doc/tei-p5-doc/en/html/DR.html\#DRPAL}{7.2.5. Speech Contents}]\end{specHead} 
    \item[{Module}]
  linking
    \item[{Attributs}]
  Attributs \hyperref[TEI.att.global]{att.global} (\textit{@xml:id}, \textit{@n}, \textit{@xml:lang}, \textit{@xml:base}, \textit{@xml:space})  (\hyperref[TEI.att.global.rendition]{att.global.rendition} (\textit{@rend}, \textit{@style}, \textit{@rendition})) (\hyperref[TEI.att.global.linking]{att.global.linking} (\textit{@corresp}, \textit{@synch}, \textit{@sameAs}, \textit{@copyOf}, \textit{@next}, \textit{@prev}, \textit{@exclude}, \textit{@select})) (\hyperref[TEI.att.global.analytic]{att.global.analytic} (\textit{@ana})) (\hyperref[TEI.att.global.facs]{att.global.facs} (\textit{@facs})) (\hyperref[TEI.att.global.change]{att.global.change} (\textit{@change})) (\hyperref[TEI.att.global.responsibility]{att.global.responsibility} (\textit{@cert}, \textit{@resp})) (\hyperref[TEI.att.global.source]{att.global.source} (\textit{@source})) \hyperref[TEI.att.segLike]{att.segLike} (\textit{@function})  (\hyperref[TEI.att.datcat]{att.datcat} (\textit{@datcat}, \textit{@valueDatcat})) (\hyperref[TEI.att.fragmentable]{att.fragmentable} (\textit{@part})) \hyperref[TEI.att.typed]{att.typed} (\textit{@type}, \textit{@subtype}) \hyperref[TEI.att.written]{att.written} (\textit{@hand}) 
    \item[{Membre du}]
  \hyperref[TEI.model.annotation]{model.annotation} \hyperref[TEI.model.choicePart]{model.choicePart} \hyperref[TEI.model.linePart]{model.linePart} \hyperref[TEI.model.segLike]{model.segLike} 
    \item[{Contenu dans}]
  
    \item[analysis: ]
   \hyperref[TEI.cl]{cl} \hyperref[TEI.m]{m} \hyperref[TEI.phr]{phr} \hyperref[TEI.s]{s} \hyperref[TEI.w]{w}\par 
    \item[core: ]
   \hyperref[TEI.abbr]{abbr} \hyperref[TEI.add]{add} \hyperref[TEI.addrLine]{addrLine} \hyperref[TEI.author]{author} \hyperref[TEI.bibl]{bibl} \hyperref[TEI.biblScope]{biblScope} \hyperref[TEI.choice]{choice} \hyperref[TEI.citedRange]{citedRange} \hyperref[TEI.corr]{corr} \hyperref[TEI.date]{date} \hyperref[TEI.del]{del} \hyperref[TEI.distinct]{distinct} \hyperref[TEI.editor]{editor} \hyperref[TEI.email]{email} \hyperref[TEI.emph]{emph} \hyperref[TEI.expan]{expan} \hyperref[TEI.foreign]{foreign} \hyperref[TEI.gloss]{gloss} \hyperref[TEI.head]{head} \hyperref[TEI.headItem]{headItem} \hyperref[TEI.headLabel]{headLabel} \hyperref[TEI.hi]{hi} \hyperref[TEI.item]{item} \hyperref[TEI.l]{l} \hyperref[TEI.label]{label} \hyperref[TEI.measure]{measure} \hyperref[TEI.mentioned]{mentioned} \hyperref[TEI.name]{name} \hyperref[TEI.note]{note} \hyperref[TEI.num]{num} \hyperref[TEI.orig]{orig} \hyperref[TEI.p]{p} \hyperref[TEI.pubPlace]{pubPlace} \hyperref[TEI.publisher]{publisher} \hyperref[TEI.q]{q} \hyperref[TEI.quote]{quote} \hyperref[TEI.ref]{ref} \hyperref[TEI.reg]{reg} \hyperref[TEI.rs]{rs} \hyperref[TEI.said]{said} \hyperref[TEI.sic]{sic} \hyperref[TEI.soCalled]{soCalled} \hyperref[TEI.speaker]{speaker} \hyperref[TEI.stage]{stage} \hyperref[TEI.street]{street} \hyperref[TEI.term]{term} \hyperref[TEI.textLang]{textLang} \hyperref[TEI.time]{time} \hyperref[TEI.title]{title} \hyperref[TEI.unclear]{unclear}\par 
    \item[figures: ]
   \hyperref[TEI.cell]{cell} \hyperref[TEI.notatedMusic]{notatedMusic}\par 
    \item[header: ]
   \hyperref[TEI.change]{change} \hyperref[TEI.distributor]{distributor} \hyperref[TEI.edition]{edition} \hyperref[TEI.extent]{extent} \hyperref[TEI.licence]{licence}\par 
    \item[linking: ]
   \hyperref[TEI.ab]{ab} \hyperref[TEI.seg]{seg}\par 
    \item[msdescription: ]
   \hyperref[TEI.accMat]{accMat} \hyperref[TEI.acquisition]{acquisition} \hyperref[TEI.additions]{additions} \hyperref[TEI.catchwords]{catchwords} \hyperref[TEI.collation]{collation} \hyperref[TEI.colophon]{colophon} \hyperref[TEI.condition]{condition} \hyperref[TEI.custEvent]{custEvent} \hyperref[TEI.decoNote]{decoNote} \hyperref[TEI.explicit]{explicit} \hyperref[TEI.filiation]{filiation} \hyperref[TEI.finalRubric]{finalRubric} \hyperref[TEI.foliation]{foliation} \hyperref[TEI.heraldry]{heraldry} \hyperref[TEI.incipit]{incipit} \hyperref[TEI.layout]{layout} \hyperref[TEI.material]{material} \hyperref[TEI.musicNotation]{musicNotation} \hyperref[TEI.objectType]{objectType} \hyperref[TEI.origDate]{origDate} \hyperref[TEI.origPlace]{origPlace} \hyperref[TEI.origin]{origin} \hyperref[TEI.provenance]{provenance} \hyperref[TEI.rubric]{rubric} \hyperref[TEI.secFol]{secFol} \hyperref[TEI.signatures]{signatures} \hyperref[TEI.source]{source} \hyperref[TEI.stamp]{stamp} \hyperref[TEI.summary]{summary} \hyperref[TEI.support]{support} \hyperref[TEI.surrogates]{surrogates} \hyperref[TEI.typeNote]{typeNote} \hyperref[TEI.watermark]{watermark}\par 
    \item[namesdates: ]
   \hyperref[TEI.addName]{addName} \hyperref[TEI.affiliation]{affiliation} \hyperref[TEI.country]{country} \hyperref[TEI.forename]{forename} \hyperref[TEI.genName]{genName} \hyperref[TEI.geogName]{geogName} \hyperref[TEI.nameLink]{nameLink} \hyperref[TEI.orgName]{orgName} \hyperref[TEI.persName]{persName} \hyperref[TEI.placeName]{placeName} \hyperref[TEI.region]{region} \hyperref[TEI.roleName]{roleName} \hyperref[TEI.settlement]{settlement} \hyperref[TEI.surname]{surname}\par 
    \item[spoken: ]
   \hyperref[TEI.annotationBlock]{annotationBlock}\par 
    \item[standOff: ]
   \hyperref[TEI.listAnnotation]{listAnnotation}\par 
    \item[textstructure: ]
   \hyperref[TEI.docAuthor]{docAuthor} \hyperref[TEI.docDate]{docDate} \hyperref[TEI.docEdition]{docEdition} \hyperref[TEI.titlePart]{titlePart}\par 
    \item[transcr: ]
   \hyperref[TEI.damage]{damage} \hyperref[TEI.fw]{fw} \hyperref[TEI.line]{line} \hyperref[TEI.metamark]{metamark} \hyperref[TEI.mod]{mod} \hyperref[TEI.restore]{restore} \hyperref[TEI.retrace]{retrace} \hyperref[TEI.secl]{secl} \hyperref[TEI.supplied]{supplied} \hyperref[TEI.surplus]{surplus} \hyperref[TEI.zone]{zone}
    \item[{Peut contenir}]
  
    \item[analysis: ]
   \hyperref[TEI.c]{c} \hyperref[TEI.cl]{cl} \hyperref[TEI.interp]{interp} \hyperref[TEI.interpGrp]{interpGrp} \hyperref[TEI.m]{m} \hyperref[TEI.pc]{pc} \hyperref[TEI.phr]{phr} \hyperref[TEI.s]{s} \hyperref[TEI.span]{span} \hyperref[TEI.spanGrp]{spanGrp} \hyperref[TEI.w]{w}\par 
    \item[core: ]
   \hyperref[TEI.abbr]{abbr} \hyperref[TEI.add]{add} \hyperref[TEI.address]{address} \hyperref[TEI.bibl]{bibl} \hyperref[TEI.biblStruct]{biblStruct} \hyperref[TEI.binaryObject]{binaryObject} \hyperref[TEI.cb]{cb} \hyperref[TEI.choice]{choice} \hyperref[TEI.cit]{cit} \hyperref[TEI.corr]{corr} \hyperref[TEI.date]{date} \hyperref[TEI.del]{del} \hyperref[TEI.desc]{desc} \hyperref[TEI.distinct]{distinct} \hyperref[TEI.email]{email} \hyperref[TEI.emph]{emph} \hyperref[TEI.expan]{expan} \hyperref[TEI.foreign]{foreign} \hyperref[TEI.gap]{gap} \hyperref[TEI.gb]{gb} \hyperref[TEI.gloss]{gloss} \hyperref[TEI.graphic]{graphic} \hyperref[TEI.hi]{hi} \hyperref[TEI.index]{index} \hyperref[TEI.l]{l} \hyperref[TEI.label]{label} \hyperref[TEI.lb]{lb} \hyperref[TEI.lg]{lg} \hyperref[TEI.list]{list} \hyperref[TEI.listBibl]{listBibl} \hyperref[TEI.measure]{measure} \hyperref[TEI.measureGrp]{measureGrp} \hyperref[TEI.media]{media} \hyperref[TEI.mentioned]{mentioned} \hyperref[TEI.milestone]{milestone} \hyperref[TEI.name]{name} \hyperref[TEI.note]{note} \hyperref[TEI.num]{num} \hyperref[TEI.orig]{orig} \hyperref[TEI.pb]{pb} \hyperref[TEI.ptr]{ptr} \hyperref[TEI.q]{q} \hyperref[TEI.quote]{quote} \hyperref[TEI.ref]{ref} \hyperref[TEI.reg]{reg} \hyperref[TEI.rs]{rs} \hyperref[TEI.said]{said} \hyperref[TEI.sic]{sic} \hyperref[TEI.soCalled]{soCalled} \hyperref[TEI.stage]{stage} \hyperref[TEI.term]{term} \hyperref[TEI.time]{time} \hyperref[TEI.title]{title} \hyperref[TEI.unclear]{unclear}\par 
    \item[derived-module-tei.istex: ]
   \hyperref[TEI.math]{math} \hyperref[TEI.mrow]{mrow}\par 
    \item[figures: ]
   \hyperref[TEI.figure]{figure} \hyperref[TEI.formula]{formula} \hyperref[TEI.notatedMusic]{notatedMusic} \hyperref[TEI.table]{table}\par 
    \item[header: ]
   \hyperref[TEI.biblFull]{biblFull} \hyperref[TEI.idno]{idno}\par 
    \item[iso-fs: ]
   \hyperref[TEI.fLib]{fLib} \hyperref[TEI.fs]{fs} \hyperref[TEI.fvLib]{fvLib}\par 
    \item[linking: ]
   \hyperref[TEI.alt]{alt} \hyperref[TEI.altGrp]{altGrp} \hyperref[TEI.anchor]{anchor} \hyperref[TEI.join]{join} \hyperref[TEI.joinGrp]{joinGrp} \hyperref[TEI.link]{link} \hyperref[TEI.linkGrp]{linkGrp} \hyperref[TEI.seg]{seg} \hyperref[TEI.timeline]{timeline}\par 
    \item[msdescription: ]
   \hyperref[TEI.catchwords]{catchwords} \hyperref[TEI.depth]{depth} \hyperref[TEI.dim]{dim} \hyperref[TEI.dimensions]{dimensions} \hyperref[TEI.height]{height} \hyperref[TEI.heraldry]{heraldry} \hyperref[TEI.locus]{locus} \hyperref[TEI.locusGrp]{locusGrp} \hyperref[TEI.material]{material} \hyperref[TEI.msDesc]{msDesc} \hyperref[TEI.objectType]{objectType} \hyperref[TEI.origDate]{origDate} \hyperref[TEI.origPlace]{origPlace} \hyperref[TEI.secFol]{secFol} \hyperref[TEI.signatures]{signatures} \hyperref[TEI.source]{source} \hyperref[TEI.stamp]{stamp} \hyperref[TEI.watermark]{watermark} \hyperref[TEI.width]{width}\par 
    \item[namesdates: ]
   \hyperref[TEI.addName]{addName} \hyperref[TEI.affiliation]{affiliation} \hyperref[TEI.country]{country} \hyperref[TEI.forename]{forename} \hyperref[TEI.genName]{genName} \hyperref[TEI.geogName]{geogName} \hyperref[TEI.listOrg]{listOrg} \hyperref[TEI.listPlace]{listPlace} \hyperref[TEI.location]{location} \hyperref[TEI.nameLink]{nameLink} \hyperref[TEI.orgName]{orgName} \hyperref[TEI.persName]{persName} \hyperref[TEI.placeName]{placeName} \hyperref[TEI.region]{region} \hyperref[TEI.roleName]{roleName} \hyperref[TEI.settlement]{settlement} \hyperref[TEI.state]{state} \hyperref[TEI.surname]{surname}\par 
    \item[spoken: ]
   \hyperref[TEI.annotationBlock]{annotationBlock}\par 
    \item[textstructure: ]
   \hyperref[TEI.floatingText]{floatingText}\par 
    \item[transcr: ]
   \hyperref[TEI.addSpan]{addSpan} \hyperref[TEI.am]{am} \hyperref[TEI.damage]{damage} \hyperref[TEI.damageSpan]{damageSpan} \hyperref[TEI.delSpan]{delSpan} \hyperref[TEI.ex]{ex} \hyperref[TEI.fw]{fw} \hyperref[TEI.handShift]{handShift} \hyperref[TEI.listTranspose]{listTranspose} \hyperref[TEI.metamark]{metamark} \hyperref[TEI.mod]{mod} \hyperref[TEI.redo]{redo} \hyperref[TEI.restore]{restore} \hyperref[TEI.retrace]{retrace} \hyperref[TEI.secl]{secl} \hyperref[TEI.space]{space} \hyperref[TEI.subst]{subst} \hyperref[TEI.substJoin]{substJoin} \hyperref[TEI.supplied]{supplied} \hyperref[TEI.surplus]{surplus} \hyperref[TEI.undo]{undo}\par des données textuelles
    \item[{Note}]
  \par
L'élément \hyperref[TEI.seg]{<seg>} peut être utilisé à la discrétion de l'encodeur pour baliser tout segment du texte intéressant pour un traitement informatique. L'un des usages de cet élément est d'encoder des caractéristiques textuelles pour lesquelles aucun balisage approprié n'est défini par ailleurs. Un autre usage consiste à fournir un identifiant pour un segment vers lequel pointe un autre élément - c'est-à-dire à fournir une cible, ou une partie de cible, pour un élément \hyperref[TEI.ptr]{<ptr>} ou pour un autre élément similaire.
    \item[{Exemple}]
  \leavevmode\bgroup\exampleFont \begin{shaded}\noindent\mbox{}{<\textbf{seg}>}Quand partez-vous ?{</\textbf{seg}>}\mbox{}\newline 
{<\textbf{seg}>}Demain.{</\textbf{seg}>}\end{shaded}\egroup 


    \item[{Exemple}]
  \leavevmode\bgroup\exampleFont \begin{shaded}\noindent\mbox{}{<\textbf{s}>}C' était à {<\textbf{seg}\hspace*{6pt}{type}="{toponyme}">}Mégara{</\textbf{seg}>}, faubourg de {<\textbf{seg}\hspace*{6pt}{type}="{topon}">}Carthage{</\textbf{seg}>}, dans les jardins d' {<\textbf{seg}\hspace*{6pt}{type}="{patronyme}">}Hamilcar{</\textbf{seg}>}. {</\textbf{s}>}\end{shaded}\egroup 


    \item[{Exemple}]
  \leavevmode\bgroup\exampleFont \begin{shaded}\noindent\mbox{}{<\textbf{seg}\hspace*{6pt}{type}="{preambule}">}La magnificence et la galanterie n'ont jamais paru en {<\textbf{seg}\hspace*{6pt}{type}="{toponyme}">}France{</\textbf{seg}>} avec tant d'éclat que {<\textbf{seg}\hspace*{6pt}{type}="{date}">}dans les dernières\mbox{}\newline 
\hspace*{6pt}\hspace*{6pt} années du règne de {<\textbf{seg}\hspace*{6pt}{type}="{patronyme}">}Henri second{</\textbf{seg}>}. {</\textbf{seg}>}\mbox{}\newline 
{</\textbf{seg}>}\end{shaded}\egroup 


    \item[{Modèle de contenu}]
  \mbox{}\hfill\\[-10pt]\begin{Verbatim}[fontsize=\small]
<content>
 <macroRef key="macro.paraContent"/>
</content>
    
\end{Verbatim}

    \item[{Schéma Declaration}]
  \mbox{}\hfill\\[-10pt]\begin{Verbatim}[fontsize=\small]
element seg
{
   tei_att.global.attributes,
   tei_att.segLike.attributes,
   tei_att.typed.attributes,
   tei_att.written.attributes,
   tei_macro.paraContent}
\end{Verbatim}

\end{reflist}  \index{semantics=<semantics>|oddindex}\index{definitionURL=@definitionURL!<semantics>|oddindex}\index{encoding=@encoding!<semantics>|oddindex}
\begin{reflist}
\item[]\begin{specHead}{TEI.semantics}{<semantics> }\end{specHead} 
    \item[{Namespace}]
  http://www.w3.org/1998/Math/MathML
    \item[{Module}]
  derived-module-tei.istex
    \item[{Attributs}]
  Attributs\hfil\\[-10pt]\begin{sansreflist}
    \item[@definitionURL]
  
\begin{reflist}
    \item[{Statut}]
  Optionel
    \item[{Type de données}]
  \xref{https://www.w3.org/TR/xmlschema-2/\#}{}
\end{reflist}  
    \item[@encoding]
  
\begin{reflist}
    \item[{Statut}]
  Optionel
    \item[{Type de données}]
  \xref{https://www.w3.org/TR/xmlschema-2/\#}{}
\end{reflist}  
\end{sansreflist}  
    \item[{Contenu dans}]
  
    \item[derived-module-tei.istex: ]
   \hyperref[TEI.math]{math}
    \item[{Peut contenir}]
  
    \item[derived-module-tei.istex: ]
   \hyperref[TEI.annotation]{annotation} \hyperref[TEI.menclose]{menclose} \hyperref[TEI.mfenced]{mfenced} \hyperref[TEI.mfrac]{mfrac} \hyperref[TEI.mi]{mi} \hyperref[TEI.mmultiscripts]{mmultiscripts} \hyperref[TEI.mn]{mn} \hyperref[TEI.mo]{mo} \hyperref[TEI.mover]{mover} \hyperref[TEI.mpadded]{mpadded} \hyperref[TEI.mphantom]{mphantom} \hyperref[TEI.mprescripts]{mprescripts} \hyperref[TEI.mrow]{mrow} \hyperref[TEI.mspace]{mspace} \hyperref[TEI.msqrt]{msqrt} \hyperref[TEI.mstyle]{mstyle} \hyperref[TEI.msub]{msub} \hyperref[TEI.msubsup]{msubsup} \hyperref[TEI.msup]{msup} \hyperref[TEI.msupsub]{msupsub} \hyperref[TEI.mtable]{mtable} \hyperref[TEI.mtd]{mtd} \hyperref[TEI.mtext]{mtext} \hyperref[TEI.mtr]{mtr} \hyperref[TEI.munder]{munder} \hyperref[TEI.munderover]{munderover} \hyperref[TEI.none]{none}\par des données textuelles
    \item[{Modèle de contenu}]
  \mbox{}\hfill\\[-10pt]\begin{Verbatim}[fontsize=\small]
<content>
 <alternate maxOccurs="unbounded"
  minOccurs="0">
  <textNode/>
  <elementRef key="mstyle"/>
  <elementRef key="mtr"/>
  <elementRef key="mtd"/>
  <elementRef key="mrow"/>
  <elementRef key="mi"/>
  <elementRef key="mn"/>
  <elementRef key="mtext"/>
  <elementRef key="mfrac"/>
  <elementRef key="mspace"/>
  <elementRef key="msqrt"/>
  <elementRef key="msub"/>
  <elementRef key="msup"/>
  <elementRef key="mo"/>
  <elementRef key="mover"/>
  <elementRef key="mfenced"/>
  <elementRef key="mtable"/>
  <elementRef key="msubsup"/>
  <elementRef key="msupsub"/>
  <elementRef key="mmultiscripts"/>
  <elementRef key="munderover"/>
  <elementRef key="mprescripts"/>
  <elementRef key="none"/>
  <elementRef key="munder"/>
  <elementRef key="mphantom"/>
  <elementRef key="mpadded"/>
  <elementRef key="menclose"/>
  <elementRef key="annotation"/>
 </alternate>
</content>
    
\end{Verbatim}

    \item[{Schéma Declaration}]
  \mbox{}\hfill\\[-10pt]\begin{Verbatim}[fontsize=\small]
element semantics
{
   attribute definitionURL { definitionURL }?,
   attribute encoding { encoding }?,
   (
      text
    | tei_mstyle    | tei_mtr    | tei_mtd    | tei_mrow    | tei_mi    | tei_mn    | tei_mtext    | tei_mfrac    | tei_mspace    | tei_msqrt    | tei_msub    | tei_msup    | tei_mo    | tei_mover    | tei_mfenced    | tei_mtable    | tei_msubsup    | tei_msupsub    | tei_mmultiscripts    | tei_munderover    | tei_mprescripts    | tei_none    | tei_munder    | tei_mphantom    | tei_mpadded    | tei_menclose    | tei_annotation   )*
}
\end{Verbatim}

\end{reflist}  \index{series=<series>|oddindex}
\begin{reflist}
\item[]\begin{specHead}{TEI.series}{<series> }(informations sur la série) contient une information sur la série dans laquelle une monographie ou un autre élément bibliographique ont paru. [\xref{http://www.tei-c.org/release/doc/tei-p5-doc/en/html/CO.html\#COBICOL}{3.11.2.1. Analytic, Monographic, and Series Levels}]\end{specHead} 
    \item[{Module}]
  core
    \item[{Attributs}]
  Attributs \hyperref[TEI.att.global]{att.global} (\textit{@xml:id}, \textit{@n}, \textit{@xml:lang}, \textit{@xml:base}, \textit{@xml:space})  (\hyperref[TEI.att.global.rendition]{att.global.rendition} (\textit{@rend}, \textit{@style}, \textit{@rendition})) (\hyperref[TEI.att.global.linking]{att.global.linking} (\textit{@corresp}, \textit{@synch}, \textit{@sameAs}, \textit{@copyOf}, \textit{@next}, \textit{@prev}, \textit{@exclude}, \textit{@select})) (\hyperref[TEI.att.global.analytic]{att.global.analytic} (\textit{@ana})) (\hyperref[TEI.att.global.facs]{att.global.facs} (\textit{@facs})) (\hyperref[TEI.att.global.change]{att.global.change} (\textit{@change})) (\hyperref[TEI.att.global.responsibility]{att.global.responsibility} (\textit{@cert}, \textit{@resp})) (\hyperref[TEI.att.global.source]{att.global.source} (\textit{@source}))
    \item[{Membre du}]
  \hyperref[TEI.model.biblPart]{model.biblPart}
    \item[{Contenu dans}]
  
    \item[core: ]
   \hyperref[TEI.bibl]{bibl} \hyperref[TEI.biblStruct]{biblStruct}
    \item[{Peut contenir}]
  
    \item[analysis: ]
   \hyperref[TEI.interp]{interp} \hyperref[TEI.interpGrp]{interpGrp} \hyperref[TEI.span]{span} \hyperref[TEI.spanGrp]{spanGrp}\par 
    \item[core: ]
   \hyperref[TEI.biblScope]{biblScope} \hyperref[TEI.cb]{cb} \hyperref[TEI.editor]{editor} \hyperref[TEI.gap]{gap} \hyperref[TEI.gb]{gb} \hyperref[TEI.index]{index} \hyperref[TEI.lb]{lb} \hyperref[TEI.milestone]{milestone} \hyperref[TEI.note]{note} \hyperref[TEI.pb]{pb} \hyperref[TEI.ptr]{ptr} \hyperref[TEI.ref]{ref} \hyperref[TEI.respStmt]{respStmt} \hyperref[TEI.textLang]{textLang} \hyperref[TEI.title]{title}\par 
    \item[figures: ]
   \hyperref[TEI.figure]{figure} \hyperref[TEI.notatedMusic]{notatedMusic}\par 
    \item[header: ]
   \hyperref[TEI.availability]{availability} \hyperref[TEI.idno]{idno}\par 
    \item[iso-fs: ]
   \hyperref[TEI.fLib]{fLib} \hyperref[TEI.fs]{fs} \hyperref[TEI.fvLib]{fvLib}\par 
    \item[linking: ]
   \hyperref[TEI.alt]{alt} \hyperref[TEI.altGrp]{altGrp} \hyperref[TEI.anchor]{anchor} \hyperref[TEI.join]{join} \hyperref[TEI.joinGrp]{joinGrp} \hyperref[TEI.link]{link} \hyperref[TEI.linkGrp]{linkGrp} \hyperref[TEI.timeline]{timeline}\par 
    \item[msdescription: ]
   \hyperref[TEI.source]{source}\par 
    \item[transcr: ]
   \hyperref[TEI.addSpan]{addSpan} \hyperref[TEI.damageSpan]{damageSpan} \hyperref[TEI.delSpan]{delSpan} \hyperref[TEI.fw]{fw} \hyperref[TEI.listTranspose]{listTranspose} \hyperref[TEI.metamark]{metamark} \hyperref[TEI.space]{space} \hyperref[TEI.substJoin]{substJoin}\par des données textuelles
    \item[{Exemple}]
  \leavevmode\bgroup\exampleFont \begin{shaded}\noindent\mbox{}{<\textbf{series}\hspace*{6pt}{xml:lang}="{de}">}\mbox{}\newline 
\hspace*{6pt}{<\textbf{title}\hspace*{6pt}{level}="{s}">}Halbgraue Reihe zur Historischen Fachinformatik{</\textbf{title}>}\mbox{}\newline 
\hspace*{6pt}{<\textbf{respStmt}>}\mbox{}\newline 
\hspace*{6pt}\hspace*{6pt}{<\textbf{resp}>}Herausgegeben von{</\textbf{resp}>}\mbox{}\newline 
\hspace*{6pt}\hspace*{6pt}{<\textbf{name}\hspace*{6pt}{type}="{person}">}Manfred Thaller{</\textbf{name}>}\mbox{}\newline 
\hspace*{6pt}\hspace*{6pt}{<\textbf{name}\hspace*{6pt}{type}="{org}">}Max-Planck-Institut für Geschichte{</\textbf{name}>}\mbox{}\newline 
\hspace*{6pt}{</\textbf{respStmt}>}\mbox{}\newline 
\hspace*{6pt}{<\textbf{title}\hspace*{6pt}{level}="{s}">}Serie A: Historische Quellenkunden{</\textbf{title}>}\mbox{}\newline 
\hspace*{6pt}{<\textbf{biblScope}>}Band 11{</\textbf{biblScope}>}\mbox{}\newline 
{</\textbf{series}>}\end{shaded}\egroup 


    \item[{Modèle de contenu}]
  \mbox{}\hfill\\[-10pt]\begin{Verbatim}[fontsize=\small]
<content>
 <alternate maxOccurs="unbounded"
  minOccurs="0">
  <textNode/>
  <classRef key="model.gLike"/>
  <elementRef key="title"/>
  <classRef key="model.ptrLike"/>
  <elementRef key="editor"/>
  <elementRef key="respStmt"/>
  <elementRef key="biblScope"/>
  <elementRef key="idno"/>
  <elementRef key="textLang"/>
  <classRef key="model.global"/>
  <elementRef key="availability"/>
 </alternate>
</content>
    
\end{Verbatim}

    \item[{Schéma Declaration}]
  \mbox{}\hfill\\[-10pt]\begin{Verbatim}[fontsize=\small]
element series
{
   tei_att.global.attributes,
   (
      text
    | tei_model.gLike    | tei_title    | tei_model.ptrLike    | tei_editor    | tei_respStmt    | tei_biblScope    | tei_idno    | tei_textLang    | tei_model.global    | tei_availability   )*
}
\end{Verbatim}

\end{reflist}  \index{seriesStmt=<seriesStmt>|oddindex}
\begin{reflist}
\item[]\begin{specHead}{TEI.seriesStmt}{<seriesStmt> }(mention de collection) regroupe toute information relative à la collection (si elle existe) à laquelle appartient une publication. [\xref{http://www.tei-c.org/release/doc/tei-p5-doc/en/html/HD.html\#HD26}{2.2.5. The Series Statement} \xref{http://www.tei-c.org/release/doc/tei-p5-doc/en/html/HD.html\#HD2}{2.2. The File Description}]\end{specHead} 
    \item[{Module}]
  header
    \item[{Attributs}]
  Attributs \hyperref[TEI.att.global]{att.global} (\textit{@xml:id}, \textit{@n}, \textit{@xml:lang}, \textit{@xml:base}, \textit{@xml:space})  (\hyperref[TEI.att.global.rendition]{att.global.rendition} (\textit{@rend}, \textit{@style}, \textit{@rendition})) (\hyperref[TEI.att.global.linking]{att.global.linking} (\textit{@corresp}, \textit{@synch}, \textit{@sameAs}, \textit{@copyOf}, \textit{@next}, \textit{@prev}, \textit{@exclude}, \textit{@select})) (\hyperref[TEI.att.global.analytic]{att.global.analytic} (\textit{@ana})) (\hyperref[TEI.att.global.facs]{att.global.facs} (\textit{@facs})) (\hyperref[TEI.att.global.change]{att.global.change} (\textit{@change})) (\hyperref[TEI.att.global.responsibility]{att.global.responsibility} (\textit{@cert}, \textit{@resp})) (\hyperref[TEI.att.global.source]{att.global.source} (\textit{@source}))
    \item[{Contenu dans}]
  
    \item[header: ]
   \hyperref[TEI.biblFull]{biblFull} \hyperref[TEI.fileDesc]{fileDesc}
    \item[{Peut contenir}]
  
    \item[core: ]
   \hyperref[TEI.biblScope]{biblScope} \hyperref[TEI.editor]{editor} \hyperref[TEI.p]{p} \hyperref[TEI.respStmt]{respStmt} \hyperref[TEI.title]{title}\par 
    \item[header: ]
   \hyperref[TEI.idno]{idno}\par 
    \item[linking: ]
   \hyperref[TEI.ab]{ab}
    \item[{Exemple}]
  \leavevmode\bgroup\exampleFont \begin{shaded}\noindent\mbox{}{<\textbf{seriesStmt}>}\mbox{}\newline 
\hspace*{6pt}{<\textbf{title}>}Babel{</\textbf{title}>}\mbox{}\newline 
\hspace*{6pt}{<\textbf{respStmt}>}\mbox{}\newline 
\hspace*{6pt}\hspace*{6pt}{<\textbf{resp}>}directeur de collection{</\textbf{resp}>}\mbox{}\newline 
\hspace*{6pt}\hspace*{6pt}{<\textbf{name}>}Jacques Dubois{</\textbf{name}>}\mbox{}\newline 
\hspace*{6pt}{</\textbf{respStmt}>}\mbox{}\newline 
\hspace*{6pt}{<\textbf{respStmt}>}\mbox{}\newline 
\hspace*{6pt}\hspace*{6pt}{<\textbf{resp}>}directeur de collection{</\textbf{resp}>}\mbox{}\newline 
\hspace*{6pt}\hspace*{6pt}{<\textbf{name}>}Hubert Nyssen{</\textbf{name}>}\mbox{}\newline 
\hspace*{6pt}{</\textbf{respStmt}>}\mbox{}\newline 
\hspace*{6pt}{<\textbf{idno}\hspace*{6pt}{type}="{ISSN}">}1140-3853{</\textbf{idno}>}\mbox{}\newline 
{</\textbf{seriesStmt}>}\end{shaded}\egroup 


    \item[{Modèle de contenu}]
  \mbox{}\hfill\\[-10pt]\begin{Verbatim}[fontsize=\small]
<content>
 <alternate maxOccurs="1" minOccurs="1">
  <classRef key="model.pLike"
   maxOccurs="unbounded" minOccurs="1"/>
  <sequence maxOccurs="1" minOccurs="1">
   <elementRef key="title"
    maxOccurs="unbounded" minOccurs="1"/>
   <alternate maxOccurs="unbounded"
    minOccurs="0">
    <elementRef key="editor"/>
    <elementRef key="respStmt"/>
   </alternate>
   <alternate maxOccurs="unbounded"
    minOccurs="0">
    <elementRef key="idno"/>
    <elementRef key="biblScope"/>
   </alternate>
  </sequence>
 </alternate>
</content>
    
\end{Verbatim}

    \item[{Schéma Declaration}]
  \mbox{}\hfill\\[-10pt]\begin{Verbatim}[fontsize=\small]
element seriesStmt
{
   tei_att.global.attributes,
   (
      tei_model.pLike+
    | (
         tei_title+,
         ( tei_editor | tei_respStmt )*,
         ( tei_idno | tei_biblScope )*
      )
   )
}
\end{Verbatim}

\end{reflist}  \index{settlement=<settlement>|oddindex}
\begin{reflist}
\item[]\begin{specHead}{TEI.settlement}{<settlement> }(lieu de peuplement) contient le nom d'un lieu de peuplement comme une cité, une ville ou un village, identifié comme une unité géo-politique ou administrative unique. [\xref{http://www.tei-c.org/release/doc/tei-p5-doc/en/html/ND.html\#NDPLAC}{13.2.3. Place Names}]\end{specHead} 
    \item[{Module}]
  namesdates
    \item[{Attributs}]
  Attributs \hyperref[TEI.att.global]{att.global} (\textit{@xml:id}, \textit{@n}, \textit{@xml:lang}, \textit{@xml:base}, \textit{@xml:space})  (\hyperref[TEI.att.global.rendition]{att.global.rendition} (\textit{@rend}, \textit{@style}, \textit{@rendition})) (\hyperref[TEI.att.global.linking]{att.global.linking} (\textit{@corresp}, \textit{@synch}, \textit{@sameAs}, \textit{@copyOf}, \textit{@next}, \textit{@prev}, \textit{@exclude}, \textit{@select})) (\hyperref[TEI.att.global.analytic]{att.global.analytic} (\textit{@ana})) (\hyperref[TEI.att.global.facs]{att.global.facs} (\textit{@facs})) (\hyperref[TEI.att.global.change]{att.global.change} (\textit{@change})) (\hyperref[TEI.att.global.responsibility]{att.global.responsibility} (\textit{@cert}, \textit{@resp})) (\hyperref[TEI.att.global.source]{att.global.source} (\textit{@source})) \hyperref[TEI.att.naming]{att.naming} (\textit{@role}, \textit{@nymRef})  (\hyperref[TEI.att.canonical]{att.canonical} (\textit{@key}, \textit{@ref})) \hyperref[TEI.att.typed]{att.typed} (\textit{@type}, \textit{@subtype}) \hyperref[TEI.att.datable]{att.datable} (\textit{@calendar}, \textit{@period})  (\hyperref[TEI.att.datable.w3c]{att.datable.w3c} (\textit{@when}, \textit{@notBefore}, \textit{@notAfter}, \textit{@from}, \textit{@to})) (\hyperref[TEI.att.datable.iso]{att.datable.iso} (\textit{@when-iso}, \textit{@notBefore-iso}, \textit{@notAfter-iso}, \textit{@from-iso}, \textit{@to-iso})) (\hyperref[TEI.att.datable.custom]{att.datable.custom} (\textit{@when-custom}, \textit{@notBefore-custom}, \textit{@notAfter-custom}, \textit{@from-custom}, \textit{@to-custom}, \textit{@datingPoint}, \textit{@datingMethod}))
    \item[{Membre du}]
  \hyperref[TEI.model.placeNamePart]{model.placeNamePart}
    \item[{Contenu dans}]
  
    \item[analysis: ]
   \hyperref[TEI.cl]{cl} \hyperref[TEI.phr]{phr} \hyperref[TEI.s]{s} \hyperref[TEI.span]{span}\par 
    \item[core: ]
   \hyperref[TEI.abbr]{abbr} \hyperref[TEI.add]{add} \hyperref[TEI.addrLine]{addrLine} \hyperref[TEI.address]{address} \hyperref[TEI.author]{author} \hyperref[TEI.bibl]{bibl} \hyperref[TEI.biblScope]{biblScope} \hyperref[TEI.citedRange]{citedRange} \hyperref[TEI.corr]{corr} \hyperref[TEI.date]{date} \hyperref[TEI.del]{del} \hyperref[TEI.desc]{desc} \hyperref[TEI.distinct]{distinct} \hyperref[TEI.editor]{editor} \hyperref[TEI.email]{email} \hyperref[TEI.emph]{emph} \hyperref[TEI.expan]{expan} \hyperref[TEI.foreign]{foreign} \hyperref[TEI.gloss]{gloss} \hyperref[TEI.head]{head} \hyperref[TEI.headItem]{headItem} \hyperref[TEI.headLabel]{headLabel} \hyperref[TEI.hi]{hi} \hyperref[TEI.item]{item} \hyperref[TEI.l]{l} \hyperref[TEI.label]{label} \hyperref[TEI.measure]{measure} \hyperref[TEI.meeting]{meeting} \hyperref[TEI.mentioned]{mentioned} \hyperref[TEI.name]{name} \hyperref[TEI.note]{note} \hyperref[TEI.num]{num} \hyperref[TEI.orig]{orig} \hyperref[TEI.p]{p} \hyperref[TEI.pubPlace]{pubPlace} \hyperref[TEI.publisher]{publisher} \hyperref[TEI.q]{q} \hyperref[TEI.quote]{quote} \hyperref[TEI.ref]{ref} \hyperref[TEI.reg]{reg} \hyperref[TEI.resp]{resp} \hyperref[TEI.rs]{rs} \hyperref[TEI.said]{said} \hyperref[TEI.sic]{sic} \hyperref[TEI.soCalled]{soCalled} \hyperref[TEI.speaker]{speaker} \hyperref[TEI.stage]{stage} \hyperref[TEI.street]{street} \hyperref[TEI.term]{term} \hyperref[TEI.textLang]{textLang} \hyperref[TEI.time]{time} \hyperref[TEI.title]{title} \hyperref[TEI.unclear]{unclear}\par 
    \item[figures: ]
   \hyperref[TEI.cell]{cell} \hyperref[TEI.figDesc]{figDesc}\par 
    \item[header: ]
   \hyperref[TEI.authority]{authority} \hyperref[TEI.change]{change} \hyperref[TEI.classCode]{classCode} \hyperref[TEI.creation]{creation} \hyperref[TEI.distributor]{distributor} \hyperref[TEI.edition]{edition} \hyperref[TEI.extent]{extent} \hyperref[TEI.funder]{funder} \hyperref[TEI.language]{language} \hyperref[TEI.licence]{licence} \hyperref[TEI.rendition]{rendition}\par 
    \item[iso-fs: ]
   \hyperref[TEI.fDescr]{fDescr} \hyperref[TEI.fsDescr]{fsDescr}\par 
    \item[linking: ]
   \hyperref[TEI.ab]{ab} \hyperref[TEI.seg]{seg}\par 
    \item[msdescription: ]
   \hyperref[TEI.accMat]{accMat} \hyperref[TEI.acquisition]{acquisition} \hyperref[TEI.additions]{additions} \hyperref[TEI.altIdentifier]{altIdentifier} \hyperref[TEI.catchwords]{catchwords} \hyperref[TEI.collation]{collation} \hyperref[TEI.colophon]{colophon} \hyperref[TEI.condition]{condition} \hyperref[TEI.custEvent]{custEvent} \hyperref[TEI.decoNote]{decoNote} \hyperref[TEI.explicit]{explicit} \hyperref[TEI.filiation]{filiation} \hyperref[TEI.finalRubric]{finalRubric} \hyperref[TEI.foliation]{foliation} \hyperref[TEI.heraldry]{heraldry} \hyperref[TEI.incipit]{incipit} \hyperref[TEI.layout]{layout} \hyperref[TEI.material]{material} \hyperref[TEI.msIdentifier]{msIdentifier} \hyperref[TEI.musicNotation]{musicNotation} \hyperref[TEI.objectType]{objectType} \hyperref[TEI.origDate]{origDate} \hyperref[TEI.origPlace]{origPlace} \hyperref[TEI.origin]{origin} \hyperref[TEI.provenance]{provenance} \hyperref[TEI.rubric]{rubric} \hyperref[TEI.secFol]{secFol} \hyperref[TEI.signatures]{signatures} \hyperref[TEI.source]{source} \hyperref[TEI.stamp]{stamp} \hyperref[TEI.summary]{summary} \hyperref[TEI.support]{support} \hyperref[TEI.surrogates]{surrogates} \hyperref[TEI.typeNote]{typeNote} \hyperref[TEI.watermark]{watermark}\par 
    \item[namesdates: ]
   \hyperref[TEI.addName]{addName} \hyperref[TEI.affiliation]{affiliation} \hyperref[TEI.country]{country} \hyperref[TEI.forename]{forename} \hyperref[TEI.genName]{genName} \hyperref[TEI.geogName]{geogName} \hyperref[TEI.location]{location} \hyperref[TEI.nameLink]{nameLink} \hyperref[TEI.org]{org} \hyperref[TEI.orgName]{orgName} \hyperref[TEI.persName]{persName} \hyperref[TEI.place]{place} \hyperref[TEI.placeName]{placeName} \hyperref[TEI.region]{region} \hyperref[TEI.roleName]{roleName} \hyperref[TEI.settlement]{settlement} \hyperref[TEI.surname]{surname}\par 
    \item[spoken: ]
   \hyperref[TEI.annotationBlock]{annotationBlock}\par 
    \item[standOff: ]
   \hyperref[TEI.listAnnotation]{listAnnotation}\par 
    \item[textstructure: ]
   \hyperref[TEI.docAuthor]{docAuthor} \hyperref[TEI.docDate]{docDate} \hyperref[TEI.docEdition]{docEdition} \hyperref[TEI.titlePart]{titlePart}\par 
    \item[transcr: ]
   \hyperref[TEI.damage]{damage} \hyperref[TEI.fw]{fw} \hyperref[TEI.metamark]{metamark} \hyperref[TEI.mod]{mod} \hyperref[TEI.restore]{restore} \hyperref[TEI.retrace]{retrace} \hyperref[TEI.secl]{secl} \hyperref[TEI.supplied]{supplied} \hyperref[TEI.surplus]{surplus}
    \item[{Peut contenir}]
  
    \item[analysis: ]
   \hyperref[TEI.c]{c} \hyperref[TEI.cl]{cl} \hyperref[TEI.interp]{interp} \hyperref[TEI.interpGrp]{interpGrp} \hyperref[TEI.m]{m} \hyperref[TEI.pc]{pc} \hyperref[TEI.phr]{phr} \hyperref[TEI.s]{s} \hyperref[TEI.span]{span} \hyperref[TEI.spanGrp]{spanGrp} \hyperref[TEI.w]{w}\par 
    \item[core: ]
   \hyperref[TEI.abbr]{abbr} \hyperref[TEI.add]{add} \hyperref[TEI.address]{address} \hyperref[TEI.binaryObject]{binaryObject} \hyperref[TEI.cb]{cb} \hyperref[TEI.choice]{choice} \hyperref[TEI.corr]{corr} \hyperref[TEI.date]{date} \hyperref[TEI.del]{del} \hyperref[TEI.distinct]{distinct} \hyperref[TEI.email]{email} \hyperref[TEI.emph]{emph} \hyperref[TEI.expan]{expan} \hyperref[TEI.foreign]{foreign} \hyperref[TEI.gap]{gap} \hyperref[TEI.gb]{gb} \hyperref[TEI.gloss]{gloss} \hyperref[TEI.graphic]{graphic} \hyperref[TEI.hi]{hi} \hyperref[TEI.index]{index} \hyperref[TEI.lb]{lb} \hyperref[TEI.measure]{measure} \hyperref[TEI.measureGrp]{measureGrp} \hyperref[TEI.media]{media} \hyperref[TEI.mentioned]{mentioned} \hyperref[TEI.milestone]{milestone} \hyperref[TEI.name]{name} \hyperref[TEI.note]{note} \hyperref[TEI.num]{num} \hyperref[TEI.orig]{orig} \hyperref[TEI.pb]{pb} \hyperref[TEI.ptr]{ptr} \hyperref[TEI.ref]{ref} \hyperref[TEI.reg]{reg} \hyperref[TEI.rs]{rs} \hyperref[TEI.sic]{sic} \hyperref[TEI.soCalled]{soCalled} \hyperref[TEI.term]{term} \hyperref[TEI.time]{time} \hyperref[TEI.title]{title} \hyperref[TEI.unclear]{unclear}\par 
    \item[derived-module-tei.istex: ]
   \hyperref[TEI.math]{math} \hyperref[TEI.mrow]{mrow}\par 
    \item[figures: ]
   \hyperref[TEI.figure]{figure} \hyperref[TEI.formula]{formula} \hyperref[TEI.notatedMusic]{notatedMusic}\par 
    \item[header: ]
   \hyperref[TEI.idno]{idno}\par 
    \item[iso-fs: ]
   \hyperref[TEI.fLib]{fLib} \hyperref[TEI.fs]{fs} \hyperref[TEI.fvLib]{fvLib}\par 
    \item[linking: ]
   \hyperref[TEI.alt]{alt} \hyperref[TEI.altGrp]{altGrp} \hyperref[TEI.anchor]{anchor} \hyperref[TEI.join]{join} \hyperref[TEI.joinGrp]{joinGrp} \hyperref[TEI.link]{link} \hyperref[TEI.linkGrp]{linkGrp} \hyperref[TEI.seg]{seg} \hyperref[TEI.timeline]{timeline}\par 
    \item[msdescription: ]
   \hyperref[TEI.catchwords]{catchwords} \hyperref[TEI.depth]{depth} \hyperref[TEI.dim]{dim} \hyperref[TEI.dimensions]{dimensions} \hyperref[TEI.height]{height} \hyperref[TEI.heraldry]{heraldry} \hyperref[TEI.locus]{locus} \hyperref[TEI.locusGrp]{locusGrp} \hyperref[TEI.material]{material} \hyperref[TEI.objectType]{objectType} \hyperref[TEI.origDate]{origDate} \hyperref[TEI.origPlace]{origPlace} \hyperref[TEI.secFol]{secFol} \hyperref[TEI.signatures]{signatures} \hyperref[TEI.source]{source} \hyperref[TEI.stamp]{stamp} \hyperref[TEI.watermark]{watermark} \hyperref[TEI.width]{width}\par 
    \item[namesdates: ]
   \hyperref[TEI.addName]{addName} \hyperref[TEI.affiliation]{affiliation} \hyperref[TEI.country]{country} \hyperref[TEI.forename]{forename} \hyperref[TEI.genName]{genName} \hyperref[TEI.geogName]{geogName} \hyperref[TEI.location]{location} \hyperref[TEI.nameLink]{nameLink} \hyperref[TEI.orgName]{orgName} \hyperref[TEI.persName]{persName} \hyperref[TEI.placeName]{placeName} \hyperref[TEI.region]{region} \hyperref[TEI.roleName]{roleName} \hyperref[TEI.settlement]{settlement} \hyperref[TEI.state]{state} \hyperref[TEI.surname]{surname}\par 
    \item[spoken: ]
   \hyperref[TEI.annotationBlock]{annotationBlock}\par 
    \item[transcr: ]
   \hyperref[TEI.addSpan]{addSpan} \hyperref[TEI.am]{am} \hyperref[TEI.damage]{damage} \hyperref[TEI.damageSpan]{damageSpan} \hyperref[TEI.delSpan]{delSpan} \hyperref[TEI.ex]{ex} \hyperref[TEI.fw]{fw} \hyperref[TEI.handShift]{handShift} \hyperref[TEI.listTranspose]{listTranspose} \hyperref[TEI.metamark]{metamark} \hyperref[TEI.mod]{mod} \hyperref[TEI.redo]{redo} \hyperref[TEI.restore]{restore} \hyperref[TEI.retrace]{retrace} \hyperref[TEI.secl]{secl} \hyperref[TEI.space]{space} \hyperref[TEI.subst]{subst} \hyperref[TEI.substJoin]{substJoin} \hyperref[TEI.supplied]{supplied} \hyperref[TEI.surplus]{surplus} \hyperref[TEI.undo]{undo}\par des données textuelles
    \item[{Exemple}]
  \leavevmode\bgroup\exampleFont \begin{shaded}\noindent\mbox{}{<\textbf{placeName}>}\mbox{}\newline 
\hspace*{6pt}{<\textbf{settlement}\hspace*{6pt}{type}="{town}">}Brest{</\textbf{settlement}>}\mbox{}\newline 
\hspace*{6pt}{<\textbf{region}>}Bretagne{</\textbf{region}>}\mbox{}\newline 
{</\textbf{placeName}>}\end{shaded}\egroup 


    \item[{Modèle de contenu}]
  \mbox{}\hfill\\[-10pt]\begin{Verbatim}[fontsize=\small]
<content>
 <macroRef key="macro.phraseSeq"/>
</content>
    
\end{Verbatim}

    \item[{Schéma Declaration}]
  \mbox{}\hfill\\[-10pt]\begin{Verbatim}[fontsize=\small]
element settlement
{
   tei_att.global.attributes,
   tei_att.naming.attributes,
   tei_att.typed.attributes,
   tei_att.datable.attributes,
   tei_macro.phraseSeq}
\end{Verbatim}

\end{reflist}  \index{sic=<sic>|oddindex}
\begin{reflist}
\item[]\begin{specHead}{TEI.sic}{<sic> }(du latin, ainsi) contient du texte reproduit quoiqu'il est apparemment incorrect ou inexact [\xref{http://www.tei-c.org/release/doc/tei-p5-doc/en/html/CO.html\#COEDCOR}{3.4.1. Apparent Errors}]\end{specHead} 
    \item[{Module}]
  core
    \item[{Attributs}]
  Attributs \hyperref[TEI.att.global]{att.global} (\textit{@xml:id}, \textit{@n}, \textit{@xml:lang}, \textit{@xml:base}, \textit{@xml:space})  (\hyperref[TEI.att.global.rendition]{att.global.rendition} (\textit{@rend}, \textit{@style}, \textit{@rendition})) (\hyperref[TEI.att.global.linking]{att.global.linking} (\textit{@corresp}, \textit{@synch}, \textit{@sameAs}, \textit{@copyOf}, \textit{@next}, \textit{@prev}, \textit{@exclude}, \textit{@select})) (\hyperref[TEI.att.global.analytic]{att.global.analytic} (\textit{@ana})) (\hyperref[TEI.att.global.facs]{att.global.facs} (\textit{@facs})) (\hyperref[TEI.att.global.change]{att.global.change} (\textit{@change})) (\hyperref[TEI.att.global.responsibility]{att.global.responsibility} (\textit{@cert}, \textit{@resp})) (\hyperref[TEI.att.global.source]{att.global.source} (\textit{@source}))
    \item[{Membre du}]
  \hyperref[TEI.model.choicePart]{model.choicePart} \hyperref[TEI.model.pPart.transcriptional]{model.pPart.transcriptional}
    \item[{Contenu dans}]
  
    \item[analysis: ]
   \hyperref[TEI.cl]{cl} \hyperref[TEI.pc]{pc} \hyperref[TEI.phr]{phr} \hyperref[TEI.s]{s} \hyperref[TEI.w]{w}\par 
    \item[core: ]
   \hyperref[TEI.abbr]{abbr} \hyperref[TEI.add]{add} \hyperref[TEI.addrLine]{addrLine} \hyperref[TEI.author]{author} \hyperref[TEI.bibl]{bibl} \hyperref[TEI.biblScope]{biblScope} \hyperref[TEI.choice]{choice} \hyperref[TEI.citedRange]{citedRange} \hyperref[TEI.corr]{corr} \hyperref[TEI.date]{date} \hyperref[TEI.del]{del} \hyperref[TEI.distinct]{distinct} \hyperref[TEI.editor]{editor} \hyperref[TEI.email]{email} \hyperref[TEI.emph]{emph} \hyperref[TEI.expan]{expan} \hyperref[TEI.foreign]{foreign} \hyperref[TEI.gloss]{gloss} \hyperref[TEI.head]{head} \hyperref[TEI.headItem]{headItem} \hyperref[TEI.headLabel]{headLabel} \hyperref[TEI.hi]{hi} \hyperref[TEI.item]{item} \hyperref[TEI.l]{l} \hyperref[TEI.label]{label} \hyperref[TEI.measure]{measure} \hyperref[TEI.mentioned]{mentioned} \hyperref[TEI.name]{name} \hyperref[TEI.note]{note} \hyperref[TEI.num]{num} \hyperref[TEI.orig]{orig} \hyperref[TEI.p]{p} \hyperref[TEI.pubPlace]{pubPlace} \hyperref[TEI.publisher]{publisher} \hyperref[TEI.q]{q} \hyperref[TEI.quote]{quote} \hyperref[TEI.ref]{ref} \hyperref[TEI.reg]{reg} \hyperref[TEI.rs]{rs} \hyperref[TEI.said]{said} \hyperref[TEI.sic]{sic} \hyperref[TEI.soCalled]{soCalled} \hyperref[TEI.speaker]{speaker} \hyperref[TEI.stage]{stage} \hyperref[TEI.street]{street} \hyperref[TEI.term]{term} \hyperref[TEI.textLang]{textLang} \hyperref[TEI.time]{time} \hyperref[TEI.title]{title} \hyperref[TEI.unclear]{unclear}\par 
    \item[figures: ]
   \hyperref[TEI.cell]{cell}\par 
    \item[header: ]
   \hyperref[TEI.change]{change} \hyperref[TEI.distributor]{distributor} \hyperref[TEI.edition]{edition} \hyperref[TEI.extent]{extent} \hyperref[TEI.licence]{licence}\par 
    \item[linking: ]
   \hyperref[TEI.ab]{ab} \hyperref[TEI.seg]{seg}\par 
    \item[msdescription: ]
   \hyperref[TEI.accMat]{accMat} \hyperref[TEI.acquisition]{acquisition} \hyperref[TEI.additions]{additions} \hyperref[TEI.catchwords]{catchwords} \hyperref[TEI.collation]{collation} \hyperref[TEI.colophon]{colophon} \hyperref[TEI.condition]{condition} \hyperref[TEI.custEvent]{custEvent} \hyperref[TEI.decoNote]{decoNote} \hyperref[TEI.explicit]{explicit} \hyperref[TEI.filiation]{filiation} \hyperref[TEI.finalRubric]{finalRubric} \hyperref[TEI.foliation]{foliation} \hyperref[TEI.heraldry]{heraldry} \hyperref[TEI.incipit]{incipit} \hyperref[TEI.layout]{layout} \hyperref[TEI.material]{material} \hyperref[TEI.musicNotation]{musicNotation} \hyperref[TEI.objectType]{objectType} \hyperref[TEI.origDate]{origDate} \hyperref[TEI.origPlace]{origPlace} \hyperref[TEI.origin]{origin} \hyperref[TEI.provenance]{provenance} \hyperref[TEI.rubric]{rubric} \hyperref[TEI.secFol]{secFol} \hyperref[TEI.signatures]{signatures} \hyperref[TEI.source]{source} \hyperref[TEI.stamp]{stamp} \hyperref[TEI.summary]{summary} \hyperref[TEI.support]{support} \hyperref[TEI.surrogates]{surrogates} \hyperref[TEI.typeNote]{typeNote} \hyperref[TEI.watermark]{watermark}\par 
    \item[namesdates: ]
   \hyperref[TEI.addName]{addName} \hyperref[TEI.affiliation]{affiliation} \hyperref[TEI.country]{country} \hyperref[TEI.forename]{forename} \hyperref[TEI.genName]{genName} \hyperref[TEI.geogName]{geogName} \hyperref[TEI.nameLink]{nameLink} \hyperref[TEI.orgName]{orgName} \hyperref[TEI.persName]{persName} \hyperref[TEI.placeName]{placeName} \hyperref[TEI.region]{region} \hyperref[TEI.roleName]{roleName} \hyperref[TEI.settlement]{settlement} \hyperref[TEI.surname]{surname}\par 
    \item[textstructure: ]
   \hyperref[TEI.docAuthor]{docAuthor} \hyperref[TEI.docDate]{docDate} \hyperref[TEI.docEdition]{docEdition} \hyperref[TEI.titlePart]{titlePart}\par 
    \item[transcr: ]
   \hyperref[TEI.am]{am} \hyperref[TEI.damage]{damage} \hyperref[TEI.fw]{fw} \hyperref[TEI.metamark]{metamark} \hyperref[TEI.mod]{mod} \hyperref[TEI.restore]{restore} \hyperref[TEI.retrace]{retrace} \hyperref[TEI.secl]{secl} \hyperref[TEI.supplied]{supplied} \hyperref[TEI.surplus]{surplus}
    \item[{Peut contenir}]
  
    \item[analysis: ]
   \hyperref[TEI.c]{c} \hyperref[TEI.cl]{cl} \hyperref[TEI.interp]{interp} \hyperref[TEI.interpGrp]{interpGrp} \hyperref[TEI.m]{m} \hyperref[TEI.pc]{pc} \hyperref[TEI.phr]{phr} \hyperref[TEI.s]{s} \hyperref[TEI.span]{span} \hyperref[TEI.spanGrp]{spanGrp} \hyperref[TEI.w]{w}\par 
    \item[core: ]
   \hyperref[TEI.abbr]{abbr} \hyperref[TEI.add]{add} \hyperref[TEI.address]{address} \hyperref[TEI.bibl]{bibl} \hyperref[TEI.biblStruct]{biblStruct} \hyperref[TEI.binaryObject]{binaryObject} \hyperref[TEI.cb]{cb} \hyperref[TEI.choice]{choice} \hyperref[TEI.cit]{cit} \hyperref[TEI.corr]{corr} \hyperref[TEI.date]{date} \hyperref[TEI.del]{del} \hyperref[TEI.desc]{desc} \hyperref[TEI.distinct]{distinct} \hyperref[TEI.email]{email} \hyperref[TEI.emph]{emph} \hyperref[TEI.expan]{expan} \hyperref[TEI.foreign]{foreign} \hyperref[TEI.gap]{gap} \hyperref[TEI.gb]{gb} \hyperref[TEI.gloss]{gloss} \hyperref[TEI.graphic]{graphic} \hyperref[TEI.hi]{hi} \hyperref[TEI.index]{index} \hyperref[TEI.l]{l} \hyperref[TEI.label]{label} \hyperref[TEI.lb]{lb} \hyperref[TEI.lg]{lg} \hyperref[TEI.list]{list} \hyperref[TEI.listBibl]{listBibl} \hyperref[TEI.measure]{measure} \hyperref[TEI.measureGrp]{measureGrp} \hyperref[TEI.media]{media} \hyperref[TEI.mentioned]{mentioned} \hyperref[TEI.milestone]{milestone} \hyperref[TEI.name]{name} \hyperref[TEI.note]{note} \hyperref[TEI.num]{num} \hyperref[TEI.orig]{orig} \hyperref[TEI.pb]{pb} \hyperref[TEI.ptr]{ptr} \hyperref[TEI.q]{q} \hyperref[TEI.quote]{quote} \hyperref[TEI.ref]{ref} \hyperref[TEI.reg]{reg} \hyperref[TEI.rs]{rs} \hyperref[TEI.said]{said} \hyperref[TEI.sic]{sic} \hyperref[TEI.soCalled]{soCalled} \hyperref[TEI.stage]{stage} \hyperref[TEI.term]{term} \hyperref[TEI.time]{time} \hyperref[TEI.title]{title} \hyperref[TEI.unclear]{unclear}\par 
    \item[derived-module-tei.istex: ]
   \hyperref[TEI.math]{math} \hyperref[TEI.mrow]{mrow}\par 
    \item[figures: ]
   \hyperref[TEI.figure]{figure} \hyperref[TEI.formula]{formula} \hyperref[TEI.notatedMusic]{notatedMusic} \hyperref[TEI.table]{table}\par 
    \item[header: ]
   \hyperref[TEI.biblFull]{biblFull} \hyperref[TEI.idno]{idno}\par 
    \item[iso-fs: ]
   \hyperref[TEI.fLib]{fLib} \hyperref[TEI.fs]{fs} \hyperref[TEI.fvLib]{fvLib}\par 
    \item[linking: ]
   \hyperref[TEI.alt]{alt} \hyperref[TEI.altGrp]{altGrp} \hyperref[TEI.anchor]{anchor} \hyperref[TEI.join]{join} \hyperref[TEI.joinGrp]{joinGrp} \hyperref[TEI.link]{link} \hyperref[TEI.linkGrp]{linkGrp} \hyperref[TEI.seg]{seg} \hyperref[TEI.timeline]{timeline}\par 
    \item[msdescription: ]
   \hyperref[TEI.catchwords]{catchwords} \hyperref[TEI.depth]{depth} \hyperref[TEI.dim]{dim} \hyperref[TEI.dimensions]{dimensions} \hyperref[TEI.height]{height} \hyperref[TEI.heraldry]{heraldry} \hyperref[TEI.locus]{locus} \hyperref[TEI.locusGrp]{locusGrp} \hyperref[TEI.material]{material} \hyperref[TEI.msDesc]{msDesc} \hyperref[TEI.objectType]{objectType} \hyperref[TEI.origDate]{origDate} \hyperref[TEI.origPlace]{origPlace} \hyperref[TEI.secFol]{secFol} \hyperref[TEI.signatures]{signatures} \hyperref[TEI.source]{source} \hyperref[TEI.stamp]{stamp} \hyperref[TEI.watermark]{watermark} \hyperref[TEI.width]{width}\par 
    \item[namesdates: ]
   \hyperref[TEI.addName]{addName} \hyperref[TEI.affiliation]{affiliation} \hyperref[TEI.country]{country} \hyperref[TEI.forename]{forename} \hyperref[TEI.genName]{genName} \hyperref[TEI.geogName]{geogName} \hyperref[TEI.listOrg]{listOrg} \hyperref[TEI.listPlace]{listPlace} \hyperref[TEI.location]{location} \hyperref[TEI.nameLink]{nameLink} \hyperref[TEI.orgName]{orgName} \hyperref[TEI.persName]{persName} \hyperref[TEI.placeName]{placeName} \hyperref[TEI.region]{region} \hyperref[TEI.roleName]{roleName} \hyperref[TEI.settlement]{settlement} \hyperref[TEI.state]{state} \hyperref[TEI.surname]{surname}\par 
    \item[spoken: ]
   \hyperref[TEI.annotationBlock]{annotationBlock}\par 
    \item[textstructure: ]
   \hyperref[TEI.floatingText]{floatingText}\par 
    \item[transcr: ]
   \hyperref[TEI.addSpan]{addSpan} \hyperref[TEI.am]{am} \hyperref[TEI.damage]{damage} \hyperref[TEI.damageSpan]{damageSpan} \hyperref[TEI.delSpan]{delSpan} \hyperref[TEI.ex]{ex} \hyperref[TEI.fw]{fw} \hyperref[TEI.handShift]{handShift} \hyperref[TEI.listTranspose]{listTranspose} \hyperref[TEI.metamark]{metamark} \hyperref[TEI.mod]{mod} \hyperref[TEI.redo]{redo} \hyperref[TEI.restore]{restore} \hyperref[TEI.retrace]{retrace} \hyperref[TEI.secl]{secl} \hyperref[TEI.space]{space} \hyperref[TEI.subst]{subst} \hyperref[TEI.substJoin]{substJoin} \hyperref[TEI.supplied]{supplied} \hyperref[TEI.surplus]{surplus} \hyperref[TEI.undo]{undo}\par des données textuelles
    \item[{Exemple}]
  \leavevmode\bgroup\exampleFont \begin{shaded}\noindent\mbox{}Des nuages, des\mbox{}\newline 
{<\textbf{sic}>}cyrrhus{</\textbf{sic}>}, des nimbus, des cumulus, tant qu'on en veut, et assurément plus que\mbox{}\newline 
 n'en voulaient le maître et le serviteur.\end{shaded}\egroup 


    \item[{Exemple}]
  Si on veut seulement attirer l'attention sur ce qui paraît être un problème dans la copie du texte, \hyperref[TEI.sic]{<sic>} est utilisé seul :\leavevmode\bgroup\exampleFont \begin{shaded}\noindent\mbox{}Tel est le\mbox{}\newline 
 chat Rutterkin des sorcières Margaret et Filippa Flower, qui furent\mbox{}\newline 
{<\textbf{sic}>}prûlées{</\textbf{sic}>}brûlées à Lincoln, le 11 mars 1619, pour avoir envoûté un parent du comte\mbox{}\newline 
 de Rutland.\end{shaded}\egroup 


    \item[{Exemple}]
  Il est également possible, en utilisant les éléments \hyperref[TEI.choice]{<choice>} et \hyperref[TEI.corr]{<corr>}, de proposer une lecture corrigée :\leavevmode\bgroup\exampleFont \begin{shaded}\noindent\mbox{}Tel est le\mbox{}\newline 
 chat Rutterkin des sorcières Margaret et Filippa Flower, qui furent{<\textbf{choice}>}\mbox{}\newline 
\hspace*{6pt}{<\textbf{sic}>}prûlées{</\textbf{sic}>}\mbox{}\newline 
\hspace*{6pt}{<\textbf{corr}>}brûlées{</\textbf{corr}>}\mbox{}\newline 
{</\textbf{choice}>} à Lincoln, le 11 mars 1619, pour avoir envoûté un parent du comte de\mbox{}\newline 
 Rutland.\end{shaded}\egroup 


    \item[{Exemple}]
  \leavevmode\bgroup\exampleFont \begin{shaded}\noindent\mbox{}Ouvrage très\mbox{}\newline 
 véridique et mirifique du Sieur Marcus Publius Dataficus du digne fils du seigneur comte,\mbox{}\newline 
 vicomte, duc et archiduc Johannus de Bessinguya{<\textbf{choice}>}\mbox{}\newline 
\hspace*{6pt}{<\textbf{sic}>} Percepteur{</\textbf{sic}>}\mbox{}\newline 
\hspace*{6pt}{<\textbf{corr}>}Precepteur{</\textbf{corr}>}\mbox{}\newline 
{</\textbf{choice}>} du digne fils du seigneur comte, vicomte, duc et archiduc Johannus de\mbox{}\newline 
 Bessinguya.\end{shaded}\egroup 


    \item[{Modèle de contenu}]
  \mbox{}\hfill\\[-10pt]\begin{Verbatim}[fontsize=\small]
<content>
 <macroRef key="macro.paraContent"/>
</content>
    
\end{Verbatim}

    \item[{Schéma Declaration}]
  \mbox{}\hfill\\[-10pt]\begin{Verbatim}[fontsize=\small]
element sic { tei_att.global.attributes, tei_macro.paraContent }
\end{Verbatim}

\end{reflist}  \index{signatures=<signatures>|oddindex}
\begin{reflist}
\item[]\begin{specHead}{TEI.signatures}{<signatures> }(signatures) Contient une étude des signatures trouvées sur un feuillet ou sur un cahier dans un manuscrit. [\xref{http://www.tei-c.org/release/doc/tei-p5-doc/en/html/MS.html\#msmisc}{10.3.7. Catchwords, Signatures, Secundo Folio}]\end{specHead} 
    \item[{Module}]
  msdescription
    \item[{Attributs}]
  Attributs \hyperref[TEI.att.global]{att.global} (\textit{@xml:id}, \textit{@n}, \textit{@xml:lang}, \textit{@xml:base}, \textit{@xml:space})  (\hyperref[TEI.att.global.rendition]{att.global.rendition} (\textit{@rend}, \textit{@style}, \textit{@rendition})) (\hyperref[TEI.att.global.linking]{att.global.linking} (\textit{@corresp}, \textit{@synch}, \textit{@sameAs}, \textit{@copyOf}, \textit{@next}, \textit{@prev}, \textit{@exclude}, \textit{@select})) (\hyperref[TEI.att.global.analytic]{att.global.analytic} (\textit{@ana})) (\hyperref[TEI.att.global.facs]{att.global.facs} (\textit{@facs})) (\hyperref[TEI.att.global.change]{att.global.change} (\textit{@change})) (\hyperref[TEI.att.global.responsibility]{att.global.responsibility} (\textit{@cert}, \textit{@resp})) (\hyperref[TEI.att.global.source]{att.global.source} (\textit{@source}))
    \item[{Membre du}]
  \hyperref[TEI.model.pPart.msdesc]{model.pPart.msdesc}
    \item[{Contenu dans}]
  
    \item[analysis: ]
   \hyperref[TEI.cl]{cl} \hyperref[TEI.phr]{phr} \hyperref[TEI.s]{s} \hyperref[TEI.span]{span}\par 
    \item[core: ]
   \hyperref[TEI.abbr]{abbr} \hyperref[TEI.add]{add} \hyperref[TEI.addrLine]{addrLine} \hyperref[TEI.author]{author} \hyperref[TEI.biblScope]{biblScope} \hyperref[TEI.citedRange]{citedRange} \hyperref[TEI.corr]{corr} \hyperref[TEI.date]{date} \hyperref[TEI.del]{del} \hyperref[TEI.desc]{desc} \hyperref[TEI.distinct]{distinct} \hyperref[TEI.editor]{editor} \hyperref[TEI.email]{email} \hyperref[TEI.emph]{emph} \hyperref[TEI.expan]{expan} \hyperref[TEI.foreign]{foreign} \hyperref[TEI.gloss]{gloss} \hyperref[TEI.head]{head} \hyperref[TEI.headItem]{headItem} \hyperref[TEI.headLabel]{headLabel} \hyperref[TEI.hi]{hi} \hyperref[TEI.item]{item} \hyperref[TEI.l]{l} \hyperref[TEI.label]{label} \hyperref[TEI.measure]{measure} \hyperref[TEI.meeting]{meeting} \hyperref[TEI.mentioned]{mentioned} \hyperref[TEI.name]{name} \hyperref[TEI.note]{note} \hyperref[TEI.num]{num} \hyperref[TEI.orig]{orig} \hyperref[TEI.p]{p} \hyperref[TEI.pubPlace]{pubPlace} \hyperref[TEI.publisher]{publisher} \hyperref[TEI.q]{q} \hyperref[TEI.quote]{quote} \hyperref[TEI.ref]{ref} \hyperref[TEI.reg]{reg} \hyperref[TEI.resp]{resp} \hyperref[TEI.rs]{rs} \hyperref[TEI.said]{said} \hyperref[TEI.sic]{sic} \hyperref[TEI.soCalled]{soCalled} \hyperref[TEI.speaker]{speaker} \hyperref[TEI.stage]{stage} \hyperref[TEI.street]{street} \hyperref[TEI.term]{term} \hyperref[TEI.textLang]{textLang} \hyperref[TEI.time]{time} \hyperref[TEI.title]{title} \hyperref[TEI.unclear]{unclear}\par 
    \item[figures: ]
   \hyperref[TEI.cell]{cell} \hyperref[TEI.figDesc]{figDesc}\par 
    \item[header: ]
   \hyperref[TEI.authority]{authority} \hyperref[TEI.change]{change} \hyperref[TEI.classCode]{classCode} \hyperref[TEI.creation]{creation} \hyperref[TEI.distributor]{distributor} \hyperref[TEI.edition]{edition} \hyperref[TEI.extent]{extent} \hyperref[TEI.funder]{funder} \hyperref[TEI.language]{language} \hyperref[TEI.licence]{licence} \hyperref[TEI.rendition]{rendition}\par 
    \item[iso-fs: ]
   \hyperref[TEI.fDescr]{fDescr} \hyperref[TEI.fsDescr]{fsDescr}\par 
    \item[linking: ]
   \hyperref[TEI.ab]{ab} \hyperref[TEI.seg]{seg}\par 
    \item[msdescription: ]
   \hyperref[TEI.accMat]{accMat} \hyperref[TEI.acquisition]{acquisition} \hyperref[TEI.additions]{additions} \hyperref[TEI.catchwords]{catchwords} \hyperref[TEI.collation]{collation} \hyperref[TEI.colophon]{colophon} \hyperref[TEI.condition]{condition} \hyperref[TEI.custEvent]{custEvent} \hyperref[TEI.decoNote]{decoNote} \hyperref[TEI.explicit]{explicit} \hyperref[TEI.filiation]{filiation} \hyperref[TEI.finalRubric]{finalRubric} \hyperref[TEI.foliation]{foliation} \hyperref[TEI.heraldry]{heraldry} \hyperref[TEI.incipit]{incipit} \hyperref[TEI.layout]{layout} \hyperref[TEI.material]{material} \hyperref[TEI.musicNotation]{musicNotation} \hyperref[TEI.objectType]{objectType} \hyperref[TEI.origDate]{origDate} \hyperref[TEI.origPlace]{origPlace} \hyperref[TEI.origin]{origin} \hyperref[TEI.provenance]{provenance} \hyperref[TEI.rubric]{rubric} \hyperref[TEI.secFol]{secFol} \hyperref[TEI.signatures]{signatures} \hyperref[TEI.source]{source} \hyperref[TEI.stamp]{stamp} \hyperref[TEI.summary]{summary} \hyperref[TEI.support]{support} \hyperref[TEI.surrogates]{surrogates} \hyperref[TEI.typeNote]{typeNote} \hyperref[TEI.watermark]{watermark}\par 
    \item[namesdates: ]
   \hyperref[TEI.addName]{addName} \hyperref[TEI.affiliation]{affiliation} \hyperref[TEI.country]{country} \hyperref[TEI.forename]{forename} \hyperref[TEI.genName]{genName} \hyperref[TEI.geogName]{geogName} \hyperref[TEI.nameLink]{nameLink} \hyperref[TEI.orgName]{orgName} \hyperref[TEI.persName]{persName} \hyperref[TEI.placeName]{placeName} \hyperref[TEI.region]{region} \hyperref[TEI.roleName]{roleName} \hyperref[TEI.settlement]{settlement} \hyperref[TEI.surname]{surname}\par 
    \item[textstructure: ]
   \hyperref[TEI.docAuthor]{docAuthor} \hyperref[TEI.docDate]{docDate} \hyperref[TEI.docEdition]{docEdition} \hyperref[TEI.titlePart]{titlePart}\par 
    \item[transcr: ]
   \hyperref[TEI.damage]{damage} \hyperref[TEI.fw]{fw} \hyperref[TEI.metamark]{metamark} \hyperref[TEI.mod]{mod} \hyperref[TEI.restore]{restore} \hyperref[TEI.retrace]{retrace} \hyperref[TEI.secl]{secl} \hyperref[TEI.supplied]{supplied} \hyperref[TEI.surplus]{surplus}
    \item[{Peut contenir}]
  
    \item[analysis: ]
   \hyperref[TEI.c]{c} \hyperref[TEI.cl]{cl} \hyperref[TEI.interp]{interp} \hyperref[TEI.interpGrp]{interpGrp} \hyperref[TEI.m]{m} \hyperref[TEI.pc]{pc} \hyperref[TEI.phr]{phr} \hyperref[TEI.s]{s} \hyperref[TEI.span]{span} \hyperref[TEI.spanGrp]{spanGrp} \hyperref[TEI.w]{w}\par 
    \item[core: ]
   \hyperref[TEI.abbr]{abbr} \hyperref[TEI.add]{add} \hyperref[TEI.address]{address} \hyperref[TEI.bibl]{bibl} \hyperref[TEI.biblStruct]{biblStruct} \hyperref[TEI.binaryObject]{binaryObject} \hyperref[TEI.cb]{cb} \hyperref[TEI.choice]{choice} \hyperref[TEI.cit]{cit} \hyperref[TEI.corr]{corr} \hyperref[TEI.date]{date} \hyperref[TEI.del]{del} \hyperref[TEI.desc]{desc} \hyperref[TEI.distinct]{distinct} \hyperref[TEI.email]{email} \hyperref[TEI.emph]{emph} \hyperref[TEI.expan]{expan} \hyperref[TEI.foreign]{foreign} \hyperref[TEI.gap]{gap} \hyperref[TEI.gb]{gb} \hyperref[TEI.gloss]{gloss} \hyperref[TEI.graphic]{graphic} \hyperref[TEI.hi]{hi} \hyperref[TEI.index]{index} \hyperref[TEI.l]{l} \hyperref[TEI.label]{label} \hyperref[TEI.lb]{lb} \hyperref[TEI.lg]{lg} \hyperref[TEI.list]{list} \hyperref[TEI.listBibl]{listBibl} \hyperref[TEI.measure]{measure} \hyperref[TEI.measureGrp]{measureGrp} \hyperref[TEI.media]{media} \hyperref[TEI.mentioned]{mentioned} \hyperref[TEI.milestone]{milestone} \hyperref[TEI.name]{name} \hyperref[TEI.note]{note} \hyperref[TEI.num]{num} \hyperref[TEI.orig]{orig} \hyperref[TEI.p]{p} \hyperref[TEI.pb]{pb} \hyperref[TEI.ptr]{ptr} \hyperref[TEI.q]{q} \hyperref[TEI.quote]{quote} \hyperref[TEI.ref]{ref} \hyperref[TEI.reg]{reg} \hyperref[TEI.rs]{rs} \hyperref[TEI.said]{said} \hyperref[TEI.sic]{sic} \hyperref[TEI.soCalled]{soCalled} \hyperref[TEI.sp]{sp} \hyperref[TEI.stage]{stage} \hyperref[TEI.term]{term} \hyperref[TEI.time]{time} \hyperref[TEI.title]{title} \hyperref[TEI.unclear]{unclear}\par 
    \item[derived-module-tei.istex: ]
   \hyperref[TEI.math]{math} \hyperref[TEI.mrow]{mrow}\par 
    \item[figures: ]
   \hyperref[TEI.figure]{figure} \hyperref[TEI.formula]{formula} \hyperref[TEI.notatedMusic]{notatedMusic} \hyperref[TEI.table]{table}\par 
    \item[header: ]
   \hyperref[TEI.biblFull]{biblFull} \hyperref[TEI.idno]{idno}\par 
    \item[iso-fs: ]
   \hyperref[TEI.fLib]{fLib} \hyperref[TEI.fs]{fs} \hyperref[TEI.fvLib]{fvLib}\par 
    \item[linking: ]
   \hyperref[TEI.ab]{ab} \hyperref[TEI.alt]{alt} \hyperref[TEI.altGrp]{altGrp} \hyperref[TEI.anchor]{anchor} \hyperref[TEI.join]{join} \hyperref[TEI.joinGrp]{joinGrp} \hyperref[TEI.link]{link} \hyperref[TEI.linkGrp]{linkGrp} \hyperref[TEI.seg]{seg} \hyperref[TEI.timeline]{timeline}\par 
    \item[msdescription: ]
   \hyperref[TEI.catchwords]{catchwords} \hyperref[TEI.depth]{depth} \hyperref[TEI.dim]{dim} \hyperref[TEI.dimensions]{dimensions} \hyperref[TEI.height]{height} \hyperref[TEI.heraldry]{heraldry} \hyperref[TEI.locus]{locus} \hyperref[TEI.locusGrp]{locusGrp} \hyperref[TEI.material]{material} \hyperref[TEI.msDesc]{msDesc} \hyperref[TEI.objectType]{objectType} \hyperref[TEI.origDate]{origDate} \hyperref[TEI.origPlace]{origPlace} \hyperref[TEI.secFol]{secFol} \hyperref[TEI.signatures]{signatures} \hyperref[TEI.source]{source} \hyperref[TEI.stamp]{stamp} \hyperref[TEI.watermark]{watermark} \hyperref[TEI.width]{width}\par 
    \item[namesdates: ]
   \hyperref[TEI.addName]{addName} \hyperref[TEI.affiliation]{affiliation} \hyperref[TEI.country]{country} \hyperref[TEI.forename]{forename} \hyperref[TEI.genName]{genName} \hyperref[TEI.geogName]{geogName} \hyperref[TEI.listOrg]{listOrg} \hyperref[TEI.listPlace]{listPlace} \hyperref[TEI.location]{location} \hyperref[TEI.nameLink]{nameLink} \hyperref[TEI.orgName]{orgName} \hyperref[TEI.persName]{persName} \hyperref[TEI.placeName]{placeName} \hyperref[TEI.region]{region} \hyperref[TEI.roleName]{roleName} \hyperref[TEI.settlement]{settlement} \hyperref[TEI.state]{state} \hyperref[TEI.surname]{surname}\par 
    \item[spoken: ]
   \hyperref[TEI.annotationBlock]{annotationBlock}\par 
    \item[textstructure: ]
   \hyperref[TEI.floatingText]{floatingText}\par 
    \item[transcr: ]
   \hyperref[TEI.addSpan]{addSpan} \hyperref[TEI.am]{am} \hyperref[TEI.damage]{damage} \hyperref[TEI.damageSpan]{damageSpan} \hyperref[TEI.delSpan]{delSpan} \hyperref[TEI.ex]{ex} \hyperref[TEI.fw]{fw} \hyperref[TEI.handShift]{handShift} \hyperref[TEI.listTranspose]{listTranspose} \hyperref[TEI.metamark]{metamark} \hyperref[TEI.mod]{mod} \hyperref[TEI.redo]{redo} \hyperref[TEI.restore]{restore} \hyperref[TEI.retrace]{retrace} \hyperref[TEI.secl]{secl} \hyperref[TEI.space]{space} \hyperref[TEI.subst]{subst} \hyperref[TEI.substJoin]{substJoin} \hyperref[TEI.supplied]{supplied} \hyperref[TEI.surplus]{surplus} \hyperref[TEI.undo]{undo}\par des données textuelles
    \item[{Exemple}]
  \leavevmode\bgroup\exampleFont \begin{shaded}\noindent\mbox{}{<\textbf{signatures}>}Quire and leaf signatures in letters, [b]-v, and roman numerals; those in quires\mbox{}\newline 
 10 (1) and 17 (s) in red ink and different from others; every third quire also signed with\mbox{}\newline 
 red crayon in arabic numerals in the center lower margin of the first leaf recto: "2" for\mbox{}\newline 
 quire 4 (f. 19), "3" for quire 7 (f. 43); "4," barely visible, for quire 10 (f. 65), "5,"\mbox{}\newline 
 in a later hand, for quire 13 (f. 89), "6," in a later hand, for quire 16 (f.\mbox{}\newline 
 113).{</\textbf{signatures}>}\end{shaded}\egroup 


    \item[{Schematron}]
   <sch:assert role="nonfatal"  test="ancestor::tei:msDesc">WARNING: deprecated use of element — The <sch:name/> element will not be allowed outside of msDesc as of 2018-10-01.</sch:assert>
    \item[{Modèle de contenu}]
  \mbox{}\hfill\\[-10pt]\begin{Verbatim}[fontsize=\small]
<content>
 <macroRef key="macro.specialPara"/>
</content>
    
\end{Verbatim}

    \item[{Schéma Declaration}]
  \mbox{}\hfill\\[-10pt]\begin{Verbatim}[fontsize=\small]
element signatures { tei_att.global.attributes, tei_macro.specialPara }
\end{Verbatim}

\end{reflist}  \index{soCalled=<soCalled>|oddindex}
\begin{reflist}
\item[]\begin{specHead}{TEI.soCalled}{<soCalled> }contient une expression ou un mot pour lesquels l'auteur ou le narrateur renonce à toute responsabilité, par exemple en utilisant de l'italique ou des guillemets. [\xref{http://www.tei-c.org/release/doc/tei-p5-doc/en/html/CO.html\#COHQQ}{3.3.3. Quotation}]\end{specHead} 
    \item[{Module}]
  core
    \item[{Attributs}]
  Attributs \hyperref[TEI.att.global]{att.global} (\textit{@xml:id}, \textit{@n}, \textit{@xml:lang}, \textit{@xml:base}, \textit{@xml:space})  (\hyperref[TEI.att.global.rendition]{att.global.rendition} (\textit{@rend}, \textit{@style}, \textit{@rendition})) (\hyperref[TEI.att.global.linking]{att.global.linking} (\textit{@corresp}, \textit{@synch}, \textit{@sameAs}, \textit{@copyOf}, \textit{@next}, \textit{@prev}, \textit{@exclude}, \textit{@select})) (\hyperref[TEI.att.global.analytic]{att.global.analytic} (\textit{@ana})) (\hyperref[TEI.att.global.facs]{att.global.facs} (\textit{@facs})) (\hyperref[TEI.att.global.change]{att.global.change} (\textit{@change})) (\hyperref[TEI.att.global.responsibility]{att.global.responsibility} (\textit{@cert}, \textit{@resp})) (\hyperref[TEI.att.global.source]{att.global.source} (\textit{@source}))
    \item[{Membre du}]
  \hyperref[TEI.model.emphLike]{model.emphLike}
    \item[{Contenu dans}]
  
    \item[analysis: ]
   \hyperref[TEI.cl]{cl} \hyperref[TEI.phr]{phr} \hyperref[TEI.s]{s} \hyperref[TEI.span]{span}\par 
    \item[core: ]
   \hyperref[TEI.abbr]{abbr} \hyperref[TEI.add]{add} \hyperref[TEI.addrLine]{addrLine} \hyperref[TEI.author]{author} \hyperref[TEI.bibl]{bibl} \hyperref[TEI.biblScope]{biblScope} \hyperref[TEI.citedRange]{citedRange} \hyperref[TEI.corr]{corr} \hyperref[TEI.date]{date} \hyperref[TEI.del]{del} \hyperref[TEI.desc]{desc} \hyperref[TEI.distinct]{distinct} \hyperref[TEI.editor]{editor} \hyperref[TEI.email]{email} \hyperref[TEI.emph]{emph} \hyperref[TEI.expan]{expan} \hyperref[TEI.foreign]{foreign} \hyperref[TEI.gloss]{gloss} \hyperref[TEI.head]{head} \hyperref[TEI.headItem]{headItem} \hyperref[TEI.headLabel]{headLabel} \hyperref[TEI.hi]{hi} \hyperref[TEI.item]{item} \hyperref[TEI.l]{l} \hyperref[TEI.label]{label} \hyperref[TEI.measure]{measure} \hyperref[TEI.meeting]{meeting} \hyperref[TEI.mentioned]{mentioned} \hyperref[TEI.name]{name} \hyperref[TEI.note]{note} \hyperref[TEI.num]{num} \hyperref[TEI.orig]{orig} \hyperref[TEI.p]{p} \hyperref[TEI.pubPlace]{pubPlace} \hyperref[TEI.publisher]{publisher} \hyperref[TEI.q]{q} \hyperref[TEI.quote]{quote} \hyperref[TEI.ref]{ref} \hyperref[TEI.reg]{reg} \hyperref[TEI.resp]{resp} \hyperref[TEI.rs]{rs} \hyperref[TEI.said]{said} \hyperref[TEI.sic]{sic} \hyperref[TEI.soCalled]{soCalled} \hyperref[TEI.speaker]{speaker} \hyperref[TEI.stage]{stage} \hyperref[TEI.street]{street} \hyperref[TEI.term]{term} \hyperref[TEI.textLang]{textLang} \hyperref[TEI.time]{time} \hyperref[TEI.title]{title} \hyperref[TEI.unclear]{unclear}\par 
    \item[figures: ]
   \hyperref[TEI.cell]{cell} \hyperref[TEI.figDesc]{figDesc}\par 
    \item[header: ]
   \hyperref[TEI.authority]{authority} \hyperref[TEI.change]{change} \hyperref[TEI.classCode]{classCode} \hyperref[TEI.creation]{creation} \hyperref[TEI.distributor]{distributor} \hyperref[TEI.edition]{edition} \hyperref[TEI.extent]{extent} \hyperref[TEI.funder]{funder} \hyperref[TEI.language]{language} \hyperref[TEI.licence]{licence} \hyperref[TEI.rendition]{rendition}\par 
    \item[iso-fs: ]
   \hyperref[TEI.fDescr]{fDescr} \hyperref[TEI.fsDescr]{fsDescr}\par 
    \item[linking: ]
   \hyperref[TEI.ab]{ab} \hyperref[TEI.seg]{seg}\par 
    \item[msdescription: ]
   \hyperref[TEI.accMat]{accMat} \hyperref[TEI.acquisition]{acquisition} \hyperref[TEI.additions]{additions} \hyperref[TEI.catchwords]{catchwords} \hyperref[TEI.collation]{collation} \hyperref[TEI.colophon]{colophon} \hyperref[TEI.condition]{condition} \hyperref[TEI.custEvent]{custEvent} \hyperref[TEI.decoNote]{decoNote} \hyperref[TEI.explicit]{explicit} \hyperref[TEI.filiation]{filiation} \hyperref[TEI.finalRubric]{finalRubric} \hyperref[TEI.foliation]{foliation} \hyperref[TEI.heraldry]{heraldry} \hyperref[TEI.incipit]{incipit} \hyperref[TEI.layout]{layout} \hyperref[TEI.material]{material} \hyperref[TEI.musicNotation]{musicNotation} \hyperref[TEI.objectType]{objectType} \hyperref[TEI.origDate]{origDate} \hyperref[TEI.origPlace]{origPlace} \hyperref[TEI.origin]{origin} \hyperref[TEI.provenance]{provenance} \hyperref[TEI.rubric]{rubric} \hyperref[TEI.secFol]{secFol} \hyperref[TEI.signatures]{signatures} \hyperref[TEI.source]{source} \hyperref[TEI.stamp]{stamp} \hyperref[TEI.summary]{summary} \hyperref[TEI.support]{support} \hyperref[TEI.surrogates]{surrogates} \hyperref[TEI.typeNote]{typeNote} \hyperref[TEI.watermark]{watermark}\par 
    \item[namesdates: ]
   \hyperref[TEI.addName]{addName} \hyperref[TEI.affiliation]{affiliation} \hyperref[TEI.country]{country} \hyperref[TEI.forename]{forename} \hyperref[TEI.genName]{genName} \hyperref[TEI.geogName]{geogName} \hyperref[TEI.nameLink]{nameLink} \hyperref[TEI.orgName]{orgName} \hyperref[TEI.persName]{persName} \hyperref[TEI.placeName]{placeName} \hyperref[TEI.region]{region} \hyperref[TEI.roleName]{roleName} \hyperref[TEI.settlement]{settlement} \hyperref[TEI.surname]{surname}\par 
    \item[textstructure: ]
   \hyperref[TEI.docAuthor]{docAuthor} \hyperref[TEI.docDate]{docDate} \hyperref[TEI.docEdition]{docEdition} \hyperref[TEI.titlePart]{titlePart}\par 
    \item[transcr: ]
   \hyperref[TEI.damage]{damage} \hyperref[TEI.fw]{fw} \hyperref[TEI.metamark]{metamark} \hyperref[TEI.mod]{mod} \hyperref[TEI.restore]{restore} \hyperref[TEI.retrace]{retrace} \hyperref[TEI.secl]{secl} \hyperref[TEI.supplied]{supplied} \hyperref[TEI.surplus]{surplus}
    \item[{Peut contenir}]
  
    \item[analysis: ]
   \hyperref[TEI.c]{c} \hyperref[TEI.cl]{cl} \hyperref[TEI.interp]{interp} \hyperref[TEI.interpGrp]{interpGrp} \hyperref[TEI.m]{m} \hyperref[TEI.pc]{pc} \hyperref[TEI.phr]{phr} \hyperref[TEI.s]{s} \hyperref[TEI.span]{span} \hyperref[TEI.spanGrp]{spanGrp} \hyperref[TEI.w]{w}\par 
    \item[core: ]
   \hyperref[TEI.abbr]{abbr} \hyperref[TEI.add]{add} \hyperref[TEI.address]{address} \hyperref[TEI.binaryObject]{binaryObject} \hyperref[TEI.cb]{cb} \hyperref[TEI.choice]{choice} \hyperref[TEI.corr]{corr} \hyperref[TEI.date]{date} \hyperref[TEI.del]{del} \hyperref[TEI.distinct]{distinct} \hyperref[TEI.email]{email} \hyperref[TEI.emph]{emph} \hyperref[TEI.expan]{expan} \hyperref[TEI.foreign]{foreign} \hyperref[TEI.gap]{gap} \hyperref[TEI.gb]{gb} \hyperref[TEI.gloss]{gloss} \hyperref[TEI.graphic]{graphic} \hyperref[TEI.hi]{hi} \hyperref[TEI.index]{index} \hyperref[TEI.lb]{lb} \hyperref[TEI.measure]{measure} \hyperref[TEI.measureGrp]{measureGrp} \hyperref[TEI.media]{media} \hyperref[TEI.mentioned]{mentioned} \hyperref[TEI.milestone]{milestone} \hyperref[TEI.name]{name} \hyperref[TEI.note]{note} \hyperref[TEI.num]{num} \hyperref[TEI.orig]{orig} \hyperref[TEI.pb]{pb} \hyperref[TEI.ptr]{ptr} \hyperref[TEI.ref]{ref} \hyperref[TEI.reg]{reg} \hyperref[TEI.rs]{rs} \hyperref[TEI.sic]{sic} \hyperref[TEI.soCalled]{soCalled} \hyperref[TEI.term]{term} \hyperref[TEI.time]{time} \hyperref[TEI.title]{title} \hyperref[TEI.unclear]{unclear}\par 
    \item[derived-module-tei.istex: ]
   \hyperref[TEI.math]{math} \hyperref[TEI.mrow]{mrow}\par 
    \item[figures: ]
   \hyperref[TEI.figure]{figure} \hyperref[TEI.formula]{formula} \hyperref[TEI.notatedMusic]{notatedMusic}\par 
    \item[header: ]
   \hyperref[TEI.idno]{idno}\par 
    \item[iso-fs: ]
   \hyperref[TEI.fLib]{fLib} \hyperref[TEI.fs]{fs} \hyperref[TEI.fvLib]{fvLib}\par 
    \item[linking: ]
   \hyperref[TEI.alt]{alt} \hyperref[TEI.altGrp]{altGrp} \hyperref[TEI.anchor]{anchor} \hyperref[TEI.join]{join} \hyperref[TEI.joinGrp]{joinGrp} \hyperref[TEI.link]{link} \hyperref[TEI.linkGrp]{linkGrp} \hyperref[TEI.seg]{seg} \hyperref[TEI.timeline]{timeline}\par 
    \item[msdescription: ]
   \hyperref[TEI.catchwords]{catchwords} \hyperref[TEI.depth]{depth} \hyperref[TEI.dim]{dim} \hyperref[TEI.dimensions]{dimensions} \hyperref[TEI.height]{height} \hyperref[TEI.heraldry]{heraldry} \hyperref[TEI.locus]{locus} \hyperref[TEI.locusGrp]{locusGrp} \hyperref[TEI.material]{material} \hyperref[TEI.objectType]{objectType} \hyperref[TEI.origDate]{origDate} \hyperref[TEI.origPlace]{origPlace} \hyperref[TEI.secFol]{secFol} \hyperref[TEI.signatures]{signatures} \hyperref[TEI.source]{source} \hyperref[TEI.stamp]{stamp} \hyperref[TEI.watermark]{watermark} \hyperref[TEI.width]{width}\par 
    \item[namesdates: ]
   \hyperref[TEI.addName]{addName} \hyperref[TEI.affiliation]{affiliation} \hyperref[TEI.country]{country} \hyperref[TEI.forename]{forename} \hyperref[TEI.genName]{genName} \hyperref[TEI.geogName]{geogName} \hyperref[TEI.location]{location} \hyperref[TEI.nameLink]{nameLink} \hyperref[TEI.orgName]{orgName} \hyperref[TEI.persName]{persName} \hyperref[TEI.placeName]{placeName} \hyperref[TEI.region]{region} \hyperref[TEI.roleName]{roleName} \hyperref[TEI.settlement]{settlement} \hyperref[TEI.state]{state} \hyperref[TEI.surname]{surname}\par 
    \item[spoken: ]
   \hyperref[TEI.annotationBlock]{annotationBlock}\par 
    \item[transcr: ]
   \hyperref[TEI.addSpan]{addSpan} \hyperref[TEI.am]{am} \hyperref[TEI.damage]{damage} \hyperref[TEI.damageSpan]{damageSpan} \hyperref[TEI.delSpan]{delSpan} \hyperref[TEI.ex]{ex} \hyperref[TEI.fw]{fw} \hyperref[TEI.handShift]{handShift} \hyperref[TEI.listTranspose]{listTranspose} \hyperref[TEI.metamark]{metamark} \hyperref[TEI.mod]{mod} \hyperref[TEI.redo]{redo} \hyperref[TEI.restore]{restore} \hyperref[TEI.retrace]{retrace} \hyperref[TEI.secl]{secl} \hyperref[TEI.space]{space} \hyperref[TEI.subst]{subst} \hyperref[TEI.substJoin]{substJoin} \hyperref[TEI.supplied]{supplied} \hyperref[TEI.surplus]{surplus} \hyperref[TEI.undo]{undo}\par des données textuelles
    \item[{Exemple}]
  \leavevmode\bgroup\exampleFont \begin{shaded}\noindent\mbox{}- On ne\mbox{}\newline 
 bouge pas, on ne touche à rien, il faut que je prévienne {<\textbf{soCalled}>}la Maison{</\textbf{soCalled}>}.\mbox{}\newline 
 C'est ainsi qu'il appelait le Quai des Orfèvres. \end{shaded}\egroup 


    \item[{Exemple}]
  \leavevmode\bgroup\exampleFont \begin{shaded}\noindent\mbox{}{<\textbf{p}>} Mais, après tout, les propos auxquels on mêlait son nom n'étaient que des propos ; du\mbox{}\newline 
 bruit, des mots, des paroles, moins que des paroles, des{<\textbf{soCalled}>}palabres{</\textbf{soCalled}>},\mbox{}\newline 
 comme dit l'énergique langue du midi.{</\textbf{p}>}\end{shaded}\egroup 


    \item[{Modèle de contenu}]
  \mbox{}\hfill\\[-10pt]\begin{Verbatim}[fontsize=\small]
<content>
 <macroRef key="macro.phraseSeq"/>
</content>
    
\end{Verbatim}

    \item[{Schéma Declaration}]
  \mbox{}\hfill\\[-10pt]\begin{Verbatim}[fontsize=\small]
element soCalled { tei_att.global.attributes, tei_macro.phraseSeq }
\end{Verbatim}

\end{reflist}  \index{source=<source>|oddindex}
\begin{reflist}
\item[]\begin{specHead}{TEI.source}{<source> }(source) décrit la source des informations contenues dans la description du manuscrit. [\xref{http://www.tei-c.org/release/doc/tei-p5-doc/en/html/MS.html\#msrh}{10.9.1.1. Record History}]\end{specHead} 
    \item[{Module}]
  msdescription
    \item[{Attributs}]
  Attributs \hyperref[TEI.att.global]{att.global} (\textit{@xml:id}, \textit{@n}, \textit{@xml:lang}, \textit{@xml:base}, \textit{@xml:space})  (\hyperref[TEI.att.global.rendition]{att.global.rendition} (\textit{@rend}, \textit{@style}, \textit{@rendition})) (\hyperref[TEI.att.global.linking]{att.global.linking} (\textit{@corresp}, \textit{@synch}, \textit{@sameAs}, \textit{@copyOf}, \textit{@next}, \textit{@prev}, \textit{@exclude}, \textit{@select})) (\hyperref[TEI.att.global.analytic]{att.global.analytic} (\textit{@ana})) (\hyperref[TEI.att.global.facs]{att.global.facs} (\textit{@facs})) (\hyperref[TEI.att.global.change]{att.global.change} (\textit{@change})) (\hyperref[TEI.att.global.responsibility]{att.global.responsibility} (\textit{@cert}, \textit{@resp})) (\hyperref[TEI.att.global.source]{att.global.source} (\textit{@source}))
    \item[{Membre du}]
  \hyperref[TEI.model.global.meta]{model.global.meta} 
    \item[{Contenu dans}]
  
    \item[analysis: ]
   \hyperref[TEI.cl]{cl} \hyperref[TEI.m]{m} \hyperref[TEI.phr]{phr} \hyperref[TEI.s]{s} \hyperref[TEI.span]{span} \hyperref[TEI.w]{w}\par 
    \item[core: ]
   \hyperref[TEI.abbr]{abbr} \hyperref[TEI.add]{add} \hyperref[TEI.addrLine]{addrLine} \hyperref[TEI.address]{address} \hyperref[TEI.author]{author} \hyperref[TEI.bibl]{bibl} \hyperref[TEI.biblScope]{biblScope} \hyperref[TEI.cit]{cit} \hyperref[TEI.citedRange]{citedRange} \hyperref[TEI.corr]{corr} \hyperref[TEI.date]{date} \hyperref[TEI.del]{del} \hyperref[TEI.distinct]{distinct} \hyperref[TEI.editor]{editor} \hyperref[TEI.email]{email} \hyperref[TEI.emph]{emph} \hyperref[TEI.expan]{expan} \hyperref[TEI.foreign]{foreign} \hyperref[TEI.gloss]{gloss} \hyperref[TEI.head]{head} \hyperref[TEI.headItem]{headItem} \hyperref[TEI.headLabel]{headLabel} \hyperref[TEI.hi]{hi} \hyperref[TEI.imprint]{imprint} \hyperref[TEI.item]{item} \hyperref[TEI.l]{l} \hyperref[TEI.label]{label} \hyperref[TEI.lg]{lg} \hyperref[TEI.list]{list} \hyperref[TEI.measure]{measure} \hyperref[TEI.mentioned]{mentioned} \hyperref[TEI.name]{name} \hyperref[TEI.note]{note} \hyperref[TEI.num]{num} \hyperref[TEI.orig]{orig} \hyperref[TEI.p]{p} \hyperref[TEI.pubPlace]{pubPlace} \hyperref[TEI.publisher]{publisher} \hyperref[TEI.q]{q} \hyperref[TEI.quote]{quote} \hyperref[TEI.ref]{ref} \hyperref[TEI.reg]{reg} \hyperref[TEI.resp]{resp} \hyperref[TEI.rs]{rs} \hyperref[TEI.said]{said} \hyperref[TEI.series]{series} \hyperref[TEI.sic]{sic} \hyperref[TEI.soCalled]{soCalled} \hyperref[TEI.sp]{sp} \hyperref[TEI.speaker]{speaker} \hyperref[TEI.stage]{stage} \hyperref[TEI.street]{street} \hyperref[TEI.term]{term} \hyperref[TEI.textLang]{textLang} \hyperref[TEI.time]{time} \hyperref[TEI.title]{title} \hyperref[TEI.unclear]{unclear}\par 
    \item[figures: ]
   \hyperref[TEI.cell]{cell} \hyperref[TEI.figure]{figure} \hyperref[TEI.table]{table}\par 
    \item[header: ]
   \hyperref[TEI.authority]{authority} \hyperref[TEI.change]{change} \hyperref[TEI.classCode]{classCode} \hyperref[TEI.distributor]{distributor} \hyperref[TEI.edition]{edition} \hyperref[TEI.extent]{extent} \hyperref[TEI.funder]{funder} \hyperref[TEI.language]{language} \hyperref[TEI.licence]{licence}\par 
    \item[linking: ]
   \hyperref[TEI.ab]{ab} \hyperref[TEI.seg]{seg}\par 
    \item[msdescription: ]
   \hyperref[TEI.accMat]{accMat} \hyperref[TEI.acquisition]{acquisition} \hyperref[TEI.additions]{additions} \hyperref[TEI.catchwords]{catchwords} \hyperref[TEI.collation]{collation} \hyperref[TEI.colophon]{colophon} \hyperref[TEI.condition]{condition} \hyperref[TEI.custEvent]{custEvent} \hyperref[TEI.decoNote]{decoNote} \hyperref[TEI.explicit]{explicit} \hyperref[TEI.filiation]{filiation} \hyperref[TEI.finalRubric]{finalRubric} \hyperref[TEI.foliation]{foliation} \hyperref[TEI.heraldry]{heraldry} \hyperref[TEI.incipit]{incipit} \hyperref[TEI.layout]{layout} \hyperref[TEI.material]{material} \hyperref[TEI.msItem]{msItem} \hyperref[TEI.musicNotation]{musicNotation} \hyperref[TEI.objectType]{objectType} \hyperref[TEI.origDate]{origDate} \hyperref[TEI.origPlace]{origPlace} \hyperref[TEI.origin]{origin} \hyperref[TEI.provenance]{provenance} \hyperref[TEI.recordHist]{recordHist} \hyperref[TEI.rubric]{rubric} \hyperref[TEI.secFol]{secFol} \hyperref[TEI.signatures]{signatures} \hyperref[TEI.source]{source} \hyperref[TEI.stamp]{stamp} \hyperref[TEI.summary]{summary} \hyperref[TEI.support]{support} \hyperref[TEI.surrogates]{surrogates} \hyperref[TEI.typeNote]{typeNote} \hyperref[TEI.watermark]{watermark}\par 
    \item[namesdates: ]
   \hyperref[TEI.addName]{addName} \hyperref[TEI.affiliation]{affiliation} \hyperref[TEI.country]{country} \hyperref[TEI.forename]{forename} \hyperref[TEI.genName]{genName} \hyperref[TEI.geogName]{geogName} \hyperref[TEI.nameLink]{nameLink} \hyperref[TEI.orgName]{orgName} \hyperref[TEI.persName]{persName} \hyperref[TEI.person]{person} \hyperref[TEI.personGrp]{personGrp} \hyperref[TEI.persona]{persona} \hyperref[TEI.placeName]{placeName} \hyperref[TEI.region]{region} \hyperref[TEI.roleName]{roleName} \hyperref[TEI.settlement]{settlement} \hyperref[TEI.surname]{surname}\par 
    \item[spoken: ]
   \hyperref[TEI.annotationBlock]{annotationBlock}\par 
    \item[standOff: ]
   \hyperref[TEI.listAnnotation]{listAnnotation}\par 
    \item[textstructure: ]
   \hyperref[TEI.back]{back} \hyperref[TEI.body]{body} \hyperref[TEI.div]{div} \hyperref[TEI.docAuthor]{docAuthor} \hyperref[TEI.docDate]{docDate} \hyperref[TEI.docEdition]{docEdition} \hyperref[TEI.docTitle]{docTitle} \hyperref[TEI.floatingText]{floatingText} \hyperref[TEI.front]{front} \hyperref[TEI.group]{group} \hyperref[TEI.text]{text} \hyperref[TEI.titlePage]{titlePage} \hyperref[TEI.titlePart]{titlePart}\par 
    \item[transcr: ]
   \hyperref[TEI.damage]{damage} \hyperref[TEI.fw]{fw} \hyperref[TEI.line]{line} \hyperref[TEI.metamark]{metamark} \hyperref[TEI.mod]{mod} \hyperref[TEI.restore]{restore} \hyperref[TEI.retrace]{retrace} \hyperref[TEI.secl]{secl} \hyperref[TEI.sourceDoc]{sourceDoc} \hyperref[TEI.supplied]{supplied} \hyperref[TEI.surface]{surface} \hyperref[TEI.surfaceGrp]{surfaceGrp} \hyperref[TEI.surplus]{surplus} \hyperref[TEI.zone]{zone}
    \item[{Peut contenir}]
  
    \item[analysis: ]
   \hyperref[TEI.c]{c} \hyperref[TEI.cl]{cl} \hyperref[TEI.interp]{interp} \hyperref[TEI.interpGrp]{interpGrp} \hyperref[TEI.m]{m} \hyperref[TEI.pc]{pc} \hyperref[TEI.phr]{phr} \hyperref[TEI.s]{s} \hyperref[TEI.span]{span} \hyperref[TEI.spanGrp]{spanGrp} \hyperref[TEI.w]{w}\par 
    \item[core: ]
   \hyperref[TEI.abbr]{abbr} \hyperref[TEI.add]{add} \hyperref[TEI.address]{address} \hyperref[TEI.bibl]{bibl} \hyperref[TEI.biblStruct]{biblStruct} \hyperref[TEI.binaryObject]{binaryObject} \hyperref[TEI.cb]{cb} \hyperref[TEI.choice]{choice} \hyperref[TEI.cit]{cit} \hyperref[TEI.corr]{corr} \hyperref[TEI.date]{date} \hyperref[TEI.del]{del} \hyperref[TEI.desc]{desc} \hyperref[TEI.distinct]{distinct} \hyperref[TEI.email]{email} \hyperref[TEI.emph]{emph} \hyperref[TEI.expan]{expan} \hyperref[TEI.foreign]{foreign} \hyperref[TEI.gap]{gap} \hyperref[TEI.gb]{gb} \hyperref[TEI.gloss]{gloss} \hyperref[TEI.graphic]{graphic} \hyperref[TEI.hi]{hi} \hyperref[TEI.index]{index} \hyperref[TEI.l]{l} \hyperref[TEI.label]{label} \hyperref[TEI.lb]{lb} \hyperref[TEI.lg]{lg} \hyperref[TEI.list]{list} \hyperref[TEI.listBibl]{listBibl} \hyperref[TEI.measure]{measure} \hyperref[TEI.measureGrp]{measureGrp} \hyperref[TEI.media]{media} \hyperref[TEI.mentioned]{mentioned} \hyperref[TEI.milestone]{milestone} \hyperref[TEI.name]{name} \hyperref[TEI.note]{note} \hyperref[TEI.num]{num} \hyperref[TEI.orig]{orig} \hyperref[TEI.p]{p} \hyperref[TEI.pb]{pb} \hyperref[TEI.ptr]{ptr} \hyperref[TEI.q]{q} \hyperref[TEI.quote]{quote} \hyperref[TEI.ref]{ref} \hyperref[TEI.reg]{reg} \hyperref[TEI.rs]{rs} \hyperref[TEI.said]{said} \hyperref[TEI.sic]{sic} \hyperref[TEI.soCalled]{soCalled} \hyperref[TEI.sp]{sp} \hyperref[TEI.stage]{stage} \hyperref[TEI.term]{term} \hyperref[TEI.time]{time} \hyperref[TEI.title]{title} \hyperref[TEI.unclear]{unclear}\par 
    \item[derived-module-tei.istex: ]
   \hyperref[TEI.math]{math} \hyperref[TEI.mrow]{mrow}\par 
    \item[figures: ]
   \hyperref[TEI.figure]{figure} \hyperref[TEI.formula]{formula} \hyperref[TEI.notatedMusic]{notatedMusic} \hyperref[TEI.table]{table}\par 
    \item[header: ]
   \hyperref[TEI.biblFull]{biblFull} \hyperref[TEI.idno]{idno}\par 
    \item[iso-fs: ]
   \hyperref[TEI.fLib]{fLib} \hyperref[TEI.fs]{fs} \hyperref[TEI.fvLib]{fvLib}\par 
    \item[linking: ]
   \hyperref[TEI.ab]{ab} \hyperref[TEI.alt]{alt} \hyperref[TEI.altGrp]{altGrp} \hyperref[TEI.anchor]{anchor} \hyperref[TEI.join]{join} \hyperref[TEI.joinGrp]{joinGrp} \hyperref[TEI.link]{link} \hyperref[TEI.linkGrp]{linkGrp} \hyperref[TEI.seg]{seg} \hyperref[TEI.timeline]{timeline}\par 
    \item[msdescription: ]
   \hyperref[TEI.catchwords]{catchwords} \hyperref[TEI.depth]{depth} \hyperref[TEI.dim]{dim} \hyperref[TEI.dimensions]{dimensions} \hyperref[TEI.height]{height} \hyperref[TEI.heraldry]{heraldry} \hyperref[TEI.locus]{locus} \hyperref[TEI.locusGrp]{locusGrp} \hyperref[TEI.material]{material} \hyperref[TEI.msDesc]{msDesc} \hyperref[TEI.objectType]{objectType} \hyperref[TEI.origDate]{origDate} \hyperref[TEI.origPlace]{origPlace} \hyperref[TEI.secFol]{secFol} \hyperref[TEI.signatures]{signatures} \hyperref[TEI.source]{source} \hyperref[TEI.stamp]{stamp} \hyperref[TEI.watermark]{watermark} \hyperref[TEI.width]{width}\par 
    \item[namesdates: ]
   \hyperref[TEI.addName]{addName} \hyperref[TEI.affiliation]{affiliation} \hyperref[TEI.country]{country} \hyperref[TEI.forename]{forename} \hyperref[TEI.genName]{genName} \hyperref[TEI.geogName]{geogName} \hyperref[TEI.listOrg]{listOrg} \hyperref[TEI.listPlace]{listPlace} \hyperref[TEI.location]{location} \hyperref[TEI.nameLink]{nameLink} \hyperref[TEI.orgName]{orgName} \hyperref[TEI.persName]{persName} \hyperref[TEI.placeName]{placeName} \hyperref[TEI.region]{region} \hyperref[TEI.roleName]{roleName} \hyperref[TEI.settlement]{settlement} \hyperref[TEI.state]{state} \hyperref[TEI.surname]{surname}\par 
    \item[spoken: ]
   \hyperref[TEI.annotationBlock]{annotationBlock}\par 
    \item[textstructure: ]
   \hyperref[TEI.floatingText]{floatingText}\par 
    \item[transcr: ]
   \hyperref[TEI.addSpan]{addSpan} \hyperref[TEI.am]{am} \hyperref[TEI.damage]{damage} \hyperref[TEI.damageSpan]{damageSpan} \hyperref[TEI.delSpan]{delSpan} \hyperref[TEI.ex]{ex} \hyperref[TEI.fw]{fw} \hyperref[TEI.handShift]{handShift} \hyperref[TEI.listTranspose]{listTranspose} \hyperref[TEI.metamark]{metamark} \hyperref[TEI.mod]{mod} \hyperref[TEI.redo]{redo} \hyperref[TEI.restore]{restore} \hyperref[TEI.retrace]{retrace} \hyperref[TEI.secl]{secl} \hyperref[TEI.space]{space} \hyperref[TEI.subst]{subst} \hyperref[TEI.substJoin]{substJoin} \hyperref[TEI.supplied]{supplied} \hyperref[TEI.surplus]{surplus} \hyperref[TEI.undo]{undo}\par des données textuelles
    \item[{Exemple}]
  \leavevmode\bgroup\exampleFont \begin{shaded}\noindent\mbox{}{<\textbf{source}>}Derived from {<\textbf{ref}>}Stanley (1960){</\textbf{ref}>}\mbox{}\newline 
{</\textbf{source}>}\end{shaded}\egroup 


    \item[{Modèle de contenu}]
  \mbox{}\hfill\\[-10pt]\begin{Verbatim}[fontsize=\small]
<content>
 <macroRef key="macro.specialPara"/>
</content>
    
\end{Verbatim}

    \item[{Schéma Declaration}]
  \mbox{}\hfill\\[-10pt]\begin{Verbatim}[fontsize=\small]
element source { tei_att.global.attributes, tei_macro.specialPara }
\end{Verbatim}

\end{reflist}  \index{sourceDesc=<sourceDesc>|oddindex}
\begin{reflist}
\item[]\begin{specHead}{TEI.sourceDesc}{<sourceDesc> }(description de la source) décrit la source à partir de laquelle un texte électronique a été dérivé ou produit, habituellement une description bibliographique pour un texte numérisé, ou une expression comme "document numérique natif " pour un texte qui n'a aucune existence précédente. [\xref{http://www.tei-c.org/release/doc/tei-p5-doc/en/html/HD.html\#HD3}{2.2.7. The Source Description}]\end{specHead} 
    \item[{Module}]
  header
    \item[{Attributs}]
  Attributs \hyperref[TEI.att.global]{att.global} (\textit{@xml:id}, \textit{@n}, \textit{@xml:lang}, \textit{@xml:base}, \textit{@xml:space})  (\hyperref[TEI.att.global.rendition]{att.global.rendition} (\textit{@rend}, \textit{@style}, \textit{@rendition})) (\hyperref[TEI.att.global.linking]{att.global.linking} (\textit{@corresp}, \textit{@synch}, \textit{@sameAs}, \textit{@copyOf}, \textit{@next}, \textit{@prev}, \textit{@exclude}, \textit{@select})) (\hyperref[TEI.att.global.analytic]{att.global.analytic} (\textit{@ana})) (\hyperref[TEI.att.global.facs]{att.global.facs} (\textit{@facs})) (\hyperref[TEI.att.global.change]{att.global.change} (\textit{@change})) (\hyperref[TEI.att.global.responsibility]{att.global.responsibility} (\textit{@cert}, \textit{@resp})) (\hyperref[TEI.att.global.source]{att.global.source} (\textit{@source})) \hyperref[TEI.att.declarable]{att.declarable} (\textit{@default}) 
    \item[{Contenu dans}]
  
    \item[header: ]
   \hyperref[TEI.biblFull]{biblFull} \hyperref[TEI.fileDesc]{fileDesc}
    \item[{Peut contenir}]
  
    \item[core: ]
   \hyperref[TEI.bibl]{bibl} \hyperref[TEI.biblStruct]{biblStruct} \hyperref[TEI.list]{list} \hyperref[TEI.listBibl]{listBibl} \hyperref[TEI.p]{p}\par 
    \item[figures: ]
   \hyperref[TEI.table]{table}\par 
    \item[header: ]
   \hyperref[TEI.biblFull]{biblFull}\par 
    \item[linking: ]
   \hyperref[TEI.ab]{ab}\par 
    \item[msdescription: ]
   \hyperref[TEI.msDesc]{msDesc}\par 
    \item[namesdates: ]
   \hyperref[TEI.listOrg]{listOrg} \hyperref[TEI.listPlace]{listPlace}
    \item[{Exemple}]
  \leavevmode\bgroup\exampleFont \begin{shaded}\noindent\mbox{}{<\textbf{sourceDesc}>}\mbox{}\newline 
\hspace*{6pt}{<\textbf{p}>}Texte original : le texte a été créé sous sa forme électronique.{</\textbf{p}>}\mbox{}\newline 
{</\textbf{sourceDesc}>}\end{shaded}\egroup 


    \item[{Modèle de contenu}]
  \mbox{}\hfill\\[-10pt]\begin{Verbatim}[fontsize=\small]
<content>
 <alternate maxOccurs="1" minOccurs="1">
  <classRef key="model.pLike"
   maxOccurs="unbounded" minOccurs="1"/>
  <alternate maxOccurs="unbounded"
   minOccurs="1">
   <classRef key="model.biblLike"/>
   <classRef key="model.sourceDescPart"/>
   <classRef key="model.listLike"/>
  </alternate>
 </alternate>
</content>
    
\end{Verbatim}

    \item[{Schéma Declaration}]
  \mbox{}\hfill\\[-10pt]\begin{Verbatim}[fontsize=\small]
element sourceDesc
{
   tei_att.global.attributes,
   tei_att.declarable.attributes,
   (
      tei_model.pLike+
    | ( tei_model.biblLike | tei_model.sourceDescPart | tei_model.listLike )+
   )
}
\end{Verbatim}

\end{reflist}  \index{sourceDoc=<sourceDoc>|oddindex}
\begin{reflist}
\item[]\begin{specHead}{TEI.sourceDoc}{<sourceDoc> }contains a transcription or other representation of a single source document potentially forming part of a \textit{dossier génétique} or collection of sources. [\xref{http://www.tei-c.org/release/doc/tei-p5-doc/en/html/PH.html\#PHFAX}{11.1. Digital Facsimiles} \xref{http://www.tei-c.org/release/doc/tei-p5-doc/en/html/PH.html\#PHZLAB}{11.2.2. Embedded Transcription}]\end{specHead} 
    \item[{Module}]
  transcr
    \item[{Attributs}]
  Attributs \hyperref[TEI.att.global]{att.global} (\textit{@xml:id}, \textit{@n}, \textit{@xml:lang}, \textit{@xml:base}, \textit{@xml:space})  (\hyperref[TEI.att.global.rendition]{att.global.rendition} (\textit{@rend}, \textit{@style}, \textit{@rendition})) (\hyperref[TEI.att.global.linking]{att.global.linking} (\textit{@corresp}, \textit{@synch}, \textit{@sameAs}, \textit{@copyOf}, \textit{@next}, \textit{@prev}, \textit{@exclude}, \textit{@select})) (\hyperref[TEI.att.global.analytic]{att.global.analytic} (\textit{@ana})) (\hyperref[TEI.att.global.facs]{att.global.facs} (\textit{@facs})) (\hyperref[TEI.att.global.change]{att.global.change} (\textit{@change})) (\hyperref[TEI.att.global.responsibility]{att.global.responsibility} (\textit{@cert}, \textit{@resp})) (\hyperref[TEI.att.global.source]{att.global.source} (\textit{@source})) \hyperref[TEI.att.declaring]{att.declaring} (\textit{@decls}) 
    \item[{Membre du}]
  \hyperref[TEI.model.resourceLike]{model.resourceLike}
    \item[{Contenu dans}]
  
    \item[core: ]
   \hyperref[TEI.teiCorpus]{teiCorpus}\par 
    \item[standOff: ]
   \hyperref[TEI.standOff]{standOff}\par 
    \item[textstructure: ]
   \hyperref[TEI.TEI]{TEI}
    \item[{Peut contenir}]
  
    \item[analysis: ]
   \hyperref[TEI.interp]{interp} \hyperref[TEI.interpGrp]{interpGrp} \hyperref[TEI.span]{span} \hyperref[TEI.spanGrp]{spanGrp}\par 
    \item[core: ]
   \hyperref[TEI.binaryObject]{binaryObject} \hyperref[TEI.cb]{cb} \hyperref[TEI.gap]{gap} \hyperref[TEI.gb]{gb} \hyperref[TEI.graphic]{graphic} \hyperref[TEI.index]{index} \hyperref[TEI.lb]{lb} \hyperref[TEI.media]{media} \hyperref[TEI.milestone]{milestone} \hyperref[TEI.note]{note} \hyperref[TEI.pb]{pb}\par 
    \item[derived-module-tei.istex: ]
   \hyperref[TEI.math]{math} \hyperref[TEI.mrow]{mrow}\par 
    \item[figures: ]
   \hyperref[TEI.figure]{figure} \hyperref[TEI.formula]{formula} \hyperref[TEI.notatedMusic]{notatedMusic}\par 
    \item[iso-fs: ]
   \hyperref[TEI.fLib]{fLib} \hyperref[TEI.fs]{fs} \hyperref[TEI.fvLib]{fvLib}\par 
    \item[linking: ]
   \hyperref[TEI.alt]{alt} \hyperref[TEI.altGrp]{altGrp} \hyperref[TEI.anchor]{anchor} \hyperref[TEI.join]{join} \hyperref[TEI.joinGrp]{joinGrp} \hyperref[TEI.link]{link} \hyperref[TEI.linkGrp]{linkGrp} \hyperref[TEI.timeline]{timeline}\par 
    \item[msdescription: ]
   \hyperref[TEI.source]{source}\par 
    \item[transcr: ]
   \hyperref[TEI.addSpan]{addSpan} \hyperref[TEI.damageSpan]{damageSpan} \hyperref[TEI.delSpan]{delSpan} \hyperref[TEI.fw]{fw} \hyperref[TEI.listTranspose]{listTranspose} \hyperref[TEI.metamark]{metamark} \hyperref[TEI.space]{space} \hyperref[TEI.substJoin]{substJoin} \hyperref[TEI.surface]{surface} \hyperref[TEI.surfaceGrp]{surfaceGrp}
    \item[{Note}]
  \par
This element may be used as an alternative to \hyperref[TEI.facsimile]{<facsimile>} for TEI documents containing only page images, or for documents containing both images and transcriptions. Transcriptions may be provided within the \hyperref[TEI.surface]{<surface>} elements making up a source document, in parallel with them as part of a \hyperref[TEI.text]{<text>} element, or in both places if the encoder wishes to distinguish these two modes of transcription.
    \item[{Exemple}]
  \leavevmode\bgroup\exampleFont \begin{shaded}\noindent\mbox{}{<\textbf{sourceDoc}>}\mbox{}\newline 
\hspace*{6pt}{<\textbf{surfaceGrp}\hspace*{6pt}{n}="{leaf1}">}\mbox{}\newline 
\hspace*{6pt}\hspace*{6pt}{<\textbf{surface}\hspace*{6pt}{facs}="{page1.png}">}\mbox{}\newline 
\hspace*{6pt}\hspace*{6pt}\hspace*{6pt}{<\textbf{zone}>}All the writing on page 1{</\textbf{zone}>}\mbox{}\newline 
\hspace*{6pt}\hspace*{6pt}{</\textbf{surface}>}\mbox{}\newline 
\hspace*{6pt}\hspace*{6pt}{<\textbf{surface}>}\mbox{}\newline 
\hspace*{6pt}\hspace*{6pt}\hspace*{6pt}{<\textbf{graphic}\hspace*{6pt}{url}="{page2-highRes.png}"/>}\mbox{}\newline 
\hspace*{6pt}\hspace*{6pt}\hspace*{6pt}{<\textbf{graphic}\hspace*{6pt}{url}="{page2-lowRes.png}"/>}\mbox{}\newline 
\hspace*{6pt}\hspace*{6pt}\hspace*{6pt}{<\textbf{zone}>}\mbox{}\newline 
\hspace*{6pt}\hspace*{6pt}\hspace*{6pt}\hspace*{6pt}{<\textbf{line}>}A line of writing on page 2{</\textbf{line}>}\mbox{}\newline 
\hspace*{6pt}\hspace*{6pt}\hspace*{6pt}\hspace*{6pt}{<\textbf{line}>}Another line of writing on page 2{</\textbf{line}>}\mbox{}\newline 
\hspace*{6pt}\hspace*{6pt}\hspace*{6pt}{</\textbf{zone}>}\mbox{}\newline 
\hspace*{6pt}\hspace*{6pt}{</\textbf{surface}>}\mbox{}\newline 
\hspace*{6pt}{</\textbf{surfaceGrp}>}\mbox{}\newline 
{</\textbf{sourceDoc}>}\end{shaded}\egroup 


    \item[{Modèle de contenu}]
  \mbox{}\hfill\\[-10pt]\begin{Verbatim}[fontsize=\small]
<content>
 <alternate maxOccurs="unbounded"
  minOccurs="1">
  <classRef key="model.global"/>
  <classRef key="model.graphicLike"/>
  <elementRef key="surface"/>
  <elementRef key="surfaceGrp"/>
 </alternate>
</content>
    
\end{Verbatim}

    \item[{Schéma Declaration}]
  \mbox{}\hfill\\[-10pt]\begin{Verbatim}[fontsize=\small]
element sourceDoc
{
   tei_att.global.attributes,
   tei_att.declaring.attributes,
   ( tei_model.global | tei_model.graphicLike | tei_surface | tei_surfaceGrp )+
}
\end{Verbatim}

\end{reflist}  \index{sp=<sp>|oddindex}
\begin{reflist}
\item[]\begin{specHead}{TEI.sp}{<sp> }(langue orale) monologue dans un texte écrit pour la scène ou un passage présenté sous cette forme dans un texte en prose ou en vers. [\xref{http://www.tei-c.org/release/doc/tei-p5-doc/en/html/CO.html\#CODR}{3.12.2. Core Tags for Drama} \xref{http://www.tei-c.org/release/doc/tei-p5-doc/en/html/CO.html\#CODV}{3.12. Passages of Verse or Drama} \xref{http://www.tei-c.org/release/doc/tei-p5-doc/en/html/DR.html\#DRSP}{7.2.2. Speeches and Speakers}]\end{specHead} 
    \item[{Module}]
  core
    \item[{Attributs}]
  Attributs \hyperref[TEI.att.global]{att.global} (\textit{@xml:id}, \textit{@n}, \textit{@xml:lang}, \textit{@xml:base}, \textit{@xml:space})  (\hyperref[TEI.att.global.rendition]{att.global.rendition} (\textit{@rend}, \textit{@style}, \textit{@rendition})) (\hyperref[TEI.att.global.linking]{att.global.linking} (\textit{@corresp}, \textit{@synch}, \textit{@sameAs}, \textit{@copyOf}, \textit{@next}, \textit{@prev}, \textit{@exclude}, \textit{@select})) (\hyperref[TEI.att.global.analytic]{att.global.analytic} (\textit{@ana})) (\hyperref[TEI.att.global.facs]{att.global.facs} (\textit{@facs})) (\hyperref[TEI.att.global.change]{att.global.change} (\textit{@change})) (\hyperref[TEI.att.global.responsibility]{att.global.responsibility} (\textit{@cert}, \textit{@resp})) (\hyperref[TEI.att.global.source]{att.global.source} (\textit{@source})) \hyperref[TEI.att.ascribed]{att.ascribed} (\textit{@who}) 
    \item[{Membre du}]
  \hyperref[TEI.model.divPart]{model.divPart}
    \item[{Contenu dans}]
  
    \item[core: ]
   \hyperref[TEI.item]{item} \hyperref[TEI.note]{note} \hyperref[TEI.q]{q} \hyperref[TEI.quote]{quote} \hyperref[TEI.said]{said} \hyperref[TEI.stage]{stage}\par 
    \item[figures: ]
   \hyperref[TEI.cell]{cell} \hyperref[TEI.figure]{figure}\par 
    \item[header: ]
   \hyperref[TEI.change]{change} \hyperref[TEI.licence]{licence}\par 
    \item[msdescription: ]
   \hyperref[TEI.accMat]{accMat} \hyperref[TEI.acquisition]{acquisition} \hyperref[TEI.additions]{additions} \hyperref[TEI.collation]{collation} \hyperref[TEI.condition]{condition} \hyperref[TEI.custEvent]{custEvent} \hyperref[TEI.decoNote]{decoNote} \hyperref[TEI.filiation]{filiation} \hyperref[TEI.foliation]{foliation} \hyperref[TEI.layout]{layout} \hyperref[TEI.musicNotation]{musicNotation} \hyperref[TEI.origin]{origin} \hyperref[TEI.provenance]{provenance} \hyperref[TEI.signatures]{signatures} \hyperref[TEI.source]{source} \hyperref[TEI.summary]{summary} \hyperref[TEI.support]{support} \hyperref[TEI.surrogates]{surrogates} \hyperref[TEI.typeNote]{typeNote}\par 
    \item[textstructure: ]
   \hyperref[TEI.body]{body} \hyperref[TEI.div]{div}\par 
    \item[transcr: ]
   \hyperref[TEI.metamark]{metamark}
    \item[{Peut contenir}]
  
    \item[analysis: ]
   \hyperref[TEI.interp]{interp} \hyperref[TEI.interpGrp]{interpGrp} \hyperref[TEI.span]{span} \hyperref[TEI.spanGrp]{spanGrp}\par 
    \item[core: ]
   \hyperref[TEI.cb]{cb} \hyperref[TEI.cit]{cit} \hyperref[TEI.gap]{gap} \hyperref[TEI.gb]{gb} \hyperref[TEI.index]{index} \hyperref[TEI.l]{l} \hyperref[TEI.lb]{lb} \hyperref[TEI.lg]{lg} \hyperref[TEI.list]{list} \hyperref[TEI.milestone]{milestone} \hyperref[TEI.note]{note} \hyperref[TEI.p]{p} \hyperref[TEI.pb]{pb} \hyperref[TEI.q]{q} \hyperref[TEI.quote]{quote} \hyperref[TEI.said]{said} \hyperref[TEI.speaker]{speaker} \hyperref[TEI.stage]{stage}\par 
    \item[figures: ]
   \hyperref[TEI.figure]{figure} \hyperref[TEI.notatedMusic]{notatedMusic} \hyperref[TEI.table]{table}\par 
    \item[iso-fs: ]
   \hyperref[TEI.fLib]{fLib} \hyperref[TEI.fs]{fs} \hyperref[TEI.fvLib]{fvLib}\par 
    \item[linking: ]
   \hyperref[TEI.ab]{ab} \hyperref[TEI.alt]{alt} \hyperref[TEI.altGrp]{altGrp} \hyperref[TEI.anchor]{anchor} \hyperref[TEI.join]{join} \hyperref[TEI.joinGrp]{joinGrp} \hyperref[TEI.link]{link} \hyperref[TEI.linkGrp]{linkGrp} \hyperref[TEI.timeline]{timeline}\par 
    \item[msdescription: ]
   \hyperref[TEI.source]{source}\par 
    \item[namesdates: ]
   \hyperref[TEI.listOrg]{listOrg} \hyperref[TEI.listPlace]{listPlace}\par 
    \item[textstructure: ]
   \hyperref[TEI.floatingText]{floatingText}\par 
    \item[transcr: ]
   \hyperref[TEI.addSpan]{addSpan} \hyperref[TEI.damageSpan]{damageSpan} \hyperref[TEI.delSpan]{delSpan} \hyperref[TEI.fw]{fw} \hyperref[TEI.listTranspose]{listTranspose} \hyperref[TEI.metamark]{metamark} \hyperref[TEI.space]{space} \hyperref[TEI.substJoin]{substJoin}
    \item[{Note}]
  \par
L'attribut {\itshape who} peut être utilisé soit en complément de l'élément \hyperref[TEI.speaker]{<speaker>}, soit comme une alternative à cet élément.
    \item[{Exemple}]
  \leavevmode\bgroup\exampleFont \begin{shaded}\noindent\mbox{}{<\textbf{sp}>}\mbox{}\newline 
\hspace*{6pt}{<\textbf{speaker}>} Valère.{</\textbf{speaker}>}\mbox{}\newline 
\hspace*{6pt}{<\textbf{p}>}Hé bien ! Sabine, quel conseil me donneras-tu ?{</\textbf{p}>}\mbox{}\newline 
{</\textbf{sp}>}\mbox{}\newline 
{<\textbf{sp}>}\mbox{}\newline 
\hspace*{6pt}{<\textbf{speaker}>} Sabine.{</\textbf{speaker}>}\mbox{}\newline 
\hspace*{6pt}{<\textbf{p}>}Vraiment, il y a bien des nouvelles. Mon oncle veut résolûment que ma cousine épouse\mbox{}\newline 
\hspace*{6pt}\hspace*{6pt} Villebrequin, et les affaires sont tellement avancées, que je crois qu'ils eussent été\mbox{}\newline 
\hspace*{6pt}\hspace*{6pt} mariés dès aujourd'hui, si vous n'étiez aimé ... Le bonhomme ne manquera pas\mbox{}\newline 
\hspace*{6pt}\hspace*{6pt} de faire loger ma cousine à ce pavillon qui est au bout de notre jardin, et par ce moyen\mbox{}\newline 
\hspace*{6pt}\hspace*{6pt} vous pourriez l'entretenir à l'insu de notre vieillard, l'épouser, et le laisser pester\mbox{}\newline 
\hspace*{6pt}\hspace*{6pt} tout son soûl avec Villebrequin.{</\textbf{p}>}\mbox{}\newline 
{</\textbf{sp}>}\end{shaded}\egroup 


    \item[{Modèle de contenu}]
  \mbox{}\hfill\\[-10pt]\begin{Verbatim}[fontsize=\small]
<content>
 <sequence maxOccurs="1" minOccurs="1">
  <classRef key="model.global"
   maxOccurs="unbounded" minOccurs="0"/>
  <sequence maxOccurs="1" minOccurs="0">
   <elementRef key="speaker"/>
   <classRef key="model.global"
    maxOccurs="unbounded" minOccurs="0"/>
  </sequence>
  <sequence maxOccurs="unbounded"
   minOccurs="1">
   <alternate maxOccurs="1" minOccurs="1">
    <elementRef key="lg"/>
    <classRef key="model.lLike"/>
    <classRef key="model.pLike"/>
    <classRef key="model.listLike"/>
    <classRef key="model.stageLike"/>
    <classRef key="model.qLike"/>
   </alternate>
   <classRef key="model.global"
    maxOccurs="unbounded" minOccurs="0"/>
  </sequence>
 </sequence>
</content>
    
\end{Verbatim}

    \item[{Schéma Declaration}]
  \mbox{}\hfill\\[-10pt]\begin{Verbatim}[fontsize=\small]
element sp
{
   tei_att.global.attributes,
   tei_att.ascribed.attributes,
   (
      tei_model.global*,
      ( tei_speaker, tei_model.global* )?,
      (
         (
            tei_lg          | tei_model.lLike          | tei_model.pLike          | tei_model.listLike          | tei_model.stageLike          | tei_model.qLike         ),
         tei_model.global*
      )+
   )
}
\end{Verbatim}

\end{reflist}  \index{space=<space>|oddindex}\index{resp=@resp!<space>|oddindex}\index{dim=@dim!<space>|oddindex}
\begin{reflist}
\item[]\begin{specHead}{TEI.space}{<space> }(espace) permet de situer un espace significatif dans le texte édité. [\xref{http://www.tei-c.org/release/doc/tei-p5-doc/en/html/PH.html\#PHSP}{11.5.1. Space}]\end{specHead} 
    \item[{Module}]
  transcr
    \item[{Attributs}]
  Attributs \hyperref[TEI.att.typed]{att.typed} (\textit{@type}, \textit{@subtype}) \hyperref[TEI.att.dimensions]{att.dimensions} (\textit{@unit}, \textit{@quantity}, \textit{@extent}, \textit{@precision}, \textit{@scope})  (\hyperref[TEI.att.ranging]{att.ranging} (\textit{@atLeast}, \textit{@atMost}, \textit{@min}, \textit{@max}, \textit{@confidence})) \hyperref[TEI.att.global]{att.global} (@xml:id, @n, @xml:lang, @xml:base, @xml:space) \hyperref[TEI.att.global.rendition]{att.global.rendition} (@rend, @style, @rendition) \hyperref[TEI.att.global.linking]{att.global.linking} (@corresp, @synch, @sameAs, @copyOf, @next, @prev, @exclude, @select) \hyperref[TEI.att.global.analytic]{att.global.analytic} (@ana) \hyperref[TEI.att.global.facs]{att.global.facs} (@facs) \hyperref[TEI.att.global.change]{att.global.change} (@change) \hyperref[TEI.att.global.responsibility]{att.global.responsibility} (\unusedattribute{resp}, @cert) \hyperref[TEI.att.global.source]{att.global.source} (@source) \hfil\\[-10pt]\begin{sansreflist}
    \item[@resp]
  (responsable) (responsible party) indicates the individual responsible for identifying and measuring the space
\begin{reflist}
    \item[{Dérivé de}]
  \hyperref[TEI.att.global.responsibility]{att.global.responsibility}
    \item[{Statut}]
  Optionel
    \item[{Type de données}]
  1–∞ occurrences de \hyperref[TEI.teidata.pointer]{teidata.pointer} séparé par un espace
\end{reflist}  
    \item[@dim]
  (dimension) indique si l'espace est vertical ou horizontal.
\begin{reflist}
    \item[{Statut}]
  Recommendé
    \item[{Type de données}]
  \hyperref[TEI.teidata.enumerated]{teidata.enumerated}
    \item[{Les valeurs autorisées sont:}]
  \begin{description}

\item[{horizontal}]l'espace est horizontal.
\item[{vertical}]l'espace est vertical.
\end{description} 
    \item[{Note}]
  \par
Pour des formes irrégulières à deux dimensions, la valeur de cet attribut doit refléter la plus importante des deux dimensions. Dans les textes écrits conventionnellement de gauche à droite, un espace composé de parties horizontales et verticales doit être considéré comme vertical.
\end{reflist}  
\end{sansreflist}  
    \item[{Membre du}]
  \hyperref[TEI.model.global.edit]{model.global.edit}
    \item[{Contenu dans}]
  
    \item[analysis: ]
   \hyperref[TEI.cl]{cl} \hyperref[TEI.m]{m} \hyperref[TEI.phr]{phr} \hyperref[TEI.s]{s} \hyperref[TEI.span]{span} \hyperref[TEI.w]{w}\par 
    \item[core: ]
   \hyperref[TEI.abbr]{abbr} \hyperref[TEI.add]{add} \hyperref[TEI.addrLine]{addrLine} \hyperref[TEI.address]{address} \hyperref[TEI.author]{author} \hyperref[TEI.bibl]{bibl} \hyperref[TEI.biblScope]{biblScope} \hyperref[TEI.cit]{cit} \hyperref[TEI.citedRange]{citedRange} \hyperref[TEI.corr]{corr} \hyperref[TEI.date]{date} \hyperref[TEI.del]{del} \hyperref[TEI.distinct]{distinct} \hyperref[TEI.editor]{editor} \hyperref[TEI.email]{email} \hyperref[TEI.emph]{emph} \hyperref[TEI.expan]{expan} \hyperref[TEI.foreign]{foreign} \hyperref[TEI.gloss]{gloss} \hyperref[TEI.head]{head} \hyperref[TEI.headItem]{headItem} \hyperref[TEI.headLabel]{headLabel} \hyperref[TEI.hi]{hi} \hyperref[TEI.imprint]{imprint} \hyperref[TEI.item]{item} \hyperref[TEI.l]{l} \hyperref[TEI.label]{label} \hyperref[TEI.lg]{lg} \hyperref[TEI.list]{list} \hyperref[TEI.measure]{measure} \hyperref[TEI.mentioned]{mentioned} \hyperref[TEI.name]{name} \hyperref[TEI.note]{note} \hyperref[TEI.num]{num} \hyperref[TEI.orig]{orig} \hyperref[TEI.p]{p} \hyperref[TEI.pubPlace]{pubPlace} \hyperref[TEI.publisher]{publisher} \hyperref[TEI.q]{q} \hyperref[TEI.quote]{quote} \hyperref[TEI.ref]{ref} \hyperref[TEI.reg]{reg} \hyperref[TEI.resp]{resp} \hyperref[TEI.rs]{rs} \hyperref[TEI.said]{said} \hyperref[TEI.series]{series} \hyperref[TEI.sic]{sic} \hyperref[TEI.soCalled]{soCalled} \hyperref[TEI.sp]{sp} \hyperref[TEI.speaker]{speaker} \hyperref[TEI.stage]{stage} \hyperref[TEI.street]{street} \hyperref[TEI.term]{term} \hyperref[TEI.textLang]{textLang} \hyperref[TEI.time]{time} \hyperref[TEI.title]{title} \hyperref[TEI.unclear]{unclear}\par 
    \item[figures: ]
   \hyperref[TEI.cell]{cell} \hyperref[TEI.figure]{figure} \hyperref[TEI.table]{table}\par 
    \item[header: ]
   \hyperref[TEI.authority]{authority} \hyperref[TEI.change]{change} \hyperref[TEI.classCode]{classCode} \hyperref[TEI.distributor]{distributor} \hyperref[TEI.edition]{edition} \hyperref[TEI.extent]{extent} \hyperref[TEI.funder]{funder} \hyperref[TEI.language]{language} \hyperref[TEI.licence]{licence}\par 
    \item[linking: ]
   \hyperref[TEI.ab]{ab} \hyperref[TEI.seg]{seg}\par 
    \item[msdescription: ]
   \hyperref[TEI.accMat]{accMat} \hyperref[TEI.acquisition]{acquisition} \hyperref[TEI.additions]{additions} \hyperref[TEI.catchwords]{catchwords} \hyperref[TEI.collation]{collation} \hyperref[TEI.colophon]{colophon} \hyperref[TEI.condition]{condition} \hyperref[TEI.custEvent]{custEvent} \hyperref[TEI.decoNote]{decoNote} \hyperref[TEI.explicit]{explicit} \hyperref[TEI.filiation]{filiation} \hyperref[TEI.finalRubric]{finalRubric} \hyperref[TEI.foliation]{foliation} \hyperref[TEI.heraldry]{heraldry} \hyperref[TEI.incipit]{incipit} \hyperref[TEI.layout]{layout} \hyperref[TEI.material]{material} \hyperref[TEI.msItem]{msItem} \hyperref[TEI.musicNotation]{musicNotation} \hyperref[TEI.objectType]{objectType} \hyperref[TEI.origDate]{origDate} \hyperref[TEI.origPlace]{origPlace} \hyperref[TEI.origin]{origin} \hyperref[TEI.provenance]{provenance} \hyperref[TEI.rubric]{rubric} \hyperref[TEI.secFol]{secFol} \hyperref[TEI.signatures]{signatures} \hyperref[TEI.source]{source} \hyperref[TEI.stamp]{stamp} \hyperref[TEI.summary]{summary} \hyperref[TEI.support]{support} \hyperref[TEI.surrogates]{surrogates} \hyperref[TEI.typeNote]{typeNote} \hyperref[TEI.watermark]{watermark}\par 
    \item[namesdates: ]
   \hyperref[TEI.addName]{addName} \hyperref[TEI.affiliation]{affiliation} \hyperref[TEI.country]{country} \hyperref[TEI.forename]{forename} \hyperref[TEI.genName]{genName} \hyperref[TEI.geogName]{geogName} \hyperref[TEI.nameLink]{nameLink} \hyperref[TEI.orgName]{orgName} \hyperref[TEI.persName]{persName} \hyperref[TEI.person]{person} \hyperref[TEI.personGrp]{personGrp} \hyperref[TEI.persona]{persona} \hyperref[TEI.placeName]{placeName} \hyperref[TEI.region]{region} \hyperref[TEI.roleName]{roleName} \hyperref[TEI.settlement]{settlement} \hyperref[TEI.surname]{surname}\par 
    \item[textstructure: ]
   \hyperref[TEI.back]{back} \hyperref[TEI.body]{body} \hyperref[TEI.div]{div} \hyperref[TEI.docAuthor]{docAuthor} \hyperref[TEI.docDate]{docDate} \hyperref[TEI.docEdition]{docEdition} \hyperref[TEI.docTitle]{docTitle} \hyperref[TEI.floatingText]{floatingText} \hyperref[TEI.front]{front} \hyperref[TEI.group]{group} \hyperref[TEI.text]{text} \hyperref[TEI.titlePage]{titlePage} \hyperref[TEI.titlePart]{titlePart}\par 
    \item[transcr: ]
   \hyperref[TEI.damage]{damage} \hyperref[TEI.fw]{fw} \hyperref[TEI.line]{line} \hyperref[TEI.metamark]{metamark} \hyperref[TEI.mod]{mod} \hyperref[TEI.restore]{restore} \hyperref[TEI.retrace]{retrace} \hyperref[TEI.secl]{secl} \hyperref[TEI.sourceDoc]{sourceDoc} \hyperref[TEI.supplied]{supplied} \hyperref[TEI.surface]{surface} \hyperref[TEI.surfaceGrp]{surfaceGrp} \hyperref[TEI.surplus]{surplus} \hyperref[TEI.zone]{zone}
    \item[{Peut contenir}]
  
    \item[core: ]
   \hyperref[TEI.desc]{desc}
    \item[{Note}]
  \par
Cet élément devrait être utilisé partout où l'on désire signaler un espace inhabituel dans le texte source, par exemple un espace réservé pour un mot à écrire plus tard, pour une rubrication ultérieure, etc. Il n'est pas destiné à être utilisé pour marquer l'espace normal entre des mots par exemple.
    \item[{Exemple}]
  \leavevmode\bgroup\exampleFont \begin{shaded}\noindent\mbox{} Lettre à lettre,\mbox{}\newline 
 un texte se forme, s'affirme, s'affermit, se fixe, se fige : une ligne assez strictement\mbox{}\newline 
 horizontale se dépose sur la {<\textbf{space}\hspace*{6pt}{quantity}="{3}"\hspace*{6pt}{unit}="{lignes}"/>}feuille blanche.\end{shaded}\egroup 


    \item[{Modèle de contenu}]
  \mbox{}\hfill\\[-10pt]\begin{Verbatim}[fontsize=\small]
<content>
 <alternate maxOccurs="unbounded"
  minOccurs="0">
  <classRef key="model.descLike"/>
  <classRef key="model.certLike"/>
 </alternate>
</content>
    
\end{Verbatim}

    \item[{Schéma Declaration}]
  \mbox{}\hfill\\[-10pt]\begin{Verbatim}[fontsize=\small]
element space
{
   tei_att.global.attribute.xmlid,
   tei_att.global.attribute.n,
   tei_att.global.attribute.xmllang,
   tei_att.global.attribute.xmlbase,
   tei_att.global.attribute.xmlspace,
   tei_att.global.rendition.attribute.rend,
   tei_att.global.rendition.attribute.style,
   tei_att.global.rendition.attribute.rendition,
   tei_att.global.linking.attribute.corresp,
   tei_att.global.linking.attribute.synch,
   tei_att.global.linking.attribute.sameAs,
   tei_att.global.linking.attribute.copyOf,
   tei_att.global.linking.attribute.next,
   tei_att.global.linking.attribute.prev,
   tei_att.global.linking.attribute.exclude,
   tei_att.global.linking.attribute.select,
   tei_att.global.analytic.attribute.ana,
   tei_att.global.facs.attribute.facs,
   tei_att.global.change.attribute.change,
   tei_att.global.responsibility.attribute.cert,
   tei_att.global.source.attribute.source,
   tei_att.typed.attributes,
   tei_att.dimensions.attributes,
   attribute resp { list { + } }?,
   attribute dim { "horizontal" | "vertical" }?,
   ( tei_model.descLike | tei_model.certLike )*
}
\end{Verbatim}

\end{reflist}  \index{span=<span>|oddindex}\index{from=@from!<span>|oddindex}\index{to=@to!<span>|oddindex}
\begin{reflist}
\item[]\begin{specHead}{TEI.span}{<span> }associe une interprétation sous forme d'annotation directement à un passage donné dans un texte. [\xref{http://www.tei-c.org/release/doc/tei-p5-doc/en/html/AI.html\#AISP}{17.3. Spans and Interpretations}]\end{specHead} 
    \item[{Module}]
  analysis
    \item[{Attributs}]
  Attributs \hyperref[TEI.att.global]{att.global} (\textit{@xml:id}, \textit{@n}, \textit{@xml:lang}, \textit{@xml:base}, \textit{@xml:space})  (\hyperref[TEI.att.global.rendition]{att.global.rendition} (\textit{@rend}, \textit{@style}, \textit{@rendition})) (\hyperref[TEI.att.global.linking]{att.global.linking} (\textit{@corresp}, \textit{@synch}, \textit{@sameAs}, \textit{@copyOf}, \textit{@next}, \textit{@prev}, \textit{@exclude}, \textit{@select})) (\hyperref[TEI.att.global.analytic]{att.global.analytic} (\textit{@ana})) (\hyperref[TEI.att.global.facs]{att.global.facs} (\textit{@facs})) (\hyperref[TEI.att.global.change]{att.global.change} (\textit{@change})) (\hyperref[TEI.att.global.responsibility]{att.global.responsibility} (\textit{@cert}, \textit{@resp})) (\hyperref[TEI.att.global.source]{att.global.source} (\textit{@source})) \hyperref[TEI.att.interpLike]{att.interpLike} (\textit{@type}, \textit{@inst}) \hyperref[TEI.att.pointing]{att.pointing} (\textit{@targetLang}, \textit{@target}, \textit{@evaluate}) \hfil\\[-10pt]\begin{sansreflist}
    \item[@from]
  précise le début du passage sur lequel porte l'annotation ; s'il n'est pas accompagné d'un attribut {\itshape to}, désigne alors l'intégralité du passage
\begin{reflist}
    \item[{Statut}]
  Optionel
    \item[{Type de données}]
  \hyperref[TEI.teidata.pointer]{teidata.pointer}
\end{reflist}  
    \item[@to]
  spécifie la fin du passage annoté.
\begin{reflist}
    \item[{Statut}]
  Optionel
    \item[{Type de données}]
  \hyperref[TEI.teidata.pointer]{teidata.pointer}
\end{reflist}  
\end{sansreflist}  
    \item[{Membre du}]
  \hyperref[TEI.model.OATarget]{model.OATarget} \hyperref[TEI.model.global.meta]{model.global.meta} 
    \item[{Contenu dans}]
  
    \item[analysis: ]
   \hyperref[TEI.cl]{cl} \hyperref[TEI.m]{m} \hyperref[TEI.phr]{phr} \hyperref[TEI.s]{s} \hyperref[TEI.span]{span} \hyperref[TEI.spanGrp]{spanGrp} \hyperref[TEI.w]{w}\par 
    \item[core: ]
   \hyperref[TEI.abbr]{abbr} \hyperref[TEI.add]{add} \hyperref[TEI.addrLine]{addrLine} \hyperref[TEI.address]{address} \hyperref[TEI.author]{author} \hyperref[TEI.bibl]{bibl} \hyperref[TEI.biblScope]{biblScope} \hyperref[TEI.cit]{cit} \hyperref[TEI.citedRange]{citedRange} \hyperref[TEI.corr]{corr} \hyperref[TEI.date]{date} \hyperref[TEI.del]{del} \hyperref[TEI.distinct]{distinct} \hyperref[TEI.editor]{editor} \hyperref[TEI.email]{email} \hyperref[TEI.emph]{emph} \hyperref[TEI.expan]{expan} \hyperref[TEI.foreign]{foreign} \hyperref[TEI.gloss]{gloss} \hyperref[TEI.head]{head} \hyperref[TEI.headItem]{headItem} \hyperref[TEI.headLabel]{headLabel} \hyperref[TEI.hi]{hi} \hyperref[TEI.imprint]{imprint} \hyperref[TEI.item]{item} \hyperref[TEI.l]{l} \hyperref[TEI.label]{label} \hyperref[TEI.lg]{lg} \hyperref[TEI.list]{list} \hyperref[TEI.measure]{measure} \hyperref[TEI.mentioned]{mentioned} \hyperref[TEI.name]{name} \hyperref[TEI.note]{note} \hyperref[TEI.num]{num} \hyperref[TEI.orig]{orig} \hyperref[TEI.p]{p} \hyperref[TEI.pubPlace]{pubPlace} \hyperref[TEI.publisher]{publisher} \hyperref[TEI.q]{q} \hyperref[TEI.quote]{quote} \hyperref[TEI.ref]{ref} \hyperref[TEI.reg]{reg} \hyperref[TEI.resp]{resp} \hyperref[TEI.rs]{rs} \hyperref[TEI.said]{said} \hyperref[TEI.series]{series} \hyperref[TEI.sic]{sic} \hyperref[TEI.soCalled]{soCalled} \hyperref[TEI.sp]{sp} \hyperref[TEI.speaker]{speaker} \hyperref[TEI.stage]{stage} \hyperref[TEI.street]{street} \hyperref[TEI.term]{term} \hyperref[TEI.textLang]{textLang} \hyperref[TEI.time]{time} \hyperref[TEI.title]{title} \hyperref[TEI.unclear]{unclear}\par 
    \item[figures: ]
   \hyperref[TEI.cell]{cell} \hyperref[TEI.figure]{figure} \hyperref[TEI.table]{table}\par 
    \item[header: ]
   \hyperref[TEI.authority]{authority} \hyperref[TEI.change]{change} \hyperref[TEI.classCode]{classCode} \hyperref[TEI.distributor]{distributor} \hyperref[TEI.edition]{edition} \hyperref[TEI.extent]{extent} \hyperref[TEI.funder]{funder} \hyperref[TEI.language]{language} \hyperref[TEI.licence]{licence}\par 
    \item[linking: ]
   \hyperref[TEI.ab]{ab} \hyperref[TEI.seg]{seg}\par 
    \item[msdescription: ]
   \hyperref[TEI.accMat]{accMat} \hyperref[TEI.acquisition]{acquisition} \hyperref[TEI.additions]{additions} \hyperref[TEI.catchwords]{catchwords} \hyperref[TEI.collation]{collation} \hyperref[TEI.colophon]{colophon} \hyperref[TEI.condition]{condition} \hyperref[TEI.custEvent]{custEvent} \hyperref[TEI.decoNote]{decoNote} \hyperref[TEI.explicit]{explicit} \hyperref[TEI.filiation]{filiation} \hyperref[TEI.finalRubric]{finalRubric} \hyperref[TEI.foliation]{foliation} \hyperref[TEI.heraldry]{heraldry} \hyperref[TEI.incipit]{incipit} \hyperref[TEI.layout]{layout} \hyperref[TEI.material]{material} \hyperref[TEI.msItem]{msItem} \hyperref[TEI.musicNotation]{musicNotation} \hyperref[TEI.objectType]{objectType} \hyperref[TEI.origDate]{origDate} \hyperref[TEI.origPlace]{origPlace} \hyperref[TEI.origin]{origin} \hyperref[TEI.provenance]{provenance} \hyperref[TEI.rubric]{rubric} \hyperref[TEI.secFol]{secFol} \hyperref[TEI.signatures]{signatures} \hyperref[TEI.source]{source} \hyperref[TEI.stamp]{stamp} \hyperref[TEI.summary]{summary} \hyperref[TEI.support]{support} \hyperref[TEI.surrogates]{surrogates} \hyperref[TEI.typeNote]{typeNote} \hyperref[TEI.watermark]{watermark}\par 
    \item[namesdates: ]
   \hyperref[TEI.addName]{addName} \hyperref[TEI.affiliation]{affiliation} \hyperref[TEI.country]{country} \hyperref[TEI.forename]{forename} \hyperref[TEI.genName]{genName} \hyperref[TEI.geogName]{geogName} \hyperref[TEI.nameLink]{nameLink} \hyperref[TEI.orgName]{orgName} \hyperref[TEI.persName]{persName} \hyperref[TEI.person]{person} \hyperref[TEI.personGrp]{personGrp} \hyperref[TEI.persona]{persona} \hyperref[TEI.placeName]{placeName} \hyperref[TEI.region]{region} \hyperref[TEI.roleName]{roleName} \hyperref[TEI.settlement]{settlement} \hyperref[TEI.surname]{surname}\par 
    \item[spoken: ]
   \hyperref[TEI.annotationBlock]{annotationBlock}\par 
    \item[standOff: ]
   \hyperref[TEI.listAnnotation]{listAnnotation}\par 
    \item[textstructure: ]
   \hyperref[TEI.back]{back} \hyperref[TEI.body]{body} \hyperref[TEI.div]{div} \hyperref[TEI.docAuthor]{docAuthor} \hyperref[TEI.docDate]{docDate} \hyperref[TEI.docEdition]{docEdition} \hyperref[TEI.docTitle]{docTitle} \hyperref[TEI.floatingText]{floatingText} \hyperref[TEI.front]{front} \hyperref[TEI.group]{group} \hyperref[TEI.text]{text} \hyperref[TEI.titlePage]{titlePage} \hyperref[TEI.titlePart]{titlePart}\par 
    \item[transcr: ]
   \hyperref[TEI.damage]{damage} \hyperref[TEI.fw]{fw} \hyperref[TEI.line]{line} \hyperref[TEI.metamark]{metamark} \hyperref[TEI.mod]{mod} \hyperref[TEI.restore]{restore} \hyperref[TEI.retrace]{retrace} \hyperref[TEI.secl]{secl} \hyperref[TEI.sourceDoc]{sourceDoc} \hyperref[TEI.supplied]{supplied} \hyperref[TEI.surface]{surface} \hyperref[TEI.surfaceGrp]{surfaceGrp} \hyperref[TEI.surplus]{surplus} \hyperref[TEI.zone]{zone}
    \item[{Peut contenir}]
  
    \item[analysis: ]
   \hyperref[TEI.interp]{interp} \hyperref[TEI.interpGrp]{interpGrp} \hyperref[TEI.span]{span} \hyperref[TEI.spanGrp]{spanGrp}\par 
    \item[core: ]
   \hyperref[TEI.abbr]{abbr} \hyperref[TEI.address]{address} \hyperref[TEI.cb]{cb} \hyperref[TEI.choice]{choice} \hyperref[TEI.date]{date} \hyperref[TEI.distinct]{distinct} \hyperref[TEI.email]{email} \hyperref[TEI.emph]{emph} \hyperref[TEI.expan]{expan} \hyperref[TEI.foreign]{foreign} \hyperref[TEI.gap]{gap} \hyperref[TEI.gb]{gb} \hyperref[TEI.gloss]{gloss} \hyperref[TEI.hi]{hi} \hyperref[TEI.index]{index} \hyperref[TEI.lb]{lb} \hyperref[TEI.measure]{measure} \hyperref[TEI.measureGrp]{measureGrp} \hyperref[TEI.mentioned]{mentioned} \hyperref[TEI.milestone]{milestone} \hyperref[TEI.name]{name} \hyperref[TEI.note]{note} \hyperref[TEI.num]{num} \hyperref[TEI.pb]{pb} \hyperref[TEI.ptr]{ptr} \hyperref[TEI.ref]{ref} \hyperref[TEI.rs]{rs} \hyperref[TEI.soCalled]{soCalled} \hyperref[TEI.term]{term} \hyperref[TEI.time]{time} \hyperref[TEI.title]{title}\par 
    \item[figures: ]
   \hyperref[TEI.figure]{figure} \hyperref[TEI.notatedMusic]{notatedMusic}\par 
    \item[header: ]
   \hyperref[TEI.idno]{idno}\par 
    \item[iso-fs: ]
   \hyperref[TEI.fLib]{fLib} \hyperref[TEI.fs]{fs} \hyperref[TEI.fvLib]{fvLib}\par 
    \item[linking: ]
   \hyperref[TEI.alt]{alt} \hyperref[TEI.altGrp]{altGrp} \hyperref[TEI.anchor]{anchor} \hyperref[TEI.join]{join} \hyperref[TEI.joinGrp]{joinGrp} \hyperref[TEI.link]{link} \hyperref[TEI.linkGrp]{linkGrp} \hyperref[TEI.timeline]{timeline}\par 
    \item[msdescription: ]
   \hyperref[TEI.catchwords]{catchwords} \hyperref[TEI.depth]{depth} \hyperref[TEI.dim]{dim} \hyperref[TEI.dimensions]{dimensions} \hyperref[TEI.height]{height} \hyperref[TEI.heraldry]{heraldry} \hyperref[TEI.locus]{locus} \hyperref[TEI.locusGrp]{locusGrp} \hyperref[TEI.material]{material} \hyperref[TEI.objectType]{objectType} \hyperref[TEI.origDate]{origDate} \hyperref[TEI.origPlace]{origPlace} \hyperref[TEI.secFol]{secFol} \hyperref[TEI.signatures]{signatures} \hyperref[TEI.source]{source} \hyperref[TEI.stamp]{stamp} \hyperref[TEI.watermark]{watermark} \hyperref[TEI.width]{width}\par 
    \item[namesdates: ]
   \hyperref[TEI.addName]{addName} \hyperref[TEI.affiliation]{affiliation} \hyperref[TEI.country]{country} \hyperref[TEI.forename]{forename} \hyperref[TEI.genName]{genName} \hyperref[TEI.geogName]{geogName} \hyperref[TEI.location]{location} \hyperref[TEI.nameLink]{nameLink} \hyperref[TEI.orgName]{orgName} \hyperref[TEI.persName]{persName} \hyperref[TEI.placeName]{placeName} \hyperref[TEI.region]{region} \hyperref[TEI.roleName]{roleName} \hyperref[TEI.settlement]{settlement} \hyperref[TEI.state]{state} \hyperref[TEI.surname]{surname}\par 
    \item[transcr: ]
   \hyperref[TEI.addSpan]{addSpan} \hyperref[TEI.am]{am} \hyperref[TEI.damageSpan]{damageSpan} \hyperref[TEI.delSpan]{delSpan} \hyperref[TEI.ex]{ex} \hyperref[TEI.fw]{fw} \hyperref[TEI.listTranspose]{listTranspose} \hyperref[TEI.metamark]{metamark} \hyperref[TEI.space]{space} \hyperref[TEI.subst]{subst} \hyperref[TEI.substJoin]{substJoin}\par des données textuelles
    \item[{Exemple}]
  \leavevmode\bgroup\exampleFont \begin{shaded}\noindent\mbox{}{<\textbf{p}\hspace*{6pt}{xml:id}="{fr\textunderscore para2}">}(la "conséquence" commence ici){</\textbf{p}>}\mbox{}\newline 
{<\textbf{p}\hspace*{6pt}{xml:id}="{fr\textunderscore para3}">}(la "conséquence" continue ici){</\textbf{p}>}\mbox{}\newline 
{<\textbf{p}\hspace*{6pt}{xml:id}="{fr\textunderscore para4}">}(la "conséquence" se termine dans ce paragraphe){</\textbf{p}>}\mbox{}\newline 
{<\textbf{span}\hspace*{6pt}{from}="{\#fr\textunderscore para2}"\hspace*{6pt}{to}="{\#fr\textunderscore para4}"\mbox{}\newline 
\hspace*{6pt}{type}="{structure}">}conséquence{</\textbf{span}>}\end{shaded}\egroup 


    \item[{Schematron}]
   <s:report test="@from and @target">Only one of the attributes @target and @from may be supplied on <s:name/> </s:report>
    \item[{Schematron}]
   <s:report test="@to and @target">Only one of the attributes @target and @to may be supplied on <s:name/> </s:report>
    \item[{Schematron}]
   <s:report test="@to and not(@from)">If @to is supplied on <s:name/>, @from must be supplied as well</s:report>
    \item[{Schematron}]
   <s:report test="contains(normalize-space(@to),' ') or contains(normalize-space(@from),'   ')">The attributes @to and @from on <s:name/> may each contain only a single value</s:report>
    \item[{Modèle de contenu}]
  \mbox{}\hfill\\[-10pt]\begin{Verbatim}[fontsize=\small]
<content>
 <macroRef key="macro.phraseSeq.limited"/>
</content>
    
\end{Verbatim}

    \item[{Schéma Declaration}]
  \mbox{}\hfill\\[-10pt]\begin{Verbatim}[fontsize=\small]
element span
{
   tei_att.global.attributes,
   tei_att.interpLike.attributes,
   tei_att.pointing.attributes,
   attribute from { text }?,
   attribute to { text }?,
   tei_macro.phraseSeq.limited}
\end{Verbatim}

\end{reflist}  \index{spanGrp=<spanGrp>|oddindex}
\begin{reflist}
\item[]\begin{specHead}{TEI.spanGrp}{<spanGrp> }(groupement de fragments de texte) regroupe des éléments \hyperref[TEI.span]{<span>} [\xref{http://www.tei-c.org/release/doc/tei-p5-doc/en/html/AI.html\#AISP}{17.3. Spans and Interpretations}]\end{specHead} 
    \item[{Module}]
  analysis
    \item[{Attributs}]
  Attributs \hyperref[TEI.att.global]{att.global} (\textit{@xml:id}, \textit{@n}, \textit{@xml:lang}, \textit{@xml:base}, \textit{@xml:space})  (\hyperref[TEI.att.global.rendition]{att.global.rendition} (\textit{@rend}, \textit{@style}, \textit{@rendition})) (\hyperref[TEI.att.global.linking]{att.global.linking} (\textit{@corresp}, \textit{@synch}, \textit{@sameAs}, \textit{@copyOf}, \textit{@next}, \textit{@prev}, \textit{@exclude}, \textit{@select})) (\hyperref[TEI.att.global.analytic]{att.global.analytic} (\textit{@ana})) (\hyperref[TEI.att.global.facs]{att.global.facs} (\textit{@facs})) (\hyperref[TEI.att.global.change]{att.global.change} (\textit{@change})) (\hyperref[TEI.att.global.responsibility]{att.global.responsibility} (\textit{@cert}, \textit{@resp})) (\hyperref[TEI.att.global.source]{att.global.source} (\textit{@source})) \hyperref[TEI.att.interpLike]{att.interpLike} (\textit{@type}, \textit{@inst}) 
    \item[{Membre du}]
  \hyperref[TEI.model.global.meta]{model.global.meta}
    \item[{Contenu dans}]
  
    \item[analysis: ]
   \hyperref[TEI.cl]{cl} \hyperref[TEI.m]{m} \hyperref[TEI.phr]{phr} \hyperref[TEI.s]{s} \hyperref[TEI.span]{span} \hyperref[TEI.w]{w}\par 
    \item[core: ]
   \hyperref[TEI.abbr]{abbr} \hyperref[TEI.add]{add} \hyperref[TEI.addrLine]{addrLine} \hyperref[TEI.address]{address} \hyperref[TEI.author]{author} \hyperref[TEI.bibl]{bibl} \hyperref[TEI.biblScope]{biblScope} \hyperref[TEI.cit]{cit} \hyperref[TEI.citedRange]{citedRange} \hyperref[TEI.corr]{corr} \hyperref[TEI.date]{date} \hyperref[TEI.del]{del} \hyperref[TEI.distinct]{distinct} \hyperref[TEI.editor]{editor} \hyperref[TEI.email]{email} \hyperref[TEI.emph]{emph} \hyperref[TEI.expan]{expan} \hyperref[TEI.foreign]{foreign} \hyperref[TEI.gloss]{gloss} \hyperref[TEI.head]{head} \hyperref[TEI.headItem]{headItem} \hyperref[TEI.headLabel]{headLabel} \hyperref[TEI.hi]{hi} \hyperref[TEI.imprint]{imprint} \hyperref[TEI.item]{item} \hyperref[TEI.l]{l} \hyperref[TEI.label]{label} \hyperref[TEI.lg]{lg} \hyperref[TEI.list]{list} \hyperref[TEI.measure]{measure} \hyperref[TEI.mentioned]{mentioned} \hyperref[TEI.name]{name} \hyperref[TEI.note]{note} \hyperref[TEI.num]{num} \hyperref[TEI.orig]{orig} \hyperref[TEI.p]{p} \hyperref[TEI.pubPlace]{pubPlace} \hyperref[TEI.publisher]{publisher} \hyperref[TEI.q]{q} \hyperref[TEI.quote]{quote} \hyperref[TEI.ref]{ref} \hyperref[TEI.reg]{reg} \hyperref[TEI.resp]{resp} \hyperref[TEI.rs]{rs} \hyperref[TEI.said]{said} \hyperref[TEI.series]{series} \hyperref[TEI.sic]{sic} \hyperref[TEI.soCalled]{soCalled} \hyperref[TEI.sp]{sp} \hyperref[TEI.speaker]{speaker} \hyperref[TEI.stage]{stage} \hyperref[TEI.street]{street} \hyperref[TEI.term]{term} \hyperref[TEI.textLang]{textLang} \hyperref[TEI.time]{time} \hyperref[TEI.title]{title} \hyperref[TEI.unclear]{unclear}\par 
    \item[figures: ]
   \hyperref[TEI.cell]{cell} \hyperref[TEI.figure]{figure} \hyperref[TEI.table]{table}\par 
    \item[header: ]
   \hyperref[TEI.authority]{authority} \hyperref[TEI.change]{change} \hyperref[TEI.classCode]{classCode} \hyperref[TEI.distributor]{distributor} \hyperref[TEI.edition]{edition} \hyperref[TEI.extent]{extent} \hyperref[TEI.funder]{funder} \hyperref[TEI.language]{language} \hyperref[TEI.licence]{licence}\par 
    \item[linking: ]
   \hyperref[TEI.ab]{ab} \hyperref[TEI.seg]{seg}\par 
    \item[msdescription: ]
   \hyperref[TEI.accMat]{accMat} \hyperref[TEI.acquisition]{acquisition} \hyperref[TEI.additions]{additions} \hyperref[TEI.catchwords]{catchwords} \hyperref[TEI.collation]{collation} \hyperref[TEI.colophon]{colophon} \hyperref[TEI.condition]{condition} \hyperref[TEI.custEvent]{custEvent} \hyperref[TEI.decoNote]{decoNote} \hyperref[TEI.explicit]{explicit} \hyperref[TEI.filiation]{filiation} \hyperref[TEI.finalRubric]{finalRubric} \hyperref[TEI.foliation]{foliation} \hyperref[TEI.heraldry]{heraldry} \hyperref[TEI.incipit]{incipit} \hyperref[TEI.layout]{layout} \hyperref[TEI.material]{material} \hyperref[TEI.msItem]{msItem} \hyperref[TEI.musicNotation]{musicNotation} \hyperref[TEI.objectType]{objectType} \hyperref[TEI.origDate]{origDate} \hyperref[TEI.origPlace]{origPlace} \hyperref[TEI.origin]{origin} \hyperref[TEI.provenance]{provenance} \hyperref[TEI.rubric]{rubric} \hyperref[TEI.secFol]{secFol} \hyperref[TEI.signatures]{signatures} \hyperref[TEI.source]{source} \hyperref[TEI.stamp]{stamp} \hyperref[TEI.summary]{summary} \hyperref[TEI.support]{support} \hyperref[TEI.surrogates]{surrogates} \hyperref[TEI.typeNote]{typeNote} \hyperref[TEI.watermark]{watermark}\par 
    \item[namesdates: ]
   \hyperref[TEI.addName]{addName} \hyperref[TEI.affiliation]{affiliation} \hyperref[TEI.country]{country} \hyperref[TEI.forename]{forename} \hyperref[TEI.genName]{genName} \hyperref[TEI.geogName]{geogName} \hyperref[TEI.nameLink]{nameLink} \hyperref[TEI.orgName]{orgName} \hyperref[TEI.persName]{persName} \hyperref[TEI.person]{person} \hyperref[TEI.personGrp]{personGrp} \hyperref[TEI.persona]{persona} \hyperref[TEI.placeName]{placeName} \hyperref[TEI.region]{region} \hyperref[TEI.roleName]{roleName} \hyperref[TEI.settlement]{settlement} \hyperref[TEI.surname]{surname}\par 
    \item[spoken: ]
   \hyperref[TEI.annotationBlock]{annotationBlock}\par 
    \item[standOff: ]
   \hyperref[TEI.listAnnotation]{listAnnotation}\par 
    \item[textstructure: ]
   \hyperref[TEI.back]{back} \hyperref[TEI.body]{body} \hyperref[TEI.div]{div} \hyperref[TEI.docAuthor]{docAuthor} \hyperref[TEI.docDate]{docDate} \hyperref[TEI.docEdition]{docEdition} \hyperref[TEI.docTitle]{docTitle} \hyperref[TEI.floatingText]{floatingText} \hyperref[TEI.front]{front} \hyperref[TEI.group]{group} \hyperref[TEI.text]{text} \hyperref[TEI.titlePage]{titlePage} \hyperref[TEI.titlePart]{titlePart}\par 
    \item[transcr: ]
   \hyperref[TEI.damage]{damage} \hyperref[TEI.fw]{fw} \hyperref[TEI.line]{line} \hyperref[TEI.metamark]{metamark} \hyperref[TEI.mod]{mod} \hyperref[TEI.restore]{restore} \hyperref[TEI.retrace]{retrace} \hyperref[TEI.secl]{secl} \hyperref[TEI.sourceDoc]{sourceDoc} \hyperref[TEI.supplied]{supplied} \hyperref[TEI.surface]{surface} \hyperref[TEI.surfaceGrp]{surfaceGrp} \hyperref[TEI.surplus]{surplus} \hyperref[TEI.zone]{zone}
    \item[{Peut contenir}]
  
    \item[analysis: ]
   \hyperref[TEI.span]{span}
    \item[{Exemple}]
  \leavevmode\bgroup\exampleFont \begin{shaded}\noindent\mbox{}{<\textbf{u}\hspace*{6pt}{xml:id}="{UU1}">}Can I have ten oranges and a kilo of bananas please?{</\textbf{u}>}\mbox{}\newline 
{<\textbf{u}\hspace*{6pt}{xml:id}="{UU2}">}Yes, anything else?{</\textbf{u}>}\mbox{}\newline 
{<\textbf{u}\hspace*{6pt}{xml:id}="{UU3}">}No thanks.{</\textbf{u}>}\mbox{}\newline 
{<\textbf{u}\hspace*{6pt}{xml:id}="{UU4}">}That'll be dollar forty.{</\textbf{u}>}\mbox{}\newline 
{<\textbf{u}\hspace*{6pt}{xml:id}="{UU5}">}Two dollars{</\textbf{u}>}\mbox{}\newline 
{<\textbf{u}\hspace*{6pt}{xml:id}="{UU6}">}Sixty, eighty, two dollars.\mbox{}\newline 
{<\textbf{anchor}\hspace*{6pt}{xml:id}="{UU6e}"/>}Thank you.{<\textbf{anchor}\hspace*{6pt}{xml:id}="{UU6f}"/>}\mbox{}\newline 
{</\textbf{u}>}\mbox{}\newline 
{<\textbf{spanGrp}\hspace*{6pt}{type}="{transactions}">}\mbox{}\newline 
\hspace*{6pt}{<\textbf{span}\hspace*{6pt}{from}="{\#UU1}">}sale request{</\textbf{span}>}\mbox{}\newline 
\hspace*{6pt}{<\textbf{span}\hspace*{6pt}{from}="{\#UU2}"\hspace*{6pt}{to}="{\#UU3}">}sale compliance{</\textbf{span}>}\mbox{}\newline 
\hspace*{6pt}{<\textbf{span}\hspace*{6pt}{from}="{\#UU4}">}sale{</\textbf{span}>}\mbox{}\newline 
\hspace*{6pt}{<\textbf{span}\hspace*{6pt}{from}="{\#UU5}"\hspace*{6pt}{to}="{\#UU6}">}purchase{</\textbf{span}>}\mbox{}\newline 
\hspace*{6pt}{<\textbf{span}\hspace*{6pt}{from}="{\#UU6e}"\hspace*{6pt}{to}="{\#UU6f}">}purchase closure{</\textbf{span}>}\mbox{}\newline 
{</\textbf{spanGrp}>}\end{shaded}\egroup 


    \item[{Modèle de contenu}]
  \mbox{}\hfill\\[-10pt]\begin{Verbatim}[fontsize=\small]
<content>
 <elementRef key="span"
  maxOccurs="unbounded" minOccurs="0"/>
</content>
    
\end{Verbatim}

    \item[{Schéma Declaration}]
  \mbox{}\hfill\\[-10pt]\begin{Verbatim}[fontsize=\small]
element spanGrp
{
   tei_att.global.attributes,
   tei_att.interpLike.attributes,
   tei_span*
}
\end{Verbatim}

\end{reflist}  \index{speaker=<speaker>|oddindex}
\begin{reflist}
\item[]\begin{specHead}{TEI.speaker}{<speaker> }forme particulière de titre ou de marque qui donne le nom d'un ou de plusieurs locuteurs dans un texte ou dans un fragment de texte écrit pour le théâtre. [\xref{http://www.tei-c.org/release/doc/tei-p5-doc/en/html/CO.html\#CODR}{3.12.2. Core Tags for Drama}]\end{specHead} 
    \item[{Module}]
  core
    \item[{Attributs}]
  Attributs \hyperref[TEI.att.global]{att.global} (\textit{@xml:id}, \textit{@n}, \textit{@xml:lang}, \textit{@xml:base}, \textit{@xml:space})  (\hyperref[TEI.att.global.rendition]{att.global.rendition} (\textit{@rend}, \textit{@style}, \textit{@rendition})) (\hyperref[TEI.att.global.linking]{att.global.linking} (\textit{@corresp}, \textit{@synch}, \textit{@sameAs}, \textit{@copyOf}, \textit{@next}, \textit{@prev}, \textit{@exclude}, \textit{@select})) (\hyperref[TEI.att.global.analytic]{att.global.analytic} (\textit{@ana})) (\hyperref[TEI.att.global.facs]{att.global.facs} (\textit{@facs})) (\hyperref[TEI.att.global.change]{att.global.change} (\textit{@change})) (\hyperref[TEI.att.global.responsibility]{att.global.responsibility} (\textit{@cert}, \textit{@resp})) (\hyperref[TEI.att.global.source]{att.global.source} (\textit{@source}))
    \item[{Contenu dans}]
  
    \item[core: ]
   \hyperref[TEI.sp]{sp}
    \item[{Peut contenir}]
  
    \item[analysis: ]
   \hyperref[TEI.c]{c} \hyperref[TEI.cl]{cl} \hyperref[TEI.interp]{interp} \hyperref[TEI.interpGrp]{interpGrp} \hyperref[TEI.m]{m} \hyperref[TEI.pc]{pc} \hyperref[TEI.phr]{phr} \hyperref[TEI.s]{s} \hyperref[TEI.span]{span} \hyperref[TEI.spanGrp]{spanGrp} \hyperref[TEI.w]{w}\par 
    \item[core: ]
   \hyperref[TEI.abbr]{abbr} \hyperref[TEI.add]{add} \hyperref[TEI.address]{address} \hyperref[TEI.binaryObject]{binaryObject} \hyperref[TEI.cb]{cb} \hyperref[TEI.choice]{choice} \hyperref[TEI.corr]{corr} \hyperref[TEI.date]{date} \hyperref[TEI.del]{del} \hyperref[TEI.distinct]{distinct} \hyperref[TEI.email]{email} \hyperref[TEI.emph]{emph} \hyperref[TEI.expan]{expan} \hyperref[TEI.foreign]{foreign} \hyperref[TEI.gap]{gap} \hyperref[TEI.gb]{gb} \hyperref[TEI.gloss]{gloss} \hyperref[TEI.graphic]{graphic} \hyperref[TEI.hi]{hi} \hyperref[TEI.index]{index} \hyperref[TEI.lb]{lb} \hyperref[TEI.measure]{measure} \hyperref[TEI.measureGrp]{measureGrp} \hyperref[TEI.media]{media} \hyperref[TEI.mentioned]{mentioned} \hyperref[TEI.milestone]{milestone} \hyperref[TEI.name]{name} \hyperref[TEI.note]{note} \hyperref[TEI.num]{num} \hyperref[TEI.orig]{orig} \hyperref[TEI.pb]{pb} \hyperref[TEI.ptr]{ptr} \hyperref[TEI.ref]{ref} \hyperref[TEI.reg]{reg} \hyperref[TEI.rs]{rs} \hyperref[TEI.sic]{sic} \hyperref[TEI.soCalled]{soCalled} \hyperref[TEI.term]{term} \hyperref[TEI.time]{time} \hyperref[TEI.title]{title} \hyperref[TEI.unclear]{unclear}\par 
    \item[derived-module-tei.istex: ]
   \hyperref[TEI.math]{math} \hyperref[TEI.mrow]{mrow}\par 
    \item[figures: ]
   \hyperref[TEI.figure]{figure} \hyperref[TEI.formula]{formula} \hyperref[TEI.notatedMusic]{notatedMusic}\par 
    \item[header: ]
   \hyperref[TEI.idno]{idno}\par 
    \item[iso-fs: ]
   \hyperref[TEI.fLib]{fLib} \hyperref[TEI.fs]{fs} \hyperref[TEI.fvLib]{fvLib}\par 
    \item[linking: ]
   \hyperref[TEI.alt]{alt} \hyperref[TEI.altGrp]{altGrp} \hyperref[TEI.anchor]{anchor} \hyperref[TEI.join]{join} \hyperref[TEI.joinGrp]{joinGrp} \hyperref[TEI.link]{link} \hyperref[TEI.linkGrp]{linkGrp} \hyperref[TEI.seg]{seg} \hyperref[TEI.timeline]{timeline}\par 
    \item[msdescription: ]
   \hyperref[TEI.catchwords]{catchwords} \hyperref[TEI.depth]{depth} \hyperref[TEI.dim]{dim} \hyperref[TEI.dimensions]{dimensions} \hyperref[TEI.height]{height} \hyperref[TEI.heraldry]{heraldry} \hyperref[TEI.locus]{locus} \hyperref[TEI.locusGrp]{locusGrp} \hyperref[TEI.material]{material} \hyperref[TEI.objectType]{objectType} \hyperref[TEI.origDate]{origDate} \hyperref[TEI.origPlace]{origPlace} \hyperref[TEI.secFol]{secFol} \hyperref[TEI.signatures]{signatures} \hyperref[TEI.source]{source} \hyperref[TEI.stamp]{stamp} \hyperref[TEI.watermark]{watermark} \hyperref[TEI.width]{width}\par 
    \item[namesdates: ]
   \hyperref[TEI.addName]{addName} \hyperref[TEI.affiliation]{affiliation} \hyperref[TEI.country]{country} \hyperref[TEI.forename]{forename} \hyperref[TEI.genName]{genName} \hyperref[TEI.geogName]{geogName} \hyperref[TEI.location]{location} \hyperref[TEI.nameLink]{nameLink} \hyperref[TEI.orgName]{orgName} \hyperref[TEI.persName]{persName} \hyperref[TEI.placeName]{placeName} \hyperref[TEI.region]{region} \hyperref[TEI.roleName]{roleName} \hyperref[TEI.settlement]{settlement} \hyperref[TEI.state]{state} \hyperref[TEI.surname]{surname}\par 
    \item[spoken: ]
   \hyperref[TEI.annotationBlock]{annotationBlock}\par 
    \item[transcr: ]
   \hyperref[TEI.addSpan]{addSpan} \hyperref[TEI.am]{am} \hyperref[TEI.damage]{damage} \hyperref[TEI.damageSpan]{damageSpan} \hyperref[TEI.delSpan]{delSpan} \hyperref[TEI.ex]{ex} \hyperref[TEI.fw]{fw} \hyperref[TEI.handShift]{handShift} \hyperref[TEI.listTranspose]{listTranspose} \hyperref[TEI.metamark]{metamark} \hyperref[TEI.mod]{mod} \hyperref[TEI.redo]{redo} \hyperref[TEI.restore]{restore} \hyperref[TEI.retrace]{retrace} \hyperref[TEI.secl]{secl} \hyperref[TEI.space]{space} \hyperref[TEI.subst]{subst} \hyperref[TEI.substJoin]{substJoin} \hyperref[TEI.supplied]{supplied} \hyperref[TEI.surplus]{surplus} \hyperref[TEI.undo]{undo}\par des données textuelles
    \item[{Note}]
  \par
Cet élément est utilisé pour indiquer quel personnage prend la parole dans une pièce de théâtre ; l'attribut who est utilisé pour pointer vers un autre élément qui fournit des informations sur ce personnage. L'un et ou l'autre peuvent être utilisés.
    \item[{Exemple}]
  \leavevmode\bgroup\exampleFont \begin{shaded}\noindent\mbox{}{<\textbf{sp}\hspace*{6pt}{who}="{ko}">}\mbox{}\newline 
\hspace*{6pt}{<\textbf{speaker}>}Koch.{</\textbf{speaker}>}\mbox{}\newline 
\hspace*{6pt}{<\textbf{p}>}Ne risquez rien du tout, Monique ; rentrez.{</\textbf{p}>}\mbox{}\newline 
{</\textbf{sp}>}\mbox{}\newline 
{<\textbf{sp}\hspace*{6pt}{who}="{mo}">}\mbox{}\newline 
\hspace*{6pt}{<\textbf{speaker}>}Monique.{</\textbf{speaker}>}\mbox{}\newline 
\hspace*{6pt}{<\textbf{p}>}Rentrer ? comment voulez-vous que je rentre ? J'ai les clés de la voiture.{</\textbf{p}>}\mbox{}\newline 
{</\textbf{sp}>}\mbox{}\newline 
{<\textbf{sp}\hspace*{6pt}{who}="{ko}">}\mbox{}\newline 
\hspace*{6pt}{<\textbf{speaker}>}Koch.{</\textbf{speaker}>}\mbox{}\newline 
\hspace*{6pt}{<\textbf{p}>} Je rentrerai par mes propres moyens. {</\textbf{p}>}\mbox{}\newline 
{</\textbf{sp}>}\mbox{}\newline 
{<\textbf{sp}\hspace*{6pt}{who}="{mo}">}\mbox{}\newline 
\hspace*{6pt}{<\textbf{speaker}>}Monique.{</\textbf{speaker}>}\mbox{}\newline 
\hspace*{6pt}{<\textbf{p}>} Vous ? vos moyens ? quels moyens ? Seigneur ! Vous ne savez même pas conduire, vous ne\mbox{}\newline 
\hspace*{6pt}\hspace*{6pt} savez pas reconnaître votre gauche de votre droite, vous auriez été incapable de\mbox{}\newline 
\hspace*{6pt}\hspace*{6pt} retrouver ce fichu quartier tout seul, vous ne savez absolument rien faire tout seul. Je\mbox{}\newline 
\hspace*{6pt}\hspace*{6pt} me demande bien comment vous pourriez rentrer. {</\textbf{p}>}\mbox{}\newline 
{</\textbf{sp}>}\mbox{}\newline 
{<\textbf{sp}\hspace*{6pt}{who}="{ko}">}\mbox{}\newline 
\hspace*{6pt}{<\textbf{speaker}>}Koch.{</\textbf{speaker}>}\mbox{}\newline 
\hspace*{6pt}{<\textbf{p}>}J'appellerai un taxi.{</\textbf{p}>}\mbox{}\newline 
{</\textbf{sp}>}\mbox{}\newline 
{<\textbf{list}\hspace*{6pt}{type}="{speakers}">}\mbox{}\newline 
\hspace*{6pt}{<\textbf{item}\hspace*{6pt}{xml:id}="{fr\textunderscore mo}"/>}\mbox{}\newline 
\hspace*{6pt}{<\textbf{item}\hspace*{6pt}{xml:id}="{fr\textunderscore ko}"/>}\mbox{}\newline 
{</\textbf{list}>}\end{shaded}\egroup 


    \item[{Modèle de contenu}]
  \mbox{}\hfill\\[-10pt]\begin{Verbatim}[fontsize=\small]
<content>
 <macroRef key="macro.phraseSeq"/>
</content>
    
\end{Verbatim}

    \item[{Schéma Declaration}]
  \mbox{}\hfill\\[-10pt]\begin{Verbatim}[fontsize=\small]
element speaker { tei_att.global.attributes, tei_macro.phraseSeq }
\end{Verbatim}

\end{reflist}  \index{stage=<stage>|oddindex}\index{type=@type!<stage>|oddindex}
\begin{reflist}
\item[]\begin{specHead}{TEI.stage}{<stage> }(indication scénique) contient tout type d'indication scénique à l'intérieur d'un texte ou fragment de texte écrit pour le théâtre. [\xref{http://www.tei-c.org/release/doc/tei-p5-doc/en/html/CO.html\#CODR}{3.12.2. Core Tags for Drama} \xref{http://www.tei-c.org/release/doc/tei-p5-doc/en/html/CO.html\#CODV}{3.12. Passages of Verse or Drama} \xref{http://www.tei-c.org/release/doc/tei-p5-doc/en/html/DR.html\#DRSTA}{7.2.4. Stage Directions}]\end{specHead} 
    \item[{Module}]
  core
    \item[{Attributs}]
  Attributs \hyperref[TEI.att.ascribed]{att.ascribed} (\textit{@who}) \hyperref[TEI.att.global]{att.global} (\textit{@xml:id}, \textit{@n}, \textit{@xml:lang}, \textit{@xml:base}, \textit{@xml:space})  (\hyperref[TEI.att.global.rendition]{att.global.rendition} (\textit{@rend}, \textit{@style}, \textit{@rendition})) (\hyperref[TEI.att.global.linking]{att.global.linking} (\textit{@corresp}, \textit{@synch}, \textit{@sameAs}, \textit{@copyOf}, \textit{@next}, \textit{@prev}, \textit{@exclude}, \textit{@select})) (\hyperref[TEI.att.global.analytic]{att.global.analytic} (\textit{@ana})) (\hyperref[TEI.att.global.facs]{att.global.facs} (\textit{@facs})) (\hyperref[TEI.att.global.change]{att.global.change} (\textit{@change})) (\hyperref[TEI.att.global.responsibility]{att.global.responsibility} (\textit{@cert}, \textit{@resp})) (\hyperref[TEI.att.global.source]{att.global.source} (\textit{@source})) \hyperref[TEI.att.placement]{att.placement} (\textit{@place}) \hfil\\[-10pt]\begin{sansreflist}
    \item[@type]
  indique le type d'indication scénique
\begin{reflist}
    \item[{Statut}]
  Recommendé
    \item[{Type de données}]
  0–∞ occurrences de \hyperref[TEI.teidata.enumerated]{teidata.enumerated} séparé par un espace
    \item[{Les valeurs suggérées comprennent:}]
  \begin{description}

\item[{setting}]décrit une mise en scène
\item[{entrance}]décrit une entrée
\item[{exit}]décrit une sortie
\item[{business}]décrit une action sur scène
\item[{novelistic}]texte explicatif de la direction de scène.
\item[{delivery}]décrit la façon dont parle un personnage
\item[{modifier}]donne certains détails à propos d'un personnage
\item[{location}]décrit un lieu
\item[{mixed}]plusieurs des indications précédentes
\end{description} 
\end{reflist}  
\end{sansreflist}  
    \item[{Membre du}]
  \hyperref[TEI.model.stageLike]{model.stageLike}
    \item[{Contenu dans}]
  
    \item[core: ]
   \hyperref[TEI.add]{add} \hyperref[TEI.corr]{corr} \hyperref[TEI.del]{del} \hyperref[TEI.desc]{desc} \hyperref[TEI.emph]{emph} \hyperref[TEI.head]{head} \hyperref[TEI.hi]{hi} \hyperref[TEI.item]{item} \hyperref[TEI.l]{l} \hyperref[TEI.lg]{lg} \hyperref[TEI.meeting]{meeting} \hyperref[TEI.note]{note} \hyperref[TEI.orig]{orig} \hyperref[TEI.p]{p} \hyperref[TEI.q]{q} \hyperref[TEI.quote]{quote} \hyperref[TEI.ref]{ref} \hyperref[TEI.reg]{reg} \hyperref[TEI.said]{said} \hyperref[TEI.sic]{sic} \hyperref[TEI.sp]{sp} \hyperref[TEI.stage]{stage} \hyperref[TEI.title]{title} \hyperref[TEI.unclear]{unclear}\par 
    \item[figures: ]
   \hyperref[TEI.cell]{cell} \hyperref[TEI.figDesc]{figDesc} \hyperref[TEI.figure]{figure}\par 
    \item[header: ]
   \hyperref[TEI.change]{change} \hyperref[TEI.licence]{licence} \hyperref[TEI.rendition]{rendition}\par 
    \item[iso-fs: ]
   \hyperref[TEI.fDescr]{fDescr} \hyperref[TEI.fsDescr]{fsDescr}\par 
    \item[linking: ]
   \hyperref[TEI.ab]{ab} \hyperref[TEI.seg]{seg}\par 
    \item[msdescription: ]
   \hyperref[TEI.accMat]{accMat} \hyperref[TEI.acquisition]{acquisition} \hyperref[TEI.additions]{additions} \hyperref[TEI.collation]{collation} \hyperref[TEI.condition]{condition} \hyperref[TEI.custEvent]{custEvent} \hyperref[TEI.decoNote]{decoNote} \hyperref[TEI.filiation]{filiation} \hyperref[TEI.foliation]{foliation} \hyperref[TEI.layout]{layout} \hyperref[TEI.musicNotation]{musicNotation} \hyperref[TEI.origin]{origin} \hyperref[TEI.provenance]{provenance} \hyperref[TEI.signatures]{signatures} \hyperref[TEI.source]{source} \hyperref[TEI.summary]{summary} \hyperref[TEI.support]{support} \hyperref[TEI.surrogates]{surrogates} \hyperref[TEI.typeNote]{typeNote}\par 
    \item[textstructure: ]
   \hyperref[TEI.body]{body} \hyperref[TEI.div]{div} \hyperref[TEI.docEdition]{docEdition} \hyperref[TEI.titlePart]{titlePart}\par 
    \item[transcr: ]
   \hyperref[TEI.damage]{damage} \hyperref[TEI.metamark]{metamark} \hyperref[TEI.mod]{mod} \hyperref[TEI.restore]{restore} \hyperref[TEI.retrace]{retrace} \hyperref[TEI.secl]{secl} \hyperref[TEI.supplied]{supplied} \hyperref[TEI.surplus]{surplus}
    \item[{Peut contenir}]
  
    \item[analysis: ]
   \hyperref[TEI.c]{c} \hyperref[TEI.cl]{cl} \hyperref[TEI.interp]{interp} \hyperref[TEI.interpGrp]{interpGrp} \hyperref[TEI.m]{m} \hyperref[TEI.pc]{pc} \hyperref[TEI.phr]{phr} \hyperref[TEI.s]{s} \hyperref[TEI.span]{span} \hyperref[TEI.spanGrp]{spanGrp} \hyperref[TEI.w]{w}\par 
    \item[core: ]
   \hyperref[TEI.abbr]{abbr} \hyperref[TEI.add]{add} \hyperref[TEI.address]{address} \hyperref[TEI.bibl]{bibl} \hyperref[TEI.biblStruct]{biblStruct} \hyperref[TEI.binaryObject]{binaryObject} \hyperref[TEI.cb]{cb} \hyperref[TEI.choice]{choice} \hyperref[TEI.cit]{cit} \hyperref[TEI.corr]{corr} \hyperref[TEI.date]{date} \hyperref[TEI.del]{del} \hyperref[TEI.desc]{desc} \hyperref[TEI.distinct]{distinct} \hyperref[TEI.email]{email} \hyperref[TEI.emph]{emph} \hyperref[TEI.expan]{expan} \hyperref[TEI.foreign]{foreign} \hyperref[TEI.gap]{gap} \hyperref[TEI.gb]{gb} \hyperref[TEI.gloss]{gloss} \hyperref[TEI.graphic]{graphic} \hyperref[TEI.hi]{hi} \hyperref[TEI.index]{index} \hyperref[TEI.l]{l} \hyperref[TEI.label]{label} \hyperref[TEI.lb]{lb} \hyperref[TEI.lg]{lg} \hyperref[TEI.list]{list} \hyperref[TEI.listBibl]{listBibl} \hyperref[TEI.measure]{measure} \hyperref[TEI.measureGrp]{measureGrp} \hyperref[TEI.media]{media} \hyperref[TEI.mentioned]{mentioned} \hyperref[TEI.milestone]{milestone} \hyperref[TEI.name]{name} \hyperref[TEI.note]{note} \hyperref[TEI.num]{num} \hyperref[TEI.orig]{orig} \hyperref[TEI.p]{p} \hyperref[TEI.pb]{pb} \hyperref[TEI.ptr]{ptr} \hyperref[TEI.q]{q} \hyperref[TEI.quote]{quote} \hyperref[TEI.ref]{ref} \hyperref[TEI.reg]{reg} \hyperref[TEI.rs]{rs} \hyperref[TEI.said]{said} \hyperref[TEI.sic]{sic} \hyperref[TEI.soCalled]{soCalled} \hyperref[TEI.sp]{sp} \hyperref[TEI.stage]{stage} \hyperref[TEI.term]{term} \hyperref[TEI.time]{time} \hyperref[TEI.title]{title} \hyperref[TEI.unclear]{unclear}\par 
    \item[derived-module-tei.istex: ]
   \hyperref[TEI.math]{math} \hyperref[TEI.mrow]{mrow}\par 
    \item[figures: ]
   \hyperref[TEI.figure]{figure} \hyperref[TEI.formula]{formula} \hyperref[TEI.notatedMusic]{notatedMusic} \hyperref[TEI.table]{table}\par 
    \item[header: ]
   \hyperref[TEI.biblFull]{biblFull} \hyperref[TEI.idno]{idno}\par 
    \item[iso-fs: ]
   \hyperref[TEI.fLib]{fLib} \hyperref[TEI.fs]{fs} \hyperref[TEI.fvLib]{fvLib}\par 
    \item[linking: ]
   \hyperref[TEI.ab]{ab} \hyperref[TEI.alt]{alt} \hyperref[TEI.altGrp]{altGrp} \hyperref[TEI.anchor]{anchor} \hyperref[TEI.join]{join} \hyperref[TEI.joinGrp]{joinGrp} \hyperref[TEI.link]{link} \hyperref[TEI.linkGrp]{linkGrp} \hyperref[TEI.seg]{seg} \hyperref[TEI.timeline]{timeline}\par 
    \item[msdescription: ]
   \hyperref[TEI.catchwords]{catchwords} \hyperref[TEI.depth]{depth} \hyperref[TEI.dim]{dim} \hyperref[TEI.dimensions]{dimensions} \hyperref[TEI.height]{height} \hyperref[TEI.heraldry]{heraldry} \hyperref[TEI.locus]{locus} \hyperref[TEI.locusGrp]{locusGrp} \hyperref[TEI.material]{material} \hyperref[TEI.msDesc]{msDesc} \hyperref[TEI.objectType]{objectType} \hyperref[TEI.origDate]{origDate} \hyperref[TEI.origPlace]{origPlace} \hyperref[TEI.secFol]{secFol} \hyperref[TEI.signatures]{signatures} \hyperref[TEI.source]{source} \hyperref[TEI.stamp]{stamp} \hyperref[TEI.watermark]{watermark} \hyperref[TEI.width]{width}\par 
    \item[namesdates: ]
   \hyperref[TEI.addName]{addName} \hyperref[TEI.affiliation]{affiliation} \hyperref[TEI.country]{country} \hyperref[TEI.forename]{forename} \hyperref[TEI.genName]{genName} \hyperref[TEI.geogName]{geogName} \hyperref[TEI.listOrg]{listOrg} \hyperref[TEI.listPlace]{listPlace} \hyperref[TEI.location]{location} \hyperref[TEI.nameLink]{nameLink} \hyperref[TEI.orgName]{orgName} \hyperref[TEI.persName]{persName} \hyperref[TEI.placeName]{placeName} \hyperref[TEI.region]{region} \hyperref[TEI.roleName]{roleName} \hyperref[TEI.settlement]{settlement} \hyperref[TEI.state]{state} \hyperref[TEI.surname]{surname}\par 
    \item[spoken: ]
   \hyperref[TEI.annotationBlock]{annotationBlock}\par 
    \item[textstructure: ]
   \hyperref[TEI.floatingText]{floatingText}\par 
    \item[transcr: ]
   \hyperref[TEI.addSpan]{addSpan} \hyperref[TEI.am]{am} \hyperref[TEI.damage]{damage} \hyperref[TEI.damageSpan]{damageSpan} \hyperref[TEI.delSpan]{delSpan} \hyperref[TEI.ex]{ex} \hyperref[TEI.fw]{fw} \hyperref[TEI.handShift]{handShift} \hyperref[TEI.listTranspose]{listTranspose} \hyperref[TEI.metamark]{metamark} \hyperref[TEI.mod]{mod} \hyperref[TEI.redo]{redo} \hyperref[TEI.restore]{restore} \hyperref[TEI.retrace]{retrace} \hyperref[TEI.secl]{secl} \hyperref[TEI.space]{space} \hyperref[TEI.subst]{subst} \hyperref[TEI.substJoin]{substJoin} \hyperref[TEI.supplied]{supplied} \hyperref[TEI.surplus]{surplus} \hyperref[TEI.undo]{undo}\par des données textuelles
    \item[{Note}]
  \par
The {\itshape who} attribute may be used to indicate more precisely the person or persons participating in the action described by the stage direction.
    \item[{Exemple}]
  \leavevmode\bgroup\exampleFont \begin{shaded}\noindent\mbox{}{<\textbf{stage}\hspace*{6pt}{type}="{setting}">}La scène est dans une place de ville.{</\textbf{stage}>}\mbox{}\newline 
{<\textbf{stage}\hspace*{6pt}{type}="{exit}">}, s'en allant.{</\textbf{stage}>}\mbox{}\newline 
{<\textbf{stage}\hspace*{6pt}{type}="{business}">}(Arnolphe ôte par trois fois le chapeau de dessus la tête d'Alain.){</\textbf{stage}>}\mbox{}\newline 
{<\textbf{stage}\hspace*{6pt}{type}="{delivery}">}, à {<\textbf{name}>}Georgette{</\textbf{name}>}.{</\textbf{stage}>}\mbox{}\newline 
{<\textbf{stage}\hspace*{6pt}{type}="{setting}">}(Tous étant rentrés.){</\textbf{stage}>}\mbox{}\newline 
{<\textbf{stage}\hspace*{6pt}{type}="{delivery}">}, riant.{</\textbf{stage}>}\mbox{}\newline 
{<\textbf{stage}\hspace*{6pt}{type}="{delivery}">}, lui montrant le logis d'{<\textbf{name}>}AGNÈS{</\textbf{name}>}.{</\textbf{stage}>}\mbox{}\newline 
{<\textbf{stage}\hspace*{6pt}{type}="{delivery}">}, à part.{</\textbf{stage}>}\mbox{}\newline 
{<\textbf{stage}\hspace*{6pt}{type}="{business}">}(Frappant à la porte.){</\textbf{stage}>}\mbox{}\newline 
{<\textbf{stage}\hspace*{6pt}{type}="{delivery}">}, assis.{</\textbf{stage}>}\mbox{}\newline 
{<\textbf{stage}\hspace*{6pt}{type}="{business}">}(Il se lève.){</\textbf{stage}>}\end{shaded}\egroup 


    \item[{Modèle de contenu}]
  \mbox{}\hfill\\[-10pt]\begin{Verbatim}[fontsize=\small]
<content>
 <macroRef key="macro.specialPara"/>
</content>
    
\end{Verbatim}

    \item[{Schéma Declaration}]
  \mbox{}\hfill\\[-10pt]\begin{Verbatim}[fontsize=\small]
element stage
{
   tei_att.ascribed.attributes,
   tei_att.global.attributes,
   tei_att.placement.attributes,
   attribute type
   {
      list
      {
         (
            "setting"
          | "entrance"
          | "exit"
          | "business"
          | "novelistic"
          | "delivery"
          | "modifier"
          | "location"
          | "mixed"
         )*
      }
   }?,
   tei_macro.specialPara}
\end{Verbatim}

\end{reflist}  \index{stamp=<stamp>|oddindex}
\begin{reflist}
\item[]\begin{specHead}{TEI.stamp}{<stamp> }(cachet) Contient un mot ou une expression décrivant un cachet ou une marque du même genre. [\xref{http://www.tei-c.org/release/doc/tei-p5-doc/en/html/MS.html\#mswat}{10.3.3. Watermarks and Stamps}]\end{specHead} 
    \item[{Module}]
  msdescription
    \item[{Attributs}]
  Attributs \hyperref[TEI.att.global]{att.global} (\textit{@xml:id}, \textit{@n}, \textit{@xml:lang}, \textit{@xml:base}, \textit{@xml:space})  (\hyperref[TEI.att.global.rendition]{att.global.rendition} (\textit{@rend}, \textit{@style}, \textit{@rendition})) (\hyperref[TEI.att.global.linking]{att.global.linking} (\textit{@corresp}, \textit{@synch}, \textit{@sameAs}, \textit{@copyOf}, \textit{@next}, \textit{@prev}, \textit{@exclude}, \textit{@select})) (\hyperref[TEI.att.global.analytic]{att.global.analytic} (\textit{@ana})) (\hyperref[TEI.att.global.facs]{att.global.facs} (\textit{@facs})) (\hyperref[TEI.att.global.change]{att.global.change} (\textit{@change})) (\hyperref[TEI.att.global.responsibility]{att.global.responsibility} (\textit{@cert}, \textit{@resp})) (\hyperref[TEI.att.global.source]{att.global.source} (\textit{@source})) \hyperref[TEI.att.typed]{att.typed} (\textit{@type}, \textit{@subtype}) \hyperref[TEI.att.datable]{att.datable} (\textit{@calendar}, \textit{@period})  (\hyperref[TEI.att.datable.w3c]{att.datable.w3c} (\textit{@when}, \textit{@notBefore}, \textit{@notAfter}, \textit{@from}, \textit{@to})) (\hyperref[TEI.att.datable.iso]{att.datable.iso} (\textit{@when-iso}, \textit{@notBefore-iso}, \textit{@notAfter-iso}, \textit{@from-iso}, \textit{@to-iso})) (\hyperref[TEI.att.datable.custom]{att.datable.custom} (\textit{@when-custom}, \textit{@notBefore-custom}, \textit{@notAfter-custom}, \textit{@from-custom}, \textit{@to-custom}, \textit{@datingPoint}, \textit{@datingMethod}))
    \item[{Membre du}]
  \hyperref[TEI.model.pPart.msdesc]{model.pPart.msdesc}
    \item[{Contenu dans}]
  
    \item[analysis: ]
   \hyperref[TEI.cl]{cl} \hyperref[TEI.phr]{phr} \hyperref[TEI.s]{s} \hyperref[TEI.span]{span}\par 
    \item[core: ]
   \hyperref[TEI.abbr]{abbr} \hyperref[TEI.add]{add} \hyperref[TEI.addrLine]{addrLine} \hyperref[TEI.author]{author} \hyperref[TEI.biblScope]{biblScope} \hyperref[TEI.citedRange]{citedRange} \hyperref[TEI.corr]{corr} \hyperref[TEI.date]{date} \hyperref[TEI.del]{del} \hyperref[TEI.desc]{desc} \hyperref[TEI.distinct]{distinct} \hyperref[TEI.editor]{editor} \hyperref[TEI.email]{email} \hyperref[TEI.emph]{emph} \hyperref[TEI.expan]{expan} \hyperref[TEI.foreign]{foreign} \hyperref[TEI.gloss]{gloss} \hyperref[TEI.head]{head} \hyperref[TEI.headItem]{headItem} \hyperref[TEI.headLabel]{headLabel} \hyperref[TEI.hi]{hi} \hyperref[TEI.item]{item} \hyperref[TEI.l]{l} \hyperref[TEI.label]{label} \hyperref[TEI.measure]{measure} \hyperref[TEI.meeting]{meeting} \hyperref[TEI.mentioned]{mentioned} \hyperref[TEI.name]{name} \hyperref[TEI.note]{note} \hyperref[TEI.num]{num} \hyperref[TEI.orig]{orig} \hyperref[TEI.p]{p} \hyperref[TEI.pubPlace]{pubPlace} \hyperref[TEI.publisher]{publisher} \hyperref[TEI.q]{q} \hyperref[TEI.quote]{quote} \hyperref[TEI.ref]{ref} \hyperref[TEI.reg]{reg} \hyperref[TEI.resp]{resp} \hyperref[TEI.rs]{rs} \hyperref[TEI.said]{said} \hyperref[TEI.sic]{sic} \hyperref[TEI.soCalled]{soCalled} \hyperref[TEI.speaker]{speaker} \hyperref[TEI.stage]{stage} \hyperref[TEI.street]{street} \hyperref[TEI.term]{term} \hyperref[TEI.textLang]{textLang} \hyperref[TEI.time]{time} \hyperref[TEI.title]{title} \hyperref[TEI.unclear]{unclear}\par 
    \item[figures: ]
   \hyperref[TEI.cell]{cell} \hyperref[TEI.figDesc]{figDesc}\par 
    \item[header: ]
   \hyperref[TEI.authority]{authority} \hyperref[TEI.change]{change} \hyperref[TEI.classCode]{classCode} \hyperref[TEI.creation]{creation} \hyperref[TEI.distributor]{distributor} \hyperref[TEI.edition]{edition} \hyperref[TEI.extent]{extent} \hyperref[TEI.funder]{funder} \hyperref[TEI.language]{language} \hyperref[TEI.licence]{licence} \hyperref[TEI.rendition]{rendition}\par 
    \item[iso-fs: ]
   \hyperref[TEI.fDescr]{fDescr} \hyperref[TEI.fsDescr]{fsDescr}\par 
    \item[linking: ]
   \hyperref[TEI.ab]{ab} \hyperref[TEI.seg]{seg}\par 
    \item[msdescription: ]
   \hyperref[TEI.accMat]{accMat} \hyperref[TEI.acquisition]{acquisition} \hyperref[TEI.additions]{additions} \hyperref[TEI.catchwords]{catchwords} \hyperref[TEI.collation]{collation} \hyperref[TEI.colophon]{colophon} \hyperref[TEI.condition]{condition} \hyperref[TEI.custEvent]{custEvent} \hyperref[TEI.decoNote]{decoNote} \hyperref[TEI.explicit]{explicit} \hyperref[TEI.filiation]{filiation} \hyperref[TEI.finalRubric]{finalRubric} \hyperref[TEI.foliation]{foliation} \hyperref[TEI.heraldry]{heraldry} \hyperref[TEI.incipit]{incipit} \hyperref[TEI.layout]{layout} \hyperref[TEI.material]{material} \hyperref[TEI.musicNotation]{musicNotation} \hyperref[TEI.objectType]{objectType} \hyperref[TEI.origDate]{origDate} \hyperref[TEI.origPlace]{origPlace} \hyperref[TEI.origin]{origin} \hyperref[TEI.provenance]{provenance} \hyperref[TEI.rubric]{rubric} \hyperref[TEI.secFol]{secFol} \hyperref[TEI.signatures]{signatures} \hyperref[TEI.source]{source} \hyperref[TEI.stamp]{stamp} \hyperref[TEI.summary]{summary} \hyperref[TEI.support]{support} \hyperref[TEI.surrogates]{surrogates} \hyperref[TEI.typeNote]{typeNote} \hyperref[TEI.watermark]{watermark}\par 
    \item[namesdates: ]
   \hyperref[TEI.addName]{addName} \hyperref[TEI.affiliation]{affiliation} \hyperref[TEI.country]{country} \hyperref[TEI.forename]{forename} \hyperref[TEI.genName]{genName} \hyperref[TEI.geogName]{geogName} \hyperref[TEI.nameLink]{nameLink} \hyperref[TEI.orgName]{orgName} \hyperref[TEI.persName]{persName} \hyperref[TEI.placeName]{placeName} \hyperref[TEI.region]{region} \hyperref[TEI.roleName]{roleName} \hyperref[TEI.settlement]{settlement} \hyperref[TEI.surname]{surname}\par 
    \item[textstructure: ]
   \hyperref[TEI.docAuthor]{docAuthor} \hyperref[TEI.docDate]{docDate} \hyperref[TEI.docEdition]{docEdition} \hyperref[TEI.titlePart]{titlePart}\par 
    \item[transcr: ]
   \hyperref[TEI.damage]{damage} \hyperref[TEI.fw]{fw} \hyperref[TEI.metamark]{metamark} \hyperref[TEI.mod]{mod} \hyperref[TEI.restore]{restore} \hyperref[TEI.retrace]{retrace} \hyperref[TEI.secl]{secl} \hyperref[TEI.supplied]{supplied} \hyperref[TEI.surplus]{surplus}
    \item[{Peut contenir}]
  
    \item[analysis: ]
   \hyperref[TEI.c]{c} \hyperref[TEI.cl]{cl} \hyperref[TEI.interp]{interp} \hyperref[TEI.interpGrp]{interpGrp} \hyperref[TEI.m]{m} \hyperref[TEI.pc]{pc} \hyperref[TEI.phr]{phr} \hyperref[TEI.s]{s} \hyperref[TEI.span]{span} \hyperref[TEI.spanGrp]{spanGrp} \hyperref[TEI.w]{w}\par 
    \item[core: ]
   \hyperref[TEI.abbr]{abbr} \hyperref[TEI.add]{add} \hyperref[TEI.address]{address} \hyperref[TEI.binaryObject]{binaryObject} \hyperref[TEI.cb]{cb} \hyperref[TEI.choice]{choice} \hyperref[TEI.corr]{corr} \hyperref[TEI.date]{date} \hyperref[TEI.del]{del} \hyperref[TEI.distinct]{distinct} \hyperref[TEI.email]{email} \hyperref[TEI.emph]{emph} \hyperref[TEI.expan]{expan} \hyperref[TEI.foreign]{foreign} \hyperref[TEI.gap]{gap} \hyperref[TEI.gb]{gb} \hyperref[TEI.gloss]{gloss} \hyperref[TEI.graphic]{graphic} \hyperref[TEI.hi]{hi} \hyperref[TEI.index]{index} \hyperref[TEI.lb]{lb} \hyperref[TEI.measure]{measure} \hyperref[TEI.measureGrp]{measureGrp} \hyperref[TEI.media]{media} \hyperref[TEI.mentioned]{mentioned} \hyperref[TEI.milestone]{milestone} \hyperref[TEI.name]{name} \hyperref[TEI.note]{note} \hyperref[TEI.num]{num} \hyperref[TEI.orig]{orig} \hyperref[TEI.pb]{pb} \hyperref[TEI.ptr]{ptr} \hyperref[TEI.ref]{ref} \hyperref[TEI.reg]{reg} \hyperref[TEI.rs]{rs} \hyperref[TEI.sic]{sic} \hyperref[TEI.soCalled]{soCalled} \hyperref[TEI.term]{term} \hyperref[TEI.time]{time} \hyperref[TEI.title]{title} \hyperref[TEI.unclear]{unclear}\par 
    \item[derived-module-tei.istex: ]
   \hyperref[TEI.math]{math} \hyperref[TEI.mrow]{mrow}\par 
    \item[figures: ]
   \hyperref[TEI.figure]{figure} \hyperref[TEI.formula]{formula} \hyperref[TEI.notatedMusic]{notatedMusic}\par 
    \item[header: ]
   \hyperref[TEI.idno]{idno}\par 
    \item[iso-fs: ]
   \hyperref[TEI.fLib]{fLib} \hyperref[TEI.fs]{fs} \hyperref[TEI.fvLib]{fvLib}\par 
    \item[linking: ]
   \hyperref[TEI.alt]{alt} \hyperref[TEI.altGrp]{altGrp} \hyperref[TEI.anchor]{anchor} \hyperref[TEI.join]{join} \hyperref[TEI.joinGrp]{joinGrp} \hyperref[TEI.link]{link} \hyperref[TEI.linkGrp]{linkGrp} \hyperref[TEI.seg]{seg} \hyperref[TEI.timeline]{timeline}\par 
    \item[msdescription: ]
   \hyperref[TEI.catchwords]{catchwords} \hyperref[TEI.depth]{depth} \hyperref[TEI.dim]{dim} \hyperref[TEI.dimensions]{dimensions} \hyperref[TEI.height]{height} \hyperref[TEI.heraldry]{heraldry} \hyperref[TEI.locus]{locus} \hyperref[TEI.locusGrp]{locusGrp} \hyperref[TEI.material]{material} \hyperref[TEI.objectType]{objectType} \hyperref[TEI.origDate]{origDate} \hyperref[TEI.origPlace]{origPlace} \hyperref[TEI.secFol]{secFol} \hyperref[TEI.signatures]{signatures} \hyperref[TEI.source]{source} \hyperref[TEI.stamp]{stamp} \hyperref[TEI.watermark]{watermark} \hyperref[TEI.width]{width}\par 
    \item[namesdates: ]
   \hyperref[TEI.addName]{addName} \hyperref[TEI.affiliation]{affiliation} \hyperref[TEI.country]{country} \hyperref[TEI.forename]{forename} \hyperref[TEI.genName]{genName} \hyperref[TEI.geogName]{geogName} \hyperref[TEI.location]{location} \hyperref[TEI.nameLink]{nameLink} \hyperref[TEI.orgName]{orgName} \hyperref[TEI.persName]{persName} \hyperref[TEI.placeName]{placeName} \hyperref[TEI.region]{region} \hyperref[TEI.roleName]{roleName} \hyperref[TEI.settlement]{settlement} \hyperref[TEI.state]{state} \hyperref[TEI.surname]{surname}\par 
    \item[spoken: ]
   \hyperref[TEI.annotationBlock]{annotationBlock}\par 
    \item[transcr: ]
   \hyperref[TEI.addSpan]{addSpan} \hyperref[TEI.am]{am} \hyperref[TEI.damage]{damage} \hyperref[TEI.damageSpan]{damageSpan} \hyperref[TEI.delSpan]{delSpan} \hyperref[TEI.ex]{ex} \hyperref[TEI.fw]{fw} \hyperref[TEI.handShift]{handShift} \hyperref[TEI.listTranspose]{listTranspose} \hyperref[TEI.metamark]{metamark} \hyperref[TEI.mod]{mod} \hyperref[TEI.redo]{redo} \hyperref[TEI.restore]{restore} \hyperref[TEI.retrace]{retrace} \hyperref[TEI.secl]{secl} \hyperref[TEI.space]{space} \hyperref[TEI.subst]{subst} \hyperref[TEI.substJoin]{substJoin} \hyperref[TEI.supplied]{supplied} \hyperref[TEI.surplus]{surplus} \hyperref[TEI.undo]{undo}\par des données textuelles
    \item[{Exemple}]
  \leavevmode\bgroup\exampleFont \begin{shaded}\noindent\mbox{}{<\textbf{rubric}>}Apologyticu TTVLLIANI AC IGNORATIA IN XPO IHV{<\textbf{lb}/>}\mbox{}\newline 
 SI NON LICET{<\textbf{lb}/>}\mbox{}\newline 
 NOBIS RO{<\textbf{lb}/>}\mbox{}\newline 
 manii imperii {<\textbf{stamp}>}Bodleian stamp{</\textbf{stamp}>}\mbox{}\newline 
\hspace*{6pt}{<\textbf{lb}/>}\mbox{}\newline 
{</\textbf{rubric}>}\end{shaded}\egroup 


    \item[{Exemple}]
  \leavevmode\bgroup\exampleFont \begin{shaded}\noindent\mbox{}{<\textbf{rubric}>}Apologyticu TTVLLIANI AC IGNORATIA IN XPO IHV{<\textbf{lb}/>} SI NON LICET{<\textbf{lb}/>} NOBIS RO{<\textbf{lb}/>}\mbox{}\newline 
 manii imperii {<\textbf{stamp}>}Bodleian stamp{</\textbf{stamp}>}\mbox{}\newline 
\hspace*{6pt}{<\textbf{lb}/>}\mbox{}\newline 
{</\textbf{rubric}>}\end{shaded}\egroup 


    \item[{Modèle de contenu}]
  \mbox{}\hfill\\[-10pt]\begin{Verbatim}[fontsize=\small]
<content>
 <macroRef key="macro.phraseSeq"/>
</content>
    
\end{Verbatim}

    \item[{Schéma Declaration}]
  \mbox{}\hfill\\[-10pt]\begin{Verbatim}[fontsize=\small]
element stamp
{
   tei_att.global.attributes,
   tei_att.typed.attributes,
   tei_att.datable.attributes,
   tei_macro.phraseSeq}
\end{Verbatim}

\end{reflist}  \index{standOff=<standOff>|oddindex}
\begin{reflist}
\item[]\begin{specHead}{TEI.standOff}{<standOff> }Container element for stand-off annotations embedded in a TEI document\end{specHead} 
    \item[{Namespace}]
  https://xml-schema.delivery.istex.fr/formats/ns1.xsd
    \item[{Module}]
  standOff
    \item[{Attributs}]
  Attributs \hyperref[TEI.att.global]{att.global} (\textit{@xml:id}, \textit{@n}, \textit{@xml:lang}, \textit{@xml:base}, \textit{@xml:space})  (\hyperref[TEI.att.global.rendition]{att.global.rendition} (\textit{@rend}, \textit{@style}, \textit{@rendition})) (\hyperref[TEI.att.global.linking]{att.global.linking} (\textit{@corresp}, \textit{@synch}, \textit{@sameAs}, \textit{@copyOf}, \textit{@next}, \textit{@prev}, \textit{@exclude}, \textit{@select})) (\hyperref[TEI.att.global.analytic]{att.global.analytic} (\textit{@ana})) (\hyperref[TEI.att.global.facs]{att.global.facs} (\textit{@facs})) (\hyperref[TEI.att.global.change]{att.global.change} (\textit{@change})) (\hyperref[TEI.att.global.responsibility]{att.global.responsibility} (\textit{@cert}, \textit{@resp})) (\hyperref[TEI.att.global.source]{att.global.source} (\textit{@source})) \hyperref[TEI.att.typed]{att.typed} (\textit{@type}, \textit{@subtype}) \hyperref[TEI.att.notated]{att.notated} (\textit{@notation}) \hyperref[TEI.att.datable.w3c]{att.datable.w3c} (\textit{@when}, \textit{@notBefore}, \textit{@notAfter}, \textit{@from}, \textit{@to}) \hyperref[TEI.att.pointing]{att.pointing} (\textit{@targetLang}, \textit{@target}, \textit{@evaluate}) 
    \item[{Membre du}]
  \hyperref[TEI.model.resourceLike]{model.resourceLike}
    \item[{Contenu dans}]
  
    \item[core: ]
   \hyperref[TEI.teiCorpus]{teiCorpus}\par 
    \item[standOff: ]
   \hyperref[TEI.standOff]{standOff}\par 
    \item[textstructure: ]
   \hyperref[TEI.TEI]{TEI}
    \item[{Peut contenir}]
  
    \item[header: ]
   \hyperref[TEI.teiHeader]{teiHeader}\par 
    \item[iso-fs: ]
   \hyperref[TEI.fsdDecl]{fsdDecl}\par 
    \item[standOff: ]
   \hyperref[TEI.listAnnotation]{listAnnotation} \hyperref[TEI.standOff]{standOff}\par 
    \item[textstructure: ]
   \hyperref[TEI.text]{text}\par 
    \item[transcr: ]
   \hyperref[TEI.facsimile]{facsimile} \hyperref[TEI.sourceDoc]{sourceDoc}
    \item[{Schematron}]
   <s:ns prefix="stdf"  uri="https://xml-schema.delivery.istex.fr/formats/ns1.xsd"/>
    \item[{Schematron}]
   <s:assert test="@type or not(ancestor::stdf:standOff)">This <s:name/> element must have a @type attribute, since it has an ancestor <s:name/> </s:assert>
    \item[{Modèle de contenu}]
  \mbox{}\hfill\\[-10pt]\begin{Verbatim}[fontsize=\small]
<content>
 <elementRef key="teiHeader" maxOccurs="1"
  minOccurs="0"/>
 <classRef key="model.resourceLike"
  maxOccurs="unbounded" minOccurs="0"/>
 <elementRef key="listAnnotation"
  maxOccurs="unbounded" minOccurs="0"/>
</content>
    
\end{Verbatim}

    \item[{Schéma Declaration}]
  \mbox{}\hfill\\[-10pt]\begin{Verbatim}[fontsize=\small]
element standOff
{
   tei_att.global.attributes,
   tei_att.typed.attributes,
   tei_att.notated.attributes,
   tei_att.datable.w3c.attributes,
   tei_att.pointing.attributes,
   tei_teiHeader?,
   tei_model.resourceLike*,
   tei_listAnnotation*
}
\end{Verbatim}

\end{reflist}  \index{state=<state>|oddindex}
\begin{reflist}
\item[]\begin{specHead}{TEI.state}{<state> }(statut) contient la description d'un statut ou d'une qualité actuels attribués à une personne, un lieu ou une organisation. [\xref{http://www.tei-c.org/release/doc/tei-p5-doc/en/html/ND.html\#NDPERSbp}{13.3.1. Basic Principles} \xref{http://www.tei-c.org/release/doc/tei-p5-doc/en/html/ND.html\#NDPERSEpc}{13.3.2.1. Personal Characteristics}]\end{specHead} 
    \item[{Module}]
  namesdates
    \item[{Attributs}]
  Attributs \hyperref[TEI.att.global]{att.global} (\textit{@xml:id}, \textit{@n}, \textit{@xml:lang}, \textit{@xml:base}, \textit{@xml:space})  (\hyperref[TEI.att.global.rendition]{att.global.rendition} (\textit{@rend}, \textit{@style}, \textit{@rendition})) (\hyperref[TEI.att.global.linking]{att.global.linking} (\textit{@corresp}, \textit{@synch}, \textit{@sameAs}, \textit{@copyOf}, \textit{@next}, \textit{@prev}, \textit{@exclude}, \textit{@select})) (\hyperref[TEI.att.global.analytic]{att.global.analytic} (\textit{@ana})) (\hyperref[TEI.att.global.facs]{att.global.facs} (\textit{@facs})) (\hyperref[TEI.att.global.change]{att.global.change} (\textit{@change})) (\hyperref[TEI.att.global.responsibility]{att.global.responsibility} (\textit{@cert}, \textit{@resp})) (\hyperref[TEI.att.global.source]{att.global.source} (\textit{@source})) \hyperref[TEI.att.datable]{att.datable} (\textit{@calendar}, \textit{@period})  (\hyperref[TEI.att.datable.w3c]{att.datable.w3c} (\textit{@when}, \textit{@notBefore}, \textit{@notAfter}, \textit{@from}, \textit{@to})) (\hyperref[TEI.att.datable.iso]{att.datable.iso} (\textit{@when-iso}, \textit{@notBefore-iso}, \textit{@notAfter-iso}, \textit{@from-iso}, \textit{@to-iso})) (\hyperref[TEI.att.datable.custom]{att.datable.custom} (\textit{@when-custom}, \textit{@notBefore-custom}, \textit{@notAfter-custom}, \textit{@from-custom}, \textit{@to-custom}, \textit{@datingPoint}, \textit{@datingMethod})) \hyperref[TEI.att.editLike]{att.editLike} (\textit{@evidence}, \textit{@instant})  (\hyperref[TEI.att.dimensions]{att.dimensions} (\textit{@unit}, \textit{@quantity}, \textit{@extent}, \textit{@precision}, \textit{@scope}) (\hyperref[TEI.att.ranging]{att.ranging} (\textit{@atLeast}, \textit{@atMost}, \textit{@min}, \textit{@max}, \textit{@confidence})) ) \hyperref[TEI.att.typed]{att.typed} (\textit{@type}, \textit{@subtype}) \hyperref[TEI.att.naming]{att.naming} (\textit{@role}, \textit{@nymRef})  (\hyperref[TEI.att.canonical]{att.canonical} (\textit{@key}, \textit{@ref}))
    \item[{Membre du}]
  \hyperref[TEI.model.persStateLike]{model.persStateLike} \hyperref[TEI.model.placeStateLike]{model.placeStateLike} 
    \item[{Contenu dans}]
  
    \item[analysis: ]
   \hyperref[TEI.cl]{cl} \hyperref[TEI.phr]{phr} \hyperref[TEI.s]{s} \hyperref[TEI.span]{span}\par 
    \item[core: ]
   \hyperref[TEI.abbr]{abbr} \hyperref[TEI.add]{add} \hyperref[TEI.addrLine]{addrLine} \hyperref[TEI.address]{address} \hyperref[TEI.author]{author} \hyperref[TEI.bibl]{bibl} \hyperref[TEI.biblScope]{biblScope} \hyperref[TEI.citedRange]{citedRange} \hyperref[TEI.corr]{corr} \hyperref[TEI.date]{date} \hyperref[TEI.del]{del} \hyperref[TEI.desc]{desc} \hyperref[TEI.distinct]{distinct} \hyperref[TEI.editor]{editor} \hyperref[TEI.email]{email} \hyperref[TEI.emph]{emph} \hyperref[TEI.expan]{expan} \hyperref[TEI.foreign]{foreign} \hyperref[TEI.gloss]{gloss} \hyperref[TEI.head]{head} \hyperref[TEI.headItem]{headItem} \hyperref[TEI.headLabel]{headLabel} \hyperref[TEI.hi]{hi} \hyperref[TEI.item]{item} \hyperref[TEI.l]{l} \hyperref[TEI.label]{label} \hyperref[TEI.measure]{measure} \hyperref[TEI.meeting]{meeting} \hyperref[TEI.mentioned]{mentioned} \hyperref[TEI.name]{name} \hyperref[TEI.note]{note} \hyperref[TEI.num]{num} \hyperref[TEI.orig]{orig} \hyperref[TEI.p]{p} \hyperref[TEI.pubPlace]{pubPlace} \hyperref[TEI.publisher]{publisher} \hyperref[TEI.q]{q} \hyperref[TEI.quote]{quote} \hyperref[TEI.ref]{ref} \hyperref[TEI.reg]{reg} \hyperref[TEI.resp]{resp} \hyperref[TEI.rs]{rs} \hyperref[TEI.said]{said} \hyperref[TEI.sic]{sic} \hyperref[TEI.soCalled]{soCalled} \hyperref[TEI.speaker]{speaker} \hyperref[TEI.stage]{stage} \hyperref[TEI.street]{street} \hyperref[TEI.term]{term} \hyperref[TEI.textLang]{textLang} \hyperref[TEI.time]{time} \hyperref[TEI.title]{title} \hyperref[TEI.unclear]{unclear}\par 
    \item[figures: ]
   \hyperref[TEI.cell]{cell} \hyperref[TEI.figDesc]{figDesc}\par 
    \item[header: ]
   \hyperref[TEI.authority]{authority} \hyperref[TEI.change]{change} \hyperref[TEI.classCode]{classCode} \hyperref[TEI.creation]{creation} \hyperref[TEI.distributor]{distributor} \hyperref[TEI.edition]{edition} \hyperref[TEI.extent]{extent} \hyperref[TEI.funder]{funder} \hyperref[TEI.language]{language} \hyperref[TEI.licence]{licence} \hyperref[TEI.rendition]{rendition}\par 
    \item[iso-fs: ]
   \hyperref[TEI.fDescr]{fDescr} \hyperref[TEI.fsDescr]{fsDescr}\par 
    \item[linking: ]
   \hyperref[TEI.ab]{ab} \hyperref[TEI.seg]{seg}\par 
    \item[msdescription: ]
   \hyperref[TEI.accMat]{accMat} \hyperref[TEI.acquisition]{acquisition} \hyperref[TEI.additions]{additions} \hyperref[TEI.catchwords]{catchwords} \hyperref[TEI.collation]{collation} \hyperref[TEI.colophon]{colophon} \hyperref[TEI.condition]{condition} \hyperref[TEI.custEvent]{custEvent} \hyperref[TEI.decoNote]{decoNote} \hyperref[TEI.explicit]{explicit} \hyperref[TEI.filiation]{filiation} \hyperref[TEI.finalRubric]{finalRubric} \hyperref[TEI.foliation]{foliation} \hyperref[TEI.heraldry]{heraldry} \hyperref[TEI.incipit]{incipit} \hyperref[TEI.layout]{layout} \hyperref[TEI.material]{material} \hyperref[TEI.musicNotation]{musicNotation} \hyperref[TEI.objectType]{objectType} \hyperref[TEI.origDate]{origDate} \hyperref[TEI.origPlace]{origPlace} \hyperref[TEI.origin]{origin} \hyperref[TEI.provenance]{provenance} \hyperref[TEI.rubric]{rubric} \hyperref[TEI.secFol]{secFol} \hyperref[TEI.signatures]{signatures} \hyperref[TEI.source]{source} \hyperref[TEI.stamp]{stamp} \hyperref[TEI.summary]{summary} \hyperref[TEI.support]{support} \hyperref[TEI.surrogates]{surrogates} \hyperref[TEI.typeNote]{typeNote} \hyperref[TEI.watermark]{watermark}\par 
    \item[namesdates: ]
   \hyperref[TEI.addName]{addName} \hyperref[TEI.affiliation]{affiliation} \hyperref[TEI.country]{country} \hyperref[TEI.forename]{forename} \hyperref[TEI.genName]{genName} \hyperref[TEI.geogName]{geogName} \hyperref[TEI.nameLink]{nameLink} \hyperref[TEI.org]{org} \hyperref[TEI.orgName]{orgName} \hyperref[TEI.persName]{persName} \hyperref[TEI.person]{person} \hyperref[TEI.personGrp]{personGrp} \hyperref[TEI.persona]{persona} \hyperref[TEI.place]{place} \hyperref[TEI.placeName]{placeName} \hyperref[TEI.region]{region} \hyperref[TEI.roleName]{roleName} \hyperref[TEI.settlement]{settlement} \hyperref[TEI.state]{state} \hyperref[TEI.surname]{surname}\par 
    \item[spoken: ]
   \hyperref[TEI.annotationBlock]{annotationBlock}\par 
    \item[standOff: ]
   \hyperref[TEI.listAnnotation]{listAnnotation}\par 
    \item[textstructure: ]
   \hyperref[TEI.docAuthor]{docAuthor} \hyperref[TEI.docDate]{docDate} \hyperref[TEI.docEdition]{docEdition} \hyperref[TEI.titlePart]{titlePart}\par 
    \item[transcr: ]
   \hyperref[TEI.damage]{damage} \hyperref[TEI.fw]{fw} \hyperref[TEI.metamark]{metamark} \hyperref[TEI.mod]{mod} \hyperref[TEI.restore]{restore} \hyperref[TEI.retrace]{retrace} \hyperref[TEI.secl]{secl} \hyperref[TEI.supplied]{supplied} \hyperref[TEI.surplus]{surplus}
    \item[{Peut contenir}]
  
    \item[core: ]
   \hyperref[TEI.bibl]{bibl} \hyperref[TEI.biblStruct]{biblStruct} \hyperref[TEI.desc]{desc} \hyperref[TEI.head]{head} \hyperref[TEI.label]{label} \hyperref[TEI.listBibl]{listBibl} \hyperref[TEI.note]{note} \hyperref[TEI.p]{p}\par 
    \item[header: ]
   \hyperref[TEI.biblFull]{biblFull}\par 
    \item[linking: ]
   \hyperref[TEI.ab]{ab}\par 
    \item[msdescription: ]
   \hyperref[TEI.msDesc]{msDesc}\par 
    \item[namesdates: ]
   \hyperref[TEI.state]{state}
    \item[{Note}]
  \par
Where there is confusion between \texttt{<trait>} and \hyperref[TEI.state]{<state>} the more general purpose element \hyperref[TEI.state]{<state>} should be used even for unchanging characteristics. If you wish to distinguish between characteristics that are generally perceived to be time-bound states and those assumed to be fixed traits, then \texttt{<trait>} is available for the more static of these. The \hyperref[TEI.state]{<state>} element encodes characteristics which are sometimes assumed to change, often at specific times or over a date range, whereas the \texttt{<trait>} elements are used to record characteristics, such as eye-colour, which are less subject to change. Traits are typically, but not necessarily, independent of the volition or action of the holder.
    \item[{Exemple}]
  \leavevmode\bgroup\exampleFont \begin{shaded}\noindent\mbox{}{<\textbf{state}\hspace*{6pt}{cert}="{high}"\hspace*{6pt}{from}="{1987-01-01}"\mbox{}\newline 
\hspace*{6pt}{to}="{1997-12-31}"\hspace*{6pt}{type}="{social}">}\mbox{}\newline 
\hspace*{6pt}{<\textbf{label}>}Citoyenneté{</\textbf{label}>}\mbox{}\newline 
\hspace*{6pt}{<\textbf{desc}>}Entre 1987 et 1997a bénéficié du statut de citoyen naturalisé du ROYAUME-UNI{</\textbf{desc}>}\mbox{}\newline 
{</\textbf{state}>}\end{shaded}\egroup 


    \item[{Modèle de contenu}]
  \mbox{}\hfill\\[-10pt]\begin{Verbatim}[fontsize=\small]
<content>
 <sequence maxOccurs="1" minOccurs="1">
  <elementRef key="precision"
   maxOccurs="unbounded" minOccurs="0"/>
  <alternate maxOccurs="1" minOccurs="1">
   <elementRef key="state"
    maxOccurs="unbounded" minOccurs="1"/>
   <sequence maxOccurs="1" minOccurs="1">
    <classRef key="model.headLike"
     maxOccurs="unbounded" minOccurs="0"/>
    <classRef key="model.pLike"
     maxOccurs="unbounded" minOccurs="1"/>
    <alternate maxOccurs="unbounded"
     minOccurs="0">
     <classRef key="model.noteLike"/>
     <classRef key="model.biblLike"/>
    </alternate>
   </sequence>
   <alternate maxOccurs="unbounded"
    minOccurs="0">
    <classRef key="model.labelLike"/>
    <classRef key="model.noteLike"/>
    <classRef key="model.biblLike"/>
   </alternate>
  </alternate>
 </sequence>
</content>
    
\end{Verbatim}

    \item[{Schéma Declaration}]
  \mbox{}\hfill\\[-10pt]\begin{Verbatim}[fontsize=\small]
element state
{
   tei_att.global.attributes,
   tei_att.datable.attributes,
   tei_att.editLike.attributes,
   tei_att.typed.attributes,
   tei_att.naming.attributes,
   (
      precision*,
      (
         tei_state+
       | (
            tei_model.headLike*,
            tei_model.pLike+,
            ( tei_model.noteLike | tei_model.biblLike )*
         )
       | ( tei_model.labelLike | tei_model.noteLike | tei_model.biblLike )*
      )
   )
}
\end{Verbatim}

\end{reflist}  \index{street=<street>|oddindex}
\begin{reflist}
\item[]\begin{specHead}{TEI.street}{<street> }adresse complète d'une rue comprenant un nom ou un numéro identifiant un bâtiment ainsi que le nom de la rue ou du chemin sur laquelle il est situé. [\xref{http://www.tei-c.org/release/doc/tei-p5-doc/en/html/CO.html\#CONAAD}{3.5.2. Addresses}]\end{specHead} 
    \item[{Module}]
  core
    \item[{Attributs}]
  Attributs \hyperref[TEI.att.global]{att.global} (\textit{@xml:id}, \textit{@n}, \textit{@xml:lang}, \textit{@xml:base}, \textit{@xml:space})  (\hyperref[TEI.att.global.rendition]{att.global.rendition} (\textit{@rend}, \textit{@style}, \textit{@rendition})) (\hyperref[TEI.att.global.linking]{att.global.linking} (\textit{@corresp}, \textit{@synch}, \textit{@sameAs}, \textit{@copyOf}, \textit{@next}, \textit{@prev}, \textit{@exclude}, \textit{@select})) (\hyperref[TEI.att.global.analytic]{att.global.analytic} (\textit{@ana})) (\hyperref[TEI.att.global.facs]{att.global.facs} (\textit{@facs})) (\hyperref[TEI.att.global.change]{att.global.change} (\textit{@change})) (\hyperref[TEI.att.global.responsibility]{att.global.responsibility} (\textit{@cert}, \textit{@resp})) (\hyperref[TEI.att.global.source]{att.global.source} (\textit{@source}))
    \item[{Membre du}]
  \hyperref[TEI.model.addrPart]{model.addrPart}
    \item[{Contenu dans}]
  
    \item[core: ]
   \hyperref[TEI.address]{address}
    \item[{Peut contenir}]
  
    \item[analysis: ]
   \hyperref[TEI.c]{c} \hyperref[TEI.cl]{cl} \hyperref[TEI.interp]{interp} \hyperref[TEI.interpGrp]{interpGrp} \hyperref[TEI.m]{m} \hyperref[TEI.pc]{pc} \hyperref[TEI.phr]{phr} \hyperref[TEI.s]{s} \hyperref[TEI.span]{span} \hyperref[TEI.spanGrp]{spanGrp} \hyperref[TEI.w]{w}\par 
    \item[core: ]
   \hyperref[TEI.abbr]{abbr} \hyperref[TEI.add]{add} \hyperref[TEI.address]{address} \hyperref[TEI.binaryObject]{binaryObject} \hyperref[TEI.cb]{cb} \hyperref[TEI.choice]{choice} \hyperref[TEI.corr]{corr} \hyperref[TEI.date]{date} \hyperref[TEI.del]{del} \hyperref[TEI.distinct]{distinct} \hyperref[TEI.email]{email} \hyperref[TEI.emph]{emph} \hyperref[TEI.expan]{expan} \hyperref[TEI.foreign]{foreign} \hyperref[TEI.gap]{gap} \hyperref[TEI.gb]{gb} \hyperref[TEI.gloss]{gloss} \hyperref[TEI.graphic]{graphic} \hyperref[TEI.hi]{hi} \hyperref[TEI.index]{index} \hyperref[TEI.lb]{lb} \hyperref[TEI.measure]{measure} \hyperref[TEI.measureGrp]{measureGrp} \hyperref[TEI.media]{media} \hyperref[TEI.mentioned]{mentioned} \hyperref[TEI.milestone]{milestone} \hyperref[TEI.name]{name} \hyperref[TEI.note]{note} \hyperref[TEI.num]{num} \hyperref[TEI.orig]{orig} \hyperref[TEI.pb]{pb} \hyperref[TEI.ptr]{ptr} \hyperref[TEI.ref]{ref} \hyperref[TEI.reg]{reg} \hyperref[TEI.rs]{rs} \hyperref[TEI.sic]{sic} \hyperref[TEI.soCalled]{soCalled} \hyperref[TEI.term]{term} \hyperref[TEI.time]{time} \hyperref[TEI.title]{title} \hyperref[TEI.unclear]{unclear}\par 
    \item[derived-module-tei.istex: ]
   \hyperref[TEI.math]{math} \hyperref[TEI.mrow]{mrow}\par 
    \item[figures: ]
   \hyperref[TEI.figure]{figure} \hyperref[TEI.formula]{formula} \hyperref[TEI.notatedMusic]{notatedMusic}\par 
    \item[header: ]
   \hyperref[TEI.idno]{idno}\par 
    \item[iso-fs: ]
   \hyperref[TEI.fLib]{fLib} \hyperref[TEI.fs]{fs} \hyperref[TEI.fvLib]{fvLib}\par 
    \item[linking: ]
   \hyperref[TEI.alt]{alt} \hyperref[TEI.altGrp]{altGrp} \hyperref[TEI.anchor]{anchor} \hyperref[TEI.join]{join} \hyperref[TEI.joinGrp]{joinGrp} \hyperref[TEI.link]{link} \hyperref[TEI.linkGrp]{linkGrp} \hyperref[TEI.seg]{seg} \hyperref[TEI.timeline]{timeline}\par 
    \item[msdescription: ]
   \hyperref[TEI.catchwords]{catchwords} \hyperref[TEI.depth]{depth} \hyperref[TEI.dim]{dim} \hyperref[TEI.dimensions]{dimensions} \hyperref[TEI.height]{height} \hyperref[TEI.heraldry]{heraldry} \hyperref[TEI.locus]{locus} \hyperref[TEI.locusGrp]{locusGrp} \hyperref[TEI.material]{material} \hyperref[TEI.objectType]{objectType} \hyperref[TEI.origDate]{origDate} \hyperref[TEI.origPlace]{origPlace} \hyperref[TEI.secFol]{secFol} \hyperref[TEI.signatures]{signatures} \hyperref[TEI.source]{source} \hyperref[TEI.stamp]{stamp} \hyperref[TEI.watermark]{watermark} \hyperref[TEI.width]{width}\par 
    \item[namesdates: ]
   \hyperref[TEI.addName]{addName} \hyperref[TEI.affiliation]{affiliation} \hyperref[TEI.country]{country} \hyperref[TEI.forename]{forename} \hyperref[TEI.genName]{genName} \hyperref[TEI.geogName]{geogName} \hyperref[TEI.location]{location} \hyperref[TEI.nameLink]{nameLink} \hyperref[TEI.orgName]{orgName} \hyperref[TEI.persName]{persName} \hyperref[TEI.placeName]{placeName} \hyperref[TEI.region]{region} \hyperref[TEI.roleName]{roleName} \hyperref[TEI.settlement]{settlement} \hyperref[TEI.state]{state} \hyperref[TEI.surname]{surname}\par 
    \item[spoken: ]
   \hyperref[TEI.annotationBlock]{annotationBlock}\par 
    \item[transcr: ]
   \hyperref[TEI.addSpan]{addSpan} \hyperref[TEI.am]{am} \hyperref[TEI.damage]{damage} \hyperref[TEI.damageSpan]{damageSpan} \hyperref[TEI.delSpan]{delSpan} \hyperref[TEI.ex]{ex} \hyperref[TEI.fw]{fw} \hyperref[TEI.handShift]{handShift} \hyperref[TEI.listTranspose]{listTranspose} \hyperref[TEI.metamark]{metamark} \hyperref[TEI.mod]{mod} \hyperref[TEI.redo]{redo} \hyperref[TEI.restore]{restore} \hyperref[TEI.retrace]{retrace} \hyperref[TEI.secl]{secl} \hyperref[TEI.space]{space} \hyperref[TEI.subst]{subst} \hyperref[TEI.substJoin]{substJoin} \hyperref[TEI.supplied]{supplied} \hyperref[TEI.surplus]{surplus} \hyperref[TEI.undo]{undo}\par des données textuelles
    \item[{Note}]
  \par
L'ordre et la présentation des noms et numéros de maisons et des noms de rues, etc. L'encodage peut varier considérablement selon les pays ; il devrait reprendre la disposition propre au pays concerné. 
    \item[{Exemple}]
  \leavevmode\bgroup\exampleFont \begin{shaded}\noindent\mbox{}{<\textbf{street}>}via della Faggiola, 36{</\textbf{street}>}\end{shaded}\egroup 


    \item[{Exemple}]
  \leavevmode\bgroup\exampleFont \begin{shaded}\noindent\mbox{}{<\textbf{street}>}110, rue de Grenelle {</\textbf{street}>}\end{shaded}\egroup 


    \item[{Exemple}]
  \leavevmode\bgroup\exampleFont \begin{shaded}\noindent\mbox{}{<\textbf{street}>}36, quai des Orfèvres{</\textbf{street}>}\end{shaded}\egroup 


    \item[{Modèle de contenu}]
  \mbox{}\hfill\\[-10pt]\begin{Verbatim}[fontsize=\small]
<content>
 <macroRef key="macro.phraseSeq"/>
</content>
    
\end{Verbatim}

    \item[{Schéma Declaration}]
  \mbox{}\hfill\\[-10pt]\begin{Verbatim}[fontsize=\small]
element street { tei_att.global.attributes, tei_macro.phraseSeq }
\end{Verbatim}

\end{reflist}  \index{string=<string>|oddindex}
\begin{reflist}
\item[]\begin{specHead}{TEI.string}{<string> }(valeur de chaîne) représente la partie valeur d'une spécification trait-valeur qui contient une chaîne de caractères. [\xref{http://www.tei-c.org/release/doc/tei-p5-doc/en/html/FS.html\#FSSY}{18.3. Other Atomic Feature Values}]\end{specHead} 
    \item[{Module}]
  iso-fs
    \item[{Attributs}]
  Attributs \hyperref[TEI.att.global]{att.global} (\textit{@xml:id}, \textit{@n}, \textit{@xml:lang}, \textit{@xml:base}, \textit{@xml:space})  (\hyperref[TEI.att.global.rendition]{att.global.rendition} (\textit{@rend}, \textit{@style}, \textit{@rendition})) (\hyperref[TEI.att.global.linking]{att.global.linking} (\textit{@corresp}, \textit{@synch}, \textit{@sameAs}, \textit{@copyOf}, \textit{@next}, \textit{@prev}, \textit{@exclude}, \textit{@select})) (\hyperref[TEI.att.global.analytic]{att.global.analytic} (\textit{@ana})) (\hyperref[TEI.att.global.facs]{att.global.facs} (\textit{@facs})) (\hyperref[TEI.att.global.change]{att.global.change} (\textit{@change})) (\hyperref[TEI.att.global.responsibility]{att.global.responsibility} (\textit{@cert}, \textit{@resp})) (\hyperref[TEI.att.global.source]{att.global.source} (\textit{@source})) \hyperref[TEI.att.datcat]{att.datcat} (\textit{@datcat}, \textit{@valueDatcat}) 
    \item[{Membre du}]
  \hyperref[TEI.model.featureVal.single]{model.featureVal.single}
    \item[{Contenu dans}]
  
    \item[iso-fs: ]
   \hyperref[TEI.f]{f} \hyperref[TEI.fvLib]{fvLib} \hyperref[TEI.if]{if} \hyperref[TEI.vAlt]{vAlt} \hyperref[TEI.vColl]{vColl} \hyperref[TEI.vDefault]{vDefault} \hyperref[TEI.vLabel]{vLabel} \hyperref[TEI.vMerge]{vMerge} \hyperref[TEI.vNot]{vNot} \hyperref[TEI.vRange]{vRange}
    \item[{Peut contenir}]
  Des données textuelles uniquement
    \item[{Exemple}]
  \leavevmode\bgroup\exampleFont \begin{shaded}\noindent\mbox{}{<\textbf{f}\hspace*{6pt}{name}="{greeting}">}\mbox{}\newline 
\hspace*{6pt}{<\textbf{string}>}Bonjour, le monde ! {</\textbf{string}>}\mbox{}\newline 
{</\textbf{f}>}\end{shaded}\egroup 


    \item[{Modèle de contenu}]
  \fbox{\ttfamily <content>\newline
 <macroRef key="macro.xtext"/>\newline
</content>\newline
    } 
    \item[{Schéma Declaration}]
  \mbox{}\hfill\\[-10pt]\begin{Verbatim}[fontsize=\small]
element string
{
   tei_att.global.attributes,
   tei_att.datcat.attributes,
   tei_macro.xtext}
\end{Verbatim}

\end{reflist}  \index{subst=<subst>|oddindex}
\begin{reflist}
\item[]\begin{specHead}{TEI.subst}{<subst> }(substitution) regroupe une ou plusieurs parties de texte supprimées et une ou plusieurs parties de texte ajoutées, lorsque cette combinaison peut être considérée comme une intervention unique sur le texte. [\xref{http://www.tei-c.org/release/doc/tei-p5-doc/en/html/PH.html\#PHSU}{11.3.1.5. Substitutions}]\end{specHead} 
    \item[{Module}]
  transcr
    \item[{Attributs}]
  Attributs \hyperref[TEI.att.global]{att.global} (\textit{@xml:id}, \textit{@n}, \textit{@xml:lang}, \textit{@xml:base}, \textit{@xml:space})  (\hyperref[TEI.att.global.rendition]{att.global.rendition} (\textit{@rend}, \textit{@style}, \textit{@rendition})) (\hyperref[TEI.att.global.linking]{att.global.linking} (\textit{@corresp}, \textit{@synch}, \textit{@sameAs}, \textit{@copyOf}, \textit{@next}, \textit{@prev}, \textit{@exclude}, \textit{@select})) (\hyperref[TEI.att.global.analytic]{att.global.analytic} (\textit{@ana})) (\hyperref[TEI.att.global.facs]{att.global.facs} (\textit{@facs})) (\hyperref[TEI.att.global.change]{att.global.change} (\textit{@change})) (\hyperref[TEI.att.global.responsibility]{att.global.responsibility} (\textit{@cert}, \textit{@resp})) (\hyperref[TEI.att.global.source]{att.global.source} (\textit{@source})) \hyperref[TEI.att.transcriptional]{att.transcriptional} (\textit{@status}, \textit{@cause}, \textit{@seq})  (\hyperref[TEI.att.editLike]{att.editLike} (\textit{@evidence}, \textit{@instant}) (\hyperref[TEI.att.dimensions]{att.dimensions} (\textit{@unit}, \textit{@quantity}, \textit{@extent}, \textit{@precision}, \textit{@scope}) (\hyperref[TEI.att.ranging]{att.ranging} (\textit{@atLeast}, \textit{@atMost}, \textit{@min}, \textit{@max}, \textit{@confidence})) ) ) (\hyperref[TEI.att.written]{att.written} (\textit{@hand}))
    \item[{Membre du}]
  \hyperref[TEI.model.pPart.editorial]{model.pPart.editorial}
    \item[{Contenu dans}]
  
    \item[analysis: ]
   \hyperref[TEI.cl]{cl} \hyperref[TEI.pc]{pc} \hyperref[TEI.phr]{phr} \hyperref[TEI.s]{s} \hyperref[TEI.span]{span} \hyperref[TEI.w]{w}\par 
    \item[core: ]
   \hyperref[TEI.abbr]{abbr} \hyperref[TEI.add]{add} \hyperref[TEI.addrLine]{addrLine} \hyperref[TEI.author]{author} \hyperref[TEI.bibl]{bibl} \hyperref[TEI.biblScope]{biblScope} \hyperref[TEI.citedRange]{citedRange} \hyperref[TEI.corr]{corr} \hyperref[TEI.date]{date} \hyperref[TEI.del]{del} \hyperref[TEI.desc]{desc} \hyperref[TEI.distinct]{distinct} \hyperref[TEI.editor]{editor} \hyperref[TEI.email]{email} \hyperref[TEI.emph]{emph} \hyperref[TEI.expan]{expan} \hyperref[TEI.foreign]{foreign} \hyperref[TEI.gloss]{gloss} \hyperref[TEI.head]{head} \hyperref[TEI.headItem]{headItem} \hyperref[TEI.headLabel]{headLabel} \hyperref[TEI.hi]{hi} \hyperref[TEI.item]{item} \hyperref[TEI.l]{l} \hyperref[TEI.label]{label} \hyperref[TEI.measure]{measure} \hyperref[TEI.meeting]{meeting} \hyperref[TEI.mentioned]{mentioned} \hyperref[TEI.name]{name} \hyperref[TEI.note]{note} \hyperref[TEI.num]{num} \hyperref[TEI.orig]{orig} \hyperref[TEI.p]{p} \hyperref[TEI.pubPlace]{pubPlace} \hyperref[TEI.publisher]{publisher} \hyperref[TEI.q]{q} \hyperref[TEI.quote]{quote} \hyperref[TEI.ref]{ref} \hyperref[TEI.reg]{reg} \hyperref[TEI.resp]{resp} \hyperref[TEI.rs]{rs} \hyperref[TEI.said]{said} \hyperref[TEI.sic]{sic} \hyperref[TEI.soCalled]{soCalled} \hyperref[TEI.speaker]{speaker} \hyperref[TEI.stage]{stage} \hyperref[TEI.street]{street} \hyperref[TEI.term]{term} \hyperref[TEI.textLang]{textLang} \hyperref[TEI.time]{time} \hyperref[TEI.title]{title} \hyperref[TEI.unclear]{unclear}\par 
    \item[figures: ]
   \hyperref[TEI.cell]{cell} \hyperref[TEI.figDesc]{figDesc}\par 
    \item[header: ]
   \hyperref[TEI.authority]{authority} \hyperref[TEI.change]{change} \hyperref[TEI.classCode]{classCode} \hyperref[TEI.creation]{creation} \hyperref[TEI.distributor]{distributor} \hyperref[TEI.edition]{edition} \hyperref[TEI.extent]{extent} \hyperref[TEI.funder]{funder} \hyperref[TEI.language]{language} \hyperref[TEI.licence]{licence} \hyperref[TEI.rendition]{rendition}\par 
    \item[iso-fs: ]
   \hyperref[TEI.fDescr]{fDescr} \hyperref[TEI.fsDescr]{fsDescr}\par 
    \item[linking: ]
   \hyperref[TEI.ab]{ab} \hyperref[TEI.seg]{seg}\par 
    \item[msdescription: ]
   \hyperref[TEI.accMat]{accMat} \hyperref[TEI.acquisition]{acquisition} \hyperref[TEI.additions]{additions} \hyperref[TEI.catchwords]{catchwords} \hyperref[TEI.collation]{collation} \hyperref[TEI.colophon]{colophon} \hyperref[TEI.condition]{condition} \hyperref[TEI.custEvent]{custEvent} \hyperref[TEI.decoNote]{decoNote} \hyperref[TEI.explicit]{explicit} \hyperref[TEI.filiation]{filiation} \hyperref[TEI.finalRubric]{finalRubric} \hyperref[TEI.foliation]{foliation} \hyperref[TEI.heraldry]{heraldry} \hyperref[TEI.incipit]{incipit} \hyperref[TEI.layout]{layout} \hyperref[TEI.material]{material} \hyperref[TEI.musicNotation]{musicNotation} \hyperref[TEI.objectType]{objectType} \hyperref[TEI.origDate]{origDate} \hyperref[TEI.origPlace]{origPlace} \hyperref[TEI.origin]{origin} \hyperref[TEI.provenance]{provenance} \hyperref[TEI.rubric]{rubric} \hyperref[TEI.secFol]{secFol} \hyperref[TEI.signatures]{signatures} \hyperref[TEI.source]{source} \hyperref[TEI.stamp]{stamp} \hyperref[TEI.summary]{summary} \hyperref[TEI.support]{support} \hyperref[TEI.surrogates]{surrogates} \hyperref[TEI.typeNote]{typeNote} \hyperref[TEI.watermark]{watermark}\par 
    \item[namesdates: ]
   \hyperref[TEI.addName]{addName} \hyperref[TEI.affiliation]{affiliation} \hyperref[TEI.country]{country} \hyperref[TEI.forename]{forename} \hyperref[TEI.genName]{genName} \hyperref[TEI.geogName]{geogName} \hyperref[TEI.nameLink]{nameLink} \hyperref[TEI.orgName]{orgName} \hyperref[TEI.persName]{persName} \hyperref[TEI.placeName]{placeName} \hyperref[TEI.region]{region} \hyperref[TEI.roleName]{roleName} \hyperref[TEI.settlement]{settlement} \hyperref[TEI.surname]{surname}\par 
    \item[textstructure: ]
   \hyperref[TEI.docAuthor]{docAuthor} \hyperref[TEI.docDate]{docDate} \hyperref[TEI.docEdition]{docEdition} \hyperref[TEI.titlePart]{titlePart}\par 
    \item[transcr: ]
   \hyperref[TEI.damage]{damage} \hyperref[TEI.fw]{fw} \hyperref[TEI.metamark]{metamark} \hyperref[TEI.mod]{mod} \hyperref[TEI.restore]{restore} \hyperref[TEI.retrace]{retrace} \hyperref[TEI.secl]{secl} \hyperref[TEI.supplied]{supplied} \hyperref[TEI.surplus]{surplus}
    \item[{Peut contenir}]
  
    \item[core: ]
   \hyperref[TEI.add]{add} \hyperref[TEI.cb]{cb} \hyperref[TEI.del]{del} \hyperref[TEI.gb]{gb} \hyperref[TEI.lb]{lb} \hyperref[TEI.milestone]{milestone} \hyperref[TEI.pb]{pb}\par 
    \item[linking: ]
   \hyperref[TEI.anchor]{anchor}\par 
    \item[transcr: ]
   \hyperref[TEI.fw]{fw}
    \item[{Exemple}]
  \leavevmode\bgroup\exampleFont \begin{shaded}\noindent\mbox{}... are all included. {<\textbf{del}\hspace*{6pt}{hand}="{\#RG}">}It is{</\textbf{del}>}\mbox{}\newline 
{<\textbf{subst}>}\mbox{}\newline 
\hspace*{6pt}{<\textbf{add}>}T{</\textbf{add}>}\mbox{}\newline 
\hspace*{6pt}{<\textbf{del}>}t{</\textbf{del}>}\mbox{}\newline 
{</\textbf{subst}>}he expressed\mbox{}\newline 
\end{shaded}\egroup 


    \item[{Exemple}]
  \leavevmode\bgroup\exampleFont \begin{shaded}\noindent\mbox{} that he and his Sister Miſs D — {<\textbf{lb}/>}who always lived with him, wd. be {<\textbf{subst}>}\mbox{}\newline 
\hspace*{6pt}{<\textbf{del}>}very{</\textbf{del}>}\mbox{}\newline 
\hspace*{6pt}{<\textbf{lb}/>}\mbox{}\newline 
\hspace*{6pt}{<\textbf{add}>}principally{</\textbf{add}>}\mbox{}\newline 
{</\textbf{subst}>} remembered in her Will.\mbox{}\newline 
\end{shaded}\egroup 


    \item[{Exemple}]
  \leavevmode\bgroup\exampleFont \begin{shaded}\noindent\mbox{}{<\textbf{ab}>}τ{<\textbf{subst}>}\mbox{}\newline 
\hspace*{6pt}\hspace*{6pt}{<\textbf{add}\hspace*{6pt}{place}="{above}">}ῶν{</\textbf{add}>}\mbox{}\newline 
\hspace*{6pt}\hspace*{6pt}{<\textbf{del}>}α{</\textbf{del}>}\mbox{}\newline 
\hspace*{6pt}{</\textbf{subst}>}\mbox{}\newline 
 συνκυρόντ{<\textbf{subst}>}\mbox{}\newline 
\hspace*{6pt}\hspace*{6pt}{<\textbf{add}\hspace*{6pt}{place}="{above}">}ων{</\textbf{add}>}\mbox{}\newline 
\hspace*{6pt}\hspace*{6pt}{<\textbf{del}>}α{</\textbf{del}>}\mbox{}\newline 
\hspace*{6pt}{</\textbf{subst}>}\mbox{}\newline 
 ἐργαστηρί{<\textbf{subst}>}\mbox{}\newline 
\hspace*{6pt}\hspace*{6pt}{<\textbf{add}\hspace*{6pt}{place}="{above}">}ων{</\textbf{add}>}\mbox{}\newline 
\hspace*{6pt}\hspace*{6pt}{<\textbf{del}>}α{</\textbf{del}>}\mbox{}\newline 
\hspace*{6pt}{</\textbf{subst}>}\mbox{}\newline 
{</\textbf{ab}>}\end{shaded}\egroup 


    \item[{Exemple}]
  \leavevmode\bgroup\exampleFont \begin{shaded}\noindent\mbox{}{<\textbf{subst}>}\mbox{}\newline 
\hspace*{6pt}{<\textbf{del}>}\mbox{}\newline 
\hspace*{6pt}\hspace*{6pt}{<\textbf{gap}\hspace*{6pt}{quantity}="{5}"\hspace*{6pt}{reason}="{illegible}"\mbox{}\newline 
\hspace*{6pt}\hspace*{6pt}\hspace*{6pt}{unit}="{character}"/>}\mbox{}\newline 
\hspace*{6pt}{</\textbf{del}>}\mbox{}\newline 
\hspace*{6pt}{<\textbf{add}>}apple{</\textbf{add}>}\mbox{}\newline 
{</\textbf{subst}>}\end{shaded}\egroup 


    \item[{Schematron}]
   <s:assert test="child::tei:add and child::tei:del"> <s:name/> must have at least one child add and at least one child del</s:assert>
    \item[{Modèle de contenu}]
  \mbox{}\hfill\\[-10pt]\begin{Verbatim}[fontsize=\small]
<content>
 <alternate maxOccurs="unbounded"
  minOccurs="1">
  <elementRef key="add"/>
  <elementRef key="del"/>
  <classRef key="model.milestoneLike"/>
 </alternate>
</content>
    
\end{Verbatim}

    \item[{Schéma Declaration}]
  \mbox{}\hfill\\[-10pt]\begin{Verbatim}[fontsize=\small]
element subst
{
   tei_att.global.attributes,
   tei_att.transcriptional.attributes,
   ( tei_add | tei_del | tei_model.milestoneLike )+
}
\end{Verbatim}

\end{reflist}  \index{substJoin=<substJoin>|oddindex}
\begin{reflist}
\item[]\begin{specHead}{TEI.substJoin}{<substJoin> }(jointure de substitution) identifies a series of possibly fragmented additions, deletions or other revisions on a manuscript that combine to make up a single intervention in the text [\xref{http://www.tei-c.org/release/doc/tei-p5-doc/en/html/PH.html\#PHSU}{11.3.1.5. Substitutions}]\end{specHead} 
    \item[{Module}]
  transcr
    \item[{Attributs}]
  Attributs \hyperref[TEI.att.global]{att.global} (\textit{@xml:id}, \textit{@n}, \textit{@xml:lang}, \textit{@xml:base}, \textit{@xml:space})  (\hyperref[TEI.att.global.rendition]{att.global.rendition} (\textit{@rend}, \textit{@style}, \textit{@rendition})) (\hyperref[TEI.att.global.linking]{att.global.linking} (\textit{@corresp}, \textit{@synch}, \textit{@sameAs}, \textit{@copyOf}, \textit{@next}, \textit{@prev}, \textit{@exclude}, \textit{@select})) (\hyperref[TEI.att.global.analytic]{att.global.analytic} (\textit{@ana})) (\hyperref[TEI.att.global.facs]{att.global.facs} (\textit{@facs})) (\hyperref[TEI.att.global.change]{att.global.change} (\textit{@change})) (\hyperref[TEI.att.global.responsibility]{att.global.responsibility} (\textit{@cert}, \textit{@resp})) (\hyperref[TEI.att.global.source]{att.global.source} (\textit{@source})) \hyperref[TEI.att.pointing]{att.pointing} (\textit{@targetLang}, \textit{@target}, \textit{@evaluate}) \hyperref[TEI.att.transcriptional]{att.transcriptional} (\textit{@status}, \textit{@cause}, \textit{@seq})  (\hyperref[TEI.att.editLike]{att.editLike} (\textit{@evidence}, \textit{@instant}) (\hyperref[TEI.att.dimensions]{att.dimensions} (\textit{@unit}, \textit{@quantity}, \textit{@extent}, \textit{@precision}, \textit{@scope}) (\hyperref[TEI.att.ranging]{att.ranging} (\textit{@atLeast}, \textit{@atMost}, \textit{@min}, \textit{@max}, \textit{@confidence})) ) ) (\hyperref[TEI.att.written]{att.written} (\textit{@hand}))
    \item[{Membre du}]
  \hyperref[TEI.model.global.meta]{model.global.meta}
    \item[{Contenu dans}]
  
    \item[analysis: ]
   \hyperref[TEI.cl]{cl} \hyperref[TEI.m]{m} \hyperref[TEI.phr]{phr} \hyperref[TEI.s]{s} \hyperref[TEI.span]{span} \hyperref[TEI.w]{w}\par 
    \item[core: ]
   \hyperref[TEI.abbr]{abbr} \hyperref[TEI.add]{add} \hyperref[TEI.addrLine]{addrLine} \hyperref[TEI.address]{address} \hyperref[TEI.author]{author} \hyperref[TEI.bibl]{bibl} \hyperref[TEI.biblScope]{biblScope} \hyperref[TEI.cit]{cit} \hyperref[TEI.citedRange]{citedRange} \hyperref[TEI.corr]{corr} \hyperref[TEI.date]{date} \hyperref[TEI.del]{del} \hyperref[TEI.distinct]{distinct} \hyperref[TEI.editor]{editor} \hyperref[TEI.email]{email} \hyperref[TEI.emph]{emph} \hyperref[TEI.expan]{expan} \hyperref[TEI.foreign]{foreign} \hyperref[TEI.gloss]{gloss} \hyperref[TEI.head]{head} \hyperref[TEI.headItem]{headItem} \hyperref[TEI.headLabel]{headLabel} \hyperref[TEI.hi]{hi} \hyperref[TEI.imprint]{imprint} \hyperref[TEI.item]{item} \hyperref[TEI.l]{l} \hyperref[TEI.label]{label} \hyperref[TEI.lg]{lg} \hyperref[TEI.list]{list} \hyperref[TEI.measure]{measure} \hyperref[TEI.mentioned]{mentioned} \hyperref[TEI.name]{name} \hyperref[TEI.note]{note} \hyperref[TEI.num]{num} \hyperref[TEI.orig]{orig} \hyperref[TEI.p]{p} \hyperref[TEI.pubPlace]{pubPlace} \hyperref[TEI.publisher]{publisher} \hyperref[TEI.q]{q} \hyperref[TEI.quote]{quote} \hyperref[TEI.ref]{ref} \hyperref[TEI.reg]{reg} \hyperref[TEI.resp]{resp} \hyperref[TEI.rs]{rs} \hyperref[TEI.said]{said} \hyperref[TEI.series]{series} \hyperref[TEI.sic]{sic} \hyperref[TEI.soCalled]{soCalled} \hyperref[TEI.sp]{sp} \hyperref[TEI.speaker]{speaker} \hyperref[TEI.stage]{stage} \hyperref[TEI.street]{street} \hyperref[TEI.term]{term} \hyperref[TEI.textLang]{textLang} \hyperref[TEI.time]{time} \hyperref[TEI.title]{title} \hyperref[TEI.unclear]{unclear}\par 
    \item[figures: ]
   \hyperref[TEI.cell]{cell} \hyperref[TEI.figure]{figure} \hyperref[TEI.table]{table}\par 
    \item[header: ]
   \hyperref[TEI.authority]{authority} \hyperref[TEI.change]{change} \hyperref[TEI.classCode]{classCode} \hyperref[TEI.distributor]{distributor} \hyperref[TEI.edition]{edition} \hyperref[TEI.extent]{extent} \hyperref[TEI.funder]{funder} \hyperref[TEI.language]{language} \hyperref[TEI.licence]{licence}\par 
    \item[linking: ]
   \hyperref[TEI.ab]{ab} \hyperref[TEI.seg]{seg}\par 
    \item[msdescription: ]
   \hyperref[TEI.accMat]{accMat} \hyperref[TEI.acquisition]{acquisition} \hyperref[TEI.additions]{additions} \hyperref[TEI.catchwords]{catchwords} \hyperref[TEI.collation]{collation} \hyperref[TEI.colophon]{colophon} \hyperref[TEI.condition]{condition} \hyperref[TEI.custEvent]{custEvent} \hyperref[TEI.decoNote]{decoNote} \hyperref[TEI.explicit]{explicit} \hyperref[TEI.filiation]{filiation} \hyperref[TEI.finalRubric]{finalRubric} \hyperref[TEI.foliation]{foliation} \hyperref[TEI.heraldry]{heraldry} \hyperref[TEI.incipit]{incipit} \hyperref[TEI.layout]{layout} \hyperref[TEI.material]{material} \hyperref[TEI.msItem]{msItem} \hyperref[TEI.musicNotation]{musicNotation} \hyperref[TEI.objectType]{objectType} \hyperref[TEI.origDate]{origDate} \hyperref[TEI.origPlace]{origPlace} \hyperref[TEI.origin]{origin} \hyperref[TEI.provenance]{provenance} \hyperref[TEI.rubric]{rubric} \hyperref[TEI.secFol]{secFol} \hyperref[TEI.signatures]{signatures} \hyperref[TEI.source]{source} \hyperref[TEI.stamp]{stamp} \hyperref[TEI.summary]{summary} \hyperref[TEI.support]{support} \hyperref[TEI.surrogates]{surrogates} \hyperref[TEI.typeNote]{typeNote} \hyperref[TEI.watermark]{watermark}\par 
    \item[namesdates: ]
   \hyperref[TEI.addName]{addName} \hyperref[TEI.affiliation]{affiliation} \hyperref[TEI.country]{country} \hyperref[TEI.forename]{forename} \hyperref[TEI.genName]{genName} \hyperref[TEI.geogName]{geogName} \hyperref[TEI.nameLink]{nameLink} \hyperref[TEI.orgName]{orgName} \hyperref[TEI.persName]{persName} \hyperref[TEI.person]{person} \hyperref[TEI.personGrp]{personGrp} \hyperref[TEI.persona]{persona} \hyperref[TEI.placeName]{placeName} \hyperref[TEI.region]{region} \hyperref[TEI.roleName]{roleName} \hyperref[TEI.settlement]{settlement} \hyperref[TEI.surname]{surname}\par 
    \item[spoken: ]
   \hyperref[TEI.annotationBlock]{annotationBlock}\par 
    \item[standOff: ]
   \hyperref[TEI.listAnnotation]{listAnnotation}\par 
    \item[textstructure: ]
   \hyperref[TEI.back]{back} \hyperref[TEI.body]{body} \hyperref[TEI.div]{div} \hyperref[TEI.docAuthor]{docAuthor} \hyperref[TEI.docDate]{docDate} \hyperref[TEI.docEdition]{docEdition} \hyperref[TEI.docTitle]{docTitle} \hyperref[TEI.floatingText]{floatingText} \hyperref[TEI.front]{front} \hyperref[TEI.group]{group} \hyperref[TEI.text]{text} \hyperref[TEI.titlePage]{titlePage} \hyperref[TEI.titlePart]{titlePart}\par 
    \item[transcr: ]
   \hyperref[TEI.damage]{damage} \hyperref[TEI.fw]{fw} \hyperref[TEI.line]{line} \hyperref[TEI.metamark]{metamark} \hyperref[TEI.mod]{mod} \hyperref[TEI.restore]{restore} \hyperref[TEI.retrace]{retrace} \hyperref[TEI.secl]{secl} \hyperref[TEI.sourceDoc]{sourceDoc} \hyperref[TEI.supplied]{supplied} \hyperref[TEI.surface]{surface} \hyperref[TEI.surfaceGrp]{surfaceGrp} \hyperref[TEI.surplus]{surplus} \hyperref[TEI.zone]{zone}
    \item[{Peut contenir}]
  
    \item[core: ]
   \hyperref[TEI.desc]{desc}
    \item[{Exemple}]
  \leavevmode\bgroup\exampleFont \begin{shaded}\noindent\mbox{} While {<\textbf{del}\hspace*{6pt}{xml:id}="{r112}">}pondering{</\textbf{del}>} thus {<\textbf{add}\hspace*{6pt}{xml:id}="{r113}">}she mus'd{</\textbf{add}>}, her pinions fann'd\mbox{}\newline 
{<\textbf{substJoin}\hspace*{6pt}{target}="{\#r112 \#r113}"/>}\end{shaded}\egroup 


    \item[{Modèle de contenu}]
  \mbox{}\hfill\\[-10pt]\begin{Verbatim}[fontsize=\small]
<content>
 <alternate maxOccurs="unbounded"
  minOccurs="0">
  <classRef key="model.descLike"/>
  <classRef key="model.certLike"/>
 </alternate>
</content>
    
\end{Verbatim}

    \item[{Schéma Declaration}]
  \mbox{}\hfill\\[-10pt]\begin{Verbatim}[fontsize=\small]
element substJoin
{
   tei_att.global.attributes,
   tei_att.pointing.attributes,
   tei_att.transcriptional.attributes,
   ( tei_model.descLike | tei_model.certLike )*
}
\end{Verbatim}

\end{reflist}  \index{summary=<summary>|oddindex}
\begin{reflist}
\item[]\begin{specHead}{TEI.summary}{<summary> }contains an overview of the available information concerning some aspect of an item (for example, its intellectual content, history, layout, typography etc.) as a complement or alternative to the more detailed information carried by more specific elements. [\xref{http://www.tei-c.org/release/doc/tei-p5-doc/en/html/MS.html\#msco}{10.6. Intellectual Content}]\end{specHead} 
    \item[{Module}]
  msdescription
    \item[{Attributs}]
  Attributs \hyperref[TEI.att.global]{att.global} (\textit{@xml:id}, \textit{@n}, \textit{@xml:lang}, \textit{@xml:base}, \textit{@xml:space})  (\hyperref[TEI.att.global.rendition]{att.global.rendition} (\textit{@rend}, \textit{@style}, \textit{@rendition})) (\hyperref[TEI.att.global.linking]{att.global.linking} (\textit{@corresp}, \textit{@synch}, \textit{@sameAs}, \textit{@copyOf}, \textit{@next}, \textit{@prev}, \textit{@exclude}, \textit{@select})) (\hyperref[TEI.att.global.analytic]{att.global.analytic} (\textit{@ana})) (\hyperref[TEI.att.global.facs]{att.global.facs} (\textit{@facs})) (\hyperref[TEI.att.global.change]{att.global.change} (\textit{@change})) (\hyperref[TEI.att.global.responsibility]{att.global.responsibility} (\textit{@cert}, \textit{@resp})) (\hyperref[TEI.att.global.source]{att.global.source} (\textit{@source}))
    \item[{Contenu dans}]
  
    \item[msdescription: ]
   \hyperref[TEI.decoDesc]{decoDesc} \hyperref[TEI.handDesc]{handDesc} \hyperref[TEI.history]{history} \hyperref[TEI.layoutDesc]{layoutDesc} \hyperref[TEI.msContents]{msContents} \hyperref[TEI.scriptDesc]{scriptDesc} \hyperref[TEI.sealDesc]{sealDesc} \hyperref[TEI.typeDesc]{typeDesc}
    \item[{Peut contenir}]
  
    \item[analysis: ]
   \hyperref[TEI.c]{c} \hyperref[TEI.cl]{cl} \hyperref[TEI.interp]{interp} \hyperref[TEI.interpGrp]{interpGrp} \hyperref[TEI.m]{m} \hyperref[TEI.pc]{pc} \hyperref[TEI.phr]{phr} \hyperref[TEI.s]{s} \hyperref[TEI.span]{span} \hyperref[TEI.spanGrp]{spanGrp} \hyperref[TEI.w]{w}\par 
    \item[core: ]
   \hyperref[TEI.abbr]{abbr} \hyperref[TEI.add]{add} \hyperref[TEI.address]{address} \hyperref[TEI.bibl]{bibl} \hyperref[TEI.biblStruct]{biblStruct} \hyperref[TEI.binaryObject]{binaryObject} \hyperref[TEI.cb]{cb} \hyperref[TEI.choice]{choice} \hyperref[TEI.cit]{cit} \hyperref[TEI.corr]{corr} \hyperref[TEI.date]{date} \hyperref[TEI.del]{del} \hyperref[TEI.desc]{desc} \hyperref[TEI.distinct]{distinct} \hyperref[TEI.email]{email} \hyperref[TEI.emph]{emph} \hyperref[TEI.expan]{expan} \hyperref[TEI.foreign]{foreign} \hyperref[TEI.gap]{gap} \hyperref[TEI.gb]{gb} \hyperref[TEI.gloss]{gloss} \hyperref[TEI.graphic]{graphic} \hyperref[TEI.hi]{hi} \hyperref[TEI.index]{index} \hyperref[TEI.l]{l} \hyperref[TEI.label]{label} \hyperref[TEI.lb]{lb} \hyperref[TEI.lg]{lg} \hyperref[TEI.list]{list} \hyperref[TEI.listBibl]{listBibl} \hyperref[TEI.measure]{measure} \hyperref[TEI.measureGrp]{measureGrp} \hyperref[TEI.media]{media} \hyperref[TEI.mentioned]{mentioned} \hyperref[TEI.milestone]{milestone} \hyperref[TEI.name]{name} \hyperref[TEI.note]{note} \hyperref[TEI.num]{num} \hyperref[TEI.orig]{orig} \hyperref[TEI.p]{p} \hyperref[TEI.pb]{pb} \hyperref[TEI.ptr]{ptr} \hyperref[TEI.q]{q} \hyperref[TEI.quote]{quote} \hyperref[TEI.ref]{ref} \hyperref[TEI.reg]{reg} \hyperref[TEI.rs]{rs} \hyperref[TEI.said]{said} \hyperref[TEI.sic]{sic} \hyperref[TEI.soCalled]{soCalled} \hyperref[TEI.sp]{sp} \hyperref[TEI.stage]{stage} \hyperref[TEI.term]{term} \hyperref[TEI.time]{time} \hyperref[TEI.title]{title} \hyperref[TEI.unclear]{unclear}\par 
    \item[derived-module-tei.istex: ]
   \hyperref[TEI.math]{math} \hyperref[TEI.mrow]{mrow}\par 
    \item[figures: ]
   \hyperref[TEI.figure]{figure} \hyperref[TEI.formula]{formula} \hyperref[TEI.notatedMusic]{notatedMusic} \hyperref[TEI.table]{table}\par 
    \item[header: ]
   \hyperref[TEI.biblFull]{biblFull} \hyperref[TEI.idno]{idno}\par 
    \item[iso-fs: ]
   \hyperref[TEI.fLib]{fLib} \hyperref[TEI.fs]{fs} \hyperref[TEI.fvLib]{fvLib}\par 
    \item[linking: ]
   \hyperref[TEI.ab]{ab} \hyperref[TEI.alt]{alt} \hyperref[TEI.altGrp]{altGrp} \hyperref[TEI.anchor]{anchor} \hyperref[TEI.join]{join} \hyperref[TEI.joinGrp]{joinGrp} \hyperref[TEI.link]{link} \hyperref[TEI.linkGrp]{linkGrp} \hyperref[TEI.seg]{seg} \hyperref[TEI.timeline]{timeline}\par 
    \item[msdescription: ]
   \hyperref[TEI.catchwords]{catchwords} \hyperref[TEI.depth]{depth} \hyperref[TEI.dim]{dim} \hyperref[TEI.dimensions]{dimensions} \hyperref[TEI.height]{height} \hyperref[TEI.heraldry]{heraldry} \hyperref[TEI.locus]{locus} \hyperref[TEI.locusGrp]{locusGrp} \hyperref[TEI.material]{material} \hyperref[TEI.msDesc]{msDesc} \hyperref[TEI.objectType]{objectType} \hyperref[TEI.origDate]{origDate} \hyperref[TEI.origPlace]{origPlace} \hyperref[TEI.secFol]{secFol} \hyperref[TEI.signatures]{signatures} \hyperref[TEI.source]{source} \hyperref[TEI.stamp]{stamp} \hyperref[TEI.watermark]{watermark} \hyperref[TEI.width]{width}\par 
    \item[namesdates: ]
   \hyperref[TEI.addName]{addName} \hyperref[TEI.affiliation]{affiliation} \hyperref[TEI.country]{country} \hyperref[TEI.forename]{forename} \hyperref[TEI.genName]{genName} \hyperref[TEI.geogName]{geogName} \hyperref[TEI.listOrg]{listOrg} \hyperref[TEI.listPlace]{listPlace} \hyperref[TEI.location]{location} \hyperref[TEI.nameLink]{nameLink} \hyperref[TEI.orgName]{orgName} \hyperref[TEI.persName]{persName} \hyperref[TEI.placeName]{placeName} \hyperref[TEI.region]{region} \hyperref[TEI.roleName]{roleName} \hyperref[TEI.settlement]{settlement} \hyperref[TEI.state]{state} \hyperref[TEI.surname]{surname}\par 
    \item[spoken: ]
   \hyperref[TEI.annotationBlock]{annotationBlock}\par 
    \item[textstructure: ]
   \hyperref[TEI.floatingText]{floatingText}\par 
    \item[transcr: ]
   \hyperref[TEI.addSpan]{addSpan} \hyperref[TEI.am]{am} \hyperref[TEI.damage]{damage} \hyperref[TEI.damageSpan]{damageSpan} \hyperref[TEI.delSpan]{delSpan} \hyperref[TEI.ex]{ex} \hyperref[TEI.fw]{fw} \hyperref[TEI.handShift]{handShift} \hyperref[TEI.listTranspose]{listTranspose} \hyperref[TEI.metamark]{metamark} \hyperref[TEI.mod]{mod} \hyperref[TEI.redo]{redo} \hyperref[TEI.restore]{restore} \hyperref[TEI.retrace]{retrace} \hyperref[TEI.secl]{secl} \hyperref[TEI.space]{space} \hyperref[TEI.subst]{subst} \hyperref[TEI.substJoin]{substJoin} \hyperref[TEI.supplied]{supplied} \hyperref[TEI.surplus]{surplus} \hyperref[TEI.undo]{undo}\par des données textuelles
    \item[{Exemple}]
  \leavevmode\bgroup\exampleFont \begin{shaded}\noindent\mbox{}{<\textbf{summary}>} Cet item est formé de trois livres, d'un prologue et d'un épilogue.{</\textbf{summary}>}\end{shaded}\egroup 


    \item[{Modèle de contenu}]
  \mbox{}\hfill\\[-10pt]\begin{Verbatim}[fontsize=\small]
<content>
 <macroRef key="macro.specialPara"/>
</content>
    
\end{Verbatim}

    \item[{Schéma Declaration}]
  \mbox{}\hfill\\[-10pt]\begin{Verbatim}[fontsize=\small]
element summary { tei_att.global.attributes, tei_macro.specialPara }
\end{Verbatim}

\end{reflist}  \index{supplied=<supplied>|oddindex}\index{reason=@reason!<supplied>|oddindex}
\begin{reflist}
\item[]\begin{specHead}{TEI.supplied}{<supplied> }(texte restitué) permet d'encoder du texte restitué par l'auteur de la transcription ou par l'éditeur pour une raison quelconque, le plus souvent parce que le texte du document original ne peut être lu, par suite de dommages matériels ou de lacunes. [\xref{http://www.tei-c.org/release/doc/tei-p5-doc/en/html/PH.html\#PHDA}{11.3.3.1. Damage, Illegibility, and Supplied Text}]\end{specHead} 
    \item[{Module}]
  transcr
    \item[{Attributs}]
  Attributs \hyperref[TEI.att.global]{att.global} (\textit{@xml:id}, \textit{@n}, \textit{@xml:lang}, \textit{@xml:base}, \textit{@xml:space})  (\hyperref[TEI.att.global.rendition]{att.global.rendition} (\textit{@rend}, \textit{@style}, \textit{@rendition})) (\hyperref[TEI.att.global.linking]{att.global.linking} (\textit{@corresp}, \textit{@synch}, \textit{@sameAs}, \textit{@copyOf}, \textit{@next}, \textit{@prev}, \textit{@exclude}, \textit{@select})) (\hyperref[TEI.att.global.analytic]{att.global.analytic} (\textit{@ana})) (\hyperref[TEI.att.global.facs]{att.global.facs} (\textit{@facs})) (\hyperref[TEI.att.global.change]{att.global.change} (\textit{@change})) (\hyperref[TEI.att.global.responsibility]{att.global.responsibility} (\textit{@cert}, \textit{@resp})) (\hyperref[TEI.att.global.source]{att.global.source} (\textit{@source})) \hyperref[TEI.att.editLike]{att.editLike} (\textit{@evidence}, \textit{@instant})  (\hyperref[TEI.att.dimensions]{att.dimensions} (\textit{@unit}, \textit{@quantity}, \textit{@extent}, \textit{@precision}, \textit{@scope}) (\hyperref[TEI.att.ranging]{att.ranging} (\textit{@atLeast}, \textit{@atMost}, \textit{@min}, \textit{@max}, \textit{@confidence})) ) \hfil\\[-10pt]\begin{sansreflist}
    \item[@reason]
  donne la raison pour laquelle on a dû restituer le texte.
\begin{reflist}
    \item[{Statut}]
  Optionel
    \item[{Type de données}]
  1–∞ occurrences de \hyperref[TEI.teidata.word]{teidata.word} séparé par un espace
\end{reflist}  
\end{sansreflist}  
    \item[{Membre du}]
  \hyperref[TEI.model.choicePart]{model.choicePart} \hyperref[TEI.model.pPart.transcriptional]{model.pPart.transcriptional}
    \item[{Contenu dans}]
  
    \item[analysis: ]
   \hyperref[TEI.cl]{cl} \hyperref[TEI.pc]{pc} \hyperref[TEI.phr]{phr} \hyperref[TEI.s]{s} \hyperref[TEI.w]{w}\par 
    \item[core: ]
   \hyperref[TEI.abbr]{abbr} \hyperref[TEI.add]{add} \hyperref[TEI.addrLine]{addrLine} \hyperref[TEI.author]{author} \hyperref[TEI.bibl]{bibl} \hyperref[TEI.biblScope]{biblScope} \hyperref[TEI.choice]{choice} \hyperref[TEI.citedRange]{citedRange} \hyperref[TEI.corr]{corr} \hyperref[TEI.date]{date} \hyperref[TEI.del]{del} \hyperref[TEI.distinct]{distinct} \hyperref[TEI.editor]{editor} \hyperref[TEI.email]{email} \hyperref[TEI.emph]{emph} \hyperref[TEI.expan]{expan} \hyperref[TEI.foreign]{foreign} \hyperref[TEI.gloss]{gloss} \hyperref[TEI.head]{head} \hyperref[TEI.headItem]{headItem} \hyperref[TEI.headLabel]{headLabel} \hyperref[TEI.hi]{hi} \hyperref[TEI.item]{item} \hyperref[TEI.l]{l} \hyperref[TEI.label]{label} \hyperref[TEI.measure]{measure} \hyperref[TEI.mentioned]{mentioned} \hyperref[TEI.name]{name} \hyperref[TEI.note]{note} \hyperref[TEI.num]{num} \hyperref[TEI.orig]{orig} \hyperref[TEI.p]{p} \hyperref[TEI.pubPlace]{pubPlace} \hyperref[TEI.publisher]{publisher} \hyperref[TEI.q]{q} \hyperref[TEI.quote]{quote} \hyperref[TEI.ref]{ref} \hyperref[TEI.reg]{reg} \hyperref[TEI.rs]{rs} \hyperref[TEI.said]{said} \hyperref[TEI.sic]{sic} \hyperref[TEI.soCalled]{soCalled} \hyperref[TEI.speaker]{speaker} \hyperref[TEI.stage]{stage} \hyperref[TEI.street]{street} \hyperref[TEI.term]{term} \hyperref[TEI.textLang]{textLang} \hyperref[TEI.time]{time} \hyperref[TEI.title]{title} \hyperref[TEI.unclear]{unclear}\par 
    \item[figures: ]
   \hyperref[TEI.cell]{cell}\par 
    \item[header: ]
   \hyperref[TEI.change]{change} \hyperref[TEI.distributor]{distributor} \hyperref[TEI.edition]{edition} \hyperref[TEI.extent]{extent} \hyperref[TEI.licence]{licence}\par 
    \item[linking: ]
   \hyperref[TEI.ab]{ab} \hyperref[TEI.seg]{seg}\par 
    \item[msdescription: ]
   \hyperref[TEI.accMat]{accMat} \hyperref[TEI.acquisition]{acquisition} \hyperref[TEI.additions]{additions} \hyperref[TEI.catchwords]{catchwords} \hyperref[TEI.collation]{collation} \hyperref[TEI.colophon]{colophon} \hyperref[TEI.condition]{condition} \hyperref[TEI.custEvent]{custEvent} \hyperref[TEI.decoNote]{decoNote} \hyperref[TEI.explicit]{explicit} \hyperref[TEI.filiation]{filiation} \hyperref[TEI.finalRubric]{finalRubric} \hyperref[TEI.foliation]{foliation} \hyperref[TEI.heraldry]{heraldry} \hyperref[TEI.incipit]{incipit} \hyperref[TEI.layout]{layout} \hyperref[TEI.material]{material} \hyperref[TEI.musicNotation]{musicNotation} \hyperref[TEI.objectType]{objectType} \hyperref[TEI.origDate]{origDate} \hyperref[TEI.origPlace]{origPlace} \hyperref[TEI.origin]{origin} \hyperref[TEI.provenance]{provenance} \hyperref[TEI.rubric]{rubric} \hyperref[TEI.secFol]{secFol} \hyperref[TEI.signatures]{signatures} \hyperref[TEI.source]{source} \hyperref[TEI.stamp]{stamp} \hyperref[TEI.summary]{summary} \hyperref[TEI.support]{support} \hyperref[TEI.surrogates]{surrogates} \hyperref[TEI.typeNote]{typeNote} \hyperref[TEI.watermark]{watermark}\par 
    \item[namesdates: ]
   \hyperref[TEI.addName]{addName} \hyperref[TEI.affiliation]{affiliation} \hyperref[TEI.country]{country} \hyperref[TEI.forename]{forename} \hyperref[TEI.genName]{genName} \hyperref[TEI.geogName]{geogName} \hyperref[TEI.nameLink]{nameLink} \hyperref[TEI.orgName]{orgName} \hyperref[TEI.persName]{persName} \hyperref[TEI.placeName]{placeName} \hyperref[TEI.region]{region} \hyperref[TEI.roleName]{roleName} \hyperref[TEI.settlement]{settlement} \hyperref[TEI.surname]{surname}\par 
    \item[textstructure: ]
   \hyperref[TEI.docAuthor]{docAuthor} \hyperref[TEI.docDate]{docDate} \hyperref[TEI.docEdition]{docEdition} \hyperref[TEI.titlePart]{titlePart}\par 
    \item[transcr: ]
   \hyperref[TEI.am]{am} \hyperref[TEI.damage]{damage} \hyperref[TEI.fw]{fw} \hyperref[TEI.metamark]{metamark} \hyperref[TEI.mod]{mod} \hyperref[TEI.restore]{restore} \hyperref[TEI.retrace]{retrace} \hyperref[TEI.secl]{secl} \hyperref[TEI.supplied]{supplied} \hyperref[TEI.surplus]{surplus}
    \item[{Peut contenir}]
  
    \item[analysis: ]
   \hyperref[TEI.c]{c} \hyperref[TEI.cl]{cl} \hyperref[TEI.interp]{interp} \hyperref[TEI.interpGrp]{interpGrp} \hyperref[TEI.m]{m} \hyperref[TEI.pc]{pc} \hyperref[TEI.phr]{phr} \hyperref[TEI.s]{s} \hyperref[TEI.span]{span} \hyperref[TEI.spanGrp]{spanGrp} \hyperref[TEI.w]{w}\par 
    \item[core: ]
   \hyperref[TEI.abbr]{abbr} \hyperref[TEI.add]{add} \hyperref[TEI.address]{address} \hyperref[TEI.bibl]{bibl} \hyperref[TEI.biblStruct]{biblStruct} \hyperref[TEI.binaryObject]{binaryObject} \hyperref[TEI.cb]{cb} \hyperref[TEI.choice]{choice} \hyperref[TEI.cit]{cit} \hyperref[TEI.corr]{corr} \hyperref[TEI.date]{date} \hyperref[TEI.del]{del} \hyperref[TEI.desc]{desc} \hyperref[TEI.distinct]{distinct} \hyperref[TEI.email]{email} \hyperref[TEI.emph]{emph} \hyperref[TEI.expan]{expan} \hyperref[TEI.foreign]{foreign} \hyperref[TEI.gap]{gap} \hyperref[TEI.gb]{gb} \hyperref[TEI.gloss]{gloss} \hyperref[TEI.graphic]{graphic} \hyperref[TEI.hi]{hi} \hyperref[TEI.index]{index} \hyperref[TEI.l]{l} \hyperref[TEI.label]{label} \hyperref[TEI.lb]{lb} \hyperref[TEI.lg]{lg} \hyperref[TEI.list]{list} \hyperref[TEI.listBibl]{listBibl} \hyperref[TEI.measure]{measure} \hyperref[TEI.measureGrp]{measureGrp} \hyperref[TEI.media]{media} \hyperref[TEI.mentioned]{mentioned} \hyperref[TEI.milestone]{milestone} \hyperref[TEI.name]{name} \hyperref[TEI.note]{note} \hyperref[TEI.num]{num} \hyperref[TEI.orig]{orig} \hyperref[TEI.pb]{pb} \hyperref[TEI.ptr]{ptr} \hyperref[TEI.q]{q} \hyperref[TEI.quote]{quote} \hyperref[TEI.ref]{ref} \hyperref[TEI.reg]{reg} \hyperref[TEI.rs]{rs} \hyperref[TEI.said]{said} \hyperref[TEI.sic]{sic} \hyperref[TEI.soCalled]{soCalled} \hyperref[TEI.stage]{stage} \hyperref[TEI.term]{term} \hyperref[TEI.time]{time} \hyperref[TEI.title]{title} \hyperref[TEI.unclear]{unclear}\par 
    \item[derived-module-tei.istex: ]
   \hyperref[TEI.math]{math} \hyperref[TEI.mrow]{mrow}\par 
    \item[figures: ]
   \hyperref[TEI.figure]{figure} \hyperref[TEI.formula]{formula} \hyperref[TEI.notatedMusic]{notatedMusic} \hyperref[TEI.table]{table}\par 
    \item[header: ]
   \hyperref[TEI.biblFull]{biblFull} \hyperref[TEI.idno]{idno}\par 
    \item[iso-fs: ]
   \hyperref[TEI.fLib]{fLib} \hyperref[TEI.fs]{fs} \hyperref[TEI.fvLib]{fvLib}\par 
    \item[linking: ]
   \hyperref[TEI.alt]{alt} \hyperref[TEI.altGrp]{altGrp} \hyperref[TEI.anchor]{anchor} \hyperref[TEI.join]{join} \hyperref[TEI.joinGrp]{joinGrp} \hyperref[TEI.link]{link} \hyperref[TEI.linkGrp]{linkGrp} \hyperref[TEI.seg]{seg} \hyperref[TEI.timeline]{timeline}\par 
    \item[msdescription: ]
   \hyperref[TEI.catchwords]{catchwords} \hyperref[TEI.depth]{depth} \hyperref[TEI.dim]{dim} \hyperref[TEI.dimensions]{dimensions} \hyperref[TEI.height]{height} \hyperref[TEI.heraldry]{heraldry} \hyperref[TEI.locus]{locus} \hyperref[TEI.locusGrp]{locusGrp} \hyperref[TEI.material]{material} \hyperref[TEI.msDesc]{msDesc} \hyperref[TEI.objectType]{objectType} \hyperref[TEI.origDate]{origDate} \hyperref[TEI.origPlace]{origPlace} \hyperref[TEI.secFol]{secFol} \hyperref[TEI.signatures]{signatures} \hyperref[TEI.source]{source} \hyperref[TEI.stamp]{stamp} \hyperref[TEI.watermark]{watermark} \hyperref[TEI.width]{width}\par 
    \item[namesdates: ]
   \hyperref[TEI.addName]{addName} \hyperref[TEI.affiliation]{affiliation} \hyperref[TEI.country]{country} \hyperref[TEI.forename]{forename} \hyperref[TEI.genName]{genName} \hyperref[TEI.geogName]{geogName} \hyperref[TEI.listOrg]{listOrg} \hyperref[TEI.listPlace]{listPlace} \hyperref[TEI.location]{location} \hyperref[TEI.nameLink]{nameLink} \hyperref[TEI.orgName]{orgName} \hyperref[TEI.persName]{persName} \hyperref[TEI.placeName]{placeName} \hyperref[TEI.region]{region} \hyperref[TEI.roleName]{roleName} \hyperref[TEI.settlement]{settlement} \hyperref[TEI.state]{state} \hyperref[TEI.surname]{surname}\par 
    \item[spoken: ]
   \hyperref[TEI.annotationBlock]{annotationBlock}\par 
    \item[textstructure: ]
   \hyperref[TEI.floatingText]{floatingText}\par 
    \item[transcr: ]
   \hyperref[TEI.addSpan]{addSpan} \hyperref[TEI.am]{am} \hyperref[TEI.damage]{damage} \hyperref[TEI.damageSpan]{damageSpan} \hyperref[TEI.delSpan]{delSpan} \hyperref[TEI.ex]{ex} \hyperref[TEI.fw]{fw} \hyperref[TEI.handShift]{handShift} \hyperref[TEI.listTranspose]{listTranspose} \hyperref[TEI.metamark]{metamark} \hyperref[TEI.mod]{mod} \hyperref[TEI.redo]{redo} \hyperref[TEI.restore]{restore} \hyperref[TEI.retrace]{retrace} \hyperref[TEI.secl]{secl} \hyperref[TEI.space]{space} \hyperref[TEI.subst]{subst} \hyperref[TEI.substJoin]{substJoin} \hyperref[TEI.supplied]{supplied} \hyperref[TEI.surplus]{surplus} \hyperref[TEI.undo]{undo}\par des données textuelles
    \item[{Note}]
  \par
Les éléments \hyperref[TEI.damage]{<damage>}, \hyperref[TEI.gap]{<gap>}, \hyperref[TEI.del]{<del>}, \hyperref[TEI.unclear]{<unclear>} et \hyperref[TEI.supplied]{<supplied>} peuvent être étroitement associés. Voir la section \xref{http://www.tei-c.org/release/doc/tei-p5-doc/en/html/PH.html\#PHCOMB}{11.3.3.2. Use of the gap, del, damage, unclear, and supplied Elements in Combination} pour savoir quel élément est approprié à chaque circonstance.
    \item[{Exemple}]
  \leavevmode\bgroup\exampleFont \begin{shaded}\noindent\mbox{}Je reste votre ts he de svt {<\textbf{supplied}\hspace*{6pt}{reason}="{illegible}"\mbox{}\newline 
\hspace*{6pt}{source}="{\#amanuensis\textunderscore copy}">}très humble et très dévoué serviteur\mbox{}\newline 
{</\textbf{supplied}>}Jean Martin \end{shaded}\egroup 


    \item[{Modèle de contenu}]
  \mbox{}\hfill\\[-10pt]\begin{Verbatim}[fontsize=\small]
<content>
 <macroRef key="macro.paraContent"/>
</content>
    
\end{Verbatim}

    \item[{Schéma Declaration}]
  \mbox{}\hfill\\[-10pt]\begin{Verbatim}[fontsize=\small]
element supplied
{
   tei_att.global.attributes,
   tei_att.editLike.attributes,
   attribute reason { list { + } }?,
   tei_macro.paraContent}
\end{Verbatim}

\end{reflist}  \index{support=<support>|oddindex}
\begin{reflist}
\item[]\begin{specHead}{TEI.support}{<support> }(support) contient la description des matériaux, techniques, etc., qui ont servi à fabriquer le support physique du texte du manuscrit. [\xref{http://www.tei-c.org/release/doc/tei-p5-doc/en/html/MS.html\#msph1}{10.7.1. Object Description}]\end{specHead} 
    \item[{Module}]
  msdescription
    \item[{Attributs}]
  Attributs \hyperref[TEI.att.global]{att.global} (\textit{@xml:id}, \textit{@n}, \textit{@xml:lang}, \textit{@xml:base}, \textit{@xml:space})  (\hyperref[TEI.att.global.rendition]{att.global.rendition} (\textit{@rend}, \textit{@style}, \textit{@rendition})) (\hyperref[TEI.att.global.linking]{att.global.linking} (\textit{@corresp}, \textit{@synch}, \textit{@sameAs}, \textit{@copyOf}, \textit{@next}, \textit{@prev}, \textit{@exclude}, \textit{@select})) (\hyperref[TEI.att.global.analytic]{att.global.analytic} (\textit{@ana})) (\hyperref[TEI.att.global.facs]{att.global.facs} (\textit{@facs})) (\hyperref[TEI.att.global.change]{att.global.change} (\textit{@change})) (\hyperref[TEI.att.global.responsibility]{att.global.responsibility} (\textit{@cert}, \textit{@resp})) (\hyperref[TEI.att.global.source]{att.global.source} (\textit{@source}))
    \item[{Contenu dans}]
  
    \item[msdescription: ]
   \hyperref[TEI.supportDesc]{supportDesc}
    \item[{Peut contenir}]
  
    \item[analysis: ]
   \hyperref[TEI.c]{c} \hyperref[TEI.cl]{cl} \hyperref[TEI.interp]{interp} \hyperref[TEI.interpGrp]{interpGrp} \hyperref[TEI.m]{m} \hyperref[TEI.pc]{pc} \hyperref[TEI.phr]{phr} \hyperref[TEI.s]{s} \hyperref[TEI.span]{span} \hyperref[TEI.spanGrp]{spanGrp} \hyperref[TEI.w]{w}\par 
    \item[core: ]
   \hyperref[TEI.abbr]{abbr} \hyperref[TEI.add]{add} \hyperref[TEI.address]{address} \hyperref[TEI.bibl]{bibl} \hyperref[TEI.biblStruct]{biblStruct} \hyperref[TEI.binaryObject]{binaryObject} \hyperref[TEI.cb]{cb} \hyperref[TEI.choice]{choice} \hyperref[TEI.cit]{cit} \hyperref[TEI.corr]{corr} \hyperref[TEI.date]{date} \hyperref[TEI.del]{del} \hyperref[TEI.desc]{desc} \hyperref[TEI.distinct]{distinct} \hyperref[TEI.email]{email} \hyperref[TEI.emph]{emph} \hyperref[TEI.expan]{expan} \hyperref[TEI.foreign]{foreign} \hyperref[TEI.gap]{gap} \hyperref[TEI.gb]{gb} \hyperref[TEI.gloss]{gloss} \hyperref[TEI.graphic]{graphic} \hyperref[TEI.hi]{hi} \hyperref[TEI.index]{index} \hyperref[TEI.l]{l} \hyperref[TEI.label]{label} \hyperref[TEI.lb]{lb} \hyperref[TEI.lg]{lg} \hyperref[TEI.list]{list} \hyperref[TEI.listBibl]{listBibl} \hyperref[TEI.measure]{measure} \hyperref[TEI.measureGrp]{measureGrp} \hyperref[TEI.media]{media} \hyperref[TEI.mentioned]{mentioned} \hyperref[TEI.milestone]{milestone} \hyperref[TEI.name]{name} \hyperref[TEI.note]{note} \hyperref[TEI.num]{num} \hyperref[TEI.orig]{orig} \hyperref[TEI.p]{p} \hyperref[TEI.pb]{pb} \hyperref[TEI.ptr]{ptr} \hyperref[TEI.q]{q} \hyperref[TEI.quote]{quote} \hyperref[TEI.ref]{ref} \hyperref[TEI.reg]{reg} \hyperref[TEI.rs]{rs} \hyperref[TEI.said]{said} \hyperref[TEI.sic]{sic} \hyperref[TEI.soCalled]{soCalled} \hyperref[TEI.sp]{sp} \hyperref[TEI.stage]{stage} \hyperref[TEI.term]{term} \hyperref[TEI.time]{time} \hyperref[TEI.title]{title} \hyperref[TEI.unclear]{unclear}\par 
    \item[derived-module-tei.istex: ]
   \hyperref[TEI.math]{math} \hyperref[TEI.mrow]{mrow}\par 
    \item[figures: ]
   \hyperref[TEI.figure]{figure} \hyperref[TEI.formula]{formula} \hyperref[TEI.notatedMusic]{notatedMusic} \hyperref[TEI.table]{table}\par 
    \item[header: ]
   \hyperref[TEI.biblFull]{biblFull} \hyperref[TEI.idno]{idno}\par 
    \item[iso-fs: ]
   \hyperref[TEI.fLib]{fLib} \hyperref[TEI.fs]{fs} \hyperref[TEI.fvLib]{fvLib}\par 
    \item[linking: ]
   \hyperref[TEI.ab]{ab} \hyperref[TEI.alt]{alt} \hyperref[TEI.altGrp]{altGrp} \hyperref[TEI.anchor]{anchor} \hyperref[TEI.join]{join} \hyperref[TEI.joinGrp]{joinGrp} \hyperref[TEI.link]{link} \hyperref[TEI.linkGrp]{linkGrp} \hyperref[TEI.seg]{seg} \hyperref[TEI.timeline]{timeline}\par 
    \item[msdescription: ]
   \hyperref[TEI.catchwords]{catchwords} \hyperref[TEI.depth]{depth} \hyperref[TEI.dim]{dim} \hyperref[TEI.dimensions]{dimensions} \hyperref[TEI.height]{height} \hyperref[TEI.heraldry]{heraldry} \hyperref[TEI.locus]{locus} \hyperref[TEI.locusGrp]{locusGrp} \hyperref[TEI.material]{material} \hyperref[TEI.msDesc]{msDesc} \hyperref[TEI.objectType]{objectType} \hyperref[TEI.origDate]{origDate} \hyperref[TEI.origPlace]{origPlace} \hyperref[TEI.secFol]{secFol} \hyperref[TEI.signatures]{signatures} \hyperref[TEI.source]{source} \hyperref[TEI.stamp]{stamp} \hyperref[TEI.watermark]{watermark} \hyperref[TEI.width]{width}\par 
    \item[namesdates: ]
   \hyperref[TEI.addName]{addName} \hyperref[TEI.affiliation]{affiliation} \hyperref[TEI.country]{country} \hyperref[TEI.forename]{forename} \hyperref[TEI.genName]{genName} \hyperref[TEI.geogName]{geogName} \hyperref[TEI.listOrg]{listOrg} \hyperref[TEI.listPlace]{listPlace} \hyperref[TEI.location]{location} \hyperref[TEI.nameLink]{nameLink} \hyperref[TEI.orgName]{orgName} \hyperref[TEI.persName]{persName} \hyperref[TEI.placeName]{placeName} \hyperref[TEI.region]{region} \hyperref[TEI.roleName]{roleName} \hyperref[TEI.settlement]{settlement} \hyperref[TEI.state]{state} \hyperref[TEI.surname]{surname}\par 
    \item[spoken: ]
   \hyperref[TEI.annotationBlock]{annotationBlock}\par 
    \item[textstructure: ]
   \hyperref[TEI.floatingText]{floatingText}\par 
    \item[transcr: ]
   \hyperref[TEI.addSpan]{addSpan} \hyperref[TEI.am]{am} \hyperref[TEI.damage]{damage} \hyperref[TEI.damageSpan]{damageSpan} \hyperref[TEI.delSpan]{delSpan} \hyperref[TEI.ex]{ex} \hyperref[TEI.fw]{fw} \hyperref[TEI.handShift]{handShift} \hyperref[TEI.listTranspose]{listTranspose} \hyperref[TEI.metamark]{metamark} \hyperref[TEI.mod]{mod} \hyperref[TEI.redo]{redo} \hyperref[TEI.restore]{restore} \hyperref[TEI.retrace]{retrace} \hyperref[TEI.secl]{secl} \hyperref[TEI.space]{space} \hyperref[TEI.subst]{subst} \hyperref[TEI.substJoin]{substJoin} \hyperref[TEI.supplied]{supplied} \hyperref[TEI.surplus]{surplus} \hyperref[TEI.undo]{undo}\par des données textuelles
    \item[{Exemple}]
  \leavevmode\bgroup\exampleFont \begin{shaded}\noindent\mbox{}{<\textbf{objectDesc}>}\mbox{}\newline 
\hspace*{6pt}{<\textbf{supportDesc}>}\mbox{}\newline 
\hspace*{6pt}\hspace*{6pt}{<\textbf{support}>} Rouleau de parchemin avec des rubans de{<\textbf{material}>}soie{</\textbf{material}>}.{</\textbf{support}>}\mbox{}\newline 
\hspace*{6pt}\hspace*{6pt}{<\textbf{extent}>}\mbox{}\newline 
\hspace*{6pt}\hspace*{6pt}\hspace*{6pt}{<\textbf{dimensions}\hspace*{6pt}{type}="{binding}">}\mbox{}\newline 
\hspace*{6pt}\hspace*{6pt}\hspace*{6pt}\hspace*{6pt}{<\textbf{height}\hspace*{6pt}{unit}="{mm}">}155{</\textbf{height}>}\mbox{}\newline 
\hspace*{6pt}\hspace*{6pt}\hspace*{6pt}\hspace*{6pt}{<\textbf{width}\hspace*{6pt}{unit}="{mm}">}95{</\textbf{width}>}\mbox{}\newline 
\hspace*{6pt}\hspace*{6pt}\hspace*{6pt}\hspace*{6pt}{<\textbf{depth}\hspace*{6pt}{unit}="{mm}">}31{</\textbf{depth}>}\mbox{}\newline 
\hspace*{6pt}\hspace*{6pt}\hspace*{6pt}{</\textbf{dimensions}>}\mbox{}\newline 
\hspace*{6pt}\hspace*{6pt}{</\textbf{extent}>}\mbox{}\newline 
\hspace*{6pt}{</\textbf{supportDesc}>}\mbox{}\newline 
{</\textbf{objectDesc}>}\end{shaded}\egroup 


    \item[{Modèle de contenu}]
  \mbox{}\hfill\\[-10pt]\begin{Verbatim}[fontsize=\small]
<content>
 <macroRef key="macro.specialPara"/>
</content>
    
\end{Verbatim}

    \item[{Schéma Declaration}]
  \mbox{}\hfill\\[-10pt]\begin{Verbatim}[fontsize=\small]
element support { tei_att.global.attributes, tei_macro.specialPara }
\end{Verbatim}

\end{reflist}  \index{supportDesc=<supportDesc>|oddindex}\index{material=@material!<supportDesc>|oddindex}
\begin{reflist}
\item[]\begin{specHead}{TEI.supportDesc}{<supportDesc> }(description du support) Regroupe les éléments décrivant le support physique du texte du manuscrit. [\xref{http://www.tei-c.org/release/doc/tei-p5-doc/en/html/MS.html\#msph1}{10.7.1. Object Description}]\end{specHead} 
    \item[{Module}]
  msdescription
    \item[{Attributs}]
  Attributs \hyperref[TEI.att.global]{att.global} (\textit{@xml:id}, \textit{@n}, \textit{@xml:lang}, \textit{@xml:base}, \textit{@xml:space})  (\hyperref[TEI.att.global.rendition]{att.global.rendition} (\textit{@rend}, \textit{@style}, \textit{@rendition})) (\hyperref[TEI.att.global.linking]{att.global.linking} (\textit{@corresp}, \textit{@synch}, \textit{@sameAs}, \textit{@copyOf}, \textit{@next}, \textit{@prev}, \textit{@exclude}, \textit{@select})) (\hyperref[TEI.att.global.analytic]{att.global.analytic} (\textit{@ana})) (\hyperref[TEI.att.global.facs]{att.global.facs} (\textit{@facs})) (\hyperref[TEI.att.global.change]{att.global.change} (\textit{@change})) (\hyperref[TEI.att.global.responsibility]{att.global.responsibility} (\textit{@cert}, \textit{@resp})) (\hyperref[TEI.att.global.source]{att.global.source} (\textit{@source})) \hfil\\[-10pt]\begin{sansreflist}
    \item[@material]
  (matériau) contient un nom abrégé propre au projet désignant le matériau qui a principalement servi pour fabriquer le support.
\begin{reflist}
    \item[{Statut}]
  Optionel
    \item[{Type de données}]
  \hyperref[TEI.teidata.enumerated]{teidata.enumerated}
    \item[{Les valeurs suggérées comprennent:}]
  \begin{description}

\item[{paper}]
\item[{parch}](parchemin)
\item[{mixed}]
\end{description} 
\end{reflist}  
\end{sansreflist}  
    \item[{Contenu dans}]
  
    \item[msdescription: ]
   \hyperref[TEI.objectDesc]{objectDesc}
    \item[{Peut contenir}]
  
    \item[core: ]
   \hyperref[TEI.p]{p}\par 
    \item[header: ]
   \hyperref[TEI.extent]{extent}\par 
    \item[linking: ]
   \hyperref[TEI.ab]{ab}\par 
    \item[msdescription: ]
   \hyperref[TEI.collation]{collation} \hyperref[TEI.condition]{condition} \hyperref[TEI.foliation]{foliation} \hyperref[TEI.support]{support}
    \item[{Exemple}]
  \leavevmode\bgroup\exampleFont \begin{shaded}\noindent\mbox{}{<\textbf{supportDesc}>}\mbox{}\newline 
\hspace*{6pt}{<\textbf{support}>} Parchment roll with {<\textbf{material}>}silk{</\textbf{material}>} ribbons.\mbox{}\newline 
\hspace*{6pt}{</\textbf{support}>}\mbox{}\newline 
{</\textbf{supportDesc}>}\end{shaded}\egroup 


    \item[{Modèle de contenu}]
  \mbox{}\hfill\\[-10pt]\begin{Verbatim}[fontsize=\small]
<content>
 <alternate maxOccurs="1" minOccurs="1">
  <classRef key="model.pLike"
   maxOccurs="unbounded" minOccurs="1"/>
  <sequence maxOccurs="1" minOccurs="1">
   <elementRef key="support" minOccurs="0"/>
   <elementRef key="extent" minOccurs="0"/>
   <elementRef key="foliation"
    maxOccurs="unbounded" minOccurs="0"/>
   <elementRef key="collation"
    minOccurs="0"/>
   <elementRef key="condition"
    minOccurs="0"/>
  </sequence>
 </alternate>
</content>
    
\end{Verbatim}

    \item[{Schéma Declaration}]
  \mbox{}\hfill\\[-10pt]\begin{Verbatim}[fontsize=\small]
element supportDesc
{
   tei_att.global.attributes,
   attribute material { "paper" | "parch" | "mixed" }?,
   (
      tei_model.pLike+
    | (
         tei_support?,
         tei_extent?,
         tei_foliation*,
         tei_collation?,
         tei_condition?
      )
   )
}
\end{Verbatim}

\end{reflist}  \index{surface=<surface>|oddindex}\index{attachment=@attachment!<surface>|oddindex}\index{flipping=@flipping!<surface>|oddindex}
\begin{reflist}
\item[]\begin{specHead}{TEI.surface}{<surface> }définit une surface écrite comme un rectangle décrit par ses coordonnées spatiales, en regroupant éventuellement une ou plusieurs représentations graphiques de cet espace et des zones rectangulaires intéressantes à l'intérieur de celui-ci. [\xref{http://www.tei-c.org/release/doc/tei-p5-doc/en/html/PH.html\#PHFAX}{11.1. Digital Facsimiles} \xref{http://www.tei-c.org/release/doc/tei-p5-doc/en/html/PH.html\#PHZLAB}{11.2.2. Embedded Transcription}]\end{specHead} 
    \item[{Module}]
  transcr
    \item[{Attributs}]
  Attributs \hyperref[TEI.att.global]{att.global} (\textit{@xml:id}, \textit{@n}, \textit{@xml:lang}, \textit{@xml:base}, \textit{@xml:space})  (\hyperref[TEI.att.global.rendition]{att.global.rendition} (\textit{@rend}, \textit{@style}, \textit{@rendition})) (\hyperref[TEI.att.global.linking]{att.global.linking} (\textit{@corresp}, \textit{@synch}, \textit{@sameAs}, \textit{@copyOf}, \textit{@next}, \textit{@prev}, \textit{@exclude}, \textit{@select})) (\hyperref[TEI.att.global.analytic]{att.global.analytic} (\textit{@ana})) (\hyperref[TEI.att.global.facs]{att.global.facs} (\textit{@facs})) (\hyperref[TEI.att.global.change]{att.global.change} (\textit{@change})) (\hyperref[TEI.att.global.responsibility]{att.global.responsibility} (\textit{@cert}, \textit{@resp})) (\hyperref[TEI.att.global.source]{att.global.source} (\textit{@source})) \hyperref[TEI.att.coordinated]{att.coordinated} (\textit{@start}, \textit{@ulx}, \textit{@uly}, \textit{@lrx}, \textit{@lry}, \textit{@points}) \hyperref[TEI.att.declaring]{att.declaring} (\textit{@decls}) \hyperref[TEI.att.typed]{att.typed} (\textit{@type}, \textit{@subtype}) \hfil\\[-10pt]\begin{sansreflist}
    \item[@attachment]
  describes the method by which this surface is or was connected to the main surface
\begin{reflist}
    \item[{Statut}]
  Optionel
    \item[{Type de données}]
  \hyperref[TEI.teidata.enumerated]{teidata.enumerated}
    \item[{Exemple de valeurs possibles:}]
  \begin{description}

\item[{glued}]glued in place
\item[{pinned}]pinned or stapled in place
\item[{sewn}]sewn in place
\end{description} 
\end{reflist}  
    \item[@flipping]
  indicates whether the surface is attached and folded in such a way as to provide two writing surfaces
\begin{reflist}
    \item[{Statut}]
  Optionel
    \item[{Type de données}]
  \hyperref[TEI.teidata.truthValue]{teidata.truthValue}
\end{reflist}  
\end{sansreflist}  
    \item[{Contenu dans}]
  
    \item[transcr: ]
   \hyperref[TEI.facsimile]{facsimile} \hyperref[TEI.sourceDoc]{sourceDoc} \hyperref[TEI.surface]{surface} \hyperref[TEI.surfaceGrp]{surfaceGrp} \hyperref[TEI.zone]{zone}
    \item[{Peut contenir}]
  
    \item[analysis: ]
   \hyperref[TEI.interp]{interp} \hyperref[TEI.interpGrp]{interpGrp} \hyperref[TEI.span]{span} \hyperref[TEI.spanGrp]{spanGrp}\par 
    \item[core: ]
   \hyperref[TEI.binaryObject]{binaryObject} \hyperref[TEI.cb]{cb} \hyperref[TEI.desc]{desc} \hyperref[TEI.gap]{gap} \hyperref[TEI.gb]{gb} \hyperref[TEI.graphic]{graphic} \hyperref[TEI.index]{index} \hyperref[TEI.label]{label} \hyperref[TEI.lb]{lb} \hyperref[TEI.media]{media} \hyperref[TEI.milestone]{milestone} \hyperref[TEI.note]{note} \hyperref[TEI.pb]{pb}\par 
    \item[derived-module-tei.istex: ]
   \hyperref[TEI.math]{math} \hyperref[TEI.mrow]{mrow}\par 
    \item[figures: ]
   \hyperref[TEI.figure]{figure} \hyperref[TEI.formula]{formula} \hyperref[TEI.notatedMusic]{notatedMusic}\par 
    \item[iso-fs: ]
   \hyperref[TEI.fLib]{fLib} \hyperref[TEI.fs]{fs} \hyperref[TEI.fvLib]{fvLib}\par 
    \item[linking: ]
   \hyperref[TEI.alt]{alt} \hyperref[TEI.altGrp]{altGrp} \hyperref[TEI.anchor]{anchor} \hyperref[TEI.join]{join} \hyperref[TEI.joinGrp]{joinGrp} \hyperref[TEI.link]{link} \hyperref[TEI.linkGrp]{linkGrp} \hyperref[TEI.timeline]{timeline}\par 
    \item[msdescription: ]
   \hyperref[TEI.source]{source}\par 
    \item[transcr: ]
   \hyperref[TEI.addSpan]{addSpan} \hyperref[TEI.damageSpan]{damageSpan} \hyperref[TEI.delSpan]{delSpan} \hyperref[TEI.fw]{fw} \hyperref[TEI.line]{line} \hyperref[TEI.listTranspose]{listTranspose} \hyperref[TEI.metamark]{metamark} \hyperref[TEI.space]{space} \hyperref[TEI.substJoin]{substJoin} \hyperref[TEI.surface]{surface} \hyperref[TEI.surfaceGrp]{surfaceGrp} \hyperref[TEI.zone]{zone}
    \item[{Note}]
  \par
L'élément \hyperref[TEI.surface]{<surface>} représente un secteur rectangulaire d’une surface physique faisant partie du matériau source. Ce peut être une feuille du papier, la façade d'un monument, un panneau publicitaire, un rouleau de papyrus ou en fait toute surface à deux dimensions.\par
L'espace de coordonnées défini par cet élément peut être considéré comme une grille d'unités horizontales{\itshape lrx} - {\itshape ulx} et verticales {\itshape uly} - {\itshape lry}. Cette grille se superpose à la totalité de toute image directement contenue par l'élément \hyperref[TEI.surface]{<surface>}. Les coordonnées employées par chaque élément \hyperref[TEI.zone]{<zone>}contenu par cette surface doivent être interprétées en référence à la même grille.
    \item[{Exemple}]
  \leavevmode\bgroup\exampleFont \begin{shaded}\noindent\mbox{}{<\textbf{facsimile}>}\mbox{}\newline 
\hspace*{6pt}{<\textbf{surface}\hspace*{6pt}{lrx}="{200}"\hspace*{6pt}{lry}="{300}"\hspace*{6pt}{ulx}="{0}"\hspace*{6pt}{uly}="{0}">}\mbox{}\newline 
\hspace*{6pt}\hspace*{6pt}{<\textbf{graphic}\hspace*{6pt}{url}="{Bovelles-49r.png}"/>}\mbox{}\newline 
\hspace*{6pt}{</\textbf{surface}>}\mbox{}\newline 
{</\textbf{facsimile}>}\end{shaded}\egroup 


    \item[{Modèle de contenu}]
  \mbox{}\hfill\\[-10pt]\begin{Verbatim}[fontsize=\small]
<content>
 <sequence maxOccurs="1" minOccurs="1">
  <alternate maxOccurs="unbounded"
   minOccurs="0">
   <classRef key="model.global"/>
   <classRef key="model.labelLike"/>
   <classRef key="model.graphicLike"/>
  </alternate>
  <sequence maxOccurs="unbounded"
   minOccurs="0">
   <alternate maxOccurs="1" minOccurs="1">
    <elementRef key="zone"/>
    <elementRef key="line"/>
    <elementRef key="surface"/>
    <elementRef key="surfaceGrp"/>
   </alternate>
   <classRef key="model.global"
    maxOccurs="unbounded" minOccurs="0"/>
  </sequence>
 </sequence>
</content>
    
\end{Verbatim}

    \item[{Schéma Declaration}]
  \mbox{}\hfill\\[-10pt]\begin{Verbatim}[fontsize=\small]
element surface
{
   tei_att.global.attributes,
   tei_att.coordinated.attributes,
   tei_att.declaring.attributes,
   tei_att.typed.attributes,
   attribute attachment { text }?,
   attribute flipping { text }?,
   (
      ( tei_model.global | tei_model.labelLike | tei_model.graphicLike )*,
      (
         ( tei_zone | tei_line | tei_surface | tei_surfaceGrp ),
         tei_model.global*
      )*
   )
}
\end{Verbatim}

\end{reflist}  \index{surfaceGrp=<surfaceGrp>|oddindex}
\begin{reflist}
\item[]\begin{specHead}{TEI.surfaceGrp}{<surfaceGrp> }defines any kind of useful grouping of written surfaces, for example the recto and verso of a single leaf, which the encoder wishes to treat as a single unit. [\xref{http://www.tei-c.org/release/doc/tei-p5-doc/en/html/PH.html\#PHFAX}{11.1. Digital Facsimiles}]\end{specHead} 
    \item[{Module}]
  transcr
    \item[{Attributs}]
  Attributs \hyperref[TEI.att.global]{att.global} (\textit{@xml:id}, \textit{@n}, \textit{@xml:lang}, \textit{@xml:base}, \textit{@xml:space})  (\hyperref[TEI.att.global.rendition]{att.global.rendition} (\textit{@rend}, \textit{@style}, \textit{@rendition})) (\hyperref[TEI.att.global.linking]{att.global.linking} (\textit{@corresp}, \textit{@synch}, \textit{@sameAs}, \textit{@copyOf}, \textit{@next}, \textit{@prev}, \textit{@exclude}, \textit{@select})) (\hyperref[TEI.att.global.analytic]{att.global.analytic} (\textit{@ana})) (\hyperref[TEI.att.global.facs]{att.global.facs} (\textit{@facs})) (\hyperref[TEI.att.global.change]{att.global.change} (\textit{@change})) (\hyperref[TEI.att.global.responsibility]{att.global.responsibility} (\textit{@cert}, \textit{@resp})) (\hyperref[TEI.att.global.source]{att.global.source} (\textit{@source})) \hyperref[TEI.att.declaring]{att.declaring} (\textit{@decls}) \hyperref[TEI.att.typed]{att.typed} (\textit{@type}, \textit{@subtype}) 
    \item[{Contenu dans}]
  
    \item[transcr: ]
   \hyperref[TEI.facsimile]{facsimile} \hyperref[TEI.sourceDoc]{sourceDoc} \hyperref[TEI.surface]{surface} \hyperref[TEI.surfaceGrp]{surfaceGrp}
    \item[{Peut contenir}]
  
    \item[analysis: ]
   \hyperref[TEI.interp]{interp} \hyperref[TEI.interpGrp]{interpGrp} \hyperref[TEI.span]{span} \hyperref[TEI.spanGrp]{spanGrp}\par 
    \item[core: ]
   \hyperref[TEI.cb]{cb} \hyperref[TEI.gap]{gap} \hyperref[TEI.gb]{gb} \hyperref[TEI.index]{index} \hyperref[TEI.lb]{lb} \hyperref[TEI.milestone]{milestone} \hyperref[TEI.note]{note} \hyperref[TEI.pb]{pb}\par 
    \item[figures: ]
   \hyperref[TEI.figure]{figure} \hyperref[TEI.notatedMusic]{notatedMusic}\par 
    \item[iso-fs: ]
   \hyperref[TEI.fLib]{fLib} \hyperref[TEI.fs]{fs} \hyperref[TEI.fvLib]{fvLib}\par 
    \item[linking: ]
   \hyperref[TEI.alt]{alt} \hyperref[TEI.altGrp]{altGrp} \hyperref[TEI.anchor]{anchor} \hyperref[TEI.join]{join} \hyperref[TEI.joinGrp]{joinGrp} \hyperref[TEI.link]{link} \hyperref[TEI.linkGrp]{linkGrp} \hyperref[TEI.timeline]{timeline}\par 
    \item[msdescription: ]
   \hyperref[TEI.source]{source}\par 
    \item[transcr: ]
   \hyperref[TEI.addSpan]{addSpan} \hyperref[TEI.damageSpan]{damageSpan} \hyperref[TEI.delSpan]{delSpan} \hyperref[TEI.fw]{fw} \hyperref[TEI.listTranspose]{listTranspose} \hyperref[TEI.metamark]{metamark} \hyperref[TEI.space]{space} \hyperref[TEI.substJoin]{substJoin} \hyperref[TEI.surface]{surface} \hyperref[TEI.surfaceGrp]{surfaceGrp}
    \item[{Note}]
  \par
Where it is useful or meaningful to do so, any grouping of multiple \hyperref[TEI.surface]{<surface>} elements may be indicated using the \hyperref[TEI.surfaceGrp]{<surfaceGrp>} elements.
    \item[{Exemple}]
  \leavevmode\bgroup\exampleFont \begin{shaded}\noindent\mbox{}{<\textbf{sourceDoc}>}\mbox{}\newline 
\hspace*{6pt}{<\textbf{surfaceGrp}>}\mbox{}\newline 
\hspace*{6pt}\hspace*{6pt}{<\textbf{surface}\hspace*{6pt}{lrx}="{200}"\hspace*{6pt}{lry}="{300}"\hspace*{6pt}{ulx}="{0}"\mbox{}\newline 
\hspace*{6pt}\hspace*{6pt}\hspace*{6pt}{uly}="{0}">}\mbox{}\newline 
\hspace*{6pt}\hspace*{6pt}\hspace*{6pt}{<\textbf{graphic}\hspace*{6pt}{url}="{Bovelles-49r.png}"/>}\mbox{}\newline 
\hspace*{6pt}\hspace*{6pt}{</\textbf{surface}>}\mbox{}\newline 
\hspace*{6pt}\hspace*{6pt}{<\textbf{surface}\hspace*{6pt}{lrx}="{200}"\hspace*{6pt}{lry}="{300}"\hspace*{6pt}{ulx}="{0}"\mbox{}\newline 
\hspace*{6pt}\hspace*{6pt}\hspace*{6pt}{uly}="{0}">}\mbox{}\newline 
\hspace*{6pt}\hspace*{6pt}\hspace*{6pt}{<\textbf{graphic}\hspace*{6pt}{url}="{Bovelles-49v.png}"/>}\mbox{}\newline 
\hspace*{6pt}\hspace*{6pt}{</\textbf{surface}>}\mbox{}\newline 
\hspace*{6pt}{</\textbf{surfaceGrp}>}\mbox{}\newline 
{</\textbf{sourceDoc}>}\end{shaded}\egroup 


    \item[{Modèle de contenu}]
  \mbox{}\hfill\\[-10pt]\begin{Verbatim}[fontsize=\small]
<content>
 <alternate maxOccurs="unbounded"
  minOccurs="1">
  <classRef key="model.global"/>
  <elementRef key="surface"/>
  <elementRef key="surfaceGrp"/>
 </alternate>
</content>
    
\end{Verbatim}

    \item[{Schéma Declaration}]
  \mbox{}\hfill\\[-10pt]\begin{Verbatim}[fontsize=\small]
element surfaceGrp
{
   tei_att.global.attributes,
   tei_att.declaring.attributes,
   tei_att.typed.attributes,
   ( tei_model.global | tei_surface | tei_surfaceGrp )+
}
\end{Verbatim}

\end{reflist}  \index{surname=<surname>|oddindex}
\begin{reflist}
\item[]\begin{specHead}{TEI.surname}{<surname> }(nom de famille) contient un nom de famille (hérité) par opposition à un nom donné, nom de baptême ou surnom. [\xref{http://www.tei-c.org/release/doc/tei-p5-doc/en/html/ND.html\#NDPER}{13.2.1. Personal Names}]\end{specHead} 
    \item[{Module}]
  namesdates
    \item[{Attributs}]
  Attributs \hyperref[TEI.att.global]{att.global} (\textit{@xml:id}, \textit{@n}, \textit{@xml:lang}, \textit{@xml:base}, \textit{@xml:space})  (\hyperref[TEI.att.global.rendition]{att.global.rendition} (\textit{@rend}, \textit{@style}, \textit{@rendition})) (\hyperref[TEI.att.global.linking]{att.global.linking} (\textit{@corresp}, \textit{@synch}, \textit{@sameAs}, \textit{@copyOf}, \textit{@next}, \textit{@prev}, \textit{@exclude}, \textit{@select})) (\hyperref[TEI.att.global.analytic]{att.global.analytic} (\textit{@ana})) (\hyperref[TEI.att.global.facs]{att.global.facs} (\textit{@facs})) (\hyperref[TEI.att.global.change]{att.global.change} (\textit{@change})) (\hyperref[TEI.att.global.responsibility]{att.global.responsibility} (\textit{@cert}, \textit{@resp})) (\hyperref[TEI.att.global.source]{att.global.source} (\textit{@source})) \hyperref[TEI.att.personal]{att.personal} (\textit{@full}, \textit{@sort})  (\hyperref[TEI.att.naming]{att.naming} (\textit{@role}, \textit{@nymRef}) (\hyperref[TEI.att.canonical]{att.canonical} (\textit{@key}, \textit{@ref})) ) \hyperref[TEI.att.typed]{att.typed} (\textit{@type}, \textit{@subtype}) 
    \item[{Membre du}]
  \hyperref[TEI.model.persNamePart]{model.persNamePart}
    \item[{Contenu dans}]
  
    \item[analysis: ]
   \hyperref[TEI.cl]{cl} \hyperref[TEI.phr]{phr} \hyperref[TEI.s]{s} \hyperref[TEI.span]{span}\par 
    \item[core: ]
   \hyperref[TEI.abbr]{abbr} \hyperref[TEI.add]{add} \hyperref[TEI.addrLine]{addrLine} \hyperref[TEI.address]{address} \hyperref[TEI.author]{author} \hyperref[TEI.bibl]{bibl} \hyperref[TEI.biblScope]{biblScope} \hyperref[TEI.citedRange]{citedRange} \hyperref[TEI.corr]{corr} \hyperref[TEI.date]{date} \hyperref[TEI.del]{del} \hyperref[TEI.desc]{desc} \hyperref[TEI.distinct]{distinct} \hyperref[TEI.editor]{editor} \hyperref[TEI.email]{email} \hyperref[TEI.emph]{emph} \hyperref[TEI.expan]{expan} \hyperref[TEI.foreign]{foreign} \hyperref[TEI.gloss]{gloss} \hyperref[TEI.head]{head} \hyperref[TEI.headItem]{headItem} \hyperref[TEI.headLabel]{headLabel} \hyperref[TEI.hi]{hi} \hyperref[TEI.item]{item} \hyperref[TEI.l]{l} \hyperref[TEI.label]{label} \hyperref[TEI.measure]{measure} \hyperref[TEI.meeting]{meeting} \hyperref[TEI.mentioned]{mentioned} \hyperref[TEI.name]{name} \hyperref[TEI.note]{note} \hyperref[TEI.num]{num} \hyperref[TEI.orig]{orig} \hyperref[TEI.p]{p} \hyperref[TEI.pubPlace]{pubPlace} \hyperref[TEI.publisher]{publisher} \hyperref[TEI.q]{q} \hyperref[TEI.quote]{quote} \hyperref[TEI.ref]{ref} \hyperref[TEI.reg]{reg} \hyperref[TEI.resp]{resp} \hyperref[TEI.rs]{rs} \hyperref[TEI.said]{said} \hyperref[TEI.sic]{sic} \hyperref[TEI.soCalled]{soCalled} \hyperref[TEI.speaker]{speaker} \hyperref[TEI.stage]{stage} \hyperref[TEI.street]{street} \hyperref[TEI.term]{term} \hyperref[TEI.textLang]{textLang} \hyperref[TEI.time]{time} \hyperref[TEI.title]{title} \hyperref[TEI.unclear]{unclear}\par 
    \item[figures: ]
   \hyperref[TEI.cell]{cell} \hyperref[TEI.figDesc]{figDesc}\par 
    \item[header: ]
   \hyperref[TEI.authority]{authority} \hyperref[TEI.change]{change} \hyperref[TEI.classCode]{classCode} \hyperref[TEI.creation]{creation} \hyperref[TEI.distributor]{distributor} \hyperref[TEI.edition]{edition} \hyperref[TEI.extent]{extent} \hyperref[TEI.funder]{funder} \hyperref[TEI.language]{language} \hyperref[TEI.licence]{licence} \hyperref[TEI.rendition]{rendition}\par 
    \item[iso-fs: ]
   \hyperref[TEI.fDescr]{fDescr} \hyperref[TEI.fsDescr]{fsDescr}\par 
    \item[linking: ]
   \hyperref[TEI.ab]{ab} \hyperref[TEI.seg]{seg}\par 
    \item[msdescription: ]
   \hyperref[TEI.accMat]{accMat} \hyperref[TEI.acquisition]{acquisition} \hyperref[TEI.additions]{additions} \hyperref[TEI.catchwords]{catchwords} \hyperref[TEI.collation]{collation} \hyperref[TEI.colophon]{colophon} \hyperref[TEI.condition]{condition} \hyperref[TEI.custEvent]{custEvent} \hyperref[TEI.decoNote]{decoNote} \hyperref[TEI.explicit]{explicit} \hyperref[TEI.filiation]{filiation} \hyperref[TEI.finalRubric]{finalRubric} \hyperref[TEI.foliation]{foliation} \hyperref[TEI.heraldry]{heraldry} \hyperref[TEI.incipit]{incipit} \hyperref[TEI.layout]{layout} \hyperref[TEI.material]{material} \hyperref[TEI.musicNotation]{musicNotation} \hyperref[TEI.objectType]{objectType} \hyperref[TEI.origDate]{origDate} \hyperref[TEI.origPlace]{origPlace} \hyperref[TEI.origin]{origin} \hyperref[TEI.provenance]{provenance} \hyperref[TEI.rubric]{rubric} \hyperref[TEI.secFol]{secFol} \hyperref[TEI.signatures]{signatures} \hyperref[TEI.source]{source} \hyperref[TEI.stamp]{stamp} \hyperref[TEI.summary]{summary} \hyperref[TEI.support]{support} \hyperref[TEI.surrogates]{surrogates} \hyperref[TEI.typeNote]{typeNote} \hyperref[TEI.watermark]{watermark}\par 
    \item[namesdates: ]
   \hyperref[TEI.addName]{addName} \hyperref[TEI.affiliation]{affiliation} \hyperref[TEI.country]{country} \hyperref[TEI.forename]{forename} \hyperref[TEI.genName]{genName} \hyperref[TEI.geogName]{geogName} \hyperref[TEI.nameLink]{nameLink} \hyperref[TEI.org]{org} \hyperref[TEI.orgName]{orgName} \hyperref[TEI.persName]{persName} \hyperref[TEI.placeName]{placeName} \hyperref[TEI.region]{region} \hyperref[TEI.roleName]{roleName} \hyperref[TEI.settlement]{settlement} \hyperref[TEI.surname]{surname}\par 
    \item[spoken: ]
   \hyperref[TEI.annotationBlock]{annotationBlock}\par 
    \item[standOff: ]
   \hyperref[TEI.listAnnotation]{listAnnotation}\par 
    \item[textstructure: ]
   \hyperref[TEI.docAuthor]{docAuthor} \hyperref[TEI.docDate]{docDate} \hyperref[TEI.docEdition]{docEdition} \hyperref[TEI.titlePart]{titlePart}\par 
    \item[transcr: ]
   \hyperref[TEI.damage]{damage} \hyperref[TEI.fw]{fw} \hyperref[TEI.metamark]{metamark} \hyperref[TEI.mod]{mod} \hyperref[TEI.restore]{restore} \hyperref[TEI.retrace]{retrace} \hyperref[TEI.secl]{secl} \hyperref[TEI.supplied]{supplied} \hyperref[TEI.surplus]{surplus}
    \item[{Peut contenir}]
  
    \item[analysis: ]
   \hyperref[TEI.c]{c} \hyperref[TEI.cl]{cl} \hyperref[TEI.interp]{interp} \hyperref[TEI.interpGrp]{interpGrp} \hyperref[TEI.m]{m} \hyperref[TEI.pc]{pc} \hyperref[TEI.phr]{phr} \hyperref[TEI.s]{s} \hyperref[TEI.span]{span} \hyperref[TEI.spanGrp]{spanGrp} \hyperref[TEI.w]{w}\par 
    \item[core: ]
   \hyperref[TEI.abbr]{abbr} \hyperref[TEI.add]{add} \hyperref[TEI.address]{address} \hyperref[TEI.binaryObject]{binaryObject} \hyperref[TEI.cb]{cb} \hyperref[TEI.choice]{choice} \hyperref[TEI.corr]{corr} \hyperref[TEI.date]{date} \hyperref[TEI.del]{del} \hyperref[TEI.distinct]{distinct} \hyperref[TEI.email]{email} \hyperref[TEI.emph]{emph} \hyperref[TEI.expan]{expan} \hyperref[TEI.foreign]{foreign} \hyperref[TEI.gap]{gap} \hyperref[TEI.gb]{gb} \hyperref[TEI.gloss]{gloss} \hyperref[TEI.graphic]{graphic} \hyperref[TEI.hi]{hi} \hyperref[TEI.index]{index} \hyperref[TEI.lb]{lb} \hyperref[TEI.measure]{measure} \hyperref[TEI.measureGrp]{measureGrp} \hyperref[TEI.media]{media} \hyperref[TEI.mentioned]{mentioned} \hyperref[TEI.milestone]{milestone} \hyperref[TEI.name]{name} \hyperref[TEI.note]{note} \hyperref[TEI.num]{num} \hyperref[TEI.orig]{orig} \hyperref[TEI.pb]{pb} \hyperref[TEI.ptr]{ptr} \hyperref[TEI.ref]{ref} \hyperref[TEI.reg]{reg} \hyperref[TEI.rs]{rs} \hyperref[TEI.sic]{sic} \hyperref[TEI.soCalled]{soCalled} \hyperref[TEI.term]{term} \hyperref[TEI.time]{time} \hyperref[TEI.title]{title} \hyperref[TEI.unclear]{unclear}\par 
    \item[derived-module-tei.istex: ]
   \hyperref[TEI.math]{math} \hyperref[TEI.mrow]{mrow}\par 
    \item[figures: ]
   \hyperref[TEI.figure]{figure} \hyperref[TEI.formula]{formula} \hyperref[TEI.notatedMusic]{notatedMusic}\par 
    \item[header: ]
   \hyperref[TEI.idno]{idno}\par 
    \item[iso-fs: ]
   \hyperref[TEI.fLib]{fLib} \hyperref[TEI.fs]{fs} \hyperref[TEI.fvLib]{fvLib}\par 
    \item[linking: ]
   \hyperref[TEI.alt]{alt} \hyperref[TEI.altGrp]{altGrp} \hyperref[TEI.anchor]{anchor} \hyperref[TEI.join]{join} \hyperref[TEI.joinGrp]{joinGrp} \hyperref[TEI.link]{link} \hyperref[TEI.linkGrp]{linkGrp} \hyperref[TEI.seg]{seg} \hyperref[TEI.timeline]{timeline}\par 
    \item[msdescription: ]
   \hyperref[TEI.catchwords]{catchwords} \hyperref[TEI.depth]{depth} \hyperref[TEI.dim]{dim} \hyperref[TEI.dimensions]{dimensions} \hyperref[TEI.height]{height} \hyperref[TEI.heraldry]{heraldry} \hyperref[TEI.locus]{locus} \hyperref[TEI.locusGrp]{locusGrp} \hyperref[TEI.material]{material} \hyperref[TEI.objectType]{objectType} \hyperref[TEI.origDate]{origDate} \hyperref[TEI.origPlace]{origPlace} \hyperref[TEI.secFol]{secFol} \hyperref[TEI.signatures]{signatures} \hyperref[TEI.source]{source} \hyperref[TEI.stamp]{stamp} \hyperref[TEI.watermark]{watermark} \hyperref[TEI.width]{width}\par 
    \item[namesdates: ]
   \hyperref[TEI.addName]{addName} \hyperref[TEI.affiliation]{affiliation} \hyperref[TEI.country]{country} \hyperref[TEI.forename]{forename} \hyperref[TEI.genName]{genName} \hyperref[TEI.geogName]{geogName} \hyperref[TEI.location]{location} \hyperref[TEI.nameLink]{nameLink} \hyperref[TEI.orgName]{orgName} \hyperref[TEI.persName]{persName} \hyperref[TEI.placeName]{placeName} \hyperref[TEI.region]{region} \hyperref[TEI.roleName]{roleName} \hyperref[TEI.settlement]{settlement} \hyperref[TEI.state]{state} \hyperref[TEI.surname]{surname}\par 
    \item[spoken: ]
   \hyperref[TEI.annotationBlock]{annotationBlock}\par 
    \item[transcr: ]
   \hyperref[TEI.addSpan]{addSpan} \hyperref[TEI.am]{am} \hyperref[TEI.damage]{damage} \hyperref[TEI.damageSpan]{damageSpan} \hyperref[TEI.delSpan]{delSpan} \hyperref[TEI.ex]{ex} \hyperref[TEI.fw]{fw} \hyperref[TEI.handShift]{handShift} \hyperref[TEI.listTranspose]{listTranspose} \hyperref[TEI.metamark]{metamark} \hyperref[TEI.mod]{mod} \hyperref[TEI.redo]{redo} \hyperref[TEI.restore]{restore} \hyperref[TEI.retrace]{retrace} \hyperref[TEI.secl]{secl} \hyperref[TEI.space]{space} \hyperref[TEI.subst]{subst} \hyperref[TEI.substJoin]{substJoin} \hyperref[TEI.supplied]{supplied} \hyperref[TEI.surplus]{surplus} \hyperref[TEI.undo]{undo}\par des données textuelles
    \item[{Exemple}]
  \leavevmode\bgroup\exampleFont \begin{shaded}\noindent\mbox{}{<\textbf{surname}\hspace*{6pt}{type}="{combine}">}Sidonie Gabrielle Colette{</\textbf{surname}>}\end{shaded}\egroup 


    \item[{Modèle de contenu}]
  \mbox{}\hfill\\[-10pt]\begin{Verbatim}[fontsize=\small]
<content>
 <macroRef key="macro.phraseSeq"/>
</content>
    
\end{Verbatim}

    \item[{Schéma Declaration}]
  \mbox{}\hfill\\[-10pt]\begin{Verbatim}[fontsize=\small]
element surname
{
   tei_att.global.attributes,
   tei_att.personal.attributes,
   tei_att.typed.attributes,
   tei_macro.phraseSeq}
\end{Verbatim}

\end{reflist}  \index{surplus=<surplus>|oddindex}\index{reason=@reason!<surplus>|oddindex}
\begin{reflist}
\item[]\begin{specHead}{TEI.surplus}{<surplus> }(Texte superflu) permet d'encoder une partie de texte présente dans la source lorsque l'éditeur la considère superflue ou redondante. [\xref{http://www.tei-c.org/release/doc/tei-p5-doc/en/html/PH.html\#PHDA}{11.3.3.1. Damage, Illegibility, and Supplied Text}]\end{specHead} 
    \item[{Module}]
  transcr
    \item[{Attributs}]
  Attributs \hyperref[TEI.att.global]{att.global} (\textit{@xml:id}, \textit{@n}, \textit{@xml:lang}, \textit{@xml:base}, \textit{@xml:space})  (\hyperref[TEI.att.global.rendition]{att.global.rendition} (\textit{@rend}, \textit{@style}, \textit{@rendition})) (\hyperref[TEI.att.global.linking]{att.global.linking} (\textit{@corresp}, \textit{@synch}, \textit{@sameAs}, \textit{@copyOf}, \textit{@next}, \textit{@prev}, \textit{@exclude}, \textit{@select})) (\hyperref[TEI.att.global.analytic]{att.global.analytic} (\textit{@ana})) (\hyperref[TEI.att.global.facs]{att.global.facs} (\textit{@facs})) (\hyperref[TEI.att.global.change]{att.global.change} (\textit{@change})) (\hyperref[TEI.att.global.responsibility]{att.global.responsibility} (\textit{@cert}, \textit{@resp})) (\hyperref[TEI.att.global.source]{att.global.source} (\textit{@source})) \hyperref[TEI.att.editLike]{att.editLike} (\textit{@evidence}, \textit{@instant})  (\hyperref[TEI.att.dimensions]{att.dimensions} (\textit{@unit}, \textit{@quantity}, \textit{@extent}, \textit{@precision}, \textit{@scope}) (\hyperref[TEI.att.ranging]{att.ranging} (\textit{@atLeast}, \textit{@atMost}, \textit{@min}, \textit{@max}, \textit{@confidence})) ) \hfil\\[-10pt]\begin{sansreflist}
    \item[@reason]
  indique les raisons pour lesquelles on considère cette partie de texte comme superflue.
\begin{reflist}
    \item[{Statut}]
  Optionel
    \item[{Type de données}]
  1–∞ occurrences de \hyperref[TEI.teidata.word]{teidata.word} séparé par un espace
\end{reflist}  
\end{sansreflist}  
    \item[{Membre du}]
  \hyperref[TEI.model.pPart.transcriptional]{model.pPart.transcriptional}
    \item[{Contenu dans}]
  
    \item[analysis: ]
   \hyperref[TEI.cl]{cl} \hyperref[TEI.pc]{pc} \hyperref[TEI.phr]{phr} \hyperref[TEI.s]{s} \hyperref[TEI.w]{w}\par 
    \item[core: ]
   \hyperref[TEI.abbr]{abbr} \hyperref[TEI.add]{add} \hyperref[TEI.addrLine]{addrLine} \hyperref[TEI.author]{author} \hyperref[TEI.bibl]{bibl} \hyperref[TEI.biblScope]{biblScope} \hyperref[TEI.citedRange]{citedRange} \hyperref[TEI.corr]{corr} \hyperref[TEI.date]{date} \hyperref[TEI.del]{del} \hyperref[TEI.distinct]{distinct} \hyperref[TEI.editor]{editor} \hyperref[TEI.email]{email} \hyperref[TEI.emph]{emph} \hyperref[TEI.expan]{expan} \hyperref[TEI.foreign]{foreign} \hyperref[TEI.gloss]{gloss} \hyperref[TEI.head]{head} \hyperref[TEI.headItem]{headItem} \hyperref[TEI.headLabel]{headLabel} \hyperref[TEI.hi]{hi} \hyperref[TEI.item]{item} \hyperref[TEI.l]{l} \hyperref[TEI.label]{label} \hyperref[TEI.measure]{measure} \hyperref[TEI.mentioned]{mentioned} \hyperref[TEI.name]{name} \hyperref[TEI.note]{note} \hyperref[TEI.num]{num} \hyperref[TEI.orig]{orig} \hyperref[TEI.p]{p} \hyperref[TEI.pubPlace]{pubPlace} \hyperref[TEI.publisher]{publisher} \hyperref[TEI.q]{q} \hyperref[TEI.quote]{quote} \hyperref[TEI.ref]{ref} \hyperref[TEI.reg]{reg} \hyperref[TEI.rs]{rs} \hyperref[TEI.said]{said} \hyperref[TEI.sic]{sic} \hyperref[TEI.soCalled]{soCalled} \hyperref[TEI.speaker]{speaker} \hyperref[TEI.stage]{stage} \hyperref[TEI.street]{street} \hyperref[TEI.term]{term} \hyperref[TEI.textLang]{textLang} \hyperref[TEI.time]{time} \hyperref[TEI.title]{title} \hyperref[TEI.unclear]{unclear}\par 
    \item[figures: ]
   \hyperref[TEI.cell]{cell}\par 
    \item[header: ]
   \hyperref[TEI.change]{change} \hyperref[TEI.distributor]{distributor} \hyperref[TEI.edition]{edition} \hyperref[TEI.extent]{extent} \hyperref[TEI.licence]{licence}\par 
    \item[linking: ]
   \hyperref[TEI.ab]{ab} \hyperref[TEI.seg]{seg}\par 
    \item[msdescription: ]
   \hyperref[TEI.accMat]{accMat} \hyperref[TEI.acquisition]{acquisition} \hyperref[TEI.additions]{additions} \hyperref[TEI.catchwords]{catchwords} \hyperref[TEI.collation]{collation} \hyperref[TEI.colophon]{colophon} \hyperref[TEI.condition]{condition} \hyperref[TEI.custEvent]{custEvent} \hyperref[TEI.decoNote]{decoNote} \hyperref[TEI.explicit]{explicit} \hyperref[TEI.filiation]{filiation} \hyperref[TEI.finalRubric]{finalRubric} \hyperref[TEI.foliation]{foliation} \hyperref[TEI.heraldry]{heraldry} \hyperref[TEI.incipit]{incipit} \hyperref[TEI.layout]{layout} \hyperref[TEI.material]{material} \hyperref[TEI.musicNotation]{musicNotation} \hyperref[TEI.objectType]{objectType} \hyperref[TEI.origDate]{origDate} \hyperref[TEI.origPlace]{origPlace} \hyperref[TEI.origin]{origin} \hyperref[TEI.provenance]{provenance} \hyperref[TEI.rubric]{rubric} \hyperref[TEI.secFol]{secFol} \hyperref[TEI.signatures]{signatures} \hyperref[TEI.source]{source} \hyperref[TEI.stamp]{stamp} \hyperref[TEI.summary]{summary} \hyperref[TEI.support]{support} \hyperref[TEI.surrogates]{surrogates} \hyperref[TEI.typeNote]{typeNote} \hyperref[TEI.watermark]{watermark}\par 
    \item[namesdates: ]
   \hyperref[TEI.addName]{addName} \hyperref[TEI.affiliation]{affiliation} \hyperref[TEI.country]{country} \hyperref[TEI.forename]{forename} \hyperref[TEI.genName]{genName} \hyperref[TEI.geogName]{geogName} \hyperref[TEI.nameLink]{nameLink} \hyperref[TEI.orgName]{orgName} \hyperref[TEI.persName]{persName} \hyperref[TEI.placeName]{placeName} \hyperref[TEI.region]{region} \hyperref[TEI.roleName]{roleName} \hyperref[TEI.settlement]{settlement} \hyperref[TEI.surname]{surname}\par 
    \item[textstructure: ]
   \hyperref[TEI.docAuthor]{docAuthor} \hyperref[TEI.docDate]{docDate} \hyperref[TEI.docEdition]{docEdition} \hyperref[TEI.titlePart]{titlePart}\par 
    \item[transcr: ]
   \hyperref[TEI.am]{am} \hyperref[TEI.damage]{damage} \hyperref[TEI.fw]{fw} \hyperref[TEI.metamark]{metamark} \hyperref[TEI.mod]{mod} \hyperref[TEI.restore]{restore} \hyperref[TEI.retrace]{retrace} \hyperref[TEI.secl]{secl} \hyperref[TEI.supplied]{supplied} \hyperref[TEI.surplus]{surplus}
    \item[{Peut contenir}]
  
    \item[analysis: ]
   \hyperref[TEI.c]{c} \hyperref[TEI.cl]{cl} \hyperref[TEI.interp]{interp} \hyperref[TEI.interpGrp]{interpGrp} \hyperref[TEI.m]{m} \hyperref[TEI.pc]{pc} \hyperref[TEI.phr]{phr} \hyperref[TEI.s]{s} \hyperref[TEI.span]{span} \hyperref[TEI.spanGrp]{spanGrp} \hyperref[TEI.w]{w}\par 
    \item[core: ]
   \hyperref[TEI.abbr]{abbr} \hyperref[TEI.add]{add} \hyperref[TEI.address]{address} \hyperref[TEI.bibl]{bibl} \hyperref[TEI.biblStruct]{biblStruct} \hyperref[TEI.binaryObject]{binaryObject} \hyperref[TEI.cb]{cb} \hyperref[TEI.choice]{choice} \hyperref[TEI.cit]{cit} \hyperref[TEI.corr]{corr} \hyperref[TEI.date]{date} \hyperref[TEI.del]{del} \hyperref[TEI.desc]{desc} \hyperref[TEI.distinct]{distinct} \hyperref[TEI.email]{email} \hyperref[TEI.emph]{emph} \hyperref[TEI.expan]{expan} \hyperref[TEI.foreign]{foreign} \hyperref[TEI.gap]{gap} \hyperref[TEI.gb]{gb} \hyperref[TEI.gloss]{gloss} \hyperref[TEI.graphic]{graphic} \hyperref[TEI.hi]{hi} \hyperref[TEI.index]{index} \hyperref[TEI.l]{l} \hyperref[TEI.label]{label} \hyperref[TEI.lb]{lb} \hyperref[TEI.lg]{lg} \hyperref[TEI.list]{list} \hyperref[TEI.listBibl]{listBibl} \hyperref[TEI.measure]{measure} \hyperref[TEI.measureGrp]{measureGrp} \hyperref[TEI.media]{media} \hyperref[TEI.mentioned]{mentioned} \hyperref[TEI.milestone]{milestone} \hyperref[TEI.name]{name} \hyperref[TEI.note]{note} \hyperref[TEI.num]{num} \hyperref[TEI.orig]{orig} \hyperref[TEI.pb]{pb} \hyperref[TEI.ptr]{ptr} \hyperref[TEI.q]{q} \hyperref[TEI.quote]{quote} \hyperref[TEI.ref]{ref} \hyperref[TEI.reg]{reg} \hyperref[TEI.rs]{rs} \hyperref[TEI.said]{said} \hyperref[TEI.sic]{sic} \hyperref[TEI.soCalled]{soCalled} \hyperref[TEI.stage]{stage} \hyperref[TEI.term]{term} \hyperref[TEI.time]{time} \hyperref[TEI.title]{title} \hyperref[TEI.unclear]{unclear}\par 
    \item[derived-module-tei.istex: ]
   \hyperref[TEI.math]{math} \hyperref[TEI.mrow]{mrow}\par 
    \item[figures: ]
   \hyperref[TEI.figure]{figure} \hyperref[TEI.formula]{formula} \hyperref[TEI.notatedMusic]{notatedMusic} \hyperref[TEI.table]{table}\par 
    \item[header: ]
   \hyperref[TEI.biblFull]{biblFull} \hyperref[TEI.idno]{idno}\par 
    \item[iso-fs: ]
   \hyperref[TEI.fLib]{fLib} \hyperref[TEI.fs]{fs} \hyperref[TEI.fvLib]{fvLib}\par 
    \item[linking: ]
   \hyperref[TEI.alt]{alt} \hyperref[TEI.altGrp]{altGrp} \hyperref[TEI.anchor]{anchor} \hyperref[TEI.join]{join} \hyperref[TEI.joinGrp]{joinGrp} \hyperref[TEI.link]{link} \hyperref[TEI.linkGrp]{linkGrp} \hyperref[TEI.seg]{seg} \hyperref[TEI.timeline]{timeline}\par 
    \item[msdescription: ]
   \hyperref[TEI.catchwords]{catchwords} \hyperref[TEI.depth]{depth} \hyperref[TEI.dim]{dim} \hyperref[TEI.dimensions]{dimensions} \hyperref[TEI.height]{height} \hyperref[TEI.heraldry]{heraldry} \hyperref[TEI.locus]{locus} \hyperref[TEI.locusGrp]{locusGrp} \hyperref[TEI.material]{material} \hyperref[TEI.msDesc]{msDesc} \hyperref[TEI.objectType]{objectType} \hyperref[TEI.origDate]{origDate} \hyperref[TEI.origPlace]{origPlace} \hyperref[TEI.secFol]{secFol} \hyperref[TEI.signatures]{signatures} \hyperref[TEI.source]{source} \hyperref[TEI.stamp]{stamp} \hyperref[TEI.watermark]{watermark} \hyperref[TEI.width]{width}\par 
    \item[namesdates: ]
   \hyperref[TEI.addName]{addName} \hyperref[TEI.affiliation]{affiliation} \hyperref[TEI.country]{country} \hyperref[TEI.forename]{forename} \hyperref[TEI.genName]{genName} \hyperref[TEI.geogName]{geogName} \hyperref[TEI.listOrg]{listOrg} \hyperref[TEI.listPlace]{listPlace} \hyperref[TEI.location]{location} \hyperref[TEI.nameLink]{nameLink} \hyperref[TEI.orgName]{orgName} \hyperref[TEI.persName]{persName} \hyperref[TEI.placeName]{placeName} \hyperref[TEI.region]{region} \hyperref[TEI.roleName]{roleName} \hyperref[TEI.settlement]{settlement} \hyperref[TEI.state]{state} \hyperref[TEI.surname]{surname}\par 
    \item[spoken: ]
   \hyperref[TEI.annotationBlock]{annotationBlock}\par 
    \item[textstructure: ]
   \hyperref[TEI.floatingText]{floatingText}\par 
    \item[transcr: ]
   \hyperref[TEI.addSpan]{addSpan} \hyperref[TEI.am]{am} \hyperref[TEI.damage]{damage} \hyperref[TEI.damageSpan]{damageSpan} \hyperref[TEI.delSpan]{delSpan} \hyperref[TEI.ex]{ex} \hyperref[TEI.fw]{fw} \hyperref[TEI.handShift]{handShift} \hyperref[TEI.listTranspose]{listTranspose} \hyperref[TEI.metamark]{metamark} \hyperref[TEI.mod]{mod} \hyperref[TEI.redo]{redo} \hyperref[TEI.restore]{restore} \hyperref[TEI.retrace]{retrace} \hyperref[TEI.secl]{secl} \hyperref[TEI.space]{space} \hyperref[TEI.subst]{subst} \hyperref[TEI.substJoin]{substJoin} \hyperref[TEI.supplied]{supplied} \hyperref[TEI.surplus]{surplus} \hyperref[TEI.undo]{undo}\par des données textuelles
    \item[{Exemple}]
  \leavevmode\bgroup\exampleFont \begin{shaded}\noindent\mbox{}I am dr Sr yrs\mbox{}\newline 
{<\textbf{surplus}\hspace*{6pt}{reason}="{repeated}">}yrs{</\textbf{surplus}>}\mbox{}\newline 
 Sydney Smith\end{shaded}\egroup 


    \item[{Modèle de contenu}]
  \mbox{}\hfill\\[-10pt]\begin{Verbatim}[fontsize=\small]
<content>
 <macroRef key="macro.paraContent"/>
</content>
    
\end{Verbatim}

    \item[{Schéma Declaration}]
  \mbox{}\hfill\\[-10pt]\begin{Verbatim}[fontsize=\small]
element surplus
{
   tei_att.global.attributes,
   tei_att.editLike.attributes,
   attribute reason { list { + } }?,
   tei_macro.paraContent}
\end{Verbatim}

\end{reflist}  \index{surrogates=<surrogates>|oddindex}
\begin{reflist}
\item[]\begin{specHead}{TEI.surrogates}{<surrogates> }(reproductions) contient des informations sur toute reproduction numérique ou photographique du manuscrit en cours de description, qu'elle soit détenue par l'institution de conservation ou ailleurs. [\xref{http://www.tei-c.org/release/doc/tei-p5-doc/en/html/MS.html\#msad}{10.9. Additional Information}]\end{specHead} 
    \item[{Module}]
  msdescription
    \item[{Attributs}]
  Attributs \hyperref[TEI.att.global]{att.global} (\textit{@xml:id}, \textit{@n}, \textit{@xml:lang}, \textit{@xml:base}, \textit{@xml:space})  (\hyperref[TEI.att.global.rendition]{att.global.rendition} (\textit{@rend}, \textit{@style}, \textit{@rendition})) (\hyperref[TEI.att.global.linking]{att.global.linking} (\textit{@corresp}, \textit{@synch}, \textit{@sameAs}, \textit{@copyOf}, \textit{@next}, \textit{@prev}, \textit{@exclude}, \textit{@select})) (\hyperref[TEI.att.global.analytic]{att.global.analytic} (\textit{@ana})) (\hyperref[TEI.att.global.facs]{att.global.facs} (\textit{@facs})) (\hyperref[TEI.att.global.change]{att.global.change} (\textit{@change})) (\hyperref[TEI.att.global.responsibility]{att.global.responsibility} (\textit{@cert}, \textit{@resp})) (\hyperref[TEI.att.global.source]{att.global.source} (\textit{@source}))
    \item[{Contenu dans}]
  
    \item[msdescription: ]
   \hyperref[TEI.additional]{additional}
    \item[{Peut contenir}]
  
    \item[analysis: ]
   \hyperref[TEI.c]{c} \hyperref[TEI.cl]{cl} \hyperref[TEI.interp]{interp} \hyperref[TEI.interpGrp]{interpGrp} \hyperref[TEI.m]{m} \hyperref[TEI.pc]{pc} \hyperref[TEI.phr]{phr} \hyperref[TEI.s]{s} \hyperref[TEI.span]{span} \hyperref[TEI.spanGrp]{spanGrp} \hyperref[TEI.w]{w}\par 
    \item[core: ]
   \hyperref[TEI.abbr]{abbr} \hyperref[TEI.add]{add} \hyperref[TEI.address]{address} \hyperref[TEI.bibl]{bibl} \hyperref[TEI.biblStruct]{biblStruct} \hyperref[TEI.binaryObject]{binaryObject} \hyperref[TEI.cb]{cb} \hyperref[TEI.choice]{choice} \hyperref[TEI.cit]{cit} \hyperref[TEI.corr]{corr} \hyperref[TEI.date]{date} \hyperref[TEI.del]{del} \hyperref[TEI.desc]{desc} \hyperref[TEI.distinct]{distinct} \hyperref[TEI.email]{email} \hyperref[TEI.emph]{emph} \hyperref[TEI.expan]{expan} \hyperref[TEI.foreign]{foreign} \hyperref[TEI.gap]{gap} \hyperref[TEI.gb]{gb} \hyperref[TEI.gloss]{gloss} \hyperref[TEI.graphic]{graphic} \hyperref[TEI.hi]{hi} \hyperref[TEI.index]{index} \hyperref[TEI.l]{l} \hyperref[TEI.label]{label} \hyperref[TEI.lb]{lb} \hyperref[TEI.lg]{lg} \hyperref[TEI.list]{list} \hyperref[TEI.listBibl]{listBibl} \hyperref[TEI.measure]{measure} \hyperref[TEI.measureGrp]{measureGrp} \hyperref[TEI.media]{media} \hyperref[TEI.mentioned]{mentioned} \hyperref[TEI.milestone]{milestone} \hyperref[TEI.name]{name} \hyperref[TEI.note]{note} \hyperref[TEI.num]{num} \hyperref[TEI.orig]{orig} \hyperref[TEI.p]{p} \hyperref[TEI.pb]{pb} \hyperref[TEI.ptr]{ptr} \hyperref[TEI.q]{q} \hyperref[TEI.quote]{quote} \hyperref[TEI.ref]{ref} \hyperref[TEI.reg]{reg} \hyperref[TEI.rs]{rs} \hyperref[TEI.said]{said} \hyperref[TEI.sic]{sic} \hyperref[TEI.soCalled]{soCalled} \hyperref[TEI.sp]{sp} \hyperref[TEI.stage]{stage} \hyperref[TEI.term]{term} \hyperref[TEI.time]{time} \hyperref[TEI.title]{title} \hyperref[TEI.unclear]{unclear}\par 
    \item[derived-module-tei.istex: ]
   \hyperref[TEI.math]{math} \hyperref[TEI.mrow]{mrow}\par 
    \item[figures: ]
   \hyperref[TEI.figure]{figure} \hyperref[TEI.formula]{formula} \hyperref[TEI.notatedMusic]{notatedMusic} \hyperref[TEI.table]{table}\par 
    \item[header: ]
   \hyperref[TEI.biblFull]{biblFull} \hyperref[TEI.idno]{idno}\par 
    \item[iso-fs: ]
   \hyperref[TEI.fLib]{fLib} \hyperref[TEI.fs]{fs} \hyperref[TEI.fvLib]{fvLib}\par 
    \item[linking: ]
   \hyperref[TEI.ab]{ab} \hyperref[TEI.alt]{alt} \hyperref[TEI.altGrp]{altGrp} \hyperref[TEI.anchor]{anchor} \hyperref[TEI.join]{join} \hyperref[TEI.joinGrp]{joinGrp} \hyperref[TEI.link]{link} \hyperref[TEI.linkGrp]{linkGrp} \hyperref[TEI.seg]{seg} \hyperref[TEI.timeline]{timeline}\par 
    \item[msdescription: ]
   \hyperref[TEI.catchwords]{catchwords} \hyperref[TEI.depth]{depth} \hyperref[TEI.dim]{dim} \hyperref[TEI.dimensions]{dimensions} \hyperref[TEI.height]{height} \hyperref[TEI.heraldry]{heraldry} \hyperref[TEI.locus]{locus} \hyperref[TEI.locusGrp]{locusGrp} \hyperref[TEI.material]{material} \hyperref[TEI.msDesc]{msDesc} \hyperref[TEI.objectType]{objectType} \hyperref[TEI.origDate]{origDate} \hyperref[TEI.origPlace]{origPlace} \hyperref[TEI.secFol]{secFol} \hyperref[TEI.signatures]{signatures} \hyperref[TEI.source]{source} \hyperref[TEI.stamp]{stamp} \hyperref[TEI.watermark]{watermark} \hyperref[TEI.width]{width}\par 
    \item[namesdates: ]
   \hyperref[TEI.addName]{addName} \hyperref[TEI.affiliation]{affiliation} \hyperref[TEI.country]{country} \hyperref[TEI.forename]{forename} \hyperref[TEI.genName]{genName} \hyperref[TEI.geogName]{geogName} \hyperref[TEI.listOrg]{listOrg} \hyperref[TEI.listPlace]{listPlace} \hyperref[TEI.location]{location} \hyperref[TEI.nameLink]{nameLink} \hyperref[TEI.orgName]{orgName} \hyperref[TEI.persName]{persName} \hyperref[TEI.placeName]{placeName} \hyperref[TEI.region]{region} \hyperref[TEI.roleName]{roleName} \hyperref[TEI.settlement]{settlement} \hyperref[TEI.state]{state} \hyperref[TEI.surname]{surname}\par 
    \item[spoken: ]
   \hyperref[TEI.annotationBlock]{annotationBlock}\par 
    \item[textstructure: ]
   \hyperref[TEI.floatingText]{floatingText}\par 
    \item[transcr: ]
   \hyperref[TEI.addSpan]{addSpan} \hyperref[TEI.am]{am} \hyperref[TEI.damage]{damage} \hyperref[TEI.damageSpan]{damageSpan} \hyperref[TEI.delSpan]{delSpan} \hyperref[TEI.ex]{ex} \hyperref[TEI.fw]{fw} \hyperref[TEI.handShift]{handShift} \hyperref[TEI.listTranspose]{listTranspose} \hyperref[TEI.metamark]{metamark} \hyperref[TEI.mod]{mod} \hyperref[TEI.redo]{redo} \hyperref[TEI.restore]{restore} \hyperref[TEI.retrace]{retrace} \hyperref[TEI.secl]{secl} \hyperref[TEI.space]{space} \hyperref[TEI.subst]{subst} \hyperref[TEI.substJoin]{substJoin} \hyperref[TEI.supplied]{supplied} \hyperref[TEI.surplus]{surplus} \hyperref[TEI.undo]{undo}\par des données textuelles
    \item[{Exemple}]
  \leavevmode\bgroup\exampleFont \begin{shaded}\noindent\mbox{}{<\textbf{surrogates}>}\mbox{}\newline 
\hspace*{6pt}{<\textbf{p}>}\mbox{}\newline 
\hspace*{6pt}\hspace*{6pt}{<\textbf{bibl}>}\mbox{}\newline 
\hspace*{6pt}\hspace*{6pt}\hspace*{6pt}{<\textbf{title}\hspace*{6pt}{type}="{gmd}">}diapositive{</\textbf{title}>}\mbox{}\newline 
\hspace*{6pt}\hspace*{6pt}\hspace*{6pt}{<\textbf{idno}>}AM 74 a, fol.{</\textbf{idno}>}\mbox{}\newline 
\hspace*{6pt}\hspace*{6pt}\hspace*{6pt}{<\textbf{date}>}May 1984{</\textbf{date}>}\mbox{}\newline 
\hspace*{6pt}\hspace*{6pt}{</\textbf{bibl}>}\mbox{}\newline 
\hspace*{6pt}\hspace*{6pt}{<\textbf{bibl}>}\mbox{}\newline 
\hspace*{6pt}\hspace*{6pt}\hspace*{6pt}{<\textbf{title}\hspace*{6pt}{type}="{gmd}">}b/w prints{</\textbf{title}>}\mbox{}\newline 
\hspace*{6pt}\hspace*{6pt}\hspace*{6pt}{<\textbf{idno}>}AM 75 a, fol.{</\textbf{idno}>}\mbox{}\newline 
\hspace*{6pt}\hspace*{6pt}\hspace*{6pt}{<\textbf{date}>}1972{</\textbf{date}>}\mbox{}\newline 
\hspace*{6pt}\hspace*{6pt}{</\textbf{bibl}>}\mbox{}\newline 
\hspace*{6pt}{</\textbf{p}>}\mbox{}\newline 
{</\textbf{surrogates}>}\end{shaded}\egroup 


    \item[{Modèle de contenu}]
  \mbox{}\hfill\\[-10pt]\begin{Verbatim}[fontsize=\small]
<content>
 <macroRef key="macro.specialPara"/>
</content>
    
\end{Verbatim}

    \item[{Schéma Declaration}]
  \mbox{}\hfill\\[-10pt]\begin{Verbatim}[fontsize=\small]
element surrogates { tei_att.global.attributes, tei_macro.specialPara }
\end{Verbatim}

\end{reflist}  \index{symbol=<symbol>|oddindex}\index{value=@value!<symbol>|oddindex}
\begin{reflist}
\item[]\begin{specHead}{TEI.symbol}{<symbol> }(valeur symbolique) représente la partie valeur d'une spécification trait-valeur qui contient un symbole extrait d'une liste finie. [\xref{http://www.tei-c.org/release/doc/tei-p5-doc/en/html/FS.html\#FSSY}{18.3. Other Atomic Feature Values}]\end{specHead} 
    \item[{Module}]
  iso-fs
    \item[{Attributs}]
  Attributs \hyperref[TEI.att.global]{att.global} (\textit{@xml:id}, \textit{@n}, \textit{@xml:lang}, \textit{@xml:base}, \textit{@xml:space})  (\hyperref[TEI.att.global.rendition]{att.global.rendition} (\textit{@rend}, \textit{@style}, \textit{@rendition})) (\hyperref[TEI.att.global.linking]{att.global.linking} (\textit{@corresp}, \textit{@synch}, \textit{@sameAs}, \textit{@copyOf}, \textit{@next}, \textit{@prev}, \textit{@exclude}, \textit{@select})) (\hyperref[TEI.att.global.analytic]{att.global.analytic} (\textit{@ana})) (\hyperref[TEI.att.global.facs]{att.global.facs} (\textit{@facs})) (\hyperref[TEI.att.global.change]{att.global.change} (\textit{@change})) (\hyperref[TEI.att.global.responsibility]{att.global.responsibility} (\textit{@cert}, \textit{@resp})) (\hyperref[TEI.att.global.source]{att.global.source} (\textit{@source})) \hyperref[TEI.att.datcat]{att.datcat} (\textit{@datcat}, \textit{@valueDatcat}) \hfil\\[-10pt]\begin{sansreflist}
    \item[@value]
  donne la valeur symbolique pour le trait, extraite d'une liste finie qui peut être spécifiée dans une déclaration de traits.
\begin{reflist}
    \item[{Statut}]
  Requis
    \item[{Type de données}]
  \hyperref[TEI.teidata.word]{teidata.word}
\end{reflist}  
\end{sansreflist}  
    \item[{Membre du}]
  \hyperref[TEI.model.featureVal.single]{model.featureVal.single}
    \item[{Contenu dans}]
  
    \item[iso-fs: ]
   \hyperref[TEI.f]{f} \hyperref[TEI.fvLib]{fvLib} \hyperref[TEI.if]{if} \hyperref[TEI.vAlt]{vAlt} \hyperref[TEI.vColl]{vColl} \hyperref[TEI.vDefault]{vDefault} \hyperref[TEI.vLabel]{vLabel} \hyperref[TEI.vMerge]{vMerge} \hyperref[TEI.vNot]{vNot} \hyperref[TEI.vRange]{vRange}
    \item[{Peut contenir}]
  Elément vide
    \item[{Exemple}]
  \leavevmode\bgroup\exampleFont \begin{shaded}\noindent\mbox{}{<\textbf{f}\hspace*{6pt}{name}="{gender}">}\mbox{}\newline 
\hspace*{6pt}{<\textbf{symbol}\hspace*{6pt}{value}="{feminine}"/>}\mbox{}\newline 
{</\textbf{f}>}\end{shaded}\egroup 


    \item[{Modèle de contenu}]
  \fbox{\ttfamily <content>\newline
</content>\newline
    } 
    \item[{Schéma Declaration}]
  \mbox{}\hfill\\[-10pt]\begin{Verbatim}[fontsize=\small]
element symbol
{
   tei_att.global.attributes,
   tei_att.datcat.attributes,
   attribute value { text },
   empty
}
\end{Verbatim}

\end{reflist}  \index{table=<table>|oddindex}\index{rows=@rows!<table>|oddindex}\index{cols=@cols!<table>|oddindex}
\begin{reflist}
\item[]\begin{specHead}{TEI.table}{<table> }(tableau) contient du texte affiché sous forme de tableau, en rangées et colonnes. [\xref{http://www.tei-c.org/release/doc/tei-p5-doc/en/html/FT.html\#FTTAB1}{14.1.1. TEI Tables}]\end{specHead} 
    \item[{Module}]
  figures
    \item[{Attributs}]
  Attributs \hyperref[TEI.att.global]{att.global} (\textit{@xml:id}, \textit{@n}, \textit{@xml:lang}, \textit{@xml:base}, \textit{@xml:space})  (\hyperref[TEI.att.global.rendition]{att.global.rendition} (\textit{@rend}, \textit{@style}, \textit{@rendition})) (\hyperref[TEI.att.global.linking]{att.global.linking} (\textit{@corresp}, \textit{@synch}, \textit{@sameAs}, \textit{@copyOf}, \textit{@next}, \textit{@prev}, \textit{@exclude}, \textit{@select})) (\hyperref[TEI.att.global.analytic]{att.global.analytic} (\textit{@ana})) (\hyperref[TEI.att.global.facs]{att.global.facs} (\textit{@facs})) (\hyperref[TEI.att.global.change]{att.global.change} (\textit{@change})) (\hyperref[TEI.att.global.responsibility]{att.global.responsibility} (\textit{@cert}, \textit{@resp})) (\hyperref[TEI.att.global.source]{att.global.source} (\textit{@source})) \hyperref[TEI.att.typed]{att.typed} (\textit{@type}, \textit{@subtype}) \hfil\\[-10pt]\begin{sansreflist}
    \item[@rows]
  indique le nombre de rangées dans le tableau.
\begin{reflist}
    \item[{Statut}]
  Optionel
    \item[{Type de données}]
  \hyperref[TEI.teidata.count]{teidata.count}
    \item[{Note}]
  \par
Les rangées sont ordonnées de haut en bas
\end{reflist}  
    \item[@cols]
  (colonnes) indique le nombre de colonnes dans chaque rangée du tableau.
\begin{reflist}
    \item[{Statut}]
  Optionel
    \item[{Type de données}]
  \hyperref[TEI.teidata.count]{teidata.count}
    \item[{Note}]
  \par
Si aucun nombre n'est fourni, une application doit calculer le nombre de colonnes.\par
Dans chaque rangée, les colonnes sont ordonnées de gauche à droite.
\end{reflist}  
\end{sansreflist}  
    \item[{Membre du}]
  \hyperref[TEI.model.listLike]{model.listLike}
    \item[{Contenu dans}]
  
    \item[core: ]
   \hyperref[TEI.add]{add} \hyperref[TEI.corr]{corr} \hyperref[TEI.del]{del} \hyperref[TEI.desc]{desc} \hyperref[TEI.emph]{emph} \hyperref[TEI.head]{head} \hyperref[TEI.hi]{hi} \hyperref[TEI.item]{item} \hyperref[TEI.l]{l} \hyperref[TEI.meeting]{meeting} \hyperref[TEI.note]{note} \hyperref[TEI.orig]{orig} \hyperref[TEI.p]{p} \hyperref[TEI.q]{q} \hyperref[TEI.quote]{quote} \hyperref[TEI.ref]{ref} \hyperref[TEI.reg]{reg} \hyperref[TEI.said]{said} \hyperref[TEI.sic]{sic} \hyperref[TEI.sp]{sp} \hyperref[TEI.stage]{stage} \hyperref[TEI.title]{title} \hyperref[TEI.unclear]{unclear}\par 
    \item[figures: ]
   \hyperref[TEI.cell]{cell} \hyperref[TEI.figDesc]{figDesc} \hyperref[TEI.figure]{figure}\par 
    \item[header: ]
   \hyperref[TEI.abstract]{abstract} \hyperref[TEI.change]{change} \hyperref[TEI.licence]{licence} \hyperref[TEI.rendition]{rendition} \hyperref[TEI.sourceDesc]{sourceDesc}\par 
    \item[iso-fs: ]
   \hyperref[TEI.fDescr]{fDescr} \hyperref[TEI.fsDescr]{fsDescr}\par 
    \item[linking: ]
   \hyperref[TEI.ab]{ab} \hyperref[TEI.seg]{seg}\par 
    \item[msdescription: ]
   \hyperref[TEI.accMat]{accMat} \hyperref[TEI.acquisition]{acquisition} \hyperref[TEI.additions]{additions} \hyperref[TEI.collation]{collation} \hyperref[TEI.condition]{condition} \hyperref[TEI.custEvent]{custEvent} \hyperref[TEI.decoNote]{decoNote} \hyperref[TEI.filiation]{filiation} \hyperref[TEI.foliation]{foliation} \hyperref[TEI.layout]{layout} \hyperref[TEI.musicNotation]{musicNotation} \hyperref[TEI.origin]{origin} \hyperref[TEI.provenance]{provenance} \hyperref[TEI.signatures]{signatures} \hyperref[TEI.source]{source} \hyperref[TEI.summary]{summary} \hyperref[TEI.support]{support} \hyperref[TEI.surrogates]{surrogates} \hyperref[TEI.typeNote]{typeNote}\par 
    \item[spoken: ]
   \hyperref[TEI.annotationBlock]{annotationBlock}\par 
    \item[standOff: ]
   \hyperref[TEI.listAnnotation]{listAnnotation}\par 
    \item[textstructure: ]
   \hyperref[TEI.back]{back} \hyperref[TEI.body]{body} \hyperref[TEI.div]{div} \hyperref[TEI.docEdition]{docEdition} \hyperref[TEI.titlePart]{titlePart}\par 
    \item[transcr: ]
   \hyperref[TEI.damage]{damage} \hyperref[TEI.metamark]{metamark} \hyperref[TEI.mod]{mod} \hyperref[TEI.restore]{restore} \hyperref[TEI.retrace]{retrace} \hyperref[TEI.secl]{secl} \hyperref[TEI.supplied]{supplied} \hyperref[TEI.surplus]{surplus}
    \item[{Peut contenir}]
  
    \item[analysis: ]
   \hyperref[TEI.interp]{interp} \hyperref[TEI.interpGrp]{interpGrp} \hyperref[TEI.span]{span} \hyperref[TEI.spanGrp]{spanGrp}\par 
    \item[core: ]
   \hyperref[TEI.binaryObject]{binaryObject} \hyperref[TEI.cb]{cb} \hyperref[TEI.gap]{gap} \hyperref[TEI.gb]{gb} \hyperref[TEI.graphic]{graphic} \hyperref[TEI.head]{head} \hyperref[TEI.index]{index} \hyperref[TEI.lb]{lb} \hyperref[TEI.media]{media} \hyperref[TEI.meeting]{meeting} \hyperref[TEI.milestone]{milestone} \hyperref[TEI.note]{note} \hyperref[TEI.pb]{pb}\par 
    \item[derived-module-tei.istex: ]
   \hyperref[TEI.math]{math} \hyperref[TEI.mrow]{mrow}\par 
    \item[figures: ]
   \hyperref[TEI.figure]{figure} \hyperref[TEI.formula]{formula} \hyperref[TEI.notatedMusic]{notatedMusic} \hyperref[TEI.row]{row}\par 
    \item[iso-fs: ]
   \hyperref[TEI.fLib]{fLib} \hyperref[TEI.fs]{fs} \hyperref[TEI.fvLib]{fvLib}\par 
    \item[linking: ]
   \hyperref[TEI.alt]{alt} \hyperref[TEI.altGrp]{altGrp} \hyperref[TEI.anchor]{anchor} \hyperref[TEI.join]{join} \hyperref[TEI.joinGrp]{joinGrp} \hyperref[TEI.link]{link} \hyperref[TEI.linkGrp]{linkGrp} \hyperref[TEI.timeline]{timeline}\par 
    \item[msdescription: ]
   \hyperref[TEI.source]{source}\par 
    \item[textstructure: ]
   \hyperref[TEI.docAuthor]{docAuthor} \hyperref[TEI.docDate]{docDate}\par 
    \item[transcr: ]
   \hyperref[TEI.addSpan]{addSpan} \hyperref[TEI.damageSpan]{damageSpan} \hyperref[TEI.delSpan]{delSpan} \hyperref[TEI.fw]{fw} \hyperref[TEI.listTranspose]{listTranspose} \hyperref[TEI.metamark]{metamark} \hyperref[TEI.space]{space} \hyperref[TEI.substJoin]{substJoin}
    \item[{Note}]
  \par
Contient un titre facultatif et une suite de rangées.\par
Toute information relative à la restitution sera exprimée avec l'attribut global {\itshape rend} appliqué au tableau, à la rangée, ou à la cellule selon le cas.
    \item[{Exemple}]
  \leavevmode\bgroup\exampleFont \begin{shaded}\noindent\mbox{}{<\textbf{table}\hspace*{6pt}{cols}="{4}"\hspace*{6pt}{rows}="{4}">}\mbox{}\newline 
\hspace*{6pt}{<\textbf{head}>}Persistance de la neige dans les Alpes suisses (Denzler). {</\textbf{head}>}\mbox{}\newline 
\hspace*{6pt}{<\textbf{row}>}\mbox{}\newline 
\hspace*{6pt}\hspace*{6pt}{<\textbf{cell}\hspace*{6pt}{role}="{label}">}A l'altitude de{</\textbf{cell}>}\mbox{}\newline 
\hspace*{6pt}\hspace*{6pt}{<\textbf{cell}\hspace*{6pt}{role}="{data}">}650 m.{</\textbf{cell}>}\mbox{}\newline 
\hspace*{6pt}\hspace*{6pt}{<\textbf{cell}\hspace*{6pt}{role}="{data}">}1300m.{</\textbf{cell}>}\mbox{}\newline 
\hspace*{6pt}\hspace*{6pt}{<\textbf{cell}\hspace*{6pt}{role}="{data}">}1950m.{</\textbf{cell}>}\mbox{}\newline 
\hspace*{6pt}\hspace*{6pt}{<\textbf{cell}\hspace*{6pt}{role}="{data}">}2700m.{</\textbf{cell}>}\mbox{}\newline 
\hspace*{6pt}{</\textbf{row}>}\mbox{}\newline 
\hspace*{6pt}{<\textbf{row}>}\mbox{}\newline 
\hspace*{6pt}\hspace*{6pt}{<\textbf{cell}\hspace*{6pt}{role}="{label}">}la neige reste{</\textbf{cell}>}\mbox{}\newline 
\hspace*{6pt}\hspace*{6pt}{<\textbf{cell}\hspace*{6pt}{role}="{data}">}77 jours.{</\textbf{cell}>}\mbox{}\newline 
\hspace*{6pt}\hspace*{6pt}{<\textbf{cell}\hspace*{6pt}{role}="{data}">} 200 jours.{</\textbf{cell}>}\mbox{}\newline 
\hspace*{6pt}\hspace*{6pt}{<\textbf{cell}\hspace*{6pt}{role}="{data}">} 245 jours.{</\textbf{cell}>}\mbox{}\newline 
\hspace*{6pt}\hspace*{6pt}{<\textbf{cell}\hspace*{6pt}{role}="{data}">} 365 jours.{</\textbf{cell}>}\mbox{}\newline 
\hspace*{6pt}{</\textbf{row}>}\mbox{}\newline 
{</\textbf{table}>}\end{shaded}\egroup 


    \item[{Modèle de contenu}]
  \mbox{}\hfill\\[-10pt]\begin{Verbatim}[fontsize=\small]
<content>
 <sequence maxOccurs="1" minOccurs="1">
  <alternate maxOccurs="unbounded"
   minOccurs="0">
   <classRef key="model.headLike"/>
   <classRef key="model.global"/>
  </alternate>
  <alternate maxOccurs="1" minOccurs="1">
   <sequence maxOccurs="unbounded"
    minOccurs="1">
    <elementRef key="row"/>
    <classRef key="model.global"
     maxOccurs="unbounded" minOccurs="0"/>
   </sequence>
   <sequence maxOccurs="unbounded"
    minOccurs="1">
    <classRef key="model.graphicLike"/>
    <classRef key="model.global"
     maxOccurs="unbounded" minOccurs="0"/>
   </sequence>
  </alternate>
  <sequence maxOccurs="unbounded"
   minOccurs="0">
   <classRef key="model.divBottom"/>
   <classRef key="model.global"
    maxOccurs="unbounded" minOccurs="0"/>
  </sequence>
 </sequence>
</content>
    
\end{Verbatim}

    \item[{Schéma Declaration}]
  \mbox{}\hfill\\[-10pt]\begin{Verbatim}[fontsize=\small]
element table
{
   tei_att.global.attributes,
   tei_att.typed.attributes,
   attribute rows { text }?,
   attribute cols { text }?,
   (
      ( tei_model.headLike | tei_model.global )*,
      (
         ( tei_row, tei_model.global* )+
       | ( tei_model.graphicLike, tei_model.global* )+
      ),
      ( tei_model.divBottom, tei_model.global* )*
   )
}
\end{Verbatim}

\end{reflist}  \index{taxonomy=<taxonomy>|oddindex}
\begin{reflist}
\item[]\begin{specHead}{TEI.taxonomy}{<taxonomy> }(taxinomie) définit une typologie soit implicitement au moyen d’une référence bibliographique, soit explicitement au moyen d’une taxinomie structurée. [\xref{http://www.tei-c.org/release/doc/tei-p5-doc/en/html/HD.html\#HD55}{2.3.7. The Classification Declaration}]\end{specHead} 
    \item[{Module}]
  header
    \item[{Attributs}]
  Attributs \hyperref[TEI.att.global]{att.global} (\textit{@xml:id}, \textit{@n}, \textit{@xml:lang}, \textit{@xml:base}, \textit{@xml:space})  (\hyperref[TEI.att.global.rendition]{att.global.rendition} (\textit{@rend}, \textit{@style}, \textit{@rendition})) (\hyperref[TEI.att.global.linking]{att.global.linking} (\textit{@corresp}, \textit{@synch}, \textit{@sameAs}, \textit{@copyOf}, \textit{@next}, \textit{@prev}, \textit{@exclude}, \textit{@select})) (\hyperref[TEI.att.global.analytic]{att.global.analytic} (\textit{@ana})) (\hyperref[TEI.att.global.facs]{att.global.facs} (\textit{@facs})) (\hyperref[TEI.att.global.change]{att.global.change} (\textit{@change})) (\hyperref[TEI.att.global.responsibility]{att.global.responsibility} (\textit{@cert}, \textit{@resp})) (\hyperref[TEI.att.global.source]{att.global.source} (\textit{@source}))
    \item[{Contenu dans}]
  
    \item[header: ]
   \hyperref[TEI.classDecl]{classDecl} \hyperref[TEI.taxonomy]{taxonomy}
    \item[{Peut contenir}]
  
    \item[core: ]
   \hyperref[TEI.bibl]{bibl} \hyperref[TEI.biblStruct]{biblStruct} \hyperref[TEI.desc]{desc} \hyperref[TEI.gloss]{gloss} \hyperref[TEI.listBibl]{listBibl}\par 
    \item[header: ]
   \hyperref[TEI.biblFull]{biblFull} \hyperref[TEI.category]{category} \hyperref[TEI.taxonomy]{taxonomy}\par 
    \item[msdescription: ]
   \hyperref[TEI.msDesc]{msDesc}
    \item[{Note}]
  \par
Nested taxonomies are common in many fields, so the \hyperref[TEI.taxonomy]{<taxonomy>} element can be nested.
    \item[{Exemple}]
  \leavevmode\bgroup\exampleFont \begin{shaded}\noindent\mbox{}{<\textbf{taxonomy}\hspace*{6pt}{xml:id}="{fr\textunderscore tax.a}">}\mbox{}\newline 
\hspace*{6pt}{<\textbf{category}\hspace*{6pt}{xml:id}="{fr\textunderscore tax.a.a}">}\mbox{}\newline 
\hspace*{6pt}\hspace*{6pt}{<\textbf{catDesc}>}littérature{</\textbf{catDesc}>}\mbox{}\newline 
\hspace*{6pt}{</\textbf{category}>}\mbox{}\newline 
\hspace*{6pt}{<\textbf{category}\hspace*{6pt}{xml:id}="{fr\textunderscore tax.a.a.1}">}\mbox{}\newline 
\hspace*{6pt}\hspace*{6pt}{<\textbf{catDesc}>}Drame bourgeois{</\textbf{catDesc}>}\mbox{}\newline 
\hspace*{6pt}{</\textbf{category}>}\mbox{}\newline 
\hspace*{6pt}{<\textbf{category}\hspace*{6pt}{xml:id}="{fr\textunderscore tax.a.a.1.α}">}\mbox{}\newline 
\hspace*{6pt}\hspace*{6pt}{<\textbf{catDesc}>}Comédie larmoyante{</\textbf{catDesc}>}\mbox{}\newline 
\hspace*{6pt}{</\textbf{category}>}\mbox{}\newline 
\hspace*{6pt}{<\textbf{category}\hspace*{6pt}{xml:id}="{fr\textunderscore tax.a.b}">}\mbox{}\newline 
\hspace*{6pt}\hspace*{6pt}{<\textbf{catDesc}>}Correspondance{</\textbf{catDesc}>}\mbox{}\newline 
\hspace*{6pt}{</\textbf{category}>}\mbox{}\newline 
\hspace*{6pt}{<\textbf{category}\hspace*{6pt}{xml:id}="{fr\textunderscore tax.a.b.1.a}">}\mbox{}\newline 
\hspace*{6pt}\hspace*{6pt}{<\textbf{catDesc}>}Dernières lettres{</\textbf{catDesc}>}\mbox{}\newline 
\hspace*{6pt}{</\textbf{category}>}\mbox{}\newline 
\hspace*{6pt}{<\textbf{category}\hspace*{6pt}{xml:id}="{fr\textunderscore tax.a.c.}">}\mbox{}\newline 
\hspace*{6pt}\hspace*{6pt}{<\textbf{catDesc}>}Littérature européenne -- 16e siècle{</\textbf{catDesc}>}\mbox{}\newline 
\hspace*{6pt}{</\textbf{category}>}\mbox{}\newline 
\hspace*{6pt}{<\textbf{category}\hspace*{6pt}{xml:id}="{fr\textunderscore tax.a.c.1}">}\mbox{}\newline 
\hspace*{6pt}\hspace*{6pt}{<\textbf{catDesc}>}Satire de la Renaissance {</\textbf{catDesc}>}\mbox{}\newline 
\hspace*{6pt}{</\textbf{category}>}\mbox{}\newline 
\hspace*{6pt}{<\textbf{category}\hspace*{6pt}{xml:id}="{fr\textunderscore tax.a.d}">}\mbox{}\newline 
\hspace*{6pt}\hspace*{6pt}{<\textbf{catDesc}>}Récits de voyage{</\textbf{catDesc}>}\mbox{}\newline 
\hspace*{6pt}{</\textbf{category}>}\mbox{}\newline 
\hspace*{6pt}{<\textbf{category}\hspace*{6pt}{xml:id}="{fr\textunderscore tax.a.d.1}">}\mbox{}\newline 
\hspace*{6pt}\hspace*{6pt}{<\textbf{catDesc}>}Récits de la mer {</\textbf{catDesc}>}\mbox{}\newline 
\hspace*{6pt}{</\textbf{category}>}\mbox{}\newline 
{</\textbf{taxonomy}>}\mbox{}\newline 
{<\textbf{bibl}>}indexation selon le système d'indexation RAMEAU, géré par la Bibliothèque nationale de\mbox{}\newline 
 France{</\textbf{bibl}>}\end{shaded}\egroup 


    \item[{Modèle de contenu}]
  \mbox{}\hfill\\[-10pt]\begin{Verbatim}[fontsize=\small]
<content>
 <alternate maxOccurs="1" minOccurs="1">
  <alternate maxOccurs="1" minOccurs="1">
   <alternate maxOccurs="unbounded"
    minOccurs="1">
    <elementRef key="category"/>
    <elementRef key="taxonomy"/>
   </alternate>
   <sequence maxOccurs="1" minOccurs="1">
    <alternate maxOccurs="unbounded"
     minOccurs="1">
     <classRef key="model.glossLike"/>
     <classRef key="model.descLike"/>
    </alternate>
    <alternate maxOccurs="unbounded"
     minOccurs="0">
     <elementRef key="category"/>
     <elementRef key="taxonomy"/>
    </alternate>
   </sequence>
  </alternate>
  <sequence maxOccurs="1" minOccurs="1">
   <classRef key="model.biblLike"/>
   <alternate maxOccurs="unbounded"
    minOccurs="0">
    <elementRef key="category"/>
    <elementRef key="taxonomy"/>
   </alternate>
  </sequence>
 </alternate>
</content>
    
\end{Verbatim}

    \item[{Schéma Declaration}]
  \mbox{}\hfill\\[-10pt]\begin{Verbatim}[fontsize=\small]
element taxonomy
{
   tei_att.global.attributes,
   (
      (
         ( tei_category | tei_taxonomy )+
       | (
            ( tei_model.glossLike | tei_model.descLike )+,
            ( tei_category | tei_taxonomy )*
         )
      )
    | ( tei_model.biblLike, ( tei_category | tei_taxonomy )* )
   )
}
\end{Verbatim}

\end{reflist}  \index{teiCorpus=<teiCorpus>|oddindex}\index{version=@version!<teiCorpus>|oddindex}
\begin{reflist}
\item[]\begin{specHead}{TEI.teiCorpus}{<teiCorpus> }contient la totalité d'un corpus encodé selon la TEI, comprenant un seul en-tête de corpus et un ou plusieurs éléments TEI dont chacun contient un seul en-tête textuel et un texte. [\xref{http://www.tei-c.org/release/doc/tei-p5-doc/en/html/DS.html\#DS}{4. Default Text Structure} \xref{http://www.tei-c.org/release/doc/tei-p5-doc/en/html/CC.html\#CCDEF}{15.1. Varieties of Composite Text}]\end{specHead} 
    \item[{Module}]
  core
    \item[{Attributs}]
  Attributs \hyperref[TEI.att.global]{att.global} (\textit{@xml:id}, \textit{@n}, \textit{@xml:lang}, \textit{@xml:base}, \textit{@xml:space})  (\hyperref[TEI.att.global.rendition]{att.global.rendition} (\textit{@rend}, \textit{@style}, \textit{@rendition})) (\hyperref[TEI.att.global.linking]{att.global.linking} (\textit{@corresp}, \textit{@synch}, \textit{@sameAs}, \textit{@copyOf}, \textit{@next}, \textit{@prev}, \textit{@exclude}, \textit{@select})) (\hyperref[TEI.att.global.analytic]{att.global.analytic} (\textit{@ana})) (\hyperref[TEI.att.global.facs]{att.global.facs} (\textit{@facs})) (\hyperref[TEI.att.global.change]{att.global.change} (\textit{@change})) (\hyperref[TEI.att.global.responsibility]{att.global.responsibility} (\textit{@cert}, \textit{@resp})) (\hyperref[TEI.att.global.source]{att.global.source} (\textit{@source})) \hyperref[TEI.att.typed]{att.typed} (\textit{@type}, \textit{@subtype}) \hfil\\[-10pt]\begin{sansreflist}
    \item[@version]
  la version du modèle TEI
\begin{reflist}
    \item[{Statut}]
  Optionel
    \item[{Type de données}]
  \hyperref[TEI.teidata.version]{teidata.version}
    \item[{Valeur par défaut}]
  5.0
\end{reflist}  
\end{sansreflist}  
    \item[{Contenu dans}]
  
    \item[core: ]
   \hyperref[TEI.teiCorpus]{teiCorpus}
    \item[{Peut contenir}]
  
    \item[core: ]
   \hyperref[TEI.teiCorpus]{teiCorpus}\par 
    \item[header: ]
   \hyperref[TEI.teiHeader]{teiHeader}\par 
    \item[iso-fs: ]
   \hyperref[TEI.fsdDecl]{fsdDecl}\par 
    \item[standOff: ]
   \hyperref[TEI.standOff]{standOff}\par 
    \item[textstructure: ]
   \hyperref[TEI.TEI]{TEI} \hyperref[TEI.text]{text}\par 
    \item[transcr: ]
   \hyperref[TEI.facsimile]{facsimile} \hyperref[TEI.sourceDoc]{sourceDoc}
    \item[{Note}]
  \par
Cet élément doit contenir un en-tête TEI pour le corpus, et une suite d'éléments \hyperref[TEI.TEI]{<TEI>}, correspondant à autant de textes.\par
Cet élément est obligatoire quand il est applicable.
    \item[{Exemple}]
  \leavevmode\bgroup\exampleFont \begin{shaded}\noindent\mbox{}{<\textbf{teiCorpus} xmlns="http://www.tei-c.org/ns/1.0">}\mbox{}\newline 
\hspace*{6pt}{<\textbf{teiHeader}>}\mbox{}\newline 
\textit{<!--[en-tête du corpus]-->}\mbox{}\newline 
\hspace*{6pt}{</\textbf{teiHeader}>}\mbox{}\newline 
\hspace*{6pt}{<\textbf{TEI}>}\mbox{}\newline 
\hspace*{6pt}\hspace*{6pt}{<\textbf{teiHeader}>}\mbox{}\newline 
\textit{<!--[en-tête du premier texte]-->}\mbox{}\newline 
\hspace*{6pt}\hspace*{6pt}{</\textbf{teiHeader}>}\mbox{}\newline 
\hspace*{6pt}\hspace*{6pt}{<\textbf{text}>}\mbox{}\newline 
\textit{<!--[premier texte du corpus]-->}\mbox{}\newline 
\hspace*{6pt}\hspace*{6pt}{</\textbf{text}>}\mbox{}\newline 
\hspace*{6pt}{</\textbf{TEI}>}\mbox{}\newline 
\hspace*{6pt}{<\textbf{TEI}>}\mbox{}\newline 
\hspace*{6pt}\hspace*{6pt}{<\textbf{teiHeader}>}\mbox{}\newline 
\textit{<!--[en-tête du deuxième texte]-->}\mbox{}\newline 
\hspace*{6pt}\hspace*{6pt}{</\textbf{teiHeader}>}\mbox{}\newline 
\hspace*{6pt}\hspace*{6pt}{<\textbf{text}>}\mbox{}\newline 
\textit{<!--[deuxième texte du corpus]-->}\mbox{}\newline 
\hspace*{6pt}\hspace*{6pt}{</\textbf{text}>}\mbox{}\newline 
\hspace*{6pt}{</\textbf{TEI}>}\mbox{}\newline 
{</\textbf{teiCorpus}>}\end{shaded}\egroup 


    \item[{Modèle de contenu}]
  \mbox{}\hfill\\[-10pt]\begin{Verbatim}[fontsize=\small]
<content>
 <sequence maxOccurs="1" minOccurs="1">
  <elementRef key="teiHeader"/>
  <alternate maxOccurs="1" minOccurs="1">
   <sequence maxOccurs="1" minOccurs="1">
    <classRef key="model.resourceLike"
     maxOccurs="unbounded" minOccurs="1"/>
    <alternate maxOccurs="unbounded"
     minOccurs="0">
     <elementRef key="TEI"/>
     <elementRef key="teiCorpus"/>
    </alternate>
   </sequence>
   <alternate maxOccurs="unbounded"
    minOccurs="1">
    <elementRef key="TEI"/>
    <elementRef key="teiCorpus"/>
   </alternate>
  </alternate>
 </sequence>
</content>
    
\end{Verbatim}

    \item[{Schéma Declaration}]
  \mbox{}\hfill\\[-10pt]\begin{Verbatim}[fontsize=\small]
element teiCorpus
{
   tei_att.global.attributes,
   tei_att.typed.attributes,
   attribute version { text }?,
   (
      tei_teiHeader,
      (
         ( tei_model.resourceLike+, ( tei_TEI | tei_teiCorpus )* )
       | ( tei_TEI | tei_teiCorpus )+
      )
   )
}
\end{Verbatim}

\end{reflist}  \index{teiHeader=<teiHeader>|oddindex}
\begin{reflist}
\item[]\begin{specHead}{TEI.teiHeader}{<teiHeader> }(en-tête TEI) fournit des informations descriptives et déclaratives qui constituent une page de titre électronique au début de tout texte conforme à la TEI. [\xref{http://www.tei-c.org/release/doc/tei-p5-doc/en/html/HD.html\#HD11}{2.1.1. The TEI Header and Its Components} \xref{http://www.tei-c.org/release/doc/tei-p5-doc/en/html/CC.html\#CCDEF}{15.1. Varieties of Composite Text}]\end{specHead} 
    \item[{Module}]
  header
    \item[{Attributs}]
  Attributs \hyperref[TEI.att.global]{att.global} (\textit{@xml:id}, \textit{@n}, \textit{@xml:lang}, \textit{@xml:base}, \textit{@xml:space})  (\hyperref[TEI.att.global.rendition]{att.global.rendition} (\textit{@rend}, \textit{@style}, \textit{@rendition})) (\hyperref[TEI.att.global.linking]{att.global.linking} (\textit{@corresp}, \textit{@synch}, \textit{@sameAs}, \textit{@copyOf}, \textit{@next}, \textit{@prev}, \textit{@exclude}, \textit{@select})) (\hyperref[TEI.att.global.analytic]{att.global.analytic} (\textit{@ana})) (\hyperref[TEI.att.global.facs]{att.global.facs} (\textit{@facs})) (\hyperref[TEI.att.global.change]{att.global.change} (\textit{@change})) (\hyperref[TEI.att.global.responsibility]{att.global.responsibility} (\textit{@cert}, \textit{@resp})) (\hyperref[TEI.att.global.source]{att.global.source} (\textit{@source}))
    \item[{Contenu dans}]
  
    \item[core: ]
   \hyperref[TEI.teiCorpus]{teiCorpus}\par 
    \item[standOff: ]
   \hyperref[TEI.standOff]{standOff}\par 
    \item[textstructure: ]
   \hyperref[TEI.TEI]{TEI}
    \item[{Peut contenir}]
  
    \item[header: ]
   \hyperref[TEI.encodingDesc]{encodingDesc} \hyperref[TEI.fileDesc]{fileDesc} \hyperref[TEI.profileDesc]{profileDesc} \hyperref[TEI.revisionDesc]{revisionDesc}
    \item[{Note}]
  \par
Un des seuls éléments obligatoires dans tout document TEI. 
    \item[{Exemple}]
  \leavevmode\bgroup\exampleFont \begin{shaded}\noindent\mbox{}{<\textbf{teiHeader}>}\mbox{}\newline 
\hspace*{6pt}{<\textbf{fileDesc}>}\mbox{}\newline 
\hspace*{6pt}\hspace*{6pt}{<\textbf{titleStmt}>}\mbox{}\newline 
\hspace*{6pt}\hspace*{6pt}\hspace*{6pt}{<\textbf{title}>}La Parisienne{</\textbf{title}>}\mbox{}\newline 
\hspace*{6pt}\hspace*{6pt}\hspace*{6pt}{<\textbf{author}>}Henry BECQUE{</\textbf{author}>}\mbox{}\newline 
\hspace*{6pt}\hspace*{6pt}{</\textbf{titleStmt}>}\mbox{}\newline 
\hspace*{6pt}\hspace*{6pt}{<\textbf{publicationStmt}>}\mbox{}\newline 
\hspace*{6pt}\hspace*{6pt}\hspace*{6pt}{<\textbf{distributor}>}ATILF (Analyse et Traitement Informatique de la Langue Française){</\textbf{distributor}>}\mbox{}\newline 
\hspace*{6pt}\hspace*{6pt}\hspace*{6pt}{<\textbf{idno}\hspace*{6pt}{type}="{FRANTEXT}">}L434{</\textbf{idno}>}\mbox{}\newline 
\hspace*{6pt}\hspace*{6pt}\hspace*{6pt}{<\textbf{address}>}\mbox{}\newline 
\hspace*{6pt}\hspace*{6pt}\hspace*{6pt}\hspace*{6pt}{<\textbf{addrLine}>}44, avenue de la Libération{</\textbf{addrLine}>}\mbox{}\newline 
\hspace*{6pt}\hspace*{6pt}\hspace*{6pt}\hspace*{6pt}{<\textbf{addrLine}>}BP 30687{</\textbf{addrLine}>}\mbox{}\newline 
\hspace*{6pt}\hspace*{6pt}\hspace*{6pt}\hspace*{6pt}{<\textbf{addrLine}>}54063 Nancy Cedex{</\textbf{addrLine}>}\mbox{}\newline 
\hspace*{6pt}\hspace*{6pt}\hspace*{6pt}\hspace*{6pt}{<\textbf{addrLine}>}FRANCE{</\textbf{addrLine}>}\mbox{}\newline 
\hspace*{6pt}\hspace*{6pt}\hspace*{6pt}{</\textbf{address}>}\mbox{}\newline 
\hspace*{6pt}\hspace*{6pt}\hspace*{6pt}{<\textbf{availability}\hspace*{6pt}{status}="{free}">}\mbox{}\newline 
\hspace*{6pt}\hspace*{6pt}\hspace*{6pt}\hspace*{6pt}{<\textbf{p}>}Dans un cadre de recherche ou d'enseignement{</\textbf{p}>}\mbox{}\newline 
\hspace*{6pt}\hspace*{6pt}\hspace*{6pt}{</\textbf{availability}>}\mbox{}\newline 
\hspace*{6pt}\hspace*{6pt}{</\textbf{publicationStmt}>}\mbox{}\newline 
\hspace*{6pt}\hspace*{6pt}{<\textbf{sourceDesc}>}\mbox{}\newline 
\hspace*{6pt}\hspace*{6pt}\hspace*{6pt}{<\textbf{biblStruct}>}\mbox{}\newline 
\hspace*{6pt}\hspace*{6pt}\hspace*{6pt}\hspace*{6pt}{<\textbf{monogr}>}\mbox{}\newline 
\hspace*{6pt}\hspace*{6pt}\hspace*{6pt}\hspace*{6pt}\hspace*{6pt}{<\textbf{imprint}>}\mbox{}\newline 
\hspace*{6pt}\hspace*{6pt}\hspace*{6pt}\hspace*{6pt}\hspace*{6pt}\hspace*{6pt}{<\textbf{publisher}>}Paris : Fasquelle, 1922.{</\textbf{publisher}>}\mbox{}\newline 
\hspace*{6pt}\hspace*{6pt}\hspace*{6pt}\hspace*{6pt}\hspace*{6pt}{</\textbf{imprint}>}\mbox{}\newline 
\hspace*{6pt}\hspace*{6pt}\hspace*{6pt}\hspace*{6pt}{</\textbf{monogr}>}\mbox{}\newline 
\hspace*{6pt}\hspace*{6pt}\hspace*{6pt}{</\textbf{biblStruct}>}\mbox{}\newline 
\hspace*{6pt}\hspace*{6pt}{</\textbf{sourceDesc}>}\mbox{}\newline 
\hspace*{6pt}{</\textbf{fileDesc}>}\mbox{}\newline 
\hspace*{6pt}{<\textbf{profileDesc}>}\mbox{}\newline 
\hspace*{6pt}\hspace*{6pt}{<\textbf{creation}>}\mbox{}\newline 
\hspace*{6pt}\hspace*{6pt}\hspace*{6pt}{<\textbf{date}>}1885{</\textbf{date}>}\mbox{}\newline 
\hspace*{6pt}\hspace*{6pt}{</\textbf{creation}>}\mbox{}\newline 
\hspace*{6pt}{</\textbf{profileDesc}>}\mbox{}\newline 
{</\textbf{teiHeader}>}\end{shaded}\egroup 


    \item[{Modèle de contenu}]
  \mbox{}\hfill\\[-10pt]\begin{Verbatim}[fontsize=\small]
<content>
 <sequence maxOccurs="1" minOccurs="1">
  <elementRef key="fileDesc"/>
  <classRef key="model.teiHeaderPart"
   maxOccurs="unbounded" minOccurs="0"/>
  <elementRef key="revisionDesc"
   minOccurs="0"/>
 </sequence>
</content>
    
\end{Verbatim}

    \item[{Schéma Declaration}]
  \mbox{}\hfill\\[-10pt]\begin{Verbatim}[fontsize=\small]
element teiHeader
{
   tei_att.global.attributes,
   ( tei_fileDesc, tei_model.teiHeaderPart*, tei_revisionDesc? )
}
\end{Verbatim}

\end{reflist}  \index{term=<term>|oddindex}\index{level=@level!<term>|oddindex}
\begin{reflist}
\item[]\begin{specHead}{TEI.term}{<term> }(terme) contient un mot simple, un mot composé ou un symbole, qui est considéré comme un terme technique. [\xref{http://www.tei-c.org/release/doc/tei-p5-doc/en/html/CO.html\#COHQU}{3.3.4. Terms, Glosses, Equivalents, and Descriptions}]\end{specHead} 
    \item[{Module}]
  core
    \item[{Attributs}]
  Attributs \hyperref[TEI.att.global]{att.global} (\textit{@xml:id}, \textit{@n}, \textit{@xml:lang}, \textit{@xml:base}, \textit{@xml:space})  (\hyperref[TEI.att.global.rendition]{att.global.rendition} (\textit{@rend}, \textit{@style}, \textit{@rendition})) (\hyperref[TEI.att.global.linking]{att.global.linking} (\textit{@corresp}, \textit{@synch}, \textit{@sameAs}, \textit{@copyOf}, \textit{@next}, \textit{@prev}, \textit{@exclude}, \textit{@select})) (\hyperref[TEI.att.global.analytic]{att.global.analytic} (\textit{@ana})) (\hyperref[TEI.att.global.facs]{att.global.facs} (\textit{@facs})) (\hyperref[TEI.att.global.change]{att.global.change} (\textit{@change})) (\hyperref[TEI.att.global.responsibility]{att.global.responsibility} (\textit{@cert}, \textit{@resp})) (\hyperref[TEI.att.global.source]{att.global.source} (\textit{@source})) \hyperref[TEI.att.declaring]{att.declaring} (\textit{@decls}) \hyperref[TEI.att.pointing]{att.pointing} (\textit{@targetLang}, \textit{@target}, \textit{@evaluate}) \hyperref[TEI.att.typed]{att.typed} (\textit{@type}, \textit{@subtype}) \hyperref[TEI.att.canonical]{att.canonical} (\textit{@key}, \textit{@ref}) \hyperref[TEI.att.sortable]{att.sortable} (\textit{@sortKey}) \hyperref[TEI.att.cReferencing]{att.cReferencing} (\textit{@cRef}) \hfil\\[-10pt]\begin{sansreflist}
    \item[@level]
  Niveau hiérarchique du terme dans le code de classement
\begin{reflist}
    \item[{Statut}]
  Optionel
    \item[{Type de données}]
  \hyperref[TEI.teidata.count]{teidata.count}
\end{reflist}  
\end{sansreflist}  
    \item[{Membre du}]
  \hyperref[TEI.model.emphLike]{model.emphLike}
    \item[{Contenu dans}]
  
    \item[analysis: ]
   \hyperref[TEI.cl]{cl} \hyperref[TEI.phr]{phr} \hyperref[TEI.s]{s} \hyperref[TEI.span]{span}\par 
    \item[core: ]
   \hyperref[TEI.abbr]{abbr} \hyperref[TEI.add]{add} \hyperref[TEI.addrLine]{addrLine} \hyperref[TEI.author]{author} \hyperref[TEI.bibl]{bibl} \hyperref[TEI.biblScope]{biblScope} \hyperref[TEI.citedRange]{citedRange} \hyperref[TEI.corr]{corr} \hyperref[TEI.date]{date} \hyperref[TEI.del]{del} \hyperref[TEI.desc]{desc} \hyperref[TEI.distinct]{distinct} \hyperref[TEI.editor]{editor} \hyperref[TEI.email]{email} \hyperref[TEI.emph]{emph} \hyperref[TEI.expan]{expan} \hyperref[TEI.foreign]{foreign} \hyperref[TEI.gloss]{gloss} \hyperref[TEI.head]{head} \hyperref[TEI.headItem]{headItem} \hyperref[TEI.headLabel]{headLabel} \hyperref[TEI.hi]{hi} \hyperref[TEI.index]{index} \hyperref[TEI.item]{item} \hyperref[TEI.l]{l} \hyperref[TEI.label]{label} \hyperref[TEI.measure]{measure} \hyperref[TEI.meeting]{meeting} \hyperref[TEI.mentioned]{mentioned} \hyperref[TEI.name]{name} \hyperref[TEI.note]{note} \hyperref[TEI.num]{num} \hyperref[TEI.orig]{orig} \hyperref[TEI.p]{p} \hyperref[TEI.pubPlace]{pubPlace} \hyperref[TEI.publisher]{publisher} \hyperref[TEI.q]{q} \hyperref[TEI.quote]{quote} \hyperref[TEI.ref]{ref} \hyperref[TEI.reg]{reg} \hyperref[TEI.resp]{resp} \hyperref[TEI.rs]{rs} \hyperref[TEI.said]{said} \hyperref[TEI.sic]{sic} \hyperref[TEI.soCalled]{soCalled} \hyperref[TEI.speaker]{speaker} \hyperref[TEI.stage]{stage} \hyperref[TEI.street]{street} \hyperref[TEI.term]{term} \hyperref[TEI.textLang]{textLang} \hyperref[TEI.time]{time} \hyperref[TEI.title]{title} \hyperref[TEI.unclear]{unclear}\par 
    \item[figures: ]
   \hyperref[TEI.cell]{cell} \hyperref[TEI.figDesc]{figDesc}\par 
    \item[header: ]
   \hyperref[TEI.authority]{authority} \hyperref[TEI.change]{change} \hyperref[TEI.classCode]{classCode} \hyperref[TEI.creation]{creation} \hyperref[TEI.distributor]{distributor} \hyperref[TEI.edition]{edition} \hyperref[TEI.extent]{extent} \hyperref[TEI.funder]{funder} \hyperref[TEI.keywords]{keywords} \hyperref[TEI.language]{language} \hyperref[TEI.licence]{licence} \hyperref[TEI.rendition]{rendition}\par 
    \item[iso-fs: ]
   \hyperref[TEI.fDescr]{fDescr} \hyperref[TEI.fsDescr]{fsDescr}\par 
    \item[linking: ]
   \hyperref[TEI.ab]{ab} \hyperref[TEI.seg]{seg}\par 
    \item[msdescription: ]
   \hyperref[TEI.accMat]{accMat} \hyperref[TEI.acquisition]{acquisition} \hyperref[TEI.additions]{additions} \hyperref[TEI.catchwords]{catchwords} \hyperref[TEI.collation]{collation} \hyperref[TEI.colophon]{colophon} \hyperref[TEI.condition]{condition} \hyperref[TEI.custEvent]{custEvent} \hyperref[TEI.decoNote]{decoNote} \hyperref[TEI.explicit]{explicit} \hyperref[TEI.filiation]{filiation} \hyperref[TEI.finalRubric]{finalRubric} \hyperref[TEI.foliation]{foliation} \hyperref[TEI.heraldry]{heraldry} \hyperref[TEI.incipit]{incipit} \hyperref[TEI.layout]{layout} \hyperref[TEI.material]{material} \hyperref[TEI.musicNotation]{musicNotation} \hyperref[TEI.objectType]{objectType} \hyperref[TEI.origDate]{origDate} \hyperref[TEI.origPlace]{origPlace} \hyperref[TEI.origin]{origin} \hyperref[TEI.provenance]{provenance} \hyperref[TEI.rubric]{rubric} \hyperref[TEI.secFol]{secFol} \hyperref[TEI.signatures]{signatures} \hyperref[TEI.source]{source} \hyperref[TEI.stamp]{stamp} \hyperref[TEI.summary]{summary} \hyperref[TEI.support]{support} \hyperref[TEI.surrogates]{surrogates} \hyperref[TEI.typeNote]{typeNote} \hyperref[TEI.watermark]{watermark}\par 
    \item[namesdates: ]
   \hyperref[TEI.addName]{addName} \hyperref[TEI.affiliation]{affiliation} \hyperref[TEI.country]{country} \hyperref[TEI.forename]{forename} \hyperref[TEI.genName]{genName} \hyperref[TEI.geogName]{geogName} \hyperref[TEI.nameLink]{nameLink} \hyperref[TEI.orgName]{orgName} \hyperref[TEI.persName]{persName} \hyperref[TEI.placeName]{placeName} \hyperref[TEI.region]{region} \hyperref[TEI.roleName]{roleName} \hyperref[TEI.settlement]{settlement} \hyperref[TEI.surname]{surname}\par 
    \item[textstructure: ]
   \hyperref[TEI.docAuthor]{docAuthor} \hyperref[TEI.docDate]{docDate} \hyperref[TEI.docEdition]{docEdition} \hyperref[TEI.titlePart]{titlePart}\par 
    \item[transcr: ]
   \hyperref[TEI.damage]{damage} \hyperref[TEI.fw]{fw} \hyperref[TEI.metamark]{metamark} \hyperref[TEI.mod]{mod} \hyperref[TEI.restore]{restore} \hyperref[TEI.retrace]{retrace} \hyperref[TEI.secl]{secl} \hyperref[TEI.supplied]{supplied} \hyperref[TEI.surplus]{surplus}
    \item[{Peut contenir}]
  
    \item[analysis: ]
   \hyperref[TEI.c]{c} \hyperref[TEI.cl]{cl} \hyperref[TEI.interp]{interp} \hyperref[TEI.interpGrp]{interpGrp} \hyperref[TEI.m]{m} \hyperref[TEI.pc]{pc} \hyperref[TEI.phr]{phr} \hyperref[TEI.s]{s} \hyperref[TEI.span]{span} \hyperref[TEI.spanGrp]{spanGrp} \hyperref[TEI.w]{w}\par 
    \item[core: ]
   \hyperref[TEI.abbr]{abbr} \hyperref[TEI.add]{add} \hyperref[TEI.address]{address} \hyperref[TEI.binaryObject]{binaryObject} \hyperref[TEI.cb]{cb} \hyperref[TEI.choice]{choice} \hyperref[TEI.corr]{corr} \hyperref[TEI.date]{date} \hyperref[TEI.del]{del} \hyperref[TEI.distinct]{distinct} \hyperref[TEI.email]{email} \hyperref[TEI.emph]{emph} \hyperref[TEI.expan]{expan} \hyperref[TEI.foreign]{foreign} \hyperref[TEI.gap]{gap} \hyperref[TEI.gb]{gb} \hyperref[TEI.gloss]{gloss} \hyperref[TEI.graphic]{graphic} \hyperref[TEI.hi]{hi} \hyperref[TEI.index]{index} \hyperref[TEI.lb]{lb} \hyperref[TEI.measure]{measure} \hyperref[TEI.measureGrp]{measureGrp} \hyperref[TEI.media]{media} \hyperref[TEI.mentioned]{mentioned} \hyperref[TEI.milestone]{milestone} \hyperref[TEI.name]{name} \hyperref[TEI.note]{note} \hyperref[TEI.num]{num} \hyperref[TEI.orig]{orig} \hyperref[TEI.pb]{pb} \hyperref[TEI.ptr]{ptr} \hyperref[TEI.ref]{ref} \hyperref[TEI.reg]{reg} \hyperref[TEI.rs]{rs} \hyperref[TEI.sic]{sic} \hyperref[TEI.soCalled]{soCalled} \hyperref[TEI.term]{term} \hyperref[TEI.time]{time} \hyperref[TEI.title]{title} \hyperref[TEI.unclear]{unclear}\par 
    \item[derived-module-tei.istex: ]
   \hyperref[TEI.math]{math} \hyperref[TEI.mrow]{mrow}\par 
    \item[figures: ]
   \hyperref[TEI.figure]{figure} \hyperref[TEI.formula]{formula} \hyperref[TEI.notatedMusic]{notatedMusic}\par 
    \item[header: ]
   \hyperref[TEI.idno]{idno}\par 
    \item[iso-fs: ]
   \hyperref[TEI.fLib]{fLib} \hyperref[TEI.fs]{fs} \hyperref[TEI.fvLib]{fvLib}\par 
    \item[linking: ]
   \hyperref[TEI.alt]{alt} \hyperref[TEI.altGrp]{altGrp} \hyperref[TEI.anchor]{anchor} \hyperref[TEI.join]{join} \hyperref[TEI.joinGrp]{joinGrp} \hyperref[TEI.link]{link} \hyperref[TEI.linkGrp]{linkGrp} \hyperref[TEI.seg]{seg} \hyperref[TEI.timeline]{timeline}\par 
    \item[msdescription: ]
   \hyperref[TEI.catchwords]{catchwords} \hyperref[TEI.depth]{depth} \hyperref[TEI.dim]{dim} \hyperref[TEI.dimensions]{dimensions} \hyperref[TEI.height]{height} \hyperref[TEI.heraldry]{heraldry} \hyperref[TEI.locus]{locus} \hyperref[TEI.locusGrp]{locusGrp} \hyperref[TEI.material]{material} \hyperref[TEI.objectType]{objectType} \hyperref[TEI.origDate]{origDate} \hyperref[TEI.origPlace]{origPlace} \hyperref[TEI.secFol]{secFol} \hyperref[TEI.signatures]{signatures} \hyperref[TEI.source]{source} \hyperref[TEI.stamp]{stamp} \hyperref[TEI.watermark]{watermark} \hyperref[TEI.width]{width}\par 
    \item[namesdates: ]
   \hyperref[TEI.addName]{addName} \hyperref[TEI.affiliation]{affiliation} \hyperref[TEI.country]{country} \hyperref[TEI.forename]{forename} \hyperref[TEI.genName]{genName} \hyperref[TEI.geogName]{geogName} \hyperref[TEI.location]{location} \hyperref[TEI.nameLink]{nameLink} \hyperref[TEI.orgName]{orgName} \hyperref[TEI.persName]{persName} \hyperref[TEI.placeName]{placeName} \hyperref[TEI.region]{region} \hyperref[TEI.roleName]{roleName} \hyperref[TEI.settlement]{settlement} \hyperref[TEI.state]{state} \hyperref[TEI.surname]{surname}\par 
    \item[spoken: ]
   \hyperref[TEI.annotationBlock]{annotationBlock}\par 
    \item[transcr: ]
   \hyperref[TEI.addSpan]{addSpan} \hyperref[TEI.am]{am} \hyperref[TEI.damage]{damage} \hyperref[TEI.damageSpan]{damageSpan} \hyperref[TEI.delSpan]{delSpan} \hyperref[TEI.ex]{ex} \hyperref[TEI.fw]{fw} \hyperref[TEI.handShift]{handShift} \hyperref[TEI.listTranspose]{listTranspose} \hyperref[TEI.metamark]{metamark} \hyperref[TEI.mod]{mod} \hyperref[TEI.redo]{redo} \hyperref[TEI.restore]{restore} \hyperref[TEI.retrace]{retrace} \hyperref[TEI.secl]{secl} \hyperref[TEI.space]{space} \hyperref[TEI.subst]{subst} \hyperref[TEI.substJoin]{substJoin} \hyperref[TEI.supplied]{supplied} \hyperref[TEI.surplus]{surplus} \hyperref[TEI.undo]{undo}\par des données textuelles
    \item[{Note}]
  \par
When this element appears within an \hyperref[TEI.index]{<index>} element, it is understood to supply the form under which an index entry is to be made for that location. Elsewhere, it is understood simply to indicate that its content is to be regarded as a technical or specialised term. It may be associated with a \hyperref[TEI.gloss]{<gloss>} element by means of its {\itshape ref} attribute; alternatively a \hyperref[TEI.gloss]{<gloss>} element may point to a \hyperref[TEI.term]{<term>} element by means of its {\itshape target} attribute.\par
In formal terminological work, there is frequently discussion over whether terms must be atomic or may include multi-word lexical items, symbolic designations, or phraseological units. The \hyperref[TEI.term]{<term>} element may be used to mark any of these. No position is taken on the philosophical issue of what a term can be; the looser definition simply allows the \hyperref[TEI.term]{<term>} element to be used by practitioners of any persuasion.\par
As with other members of the \textsf{att.canonical} class, instances of this element occuring in a text may be associated with a canonical definition, either by means of a URI (using the {\itshape ref} attribute), or by means of some system-specific code value (using the {\itshape key} attribute). Because the mutually exclusive {\itshape target} and {\itshape cRef} attributes overlap with the function of the {\itshape ref} attribute, they are deprecated and may be removed at a subsequent release.
    \item[{Exemple}]
  \leavevmode\bgroup\exampleFont \begin{shaded}\noindent\mbox{}{<\textbf{p}>}SGANARELLE.{</\textbf{p}>}\mbox{}\newline 
{<\textbf{p}>}Qui est causée par l'âcreté des {<\textbf{term}>}humeurs{</\textbf{term}>} engendrées dans la concavité du\mbox{}\newline 
{<\textbf{term}>}diaphragme{</\textbf{term}>}, il arrive que ces {<\textbf{term}>}vapeurs{</\textbf{term}>}... Ossabandus, nequeys,\mbox{}\newline 
 nequer, potarinum, quipsa milus. Voilà justement ce qui fait que votre fille est muette.{</\textbf{p}>}\end{shaded}\egroup 


    \item[{Exemple}]
  \leavevmode\bgroup\exampleFont \begin{shaded}\noindent\mbox{} D'après la\mbox{}\newline 
 théorie d'Austin, les{<\textbf{term}>} verbes performatifs{</\textbf{term}>} seraient ceux qui non seulement\mbox{}\newline 
 décrivent l'action de celui qui les utilise, mais aussi, et en même temps, qui impliqueraient\mbox{}\newline 
 cette action elle-même. \end{shaded}\egroup 


    \item[{Modèle de contenu}]
  \mbox{}\hfill\\[-10pt]\begin{Verbatim}[fontsize=\small]
<content>
 <macroRef key="macro.phraseSeq"/>
</content>
    
\end{Verbatim}

    \item[{Schéma Declaration}]
  \mbox{}\hfill\\[-10pt]\begin{Verbatim}[fontsize=\small]
element term
{
   tei_att.global.attributes,
   tei_att.declaring.attributes,
   tei_att.pointing.attributes,
   tei_att.typed.attributes,
   tei_att.canonical.attributes,
   tei_att.sortable.attributes,
   tei_att.cReferencing.attributes,
   attribute level { text }?,
   tei_macro.phraseSeq}
\end{Verbatim}

\end{reflist}  \index{text=<text>|oddindex}
\begin{reflist}
\item[]\begin{specHead}{TEI.text}{<text> }(texte) contient un seul texte quelconque, simple ou composite, par exemple un poème ou une pièce de théâtre, un recueil d’essais, un roman, un dictionnaire ou un échantillon de corpus. [\xref{http://www.tei-c.org/release/doc/tei-p5-doc/en/html/DS.html\#DS}{4. Default Text Structure} \xref{http://www.tei-c.org/release/doc/tei-p5-doc/en/html/CC.html\#CCDEF}{15.1. Varieties of Composite Text}]\end{specHead} 
    \item[{Module}]
  textstructure
    \item[{Attributs}]
  Attributs \hyperref[TEI.att.global]{att.global} (\textit{@xml:id}, \textit{@n}, \textit{@xml:lang}, \textit{@xml:base}, \textit{@xml:space})  (\hyperref[TEI.att.global.rendition]{att.global.rendition} (\textit{@rend}, \textit{@style}, \textit{@rendition})) (\hyperref[TEI.att.global.linking]{att.global.linking} (\textit{@corresp}, \textit{@synch}, \textit{@sameAs}, \textit{@copyOf}, \textit{@next}, \textit{@prev}, \textit{@exclude}, \textit{@select})) (\hyperref[TEI.att.global.analytic]{att.global.analytic} (\textit{@ana})) (\hyperref[TEI.att.global.facs]{att.global.facs} (\textit{@facs})) (\hyperref[TEI.att.global.change]{att.global.change} (\textit{@change})) (\hyperref[TEI.att.global.responsibility]{att.global.responsibility} (\textit{@cert}, \textit{@resp})) (\hyperref[TEI.att.global.source]{att.global.source} (\textit{@source})) \hyperref[TEI.att.declaring]{att.declaring} (\textit{@decls}) \hyperref[TEI.att.typed]{att.typed} (\textit{@type}, \textit{@subtype}) \hyperref[TEI.att.written]{att.written} (\textit{@hand}) 
    \item[{Membre du}]
  \hyperref[TEI.model.annotation]{model.annotation} \hyperref[TEI.model.resourceLike]{model.resourceLike}
    \item[{Contenu dans}]
  
    \item[core: ]
   \hyperref[TEI.teiCorpus]{teiCorpus}\par 
    \item[spoken: ]
   \hyperref[TEI.annotationBlock]{annotationBlock}\par 
    \item[standOff: ]
   \hyperref[TEI.listAnnotation]{listAnnotation} \hyperref[TEI.standOff]{standOff}\par 
    \item[textstructure: ]
   \hyperref[TEI.TEI]{TEI} \hyperref[TEI.group]{group}
    \item[{Peut contenir}]
  
    \item[analysis: ]
   \hyperref[TEI.interp]{interp} \hyperref[TEI.interpGrp]{interpGrp} \hyperref[TEI.span]{span} \hyperref[TEI.spanGrp]{spanGrp}\par 
    \item[core: ]
   \hyperref[TEI.cb]{cb} \hyperref[TEI.gap]{gap} \hyperref[TEI.gb]{gb} \hyperref[TEI.index]{index} \hyperref[TEI.lb]{lb} \hyperref[TEI.milestone]{milestone} \hyperref[TEI.note]{note} \hyperref[TEI.pb]{pb}\par 
    \item[figures: ]
   \hyperref[TEI.figure]{figure} \hyperref[TEI.notatedMusic]{notatedMusic}\par 
    \item[iso-fs: ]
   \hyperref[TEI.fLib]{fLib} \hyperref[TEI.fs]{fs} \hyperref[TEI.fvLib]{fvLib}\par 
    \item[linking: ]
   \hyperref[TEI.alt]{alt} \hyperref[TEI.altGrp]{altGrp} \hyperref[TEI.anchor]{anchor} \hyperref[TEI.join]{join} \hyperref[TEI.joinGrp]{joinGrp} \hyperref[TEI.link]{link} \hyperref[TEI.linkGrp]{linkGrp} \hyperref[TEI.timeline]{timeline}\par 
    \item[msdescription: ]
   \hyperref[TEI.source]{source}\par 
    \item[textstructure: ]
   \hyperref[TEI.back]{back} \hyperref[TEI.body]{body} \hyperref[TEI.front]{front} \hyperref[TEI.group]{group}\par 
    \item[transcr: ]
   \hyperref[TEI.addSpan]{addSpan} \hyperref[TEI.damageSpan]{damageSpan} \hyperref[TEI.delSpan]{delSpan} \hyperref[TEI.fw]{fw} \hyperref[TEI.listTranspose]{listTranspose} \hyperref[TEI.metamark]{metamark} \hyperref[TEI.space]{space} \hyperref[TEI.substJoin]{substJoin}
    \item[{Note}]
  \par
Cet élément ne devrait pas être utilisé pour encoder un texte inséré à un endroit non prévisible à l'intérieur de la structure d'un autre texte, comme par exemple dans un récit qui est enchâssé ou cité dans un autre ; c'est l'élément \hyperref[TEI.floatingText]{<floatingText>} qui doit être utilisé à cet effet.
    \item[{Exemple}]
  l'élément \hyperref[TEI.annotationBlock]{<annotationBlock>}contient la réocérisation du texte extrait du pdf dans \hyperref[TEI.text]{<text>}\leavevmode\bgroup\exampleFont \begin{shaded}\noindent\mbox{}{<\textbf{listAnnotation}\hspace*{6pt}{type}="{subject}">}\mbox{}\newline 
\hspace*{6pt}{<\textbf{annotationBlock}\hspace*{6pt}{corresp}="{\#subject-01}">}\mbox{}\newline 
\hspace*{6pt}\hspace*{6pt}{<\textbf{text}>}\mbox{}\newline 
\hspace*{6pt}\hspace*{6pt}\hspace*{6pt}{<\textbf{body}\hspace*{6pt}{resp}="{\#ISTEX}"\hspace*{6pt}{source}="{\#ocr-001}">}\mbox{}\newline 
\hspace*{6pt}\hspace*{6pt}\hspace*{6pt}\hspace*{6pt}{<\textbf{div}>}\mbox{}\newline 
\hspace*{6pt}\hspace*{6pt}\hspace*{6pt}\hspace*{6pt}\hspace*{6pt}{<\textbf{p}>}Action Algebras and Model Algebras in\mbox{}\newline 
\hspace*{6pt}\hspace*{6pt}\hspace*{6pt}\hspace*{6pt}\hspace*{6pt}\hspace*{6pt}\hspace*{6pt}\hspace*{6pt}\hspace*{6pt}\hspace*{6pt} Denotational Semantics Luiz Carlos Castro Guedes1\mbox{}\newline 
\hspace*{6pt}\hspace*{6pt}\hspace*{6pt}\hspace*{6pt}\hspace*{6pt}\hspace*{6pt}\hspace*{6pt}\hspace*{6pt}\hspace*{6pt}\hspace*{6pt} and Edward Hermann Haeusler2 1 2 1 Instituto de\mbox{}\newline 
\hspace*{6pt}\hspace*{6pt}\hspace*{6pt}\hspace*{6pt}\hspace*{6pt}\hspace*{6pt}\hspace*{6pt}\hspace*{6pt}\hspace*{6pt}\hspace*{6pt} Computa¸ao, UFF, Niteroi, Brasil c˜ Departamento\mbox{}\newline 
\hspace*{6pt}\hspace*{6pt}\hspace*{6pt}\hspace*{6pt}\hspace*{6pt}\hspace*{6pt}\hspace*{6pt}\hspace*{6pt}\hspace*{6pt}\hspace*{6pt} de Inform´tica, PUC-Rio, Rio de Janeiro, Brasil a\mbox{}\newline 
\hspace*{6pt}\hspace*{6pt}\hspace*{6pt}\hspace*{6pt}\hspace*{6pt}\hspace*{6pt}\hspace*{6pt}\hspace*{6pt}\hspace*{6pt}\hspace*{6pt} Introduction This article describes some results\mbox{}\newline 
\hspace*{6pt}\hspace*{6pt}\hspace*{6pt}\hspace*{6pt}\hspace*{6pt}\hspace*{6pt}\hspace*{6pt}\hspace*{6pt}\hspace*{6pt}\hspace*{6pt} concerning the conceptual separation of model\mbox{}\newline 
\hspace*{6pt}\hspace*{6pt}\hspace*{6pt}\hspace*{6pt}\hspace*{6pt}\hspace*{6pt}\hspace*{6pt}\hspace*{6pt}\hspace*{6pt}\hspace*{6pt} dependent and language inherent aspects in a\mbox{}\newline 
\hspace*{6pt}\hspace*{6pt}\hspace*{6pt}\hspace*{6pt}\hspace*{6pt}\hspace*{6pt}\hspace*{6pt}\hspace*{6pt}\hspace*{6pt}\hspace*{6pt} denotational semantics of a programming language.\mbox{}\newline 
\hspace*{6pt}\hspace*{6pt}\hspace*{6pt}\hspace*{6pt}\hspace*{6pt}\hspace*{6pt}\hspace*{6pt}\hspace*{6pt}\hspace*{6pt}\hspace*{6pt} Before going into the technical explanation, the\mbox{}\newline 
\hspace*{6pt}\hspace*{6pt}\hspace*{6pt}\hspace*{6pt}\hspace*{6pt}\hspace*{6pt}\hspace*{6pt}\hspace*{6pt}\hspace*{6pt}\hspace*{6pt} authors wish to relate a story that illustrates\mbox{}\newline 
\hspace*{6pt}\hspace*{6pt}\hspace*{6pt}\hspace*{6pt}\hspace*{6pt}\hspace*{6pt}\hspace*{6pt}\hspace*{6pt}\hspace*{6pt}\hspace*{6pt} how correctly and precisely posed questions can\mbox{}\newline 
\hspace*{6pt}\hspace*{6pt}\hspace*{6pt}\hspace*{6pt}\hspace*{6pt}\hspace*{6pt}\hspace*{6pt}\hspace*{6pt}\hspace*{6pt}\hspace*{6pt} influence the direction of research. By means of\mbox{}\newline 
\hspace*{6pt}\hspace*{6pt}\hspace*{6pt}\hspace*{6pt}\hspace*{6pt}\hspace*{6pt}\hspace*{6pt}\hspace*{6pt}\hspace*{6pt}\hspace*{6pt} his questions, Professor Mosses aided the PhD\mbox{}\newline 
\hspace*{6pt}\hspace*{6pt}\hspace*{6pt}\hspace*{6pt}\hspace*{6pt}\hspace*{6pt}\hspace*{6pt}\hspace*{6pt}\hspace*{6pt}\hspace*{6pt} research of one of the authors of this article and\mbox{}\newline 
\hspace*{6pt}\hspace*{6pt}\hspace*{6pt}\hspace*{6pt}\hspace*{6pt}\hspace*{6pt}\hspace*{6pt}\hspace*{6pt}\hspace*{6pt}\hspace*{6pt} taught the other, who at the time was a novice\mbox{}\newline 
\hspace*{6pt}\hspace*{6pt}\hspace*{6pt}\hspace*{6pt}\hspace*{6pt}\hspace*{6pt}\hspace*{6pt}\hspace*{6pt}\hspace*{6pt}\hspace*{6pt} supervisor, the real meaning of careful PhD\mbox{}\newline 
\hspace*{6pt}\hspace*{6pt}\hspace*{6pt}\hspace*{6pt}\hspace*{6pt}\hspace*{6pt}\hspace*{6pt}\hspace*{6pt}\hspace*{6pt}\hspace*{6pt} supervision. The student’s research had been\mbox{}\newline 
\hspace*{6pt}\hspace*{6pt}\hspace*{6pt}\hspace*{6pt}\hspace*{6pt}\hspace*{6pt}\hspace*{6pt}\hspace*{6pt}\hspace*{6pt}\hspace*{6pt} partially developed towards the implementation of\mbox{}\newline 
\hspace*{6pt}\hspace*{6pt}\hspace*{6pt}\hspace*{6pt}\hspace*{6pt}\hspace*{6pt}\hspace*{6pt}\hspace*{6pt}\hspace*{6pt}\hspace*{6pt} programming languages through denotational\mbox{}\newline 
\hspace*{6pt}\hspace*{6pt}\hspace*{6pt}\hspace*{6pt}\hspace*{6pt}\hspace*{6pt}\hspace*{6pt}\hspace*{6pt}\hspace*{6pt}\hspace*{6pt} semantics specification, and the student had\mbox{}\newline 
\hspace*{6pt}\hspace*{6pt}\hspace*{6pt}\hspace*{6pt}\hspace*{6pt}\hspace*{6pt}\hspace*{6pt}\hspace*{6pt}\hspace*{6pt}\hspace*{6pt} developed a prototype{</\textbf{p}>}\mbox{}\newline 
\hspace*{6pt}\hspace*{6pt}\hspace*{6pt}\hspace*{6pt}{</\textbf{div}>}\mbox{}\newline 
\hspace*{6pt}\hspace*{6pt}\hspace*{6pt}{</\textbf{body}>}\mbox{}\newline 
\hspace*{6pt}\hspace*{6pt}{</\textbf{text}>}\mbox{}\newline 
\hspace*{6pt}{</\textbf{annotationBlock}>}\mbox{}\newline 
{</\textbf{listAnnotation}>}\end{shaded}\egroup 


    \item[{Modèle de contenu}]
  \mbox{}\hfill\\[-10pt]\begin{Verbatim}[fontsize=\small]
<content>
 <sequence maxOccurs="1" minOccurs="1">
  <classRef key="model.global"
   maxOccurs="unbounded" minOccurs="0"/>
  <sequence maxOccurs="1" minOccurs="0">
   <elementRef key="front"/>
   <classRef key="model.global"
    maxOccurs="unbounded" minOccurs="0"/>
  </sequence>
  <alternate maxOccurs="1" minOccurs="1">
   <elementRef key="body"/>
   <elementRef key="group"/>
  </alternate>
  <classRef key="model.global"
   maxOccurs="unbounded" minOccurs="0"/>
  <sequence maxOccurs="1" minOccurs="0">
   <elementRef key="back"/>
   <classRef key="model.global"
    maxOccurs="unbounded" minOccurs="0"/>
  </sequence>
 </sequence>
</content>
    
\end{Verbatim}

    \item[{Schéma Declaration}]
  \mbox{}\hfill\\[-10pt]\begin{Verbatim}[fontsize=\small]
element text
{
   tei_att.global.attributes,
   tei_att.declaring.attributes,
   tei_att.typed.attributes,
   tei_att.written.attributes,
   (
      tei_model.global*,
      ( tei_front, tei_model.global* )?,
      ( tei_body | tei_group ),
      tei_model.global*,
      ( tei_back, tei_model.global* )?
   )
}
\end{Verbatim}

\end{reflist}  \index{textClass=<textClass>|oddindex}
\begin{reflist}
\item[]\begin{specHead}{TEI.textClass}{<textClass> }(classification du texte) regroupe des informations décrivant la nature ou le sujet d’un texte selon des termes issus d’un système de classification standardisé, d’un thésaurus, etc. [\xref{http://www.tei-c.org/release/doc/tei-p5-doc/en/html/HD.html\#HD43}{2.4.3. The Text Classification}]\end{specHead} 
    \item[{Module}]
  header
    \item[{Attributs}]
  Attributs \hyperref[TEI.att.global]{att.global} (\textit{@xml:id}, \textit{@n}, \textit{@xml:lang}, \textit{@xml:base}, \textit{@xml:space})  (\hyperref[TEI.att.global.rendition]{att.global.rendition} (\textit{@rend}, \textit{@style}, \textit{@rendition})) (\hyperref[TEI.att.global.linking]{att.global.linking} (\textit{@corresp}, \textit{@synch}, \textit{@sameAs}, \textit{@copyOf}, \textit{@next}, \textit{@prev}, \textit{@exclude}, \textit{@select})) (\hyperref[TEI.att.global.analytic]{att.global.analytic} (\textit{@ana})) (\hyperref[TEI.att.global.facs]{att.global.facs} (\textit{@facs})) (\hyperref[TEI.att.global.change]{att.global.change} (\textit{@change})) (\hyperref[TEI.att.global.responsibility]{att.global.responsibility} (\textit{@cert}, \textit{@resp})) (\hyperref[TEI.att.global.source]{att.global.source} (\textit{@source})) \hyperref[TEI.att.declarable]{att.declarable} (\textit{@default}) 
    \item[{Membre du}]
  \hyperref[TEI.model.profileDescPart]{model.profileDescPart}
    \item[{Contenu dans}]
  
    \item[header: ]
   \hyperref[TEI.profileDesc]{profileDesc}
    \item[{Peut contenir}]
  
    \item[header: ]
   \hyperref[TEI.classCode]{classCode} \hyperref[TEI.keywords]{keywords}
    \item[{Exemple}]
  \leavevmode\bgroup\exampleFont \begin{shaded}\noindent\mbox{}{<\textbf{textClass}>}\mbox{}\newline 
\hspace*{6pt}{<\textbf{keywords}\hspace*{6pt}{scheme}="{\#fr\textunderscore RAMEAU}">}\mbox{}\newline 
\hspace*{6pt}\hspace*{6pt}{<\textbf{list}>}\mbox{}\newline 
\hspace*{6pt}\hspace*{6pt}\hspace*{6pt}{<\textbf{item}>}Littérature française -- 20ème siècle -- Histoire et critique{</\textbf{item}>}\mbox{}\newline 
\hspace*{6pt}\hspace*{6pt}\hspace*{6pt}{<\textbf{item}>}Littérature française -- Histoire et critique -- Théorie, etc.{</\textbf{item}>}\mbox{}\newline 
\hspace*{6pt}\hspace*{6pt}\hspace*{6pt}{<\textbf{item}>}Français (langue) -- Style -- Bases de données.{</\textbf{item}>}\mbox{}\newline 
\hspace*{6pt}\hspace*{6pt}{</\textbf{list}>}\mbox{}\newline 
\hspace*{6pt}{</\textbf{keywords}>}\mbox{}\newline 
{</\textbf{textClass}>}\end{shaded}\egroup 


    \item[{Exemple}]
  \leavevmode\bgroup\exampleFont \begin{shaded}\noindent\mbox{}{<\textbf{textClass}>}\mbox{}\newline 
\hspace*{6pt}{<\textbf{catRef}\hspace*{6pt}{target}="{\#fr\textunderscore forme\textunderscore prose}"/>}\mbox{}\newline 
{</\textbf{textClass}>}\end{shaded}\egroup 


    \item[{Modèle de contenu}]
  \mbox{}\hfill\\[-10pt]\begin{Verbatim}[fontsize=\small]
<content>
 <alternate maxOccurs="unbounded"
  minOccurs="0">
  <elementRef key="classCode"/>
  <elementRef key="catRef"/>
  <elementRef key="keywords"/>
 </alternate>
</content>
    
\end{Verbatim}

    \item[{Schéma Declaration}]
  \mbox{}\hfill\\[-10pt]\begin{Verbatim}[fontsize=\small]
element textClass
{
   tei_att.global.attributes,
   tei_att.declarable.attributes,
   ( tei_classCode | catRef | tei_keywords )*
}
\end{Verbatim}

\end{reflist}  \index{textLang=<textLang>|oddindex}\index{mainLang=@mainLang!<textLang>|oddindex}\index{otherLangs=@otherLangs!<textLang>|oddindex}
\begin{reflist}
\item[]\begin{specHead}{TEI.textLang}{<textLang> }(langues du texte) décrit les langues et systèmes d'écriture utilisés dans un manuscrit (et non dans la description du manuscrit, dont les langues et systèmes d'écriture sont décrits dans l'élément \hyperref[TEI.langUsage]{<langUsage>}). [\xref{http://www.tei-c.org/release/doc/tei-p5-doc/en/html/CO.html\#COBICOI}{3.11.2.4. Imprint, Size of a Document, and Reprint Information} \xref{http://www.tei-c.org/release/doc/tei-p5-doc/en/html/MS.html\#mslangs}{10.6.6. Languages and Writing Systems}]\end{specHead} 
    \item[{Module}]
  core
    \item[{Attributs}]
  Attributs \hyperref[TEI.att.global]{att.global} (\textit{@xml:id}, \textit{@n}, \textit{@xml:lang}, \textit{@xml:base}, \textit{@xml:space})  (\hyperref[TEI.att.global.rendition]{att.global.rendition} (\textit{@rend}, \textit{@style}, \textit{@rendition})) (\hyperref[TEI.att.global.linking]{att.global.linking} (\textit{@corresp}, \textit{@synch}, \textit{@sameAs}, \textit{@copyOf}, \textit{@next}, \textit{@prev}, \textit{@exclude}, \textit{@select})) (\hyperref[TEI.att.global.analytic]{att.global.analytic} (\textit{@ana})) (\hyperref[TEI.att.global.facs]{att.global.facs} (\textit{@facs})) (\hyperref[TEI.att.global.change]{att.global.change} (\textit{@change})) (\hyperref[TEI.att.global.responsibility]{att.global.responsibility} (\textit{@cert}, \textit{@resp})) (\hyperref[TEI.att.global.source]{att.global.source} (\textit{@source})) \hfil\\[-10pt]\begin{sansreflist}
    \item[@mainLang]
  (langue principale) contient un code identifiant la langue principale du manuscrit.
\begin{reflist}
    \item[{Statut}]
  Optionel
    \item[{Type de données}]
  \hyperref[TEI.teidata.language]{teidata.language}
\end{reflist}  
    \item[@otherLangs]
  (autres langues) contient un ou plusieurs codes identifiant toute autre langue utilisée dans le manuscrit.
\begin{reflist}
    \item[{Statut}]
  Optionel
    \item[{Type de données}]
  0–∞ occurrences de \hyperref[TEI.teidata.language]{teidata.language} séparé par un espace
\end{reflist}  
\end{sansreflist}  
    \item[{Membre du}]
  \hyperref[TEI.model.biblPart]{model.biblPart} \hyperref[TEI.model.msItemPart]{model.msItemPart} 
    \item[{Contenu dans}]
  
    \item[core: ]
   \hyperref[TEI.analytic]{analytic} \hyperref[TEI.bibl]{bibl} \hyperref[TEI.monogr]{monogr} \hyperref[TEI.series]{series}\par 
    \item[msdescription: ]
   \hyperref[TEI.msContents]{msContents} \hyperref[TEI.msItem]{msItem} \hyperref[TEI.msItemStruct]{msItemStruct}
    \item[{Peut contenir}]
  
    \item[analysis: ]
   \hyperref[TEI.c]{c} \hyperref[TEI.cl]{cl} \hyperref[TEI.interp]{interp} \hyperref[TEI.interpGrp]{interpGrp} \hyperref[TEI.m]{m} \hyperref[TEI.pc]{pc} \hyperref[TEI.phr]{phr} \hyperref[TEI.s]{s} \hyperref[TEI.span]{span} \hyperref[TEI.spanGrp]{spanGrp} \hyperref[TEI.w]{w}\par 
    \item[core: ]
   \hyperref[TEI.abbr]{abbr} \hyperref[TEI.add]{add} \hyperref[TEI.address]{address} \hyperref[TEI.binaryObject]{binaryObject} \hyperref[TEI.cb]{cb} \hyperref[TEI.choice]{choice} \hyperref[TEI.corr]{corr} \hyperref[TEI.date]{date} \hyperref[TEI.del]{del} \hyperref[TEI.distinct]{distinct} \hyperref[TEI.email]{email} \hyperref[TEI.emph]{emph} \hyperref[TEI.expan]{expan} \hyperref[TEI.foreign]{foreign} \hyperref[TEI.gap]{gap} \hyperref[TEI.gb]{gb} \hyperref[TEI.gloss]{gloss} \hyperref[TEI.graphic]{graphic} \hyperref[TEI.hi]{hi} \hyperref[TEI.index]{index} \hyperref[TEI.lb]{lb} \hyperref[TEI.measure]{measure} \hyperref[TEI.measureGrp]{measureGrp} \hyperref[TEI.media]{media} \hyperref[TEI.mentioned]{mentioned} \hyperref[TEI.milestone]{milestone} \hyperref[TEI.name]{name} \hyperref[TEI.note]{note} \hyperref[TEI.num]{num} \hyperref[TEI.orig]{orig} \hyperref[TEI.pb]{pb} \hyperref[TEI.ptr]{ptr} \hyperref[TEI.ref]{ref} \hyperref[TEI.reg]{reg} \hyperref[TEI.rs]{rs} \hyperref[TEI.sic]{sic} \hyperref[TEI.soCalled]{soCalled} \hyperref[TEI.term]{term} \hyperref[TEI.time]{time} \hyperref[TEI.title]{title} \hyperref[TEI.unclear]{unclear}\par 
    \item[derived-module-tei.istex: ]
   \hyperref[TEI.math]{math} \hyperref[TEI.mrow]{mrow}\par 
    \item[figures: ]
   \hyperref[TEI.figure]{figure} \hyperref[TEI.formula]{formula} \hyperref[TEI.notatedMusic]{notatedMusic}\par 
    \item[header: ]
   \hyperref[TEI.idno]{idno}\par 
    \item[iso-fs: ]
   \hyperref[TEI.fLib]{fLib} \hyperref[TEI.fs]{fs} \hyperref[TEI.fvLib]{fvLib}\par 
    \item[linking: ]
   \hyperref[TEI.alt]{alt} \hyperref[TEI.altGrp]{altGrp} \hyperref[TEI.anchor]{anchor} \hyperref[TEI.join]{join} \hyperref[TEI.joinGrp]{joinGrp} \hyperref[TEI.link]{link} \hyperref[TEI.linkGrp]{linkGrp} \hyperref[TEI.seg]{seg} \hyperref[TEI.timeline]{timeline}\par 
    \item[msdescription: ]
   \hyperref[TEI.catchwords]{catchwords} \hyperref[TEI.depth]{depth} \hyperref[TEI.dim]{dim} \hyperref[TEI.dimensions]{dimensions} \hyperref[TEI.height]{height} \hyperref[TEI.heraldry]{heraldry} \hyperref[TEI.locus]{locus} \hyperref[TEI.locusGrp]{locusGrp} \hyperref[TEI.material]{material} \hyperref[TEI.objectType]{objectType} \hyperref[TEI.origDate]{origDate} \hyperref[TEI.origPlace]{origPlace} \hyperref[TEI.secFol]{secFol} \hyperref[TEI.signatures]{signatures} \hyperref[TEI.source]{source} \hyperref[TEI.stamp]{stamp} \hyperref[TEI.watermark]{watermark} \hyperref[TEI.width]{width}\par 
    \item[namesdates: ]
   \hyperref[TEI.addName]{addName} \hyperref[TEI.affiliation]{affiliation} \hyperref[TEI.country]{country} \hyperref[TEI.forename]{forename} \hyperref[TEI.genName]{genName} \hyperref[TEI.geogName]{geogName} \hyperref[TEI.location]{location} \hyperref[TEI.nameLink]{nameLink} \hyperref[TEI.orgName]{orgName} \hyperref[TEI.persName]{persName} \hyperref[TEI.placeName]{placeName} \hyperref[TEI.region]{region} \hyperref[TEI.roleName]{roleName} \hyperref[TEI.settlement]{settlement} \hyperref[TEI.state]{state} \hyperref[TEI.surname]{surname}\par 
    \item[spoken: ]
   \hyperref[TEI.annotationBlock]{annotationBlock}\par 
    \item[transcr: ]
   \hyperref[TEI.addSpan]{addSpan} \hyperref[TEI.am]{am} \hyperref[TEI.damage]{damage} \hyperref[TEI.damageSpan]{damageSpan} \hyperref[TEI.delSpan]{delSpan} \hyperref[TEI.ex]{ex} \hyperref[TEI.fw]{fw} \hyperref[TEI.handShift]{handShift} \hyperref[TEI.listTranspose]{listTranspose} \hyperref[TEI.metamark]{metamark} \hyperref[TEI.mod]{mod} \hyperref[TEI.redo]{redo} \hyperref[TEI.restore]{restore} \hyperref[TEI.retrace]{retrace} \hyperref[TEI.secl]{secl} \hyperref[TEI.space]{space} \hyperref[TEI.subst]{subst} \hyperref[TEI.substJoin]{substJoin} \hyperref[TEI.supplied]{supplied} \hyperref[TEI.surplus]{surplus} \hyperref[TEI.undo]{undo}\par des données textuelles
    \item[{Note}]
  \par
This element should not be used to document the languages or writing systems used for the bibliographic or manuscript description itself: as for all other TEI elements, such information should be provided by means of the global {\itshape xml:lang} attribute attached to the element containing the description.\par
In all cases, languages should be identified by means of a standardized ‘language tag’ generated according to \xref{https://tools.ietf.org/html/bcp47}{BCP 47}. Additional documentation for the language may be provided by a \hyperref[TEI.language]{<language>} element in the TEI Header.
    \item[{Exemple}]
  \leavevmode\bgroup\exampleFont \begin{shaded}\noindent\mbox{}{<\textbf{textLang}\hspace*{6pt}{mainLang}="{en}"\hspace*{6pt}{otherLangs}="{la}">} En français essentiellement, avec des gloses en\mbox{}\newline 
 latin.{</\textbf{textLang}>}\end{shaded}\egroup 


    \item[{Modèle de contenu}]
  \mbox{}\hfill\\[-10pt]\begin{Verbatim}[fontsize=\small]
<content>
 <macroRef key="macro.phraseSeq"/>
</content>
    
\end{Verbatim}

    \item[{Schéma Declaration}]
  \mbox{}\hfill\\[-10pt]\begin{Verbatim}[fontsize=\small]
element textLang
{
   tei_att.global.attributes,
   attribute mainLang { text }?,
   attribute otherLangs { list { * } }?,
   tei_macro.phraseSeq}
\end{Verbatim}

\end{reflist}  \index{then=<then>|oddindex}
\begin{reflist}
\item[]\begin{specHead}{TEI.then}{<then> }sépare la condition de la valeur par défaut dans un if, ou l'antécédent de la conséquence dans un élément cond [\xref{http://www.tei-c.org/release/doc/tei-p5-doc/en/html/FS.html\#FD}{18.11. Feature System Declaration}]\end{specHead} 
    \item[{Module}]
  iso-fs
    \item[{Attributs}]
  Attributs \hyperref[TEI.att.global]{att.global} (\textit{@xml:id}, \textit{@n}, \textit{@xml:lang}, \textit{@xml:base}, \textit{@xml:space})  (\hyperref[TEI.att.global.rendition]{att.global.rendition} (\textit{@rend}, \textit{@style}, \textit{@rendition})) (\hyperref[TEI.att.global.linking]{att.global.linking} (\textit{@corresp}, \textit{@synch}, \textit{@sameAs}, \textit{@copyOf}, \textit{@next}, \textit{@prev}, \textit{@exclude}, \textit{@select})) (\hyperref[TEI.att.global.analytic]{att.global.analytic} (\textit{@ana})) (\hyperref[TEI.att.global.facs]{att.global.facs} (\textit{@facs})) (\hyperref[TEI.att.global.change]{att.global.change} (\textit{@change})) (\hyperref[TEI.att.global.responsibility]{att.global.responsibility} (\textit{@cert}, \textit{@resp})) (\hyperref[TEI.att.global.source]{att.global.source} (\textit{@source}))
    \item[{Contenu dans}]
  
    \item[iso-fs: ]
   \hyperref[TEI.cond]{cond} \hyperref[TEI.if]{if}
    \item[{Peut contenir}]
  Elément vide
    \item[{Note}]
  \par
Cet élément est fourni essentiellement pour rendre plus lisible par l'homme une déclaration d'un système de traits.
    \item[{Exemple}]
  \leavevmode\bgroup\exampleFont \begin{shaded}\noindent\mbox{}{<\textbf{cond}>}\mbox{}\newline 
\hspace*{6pt}{<\textbf{fs}>}\mbox{}\newline 
\hspace*{6pt}\hspace*{6pt}{<\textbf{f}\hspace*{6pt}{name}="{BAR}">}\mbox{}\newline 
\hspace*{6pt}\hspace*{6pt}\hspace*{6pt}{<\textbf{symbol}\hspace*{6pt}{value}="{1}"/>}\mbox{}\newline 
\hspace*{6pt}\hspace*{6pt}{</\textbf{f}>}\mbox{}\newline 
\hspace*{6pt}{</\textbf{fs}>}\mbox{}\newline 
\hspace*{6pt}{<\textbf{then}/>}\mbox{}\newline 
\hspace*{6pt}{<\textbf{fs}>}\mbox{}\newline 
\hspace*{6pt}\hspace*{6pt}{<\textbf{f}\hspace*{6pt}{name}="{FOO}">}\mbox{}\newline 
\hspace*{6pt}\hspace*{6pt}\hspace*{6pt}{<\textbf{binary}\hspace*{6pt}{value}="{false}"/>}\mbox{}\newline 
\hspace*{6pt}\hspace*{6pt}{</\textbf{f}>}\mbox{}\newline 
\hspace*{6pt}{</\textbf{fs}>}\mbox{}\newline 
{</\textbf{cond}>}\end{shaded}\egroup 


    \item[{Modèle de contenu}]
  \fbox{\ttfamily <content>\newline
</content>\newline
    } 
    \item[{Schéma Declaration}]
  \fbox{\ttfamily element then ❴ tei\textunderscore att.global.attributes, empty ❵} 
\end{reflist}  \index{time=<time>|oddindex}
\begin{reflist}
\item[]\begin{specHead}{TEI.time}{<time> }(temps) contient une expression qui précise un moment de la journée sous n'importe quelle forme. [\xref{http://www.tei-c.org/release/doc/tei-p5-doc/en/html/CO.html\#CONADA}{3.5.4. Dates and Times}]\end{specHead} 
    \item[{Module}]
  core
    \item[{Attributs}]
  Attributs \hyperref[TEI.att.global]{att.global} (\textit{@xml:id}, \textit{@n}, \textit{@xml:lang}, \textit{@xml:base}, \textit{@xml:space})  (\hyperref[TEI.att.global.rendition]{att.global.rendition} (\textit{@rend}, \textit{@style}, \textit{@rendition})) (\hyperref[TEI.att.global.linking]{att.global.linking} (\textit{@corresp}, \textit{@synch}, \textit{@sameAs}, \textit{@copyOf}, \textit{@next}, \textit{@prev}, \textit{@exclude}, \textit{@select})) (\hyperref[TEI.att.global.analytic]{att.global.analytic} (\textit{@ana})) (\hyperref[TEI.att.global.facs]{att.global.facs} (\textit{@facs})) (\hyperref[TEI.att.global.change]{att.global.change} (\textit{@change})) (\hyperref[TEI.att.global.responsibility]{att.global.responsibility} (\textit{@cert}, \textit{@resp})) (\hyperref[TEI.att.global.source]{att.global.source} (\textit{@source})) \hyperref[TEI.att.datable]{att.datable} (\textit{@calendar}, \textit{@period})  (\hyperref[TEI.att.datable.w3c]{att.datable.w3c} (\textit{@when}, \textit{@notBefore}, \textit{@notAfter}, \textit{@from}, \textit{@to})) (\hyperref[TEI.att.datable.iso]{att.datable.iso} (\textit{@when-iso}, \textit{@notBefore-iso}, \textit{@notAfter-iso}, \textit{@from-iso}, \textit{@to-iso})) (\hyperref[TEI.att.datable.custom]{att.datable.custom} (\textit{@when-custom}, \textit{@notBefore-custom}, \textit{@notAfter-custom}, \textit{@from-custom}, \textit{@to-custom}, \textit{@datingPoint}, \textit{@datingMethod})) \hyperref[TEI.att.duration]{att.duration} (\hyperref[TEI.att.duration.w3c]{att.duration.w3c} (\textit{@dur})) (\hyperref[TEI.att.duration.iso]{att.duration.iso} (\textit{@dur-iso})) \hyperref[TEI.att.editLike]{att.editLike} (\textit{@evidence}, \textit{@instant})  (\hyperref[TEI.att.dimensions]{att.dimensions} (\textit{@unit}, \textit{@quantity}, \textit{@extent}, \textit{@precision}, \textit{@scope}) (\hyperref[TEI.att.ranging]{att.ranging} (\textit{@atLeast}, \textit{@atMost}, \textit{@min}, \textit{@max}, \textit{@confidence})) ) \hyperref[TEI.att.typed]{att.typed} (\textit{@type}, \textit{@subtype}) 
    \item[{Membre du}]
  \hyperref[TEI.model.dateLike]{model.dateLike}
    \item[{Contenu dans}]
  
    \item[analysis: ]
   \hyperref[TEI.cl]{cl} \hyperref[TEI.phr]{phr} \hyperref[TEI.s]{s} \hyperref[TEI.span]{span}\par 
    \item[core: ]
   \hyperref[TEI.abbr]{abbr} \hyperref[TEI.add]{add} \hyperref[TEI.addrLine]{addrLine} \hyperref[TEI.author]{author} \hyperref[TEI.bibl]{bibl} \hyperref[TEI.biblScope]{biblScope} \hyperref[TEI.citedRange]{citedRange} \hyperref[TEI.corr]{corr} \hyperref[TEI.date]{date} \hyperref[TEI.del]{del} \hyperref[TEI.desc]{desc} \hyperref[TEI.distinct]{distinct} \hyperref[TEI.editor]{editor} \hyperref[TEI.email]{email} \hyperref[TEI.emph]{emph} \hyperref[TEI.expan]{expan} \hyperref[TEI.foreign]{foreign} \hyperref[TEI.gloss]{gloss} \hyperref[TEI.head]{head} \hyperref[TEI.headItem]{headItem} \hyperref[TEI.headLabel]{headLabel} \hyperref[TEI.hi]{hi} \hyperref[TEI.imprint]{imprint} \hyperref[TEI.item]{item} \hyperref[TEI.l]{l} \hyperref[TEI.label]{label} \hyperref[TEI.measure]{measure} \hyperref[TEI.meeting]{meeting} \hyperref[TEI.mentioned]{mentioned} \hyperref[TEI.name]{name} \hyperref[TEI.note]{note} \hyperref[TEI.num]{num} \hyperref[TEI.orig]{orig} \hyperref[TEI.p]{p} \hyperref[TEI.pubPlace]{pubPlace} \hyperref[TEI.publisher]{publisher} \hyperref[TEI.q]{q} \hyperref[TEI.quote]{quote} \hyperref[TEI.ref]{ref} \hyperref[TEI.reg]{reg} \hyperref[TEI.resp]{resp} \hyperref[TEI.rs]{rs} \hyperref[TEI.said]{said} \hyperref[TEI.sic]{sic} \hyperref[TEI.soCalled]{soCalled} \hyperref[TEI.speaker]{speaker} \hyperref[TEI.stage]{stage} \hyperref[TEI.street]{street} \hyperref[TEI.term]{term} \hyperref[TEI.textLang]{textLang} \hyperref[TEI.time]{time} \hyperref[TEI.title]{title} \hyperref[TEI.unclear]{unclear}\par 
    \item[figures: ]
   \hyperref[TEI.cell]{cell} \hyperref[TEI.figDesc]{figDesc}\par 
    \item[header: ]
   \hyperref[TEI.authority]{authority} \hyperref[TEI.change]{change} \hyperref[TEI.classCode]{classCode} \hyperref[TEI.creation]{creation} \hyperref[TEI.distributor]{distributor} \hyperref[TEI.edition]{edition} \hyperref[TEI.extent]{extent} \hyperref[TEI.funder]{funder} \hyperref[TEI.language]{language} \hyperref[TEI.licence]{licence} \hyperref[TEI.rendition]{rendition}\par 
    \item[iso-fs: ]
   \hyperref[TEI.fDescr]{fDescr} \hyperref[TEI.fsDescr]{fsDescr}\par 
    \item[linking: ]
   \hyperref[TEI.ab]{ab} \hyperref[TEI.seg]{seg}\par 
    \item[msdescription: ]
   \hyperref[TEI.accMat]{accMat} \hyperref[TEI.acquisition]{acquisition} \hyperref[TEI.additions]{additions} \hyperref[TEI.catchwords]{catchwords} \hyperref[TEI.collation]{collation} \hyperref[TEI.colophon]{colophon} \hyperref[TEI.condition]{condition} \hyperref[TEI.custEvent]{custEvent} \hyperref[TEI.decoNote]{decoNote} \hyperref[TEI.explicit]{explicit} \hyperref[TEI.filiation]{filiation} \hyperref[TEI.finalRubric]{finalRubric} \hyperref[TEI.foliation]{foliation} \hyperref[TEI.heraldry]{heraldry} \hyperref[TEI.incipit]{incipit} \hyperref[TEI.layout]{layout} \hyperref[TEI.material]{material} \hyperref[TEI.musicNotation]{musicNotation} \hyperref[TEI.objectType]{objectType} \hyperref[TEI.origDate]{origDate} \hyperref[TEI.origPlace]{origPlace} \hyperref[TEI.origin]{origin} \hyperref[TEI.provenance]{provenance} \hyperref[TEI.rubric]{rubric} \hyperref[TEI.secFol]{secFol} \hyperref[TEI.signatures]{signatures} \hyperref[TEI.source]{source} \hyperref[TEI.stamp]{stamp} \hyperref[TEI.summary]{summary} \hyperref[TEI.support]{support} \hyperref[TEI.surrogates]{surrogates} \hyperref[TEI.typeNote]{typeNote} \hyperref[TEI.watermark]{watermark}\par 
    \item[namesdates: ]
   \hyperref[TEI.addName]{addName} \hyperref[TEI.affiliation]{affiliation} \hyperref[TEI.country]{country} \hyperref[TEI.forename]{forename} \hyperref[TEI.genName]{genName} \hyperref[TEI.geogName]{geogName} \hyperref[TEI.nameLink]{nameLink} \hyperref[TEI.orgName]{orgName} \hyperref[TEI.persName]{persName} \hyperref[TEI.placeName]{placeName} \hyperref[TEI.region]{region} \hyperref[TEI.roleName]{roleName} \hyperref[TEI.settlement]{settlement} \hyperref[TEI.surname]{surname}\par 
    \item[spoken: ]
   \hyperref[TEI.annotationBlock]{annotationBlock}\par 
    \item[standOff: ]
   \hyperref[TEI.listAnnotation]{listAnnotation}\par 
    \item[textstructure: ]
   \hyperref[TEI.docAuthor]{docAuthor} \hyperref[TEI.docDate]{docDate} \hyperref[TEI.docEdition]{docEdition} \hyperref[TEI.titlePart]{titlePart}\par 
    \item[transcr: ]
   \hyperref[TEI.damage]{damage} \hyperref[TEI.fw]{fw} \hyperref[TEI.metamark]{metamark} \hyperref[TEI.mod]{mod} \hyperref[TEI.restore]{restore} \hyperref[TEI.retrace]{retrace} \hyperref[TEI.secl]{secl} \hyperref[TEI.supplied]{supplied} \hyperref[TEI.surplus]{surplus}
    \item[{Peut contenir}]
  
    \item[analysis: ]
   \hyperref[TEI.c]{c} \hyperref[TEI.cl]{cl} \hyperref[TEI.interp]{interp} \hyperref[TEI.interpGrp]{interpGrp} \hyperref[TEI.m]{m} \hyperref[TEI.pc]{pc} \hyperref[TEI.phr]{phr} \hyperref[TEI.s]{s} \hyperref[TEI.span]{span} \hyperref[TEI.spanGrp]{spanGrp} \hyperref[TEI.w]{w}\par 
    \item[core: ]
   \hyperref[TEI.abbr]{abbr} \hyperref[TEI.add]{add} \hyperref[TEI.address]{address} \hyperref[TEI.binaryObject]{binaryObject} \hyperref[TEI.cb]{cb} \hyperref[TEI.choice]{choice} \hyperref[TEI.corr]{corr} \hyperref[TEI.date]{date} \hyperref[TEI.del]{del} \hyperref[TEI.distinct]{distinct} \hyperref[TEI.email]{email} \hyperref[TEI.emph]{emph} \hyperref[TEI.expan]{expan} \hyperref[TEI.foreign]{foreign} \hyperref[TEI.gap]{gap} \hyperref[TEI.gb]{gb} \hyperref[TEI.gloss]{gloss} \hyperref[TEI.graphic]{graphic} \hyperref[TEI.hi]{hi} \hyperref[TEI.index]{index} \hyperref[TEI.lb]{lb} \hyperref[TEI.measure]{measure} \hyperref[TEI.measureGrp]{measureGrp} \hyperref[TEI.media]{media} \hyperref[TEI.mentioned]{mentioned} \hyperref[TEI.milestone]{milestone} \hyperref[TEI.name]{name} \hyperref[TEI.note]{note} \hyperref[TEI.num]{num} \hyperref[TEI.orig]{orig} \hyperref[TEI.pb]{pb} \hyperref[TEI.ptr]{ptr} \hyperref[TEI.ref]{ref} \hyperref[TEI.reg]{reg} \hyperref[TEI.rs]{rs} \hyperref[TEI.sic]{sic} \hyperref[TEI.soCalled]{soCalled} \hyperref[TEI.term]{term} \hyperref[TEI.time]{time} \hyperref[TEI.title]{title} \hyperref[TEI.unclear]{unclear}\par 
    \item[derived-module-tei.istex: ]
   \hyperref[TEI.math]{math} \hyperref[TEI.mrow]{mrow}\par 
    \item[figures: ]
   \hyperref[TEI.figure]{figure} \hyperref[TEI.formula]{formula} \hyperref[TEI.notatedMusic]{notatedMusic}\par 
    \item[header: ]
   \hyperref[TEI.idno]{idno}\par 
    \item[iso-fs: ]
   \hyperref[TEI.fLib]{fLib} \hyperref[TEI.fs]{fs} \hyperref[TEI.fvLib]{fvLib}\par 
    \item[linking: ]
   \hyperref[TEI.alt]{alt} \hyperref[TEI.altGrp]{altGrp} \hyperref[TEI.anchor]{anchor} \hyperref[TEI.join]{join} \hyperref[TEI.joinGrp]{joinGrp} \hyperref[TEI.link]{link} \hyperref[TEI.linkGrp]{linkGrp} \hyperref[TEI.seg]{seg} \hyperref[TEI.timeline]{timeline}\par 
    \item[msdescription: ]
   \hyperref[TEI.catchwords]{catchwords} \hyperref[TEI.depth]{depth} \hyperref[TEI.dim]{dim} \hyperref[TEI.dimensions]{dimensions} \hyperref[TEI.height]{height} \hyperref[TEI.heraldry]{heraldry} \hyperref[TEI.locus]{locus} \hyperref[TEI.locusGrp]{locusGrp} \hyperref[TEI.material]{material} \hyperref[TEI.objectType]{objectType} \hyperref[TEI.origDate]{origDate} \hyperref[TEI.origPlace]{origPlace} \hyperref[TEI.secFol]{secFol} \hyperref[TEI.signatures]{signatures} \hyperref[TEI.source]{source} \hyperref[TEI.stamp]{stamp} \hyperref[TEI.watermark]{watermark} \hyperref[TEI.width]{width}\par 
    \item[namesdates: ]
   \hyperref[TEI.addName]{addName} \hyperref[TEI.affiliation]{affiliation} \hyperref[TEI.country]{country} \hyperref[TEI.forename]{forename} \hyperref[TEI.genName]{genName} \hyperref[TEI.geogName]{geogName} \hyperref[TEI.location]{location} \hyperref[TEI.nameLink]{nameLink} \hyperref[TEI.orgName]{orgName} \hyperref[TEI.persName]{persName} \hyperref[TEI.placeName]{placeName} \hyperref[TEI.region]{region} \hyperref[TEI.roleName]{roleName} \hyperref[TEI.settlement]{settlement} \hyperref[TEI.state]{state} \hyperref[TEI.surname]{surname}\par 
    \item[spoken: ]
   \hyperref[TEI.annotationBlock]{annotationBlock}\par 
    \item[transcr: ]
   \hyperref[TEI.addSpan]{addSpan} \hyperref[TEI.am]{am} \hyperref[TEI.damage]{damage} \hyperref[TEI.damageSpan]{damageSpan} \hyperref[TEI.delSpan]{delSpan} \hyperref[TEI.ex]{ex} \hyperref[TEI.fw]{fw} \hyperref[TEI.handShift]{handShift} \hyperref[TEI.listTranspose]{listTranspose} \hyperref[TEI.metamark]{metamark} \hyperref[TEI.mod]{mod} \hyperref[TEI.redo]{redo} \hyperref[TEI.restore]{restore} \hyperref[TEI.retrace]{retrace} \hyperref[TEI.secl]{secl} \hyperref[TEI.space]{space} \hyperref[TEI.subst]{subst} \hyperref[TEI.substJoin]{substJoin} \hyperref[TEI.supplied]{supplied} \hyperref[TEI.surplus]{surplus} \hyperref[TEI.undo]{undo}\par des données textuelles
    \item[{Exemple}]
  \leavevmode\bgroup\exampleFont \begin{shaded}\noindent\mbox{} Bonsoir, il est {<\textbf{time}\hspace*{6pt}{when}="{00:00:00}">}minuit{</\textbf{time}>} ici, l'heure de dormir, et chez vous\mbox{}\newline 
 à Paris, il est seulement {<\textbf{time}\hspace*{6pt}{when}="{07:00:00}">}7 h.{</\textbf{time}>} Je te\mbox{}\newline 
 rapporterai plein de souvenirs pour te faire partager cette\mbox{}\newline 
 expérience unique. \end{shaded}\egroup 


    \item[{Modèle de contenu}]
  \mbox{}\hfill\\[-10pt]\begin{Verbatim}[fontsize=\small]
<content>
 <alternate maxOccurs="unbounded"
  minOccurs="0">
  <textNode/>
  <classRef key="model.gLike"/>
  <classRef key="model.phrase"/>
  <classRef key="model.global"/>
 </alternate>
</content>
    
\end{Verbatim}

    \item[{Schéma Declaration}]
  \mbox{}\hfill\\[-10pt]\begin{Verbatim}[fontsize=\small]
element time
{
   tei_att.global.attributes,
   tei_att.datable.attributes,
   tei_att.duration.attributes,
   tei_att.editLike.attributes,
   tei_att.typed.attributes,
   ( text | tei_model.gLike | tei_model.phrase | tei_model.global )*
}
\end{Verbatim}

\end{reflist}  \index{timeline=<timeline>|oddindex}\index{origin=@origin!<timeline>|oddindex}\index{unit=@unit!<timeline>|oddindex}\index{interval=@interval!<timeline>|oddindex}
\begin{reflist}
\item[]\begin{specHead}{TEI.timeline}{<timeline> }(frise chronologique) fournit un ensemble de points ordonnés dans le temps qui peuvent être liés à des éléments de la parole transcrite pour créer un alignement temporel de ce texte. [\xref{http://www.tei-c.org/release/doc/tei-p5-doc/en/html/SA.html\#SASYMP}{16.4.2. Placing Synchronous Events in Time}]\end{specHead} 
    \item[{Module}]
  linking
    \item[{Attributs}]
  Attributs \hyperref[TEI.att.global]{att.global} (\textit{@xml:id}, \textit{@n}, \textit{@xml:lang}, \textit{@xml:base}, \textit{@xml:space})  (\hyperref[TEI.att.global.rendition]{att.global.rendition} (\textit{@rend}, \textit{@style}, \textit{@rendition})) (\hyperref[TEI.att.global.linking]{att.global.linking} (\textit{@corresp}, \textit{@synch}, \textit{@sameAs}, \textit{@copyOf}, \textit{@next}, \textit{@prev}, \textit{@exclude}, \textit{@select})) (\hyperref[TEI.att.global.analytic]{att.global.analytic} (\textit{@ana})) (\hyperref[TEI.att.global.facs]{att.global.facs} (\textit{@facs})) (\hyperref[TEI.att.global.change]{att.global.change} (\textit{@change})) (\hyperref[TEI.att.global.responsibility]{att.global.responsibility} (\textit{@cert}, \textit{@resp})) (\hyperref[TEI.att.global.source]{att.global.source} (\textit{@source})) \hfil\\[-10pt]\begin{sansreflist}
    \item[@origin]
  désigne le début de la frise chronologique, c'est-à-dire le moment où elle commence.
\begin{reflist}
    \item[{Statut}]
  Optionel
    \item[{Type de données}]
  \hyperref[TEI.teidata.pointer]{teidata.pointer}
    \item[{Note}]
  \par
Si cet attribut n'est pas fourni, cela implique que le moment où commence la frise n'est pas connu.
\end{reflist}  
    \item[@unit]
  spécifie l'unité de temps correspondant à la valeur de l'attribut {\itshape interval} de la frise chronologique ou des points temporels qui la constituent.
\begin{reflist}
    \item[{Statut}]
  Optionel
    \item[{Type de données}]
  \hyperref[TEI.teidata.enumerated]{teidata.enumerated}
    \item[{Les valeurs suggérées comprennent:}]
  \begin{description}

\item[{d}](jours)
\item[{h}](heures)
\item[{min}](minutes)
\item[{s}](secondes)
\item[{ms}](millisecondes)
\end{description} 
\end{reflist}  
    \item[@interval]
  spécifie la partie numérique d'un intervalle de temps.
\begin{reflist}
    \item[{Statut}]
  Optionel
    \item[{Type de données}]
  \hyperref[TEI.teidata.interval]{teidata.interval}
    \item[{Note}]
  \par
La valeur irregular indique une incertitude sur tous les intervalles de la frise chronologique ; la valeur regular indique que tous les intervalles sont espacés régulièrement, mais que leur taille est inconnue ; des valeurs numériques indiquent des intervalles régulièrement espacés, de la taille spécifiée. Si on attribue à certains points temporels de la frise chronologique des valeurs différentes pour l'attribut {\itshape interval}, ces valeurs prennent le pas pour ces points sur la valeur donnée dans la frise chronologique.
\end{reflist}  
\end{sansreflist}  
    \item[{Membre du}]
  \hyperref[TEI.model.global.meta]{model.global.meta}
    \item[{Contenu dans}]
  
    \item[analysis: ]
   \hyperref[TEI.cl]{cl} \hyperref[TEI.m]{m} \hyperref[TEI.phr]{phr} \hyperref[TEI.s]{s} \hyperref[TEI.span]{span} \hyperref[TEI.w]{w}\par 
    \item[core: ]
   \hyperref[TEI.abbr]{abbr} \hyperref[TEI.add]{add} \hyperref[TEI.addrLine]{addrLine} \hyperref[TEI.address]{address} \hyperref[TEI.author]{author} \hyperref[TEI.bibl]{bibl} \hyperref[TEI.biblScope]{biblScope} \hyperref[TEI.cit]{cit} \hyperref[TEI.citedRange]{citedRange} \hyperref[TEI.corr]{corr} \hyperref[TEI.date]{date} \hyperref[TEI.del]{del} \hyperref[TEI.distinct]{distinct} \hyperref[TEI.editor]{editor} \hyperref[TEI.email]{email} \hyperref[TEI.emph]{emph} \hyperref[TEI.expan]{expan} \hyperref[TEI.foreign]{foreign} \hyperref[TEI.gloss]{gloss} \hyperref[TEI.head]{head} \hyperref[TEI.headItem]{headItem} \hyperref[TEI.headLabel]{headLabel} \hyperref[TEI.hi]{hi} \hyperref[TEI.imprint]{imprint} \hyperref[TEI.item]{item} \hyperref[TEI.l]{l} \hyperref[TEI.label]{label} \hyperref[TEI.lg]{lg} \hyperref[TEI.list]{list} \hyperref[TEI.measure]{measure} \hyperref[TEI.mentioned]{mentioned} \hyperref[TEI.name]{name} \hyperref[TEI.note]{note} \hyperref[TEI.num]{num} \hyperref[TEI.orig]{orig} \hyperref[TEI.p]{p} \hyperref[TEI.pubPlace]{pubPlace} \hyperref[TEI.publisher]{publisher} \hyperref[TEI.q]{q} \hyperref[TEI.quote]{quote} \hyperref[TEI.ref]{ref} \hyperref[TEI.reg]{reg} \hyperref[TEI.resp]{resp} \hyperref[TEI.rs]{rs} \hyperref[TEI.said]{said} \hyperref[TEI.series]{series} \hyperref[TEI.sic]{sic} \hyperref[TEI.soCalled]{soCalled} \hyperref[TEI.sp]{sp} \hyperref[TEI.speaker]{speaker} \hyperref[TEI.stage]{stage} \hyperref[TEI.street]{street} \hyperref[TEI.term]{term} \hyperref[TEI.textLang]{textLang} \hyperref[TEI.time]{time} \hyperref[TEI.title]{title} \hyperref[TEI.unclear]{unclear}\par 
    \item[figures: ]
   \hyperref[TEI.cell]{cell} \hyperref[TEI.figure]{figure} \hyperref[TEI.table]{table}\par 
    \item[header: ]
   \hyperref[TEI.authority]{authority} \hyperref[TEI.change]{change} \hyperref[TEI.classCode]{classCode} \hyperref[TEI.distributor]{distributor} \hyperref[TEI.edition]{edition} \hyperref[TEI.extent]{extent} \hyperref[TEI.funder]{funder} \hyperref[TEI.language]{language} \hyperref[TEI.licence]{licence}\par 
    \item[linking: ]
   \hyperref[TEI.ab]{ab} \hyperref[TEI.seg]{seg}\par 
    \item[msdescription: ]
   \hyperref[TEI.accMat]{accMat} \hyperref[TEI.acquisition]{acquisition} \hyperref[TEI.additions]{additions} \hyperref[TEI.catchwords]{catchwords} \hyperref[TEI.collation]{collation} \hyperref[TEI.colophon]{colophon} \hyperref[TEI.condition]{condition} \hyperref[TEI.custEvent]{custEvent} \hyperref[TEI.decoNote]{decoNote} \hyperref[TEI.explicit]{explicit} \hyperref[TEI.filiation]{filiation} \hyperref[TEI.finalRubric]{finalRubric} \hyperref[TEI.foliation]{foliation} \hyperref[TEI.heraldry]{heraldry} \hyperref[TEI.incipit]{incipit} \hyperref[TEI.layout]{layout} \hyperref[TEI.material]{material} \hyperref[TEI.msItem]{msItem} \hyperref[TEI.musicNotation]{musicNotation} \hyperref[TEI.objectType]{objectType} \hyperref[TEI.origDate]{origDate} \hyperref[TEI.origPlace]{origPlace} \hyperref[TEI.origin]{origin} \hyperref[TEI.provenance]{provenance} \hyperref[TEI.rubric]{rubric} \hyperref[TEI.secFol]{secFol} \hyperref[TEI.signatures]{signatures} \hyperref[TEI.source]{source} \hyperref[TEI.stamp]{stamp} \hyperref[TEI.summary]{summary} \hyperref[TEI.support]{support} \hyperref[TEI.surrogates]{surrogates} \hyperref[TEI.typeNote]{typeNote} \hyperref[TEI.watermark]{watermark}\par 
    \item[namesdates: ]
   \hyperref[TEI.addName]{addName} \hyperref[TEI.affiliation]{affiliation} \hyperref[TEI.country]{country} \hyperref[TEI.forename]{forename} \hyperref[TEI.genName]{genName} \hyperref[TEI.geogName]{geogName} \hyperref[TEI.nameLink]{nameLink} \hyperref[TEI.orgName]{orgName} \hyperref[TEI.persName]{persName} \hyperref[TEI.person]{person} \hyperref[TEI.personGrp]{personGrp} \hyperref[TEI.persona]{persona} \hyperref[TEI.placeName]{placeName} \hyperref[TEI.region]{region} \hyperref[TEI.roleName]{roleName} \hyperref[TEI.settlement]{settlement} \hyperref[TEI.surname]{surname}\par 
    \item[spoken: ]
   \hyperref[TEI.annotationBlock]{annotationBlock}\par 
    \item[standOff: ]
   \hyperref[TEI.listAnnotation]{listAnnotation}\par 
    \item[textstructure: ]
   \hyperref[TEI.back]{back} \hyperref[TEI.body]{body} \hyperref[TEI.div]{div} \hyperref[TEI.docAuthor]{docAuthor} \hyperref[TEI.docDate]{docDate} \hyperref[TEI.docEdition]{docEdition} \hyperref[TEI.docTitle]{docTitle} \hyperref[TEI.floatingText]{floatingText} \hyperref[TEI.front]{front} \hyperref[TEI.group]{group} \hyperref[TEI.text]{text} \hyperref[TEI.titlePage]{titlePage} \hyperref[TEI.titlePart]{titlePart}\par 
    \item[transcr: ]
   \hyperref[TEI.damage]{damage} \hyperref[TEI.fw]{fw} \hyperref[TEI.line]{line} \hyperref[TEI.metamark]{metamark} \hyperref[TEI.mod]{mod} \hyperref[TEI.restore]{restore} \hyperref[TEI.retrace]{retrace} \hyperref[TEI.secl]{secl} \hyperref[TEI.sourceDoc]{sourceDoc} \hyperref[TEI.supplied]{supplied} \hyperref[TEI.surface]{surface} \hyperref[TEI.surfaceGrp]{surfaceGrp} \hyperref[TEI.surplus]{surplus} \hyperref[TEI.zone]{zone}
    \item[{Peut contenir}]
  
    \item[linking: ]
   \hyperref[TEI.when]{when}
    \item[{Exemple}]
  \leavevmode\bgroup\exampleFont \begin{shaded}\noindent\mbox{}{<\textbf{timeline}\hspace*{6pt}{unit}="{ms}"\hspace*{6pt}{xml:id}="{TL01}">}\mbox{}\newline 
\hspace*{6pt}{<\textbf{when}\hspace*{6pt}{absolute}="{11:30:00}"\hspace*{6pt}{xml:id}="{TL-w0}"/>}\mbox{}\newline 
\hspace*{6pt}{<\textbf{when}\hspace*{6pt}{interval}="{unknown}"\hspace*{6pt}{since}="{\#TL-w0}"\mbox{}\newline 
\hspace*{6pt}\hspace*{6pt}{xml:id}="{TL-w1}"/>}\mbox{}\newline 
\hspace*{6pt}{<\textbf{when}\hspace*{6pt}{interval}="{100}"\hspace*{6pt}{since}="{\#TL-w1}"\mbox{}\newline 
\hspace*{6pt}\hspace*{6pt}{xml:id}="{TL-w2}"/>}\mbox{}\newline 
\hspace*{6pt}{<\textbf{when}\hspace*{6pt}{interval}="{200}"\hspace*{6pt}{since}="{\#TL-w2}"\mbox{}\newline 
\hspace*{6pt}\hspace*{6pt}{xml:id}="{TL-w3}"/>}\mbox{}\newline 
\hspace*{6pt}{<\textbf{when}\hspace*{6pt}{interval}="{150}"\hspace*{6pt}{since}="{\#TL-w3}"\mbox{}\newline 
\hspace*{6pt}\hspace*{6pt}{xml:id}="{TL-w4}"/>}\mbox{}\newline 
\hspace*{6pt}{<\textbf{when}\hspace*{6pt}{interval}="{250}"\hspace*{6pt}{since}="{\#TL-w4}"\mbox{}\newline 
\hspace*{6pt}\hspace*{6pt}{xml:id}="{TL-w5}"/>}\mbox{}\newline 
\hspace*{6pt}{<\textbf{when}\hspace*{6pt}{interval}="{100}"\hspace*{6pt}{since}="{\#TL-w5}"\mbox{}\newline 
\hspace*{6pt}\hspace*{6pt}{xml:id}="{TL-w6}"/>}\mbox{}\newline 
{</\textbf{timeline}>}\end{shaded}\egroup 


    \item[{Exemple}]
  \leavevmode\bgroup\exampleFont \begin{shaded}\noindent\mbox{}{<\textbf{timeline}\hspace*{6pt}{origin}="{\#fr\textunderscore TL-w0}"\hspace*{6pt}{unit}="{ms}"\mbox{}\newline 
\hspace*{6pt}{xml:id}="{fr\textunderscore TL01}">}\mbox{}\newline 
\hspace*{6pt}{<\textbf{when}\hspace*{6pt}{absolute}="{11:30:00}"\mbox{}\newline 
\hspace*{6pt}\hspace*{6pt}{xml:id}="{fr\textunderscore TL-w0}"/>}\mbox{}\newline 
\hspace*{6pt}{<\textbf{when}\hspace*{6pt}{interval}="{unknown}"\hspace*{6pt}{since}="{\#fr\textunderscore TL-w0}"\mbox{}\newline 
\hspace*{6pt}\hspace*{6pt}{xml:id}="{fr\textunderscore TL-w1}"/>}\mbox{}\newline 
\hspace*{6pt}{<\textbf{when}\hspace*{6pt}{interval}="{100}"\hspace*{6pt}{since}="{\#fr\textunderscore TL-w1}"\mbox{}\newline 
\hspace*{6pt}\hspace*{6pt}{xml:id}="{fr\textunderscore TL-w2}"/>}\mbox{}\newline 
\hspace*{6pt}{<\textbf{when}\hspace*{6pt}{interval}="{200}"\hspace*{6pt}{since}="{\#fr\textunderscore TL-w2}"\mbox{}\newline 
\hspace*{6pt}\hspace*{6pt}{xml:id}="{fr\textunderscore TL-w3}"/>}\mbox{}\newline 
\hspace*{6pt}{<\textbf{when}\hspace*{6pt}{interval}="{150}"\hspace*{6pt}{since}="{\#fr\textunderscore TL-w3}"\mbox{}\newline 
\hspace*{6pt}\hspace*{6pt}{xml:id}="{fr\textunderscore TL-w4}"/>}\mbox{}\newline 
\hspace*{6pt}{<\textbf{when}\hspace*{6pt}{interval}="{250}"\hspace*{6pt}{since}="{\#fr\textunderscore TL-w4}"\mbox{}\newline 
\hspace*{6pt}\hspace*{6pt}{xml:id}="{fr\textunderscore TL-w5}"/>}\mbox{}\newline 
\hspace*{6pt}{<\textbf{when}\hspace*{6pt}{interval}="{100}"\hspace*{6pt}{since}="{\#fr\textunderscore TL-w5}"\mbox{}\newline 
\hspace*{6pt}\hspace*{6pt}{xml:id}="{fr\textunderscore TL-w6}"/>}\mbox{}\newline 
{</\textbf{timeline}>}\end{shaded}\egroup 


    \item[{Modèle de contenu}]
  \mbox{}\hfill\\[-10pt]\begin{Verbatim}[fontsize=\small]
<content>
 <elementRef key="when"
  maxOccurs="unbounded" minOccurs="1"/>
</content>
    
\end{Verbatim}

    \item[{Schéma Declaration}]
  \mbox{}\hfill\\[-10pt]\begin{Verbatim}[fontsize=\small]
element timeline
{
   tei_att.global.attributes,
   attribute origin { text }?,
   attribute unit { "d" | "h" | "min" | "s" | "ms" }?,
   attribute interval { text }?,
   tei_when+
}
\end{Verbatim}

\end{reflist}  \index{title=<title>|oddindex}\index{type=@type!<title>|oddindex}\index{level=@level!<title>|oddindex}
\begin{reflist}
\item[]\begin{specHead}{TEI.title}{<title> }(titre) contient le titre complet d'une oeuvre quelconque [\xref{http://www.tei-c.org/release/doc/tei-p5-doc/en/html/CO.html\#COBICOR}{3.11.2.2. Titles, Authors, and Editors} \xref{http://www.tei-c.org/release/doc/tei-p5-doc/en/html/HD.html\#HD21}{2.2.1. The Title Statement} \xref{http://www.tei-c.org/release/doc/tei-p5-doc/en/html/HD.html\#HD26}{2.2.5. The Series Statement}]\end{specHead} 
    \item[{Module}]
  core
    \item[{Attributs}]
  Attributs \hyperref[TEI.att.global]{att.global} (\textit{@xml:id}, \textit{@n}, \textit{@xml:lang}, \textit{@xml:base}, \textit{@xml:space})  (\hyperref[TEI.att.global.rendition]{att.global.rendition} (\textit{@rend}, \textit{@style}, \textit{@rendition})) (\hyperref[TEI.att.global.linking]{att.global.linking} (\textit{@corresp}, \textit{@synch}, \textit{@sameAs}, \textit{@copyOf}, \textit{@next}, \textit{@prev}, \textit{@exclude}, \textit{@select})) (\hyperref[TEI.att.global.analytic]{att.global.analytic} (\textit{@ana})) (\hyperref[TEI.att.global.facs]{att.global.facs} (\textit{@facs})) (\hyperref[TEI.att.global.change]{att.global.change} (\textit{@change})) (\hyperref[TEI.att.global.responsibility]{att.global.responsibility} (\textit{@cert}, \textit{@resp})) (\hyperref[TEI.att.global.source]{att.global.source} (\textit{@source})) \hyperref[TEI.att.canonical]{att.canonical} (\textit{@key}, \textit{@ref}) \hyperref[TEI.att.datable]{att.datable} (\textit{@calendar}, \textit{@period})  (\hyperref[TEI.att.datable.w3c]{att.datable.w3c} (\textit{@when}, \textit{@notBefore}, \textit{@notAfter}, \textit{@from}, \textit{@to})) (\hyperref[TEI.att.datable.iso]{att.datable.iso} (\textit{@when-iso}, \textit{@notBefore-iso}, \textit{@notAfter-iso}, \textit{@from-iso}, \textit{@to-iso})) (\hyperref[TEI.att.datable.custom]{att.datable.custom} (\textit{@when-custom}, \textit{@notBefore-custom}, \textit{@notAfter-custom}, \textit{@from-custom}, \textit{@to-custom}, \textit{@datingPoint}, \textit{@datingMethod})) \hyperref[TEI.att.typed]{att.typed} (\unusedattribute{type}, @subtype) \hfil\\[-10pt]\begin{sansreflist}
    \item[@type]
  caractérise le titre selon une typologie adaptée.
\begin{reflist}
    \item[{Dérivé de}]
  \hyperref[TEI.att.typed]{att.typed}
    \item[{Statut}]
  Optionel
    \item[{Type de données}]
  \hyperref[TEI.teidata.enumerated]{teidata.enumerated}
    \item[{Exemple de valeurs possibles:}]
  \begin{description}

\item[{main}]titre principal
\item[{sub}](titre de niveau inférieur, titre de partie) sous-titre, titre de partie.
\item[{alt}](titre alternatif, souvent dans une autre langue, par lequel l'oeuvre est également connu) autre titre, souvent exprimé dans une autre langue, par lequel l'ouvrage est aussi connu
\item[{short}]forme abrégée du titre
\item[{desc}](paraphrase descriptive de l'oeuvre ayant les fonctions d'un titre) paraphrase descriptive de l'oeuvre fonctionnant comme un titre
\end{description} 
    \item[{Note}]
  \par
Cet attribut est utile pour analyser les titres et les traiter en fonction de leur type ; lorsqu'un tel traitement spécifique n'est pas nécessaire, il n'est pas utile de donner une telle analyse, et le titre entier, sous-titres et titres parallèles inclus, peuvent être encodés dans un élément \hyperref[TEI.title]{<title>}.
\end{reflist}  
    \item[@level]
  indique le niveau bibliographique d'un titre, c'est-à-dire si ce titre identifie un article, un livre, une revue, une collection, ou un document non publié
\begin{reflist}
    \item[{Statut}]
  Optionel
    \item[{Type de données}]
  \hyperref[TEI.teidata.enumerated]{teidata.enumerated}
    \item[{Les valeurs autorisées sont:}]
  \begin{description}

\item[{a}](analytique) titre analytique (article, poème ou autre, publié comme partie d'un ensemble plus grand)
\item[{m}](monographique) titre de monographie (livre, ensemble ou autre, publié comme un document distinct, y compris les volumes isolés d'ouvrages en plusieurs volumes)
\item[{j}](journal) titre de revue
\item[{s}](série) titre de publication en série
\item[{u}](unpublished) titre de matéria non publié (thèses et dissertations comprises, à l'exception de leurs éditions commerciales)
\end{description} 
    \item[{Note}]
  \par
Si le titre apparaît comme fils de l'élément \hyperref[TEI.analytic]{<analytic>}, l'attribut {\itshape level}, s'il est renseigné, doit avoir la valeur ‘a’ ; si le titre apparaît comme fils de l'élément \hyperref[TEI.monogr]{<monogr>}, l'attribut {\itshape level} doit avoir la valeur ‘m’, ‘j’ ou ‘u’ ; si le titre apparaît comme fils de l'élément \hyperref[TEI.series]{<series>}, l'attribut {\itshape level} doit avoir la valeur ‘s’. Si le titre apparaît dans l'élément \hyperref[TEI.msItem]{<msItem>}, l'attribut level ne doit pas être utilisé.
\end{reflist}  
\end{sansreflist}  
    \item[{Membre du}]
  \hyperref[TEI.model.emphLike]{model.emphLike} \hyperref[TEI.model.msQuoteLike]{model.msQuoteLike} 
    \item[{Contenu dans}]
  
    \item[analysis: ]
   \hyperref[TEI.cl]{cl} \hyperref[TEI.phr]{phr} \hyperref[TEI.s]{s} \hyperref[TEI.span]{span}\par 
    \item[core: ]
   \hyperref[TEI.abbr]{abbr} \hyperref[TEI.add]{add} \hyperref[TEI.addrLine]{addrLine} \hyperref[TEI.analytic]{analytic} \hyperref[TEI.author]{author} \hyperref[TEI.bibl]{bibl} \hyperref[TEI.biblScope]{biblScope} \hyperref[TEI.citedRange]{citedRange} \hyperref[TEI.corr]{corr} \hyperref[TEI.date]{date} \hyperref[TEI.del]{del} \hyperref[TEI.desc]{desc} \hyperref[TEI.distinct]{distinct} \hyperref[TEI.editor]{editor} \hyperref[TEI.email]{email} \hyperref[TEI.emph]{emph} \hyperref[TEI.expan]{expan} \hyperref[TEI.foreign]{foreign} \hyperref[TEI.gloss]{gloss} \hyperref[TEI.head]{head} \hyperref[TEI.headItem]{headItem} \hyperref[TEI.headLabel]{headLabel} \hyperref[TEI.hi]{hi} \hyperref[TEI.item]{item} \hyperref[TEI.l]{l} \hyperref[TEI.label]{label} \hyperref[TEI.measure]{measure} \hyperref[TEI.meeting]{meeting} \hyperref[TEI.mentioned]{mentioned} \hyperref[TEI.monogr]{monogr} \hyperref[TEI.name]{name} \hyperref[TEI.note]{note} \hyperref[TEI.num]{num} \hyperref[TEI.orig]{orig} \hyperref[TEI.p]{p} \hyperref[TEI.pubPlace]{pubPlace} \hyperref[TEI.publisher]{publisher} \hyperref[TEI.q]{q} \hyperref[TEI.quote]{quote} \hyperref[TEI.ref]{ref} \hyperref[TEI.reg]{reg} \hyperref[TEI.resp]{resp} \hyperref[TEI.rs]{rs} \hyperref[TEI.said]{said} \hyperref[TEI.series]{series} \hyperref[TEI.sic]{sic} \hyperref[TEI.soCalled]{soCalled} \hyperref[TEI.speaker]{speaker} \hyperref[TEI.stage]{stage} \hyperref[TEI.street]{street} \hyperref[TEI.term]{term} \hyperref[TEI.textLang]{textLang} \hyperref[TEI.time]{time} \hyperref[TEI.title]{title} \hyperref[TEI.unclear]{unclear}\par 
    \item[figures: ]
   \hyperref[TEI.cell]{cell} \hyperref[TEI.figDesc]{figDesc}\par 
    \item[header: ]
   \hyperref[TEI.authority]{authority} \hyperref[TEI.change]{change} \hyperref[TEI.classCode]{classCode} \hyperref[TEI.creation]{creation} \hyperref[TEI.distributor]{distributor} \hyperref[TEI.edition]{edition} \hyperref[TEI.extent]{extent} \hyperref[TEI.funder]{funder} \hyperref[TEI.language]{language} \hyperref[TEI.licence]{licence} \hyperref[TEI.rendition]{rendition} \hyperref[TEI.seriesStmt]{seriesStmt} \hyperref[TEI.titleStmt]{titleStmt}\par 
    \item[iso-fs: ]
   \hyperref[TEI.fDescr]{fDescr} \hyperref[TEI.fsDescr]{fsDescr}\par 
    \item[linking: ]
   \hyperref[TEI.ab]{ab} \hyperref[TEI.seg]{seg}\par 
    \item[msdescription: ]
   \hyperref[TEI.accMat]{accMat} \hyperref[TEI.acquisition]{acquisition} \hyperref[TEI.additions]{additions} \hyperref[TEI.catchwords]{catchwords} \hyperref[TEI.collation]{collation} \hyperref[TEI.colophon]{colophon} \hyperref[TEI.condition]{condition} \hyperref[TEI.custEvent]{custEvent} \hyperref[TEI.decoNote]{decoNote} \hyperref[TEI.explicit]{explicit} \hyperref[TEI.filiation]{filiation} \hyperref[TEI.finalRubric]{finalRubric} \hyperref[TEI.foliation]{foliation} \hyperref[TEI.heraldry]{heraldry} \hyperref[TEI.incipit]{incipit} \hyperref[TEI.layout]{layout} \hyperref[TEI.material]{material} \hyperref[TEI.msItem]{msItem} \hyperref[TEI.msItemStruct]{msItemStruct} \hyperref[TEI.musicNotation]{musicNotation} \hyperref[TEI.objectType]{objectType} \hyperref[TEI.origDate]{origDate} \hyperref[TEI.origPlace]{origPlace} \hyperref[TEI.origin]{origin} \hyperref[TEI.provenance]{provenance} \hyperref[TEI.rubric]{rubric} \hyperref[TEI.secFol]{secFol} \hyperref[TEI.signatures]{signatures} \hyperref[TEI.source]{source} \hyperref[TEI.stamp]{stamp} \hyperref[TEI.summary]{summary} \hyperref[TEI.support]{support} \hyperref[TEI.surrogates]{surrogates} \hyperref[TEI.typeNote]{typeNote} \hyperref[TEI.watermark]{watermark}\par 
    \item[namesdates: ]
   \hyperref[TEI.addName]{addName} \hyperref[TEI.affiliation]{affiliation} \hyperref[TEI.country]{country} \hyperref[TEI.forename]{forename} \hyperref[TEI.genName]{genName} \hyperref[TEI.geogName]{geogName} \hyperref[TEI.nameLink]{nameLink} \hyperref[TEI.orgName]{orgName} \hyperref[TEI.persName]{persName} \hyperref[TEI.placeName]{placeName} \hyperref[TEI.region]{region} \hyperref[TEI.roleName]{roleName} \hyperref[TEI.settlement]{settlement} \hyperref[TEI.surname]{surname}\par 
    \item[textstructure: ]
   \hyperref[TEI.docAuthor]{docAuthor} \hyperref[TEI.docDate]{docDate} \hyperref[TEI.docEdition]{docEdition} \hyperref[TEI.titlePart]{titlePart}\par 
    \item[transcr: ]
   \hyperref[TEI.damage]{damage} \hyperref[TEI.fw]{fw} \hyperref[TEI.metamark]{metamark} \hyperref[TEI.mod]{mod} \hyperref[TEI.restore]{restore} \hyperref[TEI.retrace]{retrace} \hyperref[TEI.secl]{secl} \hyperref[TEI.supplied]{supplied} \hyperref[TEI.surplus]{surplus}
    \item[{Peut contenir}]
  
    \item[analysis: ]
   \hyperref[TEI.c]{c} \hyperref[TEI.cl]{cl} \hyperref[TEI.interp]{interp} \hyperref[TEI.interpGrp]{interpGrp} \hyperref[TEI.m]{m} \hyperref[TEI.pc]{pc} \hyperref[TEI.phr]{phr} \hyperref[TEI.s]{s} \hyperref[TEI.span]{span} \hyperref[TEI.spanGrp]{spanGrp} \hyperref[TEI.w]{w}\par 
    \item[core: ]
   \hyperref[TEI.abbr]{abbr} \hyperref[TEI.add]{add} \hyperref[TEI.address]{address} \hyperref[TEI.bibl]{bibl} \hyperref[TEI.biblStruct]{biblStruct} \hyperref[TEI.binaryObject]{binaryObject} \hyperref[TEI.cb]{cb} \hyperref[TEI.choice]{choice} \hyperref[TEI.cit]{cit} \hyperref[TEI.corr]{corr} \hyperref[TEI.date]{date} \hyperref[TEI.del]{del} \hyperref[TEI.desc]{desc} \hyperref[TEI.distinct]{distinct} \hyperref[TEI.email]{email} \hyperref[TEI.emph]{emph} \hyperref[TEI.expan]{expan} \hyperref[TEI.foreign]{foreign} \hyperref[TEI.gap]{gap} \hyperref[TEI.gb]{gb} \hyperref[TEI.gloss]{gloss} \hyperref[TEI.graphic]{graphic} \hyperref[TEI.hi]{hi} \hyperref[TEI.index]{index} \hyperref[TEI.l]{l} \hyperref[TEI.label]{label} \hyperref[TEI.lb]{lb} \hyperref[TEI.lg]{lg} \hyperref[TEI.list]{list} \hyperref[TEI.listBibl]{listBibl} \hyperref[TEI.measure]{measure} \hyperref[TEI.measureGrp]{measureGrp} \hyperref[TEI.media]{media} \hyperref[TEI.mentioned]{mentioned} \hyperref[TEI.milestone]{milestone} \hyperref[TEI.name]{name} \hyperref[TEI.note]{note} \hyperref[TEI.num]{num} \hyperref[TEI.orig]{orig} \hyperref[TEI.pb]{pb} \hyperref[TEI.ptr]{ptr} \hyperref[TEI.q]{q} \hyperref[TEI.quote]{quote} \hyperref[TEI.ref]{ref} \hyperref[TEI.reg]{reg} \hyperref[TEI.rs]{rs} \hyperref[TEI.said]{said} \hyperref[TEI.sic]{sic} \hyperref[TEI.soCalled]{soCalled} \hyperref[TEI.stage]{stage} \hyperref[TEI.term]{term} \hyperref[TEI.time]{time} \hyperref[TEI.title]{title} \hyperref[TEI.unclear]{unclear}\par 
    \item[derived-module-tei.istex: ]
   \hyperref[TEI.math]{math} \hyperref[TEI.mrow]{mrow}\par 
    \item[figures: ]
   \hyperref[TEI.figure]{figure} \hyperref[TEI.formula]{formula} \hyperref[TEI.notatedMusic]{notatedMusic} \hyperref[TEI.table]{table}\par 
    \item[header: ]
   \hyperref[TEI.biblFull]{biblFull} \hyperref[TEI.idno]{idno}\par 
    \item[iso-fs: ]
   \hyperref[TEI.fLib]{fLib} \hyperref[TEI.fs]{fs} \hyperref[TEI.fvLib]{fvLib}\par 
    \item[linking: ]
   \hyperref[TEI.alt]{alt} \hyperref[TEI.altGrp]{altGrp} \hyperref[TEI.anchor]{anchor} \hyperref[TEI.join]{join} \hyperref[TEI.joinGrp]{joinGrp} \hyperref[TEI.link]{link} \hyperref[TEI.linkGrp]{linkGrp} \hyperref[TEI.seg]{seg} \hyperref[TEI.timeline]{timeline}\par 
    \item[msdescription: ]
   \hyperref[TEI.catchwords]{catchwords} \hyperref[TEI.depth]{depth} \hyperref[TEI.dim]{dim} \hyperref[TEI.dimensions]{dimensions} \hyperref[TEI.height]{height} \hyperref[TEI.heraldry]{heraldry} \hyperref[TEI.locus]{locus} \hyperref[TEI.locusGrp]{locusGrp} \hyperref[TEI.material]{material} \hyperref[TEI.msDesc]{msDesc} \hyperref[TEI.objectType]{objectType} \hyperref[TEI.origDate]{origDate} \hyperref[TEI.origPlace]{origPlace} \hyperref[TEI.secFol]{secFol} \hyperref[TEI.signatures]{signatures} \hyperref[TEI.source]{source} \hyperref[TEI.stamp]{stamp} \hyperref[TEI.watermark]{watermark} \hyperref[TEI.width]{width}\par 
    \item[namesdates: ]
   \hyperref[TEI.addName]{addName} \hyperref[TEI.affiliation]{affiliation} \hyperref[TEI.country]{country} \hyperref[TEI.forename]{forename} \hyperref[TEI.genName]{genName} \hyperref[TEI.geogName]{geogName} \hyperref[TEI.listOrg]{listOrg} \hyperref[TEI.listPlace]{listPlace} \hyperref[TEI.location]{location} \hyperref[TEI.nameLink]{nameLink} \hyperref[TEI.orgName]{orgName} \hyperref[TEI.persName]{persName} \hyperref[TEI.placeName]{placeName} \hyperref[TEI.region]{region} \hyperref[TEI.roleName]{roleName} \hyperref[TEI.settlement]{settlement} \hyperref[TEI.state]{state} \hyperref[TEI.surname]{surname}\par 
    \item[spoken: ]
   \hyperref[TEI.annotationBlock]{annotationBlock}\par 
    \item[textstructure: ]
   \hyperref[TEI.floatingText]{floatingText}\par 
    \item[transcr: ]
   \hyperref[TEI.addSpan]{addSpan} \hyperref[TEI.am]{am} \hyperref[TEI.damage]{damage} \hyperref[TEI.damageSpan]{damageSpan} \hyperref[TEI.delSpan]{delSpan} \hyperref[TEI.ex]{ex} \hyperref[TEI.fw]{fw} \hyperref[TEI.handShift]{handShift} \hyperref[TEI.listTranspose]{listTranspose} \hyperref[TEI.metamark]{metamark} \hyperref[TEI.mod]{mod} \hyperref[TEI.redo]{redo} \hyperref[TEI.restore]{restore} \hyperref[TEI.retrace]{retrace} \hyperref[TEI.secl]{secl} \hyperref[TEI.space]{space} \hyperref[TEI.subst]{subst} \hyperref[TEI.substJoin]{substJoin} \hyperref[TEI.supplied]{supplied} \hyperref[TEI.surplus]{surplus} \hyperref[TEI.undo]{undo}\par des données textuelles
    \item[{Note}]
  \par
Les attributs {\itshape key} et {\itshape ref}, hérités de la classe \textsf{att.canonical} sont utilisés pour indiquer la forme canonique du titre ; le premier donne (par exemple) l’identifiant d’un enregistrement dans un système externe de bibliothèque ; le second pointe vers un élément XML contenant la forme canonique du titre.
    \item[{Exemple}]
  \leavevmode\bgroup\exampleFont \begin{shaded}\noindent\mbox{}{<\textbf{title}>}La vie mode d'emploi. Romans.{</\textbf{title}>}\end{shaded}\egroup 


    \item[{Exemple}]
  \leavevmode\bgroup\exampleFont \begin{shaded}\noindent\mbox{}{<\textbf{title}>}Analyser les textes de communication{</\textbf{title}>}\end{shaded}\egroup 


    \item[{Exemple}]
  \leavevmode\bgroup\exampleFont \begin{shaded}\noindent\mbox{}{<\textbf{title}>}Mélanges de linguistique française et de philologie et littérature médiévales\mbox{}\newline 
 offerts à Monsieur Paul Imbs.{</\textbf{title}>}\end{shaded}\egroup 


    \item[{Exemple}]
  \leavevmode\bgroup\exampleFont \begin{shaded}\noindent\mbox{}{<\textbf{title}>}Les fleurs du Mal de Charles Baudelaire : une édition électronique{</\textbf{title}>}\end{shaded}\egroup 


    \item[{Exemple}]
  \leavevmode\bgroup\exampleFont \begin{shaded}\noindent\mbox{}{<\textbf{p}>}quand il rentre de ses chantiers de maçonnerie il dit rien, il pose son cul sur une\mbox{}\newline 
 chaise, toujours au même endroit, et il lit {<\textbf{title}>}l'Humanité.{</\textbf{title}>}\mbox{}\newline 
{</\textbf{p}>}\end{shaded}\egroup 


    \item[{Exemple}]
  \leavevmode\bgroup\exampleFont \begin{shaded}\noindent\mbox{}{<\textbf{title}\hspace*{6pt}{type}="{full}">}\mbox{}\newline 
\hspace*{6pt}{<\textbf{title}\hspace*{6pt}{type}="{main}">}Synthèse{</\textbf{title}>}\mbox{}\newline 
\hspace*{6pt}{<\textbf{title}\hspace*{6pt}{type}="{sub}">}an international journal for\mbox{}\newline 
\hspace*{6pt}\hspace*{6pt} epistemology, methodology and history of\mbox{}\newline 
\hspace*{6pt}\hspace*{6pt} science{</\textbf{title}>}\mbox{}\newline 
{</\textbf{title}>}\end{shaded}\egroup 


    \item[{Modèle de contenu}]
  \mbox{}\hfill\\[-10pt]\begin{Verbatim}[fontsize=\small]
<content>
 <macroRef key="macro.paraContent"/>
</content>
    
\end{Verbatim}

    \item[{Schéma Declaration}]
  \mbox{}\hfill\\[-10pt]\begin{Verbatim}[fontsize=\small]
element title
{
   tei_att.global.attributes,
   tei_att.canonical.attributes,
   tei_att.typed.attribute.subtype,
   tei_att.datable.attributes,
   attribute type { text }?,
   attribute level { "a" | "m" | "j" | "s" | "u" }?,
   tei_macro.paraContent}
\end{Verbatim}

\end{reflist}  \index{titlePage=<titlePage>|oddindex}\index{type=@type!<titlePage>|oddindex}
\begin{reflist}
\item[]\begin{specHead}{TEI.titlePage}{<titlePage> }(page de titre) contient la page de titre d’un texte qui figure dans les parties liminaires. [\xref{http://www.tei-c.org/release/doc/tei-p5-doc/en/html/DS.html\#DSTITL}{4.6. Title Pages}]\end{specHead} 
    \item[{Module}]
  textstructure
    \item[{Attributs}]
  Attributs \hyperref[TEI.att.global]{att.global} (\textit{@xml:id}, \textit{@n}, \textit{@xml:lang}, \textit{@xml:base}, \textit{@xml:space})  (\hyperref[TEI.att.global.rendition]{att.global.rendition} (\textit{@rend}, \textit{@style}, \textit{@rendition})) (\hyperref[TEI.att.global.linking]{att.global.linking} (\textit{@corresp}, \textit{@synch}, \textit{@sameAs}, \textit{@copyOf}, \textit{@next}, \textit{@prev}, \textit{@exclude}, \textit{@select})) (\hyperref[TEI.att.global.analytic]{att.global.analytic} (\textit{@ana})) (\hyperref[TEI.att.global.facs]{att.global.facs} (\textit{@facs})) (\hyperref[TEI.att.global.change]{att.global.change} (\textit{@change})) (\hyperref[TEI.att.global.responsibility]{att.global.responsibility} (\textit{@cert}, \textit{@resp})) (\hyperref[TEI.att.global.source]{att.global.source} (\textit{@source})) \hfil\\[-10pt]\begin{sansreflist}
    \item[@type]
  classe la page de titre selon la typologie appropriée.
\begin{reflist}
    \item[{Statut}]
  Optionel
    \item[{Type de données}]
  \hyperref[TEI.teidata.enumerated]{teidata.enumerated}
    \item[{Note}]
  \par
Cet attribut est utile parce que c'est le même élément \hyperref[TEI.titlePage]{<titlePage>} qui est utilisé pour les pages de titre de volumes, de collections, etc., et pour la page de titre ‘principale’ d'un ouvrage.
\end{reflist}  
\end{sansreflist}  
    \item[{Membre du}]
  \hyperref[TEI.model.frontPart]{model.frontPart} 
    \item[{Contenu dans}]
  
    \item[msdescription: ]
   \hyperref[TEI.msContents]{msContents}\par 
    \item[textstructure: ]
   \hyperref[TEI.back]{back} \hyperref[TEI.front]{front}
    \item[{Peut contenir}]
  
    \item[analysis: ]
   \hyperref[TEI.interp]{interp} \hyperref[TEI.interpGrp]{interpGrp} \hyperref[TEI.span]{span} \hyperref[TEI.spanGrp]{spanGrp}\par 
    \item[core: ]
   \hyperref[TEI.binaryObject]{binaryObject} \hyperref[TEI.cb]{cb} \hyperref[TEI.gap]{gap} \hyperref[TEI.gb]{gb} \hyperref[TEI.graphic]{graphic} \hyperref[TEI.index]{index} \hyperref[TEI.lb]{lb} \hyperref[TEI.milestone]{milestone} \hyperref[TEI.note]{note} \hyperref[TEI.pb]{pb}\par 
    \item[figures: ]
   \hyperref[TEI.figure]{figure} \hyperref[TEI.notatedMusic]{notatedMusic}\par 
    \item[iso-fs: ]
   \hyperref[TEI.fLib]{fLib} \hyperref[TEI.fs]{fs} \hyperref[TEI.fvLib]{fvLib}\par 
    \item[linking: ]
   \hyperref[TEI.alt]{alt} \hyperref[TEI.altGrp]{altGrp} \hyperref[TEI.anchor]{anchor} \hyperref[TEI.join]{join} \hyperref[TEI.joinGrp]{joinGrp} \hyperref[TEI.link]{link} \hyperref[TEI.linkGrp]{linkGrp} \hyperref[TEI.timeline]{timeline}\par 
    \item[msdescription: ]
   \hyperref[TEI.source]{source}\par 
    \item[textstructure: ]
   \hyperref[TEI.docAuthor]{docAuthor} \hyperref[TEI.docDate]{docDate} \hyperref[TEI.docEdition]{docEdition} \hyperref[TEI.docTitle]{docTitle} \hyperref[TEI.titlePart]{titlePart}\par 
    \item[transcr: ]
   \hyperref[TEI.addSpan]{addSpan} \hyperref[TEI.damageSpan]{damageSpan} \hyperref[TEI.delSpan]{delSpan} \hyperref[TEI.fw]{fw} \hyperref[TEI.listTranspose]{listTranspose} \hyperref[TEI.metamark]{metamark} \hyperref[TEI.space]{space} \hyperref[TEI.substJoin]{substJoin}
    \item[{Exemple}]
  \leavevmode\bgroup\exampleFont \begin{shaded}\noindent\mbox{}{<\textbf{titlePage}>}\mbox{}\newline 
\hspace*{6pt}{<\textbf{docTitle}>}\mbox{}\newline 
\hspace*{6pt}\hspace*{6pt}{<\textbf{titlePart}\hspace*{6pt}{type}="{main}">} Histoire du Roi de Bohême{</\textbf{titlePart}>}\mbox{}\newline 
\hspace*{6pt}\hspace*{6pt}{<\textbf{titlePart}\hspace*{6pt}{type}="{sub}">} et de ses sept châteaux {</\textbf{titlePart}>}\mbox{}\newline 
\hspace*{6pt}{</\textbf{docTitle}>}\mbox{}\newline 
\hspace*{6pt}{<\textbf{titlePart}>}Pastiche.{</\textbf{titlePart}>}\mbox{}\newline 
\hspace*{6pt}{<\textbf{byline}>}Par {<\textbf{docAuthor}>}Charles Nodier{</\textbf{docAuthor}>}\mbox{}\newline 
\hspace*{6pt}{</\textbf{byline}>}\mbox{}\newline 
\hspace*{6pt}{<\textbf{epigraph}>}\mbox{}\newline 
\hspace*{6pt}\hspace*{6pt}{<\textbf{q}>}O imitatores, servum pecus! {</\textbf{q}>}\mbox{}\newline 
\hspace*{6pt}\hspace*{6pt}{<\textbf{bibl}>}Horat., Epist. I. XIX, 19.{</\textbf{bibl}>}\mbox{}\newline 
\hspace*{6pt}{</\textbf{epigraph}>}\mbox{}\newline 
\hspace*{6pt}{<\textbf{docImprint}>}\mbox{}\newline 
\hspace*{6pt}\hspace*{6pt}{<\textbf{name}>}PARIS{</\textbf{name}>}, {<\textbf{name}>}Delangle Frères{</\textbf{name}>} Éditeurs-libraires,\mbox{}\newline 
\hspace*{6pt}{<\textbf{name}>}Place de la Bourse{</\textbf{name}>}\mbox{}\newline 
\hspace*{6pt}{</\textbf{docImprint}>}\mbox{}\newline 
\hspace*{6pt}{<\textbf{docDate}>}MDCCCXXX{</\textbf{docDate}>}\mbox{}\newline 
{</\textbf{titlePage}>}\end{shaded}\egroup 


    \item[{Modèle de contenu}]
  \mbox{}\hfill\\[-10pt]\begin{Verbatim}[fontsize=\small]
<content>
 <sequence maxOccurs="1" minOccurs="1">
  <classRef key="model.global"
   maxOccurs="unbounded" minOccurs="0"/>
  <classRef key="model.titlepagePart"/>
  <alternate maxOccurs="unbounded"
   minOccurs="0">
   <classRef key="model.titlepagePart"/>
   <classRef key="model.global"/>
  </alternate>
 </sequence>
</content>
    
\end{Verbatim}

    \item[{Schéma Declaration}]
  \mbox{}\hfill\\[-10pt]\begin{Verbatim}[fontsize=\small]
element titlePage
{
   tei_att.global.attributes,
   attribute type { text }?,
   (
      tei_model.global*,
      tei_model.titlepagePart,
      ( tei_model.titlepagePart | tei_model.global )*
   )
}
\end{Verbatim}

\end{reflist}  \index{titlePart=<titlePart>|oddindex}\index{type=@type!<titlePart>|oddindex}
\begin{reflist}
\item[]\begin{specHead}{TEI.titlePart}{<titlePart> }contient une section ou division du titre d’un ouvrage telle qu’elle est indiquée sur la page de titre. [\xref{http://www.tei-c.org/release/doc/tei-p5-doc/en/html/DS.html\#DSTITL}{4.6. Title Pages}]\end{specHead} 
    \item[{Module}]
  textstructure
    \item[{Attributs}]
  Attributs \hyperref[TEI.att.global]{att.global} (\textit{@xml:id}, \textit{@n}, \textit{@xml:lang}, \textit{@xml:base}, \textit{@xml:space})  (\hyperref[TEI.att.global.rendition]{att.global.rendition} (\textit{@rend}, \textit{@style}, \textit{@rendition})) (\hyperref[TEI.att.global.linking]{att.global.linking} (\textit{@corresp}, \textit{@synch}, \textit{@sameAs}, \textit{@copyOf}, \textit{@next}, \textit{@prev}, \textit{@exclude}, \textit{@select})) (\hyperref[TEI.att.global.analytic]{att.global.analytic} (\textit{@ana})) (\hyperref[TEI.att.global.facs]{att.global.facs} (\textit{@facs})) (\hyperref[TEI.att.global.change]{att.global.change} (\textit{@change})) (\hyperref[TEI.att.global.responsibility]{att.global.responsibility} (\textit{@cert}, \textit{@resp})) (\hyperref[TEI.att.global.source]{att.global.source} (\textit{@source})) \hfil\\[-10pt]\begin{sansreflist}
    \item[@type]
  précise le rôle de cette subdivision du titre.
\begin{reflist}
    \item[{Statut}]
  Optionel
    \item[{Type de données}]
  \hyperref[TEI.teidata.enumerated]{teidata.enumerated}
    \item[{Les valeurs suggérées comprennent:}]
  \begin{description}

\item[{main}]titre principal de l'oeuvre.{[Valeur par défaut] }
\item[{sub}](sous-titre de l’ouvrage.) sous-titre de l'oeuvre.
\item[{alt}](titre alternatif de l’ouvrage.) autre titre de l'oeuvre.
\item[{short}]forme abrégée du titre.
\item[{desc}](description paraphrastique de l’ouvrage.) texte qui paraphrase l'oeuvre.
\end{description} 
\end{reflist}  
\end{sansreflist}  
    \item[{Membre du}]
  \hyperref[TEI.model.pLike.front]{model.pLike.front} \hyperref[TEI.model.titlepagePart]{model.titlepagePart}
    \item[{Contenu dans}]
  
    \item[msdescription: ]
   \hyperref[TEI.msItem]{msItem}\par 
    \item[textstructure: ]
   \hyperref[TEI.back]{back} \hyperref[TEI.docTitle]{docTitle} \hyperref[TEI.front]{front} \hyperref[TEI.titlePage]{titlePage}
    \item[{Peut contenir}]
  
    \item[analysis: ]
   \hyperref[TEI.c]{c} \hyperref[TEI.cl]{cl} \hyperref[TEI.interp]{interp} \hyperref[TEI.interpGrp]{interpGrp} \hyperref[TEI.m]{m} \hyperref[TEI.pc]{pc} \hyperref[TEI.phr]{phr} \hyperref[TEI.s]{s} \hyperref[TEI.span]{span} \hyperref[TEI.spanGrp]{spanGrp} \hyperref[TEI.w]{w}\par 
    \item[core: ]
   \hyperref[TEI.abbr]{abbr} \hyperref[TEI.add]{add} \hyperref[TEI.address]{address} \hyperref[TEI.bibl]{bibl} \hyperref[TEI.biblStruct]{biblStruct} \hyperref[TEI.binaryObject]{binaryObject} \hyperref[TEI.cb]{cb} \hyperref[TEI.choice]{choice} \hyperref[TEI.cit]{cit} \hyperref[TEI.corr]{corr} \hyperref[TEI.date]{date} \hyperref[TEI.del]{del} \hyperref[TEI.desc]{desc} \hyperref[TEI.distinct]{distinct} \hyperref[TEI.email]{email} \hyperref[TEI.emph]{emph} \hyperref[TEI.expan]{expan} \hyperref[TEI.foreign]{foreign} \hyperref[TEI.gap]{gap} \hyperref[TEI.gb]{gb} \hyperref[TEI.gloss]{gloss} \hyperref[TEI.graphic]{graphic} \hyperref[TEI.hi]{hi} \hyperref[TEI.index]{index} \hyperref[TEI.l]{l} \hyperref[TEI.label]{label} \hyperref[TEI.lb]{lb} \hyperref[TEI.lg]{lg} \hyperref[TEI.list]{list} \hyperref[TEI.listBibl]{listBibl} \hyperref[TEI.measure]{measure} \hyperref[TEI.measureGrp]{measureGrp} \hyperref[TEI.media]{media} \hyperref[TEI.mentioned]{mentioned} \hyperref[TEI.milestone]{milestone} \hyperref[TEI.name]{name} \hyperref[TEI.note]{note} \hyperref[TEI.num]{num} \hyperref[TEI.orig]{orig} \hyperref[TEI.pb]{pb} \hyperref[TEI.ptr]{ptr} \hyperref[TEI.q]{q} \hyperref[TEI.quote]{quote} \hyperref[TEI.ref]{ref} \hyperref[TEI.reg]{reg} \hyperref[TEI.rs]{rs} \hyperref[TEI.said]{said} \hyperref[TEI.sic]{sic} \hyperref[TEI.soCalled]{soCalled} \hyperref[TEI.stage]{stage} \hyperref[TEI.term]{term} \hyperref[TEI.time]{time} \hyperref[TEI.title]{title} \hyperref[TEI.unclear]{unclear}\par 
    \item[derived-module-tei.istex: ]
   \hyperref[TEI.math]{math} \hyperref[TEI.mrow]{mrow}\par 
    \item[figures: ]
   \hyperref[TEI.figure]{figure} \hyperref[TEI.formula]{formula} \hyperref[TEI.notatedMusic]{notatedMusic} \hyperref[TEI.table]{table}\par 
    \item[header: ]
   \hyperref[TEI.biblFull]{biblFull} \hyperref[TEI.idno]{idno}\par 
    \item[iso-fs: ]
   \hyperref[TEI.fLib]{fLib} \hyperref[TEI.fs]{fs} \hyperref[TEI.fvLib]{fvLib}\par 
    \item[linking: ]
   \hyperref[TEI.alt]{alt} \hyperref[TEI.altGrp]{altGrp} \hyperref[TEI.anchor]{anchor} \hyperref[TEI.join]{join} \hyperref[TEI.joinGrp]{joinGrp} \hyperref[TEI.link]{link} \hyperref[TEI.linkGrp]{linkGrp} \hyperref[TEI.seg]{seg} \hyperref[TEI.timeline]{timeline}\par 
    \item[msdescription: ]
   \hyperref[TEI.catchwords]{catchwords} \hyperref[TEI.depth]{depth} \hyperref[TEI.dim]{dim} \hyperref[TEI.dimensions]{dimensions} \hyperref[TEI.height]{height} \hyperref[TEI.heraldry]{heraldry} \hyperref[TEI.locus]{locus} \hyperref[TEI.locusGrp]{locusGrp} \hyperref[TEI.material]{material} \hyperref[TEI.msDesc]{msDesc} \hyperref[TEI.objectType]{objectType} \hyperref[TEI.origDate]{origDate} \hyperref[TEI.origPlace]{origPlace} \hyperref[TEI.secFol]{secFol} \hyperref[TEI.signatures]{signatures} \hyperref[TEI.source]{source} \hyperref[TEI.stamp]{stamp} \hyperref[TEI.watermark]{watermark} \hyperref[TEI.width]{width}\par 
    \item[namesdates: ]
   \hyperref[TEI.addName]{addName} \hyperref[TEI.affiliation]{affiliation} \hyperref[TEI.country]{country} \hyperref[TEI.forename]{forename} \hyperref[TEI.genName]{genName} \hyperref[TEI.geogName]{geogName} \hyperref[TEI.listOrg]{listOrg} \hyperref[TEI.listPlace]{listPlace} \hyperref[TEI.location]{location} \hyperref[TEI.nameLink]{nameLink} \hyperref[TEI.orgName]{orgName} \hyperref[TEI.persName]{persName} \hyperref[TEI.placeName]{placeName} \hyperref[TEI.region]{region} \hyperref[TEI.roleName]{roleName} \hyperref[TEI.settlement]{settlement} \hyperref[TEI.state]{state} \hyperref[TEI.surname]{surname}\par 
    \item[spoken: ]
   \hyperref[TEI.annotationBlock]{annotationBlock}\par 
    \item[textstructure: ]
   \hyperref[TEI.floatingText]{floatingText}\par 
    \item[transcr: ]
   \hyperref[TEI.addSpan]{addSpan} \hyperref[TEI.am]{am} \hyperref[TEI.damage]{damage} \hyperref[TEI.damageSpan]{damageSpan} \hyperref[TEI.delSpan]{delSpan} \hyperref[TEI.ex]{ex} \hyperref[TEI.fw]{fw} \hyperref[TEI.handShift]{handShift} \hyperref[TEI.listTranspose]{listTranspose} \hyperref[TEI.metamark]{metamark} \hyperref[TEI.mod]{mod} \hyperref[TEI.redo]{redo} \hyperref[TEI.restore]{restore} \hyperref[TEI.retrace]{retrace} \hyperref[TEI.secl]{secl} \hyperref[TEI.space]{space} \hyperref[TEI.subst]{subst} \hyperref[TEI.substJoin]{substJoin} \hyperref[TEI.supplied]{supplied} \hyperref[TEI.surplus]{surplus} \hyperref[TEI.undo]{undo}\par des données textuelles
    \item[{Exemple}]
  \leavevmode\bgroup\exampleFont \begin{shaded}\noindent\mbox{}{<\textbf{docTitle}>}\mbox{}\newline 
\hspace*{6pt}{<\textbf{titlePart}\hspace*{6pt}{type}="{main}">}Cinq semaines en ballon.{</\textbf{titlePart}>}\mbox{}\newline 
\hspace*{6pt}{<\textbf{titlePart}\hspace*{6pt}{type}="{desc}">}Voyage de découvertes en Afrique par 3 anglais.{</\textbf{titlePart}>}\mbox{}\newline 
{</\textbf{docTitle}>}\end{shaded}\egroup 


    \item[{Modèle de contenu}]
  \mbox{}\hfill\\[-10pt]\begin{Verbatim}[fontsize=\small]
<content>
 <macroRef key="macro.paraContent"/>
</content>
    
\end{Verbatim}

    \item[{Schéma Declaration}]
  \mbox{}\hfill\\[-10pt]\begin{Verbatim}[fontsize=\small]
element titlePart
{
   tei_att.global.attributes,
   attribute type { "main" | "sub" | "alt" | "short" | "desc" }?,
   tei_macro.paraContent}
\end{Verbatim}

\end{reflist}  \index{titleStmt=<titleStmt>|oddindex}
\begin{reflist}
\item[]\begin{specHead}{TEI.titleStmt}{<titleStmt> }(mention de titre) regroupe les informations sur le titre d’une œuvre et les personnes ou institutions responsables de son contenu intellectuel. [\xref{http://www.tei-c.org/release/doc/tei-p5-doc/en/html/HD.html\#HD21}{2.2.1. The Title Statement} \xref{http://www.tei-c.org/release/doc/tei-p5-doc/en/html/HD.html\#HD2}{2.2. The File Description}]\end{specHead} 
    \item[{Module}]
  header
    \item[{Attributs}]
  Attributs \hyperref[TEI.att.global]{att.global} (\textit{@xml:id}, \textit{@n}, \textit{@xml:lang}, \textit{@xml:base}, \textit{@xml:space})  (\hyperref[TEI.att.global.rendition]{att.global.rendition} (\textit{@rend}, \textit{@style}, \textit{@rendition})) (\hyperref[TEI.att.global.linking]{att.global.linking} (\textit{@corresp}, \textit{@synch}, \textit{@sameAs}, \textit{@copyOf}, \textit{@next}, \textit{@prev}, \textit{@exclude}, \textit{@select})) (\hyperref[TEI.att.global.analytic]{att.global.analytic} (\textit{@ana})) (\hyperref[TEI.att.global.facs]{att.global.facs} (\textit{@facs})) (\hyperref[TEI.att.global.change]{att.global.change} (\textit{@change})) (\hyperref[TEI.att.global.responsibility]{att.global.responsibility} (\textit{@cert}, \textit{@resp})) (\hyperref[TEI.att.global.source]{att.global.source} (\textit{@source}))
    \item[{Contenu dans}]
  
    \item[header: ]
   \hyperref[TEI.biblFull]{biblFull} \hyperref[TEI.fileDesc]{fileDesc}
    \item[{Peut contenir}]
  
    \item[core: ]
   \hyperref[TEI.author]{author} \hyperref[TEI.editor]{editor} \hyperref[TEI.meeting]{meeting} \hyperref[TEI.respStmt]{respStmt} \hyperref[TEI.title]{title}\par 
    \item[header: ]
   \hyperref[TEI.funder]{funder}
    \item[{Exemple}]
  \leavevmode\bgroup\exampleFont \begin{shaded}\noindent\mbox{}{<\textbf{titleStmt}>}\mbox{}\newline 
\hspace*{6pt}{<\textbf{title}>}Le sanctoral du lectionnaire de l'office dominicain (1254-1256){</\textbf{title}>}\mbox{}\newline 
\hspace*{6pt}{<\textbf{funder}>}2008—..., École nationale des chartes{</\textbf{funder}>}\mbox{}\newline 
\hspace*{6pt}{<\textbf{principal}>}Anne-Élisabeth Urfels-Capot{</\textbf{principal}>}\mbox{}\newline 
\hspace*{6pt}{<\textbf{respStmt}>}\mbox{}\newline 
\hspace*{6pt}\hspace*{6pt}{<\textbf{resp}>}responsable des publications{</\textbf{resp}>}\mbox{}\newline 
\hspace*{6pt}\hspace*{6pt}{<\textbf{name}>}Olivier Canteaut (École nationale des chartes){</\textbf{name}>}\mbox{}\newline 
\hspace*{6pt}{</\textbf{respStmt}>}\mbox{}\newline 
\hspace*{6pt}{<\textbf{respStmt}>}\mbox{}\newline 
\hspace*{6pt}\hspace*{6pt}{<\textbf{resp}>} 2009—..., Éditeur électronique : du TEI à l'écran{</\textbf{resp}>}\mbox{}\newline 
\hspace*{6pt}\hspace*{6pt}{<\textbf{name}\hspace*{6pt}{ref}="{vincent.jolivet@enc.sorbonne.fr}">}Vincent Jolivet (École nationale\mbox{}\newline 
\hspace*{6pt}\hspace*{6pt}\hspace*{6pt}\hspace*{6pt} des chartes){</\textbf{name}>}\mbox{}\newline 
\hspace*{6pt}{</\textbf{respStmt}>}\mbox{}\newline 
\hspace*{6pt}{<\textbf{respStmt}>}\mbox{}\newline 
\hspace*{6pt}\hspace*{6pt}{<\textbf{resp}>}2009, Éditeur scientifique{</\textbf{resp}>}\mbox{}\newline 
\hspace*{6pt}\hspace*{6pt}{<\textbf{name}\hspace*{6pt}{ref}="{pascale.bourgain@enc.sorbonne.fr}">}Pascale Bourgain (École\mbox{}\newline 
\hspace*{6pt}\hspace*{6pt}\hspace*{6pt}\hspace*{6pt} nationale des chartes){</\textbf{name}>}\mbox{}\newline 
\hspace*{6pt}{</\textbf{respStmt}>}\mbox{}\newline 
\hspace*{6pt}{<\textbf{respStmt}>}\mbox{}\newline 
\hspace*{6pt}\hspace*{6pt}{<\textbf{resp}>}2008, Conversion du document bureautique vers TEI{</\textbf{resp}>}\mbox{}\newline 
\hspace*{6pt}\hspace*{6pt}{<\textbf{name}\hspace*{6pt}{ref}="{frederic.glorieux@enc.sorbonne.fr}">}Frédéric Glorieux (École\mbox{}\newline 
\hspace*{6pt}\hspace*{6pt}\hspace*{6pt}\hspace*{6pt} nationale des chartes){</\textbf{name}>}\mbox{}\newline 
\hspace*{6pt}{</\textbf{respStmt}>}\mbox{}\newline 
{</\textbf{titleStmt}>}\end{shaded}\egroup 


    \item[{Modèle de contenu}]
  \mbox{}\hfill\\[-10pt]\begin{Verbatim}[fontsize=\small]
<content>
 <sequence maxOccurs="1" minOccurs="1">
  <elementRef key="title"
   maxOccurs="unbounded" minOccurs="1"/>
  <classRef key="model.respLike"
   maxOccurs="unbounded" minOccurs="0"/>
 </sequence>
</content>
    
\end{Verbatim}

    \item[{Schéma Declaration}]
  \mbox{}\hfill\\[-10pt]\begin{Verbatim}[fontsize=\small]
element titleStmt
{
   tei_att.global.attributes,
   ( tei_title+, tei_model.respLike* )
}
\end{Verbatim}

\end{reflist}  \index{transpose=<transpose>|oddindex}
\begin{reflist}
\item[]\begin{specHead}{TEI.transpose}{<transpose> }describes a single textual transposition as an ordered list of at least two pointers specifying the order in which the elements indicated should be re-combined. [\xref{http://www.tei-c.org/release/doc/tei-p5-doc/en/html/PH.html\#transpo}{11.3.4.5. Transpositions}]\end{specHead} 
    \item[{Module}]
  transcr
    \item[{Attributs}]
  Attributs \hyperref[TEI.att.global]{att.global} (\textit{@xml:id}, \textit{@n}, \textit{@xml:lang}, \textit{@xml:base}, \textit{@xml:space})  (\hyperref[TEI.att.global.rendition]{att.global.rendition} (\textit{@rend}, \textit{@style}, \textit{@rendition})) (\hyperref[TEI.att.global.linking]{att.global.linking} (\textit{@corresp}, \textit{@synch}, \textit{@sameAs}, \textit{@copyOf}, \textit{@next}, \textit{@prev}, \textit{@exclude}, \textit{@select})) (\hyperref[TEI.att.global.analytic]{att.global.analytic} (\textit{@ana})) (\hyperref[TEI.att.global.facs]{att.global.facs} (\textit{@facs})) (\hyperref[TEI.att.global.change]{att.global.change} (\textit{@change})) (\hyperref[TEI.att.global.responsibility]{att.global.responsibility} (\textit{@cert}, \textit{@resp})) (\hyperref[TEI.att.global.source]{att.global.source} (\textit{@source}))
    \item[{Contenu dans}]
  
    \item[transcr: ]
   \hyperref[TEI.listTranspose]{listTranspose}
    \item[{Peut contenir}]
  
    \item[core: ]
   \hyperref[TEI.ptr]{ptr}
    \item[{Note}]
  \par
Transposition is usually indicated in a document by a metamark such as a wavy line or numbering. \par
The order in which \hyperref[TEI.ptr]{<ptr>} elements appear within a \hyperref[TEI.transpose]{<transpose>} element should correspond with the desired order, as indicated by the metamark.
    \item[{Exemple}]
  \leavevmode\bgroup\exampleFont \begin{shaded}\noindent\mbox{}{<\textbf{transpose}>}\mbox{}\newline 
\hspace*{6pt}{<\textbf{ptr}\hspace*{6pt}{target}="{\#ib02}"/>}\mbox{}\newline 
\hspace*{6pt}{<\textbf{ptr}\hspace*{6pt}{target}="{\#ib01}"/>}\mbox{}\newline 
{</\textbf{transpose}>}\end{shaded}\egroup 

The transposition recorded here indicates that the content of the element with identifier \texttt{ib02} should appear before the content of the element with identifier \texttt{ib01}.
    \item[{Modèle de contenu}]
  \mbox{}\hfill\\[-10pt]\begin{Verbatim}[fontsize=\small]
<content>
 <sequence maxOccurs="1" minOccurs="1">
  <elementRef key="ptr"/>
  <elementRef key="ptr"
   maxOccurs="unbounded" minOccurs="1"/>
 </sequence>
</content>
    
\end{Verbatim}

    \item[{Schéma Declaration}]
  \mbox{}\hfill\\[-10pt]\begin{Verbatim}[fontsize=\small]
element transpose { tei_att.global.attributes, ( tei_ptr, tei_ptr+ ) }
\end{Verbatim}

\end{reflist}  \index{typeDesc=<typeDesc>|oddindex}
\begin{reflist}
\item[]\begin{specHead}{TEI.typeDesc}{<typeDesc> }(description des styles de caractère) contient la description des styles de caractères ou d'autres aspects de l'impression d'un incunable ou d'une autre source imprimée. [\xref{http://www.tei-c.org/release/doc/tei-p5-doc/en/html/MS.html\#msphwr}{10.7.2.1. Writing}]\end{specHead} 
    \item[{Module}]
  msdescription
    \item[{Attributs}]
  Attributs \hyperref[TEI.att.global]{att.global} (\textit{@xml:id}, \textit{@n}, \textit{@xml:lang}, \textit{@xml:base}, \textit{@xml:space})  (\hyperref[TEI.att.global.rendition]{att.global.rendition} (\textit{@rend}, \textit{@style}, \textit{@rendition})) (\hyperref[TEI.att.global.linking]{att.global.linking} (\textit{@corresp}, \textit{@synch}, \textit{@sameAs}, \textit{@copyOf}, \textit{@next}, \textit{@prev}, \textit{@exclude}, \textit{@select})) (\hyperref[TEI.att.global.analytic]{att.global.analytic} (\textit{@ana})) (\hyperref[TEI.att.global.facs]{att.global.facs} (\textit{@facs})) (\hyperref[TEI.att.global.change]{att.global.change} (\textit{@change})) (\hyperref[TEI.att.global.responsibility]{att.global.responsibility} (\textit{@cert}, \textit{@resp})) (\hyperref[TEI.att.global.source]{att.global.source} (\textit{@source}))
    \item[{Membre du}]
  \hyperref[TEI.model.physDescPart]{model.physDescPart}
    \item[{Contenu dans}]
  
    \item[msdescription: ]
   \hyperref[TEI.physDesc]{physDesc}
    \item[{Peut contenir}]
  
    \item[core: ]
   \hyperref[TEI.p]{p}\par 
    \item[linking: ]
   \hyperref[TEI.ab]{ab}\par 
    \item[msdescription: ]
   \hyperref[TEI.summary]{summary} \hyperref[TEI.typeNote]{typeNote}
    \item[{Exemple}]
  \leavevmode\bgroup\exampleFont \begin{shaded}\noindent\mbox{}{<\textbf{typeDesc}>}\mbox{}\newline 
\hspace*{6pt}{<\textbf{p}>}Uses an unidentified black letter font, probably from the\mbox{}\newline 
\hspace*{6pt}\hspace*{6pt} 15th century{</\textbf{p}>}\mbox{}\newline 
{</\textbf{typeDesc}>}\end{shaded}\egroup 


    \item[{Exemple}]
  \leavevmode\bgroup\exampleFont \begin{shaded}\noindent\mbox{}{<\textbf{typeDesc}>}\mbox{}\newline 
\hspace*{6pt}{<\textbf{summary}>}Contains a mixture of blackletter and Roman (antiqua) typefaces{</\textbf{summary}>}\mbox{}\newline 
\hspace*{6pt}{<\textbf{typeNote}\hspace*{6pt}{xml:id}="{Frak1}">}Blackletter face, showing\mbox{}\newline 
\hspace*{6pt}\hspace*{6pt} similarities to those produced in Wuerzburg after 1470.{</\textbf{typeNote}>}\mbox{}\newline 
\hspace*{6pt}{<\textbf{typeNote}\hspace*{6pt}{xml:id}="{Rom1}">}Roman face of Venetian origins.{</\textbf{typeNote}>}\mbox{}\newline 
{</\textbf{typeDesc}>}\end{shaded}\egroup 


    \item[{Modèle de contenu}]
  \mbox{}\hfill\\[-10pt]\begin{Verbatim}[fontsize=\small]
<content>
 <alternate maxOccurs="1" minOccurs="1">
  <classRef key="model.pLike"
   maxOccurs="unbounded" minOccurs="1"/>
  <sequence maxOccurs="1" minOccurs="1">
   <elementRef key="summary" minOccurs="0"/>
   <elementRef key="typeNote"
    maxOccurs="unbounded" minOccurs="1"/>
  </sequence>
 </alternate>
</content>
    
\end{Verbatim}

    \item[{Schéma Declaration}]
  \mbox{}\hfill\\[-10pt]\begin{Verbatim}[fontsize=\small]
element typeDesc
{
   tei_att.global.attributes,
   ( tei_model.pLike+ | ( tei_summary?, tei_typeNote+ ) )
}
\end{Verbatim}

\end{reflist}  \index{typeNote=<typeNote>|oddindex}
\begin{reflist}
\item[]\begin{specHead}{TEI.typeNote}{<typeNote> }(note sur les caractères typographiques.) décrit une police particulière ou un autre trait typographique significatif que l’on note dans la description d'une ressource imprimée. [\xref{http://www.tei-c.org/release/doc/tei-p5-doc/en/html/MS.html\#msph2}{10.7.2. Writing, Decoration, and Other Notations}]\end{specHead} 
    \item[{Module}]
  msdescription
    \item[{Attributs}]
  Attributs \hyperref[TEI.att.global]{att.global} (\textit{@xml:id}, \textit{@n}, \textit{@xml:lang}, \textit{@xml:base}, \textit{@xml:space})  (\hyperref[TEI.att.global.rendition]{att.global.rendition} (\textit{@rend}, \textit{@style}, \textit{@rendition})) (\hyperref[TEI.att.global.linking]{att.global.linking} (\textit{@corresp}, \textit{@synch}, \textit{@sameAs}, \textit{@copyOf}, \textit{@next}, \textit{@prev}, \textit{@exclude}, \textit{@select})) (\hyperref[TEI.att.global.analytic]{att.global.analytic} (\textit{@ana})) (\hyperref[TEI.att.global.facs]{att.global.facs} (\textit{@facs})) (\hyperref[TEI.att.global.change]{att.global.change} (\textit{@change})) (\hyperref[TEI.att.global.responsibility]{att.global.responsibility} (\textit{@cert}, \textit{@resp})) (\hyperref[TEI.att.global.source]{att.global.source} (\textit{@source})) \hyperref[TEI.att.handFeatures]{att.handFeatures} (\textit{@scribe}, \textit{@scribeRef}, \textit{@script}, \textit{@scriptRef}, \textit{@medium}, \textit{@scope}) 
    \item[{Contenu dans}]
  
    \item[msdescription: ]
   \hyperref[TEI.typeDesc]{typeDesc}
    \item[{Peut contenir}]
  
    \item[analysis: ]
   \hyperref[TEI.c]{c} \hyperref[TEI.cl]{cl} \hyperref[TEI.interp]{interp} \hyperref[TEI.interpGrp]{interpGrp} \hyperref[TEI.m]{m} \hyperref[TEI.pc]{pc} \hyperref[TEI.phr]{phr} \hyperref[TEI.s]{s} \hyperref[TEI.span]{span} \hyperref[TEI.spanGrp]{spanGrp} \hyperref[TEI.w]{w}\par 
    \item[core: ]
   \hyperref[TEI.abbr]{abbr} \hyperref[TEI.add]{add} \hyperref[TEI.address]{address} \hyperref[TEI.bibl]{bibl} \hyperref[TEI.biblStruct]{biblStruct} \hyperref[TEI.binaryObject]{binaryObject} \hyperref[TEI.cb]{cb} \hyperref[TEI.choice]{choice} \hyperref[TEI.cit]{cit} \hyperref[TEI.corr]{corr} \hyperref[TEI.date]{date} \hyperref[TEI.del]{del} \hyperref[TEI.desc]{desc} \hyperref[TEI.distinct]{distinct} \hyperref[TEI.email]{email} \hyperref[TEI.emph]{emph} \hyperref[TEI.expan]{expan} \hyperref[TEI.foreign]{foreign} \hyperref[TEI.gap]{gap} \hyperref[TEI.gb]{gb} \hyperref[TEI.gloss]{gloss} \hyperref[TEI.graphic]{graphic} \hyperref[TEI.hi]{hi} \hyperref[TEI.index]{index} \hyperref[TEI.l]{l} \hyperref[TEI.label]{label} \hyperref[TEI.lb]{lb} \hyperref[TEI.lg]{lg} \hyperref[TEI.list]{list} \hyperref[TEI.listBibl]{listBibl} \hyperref[TEI.measure]{measure} \hyperref[TEI.measureGrp]{measureGrp} \hyperref[TEI.media]{media} \hyperref[TEI.mentioned]{mentioned} \hyperref[TEI.milestone]{milestone} \hyperref[TEI.name]{name} \hyperref[TEI.note]{note} \hyperref[TEI.num]{num} \hyperref[TEI.orig]{orig} \hyperref[TEI.p]{p} \hyperref[TEI.pb]{pb} \hyperref[TEI.ptr]{ptr} \hyperref[TEI.q]{q} \hyperref[TEI.quote]{quote} \hyperref[TEI.ref]{ref} \hyperref[TEI.reg]{reg} \hyperref[TEI.rs]{rs} \hyperref[TEI.said]{said} \hyperref[TEI.sic]{sic} \hyperref[TEI.soCalled]{soCalled} \hyperref[TEI.sp]{sp} \hyperref[TEI.stage]{stage} \hyperref[TEI.term]{term} \hyperref[TEI.time]{time} \hyperref[TEI.title]{title} \hyperref[TEI.unclear]{unclear}\par 
    \item[derived-module-tei.istex: ]
   \hyperref[TEI.math]{math} \hyperref[TEI.mrow]{mrow}\par 
    \item[figures: ]
   \hyperref[TEI.figure]{figure} \hyperref[TEI.formula]{formula} \hyperref[TEI.notatedMusic]{notatedMusic} \hyperref[TEI.table]{table}\par 
    \item[header: ]
   \hyperref[TEI.biblFull]{biblFull} \hyperref[TEI.idno]{idno}\par 
    \item[iso-fs: ]
   \hyperref[TEI.fLib]{fLib} \hyperref[TEI.fs]{fs} \hyperref[TEI.fvLib]{fvLib}\par 
    \item[linking: ]
   \hyperref[TEI.ab]{ab} \hyperref[TEI.alt]{alt} \hyperref[TEI.altGrp]{altGrp} \hyperref[TEI.anchor]{anchor} \hyperref[TEI.join]{join} \hyperref[TEI.joinGrp]{joinGrp} \hyperref[TEI.link]{link} \hyperref[TEI.linkGrp]{linkGrp} \hyperref[TEI.seg]{seg} \hyperref[TEI.timeline]{timeline}\par 
    \item[msdescription: ]
   \hyperref[TEI.catchwords]{catchwords} \hyperref[TEI.depth]{depth} \hyperref[TEI.dim]{dim} \hyperref[TEI.dimensions]{dimensions} \hyperref[TEI.height]{height} \hyperref[TEI.heraldry]{heraldry} \hyperref[TEI.locus]{locus} \hyperref[TEI.locusGrp]{locusGrp} \hyperref[TEI.material]{material} \hyperref[TEI.msDesc]{msDesc} \hyperref[TEI.objectType]{objectType} \hyperref[TEI.origDate]{origDate} \hyperref[TEI.origPlace]{origPlace} \hyperref[TEI.secFol]{secFol} \hyperref[TEI.signatures]{signatures} \hyperref[TEI.source]{source} \hyperref[TEI.stamp]{stamp} \hyperref[TEI.watermark]{watermark} \hyperref[TEI.width]{width}\par 
    \item[namesdates: ]
   \hyperref[TEI.addName]{addName} \hyperref[TEI.affiliation]{affiliation} \hyperref[TEI.country]{country} \hyperref[TEI.forename]{forename} \hyperref[TEI.genName]{genName} \hyperref[TEI.geogName]{geogName} \hyperref[TEI.listOrg]{listOrg} \hyperref[TEI.listPlace]{listPlace} \hyperref[TEI.location]{location} \hyperref[TEI.nameLink]{nameLink} \hyperref[TEI.orgName]{orgName} \hyperref[TEI.persName]{persName} \hyperref[TEI.placeName]{placeName} \hyperref[TEI.region]{region} \hyperref[TEI.roleName]{roleName} \hyperref[TEI.settlement]{settlement} \hyperref[TEI.state]{state} \hyperref[TEI.surname]{surname}\par 
    \item[spoken: ]
   \hyperref[TEI.annotationBlock]{annotationBlock}\par 
    \item[textstructure: ]
   \hyperref[TEI.floatingText]{floatingText}\par 
    \item[transcr: ]
   \hyperref[TEI.addSpan]{addSpan} \hyperref[TEI.am]{am} \hyperref[TEI.damage]{damage} \hyperref[TEI.damageSpan]{damageSpan} \hyperref[TEI.delSpan]{delSpan} \hyperref[TEI.ex]{ex} \hyperref[TEI.fw]{fw} \hyperref[TEI.handShift]{handShift} \hyperref[TEI.listTranspose]{listTranspose} \hyperref[TEI.metamark]{metamark} \hyperref[TEI.mod]{mod} \hyperref[TEI.redo]{redo} \hyperref[TEI.restore]{restore} \hyperref[TEI.retrace]{retrace} \hyperref[TEI.secl]{secl} \hyperref[TEI.space]{space} \hyperref[TEI.subst]{subst} \hyperref[TEI.substJoin]{substJoin} \hyperref[TEI.supplied]{supplied} \hyperref[TEI.surplus]{surplus} \hyperref[TEI.undo]{undo}\par des données textuelles
    \item[{Exemple}]
  \leavevmode\bgroup\exampleFont \begin{shaded}\noindent\mbox{}{<\textbf{typeNote}\hspace*{6pt}{scope}="{sole}">} Printed in an Antiqua typeface showing strong Italianate influence.\mbox{}\newline 
{</\textbf{typeNote}>}\end{shaded}\egroup 


    \item[{Modèle de contenu}]
  \mbox{}\hfill\\[-10pt]\begin{Verbatim}[fontsize=\small]
<content>
 <macroRef key="macro.specialPara"/>
</content>
    
\end{Verbatim}

    \item[{Schéma Declaration}]
  \mbox{}\hfill\\[-10pt]\begin{Verbatim}[fontsize=\small]
element typeNote
{
   tei_att.global.attributes,
   tei_att.handFeatures.attributes,
   tei_macro.specialPara}
\end{Verbatim}

\end{reflist}  \index{unclear=<unclear>|oddindex}\index{reason=@reason!<unclear>|oddindex}\index{hand=@hand!<unclear>|oddindex}\index{agent=@agent!<unclear>|oddindex}
\begin{reflist}
\item[]\begin{specHead}{TEI.unclear}{<unclear> }(incertain) contient un mot, une expression ou bien un passage qui ne peut être transcrit avec certitude parce qu'il est illisible ou inaudible dans la source. [\xref{http://www.tei-c.org/release/doc/tei-p5-doc/en/html/PH.html\#PHDA}{11.3.3.1. Damage, Illegibility, and Supplied Text} \xref{http://www.tei-c.org/release/doc/tei-p5-doc/en/html/CO.html\#COEDADD}{3.4.3. Additions, Deletions, and Omissions}]\end{specHead} 
    \item[{Module}]
  core
    \item[{Attributs}]
  Attributs \hyperref[TEI.att.global]{att.global} (\textit{@xml:id}, \textit{@n}, \textit{@xml:lang}, \textit{@xml:base}, \textit{@xml:space})  (\hyperref[TEI.att.global.rendition]{att.global.rendition} (\textit{@rend}, \textit{@style}, \textit{@rendition})) (\hyperref[TEI.att.global.linking]{att.global.linking} (\textit{@corresp}, \textit{@synch}, \textit{@sameAs}, \textit{@copyOf}, \textit{@next}, \textit{@prev}, \textit{@exclude}, \textit{@select})) (\hyperref[TEI.att.global.analytic]{att.global.analytic} (\textit{@ana})) (\hyperref[TEI.att.global.facs]{att.global.facs} (\textit{@facs})) (\hyperref[TEI.att.global.change]{att.global.change} (\textit{@change})) (\hyperref[TEI.att.global.responsibility]{att.global.responsibility} (\textit{@cert}, \textit{@resp})) (\hyperref[TEI.att.global.source]{att.global.source} (\textit{@source})) \hyperref[TEI.att.editLike]{att.editLike} (\textit{@evidence}, \textit{@instant})  (\hyperref[TEI.att.dimensions]{att.dimensions} (\textit{@unit}, \textit{@quantity}, \textit{@extent}, \textit{@precision}, \textit{@scope}) (\hyperref[TEI.att.ranging]{att.ranging} (\textit{@atLeast}, \textit{@atMost}, \textit{@min}, \textit{@max}, \textit{@confidence})) ) \hfil\\[-10pt]\begin{sansreflist}
    \item[@reason]
  indique pourquoi il est difficile de transcrire le document
\begin{reflist}
    \item[{Statut}]
  Optionel
    \item[{Type de données}]
  1–∞ occurrences de \hyperref[TEI.teidata.word]{teidata.word} séparé par un espace
    \item[]\exampleFont {<\textbf{div}>}\mbox{}\newline 
\hspace*{6pt}{<\textbf{head}>}Rx{</\textbf{head}>}\mbox{}\newline 
\hspace*{6pt}{<\textbf{p}>}500 mg {<\textbf{unclear}\hspace*{6pt}{reason}="{illegible}">}placebo{</\textbf{unclear}>}\mbox{}\newline 
\hspace*{6pt}{</\textbf{p}>}\mbox{}\newline 
{</\textbf{div}>}
\end{reflist}  
    \item[@hand]
  lorsque la difficulté de transcription vient d'une action attribuable à une main identifiable (suppression partielle, etc.), indique quelle est cette main
\begin{reflist}
    \item[\xref{http://www.tei-c.org/Activities/Council/Working/tcw27.xml}{Deprecated}]
  will be removed on 2017-08-01
    \item[{Statut}]
  Optionel
    \item[{Type de données}]
  \hyperref[TEI.teidata.pointer]{teidata.pointer}
\end{reflist}  
    \item[@agent]
  lorsque la difficulté de transcription vient d'un dommage, catégorise la cause du dommage si celle-ci peut être identifiée
\begin{reflist}
    \item[{Statut}]
  Optionel
    \item[{Type de données}]
  \hyperref[TEI.teidata.enumerated]{teidata.enumerated}
    \item[{Exemple de valeurs possibles:}]
  \begin{description}

\item[{rubbing}]des dommages résultent du frottement des bords de la feuille
\item[{mildew}]des dégâts résultent de la moisissure sur la surface de la feuille
\item[{smoke}]des dégâts résultent de la fumée
\end{description} 
\end{reflist}  
\end{sansreflist}  
    \item[{Membre du}]
  \hyperref[TEI.model.choicePart]{model.choicePart} \hyperref[TEI.model.linePart]{model.linePart} \hyperref[TEI.model.pPart.transcriptional]{model.pPart.transcriptional}
    \item[{Contenu dans}]
  
    \item[analysis: ]
   \hyperref[TEI.cl]{cl} \hyperref[TEI.pc]{pc} \hyperref[TEI.phr]{phr} \hyperref[TEI.s]{s} \hyperref[TEI.w]{w}\par 
    \item[core: ]
   \hyperref[TEI.abbr]{abbr} \hyperref[TEI.add]{add} \hyperref[TEI.addrLine]{addrLine} \hyperref[TEI.author]{author} \hyperref[TEI.bibl]{bibl} \hyperref[TEI.biblScope]{biblScope} \hyperref[TEI.choice]{choice} \hyperref[TEI.citedRange]{citedRange} \hyperref[TEI.corr]{corr} \hyperref[TEI.date]{date} \hyperref[TEI.del]{del} \hyperref[TEI.distinct]{distinct} \hyperref[TEI.editor]{editor} \hyperref[TEI.email]{email} \hyperref[TEI.emph]{emph} \hyperref[TEI.expan]{expan} \hyperref[TEI.foreign]{foreign} \hyperref[TEI.gloss]{gloss} \hyperref[TEI.head]{head} \hyperref[TEI.headItem]{headItem} \hyperref[TEI.headLabel]{headLabel} \hyperref[TEI.hi]{hi} \hyperref[TEI.item]{item} \hyperref[TEI.l]{l} \hyperref[TEI.label]{label} \hyperref[TEI.measure]{measure} \hyperref[TEI.mentioned]{mentioned} \hyperref[TEI.name]{name} \hyperref[TEI.note]{note} \hyperref[TEI.num]{num} \hyperref[TEI.orig]{orig} \hyperref[TEI.p]{p} \hyperref[TEI.pubPlace]{pubPlace} \hyperref[TEI.publisher]{publisher} \hyperref[TEI.q]{q} \hyperref[TEI.quote]{quote} \hyperref[TEI.ref]{ref} \hyperref[TEI.reg]{reg} \hyperref[TEI.rs]{rs} \hyperref[TEI.said]{said} \hyperref[TEI.sic]{sic} \hyperref[TEI.soCalled]{soCalled} \hyperref[TEI.speaker]{speaker} \hyperref[TEI.stage]{stage} \hyperref[TEI.street]{street} \hyperref[TEI.term]{term} \hyperref[TEI.textLang]{textLang} \hyperref[TEI.time]{time} \hyperref[TEI.title]{title} \hyperref[TEI.unclear]{unclear}\par 
    \item[figures: ]
   \hyperref[TEI.cell]{cell}\par 
    \item[header: ]
   \hyperref[TEI.change]{change} \hyperref[TEI.distributor]{distributor} \hyperref[TEI.edition]{edition} \hyperref[TEI.extent]{extent} \hyperref[TEI.licence]{licence}\par 
    \item[linking: ]
   \hyperref[TEI.ab]{ab} \hyperref[TEI.seg]{seg}\par 
    \item[msdescription: ]
   \hyperref[TEI.accMat]{accMat} \hyperref[TEI.acquisition]{acquisition} \hyperref[TEI.additions]{additions} \hyperref[TEI.catchwords]{catchwords} \hyperref[TEI.collation]{collation} \hyperref[TEI.colophon]{colophon} \hyperref[TEI.condition]{condition} \hyperref[TEI.custEvent]{custEvent} \hyperref[TEI.decoNote]{decoNote} \hyperref[TEI.explicit]{explicit} \hyperref[TEI.filiation]{filiation} \hyperref[TEI.finalRubric]{finalRubric} \hyperref[TEI.foliation]{foliation} \hyperref[TEI.heraldry]{heraldry} \hyperref[TEI.incipit]{incipit} \hyperref[TEI.layout]{layout} \hyperref[TEI.material]{material} \hyperref[TEI.musicNotation]{musicNotation} \hyperref[TEI.objectType]{objectType} \hyperref[TEI.origDate]{origDate} \hyperref[TEI.origPlace]{origPlace} \hyperref[TEI.origin]{origin} \hyperref[TEI.provenance]{provenance} \hyperref[TEI.rubric]{rubric} \hyperref[TEI.secFol]{secFol} \hyperref[TEI.signatures]{signatures} \hyperref[TEI.source]{source} \hyperref[TEI.stamp]{stamp} \hyperref[TEI.summary]{summary} \hyperref[TEI.support]{support} \hyperref[TEI.surrogates]{surrogates} \hyperref[TEI.typeNote]{typeNote} \hyperref[TEI.watermark]{watermark}\par 
    \item[namesdates: ]
   \hyperref[TEI.addName]{addName} \hyperref[TEI.affiliation]{affiliation} \hyperref[TEI.country]{country} \hyperref[TEI.forename]{forename} \hyperref[TEI.genName]{genName} \hyperref[TEI.geogName]{geogName} \hyperref[TEI.nameLink]{nameLink} \hyperref[TEI.orgName]{orgName} \hyperref[TEI.persName]{persName} \hyperref[TEI.placeName]{placeName} \hyperref[TEI.region]{region} \hyperref[TEI.roleName]{roleName} \hyperref[TEI.settlement]{settlement} \hyperref[TEI.surname]{surname}\par 
    \item[textstructure: ]
   \hyperref[TEI.docAuthor]{docAuthor} \hyperref[TEI.docDate]{docDate} \hyperref[TEI.docEdition]{docEdition} \hyperref[TEI.titlePart]{titlePart}\par 
    \item[transcr: ]
   \hyperref[TEI.am]{am} \hyperref[TEI.damage]{damage} \hyperref[TEI.fw]{fw} \hyperref[TEI.line]{line} \hyperref[TEI.metamark]{metamark} \hyperref[TEI.mod]{mod} \hyperref[TEI.restore]{restore} \hyperref[TEI.retrace]{retrace} \hyperref[TEI.secl]{secl} \hyperref[TEI.supplied]{supplied} \hyperref[TEI.surplus]{surplus} \hyperref[TEI.zone]{zone}
    \item[{Peut contenir}]
  
    \item[analysis: ]
   \hyperref[TEI.c]{c} \hyperref[TEI.cl]{cl} \hyperref[TEI.interp]{interp} \hyperref[TEI.interpGrp]{interpGrp} \hyperref[TEI.m]{m} \hyperref[TEI.pc]{pc} \hyperref[TEI.phr]{phr} \hyperref[TEI.s]{s} \hyperref[TEI.span]{span} \hyperref[TEI.spanGrp]{spanGrp} \hyperref[TEI.w]{w}\par 
    \item[core: ]
   \hyperref[TEI.abbr]{abbr} \hyperref[TEI.add]{add} \hyperref[TEI.address]{address} \hyperref[TEI.bibl]{bibl} \hyperref[TEI.biblStruct]{biblStruct} \hyperref[TEI.binaryObject]{binaryObject} \hyperref[TEI.cb]{cb} \hyperref[TEI.choice]{choice} \hyperref[TEI.cit]{cit} \hyperref[TEI.corr]{corr} \hyperref[TEI.date]{date} \hyperref[TEI.del]{del} \hyperref[TEI.desc]{desc} \hyperref[TEI.distinct]{distinct} \hyperref[TEI.email]{email} \hyperref[TEI.emph]{emph} \hyperref[TEI.expan]{expan} \hyperref[TEI.foreign]{foreign} \hyperref[TEI.gap]{gap} \hyperref[TEI.gb]{gb} \hyperref[TEI.gloss]{gloss} \hyperref[TEI.graphic]{graphic} \hyperref[TEI.hi]{hi} \hyperref[TEI.index]{index} \hyperref[TEI.l]{l} \hyperref[TEI.label]{label} \hyperref[TEI.lb]{lb} \hyperref[TEI.lg]{lg} \hyperref[TEI.list]{list} \hyperref[TEI.listBibl]{listBibl} \hyperref[TEI.measure]{measure} \hyperref[TEI.measureGrp]{measureGrp} \hyperref[TEI.media]{media} \hyperref[TEI.mentioned]{mentioned} \hyperref[TEI.milestone]{milestone} \hyperref[TEI.name]{name} \hyperref[TEI.note]{note} \hyperref[TEI.num]{num} \hyperref[TEI.orig]{orig} \hyperref[TEI.pb]{pb} \hyperref[TEI.ptr]{ptr} \hyperref[TEI.q]{q} \hyperref[TEI.quote]{quote} \hyperref[TEI.ref]{ref} \hyperref[TEI.reg]{reg} \hyperref[TEI.rs]{rs} \hyperref[TEI.said]{said} \hyperref[TEI.sic]{sic} \hyperref[TEI.soCalled]{soCalled} \hyperref[TEI.stage]{stage} \hyperref[TEI.term]{term} \hyperref[TEI.time]{time} \hyperref[TEI.title]{title} \hyperref[TEI.unclear]{unclear}\par 
    \item[derived-module-tei.istex: ]
   \hyperref[TEI.math]{math} \hyperref[TEI.mrow]{mrow}\par 
    \item[figures: ]
   \hyperref[TEI.figure]{figure} \hyperref[TEI.formula]{formula} \hyperref[TEI.notatedMusic]{notatedMusic} \hyperref[TEI.table]{table}\par 
    \item[header: ]
   \hyperref[TEI.biblFull]{biblFull} \hyperref[TEI.idno]{idno}\par 
    \item[iso-fs: ]
   \hyperref[TEI.fLib]{fLib} \hyperref[TEI.fs]{fs} \hyperref[TEI.fvLib]{fvLib}\par 
    \item[linking: ]
   \hyperref[TEI.alt]{alt} \hyperref[TEI.altGrp]{altGrp} \hyperref[TEI.anchor]{anchor} \hyperref[TEI.join]{join} \hyperref[TEI.joinGrp]{joinGrp} \hyperref[TEI.link]{link} \hyperref[TEI.linkGrp]{linkGrp} \hyperref[TEI.seg]{seg} \hyperref[TEI.timeline]{timeline}\par 
    \item[msdescription: ]
   \hyperref[TEI.catchwords]{catchwords} \hyperref[TEI.depth]{depth} \hyperref[TEI.dim]{dim} \hyperref[TEI.dimensions]{dimensions} \hyperref[TEI.height]{height} \hyperref[TEI.heraldry]{heraldry} \hyperref[TEI.locus]{locus} \hyperref[TEI.locusGrp]{locusGrp} \hyperref[TEI.material]{material} \hyperref[TEI.msDesc]{msDesc} \hyperref[TEI.objectType]{objectType} \hyperref[TEI.origDate]{origDate} \hyperref[TEI.origPlace]{origPlace} \hyperref[TEI.secFol]{secFol} \hyperref[TEI.signatures]{signatures} \hyperref[TEI.source]{source} \hyperref[TEI.stamp]{stamp} \hyperref[TEI.watermark]{watermark} \hyperref[TEI.width]{width}\par 
    \item[namesdates: ]
   \hyperref[TEI.addName]{addName} \hyperref[TEI.affiliation]{affiliation} \hyperref[TEI.country]{country} \hyperref[TEI.forename]{forename} \hyperref[TEI.genName]{genName} \hyperref[TEI.geogName]{geogName} \hyperref[TEI.listOrg]{listOrg} \hyperref[TEI.listPlace]{listPlace} \hyperref[TEI.location]{location} \hyperref[TEI.nameLink]{nameLink} \hyperref[TEI.orgName]{orgName} \hyperref[TEI.persName]{persName} \hyperref[TEI.placeName]{placeName} \hyperref[TEI.region]{region} \hyperref[TEI.roleName]{roleName} \hyperref[TEI.settlement]{settlement} \hyperref[TEI.state]{state} \hyperref[TEI.surname]{surname}\par 
    \item[spoken: ]
   \hyperref[TEI.annotationBlock]{annotationBlock}\par 
    \item[textstructure: ]
   \hyperref[TEI.floatingText]{floatingText}\par 
    \item[transcr: ]
   \hyperref[TEI.addSpan]{addSpan} \hyperref[TEI.am]{am} \hyperref[TEI.damage]{damage} \hyperref[TEI.damageSpan]{damageSpan} \hyperref[TEI.delSpan]{delSpan} \hyperref[TEI.ex]{ex} \hyperref[TEI.fw]{fw} \hyperref[TEI.handShift]{handShift} \hyperref[TEI.listTranspose]{listTranspose} \hyperref[TEI.metamark]{metamark} \hyperref[TEI.mod]{mod} \hyperref[TEI.redo]{redo} \hyperref[TEI.restore]{restore} \hyperref[TEI.retrace]{retrace} \hyperref[TEI.secl]{secl} \hyperref[TEI.space]{space} \hyperref[TEI.subst]{subst} \hyperref[TEI.substJoin]{substJoin} \hyperref[TEI.supplied]{supplied} \hyperref[TEI.surplus]{surplus} \hyperref[TEI.undo]{undo}\par des données textuelles
    \item[{Note}]
  \par
Le même élément est utilisé pour tous les cas d'incertitude portant sur la transcription du contenu d'éléments, qu'il s'agisse de documents écrits ou oraux. Pour les autres aspects concernant la certitude, l'incertitude, et la fiabilité du balisage et de la transcription, voir le chapitre \xref{http://www.tei-c.org/release/doc/tei-p5-doc/en/html/CE.html\#CE}{21. Certainty, Precision, and Responsibility}.\par
Les éléments \hyperref[TEI.damage]{<damage>}, \hyperref[TEI.gap]{<gap>}, \hyperref[TEI.del]{<del>}, \hyperref[TEI.unclear]{<unclear>} et \hyperref[TEI.supplied]{<supplied>} peuvent être utilisés en étroite conjonction. Voir la section pour plus de détails sur l'élément le plus pertinent en fonction des circonstances.
    \item[{Exemple}]
  \leavevmode\bgroup\exampleFont \begin{shaded}\noindent\mbox{}{<\textbf{add}\hspace*{6pt}{place}="{inspace}">}Envoyez-moi une épreuve {<\textbf{unclear}\hspace*{6pt}{cert}="{medium}">}W{</\textbf{unclear}>}\mbox{}\newline 
\hspace*{6pt}{<\textbf{gap}\hspace*{6pt}{reason}="{inDéchiffrable}"/>}\mbox{}\newline 
{</\textbf{add}>}\end{shaded}\egroup 


    \item[{Exemple}]
  \leavevmode\bgroup\exampleFont \begin{shaded}\noindent\mbox{}and from time to time invited in like manner\mbox{}\newline 
 his att{<\textbf{unclear}>}ention{</\textbf{unclear}>}\end{shaded}\egroup 

Dans ce cas, les lettres à la fin du mot sont difficiles à lire.
    \item[{Modèle de contenu}]
  \mbox{}\hfill\\[-10pt]\begin{Verbatim}[fontsize=\small]
<content>
 <macroRef key="macro.paraContent"/>
</content>
    
\end{Verbatim}

    \item[{Schéma Declaration}]
  \mbox{}\hfill\\[-10pt]\begin{Verbatim}[fontsize=\small]
element unclear
{
   tei_att.global.attributes,
   tei_att.editLike.attributes,
   attribute reason { list { + } }?,
   attribute hand { text }?,
   attribute agent { text }?,
   tei_macro.paraContent}
\end{Verbatim}

\end{reflist}  \index{undo=<undo>|oddindex}\index{target=@target!<undo>|oddindex}
\begin{reflist}
\item[]\begin{specHead}{TEI.undo}{<undo> }indicates one or more marked-up interventions in a document which have subsequently been marked for cancellation. [\xref{http://www.tei-c.org/release/doc/tei-p5-doc/en/html/PH.html\#undo}{11.3.4.4. Confirmation, Cancellation, and Reinstatement of Modifications}]\end{specHead} 
    \item[{Module}]
  transcr
    \item[{Attributs}]
  Attributs \hyperref[TEI.att.global]{att.global} (\textit{@xml:id}, \textit{@n}, \textit{@xml:lang}, \textit{@xml:base}, \textit{@xml:space})  (\hyperref[TEI.att.global.rendition]{att.global.rendition} (\textit{@rend}, \textit{@style}, \textit{@rendition})) (\hyperref[TEI.att.global.linking]{att.global.linking} (\textit{@corresp}, \textit{@synch}, \textit{@sameAs}, \textit{@copyOf}, \textit{@next}, \textit{@prev}, \textit{@exclude}, \textit{@select})) (\hyperref[TEI.att.global.analytic]{att.global.analytic} (\textit{@ana})) (\hyperref[TEI.att.global.facs]{att.global.facs} (\textit{@facs})) (\hyperref[TEI.att.global.change]{att.global.change} (\textit{@change})) (\hyperref[TEI.att.global.responsibility]{att.global.responsibility} (\textit{@cert}, \textit{@resp})) (\hyperref[TEI.att.global.source]{att.global.source} (\textit{@source})) \hyperref[TEI.att.spanning]{att.spanning} (\textit{@spanTo}) \hyperref[TEI.att.transcriptional]{att.transcriptional} (\textit{@status}, \textit{@cause}, \textit{@seq})  (\hyperref[TEI.att.editLike]{att.editLike} (\textit{@evidence}, \textit{@instant}) (\hyperref[TEI.att.dimensions]{att.dimensions} (\textit{@unit}, \textit{@quantity}, \textit{@extent}, \textit{@precision}, \textit{@scope}) (\hyperref[TEI.att.ranging]{att.ranging} (\textit{@atLeast}, \textit{@atMost}, \textit{@min}, \textit{@max}, \textit{@confidence})) ) ) (\hyperref[TEI.att.written]{att.written} (\textit{@hand})) \hfil\\[-10pt]\begin{sansreflist}
    \item[@target]
  points to one or more elements representing the interventions which are to be reverted or undone.
\begin{reflist}
    \item[{Statut}]
  Optionel
    \item[{Type de données}]
  1–∞ occurrences de \hyperref[TEI.teidata.pointer]{teidata.pointer} séparé par un espace
\end{reflist}  
\end{sansreflist}  
    \item[{Membre du}]
  \hyperref[TEI.model.linePart]{model.linePart} \hyperref[TEI.model.pPart.transcriptional]{model.pPart.transcriptional}
    \item[{Contenu dans}]
  
    \item[analysis: ]
   \hyperref[TEI.cl]{cl} \hyperref[TEI.pc]{pc} \hyperref[TEI.phr]{phr} \hyperref[TEI.s]{s} \hyperref[TEI.w]{w}\par 
    \item[core: ]
   \hyperref[TEI.abbr]{abbr} \hyperref[TEI.add]{add} \hyperref[TEI.addrLine]{addrLine} \hyperref[TEI.author]{author} \hyperref[TEI.bibl]{bibl} \hyperref[TEI.biblScope]{biblScope} \hyperref[TEI.citedRange]{citedRange} \hyperref[TEI.corr]{corr} \hyperref[TEI.date]{date} \hyperref[TEI.del]{del} \hyperref[TEI.distinct]{distinct} \hyperref[TEI.editor]{editor} \hyperref[TEI.email]{email} \hyperref[TEI.emph]{emph} \hyperref[TEI.expan]{expan} \hyperref[TEI.foreign]{foreign} \hyperref[TEI.gloss]{gloss} \hyperref[TEI.head]{head} \hyperref[TEI.headItem]{headItem} \hyperref[TEI.headLabel]{headLabel} \hyperref[TEI.hi]{hi} \hyperref[TEI.item]{item} \hyperref[TEI.l]{l} \hyperref[TEI.label]{label} \hyperref[TEI.measure]{measure} \hyperref[TEI.mentioned]{mentioned} \hyperref[TEI.name]{name} \hyperref[TEI.note]{note} \hyperref[TEI.num]{num} \hyperref[TEI.orig]{orig} \hyperref[TEI.p]{p} \hyperref[TEI.pubPlace]{pubPlace} \hyperref[TEI.publisher]{publisher} \hyperref[TEI.q]{q} \hyperref[TEI.quote]{quote} \hyperref[TEI.ref]{ref} \hyperref[TEI.reg]{reg} \hyperref[TEI.rs]{rs} \hyperref[TEI.said]{said} \hyperref[TEI.sic]{sic} \hyperref[TEI.soCalled]{soCalled} \hyperref[TEI.speaker]{speaker} \hyperref[TEI.stage]{stage} \hyperref[TEI.street]{street} \hyperref[TEI.term]{term} \hyperref[TEI.textLang]{textLang} \hyperref[TEI.time]{time} \hyperref[TEI.title]{title} \hyperref[TEI.unclear]{unclear}\par 
    \item[figures: ]
   \hyperref[TEI.cell]{cell}\par 
    \item[header: ]
   \hyperref[TEI.change]{change} \hyperref[TEI.distributor]{distributor} \hyperref[TEI.edition]{edition} \hyperref[TEI.extent]{extent} \hyperref[TEI.licence]{licence}\par 
    \item[linking: ]
   \hyperref[TEI.ab]{ab} \hyperref[TEI.seg]{seg}\par 
    \item[msdescription: ]
   \hyperref[TEI.accMat]{accMat} \hyperref[TEI.acquisition]{acquisition} \hyperref[TEI.additions]{additions} \hyperref[TEI.catchwords]{catchwords} \hyperref[TEI.collation]{collation} \hyperref[TEI.colophon]{colophon} \hyperref[TEI.condition]{condition} \hyperref[TEI.custEvent]{custEvent} \hyperref[TEI.decoNote]{decoNote} \hyperref[TEI.explicit]{explicit} \hyperref[TEI.filiation]{filiation} \hyperref[TEI.finalRubric]{finalRubric} \hyperref[TEI.foliation]{foliation} \hyperref[TEI.heraldry]{heraldry} \hyperref[TEI.incipit]{incipit} \hyperref[TEI.layout]{layout} \hyperref[TEI.material]{material} \hyperref[TEI.musicNotation]{musicNotation} \hyperref[TEI.objectType]{objectType} \hyperref[TEI.origDate]{origDate} \hyperref[TEI.origPlace]{origPlace} \hyperref[TEI.origin]{origin} \hyperref[TEI.provenance]{provenance} \hyperref[TEI.rubric]{rubric} \hyperref[TEI.secFol]{secFol} \hyperref[TEI.signatures]{signatures} \hyperref[TEI.source]{source} \hyperref[TEI.stamp]{stamp} \hyperref[TEI.summary]{summary} \hyperref[TEI.support]{support} \hyperref[TEI.surrogates]{surrogates} \hyperref[TEI.typeNote]{typeNote} \hyperref[TEI.watermark]{watermark}\par 
    \item[namesdates: ]
   \hyperref[TEI.addName]{addName} \hyperref[TEI.affiliation]{affiliation} \hyperref[TEI.country]{country} \hyperref[TEI.forename]{forename} \hyperref[TEI.genName]{genName} \hyperref[TEI.geogName]{geogName} \hyperref[TEI.nameLink]{nameLink} \hyperref[TEI.orgName]{orgName} \hyperref[TEI.persName]{persName} \hyperref[TEI.placeName]{placeName} \hyperref[TEI.region]{region} \hyperref[TEI.roleName]{roleName} \hyperref[TEI.settlement]{settlement} \hyperref[TEI.surname]{surname}\par 
    \item[textstructure: ]
   \hyperref[TEI.docAuthor]{docAuthor} \hyperref[TEI.docDate]{docDate} \hyperref[TEI.docEdition]{docEdition} \hyperref[TEI.titlePart]{titlePart}\par 
    \item[transcr: ]
   \hyperref[TEI.am]{am} \hyperref[TEI.damage]{damage} \hyperref[TEI.fw]{fw} \hyperref[TEI.line]{line} \hyperref[TEI.metamark]{metamark} \hyperref[TEI.mod]{mod} \hyperref[TEI.restore]{restore} \hyperref[TEI.retrace]{retrace} \hyperref[TEI.secl]{secl} \hyperref[TEI.supplied]{supplied} \hyperref[TEI.surplus]{surplus} \hyperref[TEI.zone]{zone}
    \item[{Peut contenir}]
  Elément vide
    \item[{Exemple}]
  \leavevmode\bgroup\exampleFont \begin{shaded}\noindent\mbox{}{<\textbf{line}>}This is {<\textbf{del}\hspace*{6pt}{change}="{\#s2}"\hspace*{6pt}{rend}="{overstrike}">}\mbox{}\newline 
\hspace*{6pt}\hspace*{6pt}{<\textbf{seg}\hspace*{6pt}{xml:id}="{undo-a}">}just some{</\textbf{seg}>}\mbox{}\newline 
\hspace*{6pt}\hspace*{6pt} sample {<\textbf{seg}\hspace*{6pt}{xml:id}="{undo-b}">}text{</\textbf{seg}>},\mbox{}\newline 
\hspace*{6pt}\hspace*{6pt} we need{</\textbf{del}>}\mbox{}\newline 
\hspace*{6pt}{<\textbf{add}\hspace*{6pt}{change}="{\#s2}">}not{</\textbf{add}>}\mbox{}\newline 
 a real example.{</\textbf{line}>}\mbox{}\newline 
{<\textbf{undo}\hspace*{6pt}{change}="{\#s3}"\hspace*{6pt}{rend}="{dotted}"\mbox{}\newline 
\hspace*{6pt}{target}="{\#undo-a \#undo-b}"/>}\end{shaded}\egroup 

This encoding represents the following sequence of events: \begin{itemize}
\item "This is just some sample text, we need a real example" is written
\item At stage s2, "just some sample text, we need" is deleted by overstriking, and "not" is added 
\item At stage s3, parts of the deletion are cancelled by underdotting, thus reinstating the words "just some" and "text".
\end{itemize} 
    \item[{Modèle de contenu}]
  \fbox{\ttfamily <content>\newline
</content>\newline
    } 
    \item[{Schéma Declaration}]
  \mbox{}\hfill\\[-10pt]\begin{Verbatim}[fontsize=\small]
element undo
{
   tei_att.global.attributes,
   tei_att.spanning.attributes,
   tei_att.transcriptional.attributes,
   attribute target { list { + } }?,
   empty
}
\end{Verbatim}

\end{reflist}  \index{vAlt=<vAlt>|oddindex}
\begin{reflist}
\item[]\begin{specHead}{TEI.vAlt}{<vAlt> }(valeur alternative) représente la partie valeur d'une spécification trait-valeur qui contient un jeu de valeurs, dont une seule peut être valide [\xref{http://www.tei-c.org/release/doc/tei-p5-doc/en/html/FS.html\#FVALT}{18.8.1. Alternation}]\end{specHead} 
    \item[{Module}]
  iso-fs
    \item[{Attributs}]
  Attributs \hyperref[TEI.att.global]{att.global} (\textit{@xml:id}, \textit{@n}, \textit{@xml:lang}, \textit{@xml:base}, \textit{@xml:space})  (\hyperref[TEI.att.global.rendition]{att.global.rendition} (\textit{@rend}, \textit{@style}, \textit{@rendition})) (\hyperref[TEI.att.global.linking]{att.global.linking} (\textit{@corresp}, \textit{@synch}, \textit{@sameAs}, \textit{@copyOf}, \textit{@next}, \textit{@prev}, \textit{@exclude}, \textit{@select})) (\hyperref[TEI.att.global.analytic]{att.global.analytic} (\textit{@ana})) (\hyperref[TEI.att.global.facs]{att.global.facs} (\textit{@facs})) (\hyperref[TEI.att.global.change]{att.global.change} (\textit{@change})) (\hyperref[TEI.att.global.responsibility]{att.global.responsibility} (\textit{@cert}, \textit{@resp})) (\hyperref[TEI.att.global.source]{att.global.source} (\textit{@source}))
    \item[{Membre du}]
  \hyperref[TEI.model.featureVal.single]{model.featureVal.single}
    \item[{Contenu dans}]
  
    \item[iso-fs: ]
   \hyperref[TEI.f]{f} \hyperref[TEI.fvLib]{fvLib} \hyperref[TEI.if]{if} \hyperref[TEI.vAlt]{vAlt} \hyperref[TEI.vColl]{vColl} \hyperref[TEI.vDefault]{vDefault} \hyperref[TEI.vLabel]{vLabel} \hyperref[TEI.vMerge]{vMerge} \hyperref[TEI.vNot]{vNot} \hyperref[TEI.vRange]{vRange}
    \item[{Peut contenir}]
  
    \item[iso-fs: ]
   \hyperref[TEI.binary]{binary} \hyperref[TEI.default]{default} \hyperref[TEI.fs]{fs} \hyperref[TEI.numeric]{numeric} \hyperref[TEI.string]{string} \hyperref[TEI.symbol]{symbol} \hyperref[TEI.vAlt]{vAlt} \hyperref[TEI.vColl]{vColl} \hyperref[TEI.vLabel]{vLabel} \hyperref[TEI.vMerge]{vMerge} \hyperref[TEI.vNot]{vNot}
    \item[{Exemple}]
  \leavevmode\bgroup\exampleFont \begin{shaded}\noindent\mbox{}{<\textbf{f}\hspace*{6pt}{name}="{gender}">}\mbox{}\newline 
\hspace*{6pt}{<\textbf{vAlt}>}\mbox{}\newline 
\hspace*{6pt}\hspace*{6pt}{<\textbf{symbol}\hspace*{6pt}{value}="{masculine}"/>}\mbox{}\newline 
\hspace*{6pt}\hspace*{6pt}{<\textbf{symbol}\hspace*{6pt}{value}="{neuter}"/>}\mbox{}\newline 
\hspace*{6pt}\hspace*{6pt}{<\textbf{symbol}\hspace*{6pt}{value}="{feminine}"/>}\mbox{}\newline 
\hspace*{6pt}{</\textbf{vAlt}>}\mbox{}\newline 
{</\textbf{f}>}\end{shaded}\egroup 


    \item[{Modèle de contenu}]
  \mbox{}\hfill\\[-10pt]\begin{Verbatim}[fontsize=\small]
<content>
 <sequence maxOccurs="1" minOccurs="1">
  <classRef key="model.featureVal"/>
  <classRef key="model.featureVal"
   maxOccurs="unbounded" minOccurs="1"/>
 </sequence>
</content>
    
\end{Verbatim}

    \item[{Schéma Declaration}]
  \mbox{}\hfill\\[-10pt]\begin{Verbatim}[fontsize=\small]
element vAlt
{
   tei_att.global.attributes,
   ( tei_model.featureVal, tei_model.featureVal+ )
}
\end{Verbatim}

\end{reflist}  \index{vColl=<vColl>|oddindex}\index{org=@org!<vColl>|oddindex}
\begin{reflist}
\item[]\begin{specHead}{TEI.vColl}{<vColl> }(collection de valeurs) représente la partie valeur d'une spécification trait-valeur qui contient des valeurs multiples organisées comme un ensemble, un paquet ou une liste. [\xref{http://www.tei-c.org/release/doc/tei-p5-doc/en/html/FS.html\#FSSS}{18.7. Collections as Complex Feature Values}]\end{specHead} 
    \item[{Module}]
  iso-fs
    \item[{Attributs}]
  Attributs \hyperref[TEI.att.global]{att.global} (\textit{@xml:id}, \textit{@n}, \textit{@xml:lang}, \textit{@xml:base}, \textit{@xml:space})  (\hyperref[TEI.att.global.rendition]{att.global.rendition} (\textit{@rend}, \textit{@style}, \textit{@rendition})) (\hyperref[TEI.att.global.linking]{att.global.linking} (\textit{@corresp}, \textit{@synch}, \textit{@sameAs}, \textit{@copyOf}, \textit{@next}, \textit{@prev}, \textit{@exclude}, \textit{@select})) (\hyperref[TEI.att.global.analytic]{att.global.analytic} (\textit{@ana})) (\hyperref[TEI.att.global.facs]{att.global.facs} (\textit{@facs})) (\hyperref[TEI.att.global.change]{att.global.change} (\textit{@change})) (\hyperref[TEI.att.global.responsibility]{att.global.responsibility} (\textit{@cert}, \textit{@resp})) (\hyperref[TEI.att.global.source]{att.global.source} (\textit{@source})) \hfil\\[-10pt]\begin{sansreflist}
    \item[@org]
  (organisation) indique l'organisation de la ou des valeurs données sous forme d'ensemble, de paquet ou de liste.
\begin{reflist}
    \item[{Statut}]
  Optionel
    \item[{Type de données}]
  \hyperref[TEI.teidata.enumerated]{teidata.enumerated}
    \item[{Les valeurs autorisées sont:}]
  \begin{description}

\item[{set}]indique que les valeurs données sont organisées en ensemble.
\item[{bag}]indique que les valeurs données sont organisées en paquet (de plusieurs ensembles). 
\item[{list}]indique que les valeurs données sont organisées en liste.
\end{description} 
\end{reflist}  
\end{sansreflist}  
    \item[{Membre du}]
  \hyperref[TEI.model.featureVal.complex]{model.featureVal.complex}
    \item[{Contenu dans}]
  
    \item[iso-fs: ]
   \hyperref[TEI.f]{f} \hyperref[TEI.fvLib]{fvLib} \hyperref[TEI.if]{if} \hyperref[TEI.vAlt]{vAlt} \hyperref[TEI.vDefault]{vDefault} \hyperref[TEI.vLabel]{vLabel} \hyperref[TEI.vMerge]{vMerge} \hyperref[TEI.vNot]{vNot} \hyperref[TEI.vRange]{vRange}
    \item[{Peut contenir}]
  
    \item[iso-fs: ]
   \hyperref[TEI.binary]{binary} \hyperref[TEI.default]{default} \hyperref[TEI.fs]{fs} \hyperref[TEI.numeric]{numeric} \hyperref[TEI.string]{string} \hyperref[TEI.symbol]{symbol} \hyperref[TEI.vAlt]{vAlt} \hyperref[TEI.vLabel]{vLabel}
    \item[{Exemple}]
  \leavevmode\bgroup\exampleFont \begin{shaded}\noindent\mbox{}{<\textbf{f}\hspace*{6pt}{name}="{name}">}\mbox{}\newline 
\hspace*{6pt}{<\textbf{vColl}>}\mbox{}\newline 
\hspace*{6pt}\hspace*{6pt}{<\textbf{string}>}Jean{</\textbf{string}>}\mbox{}\newline 
\hspace*{6pt}\hspace*{6pt}{<\textbf{string}>}Luc{</\textbf{string}>}\mbox{}\newline 
\hspace*{6pt}\hspace*{6pt}{<\textbf{string}>}Godard{</\textbf{string}>}\mbox{}\newline 
\hspace*{6pt}{</\textbf{vColl}>}\mbox{}\newline 
{</\textbf{f}>}\end{shaded}\egroup 


    \item[{Exemple}]
  \leavevmode\bgroup\exampleFont \begin{shaded}\noindent\mbox{}{<\textbf{f}\hspace*{6pt}{name}="{name}">}\mbox{}\newline 
\hspace*{6pt}{<\textbf{vColl}>}\mbox{}\newline 
\hspace*{6pt}\hspace*{6pt}{<\textbf{string}>}Jean{</\textbf{string}>}\mbox{}\newline 
\hspace*{6pt}\hspace*{6pt}{<\textbf{string}>}Luc{</\textbf{string}>}\mbox{}\newline 
\hspace*{6pt}\hspace*{6pt}{<\textbf{string}>}Godard{</\textbf{string}>}\mbox{}\newline 
\hspace*{6pt}{</\textbf{vColl}>}\mbox{}\newline 
{</\textbf{f}>}\end{shaded}\egroup 


    \item[{Exemple}]
  \leavevmode\bgroup\exampleFont \begin{shaded}\noindent\mbox{}{<\textbf{fs}>}\mbox{}\newline 
\hspace*{6pt}{<\textbf{f}\hspace*{6pt}{name}="{lex}">}\mbox{}\newline 
\hspace*{6pt}\hspace*{6pt}{<\textbf{symbol}\hspace*{6pt}{value}="{auxquels}"/>}\mbox{}\newline 
\hspace*{6pt}{</\textbf{f}>}\mbox{}\newline 
\hspace*{6pt}{<\textbf{f}\hspace*{6pt}{name}="{maf}">}\mbox{}\newline 
\hspace*{6pt}\hspace*{6pt}{<\textbf{vColl}\hspace*{6pt}{org}="{list}">}\mbox{}\newline 
\hspace*{6pt}\hspace*{6pt}\hspace*{6pt}{<\textbf{fs}>}\mbox{}\newline 
\hspace*{6pt}\hspace*{6pt}\hspace*{6pt}\hspace*{6pt}{<\textbf{f}\hspace*{6pt}{name}="{cat}">}\mbox{}\newline 
\hspace*{6pt}\hspace*{6pt}\hspace*{6pt}\hspace*{6pt}\hspace*{6pt}{<\textbf{symbol}\hspace*{6pt}{value}="{prep}"/>}\mbox{}\newline 
\hspace*{6pt}\hspace*{6pt}\hspace*{6pt}\hspace*{6pt}{</\textbf{f}>}\mbox{}\newline 
\hspace*{6pt}\hspace*{6pt}\hspace*{6pt}{</\textbf{fs}>}\mbox{}\newline 
\hspace*{6pt}\hspace*{6pt}\hspace*{6pt}{<\textbf{fs}>}\mbox{}\newline 
\hspace*{6pt}\hspace*{6pt}\hspace*{6pt}\hspace*{6pt}{<\textbf{f}\hspace*{6pt}{name}="{cat}">}\mbox{}\newline 
\hspace*{6pt}\hspace*{6pt}\hspace*{6pt}\hspace*{6pt}\hspace*{6pt}{<\textbf{symbol}\hspace*{6pt}{value}="{pronoun}"/>}\mbox{}\newline 
\hspace*{6pt}\hspace*{6pt}\hspace*{6pt}\hspace*{6pt}{</\textbf{f}>}\mbox{}\newline 
\hspace*{6pt}\hspace*{6pt}\hspace*{6pt}\hspace*{6pt}{<\textbf{f}\hspace*{6pt}{name}="{kind}">}\mbox{}\newline 
\hspace*{6pt}\hspace*{6pt}\hspace*{6pt}\hspace*{6pt}\hspace*{6pt}{<\textbf{symbol}\hspace*{6pt}{value}="{rel}"/>}\mbox{}\newline 
\hspace*{6pt}\hspace*{6pt}\hspace*{6pt}\hspace*{6pt}{</\textbf{f}>}\mbox{}\newline 
\hspace*{6pt}\hspace*{6pt}\hspace*{6pt}\hspace*{6pt}{<\textbf{f}\hspace*{6pt}{name}="{num}">}\mbox{}\newline 
\hspace*{6pt}\hspace*{6pt}\hspace*{6pt}\hspace*{6pt}\hspace*{6pt}{<\textbf{symbol}\hspace*{6pt}{value}="{pl}"/>}\mbox{}\newline 
\hspace*{6pt}\hspace*{6pt}\hspace*{6pt}\hspace*{6pt}{</\textbf{f}>}\mbox{}\newline 
\hspace*{6pt}\hspace*{6pt}\hspace*{6pt}\hspace*{6pt}{<\textbf{f}\hspace*{6pt}{name}="{gender}">}\mbox{}\newline 
\hspace*{6pt}\hspace*{6pt}\hspace*{6pt}\hspace*{6pt}\hspace*{6pt}{<\textbf{symbol}\hspace*{6pt}{value}="{masc}"/>}\mbox{}\newline 
\hspace*{6pt}\hspace*{6pt}\hspace*{6pt}\hspace*{6pt}{</\textbf{f}>}\mbox{}\newline 
\hspace*{6pt}\hspace*{6pt}\hspace*{6pt}{</\textbf{fs}>}\mbox{}\newline 
\hspace*{6pt}\hspace*{6pt}{</\textbf{vColl}>}\mbox{}\newline 
\hspace*{6pt}{</\textbf{f}>}\mbox{}\newline 
{</\textbf{fs}>}\end{shaded}\egroup 


    \item[{Exemple}]
  \leavevmode\bgroup\exampleFont \begin{shaded}\noindent\mbox{}{<\textbf{fs}>}\mbox{}\newline 
\hspace*{6pt}{<\textbf{f}\hspace*{6pt}{name}="{lex}">}\mbox{}\newline 
\hspace*{6pt}\hspace*{6pt}{<\textbf{symbol}\hspace*{6pt}{value}="{auxquels}"/>}\mbox{}\newline 
\hspace*{6pt}{</\textbf{f}>}\mbox{}\newline 
\hspace*{6pt}{<\textbf{f}\hspace*{6pt}{name}="{maf}">}\mbox{}\newline 
\hspace*{6pt}\hspace*{6pt}{<\textbf{vColl}\hspace*{6pt}{org}="{list}">}\mbox{}\newline 
\hspace*{6pt}\hspace*{6pt}\hspace*{6pt}{<\textbf{fs}>}\mbox{}\newline 
\hspace*{6pt}\hspace*{6pt}\hspace*{6pt}\hspace*{6pt}{<\textbf{f}\hspace*{6pt}{name}="{cat}">}\mbox{}\newline 
\hspace*{6pt}\hspace*{6pt}\hspace*{6pt}\hspace*{6pt}\hspace*{6pt}{<\textbf{symbol}\hspace*{6pt}{value}="{prep}"/>}\mbox{}\newline 
\hspace*{6pt}\hspace*{6pt}\hspace*{6pt}\hspace*{6pt}{</\textbf{f}>}\mbox{}\newline 
\hspace*{6pt}\hspace*{6pt}\hspace*{6pt}{</\textbf{fs}>}\mbox{}\newline 
\hspace*{6pt}\hspace*{6pt}\hspace*{6pt}{<\textbf{fs}>}\mbox{}\newline 
\hspace*{6pt}\hspace*{6pt}\hspace*{6pt}\hspace*{6pt}{<\textbf{f}\hspace*{6pt}{name}="{cat}">}\mbox{}\newline 
\hspace*{6pt}\hspace*{6pt}\hspace*{6pt}\hspace*{6pt}\hspace*{6pt}{<\textbf{symbol}\hspace*{6pt}{value}="{pronoun}"/>}\mbox{}\newline 
\hspace*{6pt}\hspace*{6pt}\hspace*{6pt}\hspace*{6pt}{</\textbf{f}>}\mbox{}\newline 
\hspace*{6pt}\hspace*{6pt}\hspace*{6pt}\hspace*{6pt}{<\textbf{f}\hspace*{6pt}{name}="{kind}">}\mbox{}\newline 
\hspace*{6pt}\hspace*{6pt}\hspace*{6pt}\hspace*{6pt}\hspace*{6pt}{<\textbf{symbol}\hspace*{6pt}{value}="{rel}"/>}\mbox{}\newline 
\hspace*{6pt}\hspace*{6pt}\hspace*{6pt}\hspace*{6pt}{</\textbf{f}>}\mbox{}\newline 
\hspace*{6pt}\hspace*{6pt}\hspace*{6pt}\hspace*{6pt}{<\textbf{f}\hspace*{6pt}{name}="{num}">}\mbox{}\newline 
\hspace*{6pt}\hspace*{6pt}\hspace*{6pt}\hspace*{6pt}\hspace*{6pt}{<\textbf{symbol}\hspace*{6pt}{value}="{pl}"/>}\mbox{}\newline 
\hspace*{6pt}\hspace*{6pt}\hspace*{6pt}\hspace*{6pt}{</\textbf{f}>}\mbox{}\newline 
\hspace*{6pt}\hspace*{6pt}\hspace*{6pt}\hspace*{6pt}{<\textbf{f}\hspace*{6pt}{name}="{gender}">}\mbox{}\newline 
\hspace*{6pt}\hspace*{6pt}\hspace*{6pt}\hspace*{6pt}\hspace*{6pt}{<\textbf{symbol}\hspace*{6pt}{value}="{masc}"/>}\mbox{}\newline 
\hspace*{6pt}\hspace*{6pt}\hspace*{6pt}\hspace*{6pt}{</\textbf{f}>}\mbox{}\newline 
\hspace*{6pt}\hspace*{6pt}\hspace*{6pt}{</\textbf{fs}>}\mbox{}\newline 
\hspace*{6pt}\hspace*{6pt}{</\textbf{vColl}>}\mbox{}\newline 
\hspace*{6pt}{</\textbf{f}>}\mbox{}\newline 
{</\textbf{fs}>}\end{shaded}\egroup 


    \item[{Modèle de contenu}]
  \mbox{}\hfill\\[-10pt]\begin{Verbatim}[fontsize=\small]
<content>
 <alternate maxOccurs="unbounded"
  minOccurs="0">
  <elementRef key="fs"/>
  <classRef key="model.featureVal.single"/>
 </alternate>
</content>
    
\end{Verbatim}

    \item[{Schéma Declaration}]
  \mbox{}\hfill\\[-10pt]\begin{Verbatim}[fontsize=\small]
element vColl
{
   tei_att.global.attributes,
   attribute org { "set" | "bag" | "list" }?,
   ( tei_fs | tei_model.featureVal.single )*
}
\end{Verbatim}

\end{reflist}  \index{vDefault=<vDefault>|oddindex}
\begin{reflist}
\item[]\begin{specHead}{TEI.vDefault}{<vDefault> }(valeur par défaut) déclare la valeur par défaut à fournir quand une structure de traits ne contient aucun cas de \hyperref[TEI.f]{<f>} pour ce nom ; si elle est inconditionnelle, on l'indique comme un élément \hyperref[TEI.fs]{<fs>} (ou plusieurs, selon la valeur de l'attribut {\itshape org} du \hyperref[TEI.fDecl]{<fDecl>} englobant) ; si elle est conditionnelle, on l'indique comme un ou plusieurs éléments \hyperref[TEI.if]{<if>} ; si aucune valeur par défaut n'est précisée ou si aucune condition ne correspond, la valeur nulle est retenue. [\xref{http://www.tei-c.org/release/doc/tei-p5-doc/en/html/FS.html\#FD}{18.11. Feature System Declaration}]\end{specHead} 
    \item[{Module}]
  iso-fs
    \item[{Attributs}]
  Attributs \hyperref[TEI.att.global]{att.global} (\textit{@xml:id}, \textit{@n}, \textit{@xml:lang}, \textit{@xml:base}, \textit{@xml:space})  (\hyperref[TEI.att.global.rendition]{att.global.rendition} (\textit{@rend}, \textit{@style}, \textit{@rendition})) (\hyperref[TEI.att.global.linking]{att.global.linking} (\textit{@corresp}, \textit{@synch}, \textit{@sameAs}, \textit{@copyOf}, \textit{@next}, \textit{@prev}, \textit{@exclude}, \textit{@select})) (\hyperref[TEI.att.global.analytic]{att.global.analytic} (\textit{@ana})) (\hyperref[TEI.att.global.facs]{att.global.facs} (\textit{@facs})) (\hyperref[TEI.att.global.change]{att.global.change} (\textit{@change})) (\hyperref[TEI.att.global.responsibility]{att.global.responsibility} (\textit{@cert}, \textit{@resp})) (\hyperref[TEI.att.global.source]{att.global.source} (\textit{@source}))
    \item[{Contenu dans}]
  
    \item[iso-fs: ]
   \hyperref[TEI.fDecl]{fDecl}
    \item[{Peut contenir}]
  
    \item[iso-fs: ]
   \hyperref[TEI.binary]{binary} \hyperref[TEI.default]{default} \hyperref[TEI.fs]{fs} \hyperref[TEI.if]{if} \hyperref[TEI.numeric]{numeric} \hyperref[TEI.string]{string} \hyperref[TEI.symbol]{symbol} \hyperref[TEI.vAlt]{vAlt} \hyperref[TEI.vColl]{vColl} \hyperref[TEI.vLabel]{vLabel} \hyperref[TEI.vMerge]{vMerge} \hyperref[TEI.vNot]{vNot}
    \item[{Note}]
  \par
Peut contenir une valeur de trait admise ou une série d'éléments \hyperref[TEI.if]{<if>}.
    \item[{Exemple}]
  \leavevmode\bgroup\exampleFont \begin{shaded}\noindent\mbox{}{<\textbf{fDecl}\hspace*{6pt}{name}="{INV}">}\mbox{}\newline 
\hspace*{6pt}{<\textbf{fDescr}>}inverted sentence{</\textbf{fDescr}>}\mbox{}\newline 
\hspace*{6pt}{<\textbf{vRange}>}\mbox{}\newline 
\hspace*{6pt}\hspace*{6pt}{<\textbf{vAlt}>}\mbox{}\newline 
\hspace*{6pt}\hspace*{6pt}\hspace*{6pt}{<\textbf{binary}\hspace*{6pt}{value}="{true}"/>}\mbox{}\newline 
\hspace*{6pt}\hspace*{6pt}\hspace*{6pt}{<\textbf{binary}\hspace*{6pt}{value}="{false}"/>}\mbox{}\newline 
\hspace*{6pt}\hspace*{6pt}{</\textbf{vAlt}>}\mbox{}\newline 
\hspace*{6pt}{</\textbf{vRange}>}\mbox{}\newline 
\hspace*{6pt}{<\textbf{vDefault}>}\mbox{}\newline 
\hspace*{6pt}\hspace*{6pt}{<\textbf{binary}\hspace*{6pt}{value}="{false}"/>}\mbox{}\newline 
\hspace*{6pt}{</\textbf{vDefault}>}\mbox{}\newline 
{</\textbf{fDecl}>}\end{shaded}\egroup 


    \item[{Modèle de contenu}]
  \mbox{}\hfill\\[-10pt]\begin{Verbatim}[fontsize=\small]
<content>
 <alternate maxOccurs="1" minOccurs="1">
  <classRef key="model.featureVal"
   maxOccurs="unbounded" minOccurs="1"/>
  <elementRef key="if"
   maxOccurs="unbounded" minOccurs="1"/>
 </alternate>
</content>
    
\end{Verbatim}

    \item[{Schéma Declaration}]
  \mbox{}\hfill\\[-10pt]\begin{Verbatim}[fontsize=\small]
element vDefault
{
   tei_att.global.attributes,
   ( tei_model.featureVal+ | tei_if+ )
}
\end{Verbatim}

\end{reflist}  \index{vLabel=<vLabel>|oddindex}\index{name=@name!<vLabel>|oddindex}
\begin{reflist}
\item[]\begin{specHead}{TEI.vLabel}{<vLabel> }(étiquette de valeur) représente la partie valeur d'une spécification trait-valeur qui apparaît en plus d’un point dans une structure de traits. [\xref{http://www.tei-c.org/release/doc/tei-p5-doc/en/html/FS.html\#FSVAR}{18.6. Re-entrant Feature Structures}]\end{specHead} 
    \item[{Module}]
  iso-fs
    \item[{Attributs}]
  Attributs \hyperref[TEI.att.global]{att.global} (\textit{@xml:id}, \textit{@n}, \textit{@xml:lang}, \textit{@xml:base}, \textit{@xml:space})  (\hyperref[TEI.att.global.rendition]{att.global.rendition} (\textit{@rend}, \textit{@style}, \textit{@rendition})) (\hyperref[TEI.att.global.linking]{att.global.linking} (\textit{@corresp}, \textit{@synch}, \textit{@sameAs}, \textit{@copyOf}, \textit{@next}, \textit{@prev}, \textit{@exclude}, \textit{@select})) (\hyperref[TEI.att.global.analytic]{att.global.analytic} (\textit{@ana})) (\hyperref[TEI.att.global.facs]{att.global.facs} (\textit{@facs})) (\hyperref[TEI.att.global.change]{att.global.change} (\textit{@change})) (\hyperref[TEI.att.global.responsibility]{att.global.responsibility} (\textit{@cert}, \textit{@resp})) (\hyperref[TEI.att.global.source]{att.global.source} (\textit{@source})) \hfil\\[-10pt]\begin{sansreflist}
    \item[@name]
  fournit un nom pour le point de partage.
\begin{reflist}
    \item[{Statut}]
  Requis
    \item[{Type de données}]
  \hyperref[TEI.teidata.word]{teidata.word}
\end{reflist}  
\end{sansreflist}  
    \item[{Membre du}]
  \hyperref[TEI.model.featureVal.single]{model.featureVal.single}
    \item[{Contenu dans}]
  
    \item[iso-fs: ]
   \hyperref[TEI.f]{f} \hyperref[TEI.fvLib]{fvLib} \hyperref[TEI.if]{if} \hyperref[TEI.vAlt]{vAlt} \hyperref[TEI.vColl]{vColl} \hyperref[TEI.vDefault]{vDefault} \hyperref[TEI.vLabel]{vLabel} \hyperref[TEI.vMerge]{vMerge} \hyperref[TEI.vNot]{vNot} \hyperref[TEI.vRange]{vRange}
    \item[{Peut contenir}]
  
    \item[iso-fs: ]
   \hyperref[TEI.binary]{binary} \hyperref[TEI.default]{default} \hyperref[TEI.fs]{fs} \hyperref[TEI.numeric]{numeric} \hyperref[TEI.string]{string} \hyperref[TEI.symbol]{symbol} \hyperref[TEI.vAlt]{vAlt} \hyperref[TEI.vColl]{vColl} \hyperref[TEI.vLabel]{vLabel} \hyperref[TEI.vMerge]{vMerge} \hyperref[TEI.vNot]{vNot}
    \item[{Exemple}]
  \leavevmode\bgroup\exampleFont \begin{shaded}\noindent\mbox{}{<\textbf{fs}>}\mbox{}\newline 
\hspace*{6pt}{<\textbf{f}\hspace*{6pt}{name}="{nominal}">}\mbox{}\newline 
\hspace*{6pt}\hspace*{6pt}{<\textbf{fs}>}\mbox{}\newline 
\hspace*{6pt}\hspace*{6pt}\hspace*{6pt}{<\textbf{f}\hspace*{6pt}{name}="{nm-num}">}\mbox{}\newline 
\hspace*{6pt}\hspace*{6pt}\hspace*{6pt}\hspace*{6pt}{<\textbf{vLabel}\hspace*{6pt}{name}="{L1}">}\mbox{}\newline 
\hspace*{6pt}\hspace*{6pt}\hspace*{6pt}\hspace*{6pt}\hspace*{6pt}{<\textbf{symbol}\hspace*{6pt}{value}="{singular}"/>}\mbox{}\newline 
\hspace*{6pt}\hspace*{6pt}\hspace*{6pt}\hspace*{6pt}{</\textbf{vLabel}>}\mbox{}\newline 
\hspace*{6pt}\hspace*{6pt}\hspace*{6pt}{</\textbf{f}>}\mbox{}\newline 
\hspace*{6pt}\hspace*{6pt}{</\textbf{fs}>}\mbox{}\newline 
\hspace*{6pt}{</\textbf{f}>}\mbox{}\newline 
\hspace*{6pt}{<\textbf{f}\hspace*{6pt}{name}="{verbal}">}\mbox{}\newline 
\hspace*{6pt}\hspace*{6pt}{<\textbf{fs}>}\mbox{}\newline 
\hspace*{6pt}\hspace*{6pt}\hspace*{6pt}{<\textbf{f}\hspace*{6pt}{name}="{vb-num}">}\mbox{}\newline 
\hspace*{6pt}\hspace*{6pt}\hspace*{6pt}\hspace*{6pt}{<\textbf{vLabel}\hspace*{6pt}{name}="{L1}"/>}\mbox{}\newline 
\hspace*{6pt}\hspace*{6pt}\hspace*{6pt}{</\textbf{f}>}\mbox{}\newline 
\hspace*{6pt}\hspace*{6pt}{</\textbf{fs}>}\mbox{}\newline 
\hspace*{6pt}{</\textbf{f}>}\mbox{}\newline 
{</\textbf{fs}>}\end{shaded}\egroup 


    \item[{Modèle de contenu}]
  \mbox{}\hfill\\[-10pt]\begin{Verbatim}[fontsize=\small]
<content>
 <classRef key="model.featureVal"
  minOccurs="0"/>
</content>
    
\end{Verbatim}

    \item[{Schéma Declaration}]
  \mbox{}\hfill\\[-10pt]\begin{Verbatim}[fontsize=\small]
element vLabel
{
   tei_att.global.attributes,
   attribute name { text },
   tei_model.featureVal?
}
\end{Verbatim}

\end{reflist}  \index{vMerge=<vMerge>|oddindex}\index{org=@org!<vMerge>|oddindex}
\begin{reflist}
\item[]\begin{specHead}{TEI.vMerge}{<vMerge> }(collection fusionnée de valeurs) représente une valeur de trait , résultant de la fusion des valeurs de trait contenues dans les éléments enfants, qui utilisent l'agencement indiqué par l'attribut {\itshape org}. [\xref{http://www.tei-c.org/release/doc/tei-p5-doc/en/html/FS.html\#FVCOLL}{18.8.3. Collection of Values}]\end{specHead} 
    \item[{Module}]
  iso-fs
    \item[{Attributs}]
  Attributs \hyperref[TEI.att.global]{att.global} (\textit{@xml:id}, \textit{@n}, \textit{@xml:lang}, \textit{@xml:base}, \textit{@xml:space})  (\hyperref[TEI.att.global.rendition]{att.global.rendition} (\textit{@rend}, \textit{@style}, \textit{@rendition})) (\hyperref[TEI.att.global.linking]{att.global.linking} (\textit{@corresp}, \textit{@synch}, \textit{@sameAs}, \textit{@copyOf}, \textit{@next}, \textit{@prev}, \textit{@exclude}, \textit{@select})) (\hyperref[TEI.att.global.analytic]{att.global.analytic} (\textit{@ana})) (\hyperref[TEI.att.global.facs]{att.global.facs} (\textit{@facs})) (\hyperref[TEI.att.global.change]{att.global.change} (\textit{@change})) (\hyperref[TEI.att.global.responsibility]{att.global.responsibility} (\textit{@cert}, \textit{@resp})) (\hyperref[TEI.att.global.source]{att.global.source} (\textit{@source})) \hfil\\[-10pt]\begin{sansreflist}
    \item[@org]
  indique l'agencement des valeurs fusionnées résultantes comme un ensemble, un paquet ou une liste.
\begin{reflist}
    \item[{Statut}]
  Optionel
    \item[{Type de données}]
  \hyperref[TEI.teidata.enumerated]{teidata.enumerated}
    \item[{Les valeurs autorisées sont:}]
  \begin{description}

\item[{set}]indique que les valeurs résultantes sont organisées en ensemble.
\item[{bag}]indique que les valeurs résultantes sont organisées en paquet (de plusieurs ensembles). 
\item[{list}]indique que les valeurs résultantes sont organisées en liste.
\end{description} 
\end{reflist}  
\end{sansreflist}  
    \item[{Membre du}]
  \hyperref[TEI.model.featureVal.complex]{model.featureVal.complex}
    \item[{Contenu dans}]
  
    \item[iso-fs: ]
   \hyperref[TEI.f]{f} \hyperref[TEI.fvLib]{fvLib} \hyperref[TEI.if]{if} \hyperref[TEI.vAlt]{vAlt} \hyperref[TEI.vDefault]{vDefault} \hyperref[TEI.vLabel]{vLabel} \hyperref[TEI.vMerge]{vMerge} \hyperref[TEI.vNot]{vNot} \hyperref[TEI.vRange]{vRange}
    \item[{Peut contenir}]
  
    \item[iso-fs: ]
   \hyperref[TEI.binary]{binary} \hyperref[TEI.default]{default} \hyperref[TEI.fs]{fs} \hyperref[TEI.numeric]{numeric} \hyperref[TEI.string]{string} \hyperref[TEI.symbol]{symbol} \hyperref[TEI.vAlt]{vAlt} \hyperref[TEI.vColl]{vColl} \hyperref[TEI.vLabel]{vLabel} \hyperref[TEI.vMerge]{vMerge} \hyperref[TEI.vNot]{vNot}
    \item[{Exemple}]
  \leavevmode\bgroup\exampleFont \begin{shaded}\noindent\mbox{}{<\textbf{vMerge}\hspace*{6pt}{org}="{list}">}\mbox{}\newline 
\hspace*{6pt}{<\textbf{vColl}\hspace*{6pt}{org}="{set}">}\mbox{}\newline 
\hspace*{6pt}\hspace*{6pt}{<\textbf{symbol}\hspace*{6pt}{value}="{masculine}"/>}\mbox{}\newline 
\hspace*{6pt}\hspace*{6pt}{<\textbf{symbol}\hspace*{6pt}{value}="{neuter}"/>}\mbox{}\newline 
\hspace*{6pt}\hspace*{6pt}{<\textbf{symbol}\hspace*{6pt}{value}="{feminine}"/>}\mbox{}\newline 
\hspace*{6pt}{</\textbf{vColl}>}\mbox{}\newline 
\hspace*{6pt}{<\textbf{symbol}\hspace*{6pt}{value}="{indeterminate}"/>}\mbox{}\newline 
{</\textbf{vMerge}>}\end{shaded}\egroup 

Cet exemple génère une liste, concaténant la valeur indéterminée avec le jeu de valeurs masculines, neutres et féminines.
    \item[{Modèle de contenu}]
  \mbox{}\hfill\\[-10pt]\begin{Verbatim}[fontsize=\small]
<content>
 <classRef key="model.featureVal"
  maxOccurs="unbounded" minOccurs="1"/>
</content>
    
\end{Verbatim}

    \item[{Schéma Declaration}]
  \mbox{}\hfill\\[-10pt]\begin{Verbatim}[fontsize=\small]
element vMerge
{
   tei_att.global.attributes,
   attribute org { "set" | "bag" | "list" }?,
   tei_model.featureVal+
}
\end{Verbatim}

\end{reflist}  \index{vNot=<vNot>|oddindex}
\begin{reflist}
\item[]\begin{specHead}{TEI.vNot}{<vNot> }(négation de valeur) représente une valeur de trait qui est la négation de son contenu. [\xref{http://www.tei-c.org/release/doc/tei-p5-doc/en/html/FS.html\#FVNOT}{18.8.2. Negation}]\end{specHead} 
    \item[{Module}]
  iso-fs
    \item[{Attributs}]
  Attributs \hyperref[TEI.att.global]{att.global} (\textit{@xml:id}, \textit{@n}, \textit{@xml:lang}, \textit{@xml:base}, \textit{@xml:space})  (\hyperref[TEI.att.global.rendition]{att.global.rendition} (\textit{@rend}, \textit{@style}, \textit{@rendition})) (\hyperref[TEI.att.global.linking]{att.global.linking} (\textit{@corresp}, \textit{@synch}, \textit{@sameAs}, \textit{@copyOf}, \textit{@next}, \textit{@prev}, \textit{@exclude}, \textit{@select})) (\hyperref[TEI.att.global.analytic]{att.global.analytic} (\textit{@ana})) (\hyperref[TEI.att.global.facs]{att.global.facs} (\textit{@facs})) (\hyperref[TEI.att.global.change]{att.global.change} (\textit{@change})) (\hyperref[TEI.att.global.responsibility]{att.global.responsibility} (\textit{@cert}, \textit{@resp})) (\hyperref[TEI.att.global.source]{att.global.source} (\textit{@source}))
    \item[{Membre du}]
  \hyperref[TEI.model.featureVal.complex]{model.featureVal.complex}
    \item[{Contenu dans}]
  
    \item[iso-fs: ]
   \hyperref[TEI.f]{f} \hyperref[TEI.fvLib]{fvLib} \hyperref[TEI.if]{if} \hyperref[TEI.vAlt]{vAlt} \hyperref[TEI.vDefault]{vDefault} \hyperref[TEI.vLabel]{vLabel} \hyperref[TEI.vMerge]{vMerge} \hyperref[TEI.vNot]{vNot} \hyperref[TEI.vRange]{vRange}
    \item[{Peut contenir}]
  
    \item[iso-fs: ]
   \hyperref[TEI.binary]{binary} \hyperref[TEI.default]{default} \hyperref[TEI.fs]{fs} \hyperref[TEI.numeric]{numeric} \hyperref[TEI.string]{string} \hyperref[TEI.symbol]{symbol} \hyperref[TEI.vAlt]{vAlt} \hyperref[TEI.vColl]{vColl} \hyperref[TEI.vLabel]{vLabel} \hyperref[TEI.vMerge]{vMerge} \hyperref[TEI.vNot]{vNot}
    \item[{Exemple}]
  \leavevmode\bgroup\exampleFont \begin{shaded}\noindent\mbox{}{<\textbf{vNot}>}\mbox{}\newline 
\hspace*{6pt}{<\textbf{symbol}\hspace*{6pt}{value}="{masculine}"/>}\mbox{}\newline 
{</\textbf{vNot}>}\end{shaded}\egroup 


    \item[{Exemple}]
  \leavevmode\bgroup\exampleFont \begin{shaded}\noindent\mbox{}{<\textbf{f}\hspace*{6pt}{name}="{mode}">}\mbox{}\newline 
\hspace*{6pt}{<\textbf{vNot}>}\mbox{}\newline 
\hspace*{6pt}\hspace*{6pt}{<\textbf{vAlt}>}\mbox{}\newline 
\hspace*{6pt}\hspace*{6pt}\hspace*{6pt}{<\textbf{symbol}\hspace*{6pt}{value}="{infinitive}"/>}\mbox{}\newline 
\hspace*{6pt}\hspace*{6pt}\hspace*{6pt}{<\textbf{symbol}\hspace*{6pt}{value}="{participle}"/>}\mbox{}\newline 
\hspace*{6pt}\hspace*{6pt}{</\textbf{vAlt}>}\mbox{}\newline 
\hspace*{6pt}{</\textbf{vNot}>}\mbox{}\newline 
{</\textbf{f}>}\end{shaded}\egroup 


    \item[{Modèle de contenu}]
  \mbox{}\hfill\\[-10pt]\begin{Verbatim}[fontsize=\small]
<content>
 <classRef key="model.featureVal"/>
</content>
    
\end{Verbatim}

    \item[{Schéma Declaration}]
  \mbox{}\hfill\\[-10pt]\begin{Verbatim}[fontsize=\small]
element vNot { tei_att.global.attributes, tei_model.featureVal }
\end{Verbatim}

\end{reflist}  \index{vRange=<vRange>|oddindex}
\begin{reflist}
\item[]\begin{specHead}{TEI.vRange}{<vRange> }(gamme de valeurs) définit la plage de valeurs autorisées pour un trait, sous la forme d'un \hyperref[TEI.fs]{<fs>}, \hyperref[TEI.vAlt]{<vAlt>}, ou d'une valeur primitive ; pour que la valeur d'un élément \hyperref[TEI.f]{<f>} soit valide, elle doit être englobée dans la plage spécifiée. Si le \hyperref[TEI.f]{<f>} contient des valeurs multiples (comme prévu par l'attribut {\itshape org}), chacune des valeurs doit être englobée dans l'élément \hyperref[TEI.vRange]{<vRange>}. [\xref{http://www.tei-c.org/release/doc/tei-p5-doc/en/html/FS.html\#FD}{18.11. Feature System Declaration}]\end{specHead} 
    \item[{Module}]
  iso-fs
    \item[{Attributs}]
  Attributs \hyperref[TEI.att.global]{att.global} (\textit{@xml:id}, \textit{@n}, \textit{@xml:lang}, \textit{@xml:base}, \textit{@xml:space})  (\hyperref[TEI.att.global.rendition]{att.global.rendition} (\textit{@rend}, \textit{@style}, \textit{@rendition})) (\hyperref[TEI.att.global.linking]{att.global.linking} (\textit{@corresp}, \textit{@synch}, \textit{@sameAs}, \textit{@copyOf}, \textit{@next}, \textit{@prev}, \textit{@exclude}, \textit{@select})) (\hyperref[TEI.att.global.analytic]{att.global.analytic} (\textit{@ana})) (\hyperref[TEI.att.global.facs]{att.global.facs} (\textit{@facs})) (\hyperref[TEI.att.global.change]{att.global.change} (\textit{@change})) (\hyperref[TEI.att.global.responsibility]{att.global.responsibility} (\textit{@cert}, \textit{@resp})) (\hyperref[TEI.att.global.source]{att.global.source} (\textit{@source}))
    \item[{Contenu dans}]
  
    \item[iso-fs: ]
   \hyperref[TEI.fDecl]{fDecl}
    \item[{Peut contenir}]
  
    \item[iso-fs: ]
   \hyperref[TEI.binary]{binary} \hyperref[TEI.default]{default} \hyperref[TEI.fs]{fs} \hyperref[TEI.numeric]{numeric} \hyperref[TEI.string]{string} \hyperref[TEI.symbol]{symbol} \hyperref[TEI.vAlt]{vAlt} \hyperref[TEI.vColl]{vColl} \hyperref[TEI.vLabel]{vLabel} \hyperref[TEI.vMerge]{vMerge} \hyperref[TEI.vNot]{vNot}
    \item[{Note}]
  \par
Peut contenir n'importe quelle spécification trait-valeur admise.
    \item[{Exemple}]
  \leavevmode\bgroup\exampleFont \begin{shaded}\noindent\mbox{}{<\textbf{fDecl}\hspace*{6pt}{name}="{INV}">}\mbox{}\newline 
\hspace*{6pt}{<\textbf{fDescr}>}inverted sentence{</\textbf{fDescr}>}\mbox{}\newline 
\hspace*{6pt}{<\textbf{vRange}>}\mbox{}\newline 
\hspace*{6pt}\hspace*{6pt}{<\textbf{vAlt}>}\mbox{}\newline 
\hspace*{6pt}\hspace*{6pt}\hspace*{6pt}{<\textbf{binary}\hspace*{6pt}{value}="{true}"/>}\mbox{}\newline 
\hspace*{6pt}\hspace*{6pt}\hspace*{6pt}{<\textbf{binary}\hspace*{6pt}{value}="{false}"/>}\mbox{}\newline 
\hspace*{6pt}\hspace*{6pt}{</\textbf{vAlt}>}\mbox{}\newline 
\hspace*{6pt}{</\textbf{vRange}>}\mbox{}\newline 
\hspace*{6pt}{<\textbf{vDefault}>}\mbox{}\newline 
\hspace*{6pt}\hspace*{6pt}{<\textbf{binary}\hspace*{6pt}{value}="{false}"/>}\mbox{}\newline 
\hspace*{6pt}{</\textbf{vDefault}>}\mbox{}\newline 
{</\textbf{fDecl}>}\end{shaded}\egroup 


    \item[{Modèle de contenu}]
  \mbox{}\hfill\\[-10pt]\begin{Verbatim}[fontsize=\small]
<content>
 <classRef key="model.featureVal"/>
</content>
    
\end{Verbatim}

    \item[{Schéma Declaration}]
  \mbox{}\hfill\\[-10pt]\begin{Verbatim}[fontsize=\small]
element vRange { tei_att.global.attributes, tei_model.featureVal }
\end{Verbatim}

\end{reflist}  \index{w=<w>|oddindex}\index{lemma=@lemma!<w>|oddindex}\index{lemmaRef=@lemmaRef!<w>|oddindex}
\begin{reflist}
\item[]\begin{specHead}{TEI.w}{<w> }(mot) représente un mot grammatical (pas nécessairement orthographique) [\xref{http://www.tei-c.org/release/doc/tei-p5-doc/en/html/AI.html\#AILC}{17.1. Linguistic Segment Categories}]\end{specHead} 
    \item[{Module}]
  analysis
    \item[{Attributs}]
  Attributs \hyperref[TEI.att.global]{att.global} (\textit{@xml:id}, \textit{@n}, \textit{@xml:lang}, \textit{@xml:base}, \textit{@xml:space})  (\hyperref[TEI.att.global.rendition]{att.global.rendition} (\textit{@rend}, \textit{@style}, \textit{@rendition})) (\hyperref[TEI.att.global.linking]{att.global.linking} (\textit{@corresp}, \textit{@synch}, \textit{@sameAs}, \textit{@copyOf}, \textit{@next}, \textit{@prev}, \textit{@exclude}, \textit{@select})) (\hyperref[TEI.att.global.analytic]{att.global.analytic} (\textit{@ana})) (\hyperref[TEI.att.global.facs]{att.global.facs} (\textit{@facs})) (\hyperref[TEI.att.global.change]{att.global.change} (\textit{@change})) (\hyperref[TEI.att.global.responsibility]{att.global.responsibility} (\textit{@cert}, \textit{@resp})) (\hyperref[TEI.att.global.source]{att.global.source} (\textit{@source})) \hyperref[TEI.att.segLike]{att.segLike} (\textit{@function})  (\hyperref[TEI.att.datcat]{att.datcat} (\textit{@datcat}, \textit{@valueDatcat})) (\hyperref[TEI.att.fragmentable]{att.fragmentable} (\textit{@part})) \hyperref[TEI.att.typed]{att.typed} (\textit{@type}, \textit{@subtype}) \hfil\\[-10pt]\begin{sansreflist}
    \item[@lemma]
  fournit le lemme du mot (entrée du dictionnaire)
\begin{reflist}
    \item[{Statut}]
  Optionel
    \item[{Type de données}]
  \hyperref[TEI.teidata.text]{teidata.text}
\end{reflist}  
    \item[@lemmaRef]
  provides a pointer to a definition of the lemma for the word, for example in an online lexicon.
\begin{reflist}
    \item[{Statut}]
  Optionel
    \item[{Type de données}]
  \hyperref[TEI.teidata.pointer]{teidata.pointer}
\end{reflist}  
\end{sansreflist}  
    \item[{Membre du}]
  \hyperref[TEI.model.linePart]{model.linePart} \hyperref[TEI.model.segLike]{model.segLike} 
    \item[{Contenu dans}]
  
    \item[analysis: ]
   \hyperref[TEI.cl]{cl} \hyperref[TEI.phr]{phr} \hyperref[TEI.s]{s} \hyperref[TEI.w]{w}\par 
    \item[core: ]
   \hyperref[TEI.abbr]{abbr} \hyperref[TEI.add]{add} \hyperref[TEI.addrLine]{addrLine} \hyperref[TEI.author]{author} \hyperref[TEI.bibl]{bibl} \hyperref[TEI.biblScope]{biblScope} \hyperref[TEI.citedRange]{citedRange} \hyperref[TEI.corr]{corr} \hyperref[TEI.date]{date} \hyperref[TEI.del]{del} \hyperref[TEI.distinct]{distinct} \hyperref[TEI.editor]{editor} \hyperref[TEI.email]{email} \hyperref[TEI.emph]{emph} \hyperref[TEI.expan]{expan} \hyperref[TEI.foreign]{foreign} \hyperref[TEI.gloss]{gloss} \hyperref[TEI.head]{head} \hyperref[TEI.headItem]{headItem} \hyperref[TEI.headLabel]{headLabel} \hyperref[TEI.hi]{hi} \hyperref[TEI.item]{item} \hyperref[TEI.l]{l} \hyperref[TEI.label]{label} \hyperref[TEI.measure]{measure} \hyperref[TEI.mentioned]{mentioned} \hyperref[TEI.name]{name} \hyperref[TEI.note]{note} \hyperref[TEI.num]{num} \hyperref[TEI.orig]{orig} \hyperref[TEI.p]{p} \hyperref[TEI.pubPlace]{pubPlace} \hyperref[TEI.publisher]{publisher} \hyperref[TEI.q]{q} \hyperref[TEI.quote]{quote} \hyperref[TEI.ref]{ref} \hyperref[TEI.reg]{reg} \hyperref[TEI.rs]{rs} \hyperref[TEI.said]{said} \hyperref[TEI.sic]{sic} \hyperref[TEI.soCalled]{soCalled} \hyperref[TEI.speaker]{speaker} \hyperref[TEI.stage]{stage} \hyperref[TEI.street]{street} \hyperref[TEI.term]{term} \hyperref[TEI.textLang]{textLang} \hyperref[TEI.time]{time} \hyperref[TEI.title]{title} \hyperref[TEI.unclear]{unclear}\par 
    \item[figures: ]
   \hyperref[TEI.cell]{cell}\par 
    \item[header: ]
   \hyperref[TEI.change]{change} \hyperref[TEI.distributor]{distributor} \hyperref[TEI.edition]{edition} \hyperref[TEI.extent]{extent} \hyperref[TEI.licence]{licence}\par 
    \item[linking: ]
   \hyperref[TEI.ab]{ab} \hyperref[TEI.seg]{seg}\par 
    \item[msdescription: ]
   \hyperref[TEI.accMat]{accMat} \hyperref[TEI.acquisition]{acquisition} \hyperref[TEI.additions]{additions} \hyperref[TEI.catchwords]{catchwords} \hyperref[TEI.collation]{collation} \hyperref[TEI.colophon]{colophon} \hyperref[TEI.condition]{condition} \hyperref[TEI.custEvent]{custEvent} \hyperref[TEI.decoNote]{decoNote} \hyperref[TEI.explicit]{explicit} \hyperref[TEI.filiation]{filiation} \hyperref[TEI.finalRubric]{finalRubric} \hyperref[TEI.foliation]{foliation} \hyperref[TEI.heraldry]{heraldry} \hyperref[TEI.incipit]{incipit} \hyperref[TEI.layout]{layout} \hyperref[TEI.material]{material} \hyperref[TEI.musicNotation]{musicNotation} \hyperref[TEI.objectType]{objectType} \hyperref[TEI.origDate]{origDate} \hyperref[TEI.origPlace]{origPlace} \hyperref[TEI.origin]{origin} \hyperref[TEI.provenance]{provenance} \hyperref[TEI.rubric]{rubric} \hyperref[TEI.secFol]{secFol} \hyperref[TEI.signatures]{signatures} \hyperref[TEI.source]{source} \hyperref[TEI.stamp]{stamp} \hyperref[TEI.summary]{summary} \hyperref[TEI.support]{support} \hyperref[TEI.surrogates]{surrogates} \hyperref[TEI.typeNote]{typeNote} \hyperref[TEI.watermark]{watermark}\par 
    \item[namesdates: ]
   \hyperref[TEI.addName]{addName} \hyperref[TEI.affiliation]{affiliation} \hyperref[TEI.country]{country} \hyperref[TEI.forename]{forename} \hyperref[TEI.genName]{genName} \hyperref[TEI.geogName]{geogName} \hyperref[TEI.nameLink]{nameLink} \hyperref[TEI.orgName]{orgName} \hyperref[TEI.persName]{persName} \hyperref[TEI.placeName]{placeName} \hyperref[TEI.region]{region} \hyperref[TEI.roleName]{roleName} \hyperref[TEI.settlement]{settlement} \hyperref[TEI.surname]{surname}\par 
    \item[textstructure: ]
   \hyperref[TEI.docAuthor]{docAuthor} \hyperref[TEI.docDate]{docDate} \hyperref[TEI.docEdition]{docEdition} \hyperref[TEI.titlePart]{titlePart}\par 
    \item[transcr: ]
   \hyperref[TEI.damage]{damage} \hyperref[TEI.fw]{fw} \hyperref[TEI.line]{line} \hyperref[TEI.metamark]{metamark} \hyperref[TEI.mod]{mod} \hyperref[TEI.restore]{restore} \hyperref[TEI.retrace]{retrace} \hyperref[TEI.secl]{secl} \hyperref[TEI.supplied]{supplied} \hyperref[TEI.surplus]{surplus} \hyperref[TEI.zone]{zone}
    \item[{Peut contenir}]
  
    \item[analysis: ]
   \hyperref[TEI.c]{c} \hyperref[TEI.interp]{interp} \hyperref[TEI.interpGrp]{interpGrp} \hyperref[TEI.m]{m} \hyperref[TEI.pc]{pc} \hyperref[TEI.span]{span} \hyperref[TEI.spanGrp]{spanGrp} \hyperref[TEI.w]{w}\par 
    \item[core: ]
   \hyperref[TEI.abbr]{abbr} \hyperref[TEI.add]{add} \hyperref[TEI.cb]{cb} \hyperref[TEI.choice]{choice} \hyperref[TEI.corr]{corr} \hyperref[TEI.del]{del} \hyperref[TEI.expan]{expan} \hyperref[TEI.gap]{gap} \hyperref[TEI.gb]{gb} \hyperref[TEI.hi]{hi} \hyperref[TEI.index]{index} \hyperref[TEI.lb]{lb} \hyperref[TEI.milestone]{milestone} \hyperref[TEI.note]{note} \hyperref[TEI.orig]{orig} \hyperref[TEI.pb]{pb} \hyperref[TEI.reg]{reg} \hyperref[TEI.sic]{sic} \hyperref[TEI.unclear]{unclear}\par 
    \item[figures: ]
   \hyperref[TEI.figure]{figure} \hyperref[TEI.notatedMusic]{notatedMusic}\par 
    \item[iso-fs: ]
   \hyperref[TEI.fLib]{fLib} \hyperref[TEI.fs]{fs} \hyperref[TEI.fvLib]{fvLib}\par 
    \item[linking: ]
   \hyperref[TEI.alt]{alt} \hyperref[TEI.altGrp]{altGrp} \hyperref[TEI.anchor]{anchor} \hyperref[TEI.join]{join} \hyperref[TEI.joinGrp]{joinGrp} \hyperref[TEI.link]{link} \hyperref[TEI.linkGrp]{linkGrp} \hyperref[TEI.seg]{seg} \hyperref[TEI.timeline]{timeline}\par 
    \item[msdescription: ]
   \hyperref[TEI.source]{source}\par 
    \item[transcr: ]
   \hyperref[TEI.addSpan]{addSpan} \hyperref[TEI.am]{am} \hyperref[TEI.damage]{damage} \hyperref[TEI.damageSpan]{damageSpan} \hyperref[TEI.delSpan]{delSpan} \hyperref[TEI.ex]{ex} \hyperref[TEI.fw]{fw} \hyperref[TEI.handShift]{handShift} \hyperref[TEI.listTranspose]{listTranspose} \hyperref[TEI.metamark]{metamark} \hyperref[TEI.mod]{mod} \hyperref[TEI.redo]{redo} \hyperref[TEI.restore]{restore} \hyperref[TEI.retrace]{retrace} \hyperref[TEI.secl]{secl} \hyperref[TEI.space]{space} \hyperref[TEI.subst]{subst} \hyperref[TEI.substJoin]{substJoin} \hyperref[TEI.supplied]{supplied} \hyperref[TEI.surplus]{surplus} \hyperref[TEI.undo]{undo}\par des données textuelles
    \item[{Exemple}]
  \leavevmode\bgroup\exampleFont \begin{shaded}\noindent\mbox{}{<\textbf{w}\hspace*{6pt}{lemma}="{nage}"\mbox{}\newline 
\hspace*{6pt}{lemmaRef}="{http://www.example.com/lexicon/hitvb.xml}"\hspace*{6pt}{type}="{verb}">}nag{<\textbf{m}\hspace*{6pt}{type}="{suffix}">}er{</\textbf{m}>}\mbox{}\newline 
{</\textbf{w}>}\end{shaded}\egroup 


    \item[{Modèle de contenu}]
  \mbox{}\hfill\\[-10pt]\begin{Verbatim}[fontsize=\small]
<content>
 <alternate maxOccurs="unbounded"
  minOccurs="0">
  <textNode/>
  <classRef key="model.gLike"/>
  <elementRef key="seg"/>
  <elementRef key="w"/>
  <elementRef key="m"/>
  <elementRef key="c"/>
  <elementRef key="pc"/>
  <classRef key="model.global"/>
  <classRef key="model.lPart"/>
  <classRef key="model.hiLike"/>
  <classRef key="model.pPart.edit"/>
 </alternate>
</content>
    
\end{Verbatim}

    \item[{Schéma Declaration}]
  \mbox{}\hfill\\[-10pt]\begin{Verbatim}[fontsize=\small]
element w
{
   tei_att.global.attributes,
   tei_att.segLike.attributes,
   tei_att.typed.attributes,
   attribute lemma { text }?,
   attribute lemmaRef { text }?,
   (
      text
    | tei_model.gLike    | tei_seg    | tei_w    | tei_m    | tei_c    | tei_pc    | tei_model.global    | tei_model.lPart    | tei_model.hiLike    | tei_model.pPart.edit   )*
}
\end{Verbatim}

\end{reflist}  \index{watermark=<watermark>|oddindex}
\begin{reflist}
\item[]\begin{specHead}{TEI.watermark}{<watermark> }(filigrane) Contient un mot ou une expression décrivant un filigrane ou une marque du même genre. [\xref{http://www.tei-c.org/release/doc/tei-p5-doc/en/html/MS.html\#mswat}{10.3.3. Watermarks and Stamps}]\end{specHead} 
    \item[{Module}]
  msdescription
    \item[{Attributs}]
  Attributs \hyperref[TEI.att.global]{att.global} (\textit{@xml:id}, \textit{@n}, \textit{@xml:lang}, \textit{@xml:base}, \textit{@xml:space})  (\hyperref[TEI.att.global.rendition]{att.global.rendition} (\textit{@rend}, \textit{@style}, \textit{@rendition})) (\hyperref[TEI.att.global.linking]{att.global.linking} (\textit{@corresp}, \textit{@synch}, \textit{@sameAs}, \textit{@copyOf}, \textit{@next}, \textit{@prev}, \textit{@exclude}, \textit{@select})) (\hyperref[TEI.att.global.analytic]{att.global.analytic} (\textit{@ana})) (\hyperref[TEI.att.global.facs]{att.global.facs} (\textit{@facs})) (\hyperref[TEI.att.global.change]{att.global.change} (\textit{@change})) (\hyperref[TEI.att.global.responsibility]{att.global.responsibility} (\textit{@cert}, \textit{@resp})) (\hyperref[TEI.att.global.source]{att.global.source} (\textit{@source}))
    \item[{Membre du}]
  \hyperref[TEI.model.pPart.msdesc]{model.pPart.msdesc}
    \item[{Contenu dans}]
  
    \item[analysis: ]
   \hyperref[TEI.cl]{cl} \hyperref[TEI.phr]{phr} \hyperref[TEI.s]{s} \hyperref[TEI.span]{span}\par 
    \item[core: ]
   \hyperref[TEI.abbr]{abbr} \hyperref[TEI.add]{add} \hyperref[TEI.addrLine]{addrLine} \hyperref[TEI.author]{author} \hyperref[TEI.biblScope]{biblScope} \hyperref[TEI.citedRange]{citedRange} \hyperref[TEI.corr]{corr} \hyperref[TEI.date]{date} \hyperref[TEI.del]{del} \hyperref[TEI.desc]{desc} \hyperref[TEI.distinct]{distinct} \hyperref[TEI.editor]{editor} \hyperref[TEI.email]{email} \hyperref[TEI.emph]{emph} \hyperref[TEI.expan]{expan} \hyperref[TEI.foreign]{foreign} \hyperref[TEI.gloss]{gloss} \hyperref[TEI.head]{head} \hyperref[TEI.headItem]{headItem} \hyperref[TEI.headLabel]{headLabel} \hyperref[TEI.hi]{hi} \hyperref[TEI.item]{item} \hyperref[TEI.l]{l} \hyperref[TEI.label]{label} \hyperref[TEI.measure]{measure} \hyperref[TEI.meeting]{meeting} \hyperref[TEI.mentioned]{mentioned} \hyperref[TEI.name]{name} \hyperref[TEI.note]{note} \hyperref[TEI.num]{num} \hyperref[TEI.orig]{orig} \hyperref[TEI.p]{p} \hyperref[TEI.pubPlace]{pubPlace} \hyperref[TEI.publisher]{publisher} \hyperref[TEI.q]{q} \hyperref[TEI.quote]{quote} \hyperref[TEI.ref]{ref} \hyperref[TEI.reg]{reg} \hyperref[TEI.resp]{resp} \hyperref[TEI.rs]{rs} \hyperref[TEI.said]{said} \hyperref[TEI.sic]{sic} \hyperref[TEI.soCalled]{soCalled} \hyperref[TEI.speaker]{speaker} \hyperref[TEI.stage]{stage} \hyperref[TEI.street]{street} \hyperref[TEI.term]{term} \hyperref[TEI.textLang]{textLang} \hyperref[TEI.time]{time} \hyperref[TEI.title]{title} \hyperref[TEI.unclear]{unclear}\par 
    \item[figures: ]
   \hyperref[TEI.cell]{cell} \hyperref[TEI.figDesc]{figDesc}\par 
    \item[header: ]
   \hyperref[TEI.authority]{authority} \hyperref[TEI.change]{change} \hyperref[TEI.classCode]{classCode} \hyperref[TEI.creation]{creation} \hyperref[TEI.distributor]{distributor} \hyperref[TEI.edition]{edition} \hyperref[TEI.extent]{extent} \hyperref[TEI.funder]{funder} \hyperref[TEI.language]{language} \hyperref[TEI.licence]{licence} \hyperref[TEI.rendition]{rendition}\par 
    \item[iso-fs: ]
   \hyperref[TEI.fDescr]{fDescr} \hyperref[TEI.fsDescr]{fsDescr}\par 
    \item[linking: ]
   \hyperref[TEI.ab]{ab} \hyperref[TEI.seg]{seg}\par 
    \item[msdescription: ]
   \hyperref[TEI.accMat]{accMat} \hyperref[TEI.acquisition]{acquisition} \hyperref[TEI.additions]{additions} \hyperref[TEI.catchwords]{catchwords} \hyperref[TEI.collation]{collation} \hyperref[TEI.colophon]{colophon} \hyperref[TEI.condition]{condition} \hyperref[TEI.custEvent]{custEvent} \hyperref[TEI.decoNote]{decoNote} \hyperref[TEI.explicit]{explicit} \hyperref[TEI.filiation]{filiation} \hyperref[TEI.finalRubric]{finalRubric} \hyperref[TEI.foliation]{foliation} \hyperref[TEI.heraldry]{heraldry} \hyperref[TEI.incipit]{incipit} \hyperref[TEI.layout]{layout} \hyperref[TEI.material]{material} \hyperref[TEI.musicNotation]{musicNotation} \hyperref[TEI.objectType]{objectType} \hyperref[TEI.origDate]{origDate} \hyperref[TEI.origPlace]{origPlace} \hyperref[TEI.origin]{origin} \hyperref[TEI.provenance]{provenance} \hyperref[TEI.rubric]{rubric} \hyperref[TEI.secFol]{secFol} \hyperref[TEI.signatures]{signatures} \hyperref[TEI.source]{source} \hyperref[TEI.stamp]{stamp} \hyperref[TEI.summary]{summary} \hyperref[TEI.support]{support} \hyperref[TEI.surrogates]{surrogates} \hyperref[TEI.typeNote]{typeNote} \hyperref[TEI.watermark]{watermark}\par 
    \item[namesdates: ]
   \hyperref[TEI.addName]{addName} \hyperref[TEI.affiliation]{affiliation} \hyperref[TEI.country]{country} \hyperref[TEI.forename]{forename} \hyperref[TEI.genName]{genName} \hyperref[TEI.geogName]{geogName} \hyperref[TEI.nameLink]{nameLink} \hyperref[TEI.orgName]{orgName} \hyperref[TEI.persName]{persName} \hyperref[TEI.placeName]{placeName} \hyperref[TEI.region]{region} \hyperref[TEI.roleName]{roleName} \hyperref[TEI.settlement]{settlement} \hyperref[TEI.surname]{surname}\par 
    \item[textstructure: ]
   \hyperref[TEI.docAuthor]{docAuthor} \hyperref[TEI.docDate]{docDate} \hyperref[TEI.docEdition]{docEdition} \hyperref[TEI.titlePart]{titlePart}\par 
    \item[transcr: ]
   \hyperref[TEI.damage]{damage} \hyperref[TEI.fw]{fw} \hyperref[TEI.metamark]{metamark} \hyperref[TEI.mod]{mod} \hyperref[TEI.restore]{restore} \hyperref[TEI.retrace]{retrace} \hyperref[TEI.secl]{secl} \hyperref[TEI.supplied]{supplied} \hyperref[TEI.surplus]{surplus}
    \item[{Peut contenir}]
  
    \item[analysis: ]
   \hyperref[TEI.c]{c} \hyperref[TEI.cl]{cl} \hyperref[TEI.interp]{interp} \hyperref[TEI.interpGrp]{interpGrp} \hyperref[TEI.m]{m} \hyperref[TEI.pc]{pc} \hyperref[TEI.phr]{phr} \hyperref[TEI.s]{s} \hyperref[TEI.span]{span} \hyperref[TEI.spanGrp]{spanGrp} \hyperref[TEI.w]{w}\par 
    \item[core: ]
   \hyperref[TEI.abbr]{abbr} \hyperref[TEI.add]{add} \hyperref[TEI.address]{address} \hyperref[TEI.binaryObject]{binaryObject} \hyperref[TEI.cb]{cb} \hyperref[TEI.choice]{choice} \hyperref[TEI.corr]{corr} \hyperref[TEI.date]{date} \hyperref[TEI.del]{del} \hyperref[TEI.distinct]{distinct} \hyperref[TEI.email]{email} \hyperref[TEI.emph]{emph} \hyperref[TEI.expan]{expan} \hyperref[TEI.foreign]{foreign} \hyperref[TEI.gap]{gap} \hyperref[TEI.gb]{gb} \hyperref[TEI.gloss]{gloss} \hyperref[TEI.graphic]{graphic} \hyperref[TEI.hi]{hi} \hyperref[TEI.index]{index} \hyperref[TEI.lb]{lb} \hyperref[TEI.measure]{measure} \hyperref[TEI.measureGrp]{measureGrp} \hyperref[TEI.media]{media} \hyperref[TEI.mentioned]{mentioned} \hyperref[TEI.milestone]{milestone} \hyperref[TEI.name]{name} \hyperref[TEI.note]{note} \hyperref[TEI.num]{num} \hyperref[TEI.orig]{orig} \hyperref[TEI.pb]{pb} \hyperref[TEI.ptr]{ptr} \hyperref[TEI.ref]{ref} \hyperref[TEI.reg]{reg} \hyperref[TEI.rs]{rs} \hyperref[TEI.sic]{sic} \hyperref[TEI.soCalled]{soCalled} \hyperref[TEI.term]{term} \hyperref[TEI.time]{time} \hyperref[TEI.title]{title} \hyperref[TEI.unclear]{unclear}\par 
    \item[derived-module-tei.istex: ]
   \hyperref[TEI.math]{math} \hyperref[TEI.mrow]{mrow}\par 
    \item[figures: ]
   \hyperref[TEI.figure]{figure} \hyperref[TEI.formula]{formula} \hyperref[TEI.notatedMusic]{notatedMusic}\par 
    \item[header: ]
   \hyperref[TEI.idno]{idno}\par 
    \item[iso-fs: ]
   \hyperref[TEI.fLib]{fLib} \hyperref[TEI.fs]{fs} \hyperref[TEI.fvLib]{fvLib}\par 
    \item[linking: ]
   \hyperref[TEI.alt]{alt} \hyperref[TEI.altGrp]{altGrp} \hyperref[TEI.anchor]{anchor} \hyperref[TEI.join]{join} \hyperref[TEI.joinGrp]{joinGrp} \hyperref[TEI.link]{link} \hyperref[TEI.linkGrp]{linkGrp} \hyperref[TEI.seg]{seg} \hyperref[TEI.timeline]{timeline}\par 
    \item[msdescription: ]
   \hyperref[TEI.catchwords]{catchwords} \hyperref[TEI.depth]{depth} \hyperref[TEI.dim]{dim} \hyperref[TEI.dimensions]{dimensions} \hyperref[TEI.height]{height} \hyperref[TEI.heraldry]{heraldry} \hyperref[TEI.locus]{locus} \hyperref[TEI.locusGrp]{locusGrp} \hyperref[TEI.material]{material} \hyperref[TEI.objectType]{objectType} \hyperref[TEI.origDate]{origDate} \hyperref[TEI.origPlace]{origPlace} \hyperref[TEI.secFol]{secFol} \hyperref[TEI.signatures]{signatures} \hyperref[TEI.source]{source} \hyperref[TEI.stamp]{stamp} \hyperref[TEI.watermark]{watermark} \hyperref[TEI.width]{width}\par 
    \item[namesdates: ]
   \hyperref[TEI.addName]{addName} \hyperref[TEI.affiliation]{affiliation} \hyperref[TEI.country]{country} \hyperref[TEI.forename]{forename} \hyperref[TEI.genName]{genName} \hyperref[TEI.geogName]{geogName} \hyperref[TEI.location]{location} \hyperref[TEI.nameLink]{nameLink} \hyperref[TEI.orgName]{orgName} \hyperref[TEI.persName]{persName} \hyperref[TEI.placeName]{placeName} \hyperref[TEI.region]{region} \hyperref[TEI.roleName]{roleName} \hyperref[TEI.settlement]{settlement} \hyperref[TEI.state]{state} \hyperref[TEI.surname]{surname}\par 
    \item[spoken: ]
   \hyperref[TEI.annotationBlock]{annotationBlock}\par 
    \item[transcr: ]
   \hyperref[TEI.addSpan]{addSpan} \hyperref[TEI.am]{am} \hyperref[TEI.damage]{damage} \hyperref[TEI.damageSpan]{damageSpan} \hyperref[TEI.delSpan]{delSpan} \hyperref[TEI.ex]{ex} \hyperref[TEI.fw]{fw} \hyperref[TEI.handShift]{handShift} \hyperref[TEI.listTranspose]{listTranspose} \hyperref[TEI.metamark]{metamark} \hyperref[TEI.mod]{mod} \hyperref[TEI.redo]{redo} \hyperref[TEI.restore]{restore} \hyperref[TEI.retrace]{retrace} \hyperref[TEI.secl]{secl} \hyperref[TEI.space]{space} \hyperref[TEI.subst]{subst} \hyperref[TEI.substJoin]{substJoin} \hyperref[TEI.supplied]{supplied} \hyperref[TEI.surplus]{surplus} \hyperref[TEI.undo]{undo}\par des données textuelles
    \item[{Exemple}]
  \leavevmode\bgroup\exampleFont \begin{shaded}\noindent\mbox{}\mbox{}\newline 
\textit{<!-- Description des gardes : gardes blanches ; gardes couleurs (marbrées, gaufrées, peintes, dominotées, etc.) généralement suivies de gardes blanches ; dans tous les cas, spécifier le nombre de gardes (début + fin du volume)-->}{<\textbf{decoNote}\hspace*{6pt}{type}="{gardes}">}Gardes (3+2), filigrane {<\textbf{watermark}>}B{</\textbf{watermark}>}. {</\textbf{decoNote}>}\end{shaded}\egroup 


    \item[{Modèle de contenu}]
  \mbox{}\hfill\\[-10pt]\begin{Verbatim}[fontsize=\small]
<content>
 <macroRef key="macro.phraseSeq"/>
</content>
    
\end{Verbatim}

    \item[{Schéma Declaration}]
  \mbox{}\hfill\\[-10pt]\begin{Verbatim}[fontsize=\small]
element watermark { tei_att.global.attributes, tei_macro.phraseSeq }
\end{Verbatim}

\end{reflist}  \index{when=<when>|oddindex}\index{absolute=@absolute!<when>|oddindex}\index{unit=@unit!<when>|oddindex}\index{interval=@interval!<when>|oddindex}\index{since=@since!<when>|oddindex}
\begin{reflist}
\item[]\begin{specHead}{TEI.when}{<when> }indique un point dans le temps, soit relatif à d'autres éléments de l'élément \hyperref[TEI.timeline]{<timeline>} dans lequel il est contenu, soit dans l'absolu. [\xref{http://www.tei-c.org/release/doc/tei-p5-doc/en/html/SA.html\#SASYMP}{16.4.2. Placing Synchronous Events in Time}]\end{specHead} 
    \item[{Module}]
  linking
    \item[{Attributs}]
  Attributs \hyperref[TEI.att.global]{att.global} (\textit{@xml:id}, \textit{@n}, \textit{@xml:lang}, \textit{@xml:base}, \textit{@xml:space})  (\hyperref[TEI.att.global.rendition]{att.global.rendition} (\textit{@rend}, \textit{@style}, \textit{@rendition})) (\hyperref[TEI.att.global.linking]{att.global.linking} (\textit{@corresp}, \textit{@synch}, \textit{@sameAs}, \textit{@copyOf}, \textit{@next}, \textit{@prev}, \textit{@exclude}, \textit{@select})) (\hyperref[TEI.att.global.analytic]{att.global.analytic} (\textit{@ana})) (\hyperref[TEI.att.global.facs]{att.global.facs} (\textit{@facs})) (\hyperref[TEI.att.global.change]{att.global.change} (\textit{@change})) (\hyperref[TEI.att.global.responsibility]{att.global.responsibility} (\textit{@cert}, \textit{@resp})) (\hyperref[TEI.att.global.source]{att.global.source} (\textit{@source})) \hfil\\[-10pt]\begin{sansreflist}
    \item[@absolute]
  contient une valeur temporelle absolue.
\begin{reflist}
    \item[{Statut}]
  Optionel
    \item[{Type de données}]
  \hyperref[TEI.teidata.temporal.w3c]{teidata.temporal.w3c}
    \item[{Note}]
  \par
Cet attribut est obligatoire pour l'élément \hyperref[TEI.when]{<when>} qui est désigné comme cible par l'attribut {\itshape origin} de l'élément <timeline>.
\end{reflist}  
    \item[@unit]
  spécifie l'unité de temps dans laquelle la valeur de l'attribut {\itshape interval} est exprimée, si elle n'est pas héritée de l'élément parent \texttt{<timeLine>}.
\begin{reflist}
    \item[{Statut}]
  Optionel
    \item[{Type de données}]
  \hyperref[TEI.teidata.enumerated]{teidata.enumerated}
    \item[{Les valeurs suggérées comprennent:}]
  \begin{description}

\item[{d}](jours)
\item[{h}](heures)
\item[{min}](minutes)
\item[{s}](secondes)
\item[{ms}](millisecondes)
\end{description} 
\end{reflist}  
    \item[@interval]
  spécifie la partie numérique d'un intervalle de temps.
\begin{reflist}
    \item[{Statut}]
  Optionel
    \item[{Type de données}]
  \hyperref[TEI.teidata.interval]{teidata.interval}
\end{reflist}  
    \item[@since]
  identifie le point de référence pour déterminer la date ou l'heure de l'élément courant \hyperref[TEI.when]{<when>} : cette date ou cette heure s'obtiennent en ajoutant la valeur de l'intervalle à la date du point de référence.
\begin{reflist}
    \item[{Statut}]
  Optionel
    \item[{Type de données}]
  \hyperref[TEI.teidata.pointer]{teidata.pointer}
    \item[{Note}]
  \par
Si cet attribut est omis, et qu'il n'y a pas d'attribut {\itshape absolute}, le point de référence retenu est alors l'attribut {\itshape origin}de l'élément englobant \hyperref[TEI.timeline]{<timeline>}.
\end{reflist}  
\end{sansreflist}  
    \item[{Contenu dans}]
  
    \item[linking: ]
   \hyperref[TEI.timeline]{timeline}
    \item[{Peut contenir}]
  Elément vide
    \item[{Note}]
  \par
L'élément \hyperref[TEI.when]{<when>} doit avoir un attribut global {\itshape xml:id} pour identifier ce point dans le temps. La valeur utilisée peut être choisie librement, pourvu qu'elle soit unique dans le document et que le nom soit syntaxiquement valide. Les valeurs contenant des nombres ne doivent pas nécessairement former une séquence.
    \item[{Exemple}]
  \leavevmode\bgroup\exampleFont \begin{shaded}\noindent\mbox{}{<\textbf{when}\hspace*{6pt}{interval}="{20}"\hspace*{6pt}{since}="{\#w2}"\hspace*{6pt}{xml:id}="{TW3}"/>}\end{shaded}\egroup 


    \item[{Exemple}]
  \leavevmode\bgroup\exampleFont \begin{shaded}\noindent\mbox{}{<\textbf{when}\hspace*{6pt}{interval}="{20}"\hspace*{6pt}{since}="{\#fr\textunderscore w2}"\mbox{}\newline 
\hspace*{6pt}{xml:id}="{fr\textunderscore TW3}"/>}\end{shaded}\egroup 


    \item[{Modèle de contenu}]
  \fbox{\ttfamily <content>\newline
</content>\newline
    } 
    \item[{Schéma Declaration}]
  \mbox{}\hfill\\[-10pt]\begin{Verbatim}[fontsize=\small]
element when
{
   tei_att.global.attributes,
   attribute absolute { text }?,
   attribute unit { "d" | "h" | "min" | "s" | "ms" }?,
   attribute interval { text }?,
   attribute since { text }?,
   empty
}
\end{Verbatim}

\end{reflist}  \index{width=<width>|oddindex}
\begin{reflist}
\item[]\begin{specHead}{TEI.width}{<width> }(largeur) contient une dimension mesurée sur l'axe perpendiculaire au dos du manuscrit. [\xref{http://www.tei-c.org/release/doc/tei-p5-doc/en/html/MS.html\#msdim}{10.3.4. Dimensions}]\end{specHead} 
    \item[{Module}]
  msdescription
    \item[{Attributs}]
  Attributs \hyperref[TEI.att.global]{att.global} (\textit{@xml:id}, \textit{@n}, \textit{@xml:lang}, \textit{@xml:base}, \textit{@xml:space})  (\hyperref[TEI.att.global.rendition]{att.global.rendition} (\textit{@rend}, \textit{@style}, \textit{@rendition})) (\hyperref[TEI.att.global.linking]{att.global.linking} (\textit{@corresp}, \textit{@synch}, \textit{@sameAs}, \textit{@copyOf}, \textit{@next}, \textit{@prev}, \textit{@exclude}, \textit{@select})) (\hyperref[TEI.att.global.analytic]{att.global.analytic} (\textit{@ana})) (\hyperref[TEI.att.global.facs]{att.global.facs} (\textit{@facs})) (\hyperref[TEI.att.global.change]{att.global.change} (\textit{@change})) (\hyperref[TEI.att.global.responsibility]{att.global.responsibility} (\textit{@cert}, \textit{@resp})) (\hyperref[TEI.att.global.source]{att.global.source} (\textit{@source})) \hyperref[TEI.att.dimensions]{att.dimensions} (\textit{@unit}, \textit{@quantity}, \textit{@extent}, \textit{@precision}, \textit{@scope})  (\hyperref[TEI.att.ranging]{att.ranging} (\textit{@atLeast}, \textit{@atMost}, \textit{@min}, \textit{@max}, \textit{@confidence}))
    \item[{Membre du}]
  \hyperref[TEI.model.dimLike]{model.dimLike} \hyperref[TEI.model.measureLike]{model.measureLike}Elément: \begin{itemize}
\item \hyperref[TEI.mpadded]{mpadded}/@width
\item \hyperref[TEI.mspace]{mspace}/@width
\end{itemize} 
    \item[{Contenu dans}]
  
    \item[analysis: ]
   \hyperref[TEI.cl]{cl} \hyperref[TEI.phr]{phr} \hyperref[TEI.s]{s} \hyperref[TEI.span]{span}\par 
    \item[core: ]
   \hyperref[TEI.abbr]{abbr} \hyperref[TEI.add]{add} \hyperref[TEI.addrLine]{addrLine} \hyperref[TEI.author]{author} \hyperref[TEI.bibl]{bibl} \hyperref[TEI.biblScope]{biblScope} \hyperref[TEI.citedRange]{citedRange} \hyperref[TEI.corr]{corr} \hyperref[TEI.date]{date} \hyperref[TEI.del]{del} \hyperref[TEI.desc]{desc} \hyperref[TEI.distinct]{distinct} \hyperref[TEI.editor]{editor} \hyperref[TEI.email]{email} \hyperref[TEI.emph]{emph} \hyperref[TEI.expan]{expan} \hyperref[TEI.foreign]{foreign} \hyperref[TEI.gloss]{gloss} \hyperref[TEI.head]{head} \hyperref[TEI.headItem]{headItem} \hyperref[TEI.headLabel]{headLabel} \hyperref[TEI.hi]{hi} \hyperref[TEI.item]{item} \hyperref[TEI.l]{l} \hyperref[TEI.label]{label} \hyperref[TEI.measure]{measure} \hyperref[TEI.measureGrp]{measureGrp} \hyperref[TEI.meeting]{meeting} \hyperref[TEI.mentioned]{mentioned} \hyperref[TEI.name]{name} \hyperref[TEI.note]{note} \hyperref[TEI.num]{num} \hyperref[TEI.orig]{orig} \hyperref[TEI.p]{p} \hyperref[TEI.pubPlace]{pubPlace} \hyperref[TEI.publisher]{publisher} \hyperref[TEI.q]{q} \hyperref[TEI.quote]{quote} \hyperref[TEI.ref]{ref} \hyperref[TEI.reg]{reg} \hyperref[TEI.resp]{resp} \hyperref[TEI.rs]{rs} \hyperref[TEI.said]{said} \hyperref[TEI.sic]{sic} \hyperref[TEI.soCalled]{soCalled} \hyperref[TEI.speaker]{speaker} \hyperref[TEI.stage]{stage} \hyperref[TEI.street]{street} \hyperref[TEI.term]{term} \hyperref[TEI.textLang]{textLang} \hyperref[TEI.time]{time} \hyperref[TEI.title]{title} \hyperref[TEI.unclear]{unclear}\par 
    \item[figures: ]
   \hyperref[TEI.cell]{cell} \hyperref[TEI.figDesc]{figDesc}\par 
    \item[header: ]
   \hyperref[TEI.authority]{authority} \hyperref[TEI.change]{change} \hyperref[TEI.classCode]{classCode} \hyperref[TEI.creation]{creation} \hyperref[TEI.distributor]{distributor} \hyperref[TEI.edition]{edition} \hyperref[TEI.extent]{extent} \hyperref[TEI.funder]{funder} \hyperref[TEI.language]{language} \hyperref[TEI.licence]{licence} \hyperref[TEI.rendition]{rendition}\par 
    \item[iso-fs: ]
   \hyperref[TEI.fDescr]{fDescr} \hyperref[TEI.fsDescr]{fsDescr}\par 
    \item[linking: ]
   \hyperref[TEI.ab]{ab} \hyperref[TEI.seg]{seg}\par 
    \item[msdescription: ]
   \hyperref[TEI.accMat]{accMat} \hyperref[TEI.acquisition]{acquisition} \hyperref[TEI.additions]{additions} \hyperref[TEI.catchwords]{catchwords} \hyperref[TEI.collation]{collation} \hyperref[TEI.colophon]{colophon} \hyperref[TEI.condition]{condition} \hyperref[TEI.custEvent]{custEvent} \hyperref[TEI.decoNote]{decoNote} \hyperref[TEI.dimensions]{dimensions} \hyperref[TEI.explicit]{explicit} \hyperref[TEI.filiation]{filiation} \hyperref[TEI.finalRubric]{finalRubric} \hyperref[TEI.foliation]{foliation} \hyperref[TEI.heraldry]{heraldry} \hyperref[TEI.incipit]{incipit} \hyperref[TEI.layout]{layout} \hyperref[TEI.material]{material} \hyperref[TEI.musicNotation]{musicNotation} \hyperref[TEI.objectType]{objectType} \hyperref[TEI.origDate]{origDate} \hyperref[TEI.origPlace]{origPlace} \hyperref[TEI.origin]{origin} \hyperref[TEI.provenance]{provenance} \hyperref[TEI.rubric]{rubric} \hyperref[TEI.secFol]{secFol} \hyperref[TEI.signatures]{signatures} \hyperref[TEI.source]{source} \hyperref[TEI.stamp]{stamp} \hyperref[TEI.summary]{summary} \hyperref[TEI.support]{support} \hyperref[TEI.surrogates]{surrogates} \hyperref[TEI.typeNote]{typeNote} \hyperref[TEI.watermark]{watermark}\par 
    \item[namesdates: ]
   \hyperref[TEI.addName]{addName} \hyperref[TEI.affiliation]{affiliation} \hyperref[TEI.country]{country} \hyperref[TEI.forename]{forename} \hyperref[TEI.genName]{genName} \hyperref[TEI.geogName]{geogName} \hyperref[TEI.location]{location} \hyperref[TEI.nameLink]{nameLink} \hyperref[TEI.orgName]{orgName} \hyperref[TEI.persName]{persName} \hyperref[TEI.placeName]{placeName} \hyperref[TEI.region]{region} \hyperref[TEI.roleName]{roleName} \hyperref[TEI.settlement]{settlement} \hyperref[TEI.surname]{surname}\par 
    \item[textstructure: ]
   \hyperref[TEI.docAuthor]{docAuthor} \hyperref[TEI.docDate]{docDate} \hyperref[TEI.docEdition]{docEdition} \hyperref[TEI.titlePart]{titlePart}\par 
    \item[transcr: ]
   \hyperref[TEI.damage]{damage} \hyperref[TEI.fw]{fw} \hyperref[TEI.metamark]{metamark} \hyperref[TEI.mod]{mod} \hyperref[TEI.restore]{restore} \hyperref[TEI.retrace]{retrace} \hyperref[TEI.secl]{secl} \hyperref[TEI.supplied]{supplied} \hyperref[TEI.surplus]{surplus}
    \item[{Peut contenir}]
  Des données textuelles uniquement
    \item[{Note}]
  \par
If used to specify the depth of a non text-bearing portion of some object, for example a monument, this element conventionally refers to the axis facing the observer, and perpendicular to that indicated by the ‘depth’ axis.
    \item[{Exemple}]
  \leavevmode\bgroup\exampleFont \begin{shaded}\noindent\mbox{}{<\textbf{width}\hspace*{6pt}{unit}="{in}">}4{</\textbf{width}>}\end{shaded}\egroup 


    \item[{Exemple}]
  \leavevmode\bgroup\exampleFont \begin{shaded}\noindent\mbox{}{<\textbf{width}\hspace*{6pt}{unit}="{mm}">}240{</\textbf{width}>}\end{shaded}\egroup 


    \item[{Modèle de contenu}]
  \fbox{\ttfamily <content>\newline
 <macroRef key="macro.xtext"/>\newline
</content>\newline
    } 
    \item[{Schéma Declaration}]
  \mbox{}\hfill\\[-10pt]\begin{Verbatim}[fontsize=\small]
element width
{
   tei_att.global.attributes,
   tei_att.dimensions.attributes,
   tei_macro.xtext}
\end{Verbatim}

\end{reflist}  \index{zone=<zone>|oddindex}\index{rotate=@rotate!<zone>|oddindex}
\begin{reflist}
\item[]\begin{specHead}{TEI.zone}{<zone> }définit une zone rectangulaire contenue dans un élément \hyperref[TEI.surface]{<surface>}. [\xref{http://www.tei-c.org/release/doc/tei-p5-doc/en/html/PH.html\#PHFAX}{11.1. Digital Facsimiles} \xref{http://www.tei-c.org/release/doc/tei-p5-doc/en/html/PH.html\#PHZLAB}{11.2.2. Embedded Transcription}]\end{specHead} 
    \item[{Module}]
  transcr
    \item[{Attributs}]
  Attributs \hyperref[TEI.att.global]{att.global} (\textit{@xml:id}, \textit{@n}, \textit{@xml:lang}, \textit{@xml:base}, \textit{@xml:space})  (\hyperref[TEI.att.global.rendition]{att.global.rendition} (\textit{@rend}, \textit{@style}, \textit{@rendition})) (\hyperref[TEI.att.global.linking]{att.global.linking} (\textit{@corresp}, \textit{@synch}, \textit{@sameAs}, \textit{@copyOf}, \textit{@next}, \textit{@prev}, \textit{@exclude}, \textit{@select})) (\hyperref[TEI.att.global.analytic]{att.global.analytic} (\textit{@ana})) (\hyperref[TEI.att.global.facs]{att.global.facs} (\textit{@facs})) (\hyperref[TEI.att.global.change]{att.global.change} (\textit{@change})) (\hyperref[TEI.att.global.responsibility]{att.global.responsibility} (\textit{@cert}, \textit{@resp})) (\hyperref[TEI.att.global.source]{att.global.source} (\textit{@source})) \hyperref[TEI.att.coordinated]{att.coordinated} (\textit{@start}, \textit{@ulx}, \textit{@uly}, \textit{@lrx}, \textit{@lry}, \textit{@points}) \hyperref[TEI.att.typed]{att.typed} (\textit{@type}, \textit{@subtype}) \hyperref[TEI.att.written]{att.written} (\textit{@hand}) \hfil\\[-10pt]\begin{sansreflist}
    \item[@rotate]
  indicates the amount by which this zone has been rotated clockwise, with respect to the normal orientation of the parent \hyperref[TEI.surface]{<surface>} element as implied by the dimensions given in the \hyperref[TEI.msDesc]{<msDesc>} element or by the coordinates of the \hyperref[TEI.surface]{<surface>} itself. The orientation is expressed in arc degrees.
\begin{reflist}
    \item[{Statut}]
  Optionel
    \item[{Type de données}]
  \hyperref[TEI.teidata.count]{teidata.count}
    \item[{Valeur par défaut}]
  0
\end{reflist}  
\end{sansreflist}  
    \item[{Membre du}]
  \hyperref[TEI.model.annotation]{model.annotation} \hyperref[TEI.model.linePart]{model.linePart} 
    \item[{Contenu dans}]
  
    \item[spoken: ]
   \hyperref[TEI.annotationBlock]{annotationBlock}\par 
    \item[standOff: ]
   \hyperref[TEI.listAnnotation]{listAnnotation}\par 
    \item[transcr: ]
   \hyperref[TEI.line]{line} \hyperref[TEI.surface]{surface} \hyperref[TEI.zone]{zone}
    \item[{Peut contenir}]
  
    \item[analysis: ]
   \hyperref[TEI.c]{c} \hyperref[TEI.interp]{interp} \hyperref[TEI.interpGrp]{interpGrp} \hyperref[TEI.pc]{pc} \hyperref[TEI.span]{span} \hyperref[TEI.spanGrp]{spanGrp} \hyperref[TEI.w]{w}\par 
    \item[core: ]
   \hyperref[TEI.add]{add} \hyperref[TEI.binaryObject]{binaryObject} \hyperref[TEI.cb]{cb} \hyperref[TEI.choice]{choice} \hyperref[TEI.del]{del} \hyperref[TEI.gap]{gap} \hyperref[TEI.gb]{gb} \hyperref[TEI.graphic]{graphic} \hyperref[TEI.hi]{hi} \hyperref[TEI.index]{index} \hyperref[TEI.lb]{lb} \hyperref[TEI.media]{media} \hyperref[TEI.milestone]{milestone} \hyperref[TEI.note]{note} \hyperref[TEI.pb]{pb} \hyperref[TEI.unclear]{unclear}\par 
    \item[derived-module-tei.istex: ]
   \hyperref[TEI.math]{math} \hyperref[TEI.mrow]{mrow}\par 
    \item[figures: ]
   \hyperref[TEI.figure]{figure} \hyperref[TEI.formula]{formula} \hyperref[TEI.notatedMusic]{notatedMusic}\par 
    \item[iso-fs: ]
   \hyperref[TEI.fLib]{fLib} \hyperref[TEI.fs]{fs} \hyperref[TEI.fvLib]{fvLib}\par 
    \item[linking: ]
   \hyperref[TEI.alt]{alt} \hyperref[TEI.altGrp]{altGrp} \hyperref[TEI.anchor]{anchor} \hyperref[TEI.join]{join} \hyperref[TEI.joinGrp]{joinGrp} \hyperref[TEI.link]{link} \hyperref[TEI.linkGrp]{linkGrp} \hyperref[TEI.seg]{seg} \hyperref[TEI.timeline]{timeline}\par 
    \item[msdescription: ]
   \hyperref[TEI.source]{source}\par 
    \item[transcr: ]
   \hyperref[TEI.addSpan]{addSpan} \hyperref[TEI.damage]{damage} \hyperref[TEI.damageSpan]{damageSpan} \hyperref[TEI.delSpan]{delSpan} \hyperref[TEI.fw]{fw} \hyperref[TEI.handShift]{handShift} \hyperref[TEI.line]{line} \hyperref[TEI.listTranspose]{listTranspose} \hyperref[TEI.metamark]{metamark} \hyperref[TEI.mod]{mod} \hyperref[TEI.redo]{redo} \hyperref[TEI.restore]{restore} \hyperref[TEI.retrace]{retrace} \hyperref[TEI.space]{space} \hyperref[TEI.substJoin]{substJoin} \hyperref[TEI.surface]{surface} \hyperref[TEI.undo]{undo} \hyperref[TEI.zone]{zone}\par des données textuelles
    \item[{Note}]
  \par
La position de chaque zone pour une surface donnée est toujours définie par rapport au système de coordonnées défini pour cette surface. Tout élément graphique contenu par une zone se représente par toute la zone.
    \item[{Exemple}]
  \leavevmode\bgroup\exampleFont \begin{shaded}\noindent\mbox{}{<\textbf{surface}\hspace*{6pt}{lrx}="{0}"\hspace*{6pt}{lry}="{0}"\hspace*{6pt}{ulx}="{14.54}"\mbox{}\newline 
\hspace*{6pt}{uly}="{16.14}">}\mbox{}\newline 
\hspace*{6pt}{<\textbf{graphic}\hspace*{6pt}{url}="{stone.jpg}"/>}\mbox{}\newline 
\hspace*{6pt}{<\textbf{zone}\hspace*{6pt}{points}="{4.6,6.3 5.25,5.85 6.2,6.6 8.19222,7.4125 9.89222,6.5875 10.9422,6.1375 
 11.4422,6.7125 8.21722,8.3125 6.2,7.65}"/>}\mbox{}\newline 
{</\textbf{surface}>}\end{shaded}\egroup 

This example defines a non-rectangular zone: see the illustration in section PH-surfzone.
    \item[{Exemple}]
  \leavevmode\bgroup\exampleFont \begin{shaded}\noindent\mbox{}{<\textbf{facsimile}>}\mbox{}\newline 
\hspace*{6pt}{<\textbf{surface}\hspace*{6pt}{lrx}="{400}"\hspace*{6pt}{lry}="{280}"\hspace*{6pt}{ulx}="{50}"\mbox{}\newline 
\hspace*{6pt}\hspace*{6pt}{uly}="{20}">}\mbox{}\newline 
\hspace*{6pt}\hspace*{6pt}{<\textbf{zone}\hspace*{6pt}{lrx}="{500}"\hspace*{6pt}{lry}="{321}"\hspace*{6pt}{ulx}="{0}"\hspace*{6pt}{uly}="{0}">}\mbox{}\newline 
\hspace*{6pt}\hspace*{6pt}\hspace*{6pt}{<\textbf{graphic}\hspace*{6pt}{url}="{graphic.png }"/>}\mbox{}\newline 
\hspace*{6pt}\hspace*{6pt}{</\textbf{zone}>}\mbox{}\newline 
\hspace*{6pt}{</\textbf{surface}>}\mbox{}\newline 
{</\textbf{facsimile}>}\end{shaded}\egroup 

This example defines a zone which has been defined as larger than its parent surface in order to match the dimensions of the graphic it contains.
    \item[{Modèle de contenu}]
  \mbox{}\hfill\\[-10pt]\begin{Verbatim}[fontsize=\small]
<content>
 <alternate maxOccurs="unbounded"
  minOccurs="0">
  <textNode/>
  <classRef key="model.graphicLike"/>
  <classRef key="model.global"/>
  <elementRef key="surface"/>
  <classRef key="model.linePart"/>
 </alternate>
</content>
    
\end{Verbatim}

    \item[{Schéma Declaration}]
  \mbox{}\hfill\\[-10pt]\begin{Verbatim}[fontsize=\small]
element zone
{
   tei_att.global.attributes,
   tei_att.coordinated.attributes,
   tei_att.typed.attributes,
   tei_att.written.attributes,
   attribute rotate { text }?,
   (
      text
    | tei_model.graphicLike    | tei_model.global    | tei_surface    | tei_model.linePart   )*
}
\end{Verbatim}

\end{reflist}  
\section[{Model classes}]{Model classes}
\begin{reflist}
\item[]\begin{specHead}{TEI.model.OAAnnotation}{model.OAAnnotation}\index{model.OAAnnotation (model class)|oddindex} Class implementing the Annotation component from the Open Annotation model\end{specHead} 
    \item[{Module}]
  standOff
    \item[{Utilisé par}]
  \hyperref[TEI.annotationBlock]{annotationBlock}
    \item[{Membres}]
  \hyperref[TEI.interp]{interp}
\end{reflist}  
\begin{reflist}
\item[]\begin{specHead}{TEI.model.OABody}{model.OABody}\index{model.OABody (model class)|oddindex} Class implementing the Body component from the Open Annotation model\end{specHead} 
    \item[{Module}]
  standOff
    \item[{Utilisé par}]
  \hyperref[TEI.annotationBlock]{annotationBlock}
    \item[{Membres}]
  \hyperref[TEI.date]{date} \hyperref[TEI.measure]{measure} \hyperref[TEI.note]{note} \hyperref[TEI.org]{org} \hyperref[TEI.person]{person} \hyperref[TEI.place]{place}
\end{reflist}  
\begin{reflist}
\item[]\begin{specHead}{TEI.model.OATarget}{model.OATarget}\index{model.OATarget (model class)|oddindex} Class implementing the Target component from the Open Annotation model\end{specHead} 
    \item[{Module}]
  standOff
    \item[{Utilisé par}]
  \hyperref[TEI.annotationBlock]{annotationBlock}
    \item[{Membres}]
  \hyperref[TEI.span]{span}
\end{reflist}  
\begin{reflist}
\item[]\begin{specHead}{TEI.model.addrPart}{model.addrPart}\index{model.addrPart (model class)|oddindex} regroupe des éléments comme des noms ou des codes postaux qui peuvent apparaître dans une adresse postale\end{specHead} 
    \item[{Module}]
  tei
    \item[{Utilisé par}]
  \hyperref[TEI.address]{address}
    \item[{Membres}]
  \hyperref[TEI.model.nameLike]{model.nameLike}[\hyperref[TEI.model.nameLike.agent]{model.nameLike.agent}[\hyperref[TEI.name]{name} \hyperref[TEI.orgName]{orgName} \hyperref[TEI.persName]{persName}] model.offsetLike \hyperref[TEI.model.persNamePart]{model.persNamePart}[\hyperref[TEI.addName]{addName} \hyperref[TEI.forename]{forename} \hyperref[TEI.genName]{genName} \hyperref[TEI.nameLink]{nameLink} \hyperref[TEI.roleName]{roleName} \hyperref[TEI.surname]{surname}] \hyperref[TEI.model.placeStateLike]{model.placeStateLike}[\hyperref[TEI.model.placeNamePart]{model.placeNamePart}[\hyperref[TEI.country]{country} \hyperref[TEI.geogName]{geogName} \hyperref[TEI.placeName]{placeName} \hyperref[TEI.region]{region} \hyperref[TEI.settlement]{settlement}] \hyperref[TEI.location]{location} \hyperref[TEI.state]{state}] \hyperref[TEI.idno]{idno} \hyperref[TEI.rs]{rs}] \hyperref[TEI.addrLine]{addrLine} \hyperref[TEI.postBox]{postBox} \hyperref[TEI.postCode]{postCode} \hyperref[TEI.street]{street}
\end{reflist}  
\begin{reflist}
\item[]\begin{specHead}{TEI.model.addressLike}{model.addressLike}\index{model.addressLike (model class)|oddindex} regroupe des éléments employés pour représenter des adresses postales ou électroniques.\end{specHead} 
    \item[{Module}]
  tei
    \item[{Utilisé par}]
  \hyperref[TEI.location]{location} \hyperref[TEI.model.pPart.data]{model.pPart.data}
    \item[{Membres}]
  \hyperref[TEI.address]{address} \hyperref[TEI.affiliation]{affiliation} \hyperref[TEI.email]{email}
\end{reflist}  
\begin{reflist}
\item[]\begin{specHead}{TEI.model.annotation}{model.annotation}\index{model.annotation (model class)|oddindex} groups together any kind of element that may be used to annotate an annotable segment\end{specHead} 
    \item[{Module}]
  standOff
    \item[{Utilisé par}]
  \hyperref[TEI.annotationBlock]{annotationBlock} \hyperref[TEI.listAnnotation]{listAnnotation}
    \item[{Membres}]
  \hyperref[TEI.model.biblLike]{model.biblLike}[\hyperref[TEI.bibl]{bibl} \hyperref[TEI.biblFull]{biblFull} \hyperref[TEI.biblStruct]{biblStruct} \hyperref[TEI.listBibl]{listBibl} \hyperref[TEI.msDesc]{msDesc}] \hyperref[TEI.model.dateLike]{model.dateLike}[\hyperref[TEI.date]{date} \hyperref[TEI.time]{time}] \hyperref[TEI.model.global.meta]{model.global.meta}[\hyperref[TEI.alt]{alt} \hyperref[TEI.altGrp]{altGrp} \hyperref[TEI.fLib]{fLib} \hyperref[TEI.fs]{fs} \hyperref[TEI.fvLib]{fvLib} \hyperref[TEI.index]{index} \hyperref[TEI.interp]{interp} \hyperref[TEI.interpGrp]{interpGrp} \hyperref[TEI.join]{join} \hyperref[TEI.joinGrp]{joinGrp} \hyperref[TEI.link]{link} \hyperref[TEI.linkGrp]{linkGrp} \hyperref[TEI.listTranspose]{listTranspose} \hyperref[TEI.source]{source} \hyperref[TEI.span]{span} \hyperref[TEI.spanGrp]{spanGrp} \hyperref[TEI.substJoin]{substJoin} \hyperref[TEI.timeline]{timeline}] \hyperref[TEI.model.listLike]{model.listLike}[\hyperref[TEI.list]{list} \hyperref[TEI.listOrg]{listOrg} \hyperref[TEI.listPlace]{listPlace} \hyperref[TEI.table]{table}] \hyperref[TEI.model.nameLike]{model.nameLike}[\hyperref[TEI.model.nameLike.agent]{model.nameLike.agent}[\hyperref[TEI.name]{name} \hyperref[TEI.orgName]{orgName} \hyperref[TEI.persName]{persName}] model.offsetLike \hyperref[TEI.model.persNamePart]{model.persNamePart}[\hyperref[TEI.addName]{addName} \hyperref[TEI.forename]{forename} \hyperref[TEI.genName]{genName} \hyperref[TEI.nameLink]{nameLink} \hyperref[TEI.roleName]{roleName} \hyperref[TEI.surname]{surname}] \hyperref[TEI.model.placeStateLike]{model.placeStateLike}[\hyperref[TEI.model.placeNamePart]{model.placeNamePart}[\hyperref[TEI.country]{country} \hyperref[TEI.geogName]{geogName} \hyperref[TEI.placeName]{placeName} \hyperref[TEI.region]{region} \hyperref[TEI.settlement]{settlement}] \hyperref[TEI.location]{location} \hyperref[TEI.state]{state}] \hyperref[TEI.idno]{idno} \hyperref[TEI.rs]{rs}] \hyperref[TEI.model.ptrLike]{model.ptrLike}[\hyperref[TEI.ptr]{ptr} \hyperref[TEI.ref]{ref}] \hyperref[TEI.annotationBlock]{annotationBlock} \hyperref[TEI.author]{author} \hyperref[TEI.keywords]{keywords} \hyperref[TEI.listAnnotation]{listAnnotation} \hyperref[TEI.seg]{seg} \hyperref[TEI.text]{text} \hyperref[TEI.zone]{zone}
\end{reflist}  
\begin{reflist}
\item[]\begin{specHead}{TEI.model.applicationLike}{model.applicationLike}\index{model.applicationLike (model class)|oddindex} regroupe des éléments utilisés pour enregistrer dans l'en-tête TEI d'un document des informations d'applications spécifiques.\end{specHead} 
    \item[{Module}]
  tei
    \item[{Utilisé par}]
  \hyperref[TEI.appInfo]{appInfo}
    \item[{Membres}]
  \hyperref[TEI.application]{application}
\end{reflist}  
\begin{reflist}
\item[]\begin{specHead}{TEI.model.availabilityPart}{model.availabilityPart}\index{model.availabilityPart (model class)|oddindex} regroupe des éléments tels que les licences ou les paragraphes indiquant la disponibilité d'un ouvrage.\end{specHead} 
    \item[{Module}]
  tei
    \item[{Utilisé par}]
  \hyperref[TEI.availability]{availability}
    \item[{Membres}]
  \hyperref[TEI.licence]{licence}
\end{reflist}  
\begin{reflist}
\item[]\begin{specHead}{TEI.model.biblLike}{model.biblLike}\index{model.biblLike (model class)|oddindex} regroupe des éléments contenant une description bibliographique.\end{specHead} 
    \item[{Module}]
  tei
    \item[{Utilisé par}]
  \hyperref[TEI.cit]{cit} \hyperref[TEI.event]{event} \hyperref[TEI.listBibl]{listBibl} \hyperref[TEI.location]{location} \hyperref[TEI.model.annotation]{model.annotation} \hyperref[TEI.model.inter]{model.inter} \hyperref[TEI.model.msItemPart]{model.msItemPart} \hyperref[TEI.model.personPart]{model.personPart} \hyperref[TEI.org]{org} \hyperref[TEI.place]{place} \hyperref[TEI.relatedItem]{relatedItem} \hyperref[TEI.sourceDesc]{sourceDesc} \hyperref[TEI.state]{state} \hyperref[TEI.taxonomy]{taxonomy}
    \item[{Membres}]
  \hyperref[TEI.bibl]{bibl} \hyperref[TEI.biblFull]{biblFull} \hyperref[TEI.biblStruct]{biblStruct} \hyperref[TEI.listBibl]{listBibl} \hyperref[TEI.msDesc]{msDesc}
\end{reflist}  
\begin{reflist}
\item[]\begin{specHead}{TEI.model.biblPart}{model.biblPart}\index{model.biblPart (model class)|oddindex} regroupe des éléments qui sont des composantes d’une description bibliographique.\end{specHead} 
    \item[{Module}]
  tei
    \item[{Utilisé par}]
  \hyperref[TEI.bibl]{bibl}
    \item[{Membres}]
  \hyperref[TEI.model.imprintPart]{model.imprintPart}[\hyperref[TEI.biblScope]{biblScope} \hyperref[TEI.distributor]{distributor} \hyperref[TEI.pubPlace]{pubPlace} \hyperref[TEI.publisher]{publisher}] \hyperref[TEI.model.respLike]{model.respLike}[\hyperref[TEI.author]{author} \hyperref[TEI.editor]{editor} \hyperref[TEI.funder]{funder} \hyperref[TEI.meeting]{meeting} \hyperref[TEI.respStmt]{respStmt}] \hyperref[TEI.availability]{availability} \hyperref[TEI.bibl]{bibl} \hyperref[TEI.citedRange]{citedRange} \hyperref[TEI.edition]{edition} \hyperref[TEI.extent]{extent} \hyperref[TEI.msIdentifier]{msIdentifier} \hyperref[TEI.relatedItem]{relatedItem} \hyperref[TEI.series]{series} \hyperref[TEI.textLang]{textLang}
\end{reflist}  
\begin{reflist}
\item[]\begin{specHead}{TEI.model.choicePart}{model.choicePart}\index{model.choicePart (model class)|oddindex} regroupe des éléments (autres que \hyperref[TEI.choice]{<choice>}) qui peuvent être utilisés en alternance avec \hyperref[TEI.choice]{<choice>}\end{specHead} 
    \item[{Module}]
  tei
    \item[{Utilisé par}]
  \hyperref[TEI.choice]{choice}
    \item[{Membres}]
  \hyperref[TEI.abbr]{abbr} \hyperref[TEI.am]{am} \hyperref[TEI.corr]{corr} \hyperref[TEI.ex]{ex} \hyperref[TEI.expan]{expan} \hyperref[TEI.orig]{orig} \hyperref[TEI.reg]{reg} \hyperref[TEI.seg]{seg} \hyperref[TEI.sic]{sic} \hyperref[TEI.supplied]{supplied} \hyperref[TEI.unclear]{unclear}
\end{reflist}  
\begin{reflist}
\item[]\begin{specHead}{TEI.model.common}{model.common}\index{model.common (model class)|oddindex} regroupe des éléments de niveau fragment - et des éléments de niveau intermédiaire.\end{specHead} 
    \item[{Module}]
  tei
    \item[{Utilisé par}]
  \hyperref[TEI.body]{body} \hyperref[TEI.div]{div} \hyperref[TEI.figure]{figure}
    \item[{Membres}]
  \hyperref[TEI.model.divPart]{model.divPart}[\hyperref[TEI.model.divPart.spoken]{model.divPart.spoken}[\hyperref[TEI.annotationBlock]{annotationBlock}] \hyperref[TEI.model.lLike]{model.lLike}[\hyperref[TEI.l]{l}] \hyperref[TEI.model.pLike]{model.pLike}[\hyperref[TEI.ab]{ab} \hyperref[TEI.p]{p}] \hyperref[TEI.lg]{lg} \hyperref[TEI.sp]{sp}] \hyperref[TEI.model.inter]{model.inter}[\hyperref[TEI.model.biblLike]{model.biblLike}[\hyperref[TEI.bibl]{bibl} \hyperref[TEI.biblFull]{biblFull} \hyperref[TEI.biblStruct]{biblStruct} \hyperref[TEI.listBibl]{listBibl} \hyperref[TEI.msDesc]{msDesc}] model.egLike \hyperref[TEI.model.labelLike]{model.labelLike}[\hyperref[TEI.desc]{desc} \hyperref[TEI.label]{label}] \hyperref[TEI.model.listLike]{model.listLike}[\hyperref[TEI.list]{list} \hyperref[TEI.listOrg]{listOrg} \hyperref[TEI.listPlace]{listPlace} \hyperref[TEI.table]{table}] model.oddDecl \hyperref[TEI.model.qLike]{model.qLike}[\hyperref[TEI.model.quoteLike]{model.quoteLike}[\hyperref[TEI.cit]{cit} \hyperref[TEI.quote]{quote}] \hyperref[TEI.floatingText]{floatingText} \hyperref[TEI.q]{q} \hyperref[TEI.said]{said}] \hyperref[TEI.model.stageLike]{model.stageLike}[\hyperref[TEI.stage]{stage}]]
    \item[{Note}]
  \par
Cette classe définit l'ensemble des éléments de niveau fragment et de niveau intermédiaire ; de nombreux modèles de contenu y font appel, y compris ceux qui sont relatifs aux divisions textuelles.
\end{reflist}  
\begin{reflist}
\item[]\begin{specHead}{TEI.model.dateLike}{model.dateLike}\index{model.dateLike (model class)|oddindex} regroupe des éléments contenant des expressions relatives au temps.\end{specHead} 
    \item[{Module}]
  tei
    \item[{Utilisé par}]
  \hyperref[TEI.imprint]{imprint} \hyperref[TEI.model.annotation]{model.annotation} \hyperref[TEI.model.pPart.data]{model.pPart.data}
    \item[{Membres}]
  \hyperref[TEI.date]{date} \hyperref[TEI.time]{time}
\end{reflist}  
\begin{reflist}
\item[]\begin{specHead}{TEI.model.descLike}{model.descLike}\index{model.descLike (model class)|oddindex} regroupe des éléments qui contiennent une description de la fonction d'un élément.\end{specHead} 
    \item[{Module}]
  tei
    \item[{Utilisé par}]
  \hyperref[TEI.category]{category} \hyperref[TEI.gap]{gap} \hyperref[TEI.graphic]{graphic} \hyperref[TEI.interp]{interp} \hyperref[TEI.interpGrp]{interpGrp} \hyperref[TEI.join]{join} \hyperref[TEI.media]{media} \hyperref[TEI.schemaRef]{schemaRef} \hyperref[TEI.space]{space} \hyperref[TEI.substJoin]{substJoin} \hyperref[TEI.taxonomy]{taxonomy}
    \item[{Membres}]
  \hyperref[TEI.desc]{desc}
\end{reflist}  
\begin{reflist}
\item[]\begin{specHead}{TEI.model.dimLike}{model.dimLike}\index{model.dimLike (model class)|oddindex} groups elements which describe a measurement forming part of the physical dimensions of some object.\end{specHead} 
    \item[{Module}]
  tei
    \item[{Utilisé par}]
  \hyperref[TEI.dimensions]{dimensions}
    \item[{Membres}]
  \hyperref[TEI.depth]{depth} \hyperref[TEI.height]{height} \hyperref[TEI.width]{width}
\end{reflist}  
\begin{reflist}
\item[]\begin{specHead}{TEI.model.divBottom}{model.divBottom}\index{model.divBottom (model class)|oddindex} regroupe des éléments placés à la fin d'une division de texte.\end{specHead} 
    \item[{Module}]
  tei
    \item[{Utilisé par}]
  \hyperref[TEI.body]{body} \hyperref[TEI.div]{div} \hyperref[TEI.figure]{figure} \hyperref[TEI.front]{front} \hyperref[TEI.group]{group} \hyperref[TEI.lg]{lg} \hyperref[TEI.list]{list} \hyperref[TEI.table]{table}
    \item[{Membres}]
  model.divBottomPart \hyperref[TEI.model.divWrapper]{model.divWrapper}[\hyperref[TEI.docAuthor]{docAuthor} \hyperref[TEI.docDate]{docDate} \hyperref[TEI.meeting]{meeting}]
\end{reflist}  
\begin{reflist}
\item[]\begin{specHead}{TEI.model.divGenLike}{model.divGenLike}\index{model.divGenLike (model class)|oddindex} regroupe des éléments utilisés pour représenter une division structurelle qui est générée plutôt que présente de manière explicite dans la source\end{specHead} 
    \item[{Module}]
  tei
    \item[{Utilisé par}]
  \hyperref[TEI.body]{body} \hyperref[TEI.div]{div}
    \item[{Membres}]
  \hyperref[TEI.divGen]{divGen}
\end{reflist}  
\begin{reflist}
\item[]\begin{specHead}{TEI.model.divLike}{model.divLike}\index{model.divLike (model class)|oddindex} regroupe des éléments utilisés pour représenter des divisions structurelles génériques non numérotées.\end{specHead} 
    \item[{Module}]
  tei
    \item[{Utilisé par}]
  \hyperref[TEI.back]{back} \hyperref[TEI.body]{body} \hyperref[TEI.div]{div} \hyperref[TEI.front]{front}
    \item[{Membres}]
  \hyperref[TEI.div]{div}
\end{reflist}  
\begin{reflist}
\item[]\begin{specHead}{TEI.model.divPart}{model.divPart}\index{model.divPart (model class)|oddindex} regroupe des éléments de niveau paragraphe apparaissant directement dans des divisions\end{specHead} 
    \item[{Module}]
  tei
    \item[{Utilisé par}]
  \hyperref[TEI.macro.specialPara]{macro.specialPara} \hyperref[TEI.model.common]{model.common}
    \item[{Membres}]
  \hyperref[TEI.model.divPart.spoken]{model.divPart.spoken}[\hyperref[TEI.annotationBlock]{annotationBlock}] \hyperref[TEI.model.lLike]{model.lLike}[\hyperref[TEI.l]{l}] \hyperref[TEI.model.pLike]{model.pLike}[\hyperref[TEI.ab]{ab} \hyperref[TEI.p]{p}] \hyperref[TEI.lg]{lg} \hyperref[TEI.sp]{sp}
    \item[{Note}]
  \par
Noter que cette classe d'éléments ne comprend pas les membres de la classe \textsf{model.inter}, qui peuvent apparaître soit à l'intérieur, soit entre des items de niveau paragraphe.
\end{reflist}  
\begin{reflist}
\item[]\begin{specHead}{TEI.model.divPart.spoken}{model.divPart.spoken}\index{model.divPart.spoken (model class)|oddindex} regroupe des éléments structurellement analogues aux paragraphes dans des textes contenant de la parole transcrite.\end{specHead} 
    \item[{Module}]
  spoken
    \item[{Utilisé par}]
  \hyperref[TEI.model.divPart]{model.divPart}
    \item[{Membres}]
  \hyperref[TEI.annotationBlock]{annotationBlock}
    \item[{Note}]
  \par
Les textes contenant de la parole transcrite peuvent être structurés de plusieurs façons; les éléments de cette classe sont habituellement des unités plus grandes, comme des tournures ou des énoncés. 
\end{reflist}  
\begin{reflist}
\item[]\begin{specHead}{TEI.model.divTop}{model.divTop}\index{model.divTop (model class)|oddindex} regroupe des éléments apparaissant au début d'une division de texte.\end{specHead} 
    \item[{Module}]
  tei
    \item[{Utilisé par}]
  \hyperref[TEI.body]{body} \hyperref[TEI.div]{div} \hyperref[TEI.group]{group} \hyperref[TEI.lg]{lg} \hyperref[TEI.list]{list}
    \item[{Membres}]
  \hyperref[TEI.model.divTopPart]{model.divTopPart}[\hyperref[TEI.model.headLike]{model.headLike}[\hyperref[TEI.head]{head}]] \hyperref[TEI.model.divWrapper]{model.divWrapper}[\hyperref[TEI.docAuthor]{docAuthor} \hyperref[TEI.docDate]{docDate} \hyperref[TEI.meeting]{meeting}]
\end{reflist}  
\begin{reflist}
\item[]\begin{specHead}{TEI.model.divTopPart}{model.divTopPart}\index{model.divTopPart (model class)|oddindex} regroupe des éléments qu'on ne trouve qu'au début d'une division de texte.\end{specHead} 
    \item[{Module}]
  tei
    \item[{Utilisé par}]
  \hyperref[TEI.model.divTop]{model.divTop}
    \item[{Membres}]
  \hyperref[TEI.model.headLike]{model.headLike}[\hyperref[TEI.head]{head}]
\end{reflist}  
\begin{reflist}
\item[]\begin{specHead}{TEI.model.divWrapper}{model.divWrapper}\index{model.divWrapper (model class)|oddindex} regroupe des éléments qui peuvent être placés au début ou à la fin de n'importe quelle division de texte.\end{specHead} 
    \item[{Module}]
  tei
    \item[{Utilisé par}]
  \hyperref[TEI.model.divBottom]{model.divBottom} \hyperref[TEI.model.divTop]{model.divTop}
    \item[{Membres}]
  \hyperref[TEI.docAuthor]{docAuthor} \hyperref[TEI.docDate]{docDate} \hyperref[TEI.meeting]{meeting}
\end{reflist}  
\begin{reflist}
\item[]\begin{specHead}{TEI.model.emphLike}{model.emphLike}\index{model.emphLike (model class)|oddindex} regroupe des éléments qui sont distingués visuellement, et auxquels on peut attribuer une signification sémantique.\end{specHead} 
    \item[{Module}]
  tei
    \item[{Utilisé par}]
  \hyperref[TEI.model.highlighted]{model.highlighted} \hyperref[TEI.model.limitedPhrase]{model.limitedPhrase}
    \item[{Membres}]
  \hyperref[TEI.distinct]{distinct} \hyperref[TEI.emph]{emph} \hyperref[TEI.foreign]{foreign} \hyperref[TEI.gloss]{gloss} \hyperref[TEI.mentioned]{mentioned} \hyperref[TEI.soCalled]{soCalled} \hyperref[TEI.term]{term} \hyperref[TEI.title]{title}
\end{reflist}  
\begin{reflist}
\item[]\begin{specHead}{TEI.model.encodingDescPart}{model.encodingDescPart}\index{model.encodingDescPart (model class)|oddindex} regroupe des éléments qui peuvent apparaître plusieurs fois dans l'élément \hyperref[TEI.encodingDesc]{<encodingDesc>}.\end{specHead} 
    \item[{Module}]
  tei
    \item[{Utilisé par}]
  \hyperref[TEI.encodingDesc]{encodingDesc}
    \item[{Membres}]
  \hyperref[TEI.appInfo]{appInfo} \hyperref[TEI.classDecl]{classDecl} \hyperref[TEI.fsdDecl]{fsdDecl} \hyperref[TEI.schemaRef]{schemaRef}
\end{reflist}  
\begin{reflist}
\item[]\begin{specHead}{TEI.model.eventLike}{model.eventLike}\index{model.eventLike (model class)|oddindex} groups elements which describe events.\end{specHead} 
    \item[{Module}]
  tei
    \item[{Utilisé par}]
  \hyperref[TEI.model.orgPart]{model.orgPart} \hyperref[TEI.model.personPart]{model.personPart} \hyperref[TEI.place]{place}
    \item[{Membres}]
  \hyperref[TEI.event]{event}
\end{reflist}  
\begin{reflist}
\item[]\begin{specHead}{TEI.model.featureVal}{model.featureVal}\index{model.featureVal (model class)|oddindex} regroupe tous les éléments qui représentent des valeurs de trait dans des structures de trait.\end{specHead} 
    \item[{Module}]
  tei
    \item[{Utilisé par}]
  \hyperref[TEI.f]{f} \hyperref[TEI.fvLib]{fvLib} \hyperref[TEI.if]{if} \hyperref[TEI.vAlt]{vAlt} \hyperref[TEI.vDefault]{vDefault} \hyperref[TEI.vLabel]{vLabel} \hyperref[TEI.vMerge]{vMerge} \hyperref[TEI.vNot]{vNot} \hyperref[TEI.vRange]{vRange}
    \item[{Membres}]
  \hyperref[TEI.model.featureVal.complex]{model.featureVal.complex}[\hyperref[TEI.fs]{fs} \hyperref[TEI.vColl]{vColl} \hyperref[TEI.vMerge]{vMerge} \hyperref[TEI.vNot]{vNot}] \hyperref[TEI.model.featureVal.single]{model.featureVal.single}[\hyperref[TEI.binary]{binary} \hyperref[TEI.default]{default} \hyperref[TEI.numeric]{numeric} \hyperref[TEI.string]{string} \hyperref[TEI.symbol]{symbol} \hyperref[TEI.vAlt]{vAlt} \hyperref[TEI.vLabel]{vLabel}]
\end{reflist}  
\begin{reflist}
\item[]\begin{specHead}{TEI.model.featureVal.complex}{model.featureVal.complex}\index{model.featureVal.complex (model class)|oddindex} regroupe des éléments qui expriment des valeurs de traits complexes dans des structures de traits\end{specHead} 
    \item[{Module}]
  tei
    \item[{Utilisé par}]
  \hyperref[TEI.model.featureVal]{model.featureVal}
    \item[{Membres}]
  \hyperref[TEI.fs]{fs} \hyperref[TEI.vColl]{vColl} \hyperref[TEI.vMerge]{vMerge} \hyperref[TEI.vNot]{vNot}
\end{reflist}  
\begin{reflist}
\item[]\begin{specHead}{TEI.model.featureVal.single}{model.featureVal.single}\index{model.featureVal.single (model class)|oddindex} regroupe des éléments employés pour représenter des valeurs de traits atomiques dans des structures de traits\end{specHead} 
    \item[{Module}]
  tei
    \item[{Utilisé par}]
  \hyperref[TEI.model.featureVal]{model.featureVal} \hyperref[TEI.vColl]{vColl}
    \item[{Membres}]
  \hyperref[TEI.binary]{binary} \hyperref[TEI.default]{default} \hyperref[TEI.numeric]{numeric} \hyperref[TEI.string]{string} \hyperref[TEI.symbol]{symbol} \hyperref[TEI.vAlt]{vAlt} \hyperref[TEI.vLabel]{vLabel}
\end{reflist}  
\begin{reflist}
\item[]\begin{specHead}{TEI.model.frontPart}{model.frontPart}\index{model.frontPart (model class)|oddindex} regroupe les éléments du niveau des divisions qui peuvent apparaître dans un \hyperref[TEI.front]{<front>} ou un \hyperref[TEI.back]{<back>}.\end{specHead} 
    \item[{Module}]
  tei
    \item[{Utilisé par}]
  \hyperref[TEI.back]{back} \hyperref[TEI.front]{front}
    \item[{Membres}]
  model.frontPart.drama \hyperref[TEI.divGen]{divGen} \hyperref[TEI.listBibl]{listBibl} \hyperref[TEI.titlePage]{titlePage}
\end{reflist}  
\begin{reflist}
\item[]\begin{specHead}{TEI.model.global}{model.global}\index{model.global (model class)|oddindex} regroupe des éléments qui peuvent apparaître à un point quelconque dans un texte TEI.\end{specHead} 
    \item[{Module}]
  tei
    \item[{Utilisé par}]
  \hyperref[TEI.address]{address} \hyperref[TEI.back]{back} \hyperref[TEI.bibl]{bibl} \hyperref[TEI.body]{body} \hyperref[TEI.cit]{cit} \hyperref[TEI.date]{date} \hyperref[TEI.div]{div} \hyperref[TEI.docTitle]{docTitle} \hyperref[TEI.figure]{figure} \hyperref[TEI.floatingText]{floatingText} \hyperref[TEI.front]{front} \hyperref[TEI.group]{group} \hyperref[TEI.head]{head} \hyperref[TEI.imprint]{imprint} \hyperref[TEI.l]{l} \hyperref[TEI.lg]{lg} \hyperref[TEI.line]{line} \hyperref[TEI.list]{list} \hyperref[TEI.m]{m} \hyperref[TEI.macro.paraContent]{macro.paraContent} \hyperref[TEI.macro.phraseSeq]{macro.phraseSeq} \hyperref[TEI.macro.phraseSeq.limited]{macro.phraseSeq.limited} \hyperref[TEI.macro.specialPara]{macro.specialPara} \hyperref[TEI.msItem]{msItem} \hyperref[TEI.origDate]{origDate} \hyperref[TEI.person]{person} \hyperref[TEI.personGrp]{personGrp} \hyperref[TEI.persona]{persona} \hyperref[TEI.series]{series} \hyperref[TEI.sourceDoc]{sourceDoc} \hyperref[TEI.sp]{sp} \hyperref[TEI.surface]{surface} \hyperref[TEI.surfaceGrp]{surfaceGrp} \hyperref[TEI.table]{table} \hyperref[TEI.text]{text} \hyperref[TEI.time]{time} \hyperref[TEI.titlePage]{titlePage} \hyperref[TEI.w]{w} \hyperref[TEI.zone]{zone}
    \item[{Membres}]
  \hyperref[TEI.model.global.edit]{model.global.edit}[\hyperref[TEI.addSpan]{addSpan} \hyperref[TEI.damageSpan]{damageSpan} \hyperref[TEI.delSpan]{delSpan} \hyperref[TEI.gap]{gap} \hyperref[TEI.space]{space}] \hyperref[TEI.model.global.meta]{model.global.meta}[\hyperref[TEI.alt]{alt} \hyperref[TEI.altGrp]{altGrp} \hyperref[TEI.fLib]{fLib} \hyperref[TEI.fs]{fs} \hyperref[TEI.fvLib]{fvLib} \hyperref[TEI.index]{index} \hyperref[TEI.interp]{interp} \hyperref[TEI.interpGrp]{interpGrp} \hyperref[TEI.join]{join} \hyperref[TEI.joinGrp]{joinGrp} \hyperref[TEI.link]{link} \hyperref[TEI.linkGrp]{linkGrp} \hyperref[TEI.listTranspose]{listTranspose} \hyperref[TEI.source]{source} \hyperref[TEI.span]{span} \hyperref[TEI.spanGrp]{spanGrp} \hyperref[TEI.substJoin]{substJoin} \hyperref[TEI.timeline]{timeline}] model.global.spoken \hyperref[TEI.model.milestoneLike]{model.milestoneLike}[\hyperref[TEI.anchor]{anchor} \hyperref[TEI.cb]{cb} \hyperref[TEI.fw]{fw} \hyperref[TEI.gb]{gb} \hyperref[TEI.lb]{lb} \hyperref[TEI.milestone]{milestone} \hyperref[TEI.pb]{pb}] \hyperref[TEI.model.noteLike]{model.noteLike}[\hyperref[TEI.note]{note}] \hyperref[TEI.figure]{figure} \hyperref[TEI.metamark]{metamark} \hyperref[TEI.notatedMusic]{notatedMusic}
\end{reflist}  
\begin{reflist}
\item[]\begin{specHead}{TEI.model.global.edit}{model.global.edit}\index{model.global.edit (model class)|oddindex} regroupe des éléments globalement disponibles qui exécutent une fonction spécifiquement éditoriale.\end{specHead} 
    \item[{Module}]
  tei
    \item[{Utilisé par}]
  \hyperref[TEI.model.global]{model.global}
    \item[{Membres}]
  \hyperref[TEI.addSpan]{addSpan} \hyperref[TEI.damageSpan]{damageSpan} \hyperref[TEI.delSpan]{delSpan} \hyperref[TEI.gap]{gap} \hyperref[TEI.space]{space}
\end{reflist}  
\begin{reflist}
\item[]\begin{specHead}{TEI.model.global.meta}{model.global.meta}\index{model.global.meta (model class)|oddindex} regroupe des éléments disponibles globalement qui décrivent le statut d'autres éléments.\end{specHead} 
    \item[{Module}]
  tei
    \item[{Utilisé par}]
  \hyperref[TEI.model.annotation]{model.annotation} \hyperref[TEI.model.global]{model.global}
    \item[{Membres}]
  \hyperref[TEI.alt]{alt} \hyperref[TEI.altGrp]{altGrp} \hyperref[TEI.fLib]{fLib} \hyperref[TEI.fs]{fs} \hyperref[TEI.fvLib]{fvLib} \hyperref[TEI.index]{index} \hyperref[TEI.interp]{interp} \hyperref[TEI.interpGrp]{interpGrp} \hyperref[TEI.join]{join} \hyperref[TEI.joinGrp]{joinGrp} \hyperref[TEI.link]{link} \hyperref[TEI.linkGrp]{linkGrp} \hyperref[TEI.listTranspose]{listTranspose} \hyperref[TEI.source]{source} \hyperref[TEI.span]{span} \hyperref[TEI.spanGrp]{spanGrp} \hyperref[TEI.substJoin]{substJoin} \hyperref[TEI.timeline]{timeline}
    \item[{Note}]
  \par
Les éléments de cette classe sont utilisés pour contenir des groupes de liens ou d'interprétations abstraites, ou pour fournir des indications quant à la certitude, etc. Il peut être commode de situer tous les éléments contenant des métadonnées, par exemple de les rassembler dans la même divison que les éléments auxquels ils sont reliés ; ou de les retrouver tous dans la division qui leur est propre. Ils peuvent cependant apparaître à un point quelconque d'un texte TEI.
\end{reflist}  
\begin{reflist}
\item[]\begin{specHead}{TEI.model.glossLike}{model.glossLike}\index{model.glossLike (model class)|oddindex} regroupe des éléments qui proposent un nom alternatif, une explication, ou une description pour n'importe quelle structure de codage\end{specHead} 
    \item[{Module}]
  tei
    \item[{Utilisé par}]
  \hyperref[TEI.category]{category} \hyperref[TEI.joinGrp]{joinGrp} \hyperref[TEI.taxonomy]{taxonomy}
    \item[{Membres}]
  \hyperref[TEI.gloss]{gloss}
\end{reflist}  
\begin{reflist}
\item[]\begin{specHead}{TEI.model.graphicLike}{model.graphicLike}\index{model.graphicLike (model class)|oddindex} regroupe des éléments contenant des images, des formules et d'autres objets semblables.\end{specHead} 
    \item[{Module}]
  tei
    \item[{Utilisé par}]
  \hyperref[TEI.facsimile]{facsimile} \hyperref[TEI.figDesc]{figDesc} \hyperref[TEI.figure]{figure} \hyperref[TEI.formula]{formula} \hyperref[TEI.model.phrase]{model.phrase} \hyperref[TEI.sourceDoc]{sourceDoc} \hyperref[TEI.surface]{surface} \hyperref[TEI.table]{table} \hyperref[TEI.zone]{zone}
    \item[{Membres}]
  \hyperref[TEI.binaryObject]{binaryObject} \hyperref[TEI.formula]{formula} \hyperref[TEI.graphic]{graphic} \hyperref[TEI.math]{math} \hyperref[TEI.media]{media} \hyperref[TEI.mrow]{mrow}
\end{reflist}  
\begin{reflist}
\item[]\begin{specHead}{TEI.model.headLike}{model.headLike}\index{model.headLike (model class)|oddindex} regroupe des éléments employés pour donner un titre ou un intitulé au début d'une division de texte\end{specHead} 
    \item[{Module}]
  tei
    \item[{Utilisé par}]
  \hyperref[TEI.abstract]{abstract} \hyperref[TEI.divGen]{divGen} \hyperref[TEI.event]{event} \hyperref[TEI.figure]{figure} \hyperref[TEI.listAnnotation]{listAnnotation} \hyperref[TEI.listBibl]{listBibl} \hyperref[TEI.listOrg]{listOrg} \hyperref[TEI.listPlace]{listPlace} \hyperref[TEI.model.divTopPart]{model.divTopPart} \hyperref[TEI.msDesc]{msDesc} \hyperref[TEI.msFrag]{msFrag} \hyperref[TEI.msPart]{msPart} \hyperref[TEI.org]{org} \hyperref[TEI.place]{place} \hyperref[TEI.state]{state} \hyperref[TEI.table]{table}
    \item[{Membres}]
  \hyperref[TEI.head]{head}
\end{reflist}  
\begin{reflist}
\item[]\begin{specHead}{TEI.model.hiLike}{model.hiLike}\index{model.hiLike (model class)|oddindex} regroupe des éléments du niveau de l’expression qui sont typographiquement distincts mais auxquels aucune fonction spécifique ne peut être attribuée.\end{specHead} 
    \item[{Module}]
  tei
    \item[{Utilisé par}]
  \hyperref[TEI.formula]{formula} \hyperref[TEI.m]{m} \hyperref[TEI.model.highlighted]{model.highlighted} \hyperref[TEI.model.limitedPhrase]{model.limitedPhrase} \hyperref[TEI.model.linePart]{model.linePart} \hyperref[TEI.w]{w}
    \item[{Membres}]
  \hyperref[TEI.hi]{hi}
\end{reflist}  
\begin{reflist}
\item[]\begin{specHead}{TEI.model.highlighted}{model.highlighted}\index{model.highlighted (model class)|oddindex} regroupe des éléments du niveau de l'expression qui sont typographiquement distincts.\end{specHead} 
    \item[{Module}]
  tei
    \item[{Utilisé par}]
  \hyperref[TEI.bibl]{bibl} \hyperref[TEI.model.phrase]{model.phrase}
    \item[{Membres}]
  \hyperref[TEI.model.emphLike]{model.emphLike}[\hyperref[TEI.distinct]{distinct} \hyperref[TEI.emph]{emph} \hyperref[TEI.foreign]{foreign} \hyperref[TEI.gloss]{gloss} \hyperref[TEI.mentioned]{mentioned} \hyperref[TEI.soCalled]{soCalled} \hyperref[TEI.term]{term} \hyperref[TEI.title]{title}] \hyperref[TEI.model.hiLike]{model.hiLike}[\hyperref[TEI.hi]{hi}]
\end{reflist}  
\begin{reflist}
\item[]\begin{specHead}{TEI.model.imprintPart}{model.imprintPart}\index{model.imprintPart (model class)|oddindex} regoupe les éléments bibliographiques qui apparaissent à l'intérieur de documents imprimés.\end{specHead} 
    \item[{Module}]
  tei
    \item[{Utilisé par}]
  \hyperref[TEI.imprint]{imprint} \hyperref[TEI.model.biblPart]{model.biblPart}
    \item[{Membres}]
  \hyperref[TEI.biblScope]{biblScope} \hyperref[TEI.distributor]{distributor} \hyperref[TEI.pubPlace]{pubPlace} \hyperref[TEI.publisher]{publisher}
\end{reflist}  
\begin{reflist}
\item[]\begin{specHead}{TEI.model.inter}{model.inter}\index{model.inter (model class)|oddindex} regroupe des éléments qui peuvent apparaître à l’intérieur ou entre des composants semblables au paragraphe.\end{specHead} 
    \item[{Module}]
  tei
    \item[{Utilisé par}]
  \hyperref[TEI.head]{head} \hyperref[TEI.l]{l} \hyperref[TEI.macro.limitedContent]{macro.limitedContent} \hyperref[TEI.macro.paraContent]{macro.paraContent} \hyperref[TEI.macro.specialPara]{macro.specialPara} \hyperref[TEI.model.common]{model.common}
    \item[{Membres}]
  \hyperref[TEI.model.biblLike]{model.biblLike}[\hyperref[TEI.bibl]{bibl} \hyperref[TEI.biblFull]{biblFull} \hyperref[TEI.biblStruct]{biblStruct} \hyperref[TEI.listBibl]{listBibl} \hyperref[TEI.msDesc]{msDesc}] model.egLike \hyperref[TEI.model.labelLike]{model.labelLike}[\hyperref[TEI.desc]{desc} \hyperref[TEI.label]{label}] \hyperref[TEI.model.listLike]{model.listLike}[\hyperref[TEI.list]{list} \hyperref[TEI.listOrg]{listOrg} \hyperref[TEI.listPlace]{listPlace} \hyperref[TEI.table]{table}] model.oddDecl \hyperref[TEI.model.qLike]{model.qLike}[\hyperref[TEI.model.quoteLike]{model.quoteLike}[\hyperref[TEI.cit]{cit} \hyperref[TEI.quote]{quote}] \hyperref[TEI.floatingText]{floatingText} \hyperref[TEI.q]{q} \hyperref[TEI.said]{said}] \hyperref[TEI.model.stageLike]{model.stageLike}[\hyperref[TEI.stage]{stage}]
\end{reflist}  
\begin{reflist}
\item[]\begin{specHead}{TEI.model.lLike}{model.lLike}\index{model.lLike (model class)|oddindex} regroupe des éléments représentant des composants de la métrique comme des vers.\end{specHead} 
    \item[{Module}]
  tei
    \item[{Utilisé par}]
  \hyperref[TEI.head]{head} \hyperref[TEI.lg]{lg} \hyperref[TEI.macro.paraContent]{macro.paraContent} \hyperref[TEI.model.divPart]{model.divPart} \hyperref[TEI.sp]{sp}
    \item[{Membres}]
  \hyperref[TEI.l]{l}
\end{reflist}  
\begin{reflist}
\item[]\begin{specHead}{TEI.model.labelLike}{model.labelLike}\index{model.labelLike (model class)|oddindex} regroupe des éléments employés pour gloser ou expliquer d'autres parties d'un document.\end{specHead} 
    \item[{Module}]
  tei
    \item[{Utilisé par}]
  \hyperref[TEI.application]{application} \hyperref[TEI.event]{event} \hyperref[TEI.lg]{lg} \hyperref[TEI.location]{location} \hyperref[TEI.model.inter]{model.inter} \hyperref[TEI.notatedMusic]{notatedMusic} \hyperref[TEI.org]{org} \hyperref[TEI.place]{place} \hyperref[TEI.state]{state} \hyperref[TEI.surface]{surface}
    \item[{Membres}]
  \hyperref[TEI.desc]{desc} \hyperref[TEI.label]{label}
\end{reflist}  
\begin{reflist}
\item[]\begin{specHead}{TEI.model.limitedPhrase}{model.limitedPhrase}\index{model.limitedPhrase (model class)|oddindex} regroupe des éléments du niveau de l'expression excluant ceux qui sont principalement destinés à la transcription des sources existantes.\end{specHead} 
    \item[{Module}]
  tei
    \item[{Utilisé par}]
  \hyperref[TEI.creation]{creation} \hyperref[TEI.macro.limitedContent]{macro.limitedContent} \hyperref[TEI.macro.phraseSeq.limited]{macro.phraseSeq.limited}
    \item[{Membres}]
  \hyperref[TEI.model.emphLike]{model.emphLike}[\hyperref[TEI.distinct]{distinct} \hyperref[TEI.emph]{emph} \hyperref[TEI.foreign]{foreign} \hyperref[TEI.gloss]{gloss} \hyperref[TEI.mentioned]{mentioned} \hyperref[TEI.soCalled]{soCalled} \hyperref[TEI.term]{term} \hyperref[TEI.title]{title}] \hyperref[TEI.model.hiLike]{model.hiLike}[\hyperref[TEI.hi]{hi}] \hyperref[TEI.model.pPart.data]{model.pPart.data}[\hyperref[TEI.model.addressLike]{model.addressLike}[\hyperref[TEI.address]{address} \hyperref[TEI.affiliation]{affiliation} \hyperref[TEI.email]{email}] \hyperref[TEI.model.dateLike]{model.dateLike}[\hyperref[TEI.date]{date} \hyperref[TEI.time]{time}] \hyperref[TEI.model.measureLike]{model.measureLike}[\hyperref[TEI.depth]{depth} \hyperref[TEI.dim]{dim} \hyperref[TEI.height]{height} \hyperref[TEI.measure]{measure} \hyperref[TEI.measureGrp]{measureGrp} \hyperref[TEI.num]{num} \hyperref[TEI.width]{width}] \hyperref[TEI.model.nameLike]{model.nameLike}[\hyperref[TEI.model.nameLike.agent]{model.nameLike.agent}[\hyperref[TEI.name]{name} \hyperref[TEI.orgName]{orgName} \hyperref[TEI.persName]{persName}] model.offsetLike \hyperref[TEI.model.persNamePart]{model.persNamePart}[\hyperref[TEI.addName]{addName} \hyperref[TEI.forename]{forename} \hyperref[TEI.genName]{genName} \hyperref[TEI.nameLink]{nameLink} \hyperref[TEI.roleName]{roleName} \hyperref[TEI.surname]{surname}] \hyperref[TEI.model.placeStateLike]{model.placeStateLike}[\hyperref[TEI.model.placeNamePart]{model.placeNamePart}[\hyperref[TEI.country]{country} \hyperref[TEI.geogName]{geogName} \hyperref[TEI.placeName]{placeName} \hyperref[TEI.region]{region} \hyperref[TEI.settlement]{settlement}] \hyperref[TEI.location]{location} \hyperref[TEI.state]{state}] \hyperref[TEI.idno]{idno} \hyperref[TEI.rs]{rs}]] \hyperref[TEI.model.pPart.editorial]{model.pPart.editorial}[\hyperref[TEI.abbr]{abbr} \hyperref[TEI.am]{am} \hyperref[TEI.choice]{choice} \hyperref[TEI.ex]{ex} \hyperref[TEI.expan]{expan} \hyperref[TEI.subst]{subst}] \hyperref[TEI.model.pPart.msdesc]{model.pPart.msdesc}[\hyperref[TEI.catchwords]{catchwords} \hyperref[TEI.dimensions]{dimensions} \hyperref[TEI.heraldry]{heraldry} \hyperref[TEI.locus]{locus} \hyperref[TEI.locusGrp]{locusGrp} \hyperref[TEI.material]{material} \hyperref[TEI.objectType]{objectType} \hyperref[TEI.origDate]{origDate} \hyperref[TEI.origPlace]{origPlace} \hyperref[TEI.secFol]{secFol} \hyperref[TEI.signatures]{signatures} \hyperref[TEI.stamp]{stamp} \hyperref[TEI.watermark]{watermark}] model.phrase.xml \hyperref[TEI.model.ptrLike]{model.ptrLike}[\hyperref[TEI.ptr]{ptr} \hyperref[TEI.ref]{ref}]
\end{reflist}  
\begin{reflist}
\item[]\begin{specHead}{TEI.model.linePart}{model.linePart}\index{model.linePart (model class)|oddindex} regroupe des éléments qui peuvent figurer dans les zones d'une transcription orientée source dans un élément \hyperref[TEI.sourceDoc]{<sourceDoc>}.\end{specHead} 
    \item[{Module}]
  tei
    \item[{Utilisé par}]
  \hyperref[TEI.line]{line} \hyperref[TEI.zone]{zone}
    \item[{Membres}]
  \hyperref[TEI.model.hiLike]{model.hiLike}[\hyperref[TEI.hi]{hi}] \hyperref[TEI.add]{add} \hyperref[TEI.c]{c} \hyperref[TEI.choice]{choice} \hyperref[TEI.damage]{damage} \hyperref[TEI.del]{del} \hyperref[TEI.handShift]{handShift} \hyperref[TEI.line]{line} \hyperref[TEI.mod]{mod} \hyperref[TEI.pc]{pc} \hyperref[TEI.redo]{redo} \hyperref[TEI.restore]{restore} \hyperref[TEI.retrace]{retrace} \hyperref[TEI.seg]{seg} \hyperref[TEI.unclear]{unclear} \hyperref[TEI.undo]{undo} \hyperref[TEI.w]{w} \hyperref[TEI.zone]{zone}
\end{reflist}  
\begin{reflist}
\item[]\begin{specHead}{TEI.model.listLike}{model.listLike}\index{model.listLike (model class)|oddindex} regroupe les éléments de type liste.\end{specHead} 
    \item[{Module}]
  tei
    \item[{Utilisé par}]
  \hyperref[TEI.abstract]{abstract} \hyperref[TEI.back]{back} \hyperref[TEI.model.annotation]{model.annotation} \hyperref[TEI.model.inter]{model.inter} \hyperref[TEI.sourceDesc]{sourceDesc} \hyperref[TEI.sp]{sp}
    \item[{Membres}]
  \hyperref[TEI.list]{list} \hyperref[TEI.listOrg]{listOrg} \hyperref[TEI.listPlace]{listPlace} \hyperref[TEI.table]{table}
\end{reflist}  
\begin{reflist}
\item[]\begin{specHead}{TEI.model.measureLike}{model.measureLike}\index{model.measureLike (model class)|oddindex} regroupe des éléments qui indiquent un nombre, une quantité, une mesure ou un extrait d'un texte qui porte une signification numérique.\end{specHead} 
    \item[{Module}]
  tei
    \item[{Utilisé par}]
  \hyperref[TEI.location]{location} \hyperref[TEI.measureGrp]{measureGrp} \hyperref[TEI.model.pPart.data]{model.pPart.data}
    \item[{Membres}]
  \hyperref[TEI.depth]{depth} \hyperref[TEI.dim]{dim} \hyperref[TEI.height]{height} \hyperref[TEI.measure]{measure} \hyperref[TEI.measureGrp]{measureGrp} \hyperref[TEI.num]{num} \hyperref[TEI.width]{width}
\end{reflist}  
\begin{reflist}
\item[]\begin{specHead}{TEI.model.milestoneLike}{model.milestoneLike}\index{model.milestoneLike (model class)|oddindex} regroupe des éléments de type borne utilisés pour représenter des systèmes de référence\end{specHead} 
    \item[{Module}]
  tei
    \item[{Utilisé par}]
  \hyperref[TEI.listBibl]{listBibl} \hyperref[TEI.model.global]{model.global} \hyperref[TEI.org]{org} \hyperref[TEI.subst]{subst}
    \item[{Membres}]
  \hyperref[TEI.anchor]{anchor} \hyperref[TEI.cb]{cb} \hyperref[TEI.fw]{fw} \hyperref[TEI.gb]{gb} \hyperref[TEI.lb]{lb} \hyperref[TEI.milestone]{milestone} \hyperref[TEI.pb]{pb}
\end{reflist}  
\begin{reflist}
\item[]\begin{specHead}{TEI.model.msItemPart}{model.msItemPart}\index{model.msItemPart (model class)|oddindex} regroupe des éléments qui peuvent apparaître dans une description de manuscrit\end{specHead} 
    \item[{Module}]
  tei
    \item[{Utilisé par}]
  \hyperref[TEI.msItem]{msItem}
    \item[{Membres}]
  \hyperref[TEI.model.biblLike]{model.biblLike}[\hyperref[TEI.bibl]{bibl} \hyperref[TEI.biblFull]{biblFull} \hyperref[TEI.biblStruct]{biblStruct} \hyperref[TEI.listBibl]{listBibl} \hyperref[TEI.msDesc]{msDesc}] \hyperref[TEI.model.msQuoteLike]{model.msQuoteLike}[\hyperref[TEI.colophon]{colophon} \hyperref[TEI.explicit]{explicit} \hyperref[TEI.finalRubric]{finalRubric} \hyperref[TEI.incipit]{incipit} \hyperref[TEI.rubric]{rubric} \hyperref[TEI.title]{title}] \hyperref[TEI.model.quoteLike]{model.quoteLike}[\hyperref[TEI.cit]{cit} \hyperref[TEI.quote]{quote}] \hyperref[TEI.model.respLike]{model.respLike}[\hyperref[TEI.author]{author} \hyperref[TEI.editor]{editor} \hyperref[TEI.funder]{funder} \hyperref[TEI.meeting]{meeting} \hyperref[TEI.respStmt]{respStmt}] \hyperref[TEI.decoNote]{decoNote} \hyperref[TEI.filiation]{filiation} \hyperref[TEI.idno]{idno} \hyperref[TEI.msItem]{msItem} \hyperref[TEI.msItemStruct]{msItemStruct} \hyperref[TEI.textLang]{textLang}
\end{reflist}  
\begin{reflist}
\item[]\begin{specHead}{TEI.model.msQuoteLike}{model.msQuoteLike}\index{model.msQuoteLike (model class)|oddindex} regroupe des éléments qui représentent des passages d'un manuscrit, tels que des titres cités comme une partie de sa description\end{specHead} 
    \item[{Module}]
  tei
    \item[{Utilisé par}]
  \hyperref[TEI.model.msItemPart]{model.msItemPart}
    \item[{Membres}]
  \hyperref[TEI.colophon]{colophon} \hyperref[TEI.explicit]{explicit} \hyperref[TEI.finalRubric]{finalRubric} \hyperref[TEI.incipit]{incipit} \hyperref[TEI.rubric]{rubric} \hyperref[TEI.title]{title}
\end{reflist}  
\begin{reflist}
\item[]\begin{specHead}{TEI.model.nameLike}{model.nameLike}\index{model.nameLike (model class)|oddindex} regroupe des éléments qui nomment une personne, un lieu ou une organisation, ou qui y font référence à.\end{specHead} 
    \item[{Module}]
  tei
    \item[{Utilisé par}]
  \hyperref[TEI.model.addrPart]{model.addrPart} \hyperref[TEI.model.annotation]{model.annotation} \hyperref[TEI.model.pPart.data]{model.pPart.data} \hyperref[TEI.org]{org}
    \item[{Membres}]
  \hyperref[TEI.model.nameLike.agent]{model.nameLike.agent}[\hyperref[TEI.name]{name} \hyperref[TEI.orgName]{orgName} \hyperref[TEI.persName]{persName}] model.offsetLike \hyperref[TEI.model.persNamePart]{model.persNamePart}[\hyperref[TEI.addName]{addName} \hyperref[TEI.forename]{forename} \hyperref[TEI.genName]{genName} \hyperref[TEI.nameLink]{nameLink} \hyperref[TEI.roleName]{roleName} \hyperref[TEI.surname]{surname}] \hyperref[TEI.model.placeStateLike]{model.placeStateLike}[\hyperref[TEI.model.placeNamePart]{model.placeNamePart}[\hyperref[TEI.country]{country} \hyperref[TEI.geogName]{geogName} \hyperref[TEI.placeName]{placeName} \hyperref[TEI.region]{region} \hyperref[TEI.settlement]{settlement}] \hyperref[TEI.location]{location} \hyperref[TEI.state]{state}] \hyperref[TEI.idno]{idno} \hyperref[TEI.rs]{rs}
    \item[{Note}]
  \par
Un ensemble de niveau supérieur regroupant les éléments d'appellation qui peuvent apparaître dans les dates, les adresses, les mentions de responsabilité, etc.
\end{reflist}  
\begin{reflist}
\item[]\begin{specHead}{TEI.model.nameLike.agent}{model.nameLike.agent}\index{model.nameLike.agent (model class)|oddindex} regroupe des éléments qui contiennent des noms d'individus ou de personnes morales.\end{specHead} 
    \item[{Module}]
  tei
    \item[{Utilisé par}]
  \hyperref[TEI.model.nameLike]{model.nameLike} \hyperref[TEI.respStmt]{respStmt}
    \item[{Membres}]
  \hyperref[TEI.name]{name} \hyperref[TEI.orgName]{orgName} \hyperref[TEI.persName]{persName}
    \item[{Note}]
  \par
Cette classe est utilisée dans le modèle de contenu des éléments qui référencent des noms de personnes ou d'organisations.
\end{reflist}  
\begin{reflist}
\item[]\begin{specHead}{TEI.model.noteLike}{model.noteLike}\index{model.noteLike (model class)|oddindex} regroupe tous les éléments globaux de type note\end{specHead} 
    \item[{Module}]
  tei
    \item[{Utilisé par}]
  \hyperref[TEI.adminInfo]{adminInfo} \hyperref[TEI.biblStruct]{biblStruct} \hyperref[TEI.event]{event} \hyperref[TEI.location]{location} \hyperref[TEI.model.global]{model.global} \hyperref[TEI.monogr]{monogr} \hyperref[TEI.msItemStruct]{msItemStruct} \hyperref[TEI.notesStmt]{notesStmt} \hyperref[TEI.org]{org} \hyperref[TEI.place]{place} \hyperref[TEI.state]{state}
    \item[{Membres}]
  \hyperref[TEI.note]{note}
\end{reflist}  
\begin{reflist}
\item[]\begin{specHead}{TEI.model.orgPart}{model.orgPart}\index{model.orgPart (model class)|oddindex} groups elements which form part of the description of an organization.\end{specHead} 
    \item[{Module}]
  tei
    \item[{Utilisé par}]
  \hyperref[TEI.org]{org}
    \item[{Membres}]
  \hyperref[TEI.model.eventLike]{model.eventLike}[\hyperref[TEI.event]{event}] \hyperref[TEI.listOrg]{listOrg} \hyperref[TEI.listPlace]{listPlace}
\end{reflist}  
\begin{reflist}
\item[]\begin{specHead}{TEI.model.pLike}{model.pLike}\index{model.pLike (model class)|oddindex} regroupe des éléments de type paragraphe.\end{specHead} 
    \item[{Module}]
  tei
    \item[{Utilisé par}]
  \hyperref[TEI.abstract]{abstract} \hyperref[TEI.application]{application} \hyperref[TEI.availability]{availability} \hyperref[TEI.back]{back} \hyperref[TEI.binding]{binding} \hyperref[TEI.bindingDesc]{bindingDesc} \hyperref[TEI.correction]{correction} \hyperref[TEI.custodialHist]{custodialHist} \hyperref[TEI.decoDesc]{decoDesc} \hyperref[TEI.editionStmt]{editionStmt} \hyperref[TEI.encodingDesc]{encodingDesc} \hyperref[TEI.event]{event} \hyperref[TEI.front]{front} \hyperref[TEI.handDesc]{handDesc} \hyperref[TEI.history]{history} \hyperref[TEI.langUsage]{langUsage} \hyperref[TEI.layoutDesc]{layoutDesc} \hyperref[TEI.model.divPart]{model.divPart} \hyperref[TEI.msContents]{msContents} \hyperref[TEI.msDesc]{msDesc} \hyperref[TEI.msFrag]{msFrag} \hyperref[TEI.msItem]{msItem} \hyperref[TEI.msItemStruct]{msItemStruct} \hyperref[TEI.msPart]{msPart} \hyperref[TEI.objectDesc]{objectDesc} \hyperref[TEI.org]{org} \hyperref[TEI.person]{person} \hyperref[TEI.personGrp]{personGrp} \hyperref[TEI.persona]{persona} \hyperref[TEI.physDesc]{physDesc} \hyperref[TEI.place]{place} \hyperref[TEI.publicationStmt]{publicationStmt} \hyperref[TEI.recordHist]{recordHist} \hyperref[TEI.scriptDesc]{scriptDesc} \hyperref[TEI.seal]{seal} \hyperref[TEI.sealDesc]{sealDesc} \hyperref[TEI.seriesStmt]{seriesStmt} \hyperref[TEI.sourceDesc]{sourceDesc} \hyperref[TEI.sp]{sp} \hyperref[TEI.state]{state} \hyperref[TEI.supportDesc]{supportDesc} \hyperref[TEI.typeDesc]{typeDesc}
    \item[{Membres}]
  \hyperref[TEI.ab]{ab} \hyperref[TEI.p]{p}
\end{reflist}  
\begin{reflist}
\item[]\begin{specHead}{TEI.model.pLike.front}{model.pLike.front}\index{model.pLike.front (model class)|oddindex} regroupe des éléments de type paragraphe qui peuvent apparaître comme des constituants directs des parties liminaires.\end{specHead} 
    \item[{Module}]
  tei
    \item[{Utilisé par}]
  \hyperref[TEI.back]{back} \hyperref[TEI.front]{front}
    \item[{Membres}]
  \hyperref[TEI.docAuthor]{docAuthor} \hyperref[TEI.docDate]{docDate} \hyperref[TEI.docEdition]{docEdition} \hyperref[TEI.docTitle]{docTitle} \hyperref[TEI.head]{head} \hyperref[TEI.titlePart]{titlePart}
\end{reflist}  
\begin{reflist}
\item[]\begin{specHead}{TEI.model.pPart.data}{model.pPart.data}\index{model.pPart.data (model class)|oddindex} regroupe des éléments de niveau expression contenant des noms, des dates, des nombres, des mesures et d'autres données semblables\end{specHead} 
    \item[{Module}]
  tei
    \item[{Utilisé par}]
  \hyperref[TEI.bibl]{bibl} \hyperref[TEI.model.limitedPhrase]{model.limitedPhrase} \hyperref[TEI.model.phrase]{model.phrase}
    \item[{Membres}]
  \hyperref[TEI.model.addressLike]{model.addressLike}[\hyperref[TEI.address]{address} \hyperref[TEI.affiliation]{affiliation} \hyperref[TEI.email]{email}] \hyperref[TEI.model.dateLike]{model.dateLike}[\hyperref[TEI.date]{date} \hyperref[TEI.time]{time}] \hyperref[TEI.model.measureLike]{model.measureLike}[\hyperref[TEI.depth]{depth} \hyperref[TEI.dim]{dim} \hyperref[TEI.height]{height} \hyperref[TEI.measure]{measure} \hyperref[TEI.measureGrp]{measureGrp} \hyperref[TEI.num]{num} \hyperref[TEI.width]{width}] \hyperref[TEI.model.nameLike]{model.nameLike}[\hyperref[TEI.model.nameLike.agent]{model.nameLike.agent}[\hyperref[TEI.name]{name} \hyperref[TEI.orgName]{orgName} \hyperref[TEI.persName]{persName}] model.offsetLike \hyperref[TEI.model.persNamePart]{model.persNamePart}[\hyperref[TEI.addName]{addName} \hyperref[TEI.forename]{forename} \hyperref[TEI.genName]{genName} \hyperref[TEI.nameLink]{nameLink} \hyperref[TEI.roleName]{roleName} \hyperref[TEI.surname]{surname}] \hyperref[TEI.model.placeStateLike]{model.placeStateLike}[\hyperref[TEI.model.placeNamePart]{model.placeNamePart}[\hyperref[TEI.country]{country} \hyperref[TEI.geogName]{geogName} \hyperref[TEI.placeName]{placeName} \hyperref[TEI.region]{region} \hyperref[TEI.settlement]{settlement}] \hyperref[TEI.location]{location} \hyperref[TEI.state]{state}] \hyperref[TEI.idno]{idno} \hyperref[TEI.rs]{rs}]
\end{reflist}  
\begin{reflist}
\item[]\begin{specHead}{TEI.model.pPart.edit}{model.pPart.edit}\index{model.pPart.edit (model class)|oddindex} regroupe des éléments de niveau expression, utilisés pour de simples interventions éditoriales de corrections et de transcriptions.\end{specHead} 
    \item[{Module}]
  tei
    \item[{Utilisé par}]
  \hyperref[TEI.bibl]{bibl} \hyperref[TEI.model.phrase]{model.phrase} \hyperref[TEI.pc]{pc} \hyperref[TEI.w]{w}
    \item[{Membres}]
  \hyperref[TEI.model.pPart.editorial]{model.pPart.editorial}[\hyperref[TEI.abbr]{abbr} \hyperref[TEI.am]{am} \hyperref[TEI.choice]{choice} \hyperref[TEI.ex]{ex} \hyperref[TEI.expan]{expan} \hyperref[TEI.subst]{subst}] \hyperref[TEI.model.pPart.transcriptional]{model.pPart.transcriptional}[\hyperref[TEI.add]{add} \hyperref[TEI.corr]{corr} \hyperref[TEI.damage]{damage} \hyperref[TEI.del]{del} \hyperref[TEI.handShift]{handShift} \hyperref[TEI.mod]{mod} \hyperref[TEI.orig]{orig} \hyperref[TEI.redo]{redo} \hyperref[TEI.reg]{reg} \hyperref[TEI.restore]{restore} \hyperref[TEI.retrace]{retrace} \hyperref[TEI.secl]{secl} \hyperref[TEI.sic]{sic} \hyperref[TEI.supplied]{supplied} \hyperref[TEI.surplus]{surplus} \hyperref[TEI.unclear]{unclear} \hyperref[TEI.undo]{undo}]
\end{reflist}  
\begin{reflist}
\item[]\begin{specHead}{TEI.model.pPart.editorial}{model.pPart.editorial}\index{model.pPart.editorial (model class)|oddindex} regroupe des éléments de niveau expression, utilisés pour de simples interventions éditoriales utiles dans la transcription comme dans la rédaction.\end{specHead} 
    \item[{Module}]
  tei
    \item[{Utilisé par}]
  \hyperref[TEI.model.limitedPhrase]{model.limitedPhrase} \hyperref[TEI.model.pPart.edit]{model.pPart.edit}
    \item[{Membres}]
  \hyperref[TEI.abbr]{abbr} \hyperref[TEI.am]{am} \hyperref[TEI.choice]{choice} \hyperref[TEI.ex]{ex} \hyperref[TEI.expan]{expan} \hyperref[TEI.subst]{subst}
\end{reflist}  
\begin{reflist}
\item[]\begin{specHead}{TEI.model.pPart.msdesc}{model.pPart.msdesc}\index{model.pPart.msdesc (model class)|oddindex} regroupe des éléments de niveau expression utilisés pour décrire des manuscrits\end{specHead} 
    \item[{Module}]
  tei
    \item[{Utilisé par}]
  \hyperref[TEI.model.limitedPhrase]{model.limitedPhrase} \hyperref[TEI.model.phrase]{model.phrase}
    \item[{Membres}]
  \hyperref[TEI.catchwords]{catchwords} \hyperref[TEI.dimensions]{dimensions} \hyperref[TEI.heraldry]{heraldry} \hyperref[TEI.locus]{locus} \hyperref[TEI.locusGrp]{locusGrp} \hyperref[TEI.material]{material} \hyperref[TEI.objectType]{objectType} \hyperref[TEI.origDate]{origDate} \hyperref[TEI.origPlace]{origPlace} \hyperref[TEI.secFol]{secFol} \hyperref[TEI.signatures]{signatures} \hyperref[TEI.stamp]{stamp} \hyperref[TEI.watermark]{watermark}
\end{reflist}  
\begin{reflist}
\item[]\begin{specHead}{TEI.model.pPart.transcriptional}{model.pPart.transcriptional}\index{model.pPart.transcriptional (model class)|oddindex} regroupe des éléments de niveau expression, utilisés pour des transcriptions éditoriales de sources pré-existantes\end{specHead} 
    \item[{Module}]
  tei
    \item[{Utilisé par}]
  \hyperref[TEI.am]{am} \hyperref[TEI.model.pPart.edit]{model.pPart.edit}
    \item[{Membres}]
  \hyperref[TEI.add]{add} \hyperref[TEI.corr]{corr} \hyperref[TEI.damage]{damage} \hyperref[TEI.del]{del} \hyperref[TEI.handShift]{handShift} \hyperref[TEI.mod]{mod} \hyperref[TEI.orig]{orig} \hyperref[TEI.redo]{redo} \hyperref[TEI.reg]{reg} \hyperref[TEI.restore]{restore} \hyperref[TEI.retrace]{retrace} \hyperref[TEI.secl]{secl} \hyperref[TEI.sic]{sic} \hyperref[TEI.supplied]{supplied} \hyperref[TEI.surplus]{surplus} \hyperref[TEI.unclear]{unclear} \hyperref[TEI.undo]{undo}
\end{reflist}  
\begin{reflist}
\item[]\begin{specHead}{TEI.model.persNamePart}{model.persNamePart}\index{model.persNamePart (model class)|oddindex} regroupe des éléments qui font partie d'un nom de personne\end{specHead} 
    \item[{Module}]
  namesdates
    \item[{Utilisé par}]
  \hyperref[TEI.model.nameLike]{model.nameLike}
    \item[{Membres}]
  \hyperref[TEI.addName]{addName} \hyperref[TEI.forename]{forename} \hyperref[TEI.genName]{genName} \hyperref[TEI.nameLink]{nameLink} \hyperref[TEI.roleName]{roleName} \hyperref[TEI.surname]{surname}
\end{reflist}  
\begin{reflist}
\item[]\begin{specHead}{TEI.model.persStateLike}{model.persStateLike}\index{model.persStateLike (model class)|oddindex} regroupe des éléments décrivant les caractéristiques d'une personne, variables mais définies dans le temps, par exemple sa profession, son lieu de résidence ou son nom.\end{specHead} 
    \item[{Module}]
  tei
    \item[{Utilisé par}]
  \hyperref[TEI.model.personPart]{model.personPart}
    \item[{Membres}]
  \hyperref[TEI.affiliation]{affiliation} \hyperref[TEI.persName]{persName} \hyperref[TEI.persona]{persona} \hyperref[TEI.state]{state}
    \item[{Note}]
  \par
Il s'agit en général des caractéristiques d'un individu résultant de sa propre action ou de celle d'autrui.
\end{reflist}  
\begin{reflist}
\item[]\begin{specHead}{TEI.model.personLike}{model.personLike}\index{model.personLike (model class)|oddindex} regroupe des éléments qui donnent des informations sur des personnes et leurs relations.\end{specHead} 
    \item[{Module}]
  tei
    \item[{Utilisé par}]
  \hyperref[TEI.org]{org}
    \item[{Membres}]
  \hyperref[TEI.org]{org} \hyperref[TEI.person]{person} \hyperref[TEI.personGrp]{personGrp}
\end{reflist}  
\begin{reflist}
\item[]\begin{specHead}{TEI.model.personPart}{model.personPart}\index{model.personPart (model class)|oddindex} regroupe des éléments qui font partie de la description d'une personne\end{specHead} 
    \item[{Module}]
  tei
    \item[{Utilisé par}]
  \hyperref[TEI.person]{person} \hyperref[TEI.personGrp]{personGrp} \hyperref[TEI.persona]{persona}
    \item[{Membres}]
  \hyperref[TEI.model.biblLike]{model.biblLike}[\hyperref[TEI.bibl]{bibl} \hyperref[TEI.biblFull]{biblFull} \hyperref[TEI.biblStruct]{biblStruct} \hyperref[TEI.listBibl]{listBibl} \hyperref[TEI.msDesc]{msDesc}] \hyperref[TEI.model.eventLike]{model.eventLike}[\hyperref[TEI.event]{event}] \hyperref[TEI.model.persStateLike]{model.persStateLike}[\hyperref[TEI.affiliation]{affiliation} \hyperref[TEI.persName]{persName} \hyperref[TEI.persona]{persona} \hyperref[TEI.state]{state}] \hyperref[TEI.idno]{idno} \hyperref[TEI.name]{name}
\end{reflist}  
\begin{reflist}
\item[]\begin{specHead}{TEI.model.phrase}{model.phrase}\index{model.phrase (model class)|oddindex} regroupe des éléments qui apparaissent au niveau des mots isolés ou des groupes de mots.\end{specHead} 
    \item[{Module}]
  tei
    \item[{Utilisé par}]
  \hyperref[TEI.date]{date} \hyperref[TEI.head]{head} \hyperref[TEI.l]{l} \hyperref[TEI.macro.paraContent]{macro.paraContent} \hyperref[TEI.macro.phraseSeq]{macro.phraseSeq} \hyperref[TEI.macro.specialPara]{macro.specialPara} \hyperref[TEI.origDate]{origDate} \hyperref[TEI.time]{time}
    \item[{Membres}]
  \hyperref[TEI.model.graphicLike]{model.graphicLike}[\hyperref[TEI.binaryObject]{binaryObject} \hyperref[TEI.formula]{formula} \hyperref[TEI.graphic]{graphic} \hyperref[TEI.math]{math} \hyperref[TEI.media]{media} \hyperref[TEI.mrow]{mrow}] \hyperref[TEI.model.highlighted]{model.highlighted}[\hyperref[TEI.model.emphLike]{model.emphLike}[\hyperref[TEI.distinct]{distinct} \hyperref[TEI.emph]{emph} \hyperref[TEI.foreign]{foreign} \hyperref[TEI.gloss]{gloss} \hyperref[TEI.mentioned]{mentioned} \hyperref[TEI.soCalled]{soCalled} \hyperref[TEI.term]{term} \hyperref[TEI.title]{title}] \hyperref[TEI.model.hiLike]{model.hiLike}[\hyperref[TEI.hi]{hi}]] model.lPart \hyperref[TEI.model.pPart.data]{model.pPart.data}[\hyperref[TEI.model.addressLike]{model.addressLike}[\hyperref[TEI.address]{address} \hyperref[TEI.affiliation]{affiliation} \hyperref[TEI.email]{email}] \hyperref[TEI.model.dateLike]{model.dateLike}[\hyperref[TEI.date]{date} \hyperref[TEI.time]{time}] \hyperref[TEI.model.measureLike]{model.measureLike}[\hyperref[TEI.depth]{depth} \hyperref[TEI.dim]{dim} \hyperref[TEI.height]{height} \hyperref[TEI.measure]{measure} \hyperref[TEI.measureGrp]{measureGrp} \hyperref[TEI.num]{num} \hyperref[TEI.width]{width}] \hyperref[TEI.model.nameLike]{model.nameLike}[\hyperref[TEI.model.nameLike.agent]{model.nameLike.agent}[\hyperref[TEI.name]{name} \hyperref[TEI.orgName]{orgName} \hyperref[TEI.persName]{persName}] model.offsetLike \hyperref[TEI.model.persNamePart]{model.persNamePart}[\hyperref[TEI.addName]{addName} \hyperref[TEI.forename]{forename} \hyperref[TEI.genName]{genName} \hyperref[TEI.nameLink]{nameLink} \hyperref[TEI.roleName]{roleName} \hyperref[TEI.surname]{surname}] \hyperref[TEI.model.placeStateLike]{model.placeStateLike}[\hyperref[TEI.model.placeNamePart]{model.placeNamePart}[\hyperref[TEI.country]{country} \hyperref[TEI.geogName]{geogName} \hyperref[TEI.placeName]{placeName} \hyperref[TEI.region]{region} \hyperref[TEI.settlement]{settlement}] \hyperref[TEI.location]{location} \hyperref[TEI.state]{state}] \hyperref[TEI.idno]{idno} \hyperref[TEI.rs]{rs}]] \hyperref[TEI.model.pPart.edit]{model.pPart.edit}[\hyperref[TEI.model.pPart.editorial]{model.pPart.editorial}[\hyperref[TEI.abbr]{abbr} \hyperref[TEI.am]{am} \hyperref[TEI.choice]{choice} \hyperref[TEI.ex]{ex} \hyperref[TEI.expan]{expan} \hyperref[TEI.subst]{subst}] \hyperref[TEI.model.pPart.transcriptional]{model.pPart.transcriptional}[\hyperref[TEI.add]{add} \hyperref[TEI.corr]{corr} \hyperref[TEI.damage]{damage} \hyperref[TEI.del]{del} \hyperref[TEI.handShift]{handShift} \hyperref[TEI.mod]{mod} \hyperref[TEI.orig]{orig} \hyperref[TEI.redo]{redo} \hyperref[TEI.reg]{reg} \hyperref[TEI.restore]{restore} \hyperref[TEI.retrace]{retrace} \hyperref[TEI.secl]{secl} \hyperref[TEI.sic]{sic} \hyperref[TEI.supplied]{supplied} \hyperref[TEI.surplus]{surplus} \hyperref[TEI.unclear]{unclear} \hyperref[TEI.undo]{undo}]] \hyperref[TEI.model.pPart.msdesc]{model.pPart.msdesc}[\hyperref[TEI.catchwords]{catchwords} \hyperref[TEI.dimensions]{dimensions} \hyperref[TEI.heraldry]{heraldry} \hyperref[TEI.locus]{locus} \hyperref[TEI.locusGrp]{locusGrp} \hyperref[TEI.material]{material} \hyperref[TEI.objectType]{objectType} \hyperref[TEI.origDate]{origDate} \hyperref[TEI.origPlace]{origPlace} \hyperref[TEI.secFol]{secFol} \hyperref[TEI.signatures]{signatures} \hyperref[TEI.stamp]{stamp} \hyperref[TEI.watermark]{watermark}] model.phrase.xml \hyperref[TEI.model.ptrLike]{model.ptrLike}[\hyperref[TEI.ptr]{ptr} \hyperref[TEI.ref]{ref}] \hyperref[TEI.model.segLike]{model.segLike}[\hyperref[TEI.annotationBlock]{annotationBlock} \hyperref[TEI.c]{c} \hyperref[TEI.cl]{cl} \hyperref[TEI.m]{m} \hyperref[TEI.pc]{pc} \hyperref[TEI.phr]{phr} \hyperref[TEI.s]{s} \hyperref[TEI.seg]{seg} \hyperref[TEI.w]{w}] model.specDescLike
    \item[{Note}]
  \par
Cette classe d'éléments peut se trouver dans des paragraphes, des entrées de listes, des vers, etc.
\end{reflist}  
\begin{reflist}
\item[]\begin{specHead}{TEI.model.physDescPart}{model.physDescPart}\index{model.physDescPart (model class)|oddindex} regroupe des éléments spécialisés constituant la description physique d'un manuscrit ou d'une source écrite de même nature\end{specHead} 
    \item[{Module}]
  tei
    \item[{Utilisé par}]
  \hyperref[TEI.physDesc]{physDesc}
    \item[{Membres}]
  \hyperref[TEI.accMat]{accMat} \hyperref[TEI.additions]{additions} \hyperref[TEI.bindingDesc]{bindingDesc} \hyperref[TEI.decoDesc]{decoDesc} \hyperref[TEI.handDesc]{handDesc} \hyperref[TEI.musicNotation]{musicNotation} \hyperref[TEI.objectDesc]{objectDesc} \hyperref[TEI.scriptDesc]{scriptDesc} \hyperref[TEI.sealDesc]{sealDesc} \hyperref[TEI.typeDesc]{typeDesc}
\end{reflist}  
\begin{reflist}
\item[]\begin{specHead}{TEI.model.placeLike}{model.placeLike}\index{model.placeLike (model class)|oddindex} regroupe des éléments qui donne des informations sur des lieux et leurs relations.\end{specHead} 
    \item[{Module}]
  tei
    \item[{Utilisé par}]
  \hyperref[TEI.listPlace]{listPlace} \hyperref[TEI.org]{org} \hyperref[TEI.place]{place}
    \item[{Membres}]
  \hyperref[TEI.place]{place}
\end{reflist}  
\begin{reflist}
\item[]\begin{specHead}{TEI.model.placeNamePart}{model.placeNamePart}\index{model.placeNamePart (model class)|oddindex} regroupe des éléments qui font partie d'un nom de lieu.\end{specHead} 
    \item[{Module}]
  tei
    \item[{Utilisé par}]
  \hyperref[TEI.altIdentifier]{altIdentifier} \hyperref[TEI.geogName]{geogName} \hyperref[TEI.location]{location} \hyperref[TEI.model.placeStateLike]{model.placeStateLike} \hyperref[TEI.msIdentifier]{msIdentifier}
    \item[{Membres}]
  \hyperref[TEI.country]{country} \hyperref[TEI.geogName]{geogName} \hyperref[TEI.placeName]{placeName} \hyperref[TEI.region]{region} \hyperref[TEI.settlement]{settlement}
\end{reflist}  
\begin{reflist}
\item[]\begin{specHead}{TEI.model.placeStateLike}{model.placeStateLike}\index{model.placeStateLike (model class)|oddindex} regroupe des éléments qui décrivent les transformations d'un lieu\end{specHead} 
    \item[{Module}]
  tei
    \item[{Utilisé par}]
  \hyperref[TEI.model.nameLike]{model.nameLike} \hyperref[TEI.place]{place}
    \item[{Membres}]
  \hyperref[TEI.model.placeNamePart]{model.placeNamePart}[\hyperref[TEI.country]{country} \hyperref[TEI.geogName]{geogName} \hyperref[TEI.placeName]{placeName} \hyperref[TEI.region]{region} \hyperref[TEI.settlement]{settlement}] \hyperref[TEI.location]{location} \hyperref[TEI.state]{state}
\end{reflist}  
\begin{reflist}
\item[]\begin{specHead}{TEI.model.profileDescPart}{model.profileDescPart}\index{model.profileDescPart (model class)|oddindex} regroupe des éléments que l'on peut utiliser plusieurs fois dans l'élément \hyperref[TEI.profileDesc]{<profileDesc>}.\end{specHead} 
    \item[{Module}]
  tei
    \item[{Utilisé par}]
  \hyperref[TEI.profileDesc]{profileDesc}
    \item[{Membres}]
  \hyperref[TEI.abstract]{abstract} \hyperref[TEI.creation]{creation} \hyperref[TEI.handNotes]{handNotes} \hyperref[TEI.langUsage]{langUsage} \hyperref[TEI.listTranspose]{listTranspose} \hyperref[TEI.textClass]{textClass}
\end{reflist}  
\begin{reflist}
\item[]\begin{specHead}{TEI.model.ptrLike}{model.ptrLike}\index{model.ptrLike (model class)|oddindex} regroupe des éléments utilisés pour localiser et faire référence à quelque chose.\end{specHead} 
    \item[{Module}]
  tei
    \item[{Utilisé par}]
  \hyperref[TEI.analytic]{analytic} \hyperref[TEI.application]{application} \hyperref[TEI.bibl]{bibl} \hyperref[TEI.biblStruct]{biblStruct} \hyperref[TEI.cit]{cit} \hyperref[TEI.model.annotation]{model.annotation} \hyperref[TEI.model.limitedPhrase]{model.limitedPhrase} \hyperref[TEI.model.phrase]{model.phrase} \hyperref[TEI.model.publicationStmtPart.detail]{model.publicationStmtPart.detail} \hyperref[TEI.monogr]{monogr} \hyperref[TEI.notatedMusic]{notatedMusic} \hyperref[TEI.relatedItem]{relatedItem} \hyperref[TEI.series]{series}
    \item[{Membres}]
  \hyperref[TEI.ptr]{ptr} \hyperref[TEI.ref]{ref}
\end{reflist}  
\begin{reflist}
\item[]\begin{specHead}{TEI.model.publicationStmtPart.agency}{model.publicationStmtPart.agency}\index{model.publicationStmtPart.agency (model class)|oddindex} regroupe des éléments qui peuvent apparaître à l'intérieur de l'élément \hyperref[TEI.publicationStmt]{<publicationStmt>} de l'En-tête TEI\end{specHead} 
    \item[{Module}]
  tei
    \item[{Utilisé par}]
  \hyperref[TEI.publicationStmt]{publicationStmt}
    \item[{Membres}]
  \hyperref[TEI.authority]{authority} \hyperref[TEI.distributor]{distributor} \hyperref[TEI.publisher]{publisher}
\end{reflist}  
\begin{reflist}
\item[]\begin{specHead}{TEI.model.publicationStmtPart.detail}{model.publicationStmtPart.detail}\index{model.publicationStmtPart.detail (model class)|oddindex} regroupe des éléments qui peuvent apparaître à l'intérieur de l'élément \hyperref[TEI.publicationStmt]{<publicationStmt>} de l'En-tête TEI\end{specHead} 
    \item[{Module}]
  tei
    \item[{Utilisé par}]
  \hyperref[TEI.publicationStmt]{publicationStmt}
    \item[{Membres}]
  \hyperref[TEI.model.ptrLike]{model.ptrLike}[\hyperref[TEI.ptr]{ptr} \hyperref[TEI.ref]{ref}] \hyperref[TEI.address]{address} \hyperref[TEI.availability]{availability} \hyperref[TEI.date]{date} \hyperref[TEI.idno]{idno} \hyperref[TEI.pubPlace]{pubPlace}
\end{reflist}  
\begin{reflist}
\item[]\begin{specHead}{TEI.model.qLike}{model.qLike}\index{model.qLike (model class)|oddindex} regroupe des éléments destinés à la mise en valeur, qui peuvent apparaître à l'intérieur ou entre des éléments de niveau fragment.\end{specHead} 
    \item[{Module}]
  tei
    \item[{Utilisé par}]
  \hyperref[TEI.cit]{cit} \hyperref[TEI.model.inter]{model.inter} \hyperref[TEI.sp]{sp}
    \item[{Membres}]
  \hyperref[TEI.model.quoteLike]{model.quoteLike}[\hyperref[TEI.cit]{cit} \hyperref[TEI.quote]{quote}] \hyperref[TEI.floatingText]{floatingText} \hyperref[TEI.q]{q} \hyperref[TEI.said]{said}
\end{reflist}  
\begin{reflist}
\item[]\begin{specHead}{TEI.model.quoteLike}{model.quoteLike}\index{model.quoteLike (model class)|oddindex} regroupe des éléments employés pour contenir directement des citations\end{specHead} 
    \item[{Module}]
  tei
    \item[{Utilisé par}]
  \hyperref[TEI.model.msItemPart]{model.msItemPart} \hyperref[TEI.model.qLike]{model.qLike}
    \item[{Membres}]
  \hyperref[TEI.cit]{cit} \hyperref[TEI.quote]{quote}
\end{reflist}  
\begin{reflist}
\item[]\begin{specHead}{TEI.model.resourceLike}{model.resourceLike}\index{model.resourceLike (model class)|oddindex} regroupe des éléments non-textuels qui, avec un en-tête et un texte, constitue un document TEI.\end{specHead} 
    \item[{Module}]
  tei
    \item[{Utilisé par}]
  \hyperref[TEI.TEI]{TEI} \hyperref[TEI.standOff]{standOff} \hyperref[TEI.teiCorpus]{teiCorpus}
    \item[{Membres}]
  \hyperref[TEI.facsimile]{facsimile} \hyperref[TEI.fsdDecl]{fsdDecl} \hyperref[TEI.sourceDoc]{sourceDoc} \hyperref[TEI.standOff]{standOff} \hyperref[TEI.text]{text}
\end{reflist}  
\begin{reflist}
\item[]\begin{specHead}{TEI.model.respLike}{model.respLike}\index{model.respLike (model class)|oddindex} regroupe des éléments qui sont utilisés pour indiquer une responsabilité intellectuelle ou une autre responsabilité significative, par exemple dans un élément bibliographique.\end{specHead} 
    \item[{Module}]
  tei
    \item[{Utilisé par}]
  \hyperref[TEI.editionStmt]{editionStmt} \hyperref[TEI.model.biblPart]{model.biblPart} \hyperref[TEI.model.msItemPart]{model.msItemPart} \hyperref[TEI.titleStmt]{titleStmt}
    \item[{Membres}]
  \hyperref[TEI.author]{author} \hyperref[TEI.editor]{editor} \hyperref[TEI.funder]{funder} \hyperref[TEI.meeting]{meeting} \hyperref[TEI.respStmt]{respStmt}
\end{reflist}  
\begin{reflist}
\item[]\begin{specHead}{TEI.model.segLike}{model.segLike}\index{model.segLike (model class)|oddindex} regroupe des éléments utilisés pour une segmentation arbitraire.\end{specHead} 
    \item[{Module}]
  tei
    \item[{Utilisé par}]
  \hyperref[TEI.bibl]{bibl} \hyperref[TEI.model.phrase]{model.phrase}
    \item[{Membres}]
  \hyperref[TEI.annotationBlock]{annotationBlock} \hyperref[TEI.c]{c} \hyperref[TEI.cl]{cl} \hyperref[TEI.m]{m} \hyperref[TEI.pc]{pc} \hyperref[TEI.phr]{phr} \hyperref[TEI.s]{s} \hyperref[TEI.seg]{seg} \hyperref[TEI.w]{w}
    \item[{Note}]
  \par
Les principes sur lesquels repose la segmentation, ainsi que tout code particulier ou valeur d'attribut utilisée, doivent être définis explicitement dans l'élément \texttt{<segmentation>} de l'élément \hyperref[TEI.encodingDesc]{<encodingDesc>} situé dans l'En-tête TEI associé.
\end{reflist}  
\begin{reflist}
\item[]\begin{specHead}{TEI.model.stageLike}{model.stageLike}\index{model.stageLike (model class)|oddindex} regroupe des éléments contenant des indications scéniques ou des indications de même nature, définies par le module relatif aux textes de théâtre\end{specHead} 
    \item[{Module}]
  tei
    \item[{Utilisé par}]
  \hyperref[TEI.lg]{lg} \hyperref[TEI.model.inter]{model.inter} \hyperref[TEI.sp]{sp}
    \item[{Membres}]
  \hyperref[TEI.stage]{stage}
    \item[{Note}]
  \par
Les indications scéniques appartiennent à la classe \textit{inter} : cela signifie qu'elles peuvent apparaître à l'intérieur d'éléments de niveau composant ou bien entre ces éléments.
\end{reflist}  
\begin{reflist}
\item[]\begin{specHead}{TEI.model.teiHeaderPart}{model.teiHeaderPart}\index{model.teiHeaderPart (model class)|oddindex} regroupe des éléments de macrostructure qui peuvent apparaître plus d'une fois dans l’en-tête TEI.\end{specHead} 
    \item[{Module}]
  tei
    \item[{Utilisé par}]
  \hyperref[TEI.teiHeader]{teiHeader}
    \item[{Membres}]
  \hyperref[TEI.encodingDesc]{encodingDesc} \hyperref[TEI.profileDesc]{profileDesc}
\end{reflist}  
\begin{reflist}
\item[]\begin{specHead}{TEI.model.titlepagePart}{model.titlepagePart}\index{model.titlepagePart (model class)|oddindex} regroupe des éléments qui peuvent apparaître comme constituants directs d'une page de titre (\hyperref[TEI.docTitle]{<docTitle>}, \texttt{<docAuth>},\texttt{<docImprint>} ou \texttt{<epigraph>})\end{specHead} 
    \item[{Module}]
  tei
    \item[{Utilisé par}]
  \hyperref[TEI.msItem]{msItem} \hyperref[TEI.titlePage]{titlePage}
    \item[{Membres}]
  \hyperref[TEI.binaryObject]{binaryObject} \hyperref[TEI.docAuthor]{docAuthor} \hyperref[TEI.docDate]{docDate} \hyperref[TEI.docEdition]{docEdition} \hyperref[TEI.docTitle]{docTitle} \hyperref[TEI.graphic]{graphic} \hyperref[TEI.titlePart]{titlePart}
\end{reflist}  
\section[{Attribute classes}]{Attribute classes}
\begin{reflist}
\item[]\begin{specHead}{TEI.att.ascribed}{att.ascribed}\index{att.ascribed (attribute class)|oddindex}\index{who=@who!att.ascribed (attribute class)|oddindex} fournit des attributs pour des éléments transcrivant la parole ou l'action qui peuvent être attribuées à un individu en particulier.\end{specHead} 
    \item[{Module}]
  tei
    \item[{Membres}]
  \hyperref[TEI.annotationBlock]{annotationBlock} \hyperref[TEI.change]{change} \hyperref[TEI.q]{q} \hyperref[TEI.said]{said} \hyperref[TEI.sp]{sp} \hyperref[TEI.stage]{stage}
    \item[{Attributs}]
  Attributs\hfil\\[-10pt]\begin{sansreflist}
    \item[@who]
  indique la personne ou le groupe de personnes à qui le contenu de l'élément est attribué.
\begin{reflist}
    \item[{Statut}]
  Optionel
    \item[{Type de données}]
  1–∞ occurrences de \hyperref[TEI.teidata.pointer]{teidata.pointer} séparé par un espace
    \item[]In the following example from Hamlet, speeches (\hyperref[TEI.sp]{<sp>}) in the body of the play are linked to \texttt{<castItem>} elements in the \texttt{<castList>} using the {\itshape who} attribute.\exampleFont {<\textbf{castItem}\hspace*{6pt}{type}="{role}">}\mbox{}\newline 
\hspace*{6pt}{<\textbf{role}\hspace*{6pt}{xml:id}="{Barnardo}">}Bernardo{</\textbf{role}>}\mbox{}\newline 
{</\textbf{castItem}>}\mbox{}\newline 
{<\textbf{castItem}\hspace*{6pt}{type}="{role}">}\mbox{}\newline 
\hspace*{6pt}{<\textbf{role}\hspace*{6pt}{xml:id}="{Francisco}">}Francisco{</\textbf{role}>}\mbox{}\newline 
\hspace*{6pt}{<\textbf{roleDesc}>}a soldier{</\textbf{roleDesc}>}\mbox{}\newline 
{</\textbf{castItem}>}\mbox{}\newline 
{<\textbf{sp}\hspace*{6pt}{who}="{\#Barnardo}">}\mbox{}\newline 
\hspace*{6pt}{<\textbf{speaker}>}Bernardo{</\textbf{speaker}>}\mbox{}\newline 
\hspace*{6pt}{<\textbf{l}\hspace*{6pt}{n}="{1}">}Who's there?{</\textbf{l}>}\mbox{}\newline 
{</\textbf{sp}>}\mbox{}\newline 
{<\textbf{sp}\hspace*{6pt}{who}="{\#Francisco}">}\mbox{}\newline 
\hspace*{6pt}{<\textbf{speaker}>}Francisco{</\textbf{speaker}>}\mbox{}\newline 
\hspace*{6pt}{<\textbf{l}\hspace*{6pt}{n}="{2}">}Nay, answer me: stand, and unfold yourself.{</\textbf{l}>}\mbox{}\newline 
{</\textbf{sp}>}
    \item[{Note}]
  \par
Pour un discours transcrit, identifiera typiquement un participant ou un groupe participant ; dans d'autres contextes, pointera vers n'importe quel élément \hyperref[TEI.person]{<person>} identifié.
\end{reflist}  
\end{sansreflist}  
\end{reflist}  
\begin{reflist}
\item[]\begin{specHead}{TEI.att.breaking}{att.breaking}\index{att.breaking (attribute class)|oddindex}\index{break=@break!att.breaking (attribute class)|oddindex} fournit un attribut pour indiquer si un élément est consideré ou pas comme marquant la fin d'un mot orthographique, comme le fait une espace.\end{specHead} 
    \item[{Module}]
  tei
    \item[{Membres}]
  \hyperref[TEI.cb]{cb} \hyperref[TEI.gb]{gb} \hyperref[TEI.lb]{lb} \hyperref[TEI.milestone]{milestone} \hyperref[TEI.pb]{pb}
    \item[{Attributs}]
  Attributs\hfil\\[-10pt]\begin{sansreflist}
    \item[@break]
  indique si l'élément qui porte cet attribut peut être considéré comme une espace blanc indiquant la fin d'un mot orthographique.
\begin{reflist}
    \item[{Statut}]
  Recommendé
    \item[{Type de données}]
  \hyperref[TEI.teidata.enumerated]{teidata.enumerated}
    \item[{Sample values include}]
  \begin{description}

\item[{yes}]l'élément qui porte cet attribut peut être considéré comme indiquant la fin d'un mot orthographique
\item[{no}]l'élément qui porte cet attribut ne peut être pas considéré comme indiquant la fin d'un mot orthographique
\item[{maybe}]l'encodage ne prends aucune position sur la question.
\end{description} 
    \item[]In the following lines from the ‘Dream of the Rood’, linebreaks occur in the middle of the words \textit{lāðost} and \textit{reord-berendum}.\exampleFont {<\textbf{ab}>} ...eƿesa tome iu icƿæs ȝeƿorden ƿita heardoſt .\mbox{}\newline 
 leodum la{<\textbf{lb}\hspace*{6pt}{break}="{no}"/>} ðost ærþan ichim lifes\mbox{}\newline 
 ƿeȝ rihtne ȝerymde reord be{<\textbf{lb}\hspace*{6pt}{break}="{no}"/>}\mbox{}\newline 
 rendum hƿæt me þaȝeƿeorðode ƿuldres ealdor ofer...\mbox{}\newline 
{</\textbf{ab}>}
\end{reflist}  
\end{sansreflist}  
\end{reflist}  
\begin{reflist}
\item[]\begin{specHead}{TEI.att.cReferencing}{att.cReferencing}\index{att.cReferencing (attribute class)|oddindex}\index{cRef=@cRef!att.cReferencing (attribute class)|oddindex} provides an attribute which may be used to supply a \textit{canonical reference} as a means of identifying the target of a pointer.\end{specHead} 
    \item[{Module}]
  tei
    \item[{Membres}]
  \hyperref[TEI.gloss]{gloss} \hyperref[TEI.ptr]{ptr} \hyperref[TEI.ref]{ref} \hyperref[TEI.term]{term}
    \item[{Attributs}]
  Attributs\hfil\\[-10pt]\begin{sansreflist}
    \item[@cRef]
  (référence canonique) précise la cible du pointeur en fournissant une référence canonique issue d'un modèle défini par un élément \texttt{<refsDecl>}dans l'En-tête TEI.
\begin{reflist}
    \item[{Statut}]
  Optionel
    \item[{Type de données}]
  \hyperref[TEI.teidata.text]{teidata.text}
    \item[{Note}]
  \par
Le résultat de l’application de l'algorithme pour la résolution des références canoniques (décrit à la section \xref{http://www.tei-c.org/release/doc/tei-p5-doc/en/html/SA.html\#SACR}{16.2.5. Canonical References}). Ce sera une référence URI valide pour la cible prévue.\par
La \texttt{<refsDecl>} à utiliser peut être indiquée à l'aide de l'attribut {\itshape decls}. Actuellement ces Principes directeurs ne permettent que l'encodage d'une unique référence canonique pour tout élément \hyperref[TEI.ptr]{<ptr>} donné.
\end{reflist}  
\end{sansreflist}  
\end{reflist}  
\begin{reflist}
\item[]\begin{specHead}{TEI.att.canonical}{att.canonical}\index{att.canonical (attribute class)|oddindex}\index{key=@key!att.canonical (attribute class)|oddindex}\index{ref=@ref!att.canonical (attribute class)|oddindex} fournit des attributs qui peuvent être utilisés pour associer une représentation telle qu'un nom ou un titre à l'information canonique concernant l'objet nommé ou auquel il est fait référence.\end{specHead} 
    \item[{Module}]
  tei
    \item[{Membres}]
  \hyperref[TEI.att.naming]{att.naming}[\hyperref[TEI.att.personal]{att.personal}[\hyperref[TEI.addName]{addName} \hyperref[TEI.forename]{forename} \hyperref[TEI.genName]{genName} \hyperref[TEI.name]{name} \hyperref[TEI.orgName]{orgName} \hyperref[TEI.persName]{persName} \hyperref[TEI.placeName]{placeName} \hyperref[TEI.roleName]{roleName} \hyperref[TEI.surname]{surname}] \hyperref[TEI.affiliation]{affiliation} \hyperref[TEI.author]{author} \hyperref[TEI.collection]{collection} \hyperref[TEI.country]{country} \hyperref[TEI.editor]{editor} \hyperref[TEI.event]{event} \hyperref[TEI.geogName]{geogName} \hyperref[TEI.institution]{institution} \hyperref[TEI.origPlace]{origPlace} \hyperref[TEI.pubPlace]{pubPlace} \hyperref[TEI.region]{region} \hyperref[TEI.repository]{repository} \hyperref[TEI.rs]{rs} \hyperref[TEI.settlement]{settlement} \hyperref[TEI.state]{state}] \hyperref[TEI.docAuthor]{docAuthor} \hyperref[TEI.docTitle]{docTitle} \hyperref[TEI.funder]{funder} \hyperref[TEI.material]{material} \hyperref[TEI.meeting]{meeting} \hyperref[TEI.objectType]{objectType} \hyperref[TEI.resp]{resp} \hyperref[TEI.respStmt]{respStmt} \hyperref[TEI.term]{term} \hyperref[TEI.title]{title}
    \item[{Attributs}]
  Attributs\hfil\\[-10pt]\begin{sansreflist}
    \item[@key]
  fournit un moyen, défini de façon externe, d'identifier l'entité (ou les entités) nommé(es), en utilisant une valeur codée d'un certain type.
\begin{reflist}
    \item[{Statut}]
  Optionel
    \item[{Type de données}]
  \hyperref[TEI.teidata.text]{teidata.text}
    \item[]\exampleFont {<\textbf{author}>}\mbox{}\newline 
\hspace*{6pt}{<\textbf{name}\hspace*{6pt}{key}="{name 427308}"\mbox{}\newline 
\hspace*{6pt}\hspace*{6pt}{type}="{organisation}">}[New Zealand Parliament, Legislative Council]{</\textbf{name}>}\mbox{}\newline 
{</\textbf{author}>}
    \item[]\exampleFont {<\textbf{author}>}\mbox{}\newline 
\hspace*{6pt}{<\textbf{name}\hspace*{6pt}{key}="{Hugo, Victor (1802-1885)}"\mbox{}\newline 
\hspace*{6pt}\hspace*{6pt}{ref}="{http://www.idref.fr/026927608}">}Victor Hugo{</\textbf{name}>}\mbox{}\newline 
{</\textbf{author}>}
    \item[{Note}]
  \par
La valeur peut être un identifiant unique dans une base de données, ou toute autre chaîne définie de façon externe identifiant le référent. 
\end{reflist}  
    \item[@ref]
  (référence) fournit un moyen explicite de localiser une définition complète de l'entité nommée au moyen d'un ou plusieurs URIs.
\begin{reflist}
    \item[{Statut}]
  Optionel
    \item[{Type de données}]
  1–∞ occurrences de \hyperref[TEI.teidata.pointer]{teidata.pointer} séparé par un espace
    \item[]\exampleFont {<\textbf{name}\hspace*{6pt}{ref}="{http://viaf.org/viaf/109557338}"\mbox{}\newline 
\hspace*{6pt}{type}="{person}">}Seamus Heaney{</\textbf{name}>}
    \item[{Note}]
  \par
La valeur doit pointer directement vers un ou plusieurs éléments XML au moyen d'un ou plusieurs URIs, séparés par un espace. Si plus d'un URI est fourni, cela implique que le nom identifie plusieurs entités distinctes.
\end{reflist}  
\end{sansreflist}  
\end{reflist}  
\begin{reflist}
\item[]\begin{specHead}{TEI.att.citing}{att.citing}\index{att.citing (attribute class)|oddindex}\index{unit=@unit!att.citing (attribute class)|oddindex}\index{from=@from!att.citing (attribute class)|oddindex}\index{to=@to!att.citing (attribute class)|oddindex} provides attributes for specifying the specific part of a bibliographic item being cited.\end{specHead} 
    \item[{Module}]
  tei
    \item[{Membres}]
  \hyperref[TEI.biblScope]{biblScope} \hyperref[TEI.citedRange]{citedRange}
    \item[{Attributs}]
  Attributs\hfil\\[-10pt]\begin{sansreflist}
    \item[@unit]
  identifie le type d'information que transmet l'élément, par exemple colonnes, pages, volume, inscription, etc.
\begin{reflist}
    \item[{Statut}]
  Optionel
    \item[{Type de données}]
  \hyperref[TEI.teidata.enumerated]{teidata.enumerated}
    \item[{Les valeurs suggérées comprennent:}]
  \begin{description}

\item[{volume}]l'élément contient un numéro de volume.
\item[{issue}]l'élément contient un numéro de livraison ou bien un numéro de volume et de livraison.
\item[{page}]l'élément contient un nombre de pages ou l'étendue de sélection des pages.
\item[{line}]the element contains a line number or line range.
\item[{chapter}](chapitre) l'élément contient une indication de chapitre (le numéro et/ou le titre)
\item[{part}]l'élément identifie une partie d'un livre ou une anthologie.
\item[{column}]the element identifies an entry number or label in a list of entries.
\item[{entry}]the element identifies a column.
\end{description} 
\end{reflist}  
    \item[@from]
  specifies the starting point of the range of units indicated by the {\itshape unit} attribute.
\begin{reflist}
    \item[{Statut}]
  Optionel
    \item[{Type de données}]
  \hyperref[TEI.teidata.word]{teidata.word}
\end{reflist}  
    \item[@to]
  specifies the end-point of the range of units indicated by the {\itshape unit} attribute.
\begin{reflist}
    \item[{Statut}]
  Optionel
    \item[{Type de données}]
  \hyperref[TEI.teidata.word]{teidata.word}
\end{reflist}  
\end{sansreflist}  
\end{reflist}  
\begin{reflist}
\item[]\begin{specHead}{TEI.att.coordinated}{att.coordinated}\index{att.coordinated (attribute class)|oddindex}\index{start=@start!att.coordinated (attribute class)|oddindex}\index{ulx=@ulx!att.coordinated (attribute class)|oddindex}\index{uly=@uly!att.coordinated (attribute class)|oddindex}\index{lrx=@lrx!att.coordinated (attribute class)|oddindex}\index{lry=@lry!att.coordinated (attribute class)|oddindex}\index{points=@points!att.coordinated (attribute class)|oddindex} attributs utilisables pour les éléments pouvant être positionnés dans un système de coordonnées à deux dimensions.\end{specHead} 
    \item[{Module}]
  transcr
    \item[{Membres}]
  \hyperref[TEI.line]{line} \hyperref[TEI.surface]{surface} \hyperref[TEI.zone]{zone}
    \item[{Attributs}]
  Attributs\hfil\\[-10pt]\begin{sansreflist}
    \item[@start]
  désigne l'élément qui, dans la transcription du texte, contient au moins le début de la section de texte représentée dans la zone ou surface.
\begin{reflist}
    \item[{Statut}]
  Optionel
    \item[{Type de données}]
  \hyperref[TEI.teidata.pointer]{teidata.pointer}
\end{reflist}  
    \item[@ulx]
  donne la valeur x de l'abscisse du coin supérieur gauche d'un rectangle.
\begin{reflist}
    \item[{Statut}]
  Optionel
    \item[{Type de données}]
  \hyperref[TEI.teidata.numeric]{teidata.numeric}
\end{reflist}  
    \item[@uly]
  donne la valeur y de l'ordonnée du coin supérieur gauche d'un rectangle.
\begin{reflist}
    \item[{Statut}]
  Optionel
    \item[{Type de données}]
  \hyperref[TEI.teidata.numeric]{teidata.numeric}
\end{reflist}  
    \item[@lrx]
  donne la valeur x de l'abscisse du coin inférieur droit d'un rectangle.
\begin{reflist}
    \item[{Statut}]
  Optionel
    \item[{Type de données}]
  \hyperref[TEI.teidata.numeric]{teidata.numeric}
\end{reflist}  
    \item[@lry]
  donne la valeur y de l'ordonnée du coin inférieur droit d'un rectangle.
\begin{reflist}
    \item[{Statut}]
  Optionel
    \item[{Type de données}]
  \hyperref[TEI.teidata.numeric]{teidata.numeric}
\end{reflist}  
    \item[@points]
  identifies a two dimensional area within the bounding box specified by the other attributes by means of a series of pairs of numbers, each of which gives the x,y coordinates of a point on a line enclosing the area.
\begin{reflist}
    \item[{Statut}]
  Optionel
    \item[{Type de données}]
  3–∞ occurrences de \hyperref[TEI.teidata.point]{teidata.point} séparé par un espace
\end{reflist}  
\end{sansreflist}  
\end{reflist}  
\begin{reflist}
\item[]\begin{specHead}{TEI.att.damaged}{att.damaged}\index{att.damaged (attribute class)|oddindex}\index{agent=@agent!att.damaged (attribute class)|oddindex}\index{degree=@degree!att.damaged (attribute class)|oddindex}\index{group=@group!att.damaged (attribute class)|oddindex} fournit des attributs décrivant la nature de tout dommage physique affectant la lecture.\end{specHead} 
    \item[{Module}]
  tei
    \item[{Membres}]
  \hyperref[TEI.damage]{damage} \hyperref[TEI.damageSpan]{damageSpan}
    \item[{Attributs}]
  Attributs \hyperref[TEI.att.dimensions]{att.dimensions} (\textit{@unit}, \textit{@quantity}, \textit{@extent}, \textit{@precision}, \textit{@scope})  (\hyperref[TEI.att.ranging]{att.ranging} (\textit{@atLeast}, \textit{@atMost}, \textit{@min}, \textit{@max}, \textit{@confidence})) \hyperref[TEI.att.written]{att.written} (\textit{@hand}) \hfil\\[-10pt]\begin{sansreflist}
    \item[@agent]
  caractérise la raison des dommages, lorsqu'elle peut être identifiée
\begin{reflist}
    \item[{Statut}]
  Optionel
    \item[{Type de données}]
  \hyperref[TEI.teidata.enumerated]{teidata.enumerated}
    \item[{Exemple de valeurs possibles:}]
  \begin{description}

\item[{rubbing}]le dommage résulte d'un frottement sur les bords du feuillet
\item[{mildew}]le dommage résulte de moisissure sur la surface du feuillet
\item[{smoke}]le dommage résulte de la fumée
\end{description} 
\end{reflist}  
    \item[@degree]
  indique le degré (la gravité) du dommage subi, selon une grille appropriée. L'attribut {\itshape degree} doit être utilisé dans le seul cas où le texte peut être lu avec certitude ; le texte restitué en utilisant d'autres sources doit être encodé au moyen de l'élément \hyperref[TEI.supplied]{<supplied>}.
\begin{reflist}
    \item[{Statut}]
  Optionel
    \item[{Type de données}]
  \hyperref[TEI.teidata.probCert]{teidata.probCert}
    \item[{Note}]
  \par
La balise \hyperref[TEI.damage]{<damage>} avec l'attribut {\itshape degree}ne doit être utilisée qu'à l'endroit où le texte peut être lu avec certitude malgré le dommage. Elle convient lorsqu'on désire faire état du dommage bien que cela n'affecte en rien la lisibilité du texte (comme cela peut être le cas avec des vestiges de matériaux gravés). Là où les dommages ont rendu le texte plus ou moins illisible, les balises \hyperref[TEI.unclear]{<unclear>} (pour l'illisibilité partielle) ou \hyperref[TEI.gap]{<gap>} (pour l'illisibilité complète, sans restitution de texte) sont à employer, l'information relative aux dommages étant donnée par les valeurs d'attributs de ces balises. Voir \xref{http://www.tei-c.org/release/doc/tei-p5-doc/en/html/PH.html\#PHCOMB}{11.3.3.2. Use of the gap, del, damage, unclear, and supplied Elements in Combination} au sujet de l'utilisation de ces balises dans des cas particuliers.
\end{reflist}  
    \item[@group]
  permet d'assigner un numéro quelconque à chaque fragment endommagé considéré comme faisant partie d'un ensemble résultant du même phénomène physique
\begin{reflist}
    \item[{Statut}]
  Optionel
    \item[{Type de données}]
  \hyperref[TEI.teidata.count]{teidata.count}
\end{reflist}  
\end{sansreflist}  
\end{reflist}  
\begin{reflist}
\item[]\begin{specHead}{TEI.att.datable}{att.datable}\index{att.datable (attribute class)|oddindex}\index{calendar=@calendar!att.datable (attribute class)|oddindex}\index{period=@period!att.datable (attribute class)|oddindex} fournit des attributs pour la normalisation d'éléments qui contiennent des mentions d'événements datés ou susceptibles de l'être\end{specHead} 
    \item[{Module}]
  tei
    \item[{Membres}]
  \hyperref[TEI.acquisition]{acquisition} \hyperref[TEI.affiliation]{affiliation} \hyperref[TEI.application]{application} \hyperref[TEI.binding]{binding} \hyperref[TEI.change]{change} \hyperref[TEI.country]{country} \hyperref[TEI.creation]{creation} \hyperref[TEI.custEvent]{custEvent} \hyperref[TEI.date]{date} \hyperref[TEI.event]{event} \hyperref[TEI.geogName]{geogName} \hyperref[TEI.idno]{idno} \hyperref[TEI.licence]{licence} \hyperref[TEI.location]{location} \hyperref[TEI.name]{name} \hyperref[TEI.orgName]{orgName} \hyperref[TEI.origDate]{origDate} \hyperref[TEI.origPlace]{origPlace} \hyperref[TEI.origin]{origin} \hyperref[TEI.persName]{persName} \hyperref[TEI.placeName]{placeName} \hyperref[TEI.provenance]{provenance} \hyperref[TEI.region]{region} \hyperref[TEI.resp]{resp} \hyperref[TEI.seal]{seal} \hyperref[TEI.settlement]{settlement} \hyperref[TEI.stamp]{stamp} \hyperref[TEI.state]{state} \hyperref[TEI.time]{time} \hyperref[TEI.title]{title}
    \item[{Attributs}]
  Attributs \hyperref[TEI.att.datable.w3c]{att.datable.w3c} (\textit{@when}, \textit{@notBefore}, \textit{@notAfter}, \textit{@from}, \textit{@to}) \hyperref[TEI.att.datable.iso]{att.datable.iso} (\textit{@when-iso}, \textit{@notBefore-iso}, \textit{@notAfter-iso}, \textit{@from-iso}, \textit{@to-iso}) \hyperref[TEI.att.datable.custom]{att.datable.custom} (\textit{@when-custom}, \textit{@notBefore-custom}, \textit{@notAfter-custom}, \textit{@from-custom}, \textit{@to-custom}, \textit{@datingPoint}, \textit{@datingMethod}) \hfil\\[-10pt]\begin{sansreflist}
    \item[@calendar]
  indique le système ou le calendrier auquel appartient la date exprimée dans le contenu de l'élément.
\begin{reflist}
    \item[{Statut}]
  Optionel
    \item[{Type de données}]
  \hyperref[TEI.teidata.pointer]{teidata.pointer}
    \item[{Schematron}]
   <sch:rule context="tei:*[@calendar]"> <sch:assert test="string-length(.) gt 0">@calendar indicates the system or calendar to which the date represented by the content of this element  belongs, but this <sch:name/> element has no textual content.</sch:assert> </sch:rule>
    \item[]\exampleFont  L'année\mbox{}\newline 
 1960 fut, en vertu du calendrier grégorien, bissextile ; le 22 juin tomba ainsi le jour\mbox{}\newline 
 de l'été, le {<\textbf{date}\hspace*{6pt}{calendar}="{\#gregorian}">}22 juin{</\textbf{date}>}.\mbox{}\newline 

\end{reflist}  
    \item[@period]
  fournit un pointeur vers un emplacement donné définissant une période de temps nommée durant laquelle l'item concerné s'inscrit.
\begin{reflist}
    \item[{Statut}]
  Optionel
    \item[{Type de données}]
  \hyperref[TEI.teidata.pointer]{teidata.pointer}
\end{reflist}  
\end{sansreflist}  
    \item[{Note}]
  \par
Cette‘superclasse’ fournit des attributs qui peuvent être employés pour donner des valeurs normalisées à des informations relatives au temps. Par défaut, les attributs de la classe \textsf{att.datable.w3c} sont fournis. Si le module pour les noms et les dates est chargé, cette classe fournit également des attributs de la classe \textsf{att.datable.iso}. En général, les valeurs possibles des attributs, limitées aux types de données W3C, forment un sous-ensemble des valeurs que l'on trouve dans la norme ISO 8601. Cependant, il n'est peut-être pas nécessaire de recourir aux possibilités très étendues des types de données de l'ISO. Il existe en effet une bien plus grande offre logicielle pour le traitement des types de données W3C.
\end{reflist}  
\begin{reflist}
\item[]\begin{specHead}{TEI.att.datable.custom}{att.datable.custom}\index{att.datable.custom (attribute class)|oddindex}\index{when-custom=@when-custom!att.datable.custom (attribute class)|oddindex}\index{notBefore-custom=@notBefore-custom!att.datable.custom (attribute class)|oddindex}\index{notAfter-custom=@notAfter-custom!att.datable.custom (attribute class)|oddindex}\index{from-custom=@from-custom!att.datable.custom (attribute class)|oddindex}\index{to-custom=@to-custom!att.datable.custom (attribute class)|oddindex}\index{datingPoint=@datingPoint!att.datable.custom (attribute class)|oddindex}\index{datingMethod=@datingMethod!att.datable.custom (attribute class)|oddindex} provides attributes for normalization of elements that contain datable events to a custom dating system (i.e. other than the Gregorian used by W3 and ISO).\end{specHead} 
    \item[{Module}]
  namesdates
    \item[{Membres}]
  \hyperref[TEI.att.datable]{att.datable}[\hyperref[TEI.acquisition]{acquisition} \hyperref[TEI.affiliation]{affiliation} \hyperref[TEI.application]{application} \hyperref[TEI.binding]{binding} \hyperref[TEI.change]{change} \hyperref[TEI.country]{country} \hyperref[TEI.creation]{creation} \hyperref[TEI.custEvent]{custEvent} \hyperref[TEI.date]{date} \hyperref[TEI.event]{event} \hyperref[TEI.geogName]{geogName} \hyperref[TEI.idno]{idno} \hyperref[TEI.licence]{licence} \hyperref[TEI.location]{location} \hyperref[TEI.name]{name} \hyperref[TEI.orgName]{orgName} \hyperref[TEI.origDate]{origDate} \hyperref[TEI.origPlace]{origPlace} \hyperref[TEI.origin]{origin} \hyperref[TEI.persName]{persName} \hyperref[TEI.placeName]{placeName} \hyperref[TEI.provenance]{provenance} \hyperref[TEI.region]{region} \hyperref[TEI.resp]{resp} \hyperref[TEI.seal]{seal} \hyperref[TEI.settlement]{settlement} \hyperref[TEI.stamp]{stamp} \hyperref[TEI.state]{state} \hyperref[TEI.time]{time} \hyperref[TEI.title]{title}]
    \item[{Attributs}]
  Attributs\hfil\\[-10pt]\begin{sansreflist}
    \item[@when-custom]
  supplies the value of a date or time in some custom standard form.
\begin{reflist}
    \item[{Statut}]
  Optionel
    \item[{Type de données}]
  1–∞ occurrences de \hyperref[TEI.teidata.word]{teidata.word} séparé par un espace
    \item[]The following are examples of custom date or time formats that are \textit{not} valid ISO or W3C format normalizations, normalized to a different dating system\exampleFont {<\textbf{p}>}Alhazen died in Cairo on the\mbox{}\newline 
{<\textbf{date}\hspace*{6pt}{when}="{1040-03-06}"\mbox{}\newline 
\hspace*{6pt}\hspace*{6pt}{when-custom}="{431-06-12}">} 12th day of Jumada t-Tania, 430 AH\mbox{}\newline 
\hspace*{6pt}{</\textbf{date}>}.{</\textbf{p}>}\mbox{}\newline 
{<\textbf{p}>}The current world will end at the\mbox{}\newline 
{<\textbf{date}\hspace*{6pt}{when}="{2012-12-21}"\mbox{}\newline 
\hspace*{6pt}\hspace*{6pt}{when-custom}="{13.0.0.0.0}">}end of B'ak'tun 13{</\textbf{date}>}.{</\textbf{p}>}\mbox{}\newline 
{<\textbf{p}>}The Battle of Meggidu\mbox{}\newline 
 ({<\textbf{date}\hspace*{6pt}{when-custom}="{Thutmose\textunderscore III:23}">}23rd year of reign of Thutmose III{</\textbf{date}>}).{</\textbf{p}>}\mbox{}\newline 
{<\textbf{p}>}Esidorus bixit in pace annos LXX plus minus sub\mbox{}\newline 
{<\textbf{date}\hspace*{6pt}{when-custom}="{Ind:4-10-11}">}die XI mensis Octobris indictione IIII{</\textbf{date}>}\mbox{}\newline 
{</\textbf{p}>}Not all custom date formulations will have Gregorian equivalents.The {\itshape when-custom} attribute and other custom dating are not contrained to a datatype by the TEI, but individual projects are recommended to regularize and document their dating formats.
\end{reflist}  
    \item[@notBefore-custom]
  specifies the earliest possible date for the event in some custom standard form.
\begin{reflist}
    \item[{Statut}]
  Optionel
    \item[{Type de données}]
  1–∞ occurrences de \hyperref[TEI.teidata.word]{teidata.word} séparé par un espace
\end{reflist}  
    \item[@notAfter-custom]
  specifies the latest possible date for the event in some custom standard form.
\begin{reflist}
    \item[{Statut}]
  Optionel
    \item[{Type de données}]
  1–∞ occurrences de \hyperref[TEI.teidata.word]{teidata.word} séparé par un espace
\end{reflist}  
    \item[@from-custom]
  indicates the starting point of the period in some custom standard form.
\begin{reflist}
    \item[{Statut}]
  Optionel
    \item[{Type de données}]
  1–∞ occurrences de \hyperref[TEI.teidata.word]{teidata.word} séparé par un espace
    \item[]\exampleFont {<\textbf{event}\hspace*{6pt}{datingMethod}="{\#julian}"\mbox{}\newline 
\hspace*{6pt}{from-custom}="{1666-09-02}"\mbox{}\newline 
\hspace*{6pt}{to-custom}="{1666-09-05}"\mbox{}\newline 
\hspace*{6pt}{xml:id}="{FIRE1}">}\mbox{}\newline 
\hspace*{6pt}{<\textbf{head}>}The Great Fire of London{</\textbf{head}>}\mbox{}\newline 
\hspace*{6pt}{<\textbf{p}>}The Great Fire of London burned through a large part\mbox{}\newline 
\hspace*{6pt}\hspace*{6pt} of the city of London.{</\textbf{p}>}\mbox{}\newline 
{</\textbf{event}>}
\end{reflist}  
    \item[@to-custom]
  indicates the ending point of the period in some custom standard form.
\begin{reflist}
    \item[{Statut}]
  Optionel
    \item[{Type de données}]
  1–∞ occurrences de \hyperref[TEI.teidata.word]{teidata.word} séparé par un espace
\end{reflist}  
    \item[@datingPoint]
  supplies a pointer to some location defining a named point in time with reference to which the datable item is understood to have occurred
\begin{reflist}
    \item[{Statut}]
  Optionel
    \item[{Type de données}]
  \hyperref[TEI.teidata.pointer]{teidata.pointer}
\end{reflist}  
    \item[@datingMethod]
  supplies a pointer to a \texttt{<calendar>} element or other means of interpreting the values of the custom dating attributes.
\begin{reflist}
    \item[{Statut}]
  Optionel
    \item[{Type de données}]
  \hyperref[TEI.teidata.pointer]{teidata.pointer}
    \item[]\exampleFont Contayning the Originall, Antiquity, Increaſe, Moderne\mbox{}\newline 
 eſtate, and deſcription of that Citie, written in the yeare\mbox{}\newline 
{<\textbf{date}\hspace*{6pt}{calendar}="{\#julian}"\mbox{}\newline 
\hspace*{6pt}{datingMethod}="{\#julian}"\mbox{}\newline 
\hspace*{6pt}{when-custom}="{1598}">}1598{</\textbf{date}>}. by Iohn Stow\mbox{}\newline 
 Citizen of London.In this example, the {\itshape calendar} attribute points to a \texttt{<calendar>} element for the Julian calendar, specifying that the text content of the \hyperref[TEI.date]{<date>} element is a Julian date, and the {\itshape datingMethod} attribute also points to the Julian calendar to indicate that the content of the {\itshape when-custom} attribute value is Julian too.
    \item[]\exampleFont {<\textbf{date}\hspace*{6pt}{datingMethod}="{\#creationOfWorld}"\mbox{}\newline 
\hspace*{6pt}{when}="{1382-06-28}"\mbox{}\newline 
\hspace*{6pt}{when-custom}="{6890-06-20}">} μηνὶ Ἰουνίου εἰς {<\textbf{num}>}κ{</\textbf{num}>} ἔτους {<\textbf{num}>}ςωϞ{</\textbf{num}>}\mbox{}\newline 
{</\textbf{date}>}In this example, a date is given in a Mediaeval text measured "from the creation of the world", which is normalised (in {\itshape when}) to the Gregorian date, but is also normalized (in {\itshape when-custom}) to a machine-actionable, numeric version of the date from the Creation.
\end{reflist}  
\end{sansreflist}  
\end{reflist}  
\begin{reflist}
\item[]\begin{specHead}{TEI.att.datable.iso}{att.datable.iso}\index{att.datable.iso (attribute class)|oddindex}\index{when-iso=@when-iso!att.datable.iso (attribute class)|oddindex}\index{notBefore-iso=@notBefore-iso!att.datable.iso (attribute class)|oddindex}\index{notAfter-iso=@notAfter-iso!att.datable.iso (attribute class)|oddindex}\index{from-iso=@from-iso!att.datable.iso (attribute class)|oddindex}\index{to-iso=@to-iso!att.datable.iso (attribute class)|oddindex} fournit des attributs pour la normalisation, selon la norme ISO 8601, d'éléments contenant des évènements datables.\end{specHead} 
    \item[{Module}]
  namesdates
    \item[{Membres}]
  \hyperref[TEI.att.datable]{att.datable}[\hyperref[TEI.acquisition]{acquisition} \hyperref[TEI.affiliation]{affiliation} \hyperref[TEI.application]{application} \hyperref[TEI.binding]{binding} \hyperref[TEI.change]{change} \hyperref[TEI.country]{country} \hyperref[TEI.creation]{creation} \hyperref[TEI.custEvent]{custEvent} \hyperref[TEI.date]{date} \hyperref[TEI.event]{event} \hyperref[TEI.geogName]{geogName} \hyperref[TEI.idno]{idno} \hyperref[TEI.licence]{licence} \hyperref[TEI.location]{location} \hyperref[TEI.name]{name} \hyperref[TEI.orgName]{orgName} \hyperref[TEI.origDate]{origDate} \hyperref[TEI.origPlace]{origPlace} \hyperref[TEI.origin]{origin} \hyperref[TEI.persName]{persName} \hyperref[TEI.placeName]{placeName} \hyperref[TEI.provenance]{provenance} \hyperref[TEI.region]{region} \hyperref[TEI.resp]{resp} \hyperref[TEI.seal]{seal} \hyperref[TEI.settlement]{settlement} \hyperref[TEI.stamp]{stamp} \hyperref[TEI.state]{state} \hyperref[TEI.time]{time} \hyperref[TEI.title]{title}]
    \item[{Attributs}]
  Attributs\hfil\\[-10pt]\begin{sansreflist}
    \item[@when-iso]
  précise une date exacte pour l'évènement selon la forme normalisée ISO 8601, c'est-à-dire aaaa-mm-jj.
\begin{reflist}
    \item[{Statut}]
  Optionel
    \item[{Type de données}]
  \hyperref[TEI.teidata.temporal.iso]{teidata.temporal.iso}
    \item[]Les exemples qui suivent sont des mentions de date, d'heure, de temps au format ISO qui ne sont \textit{pas }normalisées au format W3C\exampleFont {<\textbf{date}\hspace*{6pt}{when-iso}="{1996-09-24T07:25+00}">}le 24 sept. 1996, à 3 h 25 du matin.{</\textbf{date}>}\mbox{}\newline 
{<\textbf{date}\hspace*{6pt}{when-iso}="{1996-09-24T03:25-04}">}le 24 sept. 1996, à 3 h 25 du matin.{</\textbf{date}>}\mbox{}\newline 
{<\textbf{time}\hspace*{6pt}{when-iso}="{1999-01-04T20:42-05}">}le 4 janvier 1999 à 8 h.42 du soir {</\textbf{time}>}\mbox{}\newline 
{<\textbf{time}\hspace*{6pt}{when-iso}="{1999-W01-1T20,70-05}">}le 4 janvier 1999 à 8 h.42 du soir {</\textbf{time}>}\mbox{}\newline 
{<\textbf{date}\hspace*{6pt}{when-iso}="{2006-05-18T10:03}">}quelques minutes après 10 heures du matin, le mardi 18\mbox{}\newline 
 mai.{</\textbf{date}>}\mbox{}\newline 
{<\textbf{time}\hspace*{6pt}{when-iso}="{03:00}">}3 h. du matin.{</\textbf{time}>}\mbox{}\newline 
{<\textbf{time}\hspace*{6pt}{when-iso}="{14}">}aux alentours de deux heures.{</\textbf{time}>}\mbox{}\newline 
{<\textbf{time}\hspace*{6pt}{when-iso}="{15,5}">}trois heures et demi.{</\textbf{time}>}Tous les exemples de l’attribut {\itshape when} dans la classe \textsf{att.datable.w3c} sont également valides en ce qui concerne cet attribut
    \item[]\exampleFont Il aime à être ponctuel . J'ai dit{<\textbf{q}>}\mbox{}\newline 
\hspace*{6pt}{<\textbf{time}\hspace*{6pt}{when-iso}="{12}">}autour du midi{</\textbf{time}>}\mbox{}\newline 
{</\textbf{q}>}, et il est apparu à {<\textbf{time}\hspace*{6pt}{when-iso}="{12:00:00}">}midi {</\textbf{time}>}à l'heure pile.La deuxième occurence de \hyperref[TEI.time]{<time>} a pu être encodée avec l'attribut {\itshape when} , puisque12:00:00 est une marque de temps en accord avec la spécification du W3C \textit{XML Schema Part 2: Datatypes Second Edition}. La première occurence ne l'est pas.
\end{reflist}  
    \item[@notBefore-iso]
  précise la première date possible pour l'évènement selon la forme normalisée, c'est-à-dire aaaa-mm-jj.
\begin{reflist}
    \item[{Statut}]
  Optionel
    \item[{Type de données}]
  \hyperref[TEI.teidata.temporal.iso]{teidata.temporal.iso}
\end{reflist}  
    \item[@notAfter-iso]
  précise la dernière date possible pour l'évènement selon la forme normalisée, c'est-à-dire aaaa-mm-jj.
\begin{reflist}
    \item[{Statut}]
  Optionel
    \item[{Type de données}]
  \hyperref[TEI.teidata.temporal.iso]{teidata.temporal.iso}
\end{reflist}  
    \item[@from-iso]
  Indique le point de départ de la période sous une forme normalisée
\begin{reflist}
    \item[{Statut}]
  Optionel
    \item[{Type de données}]
  \hyperref[TEI.teidata.temporal.iso]{teidata.temporal.iso}
\end{reflist}  
    \item[@to-iso]
  Indique le point final de la période sous une forme normalisée
\begin{reflist}
    \item[{Statut}]
  Optionel
    \item[{Type de données}]
  \hyperref[TEI.teidata.temporal.iso]{teidata.temporal.iso}
\end{reflist}  
\end{sansreflist}  
    \item[{Note}]
  \par
La valeur de {\itshape when-iso} doit être une représentation normalisée de la date, de la durée ou d'une combinaison de date et de durée, dans l'un des formats spécifiés dans ISO 8601, selon le calendrier grégorien.
    \item[{Note}]
  \par
Si les attributs {\itshape when-iso} et {\itshape dur-iso} sont tous les deux spécifiés, les valeurs doivent être interprétées comme indiquant un intervalle de temps au moyen de son point de départ (ou date) et de sa durée. C'est à dire, \par\bgroup\exampleFont \begin{shaded}\noindent\mbox{}{<\textbf{date}\hspace*{6pt}{dur-iso}="{P8D}"\hspace*{6pt}{when-iso}="{2007-06-01}"/>}\end{shaded}\egroup\par \noindent  indique la même période temporelle que \par\bgroup\exampleFont \begin{shaded}\noindent\mbox{}{<\textbf{date}\hspace*{6pt}{when-iso}="{2007-06-01/P8D}"/>}\end{shaded}\egroup\par \par
En fournissant une forme dite "régularisée", il n'est rien affirmé sur la correction ou l'incorrection de la forme dans le texte source ; la forme régularisée est simplement celle qui est choisie comme forme principale afin de réunir les variantes de forme sous une seule rubrique.
\end{reflist}  
\begin{reflist}
\item[]\begin{specHead}{TEI.att.datable.w3c}{att.datable.w3c}\index{att.datable.w3c (attribute class)|oddindex}\index{when=@when!att.datable.w3c (attribute class)|oddindex}\index{notBefore=@notBefore!att.datable.w3c (attribute class)|oddindex}\index{notAfter=@notAfter!att.datable.w3c (attribute class)|oddindex}\index{from=@from!att.datable.w3c (attribute class)|oddindex}\index{to=@to!att.datable.w3c (attribute class)|oddindex} fournit des attributs pour la normalisation d'éléments qui contiennent des mentions d'événements datés ou susceptibles de l'être\end{specHead} 
    \item[{Module}]
  tei
    \item[{Membres}]
  \hyperref[TEI.att.datable]{att.datable}[\hyperref[TEI.acquisition]{acquisition} \hyperref[TEI.affiliation]{affiliation} \hyperref[TEI.application]{application} \hyperref[TEI.binding]{binding} \hyperref[TEI.change]{change} \hyperref[TEI.country]{country} \hyperref[TEI.creation]{creation} \hyperref[TEI.custEvent]{custEvent} \hyperref[TEI.date]{date} \hyperref[TEI.event]{event} \hyperref[TEI.geogName]{geogName} \hyperref[TEI.idno]{idno} \hyperref[TEI.licence]{licence} \hyperref[TEI.location]{location} \hyperref[TEI.name]{name} \hyperref[TEI.orgName]{orgName} \hyperref[TEI.origDate]{origDate} \hyperref[TEI.origPlace]{origPlace} \hyperref[TEI.origin]{origin} \hyperref[TEI.persName]{persName} \hyperref[TEI.placeName]{placeName} \hyperref[TEI.provenance]{provenance} \hyperref[TEI.region]{region} \hyperref[TEI.resp]{resp} \hyperref[TEI.seal]{seal} \hyperref[TEI.settlement]{settlement} \hyperref[TEI.stamp]{stamp} \hyperref[TEI.state]{state} \hyperref[TEI.time]{time} \hyperref[TEI.title]{title}] \hyperref[TEI.standOff]{standOff}
    \item[{Attributs}]
  Attributs\hfil\\[-10pt]\begin{sansreflist}
    \item[@when]
  spécifie une date exacte pour un événement sous une forme normalisée, par ex. aaaa-mm-jj.
\begin{reflist}
    \item[{Statut}]
  Optionel
    \item[{Type de données}]
  \hyperref[TEI.teidata.temporal.w3c]{teidata.temporal.w3c}
    \item[]\exampleFont {<\textbf{p}>}\mbox{}\newline 
\hspace*{6pt}{<\textbf{date}\hspace*{6pt}{when}="{1945-10-24}">}24 Oct 45{</\textbf{date}>}\mbox{}\newline 
\hspace*{6pt}{<\textbf{date}\hspace*{6pt}{when}="{1996-09-24T07:25:00Z}">}24 septembre 1996 à 3h 25 du matin{</\textbf{date}>}\mbox{}\newline 
\hspace*{6pt}{<\textbf{time}\hspace*{6pt}{when}="{1999-01-04T20:42:00-05:00}">}4 janvier 1999 à 8h de l'après-midi.{</\textbf{time}>}\mbox{}\newline 
\hspace*{6pt}{<\textbf{time}\hspace*{6pt}{when}="{14:12:38}">}14 h 12 minutes et 38 secondes{</\textbf{time}>}\mbox{}\newline 
\hspace*{6pt}{<\textbf{date}\hspace*{6pt}{when}="{1962-10}">}octobre 1962{</\textbf{date}>}\mbox{}\newline 
\hspace*{6pt}{<\textbf{date}\hspace*{6pt}{when}="{--06-12}">}12 juin{</\textbf{date}>}\mbox{}\newline 
\hspace*{6pt}{<\textbf{date}\hspace*{6pt}{when}="{---01}">}premier du mois{</\textbf{date}>}\mbox{}\newline 
\hspace*{6pt}{<\textbf{date}\hspace*{6pt}{when}="{--08}">}Août{</\textbf{date}>}\mbox{}\newline 
\hspace*{6pt}{<\textbf{date}\hspace*{6pt}{when}="{2006}">}MMVI{</\textbf{date}>}\mbox{}\newline 
\hspace*{6pt}{<\textbf{date}\hspace*{6pt}{when}="{0056}">}56 ap. J.-C.{</\textbf{date}>}\mbox{}\newline 
\hspace*{6pt}{<\textbf{date}\hspace*{6pt}{when}="{-0056}">}56 av. J.-C.{</\textbf{date}>}\mbox{}\newline 
{</\textbf{p}>}
    \item[]\exampleFont Shakespeare meurt dix jours plus tard, à Stratford-on-Avon,\mbox{}\newline 
 Warwickshire, dans l'Angleterre protestante et dans le\mbox{}\newline 
 calendrier julien, le\mbox{}\newline 
{<\textbf{date}\hspace*{6pt}{calendar}="{\#julian}"\mbox{}\newline 
\hspace*{6pt}{when}="{--05-03}">}mardi 23 avril ancien style{</\textbf{date}>},\mbox{}\newline 
 c'est-à-dire le\mbox{}\newline 
{<\textbf{date}\hspace*{6pt}{calendar}="{\#gregorian}"\mbox{}\newline 
\hspace*{6pt}{when}="{--05-03}">}3 mai{</\textbf{date}>} dans\mbox{}\newline 
 le calendrier grégorien.
\end{reflist}  
    \item[@notBefore]
  spécifie la date la plus ancienne pour l'événement sous une forme normalisée, par ex. aaaa-mm-jj
\begin{reflist}
    \item[{Statut}]
  Optionel
    \item[{Type de données}]
  \hyperref[TEI.teidata.temporal.w3c]{teidata.temporal.w3c}
\end{reflist}  
    \item[@notAfter]
  spécifie la date la plus récente possible pour l'événement sous une forme normalisée, par ex. aaaa-mm-jj
\begin{reflist}
    \item[{Statut}]
  Optionel
    \item[{Type de données}]
  \hyperref[TEI.teidata.temporal.w3c]{teidata.temporal.w3c}
\end{reflist}  
    \item[@from]
  indique le point de départ d'une période sous une forme normalisée, par ex. aaaa-mm-jj
\begin{reflist}
    \item[{Statut}]
  Optionel
    \item[{Type de données}]
  \hyperref[TEI.teidata.temporal.w3c]{teidata.temporal.w3c}
\end{reflist}  
    \item[@to]
  indique le terme de la période sous une forme normalisée, par ex. aaaa-mm-jj
\begin{reflist}
    \item[{Statut}]
  Optionel
    \item[{Type de données}]
  \hyperref[TEI.teidata.temporal.w3c]{teidata.temporal.w3c}
\end{reflist}  
\end{sansreflist}  
    \item[{Schematron}]
   <sch:rule context="tei:*[@when]"> <sch:report role="nonfatal"  test="@notBefore|@notAfter|@from|@to">The @when attribute cannot be used with any other att.datable.w3c attributes.</sch:report> </sch:rule>
    \item[{Schematron}]
   <sch:rule context="tei:*[@from]"> <sch:report role="nonfatal"  test="@notBefore">The @from and @notBefore attributes cannot be used together.</sch:report> </sch:rule>
    \item[{Schematron}]
   <sch:rule context="tei:*[@to]"> <sch:report role="nonfatal"  test="@notAfter">The @to and @notAfter attributes cannot be used together.</sch:report> </sch:rule>
    \item[{Exemple}]
  \leavevmode\bgroup\exampleFont \begin{shaded}\noindent\mbox{}{<\textbf{date}\hspace*{6pt}{from}="{1863-05-28}"\hspace*{6pt}{to}="{1863-06-01}">}28 May through 1 June 1863{</\textbf{date}>}\end{shaded}\egroup 


    \item[{Note}]
  \par
La valeur de l'attribut {\itshape when} doit être une représentation normalisée de la date ou de l'heure, ou des deux, dans l'un des formats spécifiés par le \textit{XML Schema Part 2: Datatypes Second Edition}, selon le calendrier grégorien.\par
Pour la date, le format le plus courant est \texttt{yyyy-mm-dd}, mais on trouve aussi \texttt{yyyy}, \texttt{--mm}, \texttt{---dd}, \texttt{yyyy-mm}, ou \texttt{--mm-dd}. Pour l'heure, on utilise le format \texttt{hh:mm:ss}.\par
Il faut noter qu'actuellement ce format ne permet pas d'utiliser la valeur 0000 pour représenter l'année précédant le début de l'ère chrétienne ; on doit utiliser la valeur -0001.
\end{reflist}  
\begin{reflist}
\item[]\begin{specHead}{TEI.att.datcat}{att.datcat}\index{att.datcat (attribute class)|oddindex}\index{datcat=@datcat!att.datcat (attribute class)|oddindex}\index{valueDatcat=@valueDatcat!att.datcat (attribute class)|oddindex} provides the {\itshape dcr:datacat} and {\itshape dcr:ValueDatacat} attributes which are used to align XML elements or attributes with the appropriate Data Categories (DCs) defined by the ISO 12620:2009 standard and stored in the Web repository called ISOCat at \xref{http://www.isocat.org/}{http://www.isocat.org/}.\end{specHead} 
    \item[{Module}]
  tei
    \item[{Membres}]
  \hyperref[TEI.att.segLike]{att.segLike}[\hyperref[TEI.c]{c} \hyperref[TEI.cl]{cl} \hyperref[TEI.m]{m} \hyperref[TEI.pc]{pc} \hyperref[TEI.phr]{phr} \hyperref[TEI.s]{s} \hyperref[TEI.seg]{seg} \hyperref[TEI.w]{w}] \hyperref[TEI.binary]{binary} \hyperref[TEI.f]{f} \hyperref[TEI.fs]{fs} \hyperref[TEI.numeric]{numeric} \hyperref[TEI.string]{string} \hyperref[TEI.symbol]{symbol}
    \item[{Attributs}]
  Attributs\hfil\\[-10pt]\begin{sansreflist}
    \item[@datcat]
  contains a PID (persistent identifier) that aligns the given element with the appropriate Data Category (or categories) in ISOcat.
\begin{reflist}
    \item[{Statut}]
  Optionel
    \item[{Type de données}]
  1–∞ occurrences de \hyperref[TEI.teidata.pointer]{teidata.pointer} séparé par un espace
\end{reflist}  
    \item[@valueDatcat]
  contains a PID (persistent identifier) that aligns the content of the given element or the value of the given attribute with the appropriate simple Data Category (or categories) in ISOcat.
\begin{reflist}
    \item[{Statut}]
  Optionel
    \item[{Type de données}]
  1–∞ occurrences de \hyperref[TEI.teidata.pointer]{teidata.pointer} séparé par un espace
\end{reflist}  
\end{sansreflist}  
    \item[{Exemple}]
  In this example {\itshape dcr:datcat} relates the feature name to the data category "partOfSpeech" and {\itshape dcr:valueDatcat} the feature value to the data category "commonNoun". Both these data categories reside in the ISOcat DCR at \xref{http://www.isocat.org}{www.isocat.org}, which is the DCR used by ISO TC37 and hosted by its registration authority, the MPI for Psycholinguistics in Nijmegen.\leavevmode\bgroup\exampleFont \begin{shaded}\noindent\mbox{}{<\textbf{fs}\mbox{}\newline 
   xmlns:dcr="http://www.isocat.org/ns/dcr">}\mbox{}\newline 
\hspace*{6pt}{<\textbf{f}\hspace*{6pt}{dcr:datcat}="{http://www.isocat.org/datcat/DC-1345}"\mbox{}\newline 
\hspace*{6pt}\hspace*{6pt}{dcr:valueDatcat}="{http://www.isocat.org/datcat/DC-1256}"\hspace*{6pt}{fVal}="{\#commonNoun}"\hspace*{6pt}{name}="{POS}"/>}\mbox{}\newline 
{</\textbf{fs}>}\end{shaded}\egroup 


\end{reflist}  
\begin{reflist}
\item[]\begin{specHead}{TEI.att.declarable}{att.declarable}\index{att.declarable (attribute class)|oddindex}\index{default=@default!att.declarable (attribute class)|oddindex} fournit des attributs pour ces éléments de l'en-tête TEI qui peuvent être choisis indépendamment au moyen de l'attribut {\itshape decls}.\end{specHead} 
    \item[{Module}]
  tei
    \item[{Membres}]
  \hyperref[TEI.availability]{availability} \hyperref[TEI.bibl]{bibl} \hyperref[TEI.biblFull]{biblFull} \hyperref[TEI.biblStruct]{biblStruct} \hyperref[TEI.correction]{correction} \hyperref[TEI.langUsage]{langUsage} \hyperref[TEI.listBibl]{listBibl} \hyperref[TEI.listOrg]{listOrg} \hyperref[TEI.listPlace]{listPlace} \hyperref[TEI.sourceDesc]{sourceDesc} \hyperref[TEI.textClass]{textClass}
    \item[{Attributs}]
  Attributs\hfil\\[-10pt]\begin{sansreflist}
    \item[@default]
  Indique si oui ou non cet élément est affecté par défaut quand son élément parent a été sélectionné.
\begin{reflist}
    \item[{Statut}]
  Optionel
    \item[{Type de données}]
  \hyperref[TEI.teidata.truthValue]{teidata.truthValue}
    \item[{Les valeurs autorisées sont:}]
  \begin{description}

\item[{true}]cet élément est choisi si son parent est choisi
\item[{false}]cet élément ne peut être sélectionné qu'explicitement, à moins qu'il ne soit le seul de ce type, auquel cas il est sélectionné si son parent a été choisi{[Valeur par défaut] }
\end{description} 
\end{reflist}  
\end{sansreflist}  
    \item[{Note}]
  \par
Les règles régissant l'association d'éléments déclarables avec des parties individuelles d'un texte TEI sont entièrement définies au chap1itre \xref{http://www.tei-c.org/release/doc/tei-p5-doc/en/html/CC.html\#CCAS}{15.3. Associating Contextual Information with a Text}. Un seul élément d'un type particulier peut avoir un attribut {\itshape default} avec une valeur true.
\end{reflist}  
\begin{reflist}
\item[]\begin{specHead}{TEI.att.declaring}{att.declaring}\index{att.declaring (attribute class)|oddindex}\index{decls=@decls!att.declaring (attribute class)|oddindex} fournit des attributs pour les éléments qui peuvent être associés indépendamment à un élément particulier déclarable dans l'en-tête TEI, ignorant ainsi la valeur dont cet élément devrait hériter par défaut\end{specHead} 
    \item[{Module}]
  tei
    \item[{Membres}]
  \hyperref[TEI.ab]{ab} \hyperref[TEI.back]{back} \hyperref[TEI.body]{body} \hyperref[TEI.div]{div} \hyperref[TEI.facsimile]{facsimile} \hyperref[TEI.floatingText]{floatingText} \hyperref[TEI.front]{front} \hyperref[TEI.gloss]{gloss} \hyperref[TEI.graphic]{graphic} \hyperref[TEI.group]{group} \hyperref[TEI.lg]{lg} \hyperref[TEI.listAnnotation]{listAnnotation} \hyperref[TEI.media]{media} \hyperref[TEI.msDesc]{msDesc} \hyperref[TEI.p]{p} \hyperref[TEI.ptr]{ptr} \hyperref[TEI.ref]{ref} \hyperref[TEI.sourceDoc]{sourceDoc} \hyperref[TEI.surface]{surface} \hyperref[TEI.surfaceGrp]{surfaceGrp} \hyperref[TEI.term]{term} \hyperref[TEI.text]{text}
    \item[{Attributs}]
  Attributs\hfil\\[-10pt]\begin{sansreflist}
    \item[@decls]
  identifie un ou plusieurs\textit{éléments déclarables}dans l'en-tête TEI, qui sont destinés à s'appliquer à l'élément portant cet attribut et à son contenu.
\begin{reflist}
    \item[{Statut}]
  Optionel
    \item[{Type de données}]
  1–∞ occurrences de \hyperref[TEI.teidata.pointer]{teidata.pointer} séparé par un espace
\end{reflist}  
\end{sansreflist}  
    \item[{Note}]
  \par
Les règles régissant l'association d'éléments déclarables avec des parties individuelles d'un texte TEI sont entièrement définies au chapitre \xref{http://www.tei-c.org/release/doc/tei-p5-doc/en/html/CC.html\#CCAS}{15.3. Associating Contextual Information with a Text}.
\end{reflist}  
\begin{reflist}
\item[]\begin{specHead}{TEI.att.dimensions}{att.dimensions}\index{att.dimensions (attribute class)|oddindex}\index{unit=@unit!att.dimensions (attribute class)|oddindex}\index{quantity=@quantity!att.dimensions (attribute class)|oddindex}\index{extent=@extent!att.dimensions (attribute class)|oddindex}\index{precision=@precision!att.dimensions (attribute class)|oddindex}\index{scope=@scope!att.dimensions (attribute class)|oddindex} fournit des attributs pour décrire la taille des objets physiques\end{specHead} 
    \item[{Module}]
  tei
    \item[{Membres}]
  \hyperref[TEI.att.damaged]{att.damaged}[\hyperref[TEI.damage]{damage} \hyperref[TEI.damageSpan]{damageSpan}] \hyperref[TEI.att.editLike]{att.editLike}[\hyperref[TEI.att.transcriptional]{att.transcriptional}[\hyperref[TEI.add]{add} \hyperref[TEI.addSpan]{addSpan} \hyperref[TEI.del]{del} \hyperref[TEI.delSpan]{delSpan} \hyperref[TEI.mod]{mod} \hyperref[TEI.redo]{redo} \hyperref[TEI.restore]{restore} \hyperref[TEI.retrace]{retrace} \hyperref[TEI.subst]{subst} \hyperref[TEI.substJoin]{substJoin} \hyperref[TEI.undo]{undo}] \hyperref[TEI.affiliation]{affiliation} \hyperref[TEI.am]{am} \hyperref[TEI.corr]{corr} \hyperref[TEI.date]{date} \hyperref[TEI.event]{event} \hyperref[TEI.ex]{ex} \hyperref[TEI.expan]{expan} \hyperref[TEI.gap]{gap} \hyperref[TEI.geogName]{geogName} \hyperref[TEI.location]{location} \hyperref[TEI.name]{name} \hyperref[TEI.org]{org} \hyperref[TEI.orgName]{orgName} \hyperref[TEI.origDate]{origDate} \hyperref[TEI.origPlace]{origPlace} \hyperref[TEI.origin]{origin} \hyperref[TEI.persName]{persName} \hyperref[TEI.person]{person} \hyperref[TEI.persona]{persona} \hyperref[TEI.place]{place} \hyperref[TEI.placeName]{placeName} \hyperref[TEI.reg]{reg} \hyperref[TEI.secl]{secl} \hyperref[TEI.state]{state} \hyperref[TEI.supplied]{supplied} \hyperref[TEI.surplus]{surplus} \hyperref[TEI.time]{time} \hyperref[TEI.unclear]{unclear}] \hyperref[TEI.depth]{depth} \hyperref[TEI.dim]{dim} \hyperref[TEI.dimensions]{dimensions} \hyperref[TEI.height]{height} \hyperref[TEI.space]{space} \hyperref[TEI.width]{width}
    \item[{Attributs}]
  Attributs \hyperref[TEI.att.ranging]{att.ranging} (\textit{@atLeast}, \textit{@atMost}, \textit{@min}, \textit{@max}, \textit{@confidence}) \hfil\\[-10pt]\begin{sansreflist}
    \item[@unit]
  noms des unités utilisées pour la mesure.
\begin{reflist}
    \item[{Statut}]
  Optionel
    \item[{Type de données}]
  \hyperref[TEI.teidata.enumerated]{teidata.enumerated}
    \item[{Les valeurs suggérées comprennent:}]
  \begin{description}

\item[{cm}](centimètres)
\item[{mm}](millimètres)
\item[{in}](pouces)
\item[{lines}]lignes de texte
\item[{chars}](characters) caractères du texte
\end{description} 
\end{reflist}  
    \item[@quantity]
  spécifie la longueur dans les unités indiquées
\begin{reflist}
    \item[{Statut}]
  Optionel
    \item[{Type de données}]
  \hyperref[TEI.teidata.numeric]{teidata.numeric}
\end{reflist}  
    \item[@extent]
  indique la dimension de l'objet en utilisant un vocabulaire spécifique à un projet qui combine la quantité et l'unité dans une chaîne seule de mots.
\begin{reflist}
    \item[{Statut}]
  Optionel
    \item[{Type de données}]
  \hyperref[TEI.teidata.text]{teidata.text}
    \item[]\exampleFont {<\textbf{gap}\hspace*{6pt}{extent}="{5 words}"/>}
    \item[]\exampleFont {<\textbf{height}\hspace*{6pt}{extent}="{half the page}"/>}
\end{reflist}  
    \item[@precision]
  caractérise la précision des valeurs spécifiées par les autres attributs.
\begin{reflist}
    \item[{Statut}]
  Optionel
    \item[{Type de données}]
  \hyperref[TEI.teidata.certainty]{teidata.certainty}
\end{reflist}  
    \item[@scope]
  spécifie l'applicabilité de cette mesure, là où plus d'un objet est mesuré.
\begin{reflist}
    \item[{Statut}]
  Optionel
    \item[{Type de données}]
  \hyperref[TEI.teidata.enumerated]{teidata.enumerated}
    \item[{Exemple de valeurs possibles:}]
  \begin{description}

\item[{all}]la mesure s'applique à tous les cas.
\item[{most}]la mesure s'applique à la plupart des cas examinés
\item[{range}]la mesure s'applique seulement à l'ensemble des exemples indiqués.
\end{description} 
\end{reflist}  
\end{sansreflist}  
\end{reflist}  
\begin{reflist}
\item[]\begin{specHead}{TEI.att.divLike}{att.divLike}\index{att.divLike (attribute class)|oddindex}\index{org=@org!att.divLike (attribute class)|oddindex}\index{sample=@sample!att.divLike (attribute class)|oddindex} fournit un jeu d'attributs communs à tous les éléments qui offrent les mêmes caractéristiques que des divisions\end{specHead} 
    \item[{Module}]
  tei
    \item[{Membres}]
  \hyperref[TEI.div]{div} \hyperref[TEI.lg]{lg}
    \item[{Attributs}]
  Attributs \hyperref[TEI.att.fragmentable]{att.fragmentable} (\textit{@part}) \hfil\\[-10pt]\begin{sansreflist}
    \item[@org]
  (organisation) précise l'organisation du contenu de la division
\begin{reflist}
    \item[{Statut}]
  Optionel
    \item[{Type de données}]
  \hyperref[TEI.teidata.enumerated]{teidata.enumerated}
    \item[{Les valeurs autorisées sont:}]
  \begin{description}

\item[{composite}]aucune déclaration n'est faite quant à l'ordre dans lequel les composants de cette division doivent être traités ou bien quant à leurs corrélations
\item[{uniform}]contenu uniforme : c'est-à-dire que les composants de cet élément sont à considérer comme formant une unité logique et doivent être traités dans l'ordre séquentiel{[Valeur par défaut] }
\end{description} 
\end{reflist}  
    \item[@sample]
  indique si cette division est un échantillon de la source originale et dans ce cas, de quelle partie.
\begin{reflist}
    \item[{Statut}]
  Optionel
    \item[{Type de données}]
  \hyperref[TEI.teidata.enumerated]{teidata.enumerated}
    \item[{Les valeurs autorisées sont:}]
  \begin{description}

\item[{initial}]par rapport à la source, lacune à la fin de la division
\item[{medial}]par rapport à la source, lacune au début et à la fin de la division
\item[{final}]par rapport à la source, lacune au début de la division
\item[{unknown}]par rapport à la source, position de l'échantillon inconnue
\item[{complete}]la division n'est pas un échantillon{[Valeur par défaut] }
\end{description} 
\end{reflist}  
\end{sansreflist}  
\end{reflist}  
\begin{reflist}
\item[]\begin{specHead}{TEI.att.docStatus}{att.docStatus}\index{att.docStatus (attribute class)|oddindex}\index{status=@status!att.docStatus (attribute class)|oddindex} provides attributes for use on metadata elements describing the status of a document.\end{specHead} 
    \item[{Module}]
  tei
    \item[{Membres}]
  \hyperref[TEI.bibl]{bibl} \hyperref[TEI.biblFull]{biblFull} \hyperref[TEI.biblStruct]{biblStruct} \hyperref[TEI.change]{change} \hyperref[TEI.msDesc]{msDesc} \hyperref[TEI.revisionDesc]{revisionDesc}
    \item[{Attributs}]
  Attributs\hfil\\[-10pt]\begin{sansreflist}
    \item[@status]
  describes the status of a document either currently or, when associated with a dated element, at the time indicated.
\begin{reflist}
    \item[{Statut}]
  Optionel
    \item[{Type de données}]
  \hyperref[TEI.teidata.enumerated]{teidata.enumerated}
    \item[{Exemple de valeurs possibles:}]
  \begin{description}

\item[{approved}]
\item[{candidate}]
\item[{cleared}]
\item[{deprecated}]
\item[{draft}]{[Valeur par défaut] }
\item[{embargoed}]
\item[{expired}]
\item[{frozen}]
\item[{galley}]
\item[{proposed}]
\item[{published}]
\item[{recommendation}]
\item[{submitted}]
\item[{unfinished}]
\item[{withdrawn}]
\end{description} 
\end{reflist}  
\end{sansreflist}  
    \item[{Exemple}]
  \leavevmode\bgroup\exampleFont \begin{shaded}\noindent\mbox{}{<\textbf{revisionDesc}\hspace*{6pt}{status}="{published}">}\mbox{}\newline 
\hspace*{6pt}{<\textbf{change}\hspace*{6pt}{status}="{published}"\mbox{}\newline 
\hspace*{6pt}\hspace*{6pt}{when}="{2010-10-21}"/>}\mbox{}\newline 
\hspace*{6pt}{<\textbf{change}\hspace*{6pt}{status}="{cleared}"\hspace*{6pt}{when}="{2010-10-02}"/>}\mbox{}\newline 
\hspace*{6pt}{<\textbf{change}\hspace*{6pt}{status}="{embargoed}"\mbox{}\newline 
\hspace*{6pt}\hspace*{6pt}{when}="{2010-08-02}"/>}\mbox{}\newline 
\hspace*{6pt}{<\textbf{change}\hspace*{6pt}{status}="{frozen}"\hspace*{6pt}{when}="{2010-05-01}"\mbox{}\newline 
\hspace*{6pt}\hspace*{6pt}{who}="{\#MSM}"/>}\mbox{}\newline 
\hspace*{6pt}{<\textbf{change}\hspace*{6pt}{status}="{draft}"\hspace*{6pt}{when}="{2010-03-01}"\mbox{}\newline 
\hspace*{6pt}\hspace*{6pt}{who}="{\#LB}"/>}\mbox{}\newline 
{</\textbf{revisionDesc}>}\end{shaded}\egroup 


\end{reflist}  
\begin{reflist}
\item[]\begin{specHead}{TEI.att.duration}{att.duration}\index{att.duration (attribute class)|oddindex} fournit des attributs pour la normalisation des éléments qui contiennent des mentions d'événements datables\end{specHead} 
    \item[{Module}]
  spoken
    \item[{Membres}]
  \hyperref[TEI.att.timed]{att.timed}[\hyperref[TEI.annotationBlock]{annotationBlock} \hyperref[TEI.bibl]{bibl} \hyperref[TEI.binaryObject]{binaryObject} \hyperref[TEI.gap]{gap} \hyperref[TEI.listAnnotation]{listAnnotation} \hyperref[TEI.media]{media}] \hyperref[TEI.date]{date} \hyperref[TEI.time]{time}
    \item[{Attributs}]
  Attributs \hyperref[TEI.att.duration.w3c]{att.duration.w3c} (\textit{@dur}) \hyperref[TEI.att.duration.iso]{att.duration.iso} (\textit{@dur-iso}) 
    \item[{Note}]
  \par
Cette‘superclasse’ fournit des attributs qui peuvent être employés pour fournir des valeurs normalisées d'information relative au temps. Par défaut, les attributs de la classe \textsf{att.duration.w3c} sont fournis. Si le module pour les noms et dates est chargé, cette classe fournit également des attributs de la classe\textsf{att.duration.iso}. En général, les valeurs possibles des attributs limitées aux types de données W3C forment un sous-ensemble des valeurs que l'on trouve dans la norme ISO 8601. Cependant, il est rarement nécessaire de recourir aux possibilités très étendues des types de données de l'ISO, il existe en effet une bien plus grande offre logicielle pour le traitement des types de données W3C.
\end{reflist}  
\begin{reflist}
\item[]\begin{specHead}{TEI.att.duration.iso}{att.duration.iso}\index{att.duration.iso (attribute class)|oddindex}\index{dur-iso=@dur-iso!att.duration.iso (attribute class)|oddindex} attributs pour l'enregistrement de durées temporelles normalisées.\end{specHead} 
    \item[{Module}]
  tei
    \item[{Membres}]
  \hyperref[TEI.att.duration]{att.duration}[\hyperref[TEI.att.timed]{att.timed}[\hyperref[TEI.annotationBlock]{annotationBlock} \hyperref[TEI.bibl]{bibl} \hyperref[TEI.binaryObject]{binaryObject} \hyperref[TEI.gap]{gap} \hyperref[TEI.listAnnotation]{listAnnotation} \hyperref[TEI.media]{media}] \hyperref[TEI.date]{date} \hyperref[TEI.time]{time}]
    \item[{Attributs}]
  Attributs\hfil\\[-10pt]\begin{sansreflist}
    \item[@dur-iso]
  (durée) indique la longueur de cet élément dans le temps
\begin{reflist}
    \item[{Statut}]
  Optionel
    \item[{Type de données}]
  \hyperref[TEI.teidata.duration.iso]{teidata.duration.iso}
\end{reflist}  
\end{sansreflist}  
    \item[{Note}]
  \par
Si les attributs {\itshape when-iso} et {\itshape dur} ou {\itshape dur-iso} sont tous les deux spécifiés, les valeurs doivent être interprétées comme indiquant un intervalle de temps au moyen de son point de départ (ou date) et de sa durée. Afin de représenter une étendue temporelle par sa durée et sa fin on doit utiliser l'attribut {\itshape when-iso}.\par
En fournissant une forme "régularisée", il n'est rien affirmé sur la correction ou l'incorrection de la forme dans le texte source ; la forme régularisée est simplement celle qui est choisie comme forme principale afin de réunir les variantes de forme sous une seule rubrique.
\end{reflist}  
\begin{reflist}
\item[]\begin{specHead}{TEI.att.duration.w3c}{att.duration.w3c}\index{att.duration.w3c (attribute class)|oddindex}\index{dur=@dur!att.duration.w3c (attribute class)|oddindex} attributs pour enregistrer des durées de temps normalisées\end{specHead} 
    \item[{Module}]
  tei
    \item[{Membres}]
  \hyperref[TEI.att.duration]{att.duration}[\hyperref[TEI.att.timed]{att.timed}[\hyperref[TEI.annotationBlock]{annotationBlock} \hyperref[TEI.bibl]{bibl} \hyperref[TEI.binaryObject]{binaryObject} \hyperref[TEI.gap]{gap} \hyperref[TEI.listAnnotation]{listAnnotation} \hyperref[TEI.media]{media}] \hyperref[TEI.date]{date} \hyperref[TEI.time]{time}]
    \item[{Attributs}]
  Attributs\hfil\\[-10pt]\begin{sansreflist}
    \item[@dur]
  (durée) indique la longueur de cet élément dans le temps
\begin{reflist}
    \item[{Statut}]
  Optionel
    \item[{Type de données}]
  \hyperref[TEI.teidata.duration.w3c]{teidata.duration.w3c}
\end{reflist}  
\end{sansreflist}  
    \item[{Note}]
  \par
Si {\itshape when} et {\itshape dur} sont indiqués en même temps, leurs valeurs doivent être interprétées comme indiquant un espace de temps à partir de son heure (ou date) de départ et de sa durée. Afin de représenter une fourchette de temps par une durée et sa fin, l'attribut {\itshape when-iso} doit être employé.\par
En fournissant une forme ‘regularized’, il n'est rien affrimé sur la correction ou l'incorrection de la forme dans le texte source ; la forme régularisée est simplement celle qui est choisie comme forme principale afin d'unifier les variantes de forme sous une seule rubrique .
\end{reflist}  
\begin{reflist}
\item[]\begin{specHead}{TEI.att.editLike}{att.editLike}\index{att.editLike (attribute class)|oddindex}\index{evidence=@evidence!att.editLike (attribute class)|oddindex}\index{instant=@instant!att.editLike (attribute class)|oddindex} fournit des attributs décrivant la nature d'une intervention savante encodée ou de tout autre interprétation.\end{specHead} 
    \item[{Module}]
  tei
    \item[{Membres}]
  \hyperref[TEI.att.transcriptional]{att.transcriptional}[\hyperref[TEI.add]{add} \hyperref[TEI.addSpan]{addSpan} \hyperref[TEI.del]{del} \hyperref[TEI.delSpan]{delSpan} \hyperref[TEI.mod]{mod} \hyperref[TEI.redo]{redo} \hyperref[TEI.restore]{restore} \hyperref[TEI.retrace]{retrace} \hyperref[TEI.subst]{subst} \hyperref[TEI.substJoin]{substJoin} \hyperref[TEI.undo]{undo}] \hyperref[TEI.affiliation]{affiliation} \hyperref[TEI.am]{am} \hyperref[TEI.corr]{corr} \hyperref[TEI.date]{date} \hyperref[TEI.event]{event} \hyperref[TEI.ex]{ex} \hyperref[TEI.expan]{expan} \hyperref[TEI.gap]{gap} \hyperref[TEI.geogName]{geogName} \hyperref[TEI.location]{location} \hyperref[TEI.name]{name} \hyperref[TEI.org]{org} \hyperref[TEI.orgName]{orgName} \hyperref[TEI.origDate]{origDate} \hyperref[TEI.origPlace]{origPlace} \hyperref[TEI.origin]{origin} \hyperref[TEI.persName]{persName} \hyperref[TEI.person]{person} \hyperref[TEI.persona]{persona} \hyperref[TEI.place]{place} \hyperref[TEI.placeName]{placeName} \hyperref[TEI.reg]{reg} \hyperref[TEI.secl]{secl} \hyperref[TEI.state]{state} \hyperref[TEI.supplied]{supplied} \hyperref[TEI.surplus]{surplus} \hyperref[TEI.time]{time} \hyperref[TEI.unclear]{unclear}
    \item[{Attributs}]
  Attributs \hyperref[TEI.att.dimensions]{att.dimensions} (\textit{@unit}, \textit{@quantity}, \textit{@extent}, \textit{@precision}, \textit{@scope})  (\hyperref[TEI.att.ranging]{att.ranging} (\textit{@atLeast}, \textit{@atMost}, \textit{@min}, \textit{@max}, \textit{@confidence})) \hfil\\[-10pt]\begin{sansreflist}
    \item[@evidence]
  indique la nature de la preuve attestant la fiabilité ou la justesse de l'intervention ou de l'interprétation.
\begin{reflist}
    \item[{Statut}]
  Optionel
    \item[{Type de données}]
  1–∞ occurrences de \hyperref[TEI.teidata.enumerated]{teidata.enumerated} séparé par un espace
    \item[{Les valeurs suggérées comprennent:}]
  \begin{description}

\item[{internal}]l'intervention est justifiée par une preuve interne
\item[{external}]l'intervention est justifiée par une preuve externe
\item[{conjecture}]l'intervention ou l'interprétation a été faite par le rédacteur, le catalogueur, ou le chercheur sur la base de leur expertise.
\end{description} 
\end{reflist}  
    \item[@instant]
  indicates whether this is an instant revision or not.
\begin{reflist}
    \item[{Statut}]
  Optionel
    \item[{Type de données}]
  \hyperref[TEI.teidata.xTruthValue]{teidata.xTruthValue}
    \item[{Valeur par défaut}]
  false
\end{reflist}  
\end{sansreflist}  
    \item[{Note}]
  \par
Les membres de cette classe d'attributs sont couramment employés pour représenter tout type d'intervention éditoriale dans un texte, par exemple une correction ou une interprétation, ou bien une datation ou une localisation de manuscrit, etc. 
\end{reflist}  
\begin{reflist}
\item[]\begin{specHead}{TEI.att.edition}{att.edition}\index{att.edition (attribute class)|oddindex}\index{ed=@ed!att.edition (attribute class)|oddindex}\index{edRef=@edRef!att.edition (attribute class)|oddindex} fournit des attributs identifiant l'édition source dont provient une quelconque caractéristique encodée.\end{specHead} 
    \item[{Module}]
  tei
    \item[{Membres}]
  \hyperref[TEI.cb]{cb} \hyperref[TEI.lb]{lb} \hyperref[TEI.milestone]{milestone} \hyperref[TEI.pb]{pb}
    \item[{Attributs}]
  Attributs\hfil\\[-10pt]\begin{sansreflist}
    \item[@ed]
  (édition) fournit un identifiant arbitraire pour l'édition source dans laquelle la caractéristique associée (par exemple, une page, une colonne ou un saut de ligne) apparaît à ce point du texte.
\begin{reflist}
    \item[{Statut}]
  Optionel
    \item[{Type de données}]
  1–∞ occurrences de \hyperref[TEI.teidata.word]{teidata.word} séparé par un espace
\end{reflist}  
    \item[@edRef]
  (edition reference) provides a pointer to the source edition in which the associated feature (for example, a page, column, or line break) occurs at this point in the text.
\begin{reflist}
    \item[{Statut}]
  Optionel
    \item[{Type de données}]
  1–∞ occurrences de \hyperref[TEI.teidata.pointer]{teidata.pointer} séparé par un espace
\end{reflist}  
\end{sansreflist}  
    \item[{Exemple}]
  \leavevmode\bgroup\exampleFont \begin{shaded}\noindent\mbox{}{<\textbf{l}>}Of Mans First Disobedience,{<\textbf{lb}\hspace*{6pt}{ed}="{1674}"/>} and{<\textbf{lb}\hspace*{6pt}{ed}="{1667}"/>} the Fruit{</\textbf{l}>}\mbox{}\newline 
{<\textbf{l}>}Of that Forbidden Tree, whose{<\textbf{lb}\hspace*{6pt}{ed}="{1667 1674}"/>} mortal tast{</\textbf{l}>}\mbox{}\newline 
{<\textbf{l}>}Brought Death into the World,{<\textbf{lb}\hspace*{6pt}{ed}="{1667}"/>} and all{<\textbf{lb}\hspace*{6pt}{ed}="{1674}"/>} our woe,{</\textbf{l}>}\end{shaded}\egroup 


    \item[{Exemple}]
  \leavevmode\bgroup\exampleFont \begin{shaded}\noindent\mbox{}{<\textbf{listBibl}>}\mbox{}\newline 
\hspace*{6pt}{<\textbf{bibl}\hspace*{6pt}{xml:id}="{stapledon1937}">}\mbox{}\newline 
\hspace*{6pt}\hspace*{6pt}{<\textbf{author}>}Olaf Stapledon{</\textbf{author}>},\mbox{}\newline 
\hspace*{6pt}{<\textbf{title}>}Starmaker{</\textbf{title}>}, {<\textbf{publisher}>}Methuen{</\textbf{publisher}>}, {<\textbf{date}>}1937{</\textbf{date}>}\mbox{}\newline 
\hspace*{6pt}{</\textbf{bibl}>}\mbox{}\newline 
\hspace*{6pt}{<\textbf{bibl}\hspace*{6pt}{xml:id}="{stapledon1968}">}\mbox{}\newline 
\hspace*{6pt}\hspace*{6pt}{<\textbf{author}>}Olaf Stapledon{</\textbf{author}>},\mbox{}\newline 
\hspace*{6pt}{<\textbf{title}>}Starmaker{</\textbf{title}>}, {<\textbf{publisher}>}Dover{</\textbf{publisher}>}, {<\textbf{date}>}1968{</\textbf{date}>}\mbox{}\newline 
\hspace*{6pt}{</\textbf{bibl}>}\mbox{}\newline 
{</\textbf{listBibl}>}\mbox{}\newline 
{<\textbf{p}>}Looking into the future aeons from the supreme moment of\mbox{}\newline 
 the cosmos, I saw the populations still with all their\mbox{}\newline 
 strength maintaining the{<\textbf{pb}\hspace*{6pt}{edRef}="{\#stapledon1968}"\hspace*{6pt}{n}="{411}"/>}essentials of their ancient culture,\mbox{}\newline 
 still living their personal lives in zest and endless\mbox{}\newline 
 novelty of action, … I saw myself still\mbox{}\newline 
 preserving, though with increasing difficulty, my lucid\mbox{}\newline 
 con-{<\textbf{pb}\hspace*{6pt}{edRef}="{\#stapledon1937}"\hspace*{6pt}{n}="{291}"/>}sciousness;{</\textbf{p}>}\end{shaded}\egroup 


\end{reflist}  
\begin{reflist}
\item[]\begin{specHead}{TEI.att.fragmentable}{att.fragmentable}\index{att.fragmentable (attribute class)|oddindex}\index{part=@part!att.fragmentable (attribute class)|oddindex} provides an attribute for representing fragmentation of a structural element, typically as a consequence of some overlapping hierarchy.\end{specHead} 
    \item[{Module}]
  tei
    \item[{Membres}]
  \hyperref[TEI.att.divLike]{att.divLike}[\hyperref[TEI.div]{div} \hyperref[TEI.lg]{lg}] \hyperref[TEI.att.segLike]{att.segLike}[\hyperref[TEI.c]{c} \hyperref[TEI.cl]{cl} \hyperref[TEI.m]{m} \hyperref[TEI.pc]{pc} \hyperref[TEI.phr]{phr} \hyperref[TEI.s]{s} \hyperref[TEI.seg]{seg} \hyperref[TEI.w]{w}] \hyperref[TEI.ab]{ab} \hyperref[TEI.l]{l} \hyperref[TEI.p]{p}
    \item[{Attributs}]
  Attributs\hfil\\[-10pt]\begin{sansreflist}
    \item[@part]
  specifies whether or not its parent element is fragmented in some way, typically by some other overlapping structure: for example a speech which is divided between two or more verse stanzas, a paragraph which is split across a page division, a verse line which is divided between two speakers.
\begin{reflist}
    \item[{Statut}]
  Optionel
    \item[{Type de données}]
  \hyperref[TEI.teidata.enumerated]{teidata.enumerated}
    \item[{Les valeurs autorisées sont:}]
  \begin{description}

\item[{Y}](yes) the element is fragmented in some (unspecified) respect
\item[{N}](no) the element is not fragmented, or no claim is made as to its completeness{[Valeur par défaut] }
\item[{I}](initial) this is the initial part of a fragmented element
\item[{M}](medial) this is a medial part of a fragmented element
\item[{F}](final) this is the final part of a fragmented element
\end{description} 
\end{reflist}  
\end{sansreflist}  
\end{reflist}  
\begin{reflist}
\item[]\begin{specHead}{TEI.att.global}{att.global}\index{att.global (attribute class)|oddindex}\index{xml:id=@xml:id!att.global (attribute class)|oddindex}\index{n=@n!att.global (attribute class)|oddindex}\index{xml:lang=@xml:lang!att.global (attribute class)|oddindex}\index{xml:base=@xml:base!att.global (attribute class)|oddindex}\index{xml:space=@xml:space!att.global (attribute class)|oddindex} fournit un jeu d'attributs communs à tous les éléments dans le système de codage TEI.\end{specHead} 
    \item[{Module}]
  tei
    \item[{Membres}]
  \hyperref[TEI.TEI]{TEI} \hyperref[TEI.ab]{ab} \hyperref[TEI.abbr]{abbr} \hyperref[TEI.abstract]{abstract} \hyperref[TEI.accMat]{accMat} \hyperref[TEI.acquisition]{acquisition} \hyperref[TEI.add]{add} \hyperref[TEI.addName]{addName} \hyperref[TEI.addSpan]{addSpan} \hyperref[TEI.additional]{additional} \hyperref[TEI.additions]{additions} \hyperref[TEI.addrLine]{addrLine} \hyperref[TEI.address]{address} \hyperref[TEI.adminInfo]{adminInfo} \hyperref[TEI.affiliation]{affiliation} \hyperref[TEI.alt]{alt} \hyperref[TEI.altGrp]{altGrp} \hyperref[TEI.altIdentifier]{altIdentifier} \hyperref[TEI.am]{am} \hyperref[TEI.analytic]{analytic} \hyperref[TEI.anchor]{anchor} \hyperref[TEI.annotationBlock]{annotationBlock} \hyperref[TEI.appInfo]{appInfo} \hyperref[TEI.application]{application} \hyperref[TEI.author]{author} \hyperref[TEI.authority]{authority} \hyperref[TEI.availability]{availability} \hyperref[TEI.back]{back} \hyperref[TEI.bibl]{bibl} \hyperref[TEI.biblFull]{biblFull} \hyperref[TEI.biblScope]{biblScope} \hyperref[TEI.biblStruct]{biblStruct} \hyperref[TEI.bicond]{bicond} \hyperref[TEI.binary]{binary} \hyperref[TEI.binaryObject]{binaryObject} \hyperref[TEI.binding]{binding} \hyperref[TEI.bindingDesc]{bindingDesc} \hyperref[TEI.body]{body} \hyperref[TEI.c]{c} \hyperref[TEI.catchwords]{catchwords} \hyperref[TEI.category]{category} \hyperref[TEI.cb]{cb} \hyperref[TEI.cell]{cell} \hyperref[TEI.change]{change} \hyperref[TEI.choice]{choice} \hyperref[TEI.cit]{cit} \hyperref[TEI.citedRange]{citedRange} \hyperref[TEI.cl]{cl} \hyperref[TEI.classCode]{classCode} \hyperref[TEI.classDecl]{classDecl} \hyperref[TEI.collation]{collation} \hyperref[TEI.collection]{collection} \hyperref[TEI.colophon]{colophon} \hyperref[TEI.cond]{cond} \hyperref[TEI.condition]{condition} \hyperref[TEI.corr]{corr} \hyperref[TEI.correction]{correction} \hyperref[TEI.country]{country} \hyperref[TEI.creation]{creation} \hyperref[TEI.custEvent]{custEvent} \hyperref[TEI.custodialHist]{custodialHist} \hyperref[TEI.damage]{damage} \hyperref[TEI.damageSpan]{damageSpan} \hyperref[TEI.date]{date} \hyperref[TEI.decoDesc]{decoDesc} \hyperref[TEI.decoNote]{decoNote} \hyperref[TEI.default]{default} \hyperref[TEI.del]{del} \hyperref[TEI.delSpan]{delSpan} \hyperref[TEI.depth]{depth} \hyperref[TEI.desc]{desc} \hyperref[TEI.dim]{dim} \hyperref[TEI.dimensions]{dimensions} \hyperref[TEI.distinct]{distinct} \hyperref[TEI.distributor]{distributor} \hyperref[TEI.div]{div} \hyperref[TEI.divGen]{divGen} \hyperref[TEI.docAuthor]{docAuthor} \hyperref[TEI.docDate]{docDate} \hyperref[TEI.docEdition]{docEdition} \hyperref[TEI.docTitle]{docTitle} \hyperref[TEI.edition]{edition} \hyperref[TEI.editionStmt]{editionStmt} \hyperref[TEI.editor]{editor} \hyperref[TEI.email]{email} \hyperref[TEI.emph]{emph} \hyperref[TEI.encodingDesc]{encodingDesc} \hyperref[TEI.event]{event} \hyperref[TEI.ex]{ex} \hyperref[TEI.expan]{expan} \hyperref[TEI.explicit]{explicit} \hyperref[TEI.extent]{extent} \hyperref[TEI.f]{f} \hyperref[TEI.fDecl]{fDecl} \hyperref[TEI.fDescr]{fDescr} \hyperref[TEI.fLib]{fLib} \hyperref[TEI.facsimile]{facsimile} \hyperref[TEI.figDesc]{figDesc} \hyperref[TEI.figure]{figure} \hyperref[TEI.fileDesc]{fileDesc} \hyperref[TEI.filiation]{filiation} \hyperref[TEI.finalRubric]{finalRubric} \hyperref[TEI.floatingText]{floatingText} \hyperref[TEI.foliation]{foliation} \hyperref[TEI.foreign]{foreign} \hyperref[TEI.forename]{forename} \hyperref[TEI.formula]{formula} \hyperref[TEI.front]{front} \hyperref[TEI.fs]{fs} \hyperref[TEI.fsConstraints]{fsConstraints} \hyperref[TEI.fsDecl]{fsDecl} \hyperref[TEI.fsDescr]{fsDescr} \hyperref[TEI.fsdDecl]{fsdDecl} \hyperref[TEI.fsdLink]{fsdLink} \hyperref[TEI.funder]{funder} \hyperref[TEI.fvLib]{fvLib} \hyperref[TEI.fw]{fw} \hyperref[TEI.gap]{gap} \hyperref[TEI.gb]{gb} \hyperref[TEI.genName]{genName} \hyperref[TEI.geogName]{geogName} \hyperref[TEI.gloss]{gloss} \hyperref[TEI.graphic]{graphic} \hyperref[TEI.group]{group} \hyperref[TEI.handDesc]{handDesc} \hyperref[TEI.handNotes]{handNotes} \hyperref[TEI.handShift]{handShift} \hyperref[TEI.head]{head} \hyperref[TEI.headItem]{headItem} \hyperref[TEI.headLabel]{headLabel} \hyperref[TEI.height]{height} \hyperref[TEI.heraldry]{heraldry} \hyperref[TEI.hi]{hi} \hyperref[TEI.history]{history} \hyperref[TEI.idno]{idno} \hyperref[TEI.if]{if} \hyperref[TEI.iff]{iff} \hyperref[TEI.imprint]{imprint} \hyperref[TEI.incipit]{incipit} \hyperref[TEI.index]{index} \hyperref[TEI.institution]{institution} \hyperref[TEI.interp]{interp} \hyperref[TEI.interpGrp]{interpGrp} \hyperref[TEI.item]{item} \hyperref[TEI.join]{join} \hyperref[TEI.joinGrp]{joinGrp} \hyperref[TEI.keywords]{keywords} \hyperref[TEI.l]{l} \hyperref[TEI.label]{label} \hyperref[TEI.langUsage]{langUsage} \hyperref[TEI.language]{language} \hyperref[TEI.layout]{layout} \hyperref[TEI.layoutDesc]{layoutDesc} \hyperref[TEI.lb]{lb} \hyperref[TEI.lg]{lg} \hyperref[TEI.licence]{licence} \hyperref[TEI.line]{line} \hyperref[TEI.link]{link} \hyperref[TEI.linkGrp]{linkGrp} \hyperref[TEI.list]{list} \hyperref[TEI.listAnnotation]{listAnnotation} \hyperref[TEI.listBibl]{listBibl} \hyperref[TEI.listOrg]{listOrg} \hyperref[TEI.listPlace]{listPlace} \hyperref[TEI.listTranspose]{listTranspose} \hyperref[TEI.location]{location} \hyperref[TEI.locus]{locus} \hyperref[TEI.locusGrp]{locusGrp} \hyperref[TEI.m]{m} \hyperref[TEI.material]{material} \hyperref[TEI.measure]{measure} \hyperref[TEI.measureGrp]{measureGrp} \hyperref[TEI.media]{media} \hyperref[TEI.meeting]{meeting} \hyperref[TEI.mentioned]{mentioned} \hyperref[TEI.metamark]{metamark} \hyperref[TEI.milestone]{milestone} \hyperref[TEI.mod]{mod} \hyperref[TEI.monogr]{monogr} \hyperref[TEI.msContents]{msContents} \hyperref[TEI.msDesc]{msDesc} \hyperref[TEI.msFrag]{msFrag} \hyperref[TEI.msIdentifier]{msIdentifier} \hyperref[TEI.msItem]{msItem} \hyperref[TEI.msItemStruct]{msItemStruct} \hyperref[TEI.msName]{msName} \hyperref[TEI.msPart]{msPart} \hyperref[TEI.musicNotation]{musicNotation} \hyperref[TEI.name]{name} \hyperref[TEI.nameLink]{nameLink} \hyperref[TEI.namespace]{namespace} \hyperref[TEI.notatedMusic]{notatedMusic} \hyperref[TEI.note]{note} \hyperref[TEI.notesStmt]{notesStmt} \hyperref[TEI.num]{num} \hyperref[TEI.numeric]{numeric} \hyperref[TEI.objectDesc]{objectDesc} \hyperref[TEI.objectType]{objectType} \hyperref[TEI.org]{org} \hyperref[TEI.orgName]{orgName} \hyperref[TEI.orig]{orig} \hyperref[TEI.origDate]{origDate} \hyperref[TEI.origPlace]{origPlace} \hyperref[TEI.origin]{origin} \hyperref[TEI.p]{p} \hyperref[TEI.pb]{pb} \hyperref[TEI.pc]{pc} \hyperref[TEI.persName]{persName} \hyperref[TEI.person]{person} \hyperref[TEI.personGrp]{personGrp} \hyperref[TEI.persona]{persona} \hyperref[TEI.phr]{phr} \hyperref[TEI.physDesc]{physDesc} \hyperref[TEI.place]{place} \hyperref[TEI.placeName]{placeName} \hyperref[TEI.postBox]{postBox} \hyperref[TEI.postCode]{postCode} \hyperref[TEI.profileDesc]{profileDesc} \hyperref[TEI.provenance]{provenance} \hyperref[TEI.ptr]{ptr} \hyperref[TEI.pubPlace]{pubPlace} \hyperref[TEI.publicationStmt]{publicationStmt} \hyperref[TEI.publisher]{publisher} \hyperref[TEI.q]{q} \hyperref[TEI.quote]{quote} \hyperref[TEI.recordHist]{recordHist} \hyperref[TEI.redo]{redo} \hyperref[TEI.ref]{ref} \hyperref[TEI.reg]{reg} \hyperref[TEI.region]{region} \hyperref[TEI.relatedItem]{relatedItem} \hyperref[TEI.rendition]{rendition} \hyperref[TEI.repository]{repository} \hyperref[TEI.resp]{resp} \hyperref[TEI.respStmt]{respStmt} \hyperref[TEI.restore]{restore} \hyperref[TEI.retrace]{retrace} \hyperref[TEI.revisionDesc]{revisionDesc} \hyperref[TEI.roleName]{roleName} \hyperref[TEI.row]{row} \hyperref[TEI.rs]{rs} \hyperref[TEI.rubric]{rubric} \hyperref[TEI.s]{s} \hyperref[TEI.said]{said} \hyperref[TEI.schemaRef]{schemaRef} \hyperref[TEI.scriptDesc]{scriptDesc} \hyperref[TEI.seal]{seal} \hyperref[TEI.sealDesc]{sealDesc} \hyperref[TEI.secFol]{secFol} \hyperref[TEI.secl]{secl} \hyperref[TEI.seg]{seg} \hyperref[TEI.series]{series} \hyperref[TEI.seriesStmt]{seriesStmt} \hyperref[TEI.settlement]{settlement} \hyperref[TEI.sic]{sic} \hyperref[TEI.signatures]{signatures} \hyperref[TEI.soCalled]{soCalled} \hyperref[TEI.source]{source} \hyperref[TEI.sourceDesc]{sourceDesc} \hyperref[TEI.sourceDoc]{sourceDoc} \hyperref[TEI.sp]{sp} \hyperref[TEI.span]{span} \hyperref[TEI.spanGrp]{spanGrp} \hyperref[TEI.speaker]{speaker} \hyperref[TEI.stage]{stage} \hyperref[TEI.stamp]{stamp} \hyperref[TEI.standOff]{standOff} \hyperref[TEI.state]{state} \hyperref[TEI.street]{street} \hyperref[TEI.string]{string} \hyperref[TEI.subst]{subst} \hyperref[TEI.substJoin]{substJoin} \hyperref[TEI.summary]{summary} \hyperref[TEI.supplied]{supplied} \hyperref[TEI.support]{support} \hyperref[TEI.supportDesc]{supportDesc} \hyperref[TEI.surface]{surface} \hyperref[TEI.surfaceGrp]{surfaceGrp} \hyperref[TEI.surname]{surname} \hyperref[TEI.surplus]{surplus} \hyperref[TEI.surrogates]{surrogates} \hyperref[TEI.symbol]{symbol} \hyperref[TEI.table]{table} \hyperref[TEI.taxonomy]{taxonomy} \hyperref[TEI.teiCorpus]{teiCorpus} \hyperref[TEI.teiHeader]{teiHeader} \hyperref[TEI.term]{term} \hyperref[TEI.text]{text} \hyperref[TEI.textClass]{textClass} \hyperref[TEI.textLang]{textLang} \hyperref[TEI.then]{then} \hyperref[TEI.time]{time} \hyperref[TEI.timeline]{timeline} \hyperref[TEI.title]{title} \hyperref[TEI.titlePage]{titlePage} \hyperref[TEI.titlePart]{titlePart} \hyperref[TEI.titleStmt]{titleStmt} \hyperref[TEI.transpose]{transpose} \hyperref[TEI.typeDesc]{typeDesc} \hyperref[TEI.typeNote]{typeNote} \hyperref[TEI.unclear]{unclear} \hyperref[TEI.undo]{undo} \hyperref[TEI.vAlt]{vAlt} \hyperref[TEI.vColl]{vColl} \hyperref[TEI.vDefault]{vDefault} \hyperref[TEI.vLabel]{vLabel} \hyperref[TEI.vMerge]{vMerge} \hyperref[TEI.vNot]{vNot} \hyperref[TEI.vRange]{vRange} \hyperref[TEI.w]{w} \hyperref[TEI.watermark]{watermark} \hyperref[TEI.when]{when} \hyperref[TEI.width]{width} \hyperref[TEI.zone]{zone}
    \item[{Attributs}]
  Attributs \hyperref[TEI.att.global.rendition]{att.global.rendition} (\textit{@rend}, \textit{@style}, \textit{@rendition}) \hyperref[TEI.att.global.linking]{att.global.linking} (\textit{@corresp}, \textit{@synch}, \textit{@sameAs}, \textit{@copyOf}, \textit{@next}, \textit{@prev}, \textit{@exclude}, \textit{@select}) \hyperref[TEI.att.global.analytic]{att.global.analytic} (\textit{@ana}) \hyperref[TEI.att.global.facs]{att.global.facs} (\textit{@facs}) \hyperref[TEI.att.global.change]{att.global.change} (\textit{@change}) \hyperref[TEI.att.global.responsibility]{att.global.responsibility} (\textit{@cert}, \textit{@resp}) \hyperref[TEI.att.global.source]{att.global.source} (\textit{@source}) \hfil\\[-10pt]\begin{sansreflist}
    \item[@xml:id]
  (identifiant) fournit un identifiant unique pour l'élément qui porte l'attribut
\begin{reflist}
    \item[{Statut}]
  Optionel
    \item[{Type de données}]
  \xref{https://www.w3.org/TR/xmlschema-2/\#ID}{ID}
    \item[{Note}]
  \par
L'attribut {\itshape xml:id} peut être employé pour indiquer une référence canonique pour un élément ; voir la section\xref{http://www.tei-c.org/release/doc/tei-p5-doc/en/html/CO.html\#CORS}{3.10. Reference Systems}.
\end{reflist}  
    \item[@n]
  (nombre) donne un nombre (ou une autre étiquette) pour un élément, qui n'est pas nécessairement unique dans le document TEI.
\begin{reflist}
    \item[{Statut}]
  Optionel
    \item[{Type de données}]
  \hyperref[TEI.teidata.text]{teidata.text}
    \item[{Note}]
  \par
L'attribut {\itshape n} peut être employé pour indiquer la numérotation de chapitres, sections, items de liste, etc. ; il peut également être employé dans les spécifications d'un système standard de référence pour le texte.
\end{reflist}  
    \item[@xml:lang]
  (langue) indique la langue du contenu de l'élément en utilisant les codes du \xref{http://www.ietf.org/rfc/rfc3066.txt}{RFC 3066}
\begin{reflist}
    \item[{Statut}]
  Optionel
    \item[{Type de données}]
  \hyperref[TEI.teidata.language]{teidata.language}
    \item[]\exampleFont {<\textbf{p}>} … The consequences of\mbox{}\newline 
 this rapid depopulation were the loss of the last\mbox{}\newline 
{<\textbf{foreign}\hspace*{6pt}{xml:lang}="{rap}">}ariki{</\textbf{foreign}>} or chief\mbox{}\newline 
 (Routledge 1920:205,210) and their connections to\mbox{}\newline 
 ancestral territorial organization.{</\textbf{p}>}
    \item[{Note}]
  \par
Si aucune valeur n'est indiquée pour {\itshape xml:lang}, la valeur de l'attribut{\itshape xml:lang} de l'élément immédiatement supérieur est héritée ; c'est pour cette raison qu'une valeur devrait toujours être attribuée à l'élément du plus haut niveau hiérarchique (\hyperref[TEI.TEI]{<TEI>}).\par
La valeur doit être conforme au BCP 47. Si la valeur est un code d'usage privé (c'est-à-dire commence par \texttt{x-} ou contient \texttt{-x-}), il devrait correspondre à la valeur d'un attribut {\itshape ident} d'un élément \hyperref[TEI.language]{<language>} fourni dans l'en-tête TEI du document courant.
\end{reflist}  
    \item[@xml:base]
  donne une référence URI de base au moyen de laquelle les applications peuvent résoudre des références d'URI relatives en références d'URI absolues
\begin{reflist}
    \item[{Statut}]
  Optionel
    \item[{Type de données}]
  \hyperref[TEI.teidata.pointer]{teidata.pointer}
    \item[]\exampleFont {<\textbf{div}\hspace*{6pt}{type}="{bibl}">}\mbox{}\newline 
\hspace*{6pt}{<\textbf{head}>}Bibliography{</\textbf{head}>}\mbox{}\newline 
\hspace*{6pt}{<\textbf{listBibl}\hspace*{6pt}{xml:base}="{http://www.lib.ucdavis.edu/BWRP/Works/}">}\mbox{}\newline 
\hspace*{6pt}\hspace*{6pt}{<\textbf{bibl}>}\mbox{}\newline 
\hspace*{6pt}\hspace*{6pt}\hspace*{6pt}{<\textbf{author}>}\mbox{}\newline 
\hspace*{6pt}\hspace*{6pt}\hspace*{6pt}\hspace*{6pt}{<\textbf{name}>}Landon, Letitia Elizabeth{</\textbf{name}>}\mbox{}\newline 
\hspace*{6pt}\hspace*{6pt}\hspace*{6pt}{</\textbf{author}>}\mbox{}\newline 
\hspace*{6pt}\hspace*{6pt}\hspace*{6pt}{<\textbf{ref}\hspace*{6pt}{target}="{LandLVowOf.sgm}">}\mbox{}\newline 
\hspace*{6pt}\hspace*{6pt}\hspace*{6pt}\hspace*{6pt}{<\textbf{title}>}The Vow of the Peacock{</\textbf{title}>}\mbox{}\newline 
\hspace*{6pt}\hspace*{6pt}\hspace*{6pt}{</\textbf{ref}>}\mbox{}\newline 
\hspace*{6pt}\hspace*{6pt}{</\textbf{bibl}>}\mbox{}\newline 
\hspace*{6pt}\hspace*{6pt}{<\textbf{bibl}>}\mbox{}\newline 
\hspace*{6pt}\hspace*{6pt}\hspace*{6pt}{<\textbf{author}>}\mbox{}\newline 
\hspace*{6pt}\hspace*{6pt}\hspace*{6pt}\hspace*{6pt}{<\textbf{name}>}Compton, Margaret Clephane{</\textbf{name}>}\mbox{}\newline 
\hspace*{6pt}\hspace*{6pt}\hspace*{6pt}{</\textbf{author}>}\mbox{}\newline 
\hspace*{6pt}\hspace*{6pt}\hspace*{6pt}{<\textbf{ref}\hspace*{6pt}{target}="{NortMIrene.sgm}">}\mbox{}\newline 
\hspace*{6pt}\hspace*{6pt}\hspace*{6pt}\hspace*{6pt}{<\textbf{title}>}Irene, a Poem in Six Cantos{</\textbf{title}>}\mbox{}\newline 
\hspace*{6pt}\hspace*{6pt}\hspace*{6pt}{</\textbf{ref}>}\mbox{}\newline 
\hspace*{6pt}\hspace*{6pt}{</\textbf{bibl}>}\mbox{}\newline 
\hspace*{6pt}\hspace*{6pt}{<\textbf{bibl}>}\mbox{}\newline 
\hspace*{6pt}\hspace*{6pt}\hspace*{6pt}{<\textbf{author}>}\mbox{}\newline 
\hspace*{6pt}\hspace*{6pt}\hspace*{6pt}\hspace*{6pt}{<\textbf{name}>}Taylor, Jane{</\textbf{name}>}\mbox{}\newline 
\hspace*{6pt}\hspace*{6pt}\hspace*{6pt}{</\textbf{author}>}\mbox{}\newline 
\hspace*{6pt}\hspace*{6pt}\hspace*{6pt}{<\textbf{ref}\hspace*{6pt}{target}="{TaylJEssay.sgm}">}\mbox{}\newline 
\hspace*{6pt}\hspace*{6pt}\hspace*{6pt}\hspace*{6pt}{<\textbf{title}>}Essays in Rhyme on Morals and Manners{</\textbf{title}>}\mbox{}\newline 
\hspace*{6pt}\hspace*{6pt}\hspace*{6pt}{</\textbf{ref}>}\mbox{}\newline 
\hspace*{6pt}\hspace*{6pt}{</\textbf{bibl}>}\mbox{}\newline 
\hspace*{6pt}{</\textbf{listBibl}>}\mbox{}\newline 
{</\textbf{div}>}
\end{reflist}  
    \item[@xml:space]
  signale que les applications doivent préserver l'espace blanc
\begin{reflist}
    \item[{Statut}]
  Optionel
    \item[{Type de données}]
  \hyperref[TEI.teidata.enumerated]{teidata.enumerated}
    \item[{Les valeurs autorisées sont:}]
  \begin{description}

\item[{default}]signals that the application's default white-space processing modes are acceptable
\item[{preserve}]indicates the intent that applications preserve all white space
\end{description} 
\end{reflist}  
\end{sansreflist}  
\end{reflist}  
\begin{reflist}
\item[]\begin{specHead}{TEI.att.global.analytic}{att.global.analytic}\index{att.global.analytic (attribute class)|oddindex}\index{ana=@ana!att.global.analytic (attribute class)|oddindex} fournit des attributs globaux complémentaires pour associer des analyses ou des interprétations spécifiques avec des portions de texte appropriées.\end{specHead} 
    \item[{Module}]
  analysis
    \item[{Membres}]
  \hyperref[TEI.att.global]{att.global}[\hyperref[TEI.TEI]{TEI} \hyperref[TEI.ab]{ab} \hyperref[TEI.abbr]{abbr} \hyperref[TEI.abstract]{abstract} \hyperref[TEI.accMat]{accMat} \hyperref[TEI.acquisition]{acquisition} \hyperref[TEI.add]{add} \hyperref[TEI.addName]{addName} \hyperref[TEI.addSpan]{addSpan} \hyperref[TEI.additional]{additional} \hyperref[TEI.additions]{additions} \hyperref[TEI.addrLine]{addrLine} \hyperref[TEI.address]{address} \hyperref[TEI.adminInfo]{adminInfo} \hyperref[TEI.affiliation]{affiliation} \hyperref[TEI.alt]{alt} \hyperref[TEI.altGrp]{altGrp} \hyperref[TEI.altIdentifier]{altIdentifier} \hyperref[TEI.am]{am} \hyperref[TEI.analytic]{analytic} \hyperref[TEI.anchor]{anchor} \hyperref[TEI.annotationBlock]{annotationBlock} \hyperref[TEI.appInfo]{appInfo} \hyperref[TEI.application]{application} \hyperref[TEI.author]{author} \hyperref[TEI.authority]{authority} \hyperref[TEI.availability]{availability} \hyperref[TEI.back]{back} \hyperref[TEI.bibl]{bibl} \hyperref[TEI.biblFull]{biblFull} \hyperref[TEI.biblScope]{biblScope} \hyperref[TEI.biblStruct]{biblStruct} \hyperref[TEI.bicond]{bicond} \hyperref[TEI.binary]{binary} \hyperref[TEI.binaryObject]{binaryObject} \hyperref[TEI.binding]{binding} \hyperref[TEI.bindingDesc]{bindingDesc} \hyperref[TEI.body]{body} \hyperref[TEI.c]{c} \hyperref[TEI.catchwords]{catchwords} \hyperref[TEI.category]{category} \hyperref[TEI.cb]{cb} \hyperref[TEI.cell]{cell} \hyperref[TEI.change]{change} \hyperref[TEI.choice]{choice} \hyperref[TEI.cit]{cit} \hyperref[TEI.citedRange]{citedRange} \hyperref[TEI.cl]{cl} \hyperref[TEI.classCode]{classCode} \hyperref[TEI.classDecl]{classDecl} \hyperref[TEI.collation]{collation} \hyperref[TEI.collection]{collection} \hyperref[TEI.colophon]{colophon} \hyperref[TEI.cond]{cond} \hyperref[TEI.condition]{condition} \hyperref[TEI.corr]{corr} \hyperref[TEI.correction]{correction} \hyperref[TEI.country]{country} \hyperref[TEI.creation]{creation} \hyperref[TEI.custEvent]{custEvent} \hyperref[TEI.custodialHist]{custodialHist} \hyperref[TEI.damage]{damage} \hyperref[TEI.damageSpan]{damageSpan} \hyperref[TEI.date]{date} \hyperref[TEI.decoDesc]{decoDesc} \hyperref[TEI.decoNote]{decoNote} \hyperref[TEI.default]{default} \hyperref[TEI.del]{del} \hyperref[TEI.delSpan]{delSpan} \hyperref[TEI.depth]{depth} \hyperref[TEI.desc]{desc} \hyperref[TEI.dim]{dim} \hyperref[TEI.dimensions]{dimensions} \hyperref[TEI.distinct]{distinct} \hyperref[TEI.distributor]{distributor} \hyperref[TEI.div]{div} \hyperref[TEI.divGen]{divGen} \hyperref[TEI.docAuthor]{docAuthor} \hyperref[TEI.docDate]{docDate} \hyperref[TEI.docEdition]{docEdition} \hyperref[TEI.docTitle]{docTitle} \hyperref[TEI.edition]{edition} \hyperref[TEI.editionStmt]{editionStmt} \hyperref[TEI.editor]{editor} \hyperref[TEI.email]{email} \hyperref[TEI.emph]{emph} \hyperref[TEI.encodingDesc]{encodingDesc} \hyperref[TEI.event]{event} \hyperref[TEI.ex]{ex} \hyperref[TEI.expan]{expan} \hyperref[TEI.explicit]{explicit} \hyperref[TEI.extent]{extent} \hyperref[TEI.f]{f} \hyperref[TEI.fDecl]{fDecl} \hyperref[TEI.fDescr]{fDescr} \hyperref[TEI.fLib]{fLib} \hyperref[TEI.facsimile]{facsimile} \hyperref[TEI.figDesc]{figDesc} \hyperref[TEI.figure]{figure} \hyperref[TEI.fileDesc]{fileDesc} \hyperref[TEI.filiation]{filiation} \hyperref[TEI.finalRubric]{finalRubric} \hyperref[TEI.floatingText]{floatingText} \hyperref[TEI.foliation]{foliation} \hyperref[TEI.foreign]{foreign} \hyperref[TEI.forename]{forename} \hyperref[TEI.formula]{formula} \hyperref[TEI.front]{front} \hyperref[TEI.fs]{fs} \hyperref[TEI.fsConstraints]{fsConstraints} \hyperref[TEI.fsDecl]{fsDecl} \hyperref[TEI.fsDescr]{fsDescr} \hyperref[TEI.fsdDecl]{fsdDecl} \hyperref[TEI.fsdLink]{fsdLink} \hyperref[TEI.funder]{funder} \hyperref[TEI.fvLib]{fvLib} \hyperref[TEI.fw]{fw} \hyperref[TEI.gap]{gap} \hyperref[TEI.gb]{gb} \hyperref[TEI.genName]{genName} \hyperref[TEI.geogName]{geogName} \hyperref[TEI.gloss]{gloss} \hyperref[TEI.graphic]{graphic} \hyperref[TEI.group]{group} \hyperref[TEI.handDesc]{handDesc} \hyperref[TEI.handNotes]{handNotes} \hyperref[TEI.handShift]{handShift} \hyperref[TEI.head]{head} \hyperref[TEI.headItem]{headItem} \hyperref[TEI.headLabel]{headLabel} \hyperref[TEI.height]{height} \hyperref[TEI.heraldry]{heraldry} \hyperref[TEI.hi]{hi} \hyperref[TEI.history]{history} \hyperref[TEI.idno]{idno} \hyperref[TEI.if]{if} \hyperref[TEI.iff]{iff} \hyperref[TEI.imprint]{imprint} \hyperref[TEI.incipit]{incipit} \hyperref[TEI.index]{index} \hyperref[TEI.institution]{institution} \hyperref[TEI.interp]{interp} \hyperref[TEI.interpGrp]{interpGrp} \hyperref[TEI.item]{item} \hyperref[TEI.join]{join} \hyperref[TEI.joinGrp]{joinGrp} \hyperref[TEI.keywords]{keywords} \hyperref[TEI.l]{l} \hyperref[TEI.label]{label} \hyperref[TEI.langUsage]{langUsage} \hyperref[TEI.language]{language} \hyperref[TEI.layout]{layout} \hyperref[TEI.layoutDesc]{layoutDesc} \hyperref[TEI.lb]{lb} \hyperref[TEI.lg]{lg} \hyperref[TEI.licence]{licence} \hyperref[TEI.line]{line} \hyperref[TEI.link]{link} \hyperref[TEI.linkGrp]{linkGrp} \hyperref[TEI.list]{list} \hyperref[TEI.listAnnotation]{listAnnotation} \hyperref[TEI.listBibl]{listBibl} \hyperref[TEI.listOrg]{listOrg} \hyperref[TEI.listPlace]{listPlace} \hyperref[TEI.listTranspose]{listTranspose} \hyperref[TEI.location]{location} \hyperref[TEI.locus]{locus} \hyperref[TEI.locusGrp]{locusGrp} \hyperref[TEI.m]{m} \hyperref[TEI.material]{material} \hyperref[TEI.measure]{measure} \hyperref[TEI.measureGrp]{measureGrp} \hyperref[TEI.media]{media} \hyperref[TEI.meeting]{meeting} \hyperref[TEI.mentioned]{mentioned} \hyperref[TEI.metamark]{metamark} \hyperref[TEI.milestone]{milestone} \hyperref[TEI.mod]{mod} \hyperref[TEI.monogr]{monogr} \hyperref[TEI.msContents]{msContents} \hyperref[TEI.msDesc]{msDesc} \hyperref[TEI.msFrag]{msFrag} \hyperref[TEI.msIdentifier]{msIdentifier} \hyperref[TEI.msItem]{msItem} \hyperref[TEI.msItemStruct]{msItemStruct} \hyperref[TEI.msName]{msName} \hyperref[TEI.msPart]{msPart} \hyperref[TEI.musicNotation]{musicNotation} \hyperref[TEI.name]{name} \hyperref[TEI.nameLink]{nameLink} \hyperref[TEI.namespace]{namespace} \hyperref[TEI.notatedMusic]{notatedMusic} \hyperref[TEI.note]{note} \hyperref[TEI.notesStmt]{notesStmt} \hyperref[TEI.num]{num} \hyperref[TEI.numeric]{numeric} \hyperref[TEI.objectDesc]{objectDesc} \hyperref[TEI.objectType]{objectType} \hyperref[TEI.org]{org} \hyperref[TEI.orgName]{orgName} \hyperref[TEI.orig]{orig} \hyperref[TEI.origDate]{origDate} \hyperref[TEI.origPlace]{origPlace} \hyperref[TEI.origin]{origin} \hyperref[TEI.p]{p} \hyperref[TEI.pb]{pb} \hyperref[TEI.pc]{pc} \hyperref[TEI.persName]{persName} \hyperref[TEI.person]{person} \hyperref[TEI.personGrp]{personGrp} \hyperref[TEI.persona]{persona} \hyperref[TEI.phr]{phr} \hyperref[TEI.physDesc]{physDesc} \hyperref[TEI.place]{place} \hyperref[TEI.placeName]{placeName} \hyperref[TEI.postBox]{postBox} \hyperref[TEI.postCode]{postCode} \hyperref[TEI.profileDesc]{profileDesc} \hyperref[TEI.provenance]{provenance} \hyperref[TEI.ptr]{ptr} \hyperref[TEI.pubPlace]{pubPlace} \hyperref[TEI.publicationStmt]{publicationStmt} \hyperref[TEI.publisher]{publisher} \hyperref[TEI.q]{q} \hyperref[TEI.quote]{quote} \hyperref[TEI.recordHist]{recordHist} \hyperref[TEI.redo]{redo} \hyperref[TEI.ref]{ref} \hyperref[TEI.reg]{reg} \hyperref[TEI.region]{region} \hyperref[TEI.relatedItem]{relatedItem} \hyperref[TEI.rendition]{rendition} \hyperref[TEI.repository]{repository} \hyperref[TEI.resp]{resp} \hyperref[TEI.respStmt]{respStmt} \hyperref[TEI.restore]{restore} \hyperref[TEI.retrace]{retrace} \hyperref[TEI.revisionDesc]{revisionDesc} \hyperref[TEI.roleName]{roleName} \hyperref[TEI.row]{row} \hyperref[TEI.rs]{rs} \hyperref[TEI.rubric]{rubric} \hyperref[TEI.s]{s} \hyperref[TEI.said]{said} \hyperref[TEI.schemaRef]{schemaRef} \hyperref[TEI.scriptDesc]{scriptDesc} \hyperref[TEI.seal]{seal} \hyperref[TEI.sealDesc]{sealDesc} \hyperref[TEI.secFol]{secFol} \hyperref[TEI.secl]{secl} \hyperref[TEI.seg]{seg} \hyperref[TEI.series]{series} \hyperref[TEI.seriesStmt]{seriesStmt} \hyperref[TEI.settlement]{settlement} \hyperref[TEI.sic]{sic} \hyperref[TEI.signatures]{signatures} \hyperref[TEI.soCalled]{soCalled} \hyperref[TEI.source]{source} \hyperref[TEI.sourceDesc]{sourceDesc} \hyperref[TEI.sourceDoc]{sourceDoc} \hyperref[TEI.sp]{sp} \hyperref[TEI.span]{span} \hyperref[TEI.spanGrp]{spanGrp} \hyperref[TEI.speaker]{speaker} \hyperref[TEI.stage]{stage} \hyperref[TEI.stamp]{stamp} \hyperref[TEI.standOff]{standOff} \hyperref[TEI.state]{state} \hyperref[TEI.street]{street} \hyperref[TEI.string]{string} \hyperref[TEI.subst]{subst} \hyperref[TEI.substJoin]{substJoin} \hyperref[TEI.summary]{summary} \hyperref[TEI.supplied]{supplied} \hyperref[TEI.support]{support} \hyperref[TEI.supportDesc]{supportDesc} \hyperref[TEI.surface]{surface} \hyperref[TEI.surfaceGrp]{surfaceGrp} \hyperref[TEI.surname]{surname} \hyperref[TEI.surplus]{surplus} \hyperref[TEI.surrogates]{surrogates} \hyperref[TEI.symbol]{symbol} \hyperref[TEI.table]{table} \hyperref[TEI.taxonomy]{taxonomy} \hyperref[TEI.teiCorpus]{teiCorpus} \hyperref[TEI.teiHeader]{teiHeader} \hyperref[TEI.term]{term} \hyperref[TEI.text]{text} \hyperref[TEI.textClass]{textClass} \hyperref[TEI.textLang]{textLang} \hyperref[TEI.then]{then} \hyperref[TEI.time]{time} \hyperref[TEI.timeline]{timeline} \hyperref[TEI.title]{title} \hyperref[TEI.titlePage]{titlePage} \hyperref[TEI.titlePart]{titlePart} \hyperref[TEI.titleStmt]{titleStmt} \hyperref[TEI.transpose]{transpose} \hyperref[TEI.typeDesc]{typeDesc} \hyperref[TEI.typeNote]{typeNote} \hyperref[TEI.unclear]{unclear} \hyperref[TEI.undo]{undo} \hyperref[TEI.vAlt]{vAlt} \hyperref[TEI.vColl]{vColl} \hyperref[TEI.vDefault]{vDefault} \hyperref[TEI.vLabel]{vLabel} \hyperref[TEI.vMerge]{vMerge} \hyperref[TEI.vNot]{vNot} \hyperref[TEI.vRange]{vRange} \hyperref[TEI.w]{w} \hyperref[TEI.watermark]{watermark} \hyperref[TEI.when]{when} \hyperref[TEI.width]{width} \hyperref[TEI.zone]{zone}]
    \item[{Attributs}]
  Attributs\hfil\\[-10pt]\begin{sansreflist}
    \item[@ana]
  (analyse) indique un ou plusieurs éléments contenant des interprétations de l'élément qui porte l'attribut {\itshape ana}.
\begin{reflist}
    \item[{Statut}]
  Optionel
    \item[{Type de données}]
  1–∞ occurrences de \hyperref[TEI.teidata.pointer]{teidata.pointer} séparé par un espace
    \item[{Note}]
  \par
Quand on donne de multiples valeurs, celles-ci peuvent refléter, soit des interprétations multiples et divergentes d'un texte ambigu soit des interprétations multiples et compatibles du même passage dans différents contextes.
\end{reflist}  
\end{sansreflist}  
\end{reflist}  
\begin{reflist}
\item[]\begin{specHead}{TEI.att.global.change}{att.global.change}\index{att.global.change (attribute class)|oddindex}\index{change=@change!att.global.change (attribute class)|oddindex} supplies the {\itshape change} attribute, allowing its member elements to specify one or more states or revision campaigns with which they are associated.\end{specHead} 
    \item[{Module}]
  transcr
    \item[{Membres}]
  \hyperref[TEI.att.global]{att.global}[\hyperref[TEI.TEI]{TEI} \hyperref[TEI.ab]{ab} \hyperref[TEI.abbr]{abbr} \hyperref[TEI.abstract]{abstract} \hyperref[TEI.accMat]{accMat} \hyperref[TEI.acquisition]{acquisition} \hyperref[TEI.add]{add} \hyperref[TEI.addName]{addName} \hyperref[TEI.addSpan]{addSpan} \hyperref[TEI.additional]{additional} \hyperref[TEI.additions]{additions} \hyperref[TEI.addrLine]{addrLine} \hyperref[TEI.address]{address} \hyperref[TEI.adminInfo]{adminInfo} \hyperref[TEI.affiliation]{affiliation} \hyperref[TEI.alt]{alt} \hyperref[TEI.altGrp]{altGrp} \hyperref[TEI.altIdentifier]{altIdentifier} \hyperref[TEI.am]{am} \hyperref[TEI.analytic]{analytic} \hyperref[TEI.anchor]{anchor} \hyperref[TEI.annotationBlock]{annotationBlock} \hyperref[TEI.appInfo]{appInfo} \hyperref[TEI.application]{application} \hyperref[TEI.author]{author} \hyperref[TEI.authority]{authority} \hyperref[TEI.availability]{availability} \hyperref[TEI.back]{back} \hyperref[TEI.bibl]{bibl} \hyperref[TEI.biblFull]{biblFull} \hyperref[TEI.biblScope]{biblScope} \hyperref[TEI.biblStruct]{biblStruct} \hyperref[TEI.bicond]{bicond} \hyperref[TEI.binary]{binary} \hyperref[TEI.binaryObject]{binaryObject} \hyperref[TEI.binding]{binding} \hyperref[TEI.bindingDesc]{bindingDesc} \hyperref[TEI.body]{body} \hyperref[TEI.c]{c} \hyperref[TEI.catchwords]{catchwords} \hyperref[TEI.category]{category} \hyperref[TEI.cb]{cb} \hyperref[TEI.cell]{cell} \hyperref[TEI.change]{change} \hyperref[TEI.choice]{choice} \hyperref[TEI.cit]{cit} \hyperref[TEI.citedRange]{citedRange} \hyperref[TEI.cl]{cl} \hyperref[TEI.classCode]{classCode} \hyperref[TEI.classDecl]{classDecl} \hyperref[TEI.collation]{collation} \hyperref[TEI.collection]{collection} \hyperref[TEI.colophon]{colophon} \hyperref[TEI.cond]{cond} \hyperref[TEI.condition]{condition} \hyperref[TEI.corr]{corr} \hyperref[TEI.correction]{correction} \hyperref[TEI.country]{country} \hyperref[TEI.creation]{creation} \hyperref[TEI.custEvent]{custEvent} \hyperref[TEI.custodialHist]{custodialHist} \hyperref[TEI.damage]{damage} \hyperref[TEI.damageSpan]{damageSpan} \hyperref[TEI.date]{date} \hyperref[TEI.decoDesc]{decoDesc} \hyperref[TEI.decoNote]{decoNote} \hyperref[TEI.default]{default} \hyperref[TEI.del]{del} \hyperref[TEI.delSpan]{delSpan} \hyperref[TEI.depth]{depth} \hyperref[TEI.desc]{desc} \hyperref[TEI.dim]{dim} \hyperref[TEI.dimensions]{dimensions} \hyperref[TEI.distinct]{distinct} \hyperref[TEI.distributor]{distributor} \hyperref[TEI.div]{div} \hyperref[TEI.divGen]{divGen} \hyperref[TEI.docAuthor]{docAuthor} \hyperref[TEI.docDate]{docDate} \hyperref[TEI.docEdition]{docEdition} \hyperref[TEI.docTitle]{docTitle} \hyperref[TEI.edition]{edition} \hyperref[TEI.editionStmt]{editionStmt} \hyperref[TEI.editor]{editor} \hyperref[TEI.email]{email} \hyperref[TEI.emph]{emph} \hyperref[TEI.encodingDesc]{encodingDesc} \hyperref[TEI.event]{event} \hyperref[TEI.ex]{ex} \hyperref[TEI.expan]{expan} \hyperref[TEI.explicit]{explicit} \hyperref[TEI.extent]{extent} \hyperref[TEI.f]{f} \hyperref[TEI.fDecl]{fDecl} \hyperref[TEI.fDescr]{fDescr} \hyperref[TEI.fLib]{fLib} \hyperref[TEI.facsimile]{facsimile} \hyperref[TEI.figDesc]{figDesc} \hyperref[TEI.figure]{figure} \hyperref[TEI.fileDesc]{fileDesc} \hyperref[TEI.filiation]{filiation} \hyperref[TEI.finalRubric]{finalRubric} \hyperref[TEI.floatingText]{floatingText} \hyperref[TEI.foliation]{foliation} \hyperref[TEI.foreign]{foreign} \hyperref[TEI.forename]{forename} \hyperref[TEI.formula]{formula} \hyperref[TEI.front]{front} \hyperref[TEI.fs]{fs} \hyperref[TEI.fsConstraints]{fsConstraints} \hyperref[TEI.fsDecl]{fsDecl} \hyperref[TEI.fsDescr]{fsDescr} \hyperref[TEI.fsdDecl]{fsdDecl} \hyperref[TEI.fsdLink]{fsdLink} \hyperref[TEI.funder]{funder} \hyperref[TEI.fvLib]{fvLib} \hyperref[TEI.fw]{fw} \hyperref[TEI.gap]{gap} \hyperref[TEI.gb]{gb} \hyperref[TEI.genName]{genName} \hyperref[TEI.geogName]{geogName} \hyperref[TEI.gloss]{gloss} \hyperref[TEI.graphic]{graphic} \hyperref[TEI.group]{group} \hyperref[TEI.handDesc]{handDesc} \hyperref[TEI.handNotes]{handNotes} \hyperref[TEI.handShift]{handShift} \hyperref[TEI.head]{head} \hyperref[TEI.headItem]{headItem} \hyperref[TEI.headLabel]{headLabel} \hyperref[TEI.height]{height} \hyperref[TEI.heraldry]{heraldry} \hyperref[TEI.hi]{hi} \hyperref[TEI.history]{history} \hyperref[TEI.idno]{idno} \hyperref[TEI.if]{if} \hyperref[TEI.iff]{iff} \hyperref[TEI.imprint]{imprint} \hyperref[TEI.incipit]{incipit} \hyperref[TEI.index]{index} \hyperref[TEI.institution]{institution} \hyperref[TEI.interp]{interp} \hyperref[TEI.interpGrp]{interpGrp} \hyperref[TEI.item]{item} \hyperref[TEI.join]{join} \hyperref[TEI.joinGrp]{joinGrp} \hyperref[TEI.keywords]{keywords} \hyperref[TEI.l]{l} \hyperref[TEI.label]{label} \hyperref[TEI.langUsage]{langUsage} \hyperref[TEI.language]{language} \hyperref[TEI.layout]{layout} \hyperref[TEI.layoutDesc]{layoutDesc} \hyperref[TEI.lb]{lb} \hyperref[TEI.lg]{lg} \hyperref[TEI.licence]{licence} \hyperref[TEI.line]{line} \hyperref[TEI.link]{link} \hyperref[TEI.linkGrp]{linkGrp} \hyperref[TEI.list]{list} \hyperref[TEI.listAnnotation]{listAnnotation} \hyperref[TEI.listBibl]{listBibl} \hyperref[TEI.listOrg]{listOrg} \hyperref[TEI.listPlace]{listPlace} \hyperref[TEI.listTranspose]{listTranspose} \hyperref[TEI.location]{location} \hyperref[TEI.locus]{locus} \hyperref[TEI.locusGrp]{locusGrp} \hyperref[TEI.m]{m} \hyperref[TEI.material]{material} \hyperref[TEI.measure]{measure} \hyperref[TEI.measureGrp]{measureGrp} \hyperref[TEI.media]{media} \hyperref[TEI.meeting]{meeting} \hyperref[TEI.mentioned]{mentioned} \hyperref[TEI.metamark]{metamark} \hyperref[TEI.milestone]{milestone} \hyperref[TEI.mod]{mod} \hyperref[TEI.monogr]{monogr} \hyperref[TEI.msContents]{msContents} \hyperref[TEI.msDesc]{msDesc} \hyperref[TEI.msFrag]{msFrag} \hyperref[TEI.msIdentifier]{msIdentifier} \hyperref[TEI.msItem]{msItem} \hyperref[TEI.msItemStruct]{msItemStruct} \hyperref[TEI.msName]{msName} \hyperref[TEI.msPart]{msPart} \hyperref[TEI.musicNotation]{musicNotation} \hyperref[TEI.name]{name} \hyperref[TEI.nameLink]{nameLink} \hyperref[TEI.namespace]{namespace} \hyperref[TEI.notatedMusic]{notatedMusic} \hyperref[TEI.note]{note} \hyperref[TEI.notesStmt]{notesStmt} \hyperref[TEI.num]{num} \hyperref[TEI.numeric]{numeric} \hyperref[TEI.objectDesc]{objectDesc} \hyperref[TEI.objectType]{objectType} \hyperref[TEI.org]{org} \hyperref[TEI.orgName]{orgName} \hyperref[TEI.orig]{orig} \hyperref[TEI.origDate]{origDate} \hyperref[TEI.origPlace]{origPlace} \hyperref[TEI.origin]{origin} \hyperref[TEI.p]{p} \hyperref[TEI.pb]{pb} \hyperref[TEI.pc]{pc} \hyperref[TEI.persName]{persName} \hyperref[TEI.person]{person} \hyperref[TEI.personGrp]{personGrp} \hyperref[TEI.persona]{persona} \hyperref[TEI.phr]{phr} \hyperref[TEI.physDesc]{physDesc} \hyperref[TEI.place]{place} \hyperref[TEI.placeName]{placeName} \hyperref[TEI.postBox]{postBox} \hyperref[TEI.postCode]{postCode} \hyperref[TEI.profileDesc]{profileDesc} \hyperref[TEI.provenance]{provenance} \hyperref[TEI.ptr]{ptr} \hyperref[TEI.pubPlace]{pubPlace} \hyperref[TEI.publicationStmt]{publicationStmt} \hyperref[TEI.publisher]{publisher} \hyperref[TEI.q]{q} \hyperref[TEI.quote]{quote} \hyperref[TEI.recordHist]{recordHist} \hyperref[TEI.redo]{redo} \hyperref[TEI.ref]{ref} \hyperref[TEI.reg]{reg} \hyperref[TEI.region]{region} \hyperref[TEI.relatedItem]{relatedItem} \hyperref[TEI.rendition]{rendition} \hyperref[TEI.repository]{repository} \hyperref[TEI.resp]{resp} \hyperref[TEI.respStmt]{respStmt} \hyperref[TEI.restore]{restore} \hyperref[TEI.retrace]{retrace} \hyperref[TEI.revisionDesc]{revisionDesc} \hyperref[TEI.roleName]{roleName} \hyperref[TEI.row]{row} \hyperref[TEI.rs]{rs} \hyperref[TEI.rubric]{rubric} \hyperref[TEI.s]{s} \hyperref[TEI.said]{said} \hyperref[TEI.schemaRef]{schemaRef} \hyperref[TEI.scriptDesc]{scriptDesc} \hyperref[TEI.seal]{seal} \hyperref[TEI.sealDesc]{sealDesc} \hyperref[TEI.secFol]{secFol} \hyperref[TEI.secl]{secl} \hyperref[TEI.seg]{seg} \hyperref[TEI.series]{series} \hyperref[TEI.seriesStmt]{seriesStmt} \hyperref[TEI.settlement]{settlement} \hyperref[TEI.sic]{sic} \hyperref[TEI.signatures]{signatures} \hyperref[TEI.soCalled]{soCalled} \hyperref[TEI.source]{source} \hyperref[TEI.sourceDesc]{sourceDesc} \hyperref[TEI.sourceDoc]{sourceDoc} \hyperref[TEI.sp]{sp} \hyperref[TEI.span]{span} \hyperref[TEI.spanGrp]{spanGrp} \hyperref[TEI.speaker]{speaker} \hyperref[TEI.stage]{stage} \hyperref[TEI.stamp]{stamp} \hyperref[TEI.standOff]{standOff} \hyperref[TEI.state]{state} \hyperref[TEI.street]{street} \hyperref[TEI.string]{string} \hyperref[TEI.subst]{subst} \hyperref[TEI.substJoin]{substJoin} \hyperref[TEI.summary]{summary} \hyperref[TEI.supplied]{supplied} \hyperref[TEI.support]{support} \hyperref[TEI.supportDesc]{supportDesc} \hyperref[TEI.surface]{surface} \hyperref[TEI.surfaceGrp]{surfaceGrp} \hyperref[TEI.surname]{surname} \hyperref[TEI.surplus]{surplus} \hyperref[TEI.surrogates]{surrogates} \hyperref[TEI.symbol]{symbol} \hyperref[TEI.table]{table} \hyperref[TEI.taxonomy]{taxonomy} \hyperref[TEI.teiCorpus]{teiCorpus} \hyperref[TEI.teiHeader]{teiHeader} \hyperref[TEI.term]{term} \hyperref[TEI.text]{text} \hyperref[TEI.textClass]{textClass} \hyperref[TEI.textLang]{textLang} \hyperref[TEI.then]{then} \hyperref[TEI.time]{time} \hyperref[TEI.timeline]{timeline} \hyperref[TEI.title]{title} \hyperref[TEI.titlePage]{titlePage} \hyperref[TEI.titlePart]{titlePart} \hyperref[TEI.titleStmt]{titleStmt} \hyperref[TEI.transpose]{transpose} \hyperref[TEI.typeDesc]{typeDesc} \hyperref[TEI.typeNote]{typeNote} \hyperref[TEI.unclear]{unclear} \hyperref[TEI.undo]{undo} \hyperref[TEI.vAlt]{vAlt} \hyperref[TEI.vColl]{vColl} \hyperref[TEI.vDefault]{vDefault} \hyperref[TEI.vLabel]{vLabel} \hyperref[TEI.vMerge]{vMerge} \hyperref[TEI.vNot]{vNot} \hyperref[TEI.vRange]{vRange} \hyperref[TEI.w]{w} \hyperref[TEI.watermark]{watermark} \hyperref[TEI.when]{when} \hyperref[TEI.width]{width} \hyperref[TEI.zone]{zone}]
    \item[{Attributs}]
  Attributs\hfil\\[-10pt]\begin{sansreflist}
    \item[@change]
  points to one or more \hyperref[TEI.change]{<change>} elements documenting a state or revision campaign to which the element bearing this attribute and its children have been assigned by the encoder.
\begin{reflist}
    \item[{Statut}]
  Optionel
    \item[{Type de données}]
  1–∞ occurrences de \hyperref[TEI.teidata.pointer]{teidata.pointer} séparé par un espace
\end{reflist}  
\end{sansreflist}  
\end{reflist}  
\begin{reflist}
\item[]\begin{specHead}{TEI.att.global.facs}{att.global.facs}\index{att.global.facs (attribute class)|oddindex}\index{facs=@facs!att.global.facs (attribute class)|oddindex} attributs utilisables pour les éléments correspondant à tout ou partie d'une image, parce qu'ils contiennent une représentation alternative de cette image, généralement mais pas nécessairement, une transcription.\end{specHead} 
    \item[{Module}]
  transcr
    \item[{Membres}]
  \hyperref[TEI.att.global]{att.global}[\hyperref[TEI.TEI]{TEI} \hyperref[TEI.ab]{ab} \hyperref[TEI.abbr]{abbr} \hyperref[TEI.abstract]{abstract} \hyperref[TEI.accMat]{accMat} \hyperref[TEI.acquisition]{acquisition} \hyperref[TEI.add]{add} \hyperref[TEI.addName]{addName} \hyperref[TEI.addSpan]{addSpan} \hyperref[TEI.additional]{additional} \hyperref[TEI.additions]{additions} \hyperref[TEI.addrLine]{addrLine} \hyperref[TEI.address]{address} \hyperref[TEI.adminInfo]{adminInfo} \hyperref[TEI.affiliation]{affiliation} \hyperref[TEI.alt]{alt} \hyperref[TEI.altGrp]{altGrp} \hyperref[TEI.altIdentifier]{altIdentifier} \hyperref[TEI.am]{am} \hyperref[TEI.analytic]{analytic} \hyperref[TEI.anchor]{anchor} \hyperref[TEI.annotationBlock]{annotationBlock} \hyperref[TEI.appInfo]{appInfo} \hyperref[TEI.application]{application} \hyperref[TEI.author]{author} \hyperref[TEI.authority]{authority} \hyperref[TEI.availability]{availability} \hyperref[TEI.back]{back} \hyperref[TEI.bibl]{bibl} \hyperref[TEI.biblFull]{biblFull} \hyperref[TEI.biblScope]{biblScope} \hyperref[TEI.biblStruct]{biblStruct} \hyperref[TEI.bicond]{bicond} \hyperref[TEI.binary]{binary} \hyperref[TEI.binaryObject]{binaryObject} \hyperref[TEI.binding]{binding} \hyperref[TEI.bindingDesc]{bindingDesc} \hyperref[TEI.body]{body} \hyperref[TEI.c]{c} \hyperref[TEI.catchwords]{catchwords} \hyperref[TEI.category]{category} \hyperref[TEI.cb]{cb} \hyperref[TEI.cell]{cell} \hyperref[TEI.change]{change} \hyperref[TEI.choice]{choice} \hyperref[TEI.cit]{cit} \hyperref[TEI.citedRange]{citedRange} \hyperref[TEI.cl]{cl} \hyperref[TEI.classCode]{classCode} \hyperref[TEI.classDecl]{classDecl} \hyperref[TEI.collation]{collation} \hyperref[TEI.collection]{collection} \hyperref[TEI.colophon]{colophon} \hyperref[TEI.cond]{cond} \hyperref[TEI.condition]{condition} \hyperref[TEI.corr]{corr} \hyperref[TEI.correction]{correction} \hyperref[TEI.country]{country} \hyperref[TEI.creation]{creation} \hyperref[TEI.custEvent]{custEvent} \hyperref[TEI.custodialHist]{custodialHist} \hyperref[TEI.damage]{damage} \hyperref[TEI.damageSpan]{damageSpan} \hyperref[TEI.date]{date} \hyperref[TEI.decoDesc]{decoDesc} \hyperref[TEI.decoNote]{decoNote} \hyperref[TEI.default]{default} \hyperref[TEI.del]{del} \hyperref[TEI.delSpan]{delSpan} \hyperref[TEI.depth]{depth} \hyperref[TEI.desc]{desc} \hyperref[TEI.dim]{dim} \hyperref[TEI.dimensions]{dimensions} \hyperref[TEI.distinct]{distinct} \hyperref[TEI.distributor]{distributor} \hyperref[TEI.div]{div} \hyperref[TEI.divGen]{divGen} \hyperref[TEI.docAuthor]{docAuthor} \hyperref[TEI.docDate]{docDate} \hyperref[TEI.docEdition]{docEdition} \hyperref[TEI.docTitle]{docTitle} \hyperref[TEI.edition]{edition} \hyperref[TEI.editionStmt]{editionStmt} \hyperref[TEI.editor]{editor} \hyperref[TEI.email]{email} \hyperref[TEI.emph]{emph} \hyperref[TEI.encodingDesc]{encodingDesc} \hyperref[TEI.event]{event} \hyperref[TEI.ex]{ex} \hyperref[TEI.expan]{expan} \hyperref[TEI.explicit]{explicit} \hyperref[TEI.extent]{extent} \hyperref[TEI.f]{f} \hyperref[TEI.fDecl]{fDecl} \hyperref[TEI.fDescr]{fDescr} \hyperref[TEI.fLib]{fLib} \hyperref[TEI.facsimile]{facsimile} \hyperref[TEI.figDesc]{figDesc} \hyperref[TEI.figure]{figure} \hyperref[TEI.fileDesc]{fileDesc} \hyperref[TEI.filiation]{filiation} \hyperref[TEI.finalRubric]{finalRubric} \hyperref[TEI.floatingText]{floatingText} \hyperref[TEI.foliation]{foliation} \hyperref[TEI.foreign]{foreign} \hyperref[TEI.forename]{forename} \hyperref[TEI.formula]{formula} \hyperref[TEI.front]{front} \hyperref[TEI.fs]{fs} \hyperref[TEI.fsConstraints]{fsConstraints} \hyperref[TEI.fsDecl]{fsDecl} \hyperref[TEI.fsDescr]{fsDescr} \hyperref[TEI.fsdDecl]{fsdDecl} \hyperref[TEI.fsdLink]{fsdLink} \hyperref[TEI.funder]{funder} \hyperref[TEI.fvLib]{fvLib} \hyperref[TEI.fw]{fw} \hyperref[TEI.gap]{gap} \hyperref[TEI.gb]{gb} \hyperref[TEI.genName]{genName} \hyperref[TEI.geogName]{geogName} \hyperref[TEI.gloss]{gloss} \hyperref[TEI.graphic]{graphic} \hyperref[TEI.group]{group} \hyperref[TEI.handDesc]{handDesc} \hyperref[TEI.handNotes]{handNotes} \hyperref[TEI.handShift]{handShift} \hyperref[TEI.head]{head} \hyperref[TEI.headItem]{headItem} \hyperref[TEI.headLabel]{headLabel} \hyperref[TEI.height]{height} \hyperref[TEI.heraldry]{heraldry} \hyperref[TEI.hi]{hi} \hyperref[TEI.history]{history} \hyperref[TEI.idno]{idno} \hyperref[TEI.if]{if} \hyperref[TEI.iff]{iff} \hyperref[TEI.imprint]{imprint} \hyperref[TEI.incipit]{incipit} \hyperref[TEI.index]{index} \hyperref[TEI.institution]{institution} \hyperref[TEI.interp]{interp} \hyperref[TEI.interpGrp]{interpGrp} \hyperref[TEI.item]{item} \hyperref[TEI.join]{join} \hyperref[TEI.joinGrp]{joinGrp} \hyperref[TEI.keywords]{keywords} \hyperref[TEI.l]{l} \hyperref[TEI.label]{label} \hyperref[TEI.langUsage]{langUsage} \hyperref[TEI.language]{language} \hyperref[TEI.layout]{layout} \hyperref[TEI.layoutDesc]{layoutDesc} \hyperref[TEI.lb]{lb} \hyperref[TEI.lg]{lg} \hyperref[TEI.licence]{licence} \hyperref[TEI.line]{line} \hyperref[TEI.link]{link} \hyperref[TEI.linkGrp]{linkGrp} \hyperref[TEI.list]{list} \hyperref[TEI.listAnnotation]{listAnnotation} \hyperref[TEI.listBibl]{listBibl} \hyperref[TEI.listOrg]{listOrg} \hyperref[TEI.listPlace]{listPlace} \hyperref[TEI.listTranspose]{listTranspose} \hyperref[TEI.location]{location} \hyperref[TEI.locus]{locus} \hyperref[TEI.locusGrp]{locusGrp} \hyperref[TEI.m]{m} \hyperref[TEI.material]{material} \hyperref[TEI.measure]{measure} \hyperref[TEI.measureGrp]{measureGrp} \hyperref[TEI.media]{media} \hyperref[TEI.meeting]{meeting} \hyperref[TEI.mentioned]{mentioned} \hyperref[TEI.metamark]{metamark} \hyperref[TEI.milestone]{milestone} \hyperref[TEI.mod]{mod} \hyperref[TEI.monogr]{monogr} \hyperref[TEI.msContents]{msContents} \hyperref[TEI.msDesc]{msDesc} \hyperref[TEI.msFrag]{msFrag} \hyperref[TEI.msIdentifier]{msIdentifier} \hyperref[TEI.msItem]{msItem} \hyperref[TEI.msItemStruct]{msItemStruct} \hyperref[TEI.msName]{msName} \hyperref[TEI.msPart]{msPart} \hyperref[TEI.musicNotation]{musicNotation} \hyperref[TEI.name]{name} \hyperref[TEI.nameLink]{nameLink} \hyperref[TEI.namespace]{namespace} \hyperref[TEI.notatedMusic]{notatedMusic} \hyperref[TEI.note]{note} \hyperref[TEI.notesStmt]{notesStmt} \hyperref[TEI.num]{num} \hyperref[TEI.numeric]{numeric} \hyperref[TEI.objectDesc]{objectDesc} \hyperref[TEI.objectType]{objectType} \hyperref[TEI.org]{org} \hyperref[TEI.orgName]{orgName} \hyperref[TEI.orig]{orig} \hyperref[TEI.origDate]{origDate} \hyperref[TEI.origPlace]{origPlace} \hyperref[TEI.origin]{origin} \hyperref[TEI.p]{p} \hyperref[TEI.pb]{pb} \hyperref[TEI.pc]{pc} \hyperref[TEI.persName]{persName} \hyperref[TEI.person]{person} \hyperref[TEI.personGrp]{personGrp} \hyperref[TEI.persona]{persona} \hyperref[TEI.phr]{phr} \hyperref[TEI.physDesc]{physDesc} \hyperref[TEI.place]{place} \hyperref[TEI.placeName]{placeName} \hyperref[TEI.postBox]{postBox} \hyperref[TEI.postCode]{postCode} \hyperref[TEI.profileDesc]{profileDesc} \hyperref[TEI.provenance]{provenance} \hyperref[TEI.ptr]{ptr} \hyperref[TEI.pubPlace]{pubPlace} \hyperref[TEI.publicationStmt]{publicationStmt} \hyperref[TEI.publisher]{publisher} \hyperref[TEI.q]{q} \hyperref[TEI.quote]{quote} \hyperref[TEI.recordHist]{recordHist} \hyperref[TEI.redo]{redo} \hyperref[TEI.ref]{ref} \hyperref[TEI.reg]{reg} \hyperref[TEI.region]{region} \hyperref[TEI.relatedItem]{relatedItem} \hyperref[TEI.rendition]{rendition} \hyperref[TEI.repository]{repository} \hyperref[TEI.resp]{resp} \hyperref[TEI.respStmt]{respStmt} \hyperref[TEI.restore]{restore} \hyperref[TEI.retrace]{retrace} \hyperref[TEI.revisionDesc]{revisionDesc} \hyperref[TEI.roleName]{roleName} \hyperref[TEI.row]{row} \hyperref[TEI.rs]{rs} \hyperref[TEI.rubric]{rubric} \hyperref[TEI.s]{s} \hyperref[TEI.said]{said} \hyperref[TEI.schemaRef]{schemaRef} \hyperref[TEI.scriptDesc]{scriptDesc} \hyperref[TEI.seal]{seal} \hyperref[TEI.sealDesc]{sealDesc} \hyperref[TEI.secFol]{secFol} \hyperref[TEI.secl]{secl} \hyperref[TEI.seg]{seg} \hyperref[TEI.series]{series} \hyperref[TEI.seriesStmt]{seriesStmt} \hyperref[TEI.settlement]{settlement} \hyperref[TEI.sic]{sic} \hyperref[TEI.signatures]{signatures} \hyperref[TEI.soCalled]{soCalled} \hyperref[TEI.source]{source} \hyperref[TEI.sourceDesc]{sourceDesc} \hyperref[TEI.sourceDoc]{sourceDoc} \hyperref[TEI.sp]{sp} \hyperref[TEI.span]{span} \hyperref[TEI.spanGrp]{spanGrp} \hyperref[TEI.speaker]{speaker} \hyperref[TEI.stage]{stage} \hyperref[TEI.stamp]{stamp} \hyperref[TEI.standOff]{standOff} \hyperref[TEI.state]{state} \hyperref[TEI.street]{street} \hyperref[TEI.string]{string} \hyperref[TEI.subst]{subst} \hyperref[TEI.substJoin]{substJoin} \hyperref[TEI.summary]{summary} \hyperref[TEI.supplied]{supplied} \hyperref[TEI.support]{support} \hyperref[TEI.supportDesc]{supportDesc} \hyperref[TEI.surface]{surface} \hyperref[TEI.surfaceGrp]{surfaceGrp} \hyperref[TEI.surname]{surname} \hyperref[TEI.surplus]{surplus} \hyperref[TEI.surrogates]{surrogates} \hyperref[TEI.symbol]{symbol} \hyperref[TEI.table]{table} \hyperref[TEI.taxonomy]{taxonomy} \hyperref[TEI.teiCorpus]{teiCorpus} \hyperref[TEI.teiHeader]{teiHeader} \hyperref[TEI.term]{term} \hyperref[TEI.text]{text} \hyperref[TEI.textClass]{textClass} \hyperref[TEI.textLang]{textLang} \hyperref[TEI.then]{then} \hyperref[TEI.time]{time} \hyperref[TEI.timeline]{timeline} \hyperref[TEI.title]{title} \hyperref[TEI.titlePage]{titlePage} \hyperref[TEI.titlePart]{titlePart} \hyperref[TEI.titleStmt]{titleStmt} \hyperref[TEI.transpose]{transpose} \hyperref[TEI.typeDesc]{typeDesc} \hyperref[TEI.typeNote]{typeNote} \hyperref[TEI.unclear]{unclear} \hyperref[TEI.undo]{undo} \hyperref[TEI.vAlt]{vAlt} \hyperref[TEI.vColl]{vColl} \hyperref[TEI.vDefault]{vDefault} \hyperref[TEI.vLabel]{vLabel} \hyperref[TEI.vMerge]{vMerge} \hyperref[TEI.vNot]{vNot} \hyperref[TEI.vRange]{vRange} \hyperref[TEI.w]{w} \hyperref[TEI.watermark]{watermark} \hyperref[TEI.when]{when} \hyperref[TEI.width]{width} \hyperref[TEI.zone]{zone}]
    \item[{Attributs}]
  Attributs\hfil\\[-10pt]\begin{sansreflist}
    \item[@facs]
  (fac-similé) pointe directement vers une image ou vers une partie d'une image correspondant au contenu de l'élément.
\begin{reflist}
    \item[{Statut}]
  Optionel
    \item[{Type de données}]
  1–∞ occurrences de \hyperref[TEI.teidata.pointer]{teidata.pointer} séparé par un espace
\end{reflist}  
\end{sansreflist}  
\end{reflist}  
\begin{reflist}
\item[]\begin{specHead}{TEI.att.global.linking}{att.global.linking}\index{att.global.linking (attribute class)|oddindex}\index{corresp=@corresp!att.global.linking (attribute class)|oddindex}\index{synch=@synch!att.global.linking (attribute class)|oddindex}\index{sameAs=@sameAs!att.global.linking (attribute class)|oddindex}\index{copyOf=@copyOf!att.global.linking (attribute class)|oddindex}\index{next=@next!att.global.linking (attribute class)|oddindex}\index{prev=@prev!att.global.linking (attribute class)|oddindex}\index{exclude=@exclude!att.global.linking (attribute class)|oddindex}\index{select=@select!att.global.linking (attribute class)|oddindex} fournit un ensemble d'attributs pour décrire les liens hypertextuels.\end{specHead} 
    \item[{Module}]
  linking
    \item[{Membres}]
  \hyperref[TEI.att.global]{att.global}[\hyperref[TEI.TEI]{TEI} \hyperref[TEI.ab]{ab} \hyperref[TEI.abbr]{abbr} \hyperref[TEI.abstract]{abstract} \hyperref[TEI.accMat]{accMat} \hyperref[TEI.acquisition]{acquisition} \hyperref[TEI.add]{add} \hyperref[TEI.addName]{addName} \hyperref[TEI.addSpan]{addSpan} \hyperref[TEI.additional]{additional} \hyperref[TEI.additions]{additions} \hyperref[TEI.addrLine]{addrLine} \hyperref[TEI.address]{address} \hyperref[TEI.adminInfo]{adminInfo} \hyperref[TEI.affiliation]{affiliation} \hyperref[TEI.alt]{alt} \hyperref[TEI.altGrp]{altGrp} \hyperref[TEI.altIdentifier]{altIdentifier} \hyperref[TEI.am]{am} \hyperref[TEI.analytic]{analytic} \hyperref[TEI.anchor]{anchor} \hyperref[TEI.annotationBlock]{annotationBlock} \hyperref[TEI.appInfo]{appInfo} \hyperref[TEI.application]{application} \hyperref[TEI.author]{author} \hyperref[TEI.authority]{authority} \hyperref[TEI.availability]{availability} \hyperref[TEI.back]{back} \hyperref[TEI.bibl]{bibl} \hyperref[TEI.biblFull]{biblFull} \hyperref[TEI.biblScope]{biblScope} \hyperref[TEI.biblStruct]{biblStruct} \hyperref[TEI.bicond]{bicond} \hyperref[TEI.binary]{binary} \hyperref[TEI.binaryObject]{binaryObject} \hyperref[TEI.binding]{binding} \hyperref[TEI.bindingDesc]{bindingDesc} \hyperref[TEI.body]{body} \hyperref[TEI.c]{c} \hyperref[TEI.catchwords]{catchwords} \hyperref[TEI.category]{category} \hyperref[TEI.cb]{cb} \hyperref[TEI.cell]{cell} \hyperref[TEI.change]{change} \hyperref[TEI.choice]{choice} \hyperref[TEI.cit]{cit} \hyperref[TEI.citedRange]{citedRange} \hyperref[TEI.cl]{cl} \hyperref[TEI.classCode]{classCode} \hyperref[TEI.classDecl]{classDecl} \hyperref[TEI.collation]{collation} \hyperref[TEI.collection]{collection} \hyperref[TEI.colophon]{colophon} \hyperref[TEI.cond]{cond} \hyperref[TEI.condition]{condition} \hyperref[TEI.corr]{corr} \hyperref[TEI.correction]{correction} \hyperref[TEI.country]{country} \hyperref[TEI.creation]{creation} \hyperref[TEI.custEvent]{custEvent} \hyperref[TEI.custodialHist]{custodialHist} \hyperref[TEI.damage]{damage} \hyperref[TEI.damageSpan]{damageSpan} \hyperref[TEI.date]{date} \hyperref[TEI.decoDesc]{decoDesc} \hyperref[TEI.decoNote]{decoNote} \hyperref[TEI.default]{default} \hyperref[TEI.del]{del} \hyperref[TEI.delSpan]{delSpan} \hyperref[TEI.depth]{depth} \hyperref[TEI.desc]{desc} \hyperref[TEI.dim]{dim} \hyperref[TEI.dimensions]{dimensions} \hyperref[TEI.distinct]{distinct} \hyperref[TEI.distributor]{distributor} \hyperref[TEI.div]{div} \hyperref[TEI.divGen]{divGen} \hyperref[TEI.docAuthor]{docAuthor} \hyperref[TEI.docDate]{docDate} \hyperref[TEI.docEdition]{docEdition} \hyperref[TEI.docTitle]{docTitle} \hyperref[TEI.edition]{edition} \hyperref[TEI.editionStmt]{editionStmt} \hyperref[TEI.editor]{editor} \hyperref[TEI.email]{email} \hyperref[TEI.emph]{emph} \hyperref[TEI.encodingDesc]{encodingDesc} \hyperref[TEI.event]{event} \hyperref[TEI.ex]{ex} \hyperref[TEI.expan]{expan} \hyperref[TEI.explicit]{explicit} \hyperref[TEI.extent]{extent} \hyperref[TEI.f]{f} \hyperref[TEI.fDecl]{fDecl} \hyperref[TEI.fDescr]{fDescr} \hyperref[TEI.fLib]{fLib} \hyperref[TEI.facsimile]{facsimile} \hyperref[TEI.figDesc]{figDesc} \hyperref[TEI.figure]{figure} \hyperref[TEI.fileDesc]{fileDesc} \hyperref[TEI.filiation]{filiation} \hyperref[TEI.finalRubric]{finalRubric} \hyperref[TEI.floatingText]{floatingText} \hyperref[TEI.foliation]{foliation} \hyperref[TEI.foreign]{foreign} \hyperref[TEI.forename]{forename} \hyperref[TEI.formula]{formula} \hyperref[TEI.front]{front} \hyperref[TEI.fs]{fs} \hyperref[TEI.fsConstraints]{fsConstraints} \hyperref[TEI.fsDecl]{fsDecl} \hyperref[TEI.fsDescr]{fsDescr} \hyperref[TEI.fsdDecl]{fsdDecl} \hyperref[TEI.fsdLink]{fsdLink} \hyperref[TEI.funder]{funder} \hyperref[TEI.fvLib]{fvLib} \hyperref[TEI.fw]{fw} \hyperref[TEI.gap]{gap} \hyperref[TEI.gb]{gb} \hyperref[TEI.genName]{genName} \hyperref[TEI.geogName]{geogName} \hyperref[TEI.gloss]{gloss} \hyperref[TEI.graphic]{graphic} \hyperref[TEI.group]{group} \hyperref[TEI.handDesc]{handDesc} \hyperref[TEI.handNotes]{handNotes} \hyperref[TEI.handShift]{handShift} \hyperref[TEI.head]{head} \hyperref[TEI.headItem]{headItem} \hyperref[TEI.headLabel]{headLabel} \hyperref[TEI.height]{height} \hyperref[TEI.heraldry]{heraldry} \hyperref[TEI.hi]{hi} \hyperref[TEI.history]{history} \hyperref[TEI.idno]{idno} \hyperref[TEI.if]{if} \hyperref[TEI.iff]{iff} \hyperref[TEI.imprint]{imprint} \hyperref[TEI.incipit]{incipit} \hyperref[TEI.index]{index} \hyperref[TEI.institution]{institution} \hyperref[TEI.interp]{interp} \hyperref[TEI.interpGrp]{interpGrp} \hyperref[TEI.item]{item} \hyperref[TEI.join]{join} \hyperref[TEI.joinGrp]{joinGrp} \hyperref[TEI.keywords]{keywords} \hyperref[TEI.l]{l} \hyperref[TEI.label]{label} \hyperref[TEI.langUsage]{langUsage} \hyperref[TEI.language]{language} \hyperref[TEI.layout]{layout} \hyperref[TEI.layoutDesc]{layoutDesc} \hyperref[TEI.lb]{lb} \hyperref[TEI.lg]{lg} \hyperref[TEI.licence]{licence} \hyperref[TEI.line]{line} \hyperref[TEI.link]{link} \hyperref[TEI.linkGrp]{linkGrp} \hyperref[TEI.list]{list} \hyperref[TEI.listAnnotation]{listAnnotation} \hyperref[TEI.listBibl]{listBibl} \hyperref[TEI.listOrg]{listOrg} \hyperref[TEI.listPlace]{listPlace} \hyperref[TEI.listTranspose]{listTranspose} \hyperref[TEI.location]{location} \hyperref[TEI.locus]{locus} \hyperref[TEI.locusGrp]{locusGrp} \hyperref[TEI.m]{m} \hyperref[TEI.material]{material} \hyperref[TEI.measure]{measure} \hyperref[TEI.measureGrp]{measureGrp} \hyperref[TEI.media]{media} \hyperref[TEI.meeting]{meeting} \hyperref[TEI.mentioned]{mentioned} \hyperref[TEI.metamark]{metamark} \hyperref[TEI.milestone]{milestone} \hyperref[TEI.mod]{mod} \hyperref[TEI.monogr]{monogr} \hyperref[TEI.msContents]{msContents} \hyperref[TEI.msDesc]{msDesc} \hyperref[TEI.msFrag]{msFrag} \hyperref[TEI.msIdentifier]{msIdentifier} \hyperref[TEI.msItem]{msItem} \hyperref[TEI.msItemStruct]{msItemStruct} \hyperref[TEI.msName]{msName} \hyperref[TEI.msPart]{msPart} \hyperref[TEI.musicNotation]{musicNotation} \hyperref[TEI.name]{name} \hyperref[TEI.nameLink]{nameLink} \hyperref[TEI.namespace]{namespace} \hyperref[TEI.notatedMusic]{notatedMusic} \hyperref[TEI.note]{note} \hyperref[TEI.notesStmt]{notesStmt} \hyperref[TEI.num]{num} \hyperref[TEI.numeric]{numeric} \hyperref[TEI.objectDesc]{objectDesc} \hyperref[TEI.objectType]{objectType} \hyperref[TEI.org]{org} \hyperref[TEI.orgName]{orgName} \hyperref[TEI.orig]{orig} \hyperref[TEI.origDate]{origDate} \hyperref[TEI.origPlace]{origPlace} \hyperref[TEI.origin]{origin} \hyperref[TEI.p]{p} \hyperref[TEI.pb]{pb} \hyperref[TEI.pc]{pc} \hyperref[TEI.persName]{persName} \hyperref[TEI.person]{person} \hyperref[TEI.personGrp]{personGrp} \hyperref[TEI.persona]{persona} \hyperref[TEI.phr]{phr} \hyperref[TEI.physDesc]{physDesc} \hyperref[TEI.place]{place} \hyperref[TEI.placeName]{placeName} \hyperref[TEI.postBox]{postBox} \hyperref[TEI.postCode]{postCode} \hyperref[TEI.profileDesc]{profileDesc} \hyperref[TEI.provenance]{provenance} \hyperref[TEI.ptr]{ptr} \hyperref[TEI.pubPlace]{pubPlace} \hyperref[TEI.publicationStmt]{publicationStmt} \hyperref[TEI.publisher]{publisher} \hyperref[TEI.q]{q} \hyperref[TEI.quote]{quote} \hyperref[TEI.recordHist]{recordHist} \hyperref[TEI.redo]{redo} \hyperref[TEI.ref]{ref} \hyperref[TEI.reg]{reg} \hyperref[TEI.region]{region} \hyperref[TEI.relatedItem]{relatedItem} \hyperref[TEI.rendition]{rendition} \hyperref[TEI.repository]{repository} \hyperref[TEI.resp]{resp} \hyperref[TEI.respStmt]{respStmt} \hyperref[TEI.restore]{restore} \hyperref[TEI.retrace]{retrace} \hyperref[TEI.revisionDesc]{revisionDesc} \hyperref[TEI.roleName]{roleName} \hyperref[TEI.row]{row} \hyperref[TEI.rs]{rs} \hyperref[TEI.rubric]{rubric} \hyperref[TEI.s]{s} \hyperref[TEI.said]{said} \hyperref[TEI.schemaRef]{schemaRef} \hyperref[TEI.scriptDesc]{scriptDesc} \hyperref[TEI.seal]{seal} \hyperref[TEI.sealDesc]{sealDesc} \hyperref[TEI.secFol]{secFol} \hyperref[TEI.secl]{secl} \hyperref[TEI.seg]{seg} \hyperref[TEI.series]{series} \hyperref[TEI.seriesStmt]{seriesStmt} \hyperref[TEI.settlement]{settlement} \hyperref[TEI.sic]{sic} \hyperref[TEI.signatures]{signatures} \hyperref[TEI.soCalled]{soCalled} \hyperref[TEI.source]{source} \hyperref[TEI.sourceDesc]{sourceDesc} \hyperref[TEI.sourceDoc]{sourceDoc} \hyperref[TEI.sp]{sp} \hyperref[TEI.span]{span} \hyperref[TEI.spanGrp]{spanGrp} \hyperref[TEI.speaker]{speaker} \hyperref[TEI.stage]{stage} \hyperref[TEI.stamp]{stamp} \hyperref[TEI.standOff]{standOff} \hyperref[TEI.state]{state} \hyperref[TEI.street]{street} \hyperref[TEI.string]{string} \hyperref[TEI.subst]{subst} \hyperref[TEI.substJoin]{substJoin} \hyperref[TEI.summary]{summary} \hyperref[TEI.supplied]{supplied} \hyperref[TEI.support]{support} \hyperref[TEI.supportDesc]{supportDesc} \hyperref[TEI.surface]{surface} \hyperref[TEI.surfaceGrp]{surfaceGrp} \hyperref[TEI.surname]{surname} \hyperref[TEI.surplus]{surplus} \hyperref[TEI.surrogates]{surrogates} \hyperref[TEI.symbol]{symbol} \hyperref[TEI.table]{table} \hyperref[TEI.taxonomy]{taxonomy} \hyperref[TEI.teiCorpus]{teiCorpus} \hyperref[TEI.teiHeader]{teiHeader} \hyperref[TEI.term]{term} \hyperref[TEI.text]{text} \hyperref[TEI.textClass]{textClass} \hyperref[TEI.textLang]{textLang} \hyperref[TEI.then]{then} \hyperref[TEI.time]{time} \hyperref[TEI.timeline]{timeline} \hyperref[TEI.title]{title} \hyperref[TEI.titlePage]{titlePage} \hyperref[TEI.titlePart]{titlePart} \hyperref[TEI.titleStmt]{titleStmt} \hyperref[TEI.transpose]{transpose} \hyperref[TEI.typeDesc]{typeDesc} \hyperref[TEI.typeNote]{typeNote} \hyperref[TEI.unclear]{unclear} \hyperref[TEI.undo]{undo} \hyperref[TEI.vAlt]{vAlt} \hyperref[TEI.vColl]{vColl} \hyperref[TEI.vDefault]{vDefault} \hyperref[TEI.vLabel]{vLabel} \hyperref[TEI.vMerge]{vMerge} \hyperref[TEI.vNot]{vNot} \hyperref[TEI.vRange]{vRange} \hyperref[TEI.w]{w} \hyperref[TEI.watermark]{watermark} \hyperref[TEI.when]{when} \hyperref[TEI.width]{width} \hyperref[TEI.zone]{zone}]
    \item[{Attributs}]
  Attributs\hfil\\[-10pt]\begin{sansreflist}
    \item[@corresp]
  (correspond) pointe vers des éléments qui ont une correspondance avec l'élément en question.
\begin{reflist}
    \item[{Statut}]
  Optionel
    \item[{Type de données}]
  1–∞ occurrences de \hyperref[TEI.teidata.pointer]{teidata.pointer} séparé par un espace
    \item[]\exampleFont {<\textbf{group}>}\mbox{}\newline 
\hspace*{6pt}{<\textbf{text}\hspace*{6pt}{xml:id}="{t1-g1-t1}"\mbox{}\newline 
\hspace*{6pt}\hspace*{6pt}{xml:lang}="{mi}">}\mbox{}\newline 
\hspace*{6pt}\hspace*{6pt}{<\textbf{body}\hspace*{6pt}{xml:id}="{t1-g1-t1-body1}">}\mbox{}\newline 
\hspace*{6pt}\hspace*{6pt}\hspace*{6pt}{<\textbf{div}\hspace*{6pt}{type}="{chapter}">}\mbox{}\newline 
\hspace*{6pt}\hspace*{6pt}\hspace*{6pt}\hspace*{6pt}{<\textbf{head}>}He Whakamaramatanga mo te Ture Hoko, Riihi hoki, i nga Whenua Maori, 1876.{</\textbf{head}>}\mbox{}\newline 
\hspace*{6pt}\hspace*{6pt}\hspace*{6pt}\hspace*{6pt}{<\textbf{p}>}…{</\textbf{p}>}\mbox{}\newline 
\hspace*{6pt}\hspace*{6pt}\hspace*{6pt}{</\textbf{div}>}\mbox{}\newline 
\hspace*{6pt}\hspace*{6pt}{</\textbf{body}>}\mbox{}\newline 
\hspace*{6pt}{</\textbf{text}>}\mbox{}\newline 
\hspace*{6pt}{<\textbf{text}\hspace*{6pt}{xml:id}="{t1-g1-t2}"\mbox{}\newline 
\hspace*{6pt}\hspace*{6pt}{xml:lang}="{en}">}\mbox{}\newline 
\hspace*{6pt}\hspace*{6pt}{<\textbf{body}\hspace*{6pt}{corresp}="{\#t1-g1-t1-body1}"\mbox{}\newline 
\hspace*{6pt}\hspace*{6pt}\hspace*{6pt}{xml:id}="{t1-g1-t2-body1}">}\mbox{}\newline 
\hspace*{6pt}\hspace*{6pt}\hspace*{6pt}{<\textbf{div}\hspace*{6pt}{type}="{chapter}">}\mbox{}\newline 
\hspace*{6pt}\hspace*{6pt}\hspace*{6pt}\hspace*{6pt}{<\textbf{head}>}An Act to regulate the Sale, Letting, and Disposal of Native Lands, 1876.{</\textbf{head}>}\mbox{}\newline 
\hspace*{6pt}\hspace*{6pt}\hspace*{6pt}\hspace*{6pt}{<\textbf{p}>}…{</\textbf{p}>}\mbox{}\newline 
\hspace*{6pt}\hspace*{6pt}\hspace*{6pt}{</\textbf{div}>}\mbox{}\newline 
\hspace*{6pt}\hspace*{6pt}{</\textbf{body}>}\mbox{}\newline 
\hspace*{6pt}{</\textbf{text}>}\mbox{}\newline 
{</\textbf{group}>}In this example a \hyperref[TEI.group]{<group>} contains two \hyperref[TEI.text]{<text>}s, each containing the same document in a different language. The correspondence is indicated using {\itshape corresp}. The language is indicated using {\itshape xml:lang}, whose value is inherited; both the tag with the {\itshape corresp} and the tag pointed to by the {\itshape corresp} inherit the value from their immediate parent.
    \item[]\exampleFont {<\textbf{place}\hspace*{6pt}{corresp}="{people.xml\#LOND2 people.xml\#GENI1}"\mbox{}\newline 
\hspace*{6pt}{xml:id}="{LOND1}">}\mbox{}\newline 
\hspace*{6pt}{<\textbf{placeName}>}London{</\textbf{placeName}>}\mbox{}\newline 
\hspace*{6pt}{<\textbf{desc}>}The city of London...{</\textbf{desc}>}\mbox{}\newline 
{</\textbf{place}>}\mbox{}\newline 
{<\textbf{person}\hspace*{6pt}{corresp}="{places.xml\#LOND1 \#GENI1}"\mbox{}\newline 
\hspace*{6pt}{xml:id}="{LOND2}">}\mbox{}\newline 
\hspace*{6pt}{<\textbf{persName}\hspace*{6pt}{type}="{lit}">}London{</\textbf{persName}>}\mbox{}\newline 
\hspace*{6pt}{<\textbf{note}>}\mbox{}\newline 
\hspace*{6pt}\hspace*{6pt}{<\textbf{p}>}Allegorical character representing the city of {<\textbf{placeName}\hspace*{6pt}{ref}="{places.xml\#LOND1}">}London{</\textbf{placeName}>}.{</\textbf{p}>}\mbox{}\newline 
\hspace*{6pt}{</\textbf{note}>}\mbox{}\newline 
{</\textbf{person}>}\mbox{}\newline 
{<\textbf{person}\hspace*{6pt}{corresp}="{places.xml\#LOND1 \#LOND2}"\mbox{}\newline 
\hspace*{6pt}{xml:id}="{GENI1}">}\mbox{}\newline 
\hspace*{6pt}{<\textbf{persName}\hspace*{6pt}{type}="{lit}">}London’s Genius{</\textbf{persName}>}\mbox{}\newline 
\hspace*{6pt}{<\textbf{note}>}\mbox{}\newline 
\hspace*{6pt}\hspace*{6pt}{<\textbf{p}>}Personification of London’s genius. Appears as an\mbox{}\newline 
\hspace*{6pt}\hspace*{6pt}\hspace*{6pt}\hspace*{6pt} allegorical character in mayoral shows.\mbox{}\newline 
\hspace*{6pt}\hspace*{6pt}{</\textbf{p}>}\mbox{}\newline 
\hspace*{6pt}{</\textbf{note}>}\mbox{}\newline 
{</\textbf{person}>}In this example, a \hyperref[TEI.place]{<place>} element containing information about the city of London is linked with two \hyperref[TEI.person]{<person>} elements in a literary personography. This correspondence represents a slightly looser relationship than the one in the preceding example; there is no sense in which an allegorical character could be substituted for the physical city, or vice versa, but there is obviously a correspondence between them.
\end{reflist}  
    \item[@synch]
  (synchrone) pointe vers des éléments qui sont synchrones avec l'élément en question.
\begin{reflist}
    \item[{Statut}]
  Optionel
    \item[{Type de données}]
  1–∞ occurrences de \hyperref[TEI.teidata.pointer]{teidata.pointer} séparé par un espace
\end{reflist}  
    \item[@sameAs]
  pointe vers un élément identique à l'élément en question.
\begin{reflist}
    \item[{Statut}]
  Optionel
    \item[{Type de données}]
  \hyperref[TEI.teidata.pointer]{teidata.pointer}
\end{reflist}  
    \item[@copyOf]
  pointe vers un élément dont l'élément en question est une copie.
\begin{reflist}
    \item[{Statut}]
  Optionel
    \item[{Type de données}]
  \hyperref[TEI.teidata.pointer]{teidata.pointer}
    \item[{Note}]
  \par
Tout contenu appartenant à l'élément en cours doit être ignoré. Le vrai contenu est celui de l'élément cible du pointeur.
\end{reflist}  
    \item[@next]
  pointe vers l'élément suivant d'un ensemble virtuel dont l'élément en question est une partie.
\begin{reflist}
    \item[{Statut}]
  Optionel
    \item[{Type de données}]
  \hyperref[TEI.teidata.pointer]{teidata.pointer}
\end{reflist}  
    \item[@prev]
  (précédent) pointe vers l'élément précédent d'un ensemble virtuel auquel appartient l'élément en question.
\begin{reflist}
    \item[{Statut}]
  Optionel
    \item[{Type de données}]
  \hyperref[TEI.teidata.pointer]{teidata.pointer}
\end{reflist}  
    \item[@exclude]
  pointe vers des éléments qui sont une alternative exclusive à l'élément en question.
\begin{reflist}
    \item[{Statut}]
  Optionel
    \item[{Type de données}]
  1–∞ occurrences de \hyperref[TEI.teidata.pointer]{teidata.pointer} séparé par un espace
\end{reflist}  
    \item[@select]
  sélectionne une ou plusieurs valeurs alternatives ; si une seule valeur est sélectionnée, l'ambiguïté ou l'incertitude est marquée comme résolue. Si plus d'une valeur alternative est sélectionnée, le degré d'ambiguïté ou d'incertitude est marqué comme réduit par le nombre de valeurs alternatives non sélectionnées.
\begin{reflist}
    \item[{Statut}]
  Optionel
    \item[{Type de données}]
  1–∞ occurrences de \hyperref[TEI.teidata.pointer]{teidata.pointer} séparé par un espace
    \item[{Note}]
  \par
Cet attribut doit être placé dans un élément hiérarchiquement supérieur à tous les éléments possibles parmi lesquelles la sélection est faite.
\end{reflist}  
\end{sansreflist}  
\end{reflist}  
\begin{reflist}
\item[]\begin{specHead}{TEI.att.global.rendition}{att.global.rendition}\index{att.global.rendition (attribute class)|oddindex}\index{rend=@rend!att.global.rendition (attribute class)|oddindex}\index{style=@style!att.global.rendition (attribute class)|oddindex}\index{rendition=@rendition!att.global.rendition (attribute class)|oddindex} provides rendering attributes common to all elements in the TEI encoding scheme.\end{specHead} 
    \item[{Module}]
  tei
    \item[{Membres}]
  \hyperref[TEI.att.global]{att.global}[\hyperref[TEI.TEI]{TEI} \hyperref[TEI.ab]{ab} \hyperref[TEI.abbr]{abbr} \hyperref[TEI.abstract]{abstract} \hyperref[TEI.accMat]{accMat} \hyperref[TEI.acquisition]{acquisition} \hyperref[TEI.add]{add} \hyperref[TEI.addName]{addName} \hyperref[TEI.addSpan]{addSpan} \hyperref[TEI.additional]{additional} \hyperref[TEI.additions]{additions} \hyperref[TEI.addrLine]{addrLine} \hyperref[TEI.address]{address} \hyperref[TEI.adminInfo]{adminInfo} \hyperref[TEI.affiliation]{affiliation} \hyperref[TEI.alt]{alt} \hyperref[TEI.altGrp]{altGrp} \hyperref[TEI.altIdentifier]{altIdentifier} \hyperref[TEI.am]{am} \hyperref[TEI.analytic]{analytic} \hyperref[TEI.anchor]{anchor} \hyperref[TEI.annotationBlock]{annotationBlock} \hyperref[TEI.appInfo]{appInfo} \hyperref[TEI.application]{application} \hyperref[TEI.author]{author} \hyperref[TEI.authority]{authority} \hyperref[TEI.availability]{availability} \hyperref[TEI.back]{back} \hyperref[TEI.bibl]{bibl} \hyperref[TEI.biblFull]{biblFull} \hyperref[TEI.biblScope]{biblScope} \hyperref[TEI.biblStruct]{biblStruct} \hyperref[TEI.bicond]{bicond} \hyperref[TEI.binary]{binary} \hyperref[TEI.binaryObject]{binaryObject} \hyperref[TEI.binding]{binding} \hyperref[TEI.bindingDesc]{bindingDesc} \hyperref[TEI.body]{body} \hyperref[TEI.c]{c} \hyperref[TEI.catchwords]{catchwords} \hyperref[TEI.category]{category} \hyperref[TEI.cb]{cb} \hyperref[TEI.cell]{cell} \hyperref[TEI.change]{change} \hyperref[TEI.choice]{choice} \hyperref[TEI.cit]{cit} \hyperref[TEI.citedRange]{citedRange} \hyperref[TEI.cl]{cl} \hyperref[TEI.classCode]{classCode} \hyperref[TEI.classDecl]{classDecl} \hyperref[TEI.collation]{collation} \hyperref[TEI.collection]{collection} \hyperref[TEI.colophon]{colophon} \hyperref[TEI.cond]{cond} \hyperref[TEI.condition]{condition} \hyperref[TEI.corr]{corr} \hyperref[TEI.correction]{correction} \hyperref[TEI.country]{country} \hyperref[TEI.creation]{creation} \hyperref[TEI.custEvent]{custEvent} \hyperref[TEI.custodialHist]{custodialHist} \hyperref[TEI.damage]{damage} \hyperref[TEI.damageSpan]{damageSpan} \hyperref[TEI.date]{date} \hyperref[TEI.decoDesc]{decoDesc} \hyperref[TEI.decoNote]{decoNote} \hyperref[TEI.default]{default} \hyperref[TEI.del]{del} \hyperref[TEI.delSpan]{delSpan} \hyperref[TEI.depth]{depth} \hyperref[TEI.desc]{desc} \hyperref[TEI.dim]{dim} \hyperref[TEI.dimensions]{dimensions} \hyperref[TEI.distinct]{distinct} \hyperref[TEI.distributor]{distributor} \hyperref[TEI.div]{div} \hyperref[TEI.divGen]{divGen} \hyperref[TEI.docAuthor]{docAuthor} \hyperref[TEI.docDate]{docDate} \hyperref[TEI.docEdition]{docEdition} \hyperref[TEI.docTitle]{docTitle} \hyperref[TEI.edition]{edition} \hyperref[TEI.editionStmt]{editionStmt} \hyperref[TEI.editor]{editor} \hyperref[TEI.email]{email} \hyperref[TEI.emph]{emph} \hyperref[TEI.encodingDesc]{encodingDesc} \hyperref[TEI.event]{event} \hyperref[TEI.ex]{ex} \hyperref[TEI.expan]{expan} \hyperref[TEI.explicit]{explicit} \hyperref[TEI.extent]{extent} \hyperref[TEI.f]{f} \hyperref[TEI.fDecl]{fDecl} \hyperref[TEI.fDescr]{fDescr} \hyperref[TEI.fLib]{fLib} \hyperref[TEI.facsimile]{facsimile} \hyperref[TEI.figDesc]{figDesc} \hyperref[TEI.figure]{figure} \hyperref[TEI.fileDesc]{fileDesc} \hyperref[TEI.filiation]{filiation} \hyperref[TEI.finalRubric]{finalRubric} \hyperref[TEI.floatingText]{floatingText} \hyperref[TEI.foliation]{foliation} \hyperref[TEI.foreign]{foreign} \hyperref[TEI.forename]{forename} \hyperref[TEI.formula]{formula} \hyperref[TEI.front]{front} \hyperref[TEI.fs]{fs} \hyperref[TEI.fsConstraints]{fsConstraints} \hyperref[TEI.fsDecl]{fsDecl} \hyperref[TEI.fsDescr]{fsDescr} \hyperref[TEI.fsdDecl]{fsdDecl} \hyperref[TEI.fsdLink]{fsdLink} \hyperref[TEI.funder]{funder} \hyperref[TEI.fvLib]{fvLib} \hyperref[TEI.fw]{fw} \hyperref[TEI.gap]{gap} \hyperref[TEI.gb]{gb} \hyperref[TEI.genName]{genName} \hyperref[TEI.geogName]{geogName} \hyperref[TEI.gloss]{gloss} \hyperref[TEI.graphic]{graphic} \hyperref[TEI.group]{group} \hyperref[TEI.handDesc]{handDesc} \hyperref[TEI.handNotes]{handNotes} \hyperref[TEI.handShift]{handShift} \hyperref[TEI.head]{head} \hyperref[TEI.headItem]{headItem} \hyperref[TEI.headLabel]{headLabel} \hyperref[TEI.height]{height} \hyperref[TEI.heraldry]{heraldry} \hyperref[TEI.hi]{hi} \hyperref[TEI.history]{history} \hyperref[TEI.idno]{idno} \hyperref[TEI.if]{if} \hyperref[TEI.iff]{iff} \hyperref[TEI.imprint]{imprint} \hyperref[TEI.incipit]{incipit} \hyperref[TEI.index]{index} \hyperref[TEI.institution]{institution} \hyperref[TEI.interp]{interp} \hyperref[TEI.interpGrp]{interpGrp} \hyperref[TEI.item]{item} \hyperref[TEI.join]{join} \hyperref[TEI.joinGrp]{joinGrp} \hyperref[TEI.keywords]{keywords} \hyperref[TEI.l]{l} \hyperref[TEI.label]{label} \hyperref[TEI.langUsage]{langUsage} \hyperref[TEI.language]{language} \hyperref[TEI.layout]{layout} \hyperref[TEI.layoutDesc]{layoutDesc} \hyperref[TEI.lb]{lb} \hyperref[TEI.lg]{lg} \hyperref[TEI.licence]{licence} \hyperref[TEI.line]{line} \hyperref[TEI.link]{link} \hyperref[TEI.linkGrp]{linkGrp} \hyperref[TEI.list]{list} \hyperref[TEI.listAnnotation]{listAnnotation} \hyperref[TEI.listBibl]{listBibl} \hyperref[TEI.listOrg]{listOrg} \hyperref[TEI.listPlace]{listPlace} \hyperref[TEI.listTranspose]{listTranspose} \hyperref[TEI.location]{location} \hyperref[TEI.locus]{locus} \hyperref[TEI.locusGrp]{locusGrp} \hyperref[TEI.m]{m} \hyperref[TEI.material]{material} \hyperref[TEI.measure]{measure} \hyperref[TEI.measureGrp]{measureGrp} \hyperref[TEI.media]{media} \hyperref[TEI.meeting]{meeting} \hyperref[TEI.mentioned]{mentioned} \hyperref[TEI.metamark]{metamark} \hyperref[TEI.milestone]{milestone} \hyperref[TEI.mod]{mod} \hyperref[TEI.monogr]{monogr} \hyperref[TEI.msContents]{msContents} \hyperref[TEI.msDesc]{msDesc} \hyperref[TEI.msFrag]{msFrag} \hyperref[TEI.msIdentifier]{msIdentifier} \hyperref[TEI.msItem]{msItem} \hyperref[TEI.msItemStruct]{msItemStruct} \hyperref[TEI.msName]{msName} \hyperref[TEI.msPart]{msPart} \hyperref[TEI.musicNotation]{musicNotation} \hyperref[TEI.name]{name} \hyperref[TEI.nameLink]{nameLink} \hyperref[TEI.namespace]{namespace} \hyperref[TEI.notatedMusic]{notatedMusic} \hyperref[TEI.note]{note} \hyperref[TEI.notesStmt]{notesStmt} \hyperref[TEI.num]{num} \hyperref[TEI.numeric]{numeric} \hyperref[TEI.objectDesc]{objectDesc} \hyperref[TEI.objectType]{objectType} \hyperref[TEI.org]{org} \hyperref[TEI.orgName]{orgName} \hyperref[TEI.orig]{orig} \hyperref[TEI.origDate]{origDate} \hyperref[TEI.origPlace]{origPlace} \hyperref[TEI.origin]{origin} \hyperref[TEI.p]{p} \hyperref[TEI.pb]{pb} \hyperref[TEI.pc]{pc} \hyperref[TEI.persName]{persName} \hyperref[TEI.person]{person} \hyperref[TEI.personGrp]{personGrp} \hyperref[TEI.persona]{persona} \hyperref[TEI.phr]{phr} \hyperref[TEI.physDesc]{physDesc} \hyperref[TEI.place]{place} \hyperref[TEI.placeName]{placeName} \hyperref[TEI.postBox]{postBox} \hyperref[TEI.postCode]{postCode} \hyperref[TEI.profileDesc]{profileDesc} \hyperref[TEI.provenance]{provenance} \hyperref[TEI.ptr]{ptr} \hyperref[TEI.pubPlace]{pubPlace} \hyperref[TEI.publicationStmt]{publicationStmt} \hyperref[TEI.publisher]{publisher} \hyperref[TEI.q]{q} \hyperref[TEI.quote]{quote} \hyperref[TEI.recordHist]{recordHist} \hyperref[TEI.redo]{redo} \hyperref[TEI.ref]{ref} \hyperref[TEI.reg]{reg} \hyperref[TEI.region]{region} \hyperref[TEI.relatedItem]{relatedItem} \hyperref[TEI.rendition]{rendition} \hyperref[TEI.repository]{repository} \hyperref[TEI.resp]{resp} \hyperref[TEI.respStmt]{respStmt} \hyperref[TEI.restore]{restore} \hyperref[TEI.retrace]{retrace} \hyperref[TEI.revisionDesc]{revisionDesc} \hyperref[TEI.roleName]{roleName} \hyperref[TEI.row]{row} \hyperref[TEI.rs]{rs} \hyperref[TEI.rubric]{rubric} \hyperref[TEI.s]{s} \hyperref[TEI.said]{said} \hyperref[TEI.schemaRef]{schemaRef} \hyperref[TEI.scriptDesc]{scriptDesc} \hyperref[TEI.seal]{seal} \hyperref[TEI.sealDesc]{sealDesc} \hyperref[TEI.secFol]{secFol} \hyperref[TEI.secl]{secl} \hyperref[TEI.seg]{seg} \hyperref[TEI.series]{series} \hyperref[TEI.seriesStmt]{seriesStmt} \hyperref[TEI.settlement]{settlement} \hyperref[TEI.sic]{sic} \hyperref[TEI.signatures]{signatures} \hyperref[TEI.soCalled]{soCalled} \hyperref[TEI.source]{source} \hyperref[TEI.sourceDesc]{sourceDesc} \hyperref[TEI.sourceDoc]{sourceDoc} \hyperref[TEI.sp]{sp} \hyperref[TEI.span]{span} \hyperref[TEI.spanGrp]{spanGrp} \hyperref[TEI.speaker]{speaker} \hyperref[TEI.stage]{stage} \hyperref[TEI.stamp]{stamp} \hyperref[TEI.standOff]{standOff} \hyperref[TEI.state]{state} \hyperref[TEI.street]{street} \hyperref[TEI.string]{string} \hyperref[TEI.subst]{subst} \hyperref[TEI.substJoin]{substJoin} \hyperref[TEI.summary]{summary} \hyperref[TEI.supplied]{supplied} \hyperref[TEI.support]{support} \hyperref[TEI.supportDesc]{supportDesc} \hyperref[TEI.surface]{surface} \hyperref[TEI.surfaceGrp]{surfaceGrp} \hyperref[TEI.surname]{surname} \hyperref[TEI.surplus]{surplus} \hyperref[TEI.surrogates]{surrogates} \hyperref[TEI.symbol]{symbol} \hyperref[TEI.table]{table} \hyperref[TEI.taxonomy]{taxonomy} \hyperref[TEI.teiCorpus]{teiCorpus} \hyperref[TEI.teiHeader]{teiHeader} \hyperref[TEI.term]{term} \hyperref[TEI.text]{text} \hyperref[TEI.textClass]{textClass} \hyperref[TEI.textLang]{textLang} \hyperref[TEI.then]{then} \hyperref[TEI.time]{time} \hyperref[TEI.timeline]{timeline} \hyperref[TEI.title]{title} \hyperref[TEI.titlePage]{titlePage} \hyperref[TEI.titlePart]{titlePart} \hyperref[TEI.titleStmt]{titleStmt} \hyperref[TEI.transpose]{transpose} \hyperref[TEI.typeDesc]{typeDesc} \hyperref[TEI.typeNote]{typeNote} \hyperref[TEI.unclear]{unclear} \hyperref[TEI.undo]{undo} \hyperref[TEI.vAlt]{vAlt} \hyperref[TEI.vColl]{vColl} \hyperref[TEI.vDefault]{vDefault} \hyperref[TEI.vLabel]{vLabel} \hyperref[TEI.vMerge]{vMerge} \hyperref[TEI.vNot]{vNot} \hyperref[TEI.vRange]{vRange} \hyperref[TEI.w]{w} \hyperref[TEI.watermark]{watermark} \hyperref[TEI.when]{when} \hyperref[TEI.width]{width} \hyperref[TEI.zone]{zone}]
    \item[{Attributs}]
  Attributs\hfil\\[-10pt]\begin{sansreflist}
    \item[@rend]
  (interprétation) indique comment l'élément en question a été rendu ou présenté dans le texte source
\begin{reflist}
    \item[{Statut}]
  Optionel
    \item[{Type de données}]
  1–∞ occurrences de \hyperref[TEI.teidata.word]{teidata.word} séparé par un espace
    \item[]\exampleFont {<\textbf{head}\hspace*{6pt}{rend}="{align(center) case(allcaps)}">}épître dédicatoire{<\textbf{lb}/>}à {<\textbf{lb}/>}Monsieur de Coucy {<\textbf{lb}/>}\mbox{}\newline 
\hspace*{6pt}{<\textbf{lb}/>}.{</\textbf{head}>}
    \item[{Note}]
  \par
Ces Principes directeurs ne font aucune recommandation contraignante pour les valeurs de l'attribut {\itshape rend}; les caractéristiques de la présentation visuelle changent trop d'un texte à l'autre et la décision d'enregistrer ou d'ignorer des caractéristiques individuelles est trop variable d'un projet à l'autre. Quelques conventions potentiellement utiles sont notées de temps en temps à des points appropriés dans ces Principes directeurs.
\end{reflist}  
    \item[@style]
  contains an expression in some formal style definition language which defines the rendering or presentation used for this element in the source text
\begin{reflist}
    \item[{Statut}]
  Optionel
    \item[{Type de données}]
  \hyperref[TEI.teidata.text]{teidata.text}
    \item[]\exampleFont {<\textbf{head}\hspace*{6pt}{style}="{text-align: center; font-variant: small-caps}">}\mbox{}\newline 
\hspace*{6pt}{<\textbf{lb}/>}To The {<\textbf{lb}/>}Duchesse {<\textbf{lb}/>}of {<\textbf{lb}/>}Newcastle, {<\textbf{lb}/>}On Her\mbox{}\newline 
{<\textbf{lb}/>}\mbox{}\newline 
\hspace*{6pt}{<\textbf{hi}\hspace*{6pt}{style}="{font-variant: normal}">}New Blazing-World{</\textbf{hi}>}. \mbox{}\newline 
{</\textbf{head}>}
\end{reflist}  
    \item[@rendition]
  pointe vers une description du rendu ou de la présentation utilisés pour cet élément dans le texte source
\begin{reflist}
    \item[{Statut}]
  Optionel
    \item[{Type de données}]
  1–∞ occurrences de \hyperref[TEI.teidata.pointer]{teidata.pointer} séparé par un espace
    \item[]\exampleFont {<\textbf{head}\hspace*{6pt}{rendition}="{\#ac \#sc}">}\mbox{}\newline 
\hspace*{6pt}{<\textbf{lb}/>}épître dédicatoire {<\textbf{lb}/>}à {<\textbf{lb}/>}Monsieur de Coucy {<\textbf{lb}/>}\mbox{}\newline 
{</\textbf{head}>}\mbox{}\newline 
{<\textbf{rendition}\hspace*{6pt}{scheme}="{css}"\mbox{}\newline 
\hspace*{6pt}{xml:id}="{fr\textunderscore sc}">}font-variant: uppercase{</\textbf{rendition}>}\mbox{}\newline 
{<\textbf{rendition}\hspace*{6pt}{scheme}="{css}"\mbox{}\newline 
\hspace*{6pt}{xml:id}="{fr\textunderscore ac}">}text-align: center{</\textbf{rendition}>}
    \item[{Note}]
  \par
L'attribut {\itshape rendition} est employé à peu près de la même manière que l'attribut {\itshape class} défini pour XHTML mais avec cette sérieuse différence que sa fonction est de décrire la présentation du texte source mais pas nécessairement de déterminer comment ce texte doit être représenté à l'écran ou sur le papier.\par
Où {\itshape rendition} et {\itshape rend} sont donnés ensembles, il faut comprendre que le dernier remplace ou complète le premier.\par
Chaque URI fourni doit indiquer un élément \hyperref[TEI.rendition]{<rendition>} définissant le rendu prévu dans les termes d'un langage approprié pour définir les styles, comme indiqué par l'attribut {\itshape scheme}.
\end{reflist}  
\end{sansreflist}  
\end{reflist}  
\begin{reflist}
\item[]\begin{specHead}{TEI.att.global.responsibility}{att.global.responsibility}\index{att.global.responsibility (attribute class)|oddindex}\index{cert=@cert!att.global.responsibility (attribute class)|oddindex}\index{resp=@resp!att.global.responsibility (attribute class)|oddindex} provides attributes indicating the agent responsible for some aspect of the text, the markup or something asserted by the markup, and the degree of certainty associated with it.\end{specHead} 
    \item[{Module}]
  tei
    \item[{Membres}]
  \hyperref[TEI.att.global]{att.global}[\hyperref[TEI.TEI]{TEI} \hyperref[TEI.ab]{ab} \hyperref[TEI.abbr]{abbr} \hyperref[TEI.abstract]{abstract} \hyperref[TEI.accMat]{accMat} \hyperref[TEI.acquisition]{acquisition} \hyperref[TEI.add]{add} \hyperref[TEI.addName]{addName} \hyperref[TEI.addSpan]{addSpan} \hyperref[TEI.additional]{additional} \hyperref[TEI.additions]{additions} \hyperref[TEI.addrLine]{addrLine} \hyperref[TEI.address]{address} \hyperref[TEI.adminInfo]{adminInfo} \hyperref[TEI.affiliation]{affiliation} \hyperref[TEI.alt]{alt} \hyperref[TEI.altGrp]{altGrp} \hyperref[TEI.altIdentifier]{altIdentifier} \hyperref[TEI.am]{am} \hyperref[TEI.analytic]{analytic} \hyperref[TEI.anchor]{anchor} \hyperref[TEI.annotationBlock]{annotationBlock} \hyperref[TEI.appInfo]{appInfo} \hyperref[TEI.application]{application} \hyperref[TEI.author]{author} \hyperref[TEI.authority]{authority} \hyperref[TEI.availability]{availability} \hyperref[TEI.back]{back} \hyperref[TEI.bibl]{bibl} \hyperref[TEI.biblFull]{biblFull} \hyperref[TEI.biblScope]{biblScope} \hyperref[TEI.biblStruct]{biblStruct} \hyperref[TEI.bicond]{bicond} \hyperref[TEI.binary]{binary} \hyperref[TEI.binaryObject]{binaryObject} \hyperref[TEI.binding]{binding} \hyperref[TEI.bindingDesc]{bindingDesc} \hyperref[TEI.body]{body} \hyperref[TEI.c]{c} \hyperref[TEI.catchwords]{catchwords} \hyperref[TEI.category]{category} \hyperref[TEI.cb]{cb} \hyperref[TEI.cell]{cell} \hyperref[TEI.change]{change} \hyperref[TEI.choice]{choice} \hyperref[TEI.cit]{cit} \hyperref[TEI.citedRange]{citedRange} \hyperref[TEI.cl]{cl} \hyperref[TEI.classCode]{classCode} \hyperref[TEI.classDecl]{classDecl} \hyperref[TEI.collation]{collation} \hyperref[TEI.collection]{collection} \hyperref[TEI.colophon]{colophon} \hyperref[TEI.cond]{cond} \hyperref[TEI.condition]{condition} \hyperref[TEI.corr]{corr} \hyperref[TEI.correction]{correction} \hyperref[TEI.country]{country} \hyperref[TEI.creation]{creation} \hyperref[TEI.custEvent]{custEvent} \hyperref[TEI.custodialHist]{custodialHist} \hyperref[TEI.damage]{damage} \hyperref[TEI.damageSpan]{damageSpan} \hyperref[TEI.date]{date} \hyperref[TEI.decoDesc]{decoDesc} \hyperref[TEI.decoNote]{decoNote} \hyperref[TEI.default]{default} \hyperref[TEI.del]{del} \hyperref[TEI.delSpan]{delSpan} \hyperref[TEI.depth]{depth} \hyperref[TEI.desc]{desc} \hyperref[TEI.dim]{dim} \hyperref[TEI.dimensions]{dimensions} \hyperref[TEI.distinct]{distinct} \hyperref[TEI.distributor]{distributor} \hyperref[TEI.div]{div} \hyperref[TEI.divGen]{divGen} \hyperref[TEI.docAuthor]{docAuthor} \hyperref[TEI.docDate]{docDate} \hyperref[TEI.docEdition]{docEdition} \hyperref[TEI.docTitle]{docTitle} \hyperref[TEI.edition]{edition} \hyperref[TEI.editionStmt]{editionStmt} \hyperref[TEI.editor]{editor} \hyperref[TEI.email]{email} \hyperref[TEI.emph]{emph} \hyperref[TEI.encodingDesc]{encodingDesc} \hyperref[TEI.event]{event} \hyperref[TEI.ex]{ex} \hyperref[TEI.expan]{expan} \hyperref[TEI.explicit]{explicit} \hyperref[TEI.extent]{extent} \hyperref[TEI.f]{f} \hyperref[TEI.fDecl]{fDecl} \hyperref[TEI.fDescr]{fDescr} \hyperref[TEI.fLib]{fLib} \hyperref[TEI.facsimile]{facsimile} \hyperref[TEI.figDesc]{figDesc} \hyperref[TEI.figure]{figure} \hyperref[TEI.fileDesc]{fileDesc} \hyperref[TEI.filiation]{filiation} \hyperref[TEI.finalRubric]{finalRubric} \hyperref[TEI.floatingText]{floatingText} \hyperref[TEI.foliation]{foliation} \hyperref[TEI.foreign]{foreign} \hyperref[TEI.forename]{forename} \hyperref[TEI.formula]{formula} \hyperref[TEI.front]{front} \hyperref[TEI.fs]{fs} \hyperref[TEI.fsConstraints]{fsConstraints} \hyperref[TEI.fsDecl]{fsDecl} \hyperref[TEI.fsDescr]{fsDescr} \hyperref[TEI.fsdDecl]{fsdDecl} \hyperref[TEI.fsdLink]{fsdLink} \hyperref[TEI.funder]{funder} \hyperref[TEI.fvLib]{fvLib} \hyperref[TEI.fw]{fw} \hyperref[TEI.gap]{gap} \hyperref[TEI.gb]{gb} \hyperref[TEI.genName]{genName} \hyperref[TEI.geogName]{geogName} \hyperref[TEI.gloss]{gloss} \hyperref[TEI.graphic]{graphic} \hyperref[TEI.group]{group} \hyperref[TEI.handDesc]{handDesc} \hyperref[TEI.handNotes]{handNotes} \hyperref[TEI.handShift]{handShift} \hyperref[TEI.head]{head} \hyperref[TEI.headItem]{headItem} \hyperref[TEI.headLabel]{headLabel} \hyperref[TEI.height]{height} \hyperref[TEI.heraldry]{heraldry} \hyperref[TEI.hi]{hi} \hyperref[TEI.history]{history} \hyperref[TEI.idno]{idno} \hyperref[TEI.if]{if} \hyperref[TEI.iff]{iff} \hyperref[TEI.imprint]{imprint} \hyperref[TEI.incipit]{incipit} \hyperref[TEI.index]{index} \hyperref[TEI.institution]{institution} \hyperref[TEI.interp]{interp} \hyperref[TEI.interpGrp]{interpGrp} \hyperref[TEI.item]{item} \hyperref[TEI.join]{join} \hyperref[TEI.joinGrp]{joinGrp} \hyperref[TEI.keywords]{keywords} \hyperref[TEI.l]{l} \hyperref[TEI.label]{label} \hyperref[TEI.langUsage]{langUsage} \hyperref[TEI.language]{language} \hyperref[TEI.layout]{layout} \hyperref[TEI.layoutDesc]{layoutDesc} \hyperref[TEI.lb]{lb} \hyperref[TEI.lg]{lg} \hyperref[TEI.licence]{licence} \hyperref[TEI.line]{line} \hyperref[TEI.link]{link} \hyperref[TEI.linkGrp]{linkGrp} \hyperref[TEI.list]{list} \hyperref[TEI.listAnnotation]{listAnnotation} \hyperref[TEI.listBibl]{listBibl} \hyperref[TEI.listOrg]{listOrg} \hyperref[TEI.listPlace]{listPlace} \hyperref[TEI.listTranspose]{listTranspose} \hyperref[TEI.location]{location} \hyperref[TEI.locus]{locus} \hyperref[TEI.locusGrp]{locusGrp} \hyperref[TEI.m]{m} \hyperref[TEI.material]{material} \hyperref[TEI.measure]{measure} \hyperref[TEI.measureGrp]{measureGrp} \hyperref[TEI.media]{media} \hyperref[TEI.meeting]{meeting} \hyperref[TEI.mentioned]{mentioned} \hyperref[TEI.metamark]{metamark} \hyperref[TEI.milestone]{milestone} \hyperref[TEI.mod]{mod} \hyperref[TEI.monogr]{monogr} \hyperref[TEI.msContents]{msContents} \hyperref[TEI.msDesc]{msDesc} \hyperref[TEI.msFrag]{msFrag} \hyperref[TEI.msIdentifier]{msIdentifier} \hyperref[TEI.msItem]{msItem} \hyperref[TEI.msItemStruct]{msItemStruct} \hyperref[TEI.msName]{msName} \hyperref[TEI.msPart]{msPart} \hyperref[TEI.musicNotation]{musicNotation} \hyperref[TEI.name]{name} \hyperref[TEI.nameLink]{nameLink} \hyperref[TEI.namespace]{namespace} \hyperref[TEI.notatedMusic]{notatedMusic} \hyperref[TEI.note]{note} \hyperref[TEI.notesStmt]{notesStmt} \hyperref[TEI.num]{num} \hyperref[TEI.numeric]{numeric} \hyperref[TEI.objectDesc]{objectDesc} \hyperref[TEI.objectType]{objectType} \hyperref[TEI.org]{org} \hyperref[TEI.orgName]{orgName} \hyperref[TEI.orig]{orig} \hyperref[TEI.origDate]{origDate} \hyperref[TEI.origPlace]{origPlace} \hyperref[TEI.origin]{origin} \hyperref[TEI.p]{p} \hyperref[TEI.pb]{pb} \hyperref[TEI.pc]{pc} \hyperref[TEI.persName]{persName} \hyperref[TEI.person]{person} \hyperref[TEI.personGrp]{personGrp} \hyperref[TEI.persona]{persona} \hyperref[TEI.phr]{phr} \hyperref[TEI.physDesc]{physDesc} \hyperref[TEI.place]{place} \hyperref[TEI.placeName]{placeName} \hyperref[TEI.postBox]{postBox} \hyperref[TEI.postCode]{postCode} \hyperref[TEI.profileDesc]{profileDesc} \hyperref[TEI.provenance]{provenance} \hyperref[TEI.ptr]{ptr} \hyperref[TEI.pubPlace]{pubPlace} \hyperref[TEI.publicationStmt]{publicationStmt} \hyperref[TEI.publisher]{publisher} \hyperref[TEI.q]{q} \hyperref[TEI.quote]{quote} \hyperref[TEI.recordHist]{recordHist} \hyperref[TEI.redo]{redo} \hyperref[TEI.ref]{ref} \hyperref[TEI.reg]{reg} \hyperref[TEI.region]{region} \hyperref[TEI.relatedItem]{relatedItem} \hyperref[TEI.rendition]{rendition} \hyperref[TEI.repository]{repository} \hyperref[TEI.resp]{resp} \hyperref[TEI.respStmt]{respStmt} \hyperref[TEI.restore]{restore} \hyperref[TEI.retrace]{retrace} \hyperref[TEI.revisionDesc]{revisionDesc} \hyperref[TEI.roleName]{roleName} \hyperref[TEI.row]{row} \hyperref[TEI.rs]{rs} \hyperref[TEI.rubric]{rubric} \hyperref[TEI.s]{s} \hyperref[TEI.said]{said} \hyperref[TEI.schemaRef]{schemaRef} \hyperref[TEI.scriptDesc]{scriptDesc} \hyperref[TEI.seal]{seal} \hyperref[TEI.sealDesc]{sealDesc} \hyperref[TEI.secFol]{secFol} \hyperref[TEI.secl]{secl} \hyperref[TEI.seg]{seg} \hyperref[TEI.series]{series} \hyperref[TEI.seriesStmt]{seriesStmt} \hyperref[TEI.settlement]{settlement} \hyperref[TEI.sic]{sic} \hyperref[TEI.signatures]{signatures} \hyperref[TEI.soCalled]{soCalled} \hyperref[TEI.source]{source} \hyperref[TEI.sourceDesc]{sourceDesc} \hyperref[TEI.sourceDoc]{sourceDoc} \hyperref[TEI.sp]{sp} \hyperref[TEI.span]{span} \hyperref[TEI.spanGrp]{spanGrp} \hyperref[TEI.speaker]{speaker} \hyperref[TEI.stage]{stage} \hyperref[TEI.stamp]{stamp} \hyperref[TEI.standOff]{standOff} \hyperref[TEI.state]{state} \hyperref[TEI.street]{street} \hyperref[TEI.string]{string} \hyperref[TEI.subst]{subst} \hyperref[TEI.substJoin]{substJoin} \hyperref[TEI.summary]{summary} \hyperref[TEI.supplied]{supplied} \hyperref[TEI.support]{support} \hyperref[TEI.supportDesc]{supportDesc} \hyperref[TEI.surface]{surface} \hyperref[TEI.surfaceGrp]{surfaceGrp} \hyperref[TEI.surname]{surname} \hyperref[TEI.surplus]{surplus} \hyperref[TEI.surrogates]{surrogates} \hyperref[TEI.symbol]{symbol} \hyperref[TEI.table]{table} \hyperref[TEI.taxonomy]{taxonomy} \hyperref[TEI.teiCorpus]{teiCorpus} \hyperref[TEI.teiHeader]{teiHeader} \hyperref[TEI.term]{term} \hyperref[TEI.text]{text} \hyperref[TEI.textClass]{textClass} \hyperref[TEI.textLang]{textLang} \hyperref[TEI.then]{then} \hyperref[TEI.time]{time} \hyperref[TEI.timeline]{timeline} \hyperref[TEI.title]{title} \hyperref[TEI.titlePage]{titlePage} \hyperref[TEI.titlePart]{titlePart} \hyperref[TEI.titleStmt]{titleStmt} \hyperref[TEI.transpose]{transpose} \hyperref[TEI.typeDesc]{typeDesc} \hyperref[TEI.typeNote]{typeNote} \hyperref[TEI.unclear]{unclear} \hyperref[TEI.undo]{undo} \hyperref[TEI.vAlt]{vAlt} \hyperref[TEI.vColl]{vColl} \hyperref[TEI.vDefault]{vDefault} \hyperref[TEI.vLabel]{vLabel} \hyperref[TEI.vMerge]{vMerge} \hyperref[TEI.vNot]{vNot} \hyperref[TEI.vRange]{vRange} \hyperref[TEI.w]{w} \hyperref[TEI.watermark]{watermark} \hyperref[TEI.when]{when} \hyperref[TEI.width]{width} \hyperref[TEI.zone]{zone}]
    \item[{Attributs}]
  Attributs\hfil\\[-10pt]\begin{sansreflist}
    \item[@cert]
  (certitude) donne le degré de certitude associée à l'intervention ou à l'interprétation.
\begin{reflist}
    \item[{Statut}]
  Optionel
    \item[{Type de données}]
  \hyperref[TEI.teidata.probCert]{teidata.probCert}
\end{reflist}  
    \item[@resp]
  (responsable) indique l'agent responsable de l'intervention ou de l'interprétation, par exemple un éditeur ou un transcripteur.
\begin{reflist}
    \item[{Statut}]
  Optionel
    \item[{Type de données}]
  1–∞ occurrences de \hyperref[TEI.teidata.pointer]{teidata.pointer} séparé par un espace
\end{reflist}  
\end{sansreflist}  
    \item[{Exemple}]
  \leavevmode\bgroup\exampleFont \begin{shaded}\noindent\mbox{}Blessed are the\mbox{}\newline 
{<\textbf{choice}>}\mbox{}\newline 
\hspace*{6pt}{<\textbf{sic}>}cheesemakers{</\textbf{sic}>}\mbox{}\newline 
\hspace*{6pt}{<\textbf{corr}\hspace*{6pt}{cert}="{high}"\hspace*{6pt}{resp}="{\#editor}">}peacemakers{</\textbf{corr}>}\mbox{}\newline 
{</\textbf{choice}>}: for they shall be called the children of God.\end{shaded}\egroup 


    \item[{Exemple}]
  \leavevmode\bgroup\exampleFont \begin{shaded}\noindent\mbox{}{<\textbf{lg}>}\mbox{}\newline 
\hspace*{6pt}{<\textbf{l}>}Punkes, Panders, baſe extortionizing\mbox{}\newline 
\hspace*{6pt}\hspace*{6pt} sla{<\textbf{choice}>}\mbox{}\newline 
\hspace*{6pt}\hspace*{6pt}\hspace*{6pt}{<\textbf{sic}>}n{</\textbf{sic}>}\mbox{}\newline 
\hspace*{6pt}\hspace*{6pt}\hspace*{6pt}{<\textbf{corr}\hspace*{6pt}{resp}="{\#JENS1\textunderscore transcriber}">}u{</\textbf{corr}>}\mbox{}\newline 
\hspace*{6pt}\hspace*{6pt}{</\textbf{choice}>}es,{</\textbf{l}>}\mbox{}\newline 
{</\textbf{lg}>}\mbox{}\newline 
{<\textbf{respStmt}\hspace*{6pt}{xml:id}="{JENS1\textunderscore transcriber}">}\mbox{}\newline 
\hspace*{6pt}{<\textbf{resp}\hspace*{6pt}{when}="{2014}">}Transcriber{</\textbf{resp}>}\mbox{}\newline 
\hspace*{6pt}{<\textbf{name}>}Janelle Jenstad{</\textbf{name}>}\mbox{}\newline 
{</\textbf{respStmt}>}\end{shaded}\egroup 


\end{reflist}  
\begin{reflist}
\item[]\begin{specHead}{TEI.att.global.source}{att.global.source}\index{att.global.source (attribute class)|oddindex}\index{source=@source!att.global.source (attribute class)|oddindex} provides an attribute used by elements to point to an external source.\end{specHead} 
    \item[{Module}]
  tei
    \item[{Membres}]
  \hyperref[TEI.att.global]{att.global}[\hyperref[TEI.TEI]{TEI} \hyperref[TEI.ab]{ab} \hyperref[TEI.abbr]{abbr} \hyperref[TEI.abstract]{abstract} \hyperref[TEI.accMat]{accMat} \hyperref[TEI.acquisition]{acquisition} \hyperref[TEI.add]{add} \hyperref[TEI.addName]{addName} \hyperref[TEI.addSpan]{addSpan} \hyperref[TEI.additional]{additional} \hyperref[TEI.additions]{additions} \hyperref[TEI.addrLine]{addrLine} \hyperref[TEI.address]{address} \hyperref[TEI.adminInfo]{adminInfo} \hyperref[TEI.affiliation]{affiliation} \hyperref[TEI.alt]{alt} \hyperref[TEI.altGrp]{altGrp} \hyperref[TEI.altIdentifier]{altIdentifier} \hyperref[TEI.am]{am} \hyperref[TEI.analytic]{analytic} \hyperref[TEI.anchor]{anchor} \hyperref[TEI.annotationBlock]{annotationBlock} \hyperref[TEI.appInfo]{appInfo} \hyperref[TEI.application]{application} \hyperref[TEI.author]{author} \hyperref[TEI.authority]{authority} \hyperref[TEI.availability]{availability} \hyperref[TEI.back]{back} \hyperref[TEI.bibl]{bibl} \hyperref[TEI.biblFull]{biblFull} \hyperref[TEI.biblScope]{biblScope} \hyperref[TEI.biblStruct]{biblStruct} \hyperref[TEI.bicond]{bicond} \hyperref[TEI.binary]{binary} \hyperref[TEI.binaryObject]{binaryObject} \hyperref[TEI.binding]{binding} \hyperref[TEI.bindingDesc]{bindingDesc} \hyperref[TEI.body]{body} \hyperref[TEI.c]{c} \hyperref[TEI.catchwords]{catchwords} \hyperref[TEI.category]{category} \hyperref[TEI.cb]{cb} \hyperref[TEI.cell]{cell} \hyperref[TEI.change]{change} \hyperref[TEI.choice]{choice} \hyperref[TEI.cit]{cit} \hyperref[TEI.citedRange]{citedRange} \hyperref[TEI.cl]{cl} \hyperref[TEI.classCode]{classCode} \hyperref[TEI.classDecl]{classDecl} \hyperref[TEI.collation]{collation} \hyperref[TEI.collection]{collection} \hyperref[TEI.colophon]{colophon} \hyperref[TEI.cond]{cond} \hyperref[TEI.condition]{condition} \hyperref[TEI.corr]{corr} \hyperref[TEI.correction]{correction} \hyperref[TEI.country]{country} \hyperref[TEI.creation]{creation} \hyperref[TEI.custEvent]{custEvent} \hyperref[TEI.custodialHist]{custodialHist} \hyperref[TEI.damage]{damage} \hyperref[TEI.damageSpan]{damageSpan} \hyperref[TEI.date]{date} \hyperref[TEI.decoDesc]{decoDesc} \hyperref[TEI.decoNote]{decoNote} \hyperref[TEI.default]{default} \hyperref[TEI.del]{del} \hyperref[TEI.delSpan]{delSpan} \hyperref[TEI.depth]{depth} \hyperref[TEI.desc]{desc} \hyperref[TEI.dim]{dim} \hyperref[TEI.dimensions]{dimensions} \hyperref[TEI.distinct]{distinct} \hyperref[TEI.distributor]{distributor} \hyperref[TEI.div]{div} \hyperref[TEI.divGen]{divGen} \hyperref[TEI.docAuthor]{docAuthor} \hyperref[TEI.docDate]{docDate} \hyperref[TEI.docEdition]{docEdition} \hyperref[TEI.docTitle]{docTitle} \hyperref[TEI.edition]{edition} \hyperref[TEI.editionStmt]{editionStmt} \hyperref[TEI.editor]{editor} \hyperref[TEI.email]{email} \hyperref[TEI.emph]{emph} \hyperref[TEI.encodingDesc]{encodingDesc} \hyperref[TEI.event]{event} \hyperref[TEI.ex]{ex} \hyperref[TEI.expan]{expan} \hyperref[TEI.explicit]{explicit} \hyperref[TEI.extent]{extent} \hyperref[TEI.f]{f} \hyperref[TEI.fDecl]{fDecl} \hyperref[TEI.fDescr]{fDescr} \hyperref[TEI.fLib]{fLib} \hyperref[TEI.facsimile]{facsimile} \hyperref[TEI.figDesc]{figDesc} \hyperref[TEI.figure]{figure} \hyperref[TEI.fileDesc]{fileDesc} \hyperref[TEI.filiation]{filiation} \hyperref[TEI.finalRubric]{finalRubric} \hyperref[TEI.floatingText]{floatingText} \hyperref[TEI.foliation]{foliation} \hyperref[TEI.foreign]{foreign} \hyperref[TEI.forename]{forename} \hyperref[TEI.formula]{formula} \hyperref[TEI.front]{front} \hyperref[TEI.fs]{fs} \hyperref[TEI.fsConstraints]{fsConstraints} \hyperref[TEI.fsDecl]{fsDecl} \hyperref[TEI.fsDescr]{fsDescr} \hyperref[TEI.fsdDecl]{fsdDecl} \hyperref[TEI.fsdLink]{fsdLink} \hyperref[TEI.funder]{funder} \hyperref[TEI.fvLib]{fvLib} \hyperref[TEI.fw]{fw} \hyperref[TEI.gap]{gap} \hyperref[TEI.gb]{gb} \hyperref[TEI.genName]{genName} \hyperref[TEI.geogName]{geogName} \hyperref[TEI.gloss]{gloss} \hyperref[TEI.graphic]{graphic} \hyperref[TEI.group]{group} \hyperref[TEI.handDesc]{handDesc} \hyperref[TEI.handNotes]{handNotes} \hyperref[TEI.handShift]{handShift} \hyperref[TEI.head]{head} \hyperref[TEI.headItem]{headItem} \hyperref[TEI.headLabel]{headLabel} \hyperref[TEI.height]{height} \hyperref[TEI.heraldry]{heraldry} \hyperref[TEI.hi]{hi} \hyperref[TEI.history]{history} \hyperref[TEI.idno]{idno} \hyperref[TEI.if]{if} \hyperref[TEI.iff]{iff} \hyperref[TEI.imprint]{imprint} \hyperref[TEI.incipit]{incipit} \hyperref[TEI.index]{index} \hyperref[TEI.institution]{institution} \hyperref[TEI.interp]{interp} \hyperref[TEI.interpGrp]{interpGrp} \hyperref[TEI.item]{item} \hyperref[TEI.join]{join} \hyperref[TEI.joinGrp]{joinGrp} \hyperref[TEI.keywords]{keywords} \hyperref[TEI.l]{l} \hyperref[TEI.label]{label} \hyperref[TEI.langUsage]{langUsage} \hyperref[TEI.language]{language} \hyperref[TEI.layout]{layout} \hyperref[TEI.layoutDesc]{layoutDesc} \hyperref[TEI.lb]{lb} \hyperref[TEI.lg]{lg} \hyperref[TEI.licence]{licence} \hyperref[TEI.line]{line} \hyperref[TEI.link]{link} \hyperref[TEI.linkGrp]{linkGrp} \hyperref[TEI.list]{list} \hyperref[TEI.listAnnotation]{listAnnotation} \hyperref[TEI.listBibl]{listBibl} \hyperref[TEI.listOrg]{listOrg} \hyperref[TEI.listPlace]{listPlace} \hyperref[TEI.listTranspose]{listTranspose} \hyperref[TEI.location]{location} \hyperref[TEI.locus]{locus} \hyperref[TEI.locusGrp]{locusGrp} \hyperref[TEI.m]{m} \hyperref[TEI.material]{material} \hyperref[TEI.measure]{measure} \hyperref[TEI.measureGrp]{measureGrp} \hyperref[TEI.media]{media} \hyperref[TEI.meeting]{meeting} \hyperref[TEI.mentioned]{mentioned} \hyperref[TEI.metamark]{metamark} \hyperref[TEI.milestone]{milestone} \hyperref[TEI.mod]{mod} \hyperref[TEI.monogr]{monogr} \hyperref[TEI.msContents]{msContents} \hyperref[TEI.msDesc]{msDesc} \hyperref[TEI.msFrag]{msFrag} \hyperref[TEI.msIdentifier]{msIdentifier} \hyperref[TEI.msItem]{msItem} \hyperref[TEI.msItemStruct]{msItemStruct} \hyperref[TEI.msName]{msName} \hyperref[TEI.msPart]{msPart} \hyperref[TEI.musicNotation]{musicNotation} \hyperref[TEI.name]{name} \hyperref[TEI.nameLink]{nameLink} \hyperref[TEI.namespace]{namespace} \hyperref[TEI.notatedMusic]{notatedMusic} \hyperref[TEI.note]{note} \hyperref[TEI.notesStmt]{notesStmt} \hyperref[TEI.num]{num} \hyperref[TEI.numeric]{numeric} \hyperref[TEI.objectDesc]{objectDesc} \hyperref[TEI.objectType]{objectType} \hyperref[TEI.org]{org} \hyperref[TEI.orgName]{orgName} \hyperref[TEI.orig]{orig} \hyperref[TEI.origDate]{origDate} \hyperref[TEI.origPlace]{origPlace} \hyperref[TEI.origin]{origin} \hyperref[TEI.p]{p} \hyperref[TEI.pb]{pb} \hyperref[TEI.pc]{pc} \hyperref[TEI.persName]{persName} \hyperref[TEI.person]{person} \hyperref[TEI.personGrp]{personGrp} \hyperref[TEI.persona]{persona} \hyperref[TEI.phr]{phr} \hyperref[TEI.physDesc]{physDesc} \hyperref[TEI.place]{place} \hyperref[TEI.placeName]{placeName} \hyperref[TEI.postBox]{postBox} \hyperref[TEI.postCode]{postCode} \hyperref[TEI.profileDesc]{profileDesc} \hyperref[TEI.provenance]{provenance} \hyperref[TEI.ptr]{ptr} \hyperref[TEI.pubPlace]{pubPlace} \hyperref[TEI.publicationStmt]{publicationStmt} \hyperref[TEI.publisher]{publisher} \hyperref[TEI.q]{q} \hyperref[TEI.quote]{quote} \hyperref[TEI.recordHist]{recordHist} \hyperref[TEI.redo]{redo} \hyperref[TEI.ref]{ref} \hyperref[TEI.reg]{reg} \hyperref[TEI.region]{region} \hyperref[TEI.relatedItem]{relatedItem} \hyperref[TEI.rendition]{rendition} \hyperref[TEI.repository]{repository} \hyperref[TEI.resp]{resp} \hyperref[TEI.respStmt]{respStmt} \hyperref[TEI.restore]{restore} \hyperref[TEI.retrace]{retrace} \hyperref[TEI.revisionDesc]{revisionDesc} \hyperref[TEI.roleName]{roleName} \hyperref[TEI.row]{row} \hyperref[TEI.rs]{rs} \hyperref[TEI.rubric]{rubric} \hyperref[TEI.s]{s} \hyperref[TEI.said]{said} \hyperref[TEI.schemaRef]{schemaRef} \hyperref[TEI.scriptDesc]{scriptDesc} \hyperref[TEI.seal]{seal} \hyperref[TEI.sealDesc]{sealDesc} \hyperref[TEI.secFol]{secFol} \hyperref[TEI.secl]{secl} \hyperref[TEI.seg]{seg} \hyperref[TEI.series]{series} \hyperref[TEI.seriesStmt]{seriesStmt} \hyperref[TEI.settlement]{settlement} \hyperref[TEI.sic]{sic} \hyperref[TEI.signatures]{signatures} \hyperref[TEI.soCalled]{soCalled} \hyperref[TEI.source]{source} \hyperref[TEI.sourceDesc]{sourceDesc} \hyperref[TEI.sourceDoc]{sourceDoc} \hyperref[TEI.sp]{sp} \hyperref[TEI.span]{span} \hyperref[TEI.spanGrp]{spanGrp} \hyperref[TEI.speaker]{speaker} \hyperref[TEI.stage]{stage} \hyperref[TEI.stamp]{stamp} \hyperref[TEI.standOff]{standOff} \hyperref[TEI.state]{state} \hyperref[TEI.street]{street} \hyperref[TEI.string]{string} \hyperref[TEI.subst]{subst} \hyperref[TEI.substJoin]{substJoin} \hyperref[TEI.summary]{summary} \hyperref[TEI.supplied]{supplied} \hyperref[TEI.support]{support} \hyperref[TEI.supportDesc]{supportDesc} \hyperref[TEI.surface]{surface} \hyperref[TEI.surfaceGrp]{surfaceGrp} \hyperref[TEI.surname]{surname} \hyperref[TEI.surplus]{surplus} \hyperref[TEI.surrogates]{surrogates} \hyperref[TEI.symbol]{symbol} \hyperref[TEI.table]{table} \hyperref[TEI.taxonomy]{taxonomy} \hyperref[TEI.teiCorpus]{teiCorpus} \hyperref[TEI.teiHeader]{teiHeader} \hyperref[TEI.term]{term} \hyperref[TEI.text]{text} \hyperref[TEI.textClass]{textClass} \hyperref[TEI.textLang]{textLang} \hyperref[TEI.then]{then} \hyperref[TEI.time]{time} \hyperref[TEI.timeline]{timeline} \hyperref[TEI.title]{title} \hyperref[TEI.titlePage]{titlePage} \hyperref[TEI.titlePart]{titlePart} \hyperref[TEI.titleStmt]{titleStmt} \hyperref[TEI.transpose]{transpose} \hyperref[TEI.typeDesc]{typeDesc} \hyperref[TEI.typeNote]{typeNote} \hyperref[TEI.unclear]{unclear} \hyperref[TEI.undo]{undo} \hyperref[TEI.vAlt]{vAlt} \hyperref[TEI.vColl]{vColl} \hyperref[TEI.vDefault]{vDefault} \hyperref[TEI.vLabel]{vLabel} \hyperref[TEI.vMerge]{vMerge} \hyperref[TEI.vNot]{vNot} \hyperref[TEI.vRange]{vRange} \hyperref[TEI.w]{w} \hyperref[TEI.watermark]{watermark} \hyperref[TEI.when]{when} \hyperref[TEI.width]{width} \hyperref[TEI.zone]{zone}]
    \item[{Attributs}]
  Attributs\hfil\\[-10pt]\begin{sansreflist}
    \item[@source]
  specifies the source from which some aspect of this element is drawn.
\begin{reflist}
    \item[{Statut}]
  Optionel
    \item[{Type de données}]
  1–∞ occurrences de \hyperref[TEI.teidata.pointer]{teidata.pointer} séparé par un espace
\end{reflist}  
\end{sansreflist}  
    \item[{Exemple}]
  \leavevmode\bgroup\exampleFont \begin{shaded}\noindent\mbox{}{<\textbf{p}>} As Willard McCarty ({<\textbf{bibl}\hspace*{6pt}{xml:id}="{mcc\textunderscore 2012}">}2012, p.2{</\textbf{bibl}>}) tells us, {<\textbf{quote}\hspace*{6pt}{source}="{\#mcc\textunderscore 2012}">}‘Collaboration’ is a problematic and should be a contested\mbox{}\newline 
\hspace*{6pt}\hspace*{6pt} term.{</\textbf{quote}>}\mbox{}\newline 
{</\textbf{p}>}\end{shaded}\egroup 


    \item[{Exemple}]
  \leavevmode\bgroup\exampleFont \begin{shaded}\noindent\mbox{}{<\textbf{p}>}\mbox{}\newline 
\hspace*{6pt}{<\textbf{quote}\hspace*{6pt}{source}="{\#chicago\textunderscore 15\textunderscore ed}">}Grammatical theories are in flux, and the more we learn, the\mbox{}\newline 
\hspace*{6pt}\hspace*{6pt} less we seem to know.{</\textbf{quote}>}\mbox{}\newline 
{</\textbf{p}>}\mbox{}\newline 
{<\textbf{bibl}\hspace*{6pt}{xml:id}="{chicago\textunderscore 15\textunderscore ed}">}\mbox{}\newline 
\hspace*{6pt}{<\textbf{title}\hspace*{6pt}{level}="{m}">}The Chicago Manual of Style{</\textbf{title}>},\mbox{}\newline 
{<\textbf{edition}>}15th edition{</\textbf{edition}>}. {<\textbf{pubPlace}>}Chicago{</\textbf{pubPlace}>}: {<\textbf{publisher}>}University of\mbox{}\newline 
\hspace*{6pt}\hspace*{6pt} Chicago Press{</\textbf{publisher}>} ({<\textbf{date}>}2003{</\textbf{date}>}), {<\textbf{biblScope}\hspace*{6pt}{unit}="{page}">}p.147{</\textbf{biblScope}>}.\mbox{}\newline 
\mbox{}\newline 
{</\textbf{bibl}>}\end{shaded}\egroup 


    \item[{Exemple}]
  \leavevmode\bgroup\exampleFont \begin{shaded}\noindent\mbox{}{<\textbf{elementRef}\hspace*{6pt}{key}="{p}"\hspace*{6pt}{source}="{tei:2.0.1}"/>}\end{shaded}\egroup 

Include in the schema an element named \hyperref[TEI.p]{<p>} available from the TEI P5 2.0.1 release.
    \item[{Exemple}]
  \leavevmode\bgroup\exampleFont \begin{shaded}\noindent\mbox{}{<\textbf{schemaSpec}\hspace*{6pt}{ident}="{myODD}"\mbox{}\newline 
\hspace*{6pt}{source}="{mycompiledODD.xml}"/>}\end{shaded}\egroup 

Create a schema using components taken from the file \textsf{mycompiledODD.xml}.
\end{reflist}  
\begin{reflist}
\item[]\begin{specHead}{TEI.att.handFeatures}{att.handFeatures}\index{att.handFeatures (attribute class)|oddindex}\index{scribe=@scribe!att.handFeatures (attribute class)|oddindex}\index{scribeRef=@scribeRef!att.handFeatures (attribute class)|oddindex}\index{script=@script!att.handFeatures (attribute class)|oddindex}\index{scriptRef=@scriptRef!att.handFeatures (attribute class)|oddindex}\index{medium=@medium!att.handFeatures (attribute class)|oddindex}\index{scope=@scope!att.handFeatures (attribute class)|oddindex} fournit des attributs décrivant les caractéristiques de la main par laquelle un manuscrit est écrit.\end{specHead} 
    \item[{Module}]
  tei
    \item[{Membres}]
  \hyperref[TEI.handShift]{handShift} \hyperref[TEI.typeNote]{typeNote}
    \item[{Attributs}]
  Attributs\hfil\\[-10pt]\begin{sansreflist}
    \item[@scribe]
  donne un nom normalisé ou un autre identifiant pour le scribe reconnu comme responsable de cette main.
\begin{reflist}
    \item[{Statut}]
  Optionel
    \item[{Type de données}]
  \hyperref[TEI.teidata.name]{teidata.name}
\end{reflist}  
    \item[@scribeRef]
  points to a full description of the scribe concerned, typically supplied by a \hyperref[TEI.person]{<person>} element elsewhere in the description.
\begin{reflist}
    \item[{Statut}]
  Optionel
    \item[{Type de données}]
  1–∞ occurrences de \hyperref[TEI.teidata.pointer]{teidata.pointer} séparé par un espace
\end{reflist}  
    \item[@script]
  caractérise la calligraphie ou le style d'écriture particuliers utilisés par cette main, par exemple \textit{écriture anglaise}, \textit{de chancellerie}, \textit{italienne}, etc.
\begin{reflist}
    \item[{Statut}]
  Optionel
    \item[{Type de données}]
  1–∞ occurrences de \hyperref[TEI.teidata.name]{teidata.name} séparé par un espace
\end{reflist}  
    \item[@scriptRef]
  points to a full description of the script or writing style used by this hand, typically supplied by a \texttt{<scriptNote>} element elsewhere in the description.
\begin{reflist}
    \item[{Statut}]
  Optionel
    \item[{Type de données}]
  1–∞ occurrences de \hyperref[TEI.teidata.pointer]{teidata.pointer} séparé par un espace
\end{reflist}  
    \item[@medium]
  décrit la teinte ou le type d'encre, par exemple \textit{brune}, ou un autre outil d'écriture, par exemple un \textit{crayon}.
\begin{reflist}
    \item[{Statut}]
  Optionel
    \item[{Type de données}]
  1–∞ occurrences de \hyperref[TEI.teidata.enumerated]{teidata.enumerated} séparé par un espace
\end{reflist}  
    \item[@scope]
  Spécifie la fréquence d'apparition de cette main dans le manuscrit.
\begin{reflist}
    \item[{Statut}]
  Optionel
    \item[{Type de données}]
  \hyperref[TEI.teidata.enumerated]{teidata.enumerated}
    \item[{Les valeurs autorisées sont:}]
  \begin{description}

\item[{sole}]il n'y a que cette main dans le manuscrit.
\item[{major}]cette main est utilisée dans la majeure partie du manuscrit.
\item[{minor}]cette main est utilisée occasionnellement dans le manuscrit.
\end{description} 
\end{reflist}  
\end{sansreflist}  
\end{reflist}  
\begin{reflist}
\item[]\begin{specHead}{TEI.att.internetMedia}{att.internetMedia}\index{att.internetMedia (attribute class)|oddindex}\index{mimeType=@mimeType!att.internetMedia (attribute class)|oddindex} fournit des attributs pour spécifier le type de ressource informatique selon une taxinomie normalisée.\end{specHead} 
    \item[{Module}]
  tei
    \item[{Membres}]
  \hyperref[TEI.att.media]{att.media}[\hyperref[TEI.binaryObject]{binaryObject} \hyperref[TEI.graphic]{graphic}] \hyperref[TEI.ptr]{ptr} \hyperref[TEI.ref]{ref}
    \item[{Attributs}]
  Attributs\hfil\\[-10pt]\begin{sansreflist}
    \item[@mimeType]
  (type de média MIME) spécifie le type MIME (multipurpose internet mail extension) applicable.
\begin{reflist}
    \item[{Statut}]
  Optionel
    \item[{Type de données}]
  1–∞ occurrences de \hyperref[TEI.teidata.word]{teidata.word} séparé par un espace
\end{reflist}  
\end{sansreflist}  
    \item[{Exemple}]
  In this example {\itshape mimeType} is used to indicate that the URL points to a TEI XML file encoded in UTF-8.\leavevmode\bgroup\exampleFont \begin{shaded}\noindent\mbox{}{<\textbf{ref}\hspace*{6pt}{mimeType}="{application/tei+xml; charset=UTF-8}"\mbox{}\newline 
\hspace*{6pt}{target}="{http://sourceforge.net/p/tei/code/HEAD/tree/trunk/P5/Source/guidelines-en.xml}"/>}\end{shaded}\egroup 


    \item[{Note}]
  \par
Cette classe d'attributs fournit des attributs pour décrire une ressource informatique, en général disponible sur internet, selon les taxinomies normalisées. Actuellement une seule taxinomie est reconnue : le système "Multipurpose Internet Mail Extensions Media Type". Ce système de typologie des types de média est définie par l'Internet Engineering Task Force dans\xref{http://www.ietf.org/rfc/rfc2046.txt}{RFC 2046}. La \xref{http://www.iana.org/assignments/media-types/}{liste des types} est maintenue par l'Internet Assigned Numbers Authority.
\end{reflist}  
\begin{reflist}
\item[]\begin{specHead}{TEI.att.interpLike}{att.interpLike}\index{att.interpLike (attribute class)|oddindex}\index{type=@type!att.interpLike (attribute class)|oddindex}\index{inst=@inst!att.interpLike (attribute class)|oddindex} fournit les attributs pour des éléments qui exposent une analyse ou une interprétation formelles.\end{specHead} 
    \item[{Module}]
  tei
    \item[{Membres}]
  \hyperref[TEI.interp]{interp} \hyperref[TEI.interpGrp]{interpGrp} \hyperref[TEI.span]{span} \hyperref[TEI.spanGrp]{spanGrp}
    \item[{Attributs}]
  Attributs\hfil\\[-10pt]\begin{sansreflist}
    \item[@type]
  indique quel genre de phénomène est noté dans le passage.
\begin{reflist}
    \item[{Statut}]
  Recommendé
    \item[{Type de données}]
  \hyperref[TEI.teidata.enumerated]{teidata.enumerated}
    \item[{Exemple de valeurs possibles:}]
  \begin{description}

\item[{image}]identifie une image dans le passage.
\item[{character}]identifie un personnage associé au passage.
\item[{theme}]identifie un thème dans le passage.
\item[{allusion}]identifie une allusion à un autre texte.
\end{description} 
\end{reflist}  
    \item[@inst]
  (cas) pointe vers les instances de l'analyse ou de l'interprétation représentées par l'élément courant.
\begin{reflist}
    \item[{Statut}]
  Optionel
    \item[{Type de données}]
  1–∞ occurrences de \hyperref[TEI.teidata.pointer]{teidata.pointer} séparé par un espace
    \item[{Note}]
  \par
L'élément courant doit être analytique. L'élément pointé doit être textuel. 
\end{reflist}  
\end{sansreflist}  
\end{reflist}  
\begin{reflist}
\item[]\begin{specHead}{TEI.att.measurement}{att.measurement}\index{att.measurement (attribute class)|oddindex}\index{unit=@unit!att.measurement (attribute class)|oddindex}\index{quantity=@quantity!att.measurement (attribute class)|oddindex}\index{commodity=@commodity!att.measurement (attribute class)|oddindex} donne des attributs pour représenter une mesure régularisée ou normalisée.\end{specHead} 
    \item[{Module}]
  tei
    \item[{Membres}]
  \hyperref[TEI.measure]{measure} \hyperref[TEI.measureGrp]{measureGrp}
    \item[{Attributs}]
  Attributs\hfil\\[-10pt]\begin{sansreflist}
    \item[@unit]
  (unité) indique les unités de mesure utilisées ; il s'agit en général du symbole normalisé pour les unités dont on a besoin.
\begin{reflist}
    \item[{Statut}]
  Optionel
    \item[{Type de données}]
  \hyperref[TEI.teidata.enumerated]{teidata.enumerated}
    \item[{Les valeurs suggérées comprennent:}]
  \begin{description}

\item[{m}](mètre) unité SI (système international) de longueur
\item[{kg}](kilogramme) unité SI de masse
\item[{s}](seconde) unité SI de temps
\item[{Hz}](hertz) unité SI de fréquence
\item[{Pa}](pascal) unité SI de pression
\item[{Ω}](ohm) unité SI de résistance électrique
\item[{L}](litre) 1 dm³
\item[{t}](tonne) 10³ kg
\item[{ha}](hectare) 1 hm²
\item[{Å}](ångström) 10⁻¹⁰ m
\item[{mL}](millilitre)
\item[{cm}](centimètre)
\item[{dB}](décibel) Voir remarques, ci-dessous.
\item[{kbit}](kilobit) 10³ ou 1000 bits
\item[{Kibit}](kibibit) 2¹⁰ ou 1024 bits
\item[{kB}](kilo-octet) 10³ ou 1000 octets
\item[{KiB}](kibioctet) 2¹⁰ ou 1024 octets
\item[{MB}](mégaoctet) 10⁶ ou 1 000 000 octets
\item[{MiB}](mébioctet) 2²⁰ ou 1 048 576 octets
\end{description} 
    \item[{Note}]
  \par
Si la mesure représentée n'est pas exprimée dans une unité particulière mais plutôt comme un certain nombre d'items distincts, l'unité count doit être employée ou l'attribut {\itshape unit} peut être laissé comme non spécifié.\par
Partout où c'est approprié, un nom d'unité reconnu par le SI (système international) doit être employé (voir plus loin \url{http://www.bipm.org/en/publications/si-brochure/}; \url{http://physics.nist.gov/cuu/Units/} ). La liste ci-dessus est plus indicative qu'exhaustive.
\end{reflist}  
    \item[@quantity]
  (quantité) spécifie le nombre des unités indiquées que comprend la mesure.
\begin{reflist}
    \item[{Statut}]
  Optionel
    \item[{Type de données}]
  \hyperref[TEI.teidata.numeric]{teidata.numeric}
\end{reflist}  
    \item[@commodity]
  (article) indique ce qui est mesuré.
\begin{reflist}
    \item[{Statut}]
  Optionel
    \item[{Type de données}]
  1–∞ occurrences de \hyperref[TEI.teidata.word]{teidata.word} séparé par un espace
    \item[{Note}]
  \par
En général, si l'article est composé d'entités distinctes, la forme plurielle doit être employée, même si la mesure ne s'applique qu'à l'une d'entre elles.
\end{reflist}  
\end{sansreflist}  
    \item[{Note}]
  \par
Cette classe d'attributs fournit un ensemble de trois attributs qui peuvent être employés soit pour régulariser les valeurs de la mesure encodée, soit pour les normaliser en conformité avec un système de mesure normalisé.
    \item[{Note}]
  \par
L'unité doit normalement être nommée avec une abréviation normalisée issue d'une unité SI (voir plus loin \url{http://www.bipm.org/en/publications/si-brochure/}; \url{http://physics.nist.gov/cuu/Units/}). Cependant les encodeurs peuvent aussi spécifier des mesures avec des unités définies de manière informelle, telles que lines ou characters.
\end{reflist}  
\begin{reflist}
\item[]\begin{specHead}{TEI.att.media}{att.media}\index{att.media (attribute class)|oddindex}\index{width=@width!att.media (attribute class)|oddindex}\index{height=@height!att.media (attribute class)|oddindex}\index{scale=@scale!att.media (attribute class)|oddindex} provides attributes for specifying display and related properties of external media.\end{specHead} 
    \item[{Module}]
  tei
    \item[{Membres}]
  \hyperref[TEI.binaryObject]{binaryObject} \hyperref[TEI.graphic]{graphic}
    \item[{Attributs}]
  Attributs \hyperref[TEI.att.internetMedia]{att.internetMedia} (\textit{@mimeType}) \hfil\\[-10pt]\begin{sansreflist}
    \item[@width]
  Where the media are displayed, indicates the display width
\begin{reflist}
    \item[{Statut}]
  Optionel
    \item[{Type de données}]
  \hyperref[TEI.teidata.outputMeasurement]{teidata.outputMeasurement}
\end{reflist}  
    \item[@height]
  Where the media are displayed, indicates the display height
\begin{reflist}
    \item[{Statut}]
  Optionel
    \item[{Type de données}]
  \hyperref[TEI.teidata.outputMeasurement]{teidata.outputMeasurement}
\end{reflist}  
    \item[@scale]
  Where the media are displayed, indicates a scale factor to be applied when generating the desired display size
\begin{reflist}
    \item[{Statut}]
  Optionel
    \item[{Type de données}]
  \hyperref[TEI.teidata.numeric]{teidata.numeric}
\end{reflist}  
\end{sansreflist}  
\end{reflist}  
\begin{reflist}
\item[]\begin{specHead}{TEI.att.milestoneUnit}{att.milestoneUnit}\index{att.milestoneUnit (attribute class)|oddindex}\index{unit=@unit!att.milestoneUnit (attribute class)|oddindex} provides an attribute to indicate the type of section which is changing at a specific milestone.\end{specHead} 
    \item[{Module}]
  core
    \item[{Membres}]
  \hyperref[TEI.milestone]{milestone}
    \item[{Attributs}]
  Attributs\hfil\\[-10pt]\begin{sansreflist}
    \item[@unit]
  fournit un nom conventionnel pour le type de section qui change à partir de cette balise de bornage
\begin{reflist}
    \item[{Statut}]
  Requis
    \item[{Type de données}]
  \hyperref[TEI.teidata.enumerated]{teidata.enumerated}
    \item[{Les valeurs suggérées comprennent:}]
  \begin{description}

\item[{page}]sauts de page matériels (synonymes de l'élément \hyperref[TEI.pb]{<pb>})
\item[{column}]sauts de colonnes
\item[{line}]sauts de ligne (synonymes de l'élément\hyperref[TEI.lb]{<lb>})
\item[{book}]n'importe quel unité désignée par les termes livre, liber, etc.
\item[{poem}]poèmes séparés dans une collection
\item[{canto}]chants ou autres parties principales dans une poésie
\item[{speaker}]changement de locuteur ou de narrateur
\item[{stanza}]strophes dans une poésie, livre, ou chant
\item[{act}]actes dans une pièce
\item[{scene}]scènes dans une pièce ou dans un acte
\item[{section}]parties de toute catégorie.
\item[{absent}]passages qui ne sont pas présents dans l'édition de référence.
\item[{unnumbered}]passages figurant dans le texte, mais qui ne sont pas destinés à être inclus comme élément de référence.
\end{description} 
    \item[]\exampleFont {<\textbf{milestone}\hspace*{6pt}{ed}="{La}"\mbox{}\newline 
\hspace*{6pt}{n}="{23}"\mbox{}\newline 
\hspace*{6pt}{unit}="{Dreissiger}"/>}\mbox{}\newline 
 ... {<\textbf{milestone}\hspace*{6pt}{ed}="{AV}"\mbox{}\newline 
\hspace*{6pt}{n}="{24}"\mbox{}\newline 
\hspace*{6pt}{unit}="{verse}"/>} ...
    \item[]\exampleFont {<\textbf{milestone}\hspace*{6pt}{ed}="{La}"\mbox{}\newline 
\hspace*{6pt}{n}="{23}"\mbox{}\newline 
\hspace*{6pt}{unit}="{Dreissiger}"/>} ... {<\textbf{milestone}\hspace*{6pt}{ed}="{AV}"\mbox{}\newline 
\hspace*{6pt}{n}="{24}"\mbox{}\newline 
\hspace*{6pt}{unit}="{verse}"/>}\mbox{}\newline 
 ...
    \item[{Note}]
  \par
Si l'élément \hyperref[TEI.milestone]{<milestone>} marque le début d'un fragment de texte qui n'est pas présent dans l'édition de référence, la valeur \textit{absent} peut être donnée à l'attribut {\itshape unit}. On comprendra alors que l'édition de référence ne contient pas le fragment de texte qui suit et qui s'achève à la balise \hyperref[TEI.milestone]{<milestone>} suivante dans le texte.\par
En plus des valeurs proposées pour cet attribut, d'autres termes peuvent être appropriés (par ex. \textit{Stephanus} pour les numéros dits de Henri Estienne dans les éditions de Platon).
    \item[{Note}]
  \par
L'attribut {\itshape type} sera utilisé pour caractériser l’unité de bornage sans autre précaution d’emploi que celle de l'identification du type d'unité, par exemple s’il s’agit d’un mot coupé ou pas.
\end{reflist}  
\end{sansreflist}  
\end{reflist}  
\begin{reflist}
\item[]\begin{specHead}{TEI.att.msClass}{att.msClass}\index{att.msClass (attribute class)|oddindex}\index{class=@class!att.msClass (attribute class)|oddindex} provides an attribute to indicate text type or classification.\end{specHead} 
    \item[{Module}]
  msdescription
    \item[{Membres}]
  \hyperref[TEI.msContents]{msContents} \hyperref[TEI.msItem]{msItem} \hyperref[TEI.msItemStruct]{msItemStruct}
    \item[{Attributs}]
  Attributs\hfil\\[-10pt]\begin{sansreflist}
    \item[@class]
  spécifie la ou les catégories ou classes auxquelles l'item appartient.
\begin{reflist}
    \item[{Statut}]
  Optionel
    \item[{Type de données}]
  1–∞ occurrences de \hyperref[TEI.teidata.pointer]{teidata.pointer} séparé par un espace
\end{reflist}  
\end{sansreflist}  
\end{reflist}  
\begin{reflist}
\item[]\begin{specHead}{TEI.att.msExcerpt}{att.msExcerpt}\index{att.msExcerpt (attribute class)|oddindex}\index{defective=@defective!att.msExcerpt (attribute class)|oddindex} (extrait d'un manuscrit) fournit des attributs pour décrire les extraits d'un manuscrit.\end{specHead} 
    \item[{Module}]
  msdescription
    \item[{Membres}]
  \hyperref[TEI.colophon]{colophon} \hyperref[TEI.explicit]{explicit} \hyperref[TEI.finalRubric]{finalRubric} \hyperref[TEI.incipit]{incipit} \hyperref[TEI.msContents]{msContents} \hyperref[TEI.msItem]{msItem} \hyperref[TEI.msItemStruct]{msItemStruct} \hyperref[TEI.quote]{quote} \hyperref[TEI.rubric]{rubric}
    \item[{Attributs}]
  Attributs\hfil\\[-10pt]\begin{sansreflist}
    \item[@defective]
  indique si le passage décrit est fautif, i.e. incomplet en raison d'une lacune ou d'une détérioration.
\begin{reflist}
    \item[{Statut}]
  Optionel
    \item[{Type de données}]
  \hyperref[TEI.teidata.xTruthValue]{teidata.xTruthValue}
    \item[{Valeur par défaut}]
  false \par \begin{tabular}{P{0.4969230769230769\textwidth}P{0.35307692307692307\textwidth}}
\xref{http://www.tei-c.org/Activities/Council/Working/tcw27.xml}{Deprecated}\tabcellsep The value will no longer be a default after 2017-09-05\end{tabular}
\end{reflist}  
\end{sansreflist}  
    \item[{Note}]
  \par
Dans le cas d'un incipit, indique si l'incipit est considéré comme fautif, c'est-à-dire qu'il présente les premiers mots du texte tels qu'ils ont été conservés, et non pas les premiers mots de l'oeuvre elle-même. Dans le cas d'un explicit, indique si l'explicit est considéré comme fautif, c'est-à-dire qu'il présente les mots finaux du texte tels qu'ils ont été préservés, et non pas ce qu'auraient été ces mots si le texte de l'oeuvre avait été complet. 
\end{reflist}  
\begin{reflist}
\item[]\begin{specHead}{TEI.att.naming}{att.naming}\index{att.naming (attribute class)|oddindex}\index{role=@role!att.naming (attribute class)|oddindex}\index{nymRef=@nymRef!att.naming (attribute class)|oddindex} fournit des attributs communs aux éléments qui font référence à des personnes, lieux, organismes, etc., nommés.\end{specHead} 
    \item[{Module}]
  tei
    \item[{Membres}]
  \hyperref[TEI.att.personal]{att.personal}[\hyperref[TEI.addName]{addName} \hyperref[TEI.forename]{forename} \hyperref[TEI.genName]{genName} \hyperref[TEI.name]{name} \hyperref[TEI.orgName]{orgName} \hyperref[TEI.persName]{persName} \hyperref[TEI.placeName]{placeName} \hyperref[TEI.roleName]{roleName} \hyperref[TEI.surname]{surname}] \hyperref[TEI.affiliation]{affiliation} \hyperref[TEI.author]{author} \hyperref[TEI.collection]{collection} \hyperref[TEI.country]{country} \hyperref[TEI.editor]{editor} \hyperref[TEI.event]{event} \hyperref[TEI.geogName]{geogName} \hyperref[TEI.institution]{institution} \hyperref[TEI.origPlace]{origPlace} \hyperref[TEI.pubPlace]{pubPlace} \hyperref[TEI.region]{region} \hyperref[TEI.repository]{repository} \hyperref[TEI.rs]{rs} \hyperref[TEI.settlement]{settlement} \hyperref[TEI.state]{state}
    \item[{Attributs}]
  Attributs \hyperref[TEI.att.canonical]{att.canonical} (\textit{@key}, \textit{@ref}) \hfil\\[-10pt]\begin{sansreflist}
    \item[@role]
  may be used to specify further information about the entity referenced by this name in the form of a set of whitespace-separated values, for example the occupation of a person, or the status of a place.
\begin{reflist}
    \item[{Statut}]
  Optionel
    \item[{Type de données}]
  1–∞ occurrences de \hyperref[TEI.teidata.enumerated]{teidata.enumerated} séparé par un espace
\end{reflist}  
    \item[@nymRef]
  (référence au nom canonique) indique comment localiser la forme canonique (\textit{nym}) des noms qui sont associés à l'objet nommé par l'élément qui le contient.
\begin{reflist}
    \item[{Statut}]
  Optionel
    \item[{Type de données}]
  1–∞ occurrences de \hyperref[TEI.teidata.pointer]{teidata.pointer} séparé par un espace
    \item[{Note}]
  \par
La valeur doit pointer directement vers un ou plusieurs éléments XML au moyen d'un ou plusieurs URIs, séparés par un espace blanc. Si plus d'un URI est fourni, alors le nom est associé à plusieurs noms canoniques distincts.
\end{reflist}  
\end{sansreflist}  
\end{reflist}  
\begin{reflist}
\item[]\begin{specHead}{TEI.att.notated}{att.notated}\index{att.notated (attribute class)|oddindex}\index{notation=@notation!att.notated (attribute class)|oddindex} provides an attribute to indicate any specialised notation used for element content.\end{specHead} 
    \item[{Module}]
  tei
    \item[{Membres}]
  \hyperref[TEI.annotationBlock]{annotationBlock} \hyperref[TEI.formula]{formula} \hyperref[TEI.listAnnotation]{listAnnotation} \hyperref[TEI.standOff]{standOff}
    \item[{Attributs}]
  Attributs\hfil\\[-10pt]\begin{sansreflist}
    \item[@notation]
  précise le nom d'une notation définie précédemment, utilisée dans le contenu de l'élément.
\begin{reflist}
    \item[{Statut}]
  Optionel
    \item[{Type de données}]
  \hyperref[TEI.teidata.enumerated]{teidata.enumerated}
\end{reflist}  
\end{sansreflist}  
\end{reflist}  
\begin{reflist}
\item[]\begin{specHead}{TEI.att.personal}{att.personal}\index{att.personal (attribute class)|oddindex}\index{full=@full!att.personal (attribute class)|oddindex}\index{sort=@sort!att.personal (attribute class)|oddindex} (attributs des composantes des noms de personnes) attributs communs des éléments qui composent un nom de personne\end{specHead} 
    \item[{Module}]
  tei
    \item[{Membres}]
  \hyperref[TEI.addName]{addName} \hyperref[TEI.forename]{forename} \hyperref[TEI.genName]{genName} \hyperref[TEI.name]{name} \hyperref[TEI.orgName]{orgName} \hyperref[TEI.persName]{persName} \hyperref[TEI.placeName]{placeName} \hyperref[TEI.roleName]{roleName} \hyperref[TEI.surname]{surname}
    \item[{Attributs}]
  Attributs \hyperref[TEI.att.naming]{att.naming} (\textit{@role}, \textit{@nymRef})  (\hyperref[TEI.att.canonical]{att.canonical} (\textit{@key}, \textit{@ref})) \hfil\\[-10pt]\begin{sansreflist}
    \item[@full]
  indique si la composante du nom est donnée en intégralité, sous forme d'abréviation ou simplement d'initiale.
\begin{reflist}
    \item[{Statut}]
  Optionel
    \item[{Type de données}]
  \hyperref[TEI.teidata.enumerated]{teidata.enumerated}
    \item[{Les valeurs autorisées sont:}]
  \begin{description}

\item[{yes}](complet) la composante du nom est orthographiée en intégralité.{[Valeur par défaut] }
\item[{abb}](abrégé) la composante du nom est donnée sous forme abrégée.
\item[{init}](initiale) la composante du nom n'est indiquée que par sa lettre initiale.
\end{description} 
\end{reflist}  
    \item[@sort]
  (ordre) précise dans quel ordre est placé la composante par rapport aux autres dans le nom d'une personne.
\begin{reflist}
    \item[{Statut}]
  Optionel
    \item[{Type de données}]
  \hyperref[TEI.teidata.count]{teidata.count}
\end{reflist}  
\end{sansreflist}  
\end{reflist}  
\begin{reflist}
\item[]\begin{specHead}{TEI.att.placement}{att.placement}\index{att.placement (attribute class)|oddindex}\index{place=@place!att.placement (attribute class)|oddindex} fournit des attributs pour décrire l'emplacement où apparaît un élément textuel dans la page ou l'objet source.\end{specHead} 
    \item[{Module}]
  tei
    \item[{Membres}]
  \hyperref[TEI.add]{add} \hyperref[TEI.addSpan]{addSpan} \hyperref[TEI.figure]{figure} \hyperref[TEI.fw]{fw} \hyperref[TEI.head]{head} \hyperref[TEI.label]{label} \hyperref[TEI.metamark]{metamark} \hyperref[TEI.notatedMusic]{notatedMusic} \hyperref[TEI.note]{note} \hyperref[TEI.stage]{stage}
    \item[{Attributs}]
  Attributs\hfil\\[-10pt]\begin{sansreflist}
    \item[@place]
  specifie où cet item se trouve.
\begin{reflist}
    \item[{Statut}]
  Recommendé
    \item[{Type de données}]
  1–∞ occurrences de \hyperref[TEI.teidata.enumerated]{teidata.enumerated} séparé par un espace
    \item[{Les valeurs suggérées comprennent:}]
  \begin{description}

\item[{below}]au-dessous de la ligne
\item[{bottom}]dans la marge inférieure
\item[{margin}]dans la marge (gauche, droite ou les deux en même temps)
\item[{top}]dans la marge supérieure
\item[{opposite}]sur la page opposée
\item[{overleaf}]de l'autre côté de la feuille
\item[{above}]au-dessus de la ligne
\item[{end}]à la fin, par exemple d'un chapitre ou d'un volume
\item[{inline}]dans le corps du texte
\item[{inspace}]dans un espace prédéfini, par exemple à gauche d'un scripteur précédent
\end{description} 
    \item[]\exampleFont {<\textbf{add}\hspace*{6pt}{place}="{margin}">}[An addition written in the margin]{</\textbf{add}>}\mbox{}\newline 
{<\textbf{add}\hspace*{6pt}{place}="{bottom opposite}">}[An addition written at the\mbox{}\newline 
 foot of the current page and also on the facing page]{</\textbf{add}>}
    \item[]\exampleFont {<\textbf{note}\hspace*{6pt}{place}="{bottom}">}Ibid, p.7{</\textbf{note}>}
\end{reflist}  
\end{sansreflist}  
\end{reflist}  
\begin{reflist}
\item[]\begin{specHead}{TEI.att.pointing}{att.pointing}\index{att.pointing (attribute class)|oddindex}\index{targetLang=@targetLang!att.pointing (attribute class)|oddindex}\index{target=@target!att.pointing (attribute class)|oddindex}\index{evaluate=@evaluate!att.pointing (attribute class)|oddindex} fournit un ensemble d'attributs utilisés par tous les éléments qui pointent vers d'autres éléments au moyen d'une ou de plusieurs références URI.\end{specHead} 
    \item[{Module}]
  tei
    \item[{Membres}]
  \hyperref[TEI.att.pointing.group]{att.pointing.group}[\hyperref[TEI.altGrp]{altGrp} \hyperref[TEI.joinGrp]{joinGrp} \hyperref[TEI.linkGrp]{linkGrp}] \hyperref[TEI.citedRange]{citedRange} \hyperref[TEI.gloss]{gloss} \hyperref[TEI.join]{join} \hyperref[TEI.licence]{licence} \hyperref[TEI.link]{link} \hyperref[TEI.locus]{locus} \hyperref[TEI.note]{note} \hyperref[TEI.ptr]{ptr} \hyperref[TEI.ref]{ref} \hyperref[TEI.span]{span} \hyperref[TEI.standOff]{standOff} \hyperref[TEI.substJoin]{substJoin} \hyperref[TEI.term]{term}
    \item[{Attributs}]
  Attributs\hfil\\[-10pt]\begin{sansreflist}
    \item[@targetLang]
  specifies the language of the content to be found at the destination referenced by {\itshape target}, using a ‘language tag’ generated according to \xref{http://www.rfc-editor.org/rfc/bcp/bcp47.txt}{BCP 47}.
\begin{reflist}
    \item[{Statut}]
  Optionel
    \item[{Type de données}]
  \hyperref[TEI.teidata.language]{teidata.language}
    \item[{Schematron}]
   <sch:rule context="tei:*[not(self::tei:schemaSpec)][@targetLang]"> <sch:assert test="@target">@targetLang should only be used on <sch:name/> if @target is specified.</sch:assert> </sch:rule>
    \item[]\exampleFont {<\textbf{linkGrp}\hspace*{6pt}{xml:id}="{pol-swh\textunderscore aln\textunderscore 2.1-linkGrp}">}\mbox{}\newline 
\hspace*{6pt}{<\textbf{ptr}\hspace*{6pt}{target}="{pol/UDHR/text.xml\#pol\textunderscore txt\textunderscore 1-head}"\mbox{}\newline 
\hspace*{6pt}\hspace*{6pt}{targetLang}="{pl}"\mbox{}\newline 
\hspace*{6pt}\hspace*{6pt}{type}="{tuv}"\mbox{}\newline 
\hspace*{6pt}\hspace*{6pt}{xml:id}="{pol-swh\textunderscore aln\textunderscore 2.1.1-ptr}"/>}\mbox{}\newline 
\hspace*{6pt}{<\textbf{ptr}\hspace*{6pt}{target}="{swh/UDHR/text.xml\#swh\textunderscore txt\textunderscore 1-head}"\mbox{}\newline 
\hspace*{6pt}\hspace*{6pt}{targetLang}="{sw}"\mbox{}\newline 
\hspace*{6pt}\hspace*{6pt}{type}="{tuv}"\mbox{}\newline 
\hspace*{6pt}\hspace*{6pt}{xml:id}="{pol-swh\textunderscore aln\textunderscore 2.1.2-ptr}"/>}\mbox{}\newline 
{</\textbf{linkGrp}>}In the example above, the \hyperref[TEI.linkGrp]{<linkGrp>} combines pointers at parallel fragments of the \textit{Universal Declaration of Human Rights}: one of them is in Polish, the other in Swahili.
\end{reflist}  
    \item[@target]
  précise la cible de la référence en donnant une ou plusieurs références URI
\begin{reflist}
    \item[{Statut}]
  Optionel
    \item[{Type de données}]
  1–∞ occurrences de \hyperref[TEI.teidata.pointer]{teidata.pointer} séparé par un espace
    \item[{Note}]
  \par
Une ou plusieurs références URI syntaxiquement valables, séparée par un espace. Puisqu'un espace est employé pour séparer des URIs, aucun espace n’est autorisé à l'intérieur d'un URI. Si un espace est requis dans un URI, il faut le représenter avec une séquence d'échappement, comme par exemple \texttt{TEI\%20Consortium}.
\end{reflist}  
    \item[@evaluate]
  (évalué) détermine le sens attendu, si la cible d'un pointeur est elle-même un pointeur.
\begin{reflist}
    \item[{Statut}]
  Optionel
    \item[{Type de données}]
  \hyperref[TEI.teidata.enumerated]{teidata.enumerated}
    \item[{Les valeurs autorisées sont:}]
  \begin{description}

\item[{all}]si l'élément pointé est lui-même un pointeur, alors on prendra la cible de ce pointeur, et ainsi de suite jusqu'à trouver un élément qui n'est pas un pointeur.
\item[{one}]si l'élément pointé est lui-même un pointeur, alors sa cible (qui est ou non un pointeur) devient la cible retenue.
\item[{none}]aucune évaluation ultérieure des cibles n'est menée au-delà de la recherche de l'élément désigné dans la cible du pointeur.
\end{description} 
    \item[{Note}]
  \par
Si aucune valeur n'est fournie, c'est au programme d'application de décider (éventuellement à partir d'une donnée entrée par l'utilisateur) jusqu'où retracer une chaîne de pointeurs.
\end{reflist}  
\end{sansreflist}  
\end{reflist}  
\begin{reflist}
\item[]\begin{specHead}{TEI.att.pointing.group}{att.pointing.group}\index{att.pointing.group (attribute class)|oddindex}\index{domains=@domains!att.pointing.group (attribute class)|oddindex}\index{targFunc=@targFunc!att.pointing.group (attribute class)|oddindex} fournit un ensemble d'attributs communs à tous les éléments qui contiennent des groupes d'éléments pointeurs.\end{specHead} 
    \item[{Module}]
  tei
    \item[{Membres}]
  \hyperref[TEI.altGrp]{altGrp} \hyperref[TEI.joinGrp]{joinGrp} \hyperref[TEI.linkGrp]{linkGrp}
    \item[{Attributs}]
  Attributs \hyperref[TEI.att.pointing]{att.pointing} (\textit{@targetLang}, \textit{@target}, \textit{@evaluate}) \hyperref[TEI.att.typed]{att.typed} (\textit{@type}, \textit{@subtype}) \hfil\\[-10pt]\begin{sansreflist}
    \item[@domains]
  spécifie, facultativement, les identifiants des éléments englobant tous les éléments indiqués par le contenu de cet élément.
\begin{reflist}
    \item[{Statut}]
  Optionel
    \item[{Type de données}]
  2–∞ occurrences de \hyperref[TEI.teidata.pointer]{teidata.pointer} séparé par un espace
    \item[{Note}]
  \par
Si cet attribut est utilisé, tout élément spécifié comme étant une cible doit être contenu dans l'élément ou les éléments qu'il nomme. Une application peut choisir de faire apparaître en erreur, ou non, les entorses à cette contrainte mais ne peut pas accéder à un élément qui aurait le bon identifiant mais se trouverait dans le mauvais contexte. Si cet attribut n'est pas utilisé, les éléments cibles peuvent apparaître n'importe où dans le document cible.
\end{reflist}  
    \item[@targFunc]
  (fonction cible) décrit la fonction de chacune des valeurs de l'attribut {\itshape target} pour les balises incluses \hyperref[TEI.link]{<link>}, \hyperref[TEI.join]{<join>}, ou \hyperref[TEI.alt]{<alt>}.
\begin{reflist}
    \item[{Statut}]
  Optionel
    \item[{Type de données}]
  2–∞ occurrences de \hyperref[TEI.teidata.word]{teidata.word} séparé par un espace
    \item[{Note}]
  \par
Le nombre de valeurs distinctes doit correspondre au nombre de valeurs dans l'attribut {\itshape target} des balises incluses \hyperref[TEI.link]{<link>}, \hyperref[TEI.join]{<join>} ou \hyperref[TEI.alt]{<alt>} (un élément intermédiaire \hyperref[TEI.ptr]{<ptr>} peut être nécessaire dans ce cas). Il devrait également correspondre au nombre de valeurs se trouvant dans l'attribut {\itshape domains} de l'élément en question, si un tel attribut a été spécifié.
\end{reflist}  
\end{sansreflist}  
\end{reflist}  
\begin{reflist}
\item[]\begin{specHead}{TEI.att.ranging}{att.ranging}\index{att.ranging (attribute class)|oddindex}\index{atLeast=@atLeast!att.ranging (attribute class)|oddindex}\index{atMost=@atMost!att.ranging (attribute class)|oddindex}\index{min=@min!att.ranging (attribute class)|oddindex}\index{max=@max!att.ranging (attribute class)|oddindex}\index{confidence=@confidence!att.ranging (attribute class)|oddindex} provides attributes for describing numerical ranges.\end{specHead} 
    \item[{Module}]
  tei
    \item[{Membres}]
  \hyperref[TEI.att.dimensions]{att.dimensions}[\hyperref[TEI.att.damaged]{att.damaged}[\hyperref[TEI.damage]{damage} \hyperref[TEI.damageSpan]{damageSpan}] \hyperref[TEI.att.editLike]{att.editLike}[\hyperref[TEI.att.transcriptional]{att.transcriptional}[\hyperref[TEI.add]{add} \hyperref[TEI.addSpan]{addSpan} \hyperref[TEI.del]{del} \hyperref[TEI.delSpan]{delSpan} \hyperref[TEI.mod]{mod} \hyperref[TEI.redo]{redo} \hyperref[TEI.restore]{restore} \hyperref[TEI.retrace]{retrace} \hyperref[TEI.subst]{subst} \hyperref[TEI.substJoin]{substJoin} \hyperref[TEI.undo]{undo}] \hyperref[TEI.affiliation]{affiliation} \hyperref[TEI.am]{am} \hyperref[TEI.corr]{corr} \hyperref[TEI.date]{date} \hyperref[TEI.event]{event} \hyperref[TEI.ex]{ex} \hyperref[TEI.expan]{expan} \hyperref[TEI.gap]{gap} \hyperref[TEI.geogName]{geogName} \hyperref[TEI.location]{location} \hyperref[TEI.name]{name} \hyperref[TEI.org]{org} \hyperref[TEI.orgName]{orgName} \hyperref[TEI.origDate]{origDate} \hyperref[TEI.origPlace]{origPlace} \hyperref[TEI.origin]{origin} \hyperref[TEI.persName]{persName} \hyperref[TEI.person]{person} \hyperref[TEI.persona]{persona} \hyperref[TEI.place]{place} \hyperref[TEI.placeName]{placeName} \hyperref[TEI.reg]{reg} \hyperref[TEI.secl]{secl} \hyperref[TEI.state]{state} \hyperref[TEI.supplied]{supplied} \hyperref[TEI.surplus]{surplus} \hyperref[TEI.time]{time} \hyperref[TEI.unclear]{unclear}] \hyperref[TEI.depth]{depth} \hyperref[TEI.dim]{dim} \hyperref[TEI.dimensions]{dimensions} \hyperref[TEI.height]{height} \hyperref[TEI.space]{space} \hyperref[TEI.width]{width}] \hyperref[TEI.num]{num}
    \item[{Attributs}]
  Attributs\hfil\\[-10pt]\begin{sansreflist}
    \item[@atLeast]
  donne une estimation de la valeur minimum pour la mesure.
\begin{reflist}
    \item[{Statut}]
  Optionel
    \item[{Type de données}]
  \hyperref[TEI.teidata.numeric]{teidata.numeric}
\end{reflist}  
    \item[@atMost]
  donne une estimation de la valeur maximum pour la mesure.
\begin{reflist}
    \item[{Statut}]
  Optionel
    \item[{Type de données}]
  \hyperref[TEI.teidata.numeric]{teidata.numeric}
\end{reflist}  
    \item[@min]
  lorsque la mesure résume plus d'une observation, fournit la valeur minimum observée.
\begin{reflist}
    \item[{Statut}]
  Optionel
    \item[{Type de données}]
  \hyperref[TEI.teidata.numeric]{teidata.numeric}
\end{reflist}  
    \item[@max]
  lorsque la mesure résume plus d'une observation, fournit la valeur maximum observée.
\begin{reflist}
    \item[{Statut}]
  Optionel
    \item[{Type de données}]
  \hyperref[TEI.teidata.numeric]{teidata.numeric}
\end{reflist}  
    \item[@confidence]
  specifies the degree of statistical confidence (between zero and one) that a value falls within the range specified by {\itshape min} and {\itshape max}, or the proportion of observed values that fall within that range.
\begin{reflist}
    \item[{Statut}]
  Optionel
    \item[{Type de données}]
  \hyperref[TEI.teidata.probability]{teidata.probability}
\end{reflist}  
\end{sansreflist}  
    \item[{Exemple}]
  \leavevmode\bgroup\exampleFont \begin{shaded}\noindent\mbox{}The MS. was lost in transmission by mail from {<\textbf{del}\hspace*{6pt}{rend}="{overstrike}">}\mbox{}\newline 
\hspace*{6pt}{<\textbf{gap}\hspace*{6pt}{atLeast}="{1}"\hspace*{6pt}{atMost}="{2}"\mbox{}\newline 
\hspace*{6pt}\hspace*{6pt}{extent}="{one or two letters}"\hspace*{6pt}{reason}="{illegible}"\hspace*{6pt}{unit}="{chars}"/>}\mbox{}\newline 
{</\textbf{del}>} Philadelphia to the Graphic office, New York.\mbox{}\newline 
\end{shaded}\egroup 


\end{reflist}  
\begin{reflist}
\item[]\begin{specHead}{TEI.att.resourced}{att.resourced}\index{att.resourced (attribute class)|oddindex}\index{url=@url!att.resourced (attribute class)|oddindex} provides attributes by which a resource (such as an externally held media file) may be located.\end{specHead} 
    \item[{Module}]
  tei
    \item[{Membres}]
  \hyperref[TEI.graphic]{graphic} \hyperref[TEI.media]{media} \hyperref[TEI.schemaRef]{schemaRef}
    \item[{Attributs}]
  Attributs\hfil\\[-10pt]\begin{sansreflist}
    \item[@url]
  (adresse URL) specifies the URL from which the media concerned may be obtained.
\begin{reflist}
    \item[{Statut}]
  Requis
    \item[{Type de données}]
  \hyperref[TEI.teidata.pointer]{teidata.pointer}
\end{reflist}  
\end{sansreflist}  
\end{reflist}  
\begin{reflist}
\item[]\begin{specHead}{TEI.att.segLike}{att.segLike}\index{att.segLike (attribute class)|oddindex}\index{function=@function!att.segLike (attribute class)|oddindex} fournit des attributs pour des éléments utilisés pour une segmentation arbitraire.\end{specHead} 
    \item[{Module}]
  tei
    \item[{Membres}]
  \hyperref[TEI.c]{c} \hyperref[TEI.cl]{cl} \hyperref[TEI.m]{m} \hyperref[TEI.pc]{pc} \hyperref[TEI.phr]{phr} \hyperref[TEI.s]{s} \hyperref[TEI.seg]{seg} \hyperref[TEI.w]{w}
    \item[{Attributs}]
  Attributs \hyperref[TEI.att.datcat]{att.datcat} (\textit{@datcat}, \textit{@valueDatcat}) \hyperref[TEI.att.fragmentable]{att.fragmentable} (\textit{@part}) \hfil\\[-10pt]\begin{sansreflist}
    \item[@function]
  (fonction) caractérise la fonction du segment.
\begin{reflist}
    \item[{Statut}]
  Optionel
    \item[{Type de données}]
  \hyperref[TEI.teidata.enumerated]{teidata.enumerated}
\end{reflist}  
\end{sansreflist}  
\end{reflist}  
\begin{reflist}
\item[]\begin{specHead}{TEI.att.sortable}{att.sortable}\index{att.sortable (attribute class)|oddindex}\index{sortKey=@sortKey!att.sortable (attribute class)|oddindex} provides attributes for elements in lists or groups that are sortable, but whose sorting key cannot be derived mechanically from the element content.\end{specHead} 
    \item[{Module}]
  tei
    \item[{Membres}]
  \hyperref[TEI.bibl]{bibl} \hyperref[TEI.biblFull]{biblFull} \hyperref[TEI.biblStruct]{biblStruct} \hyperref[TEI.event]{event} \hyperref[TEI.idno]{idno} \hyperref[TEI.item]{item} \hyperref[TEI.list]{list} \hyperref[TEI.listBibl]{listBibl} \hyperref[TEI.listOrg]{listOrg} \hyperref[TEI.listPlace]{listPlace} \hyperref[TEI.msDesc]{msDesc} \hyperref[TEI.org]{org} \hyperref[TEI.person]{person} \hyperref[TEI.personGrp]{personGrp} \hyperref[TEI.persona]{persona} \hyperref[TEI.place]{place} \hyperref[TEI.term]{term}
    \item[{Attributs}]
  Attributs\hfil\\[-10pt]\begin{sansreflist}
    \item[@sortKey]
  supplies the sort key for this element in an index, list or group which contains it.
\begin{reflist}
    \item[{Statut}]
  Optionel
    \item[{Type de données}]
  \hyperref[TEI.teidata.word]{teidata.word}
    \item[]\exampleFont  Je me suis\mbox{}\newline 
 procuré une {<\textbf{term}>}clef anglaise{</\textbf{term}>} pour dévisser les écrous qui attachent le canot à\mbox{}\newline 
 la coque du Nautilus. Ainsi tout est prêt.
    \item[{Note}]
  \par
La clé de tri est utilisée pour déterminer la séquence et le groupement d'entrées dans un index. Elle fournit une séquence de caractères qui, lorsqu'ils sont triés avec les autres valeurs, produisent l'ordre souhaité ; les détails de construction d'une clé de tri dépendent des applications. .\par
La structure d'un dictionnaire diffère souvent de l'ordre de collation des jeux de caractères lisibles par la machine ; dans des dictionnaires de langue anglaise, une entrée pour \textit{4-H} apparaîtra souvent alphabétiquement sous ‘fourh’, et \textit{McCoy}peut être classé alphabétiquement sous ‘maccoy’, tandis que \textit{A1}, \textit{A4} et \textit{A5} apparaîtront tous dans un ordre alphanumérique entre ‘a-’ et ‘AA’. La clef de tri est exigée si l'orthographe de l'entrée du dictionnaire n'est pas suffisante pour déterminer son emplacement.
\end{reflist}  
\end{sansreflist}  
\end{reflist}  
\begin{reflist}
\item[]\begin{specHead}{TEI.att.spanning}{att.spanning}\index{att.spanning (attribute class)|oddindex}\index{spanTo=@spanTo!att.spanning (attribute class)|oddindex} fournit des attributs pour les éléments qui délimitent un passage de texte par des mécanismes de pointage plutôt qu'en entourant le passage.\end{specHead} 
    \item[{Module}]
  tei
    \item[{Membres}]
  \hyperref[TEI.addSpan]{addSpan} \hyperref[TEI.cb]{cb} \hyperref[TEI.damageSpan]{damageSpan} \hyperref[TEI.delSpan]{delSpan} \hyperref[TEI.gb]{gb} \hyperref[TEI.index]{index} \hyperref[TEI.lb]{lb} \hyperref[TEI.metamark]{metamark} \hyperref[TEI.milestone]{milestone} \hyperref[TEI.mod]{mod} \hyperref[TEI.pb]{pb} \hyperref[TEI.redo]{redo} \hyperref[TEI.retrace]{retrace} \hyperref[TEI.undo]{undo}
    \item[{Attributs}]
  Attributs\hfil\\[-10pt]\begin{sansreflist}
    \item[@spanTo]
  indique la fin d'un passage introduit par l'élément portant cet attribut.
\begin{reflist}
    \item[{Statut}]
  Optionel
    \item[{Type de données}]
  \hyperref[TEI.teidata.pointer]{teidata.pointer}
    \item[{Schematron}]
  The @spanTo attribute must point to an element following the current element <sch:rule context="tei:*[@spanTo]"> <sch:assert test="id(substring(@spanTo,2)) and following::*[@xml:id=substring(current()/@spanTo,2)]">The element indicated by @spanTo (<sch:value-of select="@spanTo"/>) must follow the current element <sch:name/> </sch:assert> </sch:rule>
\end{reflist}  
\end{sansreflist}  
    \item[{Note}]
  \par
Le passage est défini comme courant depuis le début du contenu de l'élément pointeur (s'il y en a un) jusqu'à la fin du contenu de l'élément pointé par l'attribut {\itshape spanTo} (s'il y en a un), dans l'ordre du document. Si aucune valeur n'est fournie pour l'attribut, il est entendu que le passage est de même étendue que l'élément pointeur.
\end{reflist}  
\begin{reflist}
\item[]\begin{specHead}{TEI.att.styleDef}{att.styleDef}\index{att.styleDef (attribute class)|oddindex}\index{scheme=@scheme!att.styleDef (attribute class)|oddindex}\index{schemeVersion=@schemeVersion!att.styleDef (attribute class)|oddindex} provides attributes to specify the name of a formal definition language used to provide formatting or rendition information.\end{specHead} 
    \item[{Module}]
  tei
    \item[{Membres}]
  \hyperref[TEI.rendition]{rendition}
    \item[{Attributs}]
  Attributs\hfil\\[-10pt]\begin{sansreflist}
    \item[@scheme]
  identifie la langue employée pour décrire le rendu
\begin{reflist}
    \item[{Statut}]
  Optionel
    \item[{Type de données}]
  \hyperref[TEI.teidata.enumerated]{teidata.enumerated}
    \item[{Les valeurs autorisées sont:}]
  \begin{description}

\item[{css}]langage CSS (Cascading Stylesheet )
\item[{xslfo}]Langage XSL (Extensible Stylesheet )Formatting Objects
\item[{free}]description en texte libre non structuré.
\item[{other}]langue de description de l'interprétation définie par l'utilisateur
\end{description} 
\end{reflist}  
    \item[@schemeVersion]
  supplies a version number for the style language provided in {\itshape scheme}.
\begin{reflist}
    \item[{Statut}]
  Optionel
    \item[{Type de données}]
  \hyperref[TEI.teidata.versionNumber]{teidata.versionNumber}
    \item[{Schematron}]
   <sch:rule context="tei:*[@schemeVersion]"> <sch:assert test="@scheme and not(@scheme = 'free')"> @schemeVersion can only be used if @scheme is specified. </sch:assert> </sch:rule>
\end{reflist}  
\end{sansreflist}  
\end{reflist}  
\begin{reflist}
\item[]\begin{specHead}{TEI.att.tableDecoration}{att.tableDecoration}\index{att.tableDecoration (attribute class)|oddindex}\index{role=@role!att.tableDecoration (attribute class)|oddindex}\index{rows=@rows!att.tableDecoration (attribute class)|oddindex}\index{cols=@cols!att.tableDecoration (attribute class)|oddindex} fournit des attributs pour mettre en forme les lignes ou les cellules d'un tableau.\end{specHead} 
    \item[{Module}]
  tei
    \item[{Membres}]
  \hyperref[TEI.cell]{cell} \hyperref[TEI.row]{row}
    \item[{Attributs}]
  Attributs\hfil\\[-10pt]\begin{sansreflist}
    \item[@role]
  (rôle) indique le type des informations contenues dans cette cellule ou dans chaque cellule de cette ligne.
\begin{reflist}
    \item[{Statut}]
  Optionel
    \item[{Type de données}]
  \hyperref[TEI.teidata.enumerated]{teidata.enumerated}
    \item[{Les valeurs suggérées comprennent:}]
  \begin{description}

\item[{label}]uniquement des informations relatives au codage ou à la description
\item[{data}]valeurs de données{[Valeur par défaut] }
\end{description} 
    \item[{Note}]
  \par
Quand cet attribut est appliqué à une ligne de tableau, sa valeur est transmise comme valeur par défaut à toutes les cellules de cette ligne. Quand il est spécifié sur une cellule, sa valeur annule et remplace toute valeur spécifiée par défaut dans l'attribut {\itshape role} de l'élément parent \hyperref[TEI.row]{<row>}.
\end{reflist}  
    \item[@rows]
  (lignes) indique le nombre de lignes occupées par la cellule ou la ligne en question.
\begin{reflist}
    \item[{Statut}]
  Optionel
    \item[{Type de données}]
  \hyperref[TEI.teidata.count]{teidata.count}
    \item[{Valeur par défaut}]
  1
    \item[{Note}]
  \par
Lorsque plusieurs cellules s'étendent sur plusieurs lignes, il peut être plus pratique d'employer des tableaux inclus.
\end{reflist}  
    \item[@cols]
  (colonnes) indique le nombre de colonnes occupées par cette cellule ou cette ligne.
\begin{reflist}
    \item[{Statut}]
  Optionel
    \item[{Type de données}]
  \hyperref[TEI.teidata.count]{teidata.count}
    \item[{Valeur par défaut}]
  1
    \item[{Note}]
  \par
Une valeur plus grande que 1 indique que cette cellule (ou cette ligne) occupe plusieurs colonnes. Lorsqu'une première cellule s'étend sur une ligne entière, il peut être préférable de la considérer comme un titre.
\end{reflist}  
\end{sansreflist}  
\end{reflist}  
\begin{reflist}
\item[]\begin{specHead}{TEI.att.timed}{att.timed}\index{att.timed (attribute class)|oddindex}\index{start=@start!att.timed (attribute class)|oddindex}\index{end=@end!att.timed (attribute class)|oddindex} fournit des attributs communs aux éléments qui expriment une durée dans le temps, soit de manière absolue, soit en se référant à une carte d'alignement.\end{specHead} 
    \item[{Module}]
  tei
    \item[{Membres}]
  \hyperref[TEI.annotationBlock]{annotationBlock} \hyperref[TEI.bibl]{bibl} \hyperref[TEI.binaryObject]{binaryObject} \hyperref[TEI.gap]{gap} \hyperref[TEI.listAnnotation]{listAnnotation} \hyperref[TEI.media]{media}
    \item[{Attributs}]
  Attributs \hyperref[TEI.att.duration]{att.duration} (\hyperref[TEI.att.duration.w3c]{att.duration.w3c} (\textit{@dur})) (\hyperref[TEI.att.duration.iso]{att.duration.iso} (\textit{@dur-iso})) \hfil\\[-10pt]\begin{sansreflist}
    \item[@start]
  indique dans un alignement temporel (un ordre chronologique) l'endroit où commence cet élément.
\begin{reflist}
    \item[{Statut}]
  Optionel
    \item[{Type de données}]
  \hyperref[TEI.teidata.pointer]{teidata.pointer}
    \item[{Note}]
  \par
Si aucune valeur n'est donnée, il est entendu que l'élément suit l'élément immédiatement précédent au même niveau hiérarchique.
\end{reflist}  
    \item[@end]
  indique l'endroit où se termine cet élément dans un alignement temporel.
\begin{reflist}
    \item[{Statut}]
  Optionel
    \item[{Type de données}]
  \hyperref[TEI.teidata.pointer]{teidata.pointer}
    \item[{Note}]
  \par
Si aucune valeur n'est donnée, il est entendu que l'élément précède l'élément immédiatement suivant au même niveau hiérarchique.
\end{reflist}  
\end{sansreflist}  
\end{reflist}  
\begin{reflist}
\item[]\begin{specHead}{TEI.att.transcriptional}{att.transcriptional}\index{att.transcriptional (attribute class)|oddindex}\index{status=@status!att.transcriptional (attribute class)|oddindex}\index{cause=@cause!att.transcriptional (attribute class)|oddindex}\index{seq=@seq!att.transcriptional (attribute class)|oddindex} fournit des attributs spécifiques au codage d'éléments relatifs à l'intervention de l'auteur ou du copiste dans un texte lors de la transcription de sources manuscrites ou assimilées.\end{specHead} 
    \item[{Module}]
  tei
    \item[{Membres}]
  \hyperref[TEI.add]{add} \hyperref[TEI.addSpan]{addSpan} \hyperref[TEI.del]{del} \hyperref[TEI.delSpan]{delSpan} \hyperref[TEI.mod]{mod} \hyperref[TEI.redo]{redo} \hyperref[TEI.restore]{restore} \hyperref[TEI.retrace]{retrace} \hyperref[TEI.subst]{subst} \hyperref[TEI.substJoin]{substJoin} \hyperref[TEI.undo]{undo}
    \item[{Attributs}]
  Attributs \hyperref[TEI.att.editLike]{att.editLike} (\textit{@evidence}, \textit{@instant})  (\hyperref[TEI.att.dimensions]{att.dimensions} (\textit{@unit}, \textit{@quantity}, \textit{@extent}, \textit{@precision}, \textit{@scope}) (\hyperref[TEI.att.ranging]{att.ranging} (\textit{@atLeast}, \textit{@atMost}, \textit{@min}, \textit{@max}, \textit{@confidence})) ) \hyperref[TEI.att.written]{att.written} (\textit{@hand}) \hfil\\[-10pt]\begin{sansreflist}
    \item[@status]
  indique la conséquence de l'intervention, par exemple dans le cas d'un effacement, une biffure, qui inclut trop ou pas assez de texte, ou dans le cas d'un ajout, une insertion, qui reproduit une portion du texte déjà présent.
\begin{reflist}
    \item[{Statut}]
  Optionel
    \item[{Type de données}]
  \hyperref[TEI.teidata.enumerated]{teidata.enumerated}
    \item[{Exemple de valeurs possibles:}]
  \begin{description}

\item[{duplicate}]tout le texte indiqué comme étant une addition reprend le texte de l'original, que la duplication soit identique mot pour mot ou moins exacte.
\item[{duplicate-partial}]la partie du texte indiquée comme étant un ajout est redondante avec un texte présent dans l'original.
\item[{excessStart}]un passage du texte situé au début de la supression est indiqué comme supprimé bien qu'à l'évidence il ne devrait pas l'être.
\item[{excessEnd}]un passage du texte situé à la fin de la supression est indiqué comme supprimé bien qu'à l'évidence il ne devrait pas l'être.
\item[{shortStart}]un passage du texte situé au début de la supression n'est pas indiqué comme supprimé bien qu'à l'évidence il devrait l'être.
\item[{shortEnd}]un passage du texte situé à la fin de la supression n'est pas indiqué comme supprimé bien qu'à l'évidence il devrait l'être.
\item[{partial}]un passage du texte dans la supression n'est pas indiqué comme disparu bien qu'à l'évidence il devrait l'être.
\item[{unremarkable}]l'indication de suppression n'est pas erronée.{[Valeur par défaut] }
\end{description} 
    \item[{Note}]
  \par
Il est rarement nécessaire de donner de l'information sur le statut de chaque suppression sauf dans le cas des éditions critiques de manuscrits d'auteur, l'information sur le statut des additions étant encore plus rare.\par
L'indication d'une suppression ou d'une addition comme erronée est indéniablement un acte d'interprétation ; le test habituel appliqué dans la pratique est l'acceptabilité linguistique du texte avec et sans les lettres ou mots en question. 
\end{reflist}  
    \item[@cause]
  documents the presumed cause for the intervention.
\begin{reflist}
    \item[{Statut}]
  Optionel
    \item[{Type de données}]
  \hyperref[TEI.teidata.enumerated]{teidata.enumerated}
\end{reflist}  
    \item[@seq]
  (séquence) assigne un numéro séquentiel relatif à l'ordre dans lequel les traits encodés portant cet attribut sont supposés être apparus.
\begin{reflist}
    \item[{Statut}]
  Optionel
    \item[{Type de données}]
  \hyperref[TEI.teidata.count]{teidata.count}
\end{reflist}  
\end{sansreflist}  
\end{reflist}  
\begin{reflist}
\item[]\begin{specHead}{TEI.att.translatable}{att.translatable}\index{att.translatable (attribute class)|oddindex}\index{versionDate=@versionDate!att.translatable (attribute class)|oddindex} fournit les attributs utilisés pour indiquer le statut d'une partie traduisible d'un document ODD.\end{specHead} 
    \item[{Module}]
  tei
    \item[{Membres}]
  \hyperref[TEI.desc]{desc} \hyperref[TEI.gloss]{gloss}
    \item[{Attributs}]
  Attributs\hfil\\[-10pt]\begin{sansreflist}
    \item[@versionDate]
  spécifie le nom de la version ou le numéro de la source dont la version traduite a été tirée.
\begin{reflist}
    \item[{Statut}]
  Optionel
    \item[{Type de données}]
  \hyperref[TEI.teidata.temporal.w3c]{teidata.temporal.w3c}
\end{reflist}  
\end{sansreflist}  
\end{reflist}  
\begin{reflist}
\item[]\begin{specHead}{TEI.att.typed}{att.typed}\index{att.typed (attribute class)|oddindex}\index{type=@type!att.typed (attribute class)|oddindex}\index{subtype=@subtype!att.typed (attribute class)|oddindex} fournit des attributs qui peuvent être utilisés pour classer ou interclasser des éléments de n'importe quelle façon.\end{specHead} 
    \item[{Module}]
  tei
    \item[{Membres}]
  \hyperref[TEI.att.pointing.group]{att.pointing.group}[\hyperref[TEI.altGrp]{altGrp} \hyperref[TEI.joinGrp]{joinGrp} \hyperref[TEI.linkGrp]{linkGrp}] \hyperref[TEI.TEI]{TEI} \hyperref[TEI.ab]{ab} \hyperref[TEI.accMat]{accMat} \hyperref[TEI.add]{add} \hyperref[TEI.addName]{addName} \hyperref[TEI.addSpan]{addSpan} \hyperref[TEI.alt]{alt} \hyperref[TEI.altIdentifier]{altIdentifier} \hyperref[TEI.am]{am} \hyperref[TEI.anchor]{anchor} \hyperref[TEI.annotationBlock]{annotationBlock} \hyperref[TEI.application]{application} \hyperref[TEI.bibl]{bibl} \hyperref[TEI.biblStruct]{biblStruct} \hyperref[TEI.binaryObject]{binaryObject} \hyperref[TEI.c]{c} \hyperref[TEI.cb]{cb} \hyperref[TEI.change]{change} \hyperref[TEI.cit]{cit} \hyperref[TEI.cl]{cl} \hyperref[TEI.collection]{collection} \hyperref[TEI.corr]{corr} \hyperref[TEI.country]{country} \hyperref[TEI.custEvent]{custEvent} \hyperref[TEI.damage]{damage} \hyperref[TEI.damageSpan]{damageSpan} \hyperref[TEI.date]{date} \hyperref[TEI.decoNote]{decoNote} \hyperref[TEI.del]{del} \hyperref[TEI.delSpan]{delSpan} \hyperref[TEI.desc]{desc} \hyperref[TEI.dim]{dim} \hyperref[TEI.div]{div} \hyperref[TEI.event]{event} \hyperref[TEI.explicit]{explicit} \hyperref[TEI.figure]{figure} \hyperref[TEI.filiation]{filiation} \hyperref[TEI.finalRubric]{finalRubric} \hyperref[TEI.floatingText]{floatingText} \hyperref[TEI.forename]{forename} \hyperref[TEI.gb]{gb} \hyperref[TEI.genName]{genName} \hyperref[TEI.geogName]{geogName} \hyperref[TEI.gloss]{gloss} \hyperref[TEI.group]{group} \hyperref[TEI.head]{head} \hyperref[TEI.incipit]{incipit} \hyperref[TEI.join]{join} \hyperref[TEI.label]{label} \hyperref[TEI.lb]{lb} \hyperref[TEI.lg]{lg} \hyperref[TEI.line]{line} \hyperref[TEI.link]{link} \hyperref[TEI.listAnnotation]{listAnnotation} \hyperref[TEI.listBibl]{listBibl} \hyperref[TEI.listOrg]{listOrg} \hyperref[TEI.listPlace]{listPlace} \hyperref[TEI.location]{location} \hyperref[TEI.locus]{locus} \hyperref[TEI.m]{m} \hyperref[TEI.measureGrp]{measureGrp} \hyperref[TEI.media]{media} \hyperref[TEI.milestone]{milestone} \hyperref[TEI.mod]{mod} \hyperref[TEI.msDesc]{msDesc} \hyperref[TEI.msFrag]{msFrag} \hyperref[TEI.msName]{msName} \hyperref[TEI.msPart]{msPart} \hyperref[TEI.name]{name} \hyperref[TEI.nameLink]{nameLink} \hyperref[TEI.notatedMusic]{notatedMusic} \hyperref[TEI.note]{note} \hyperref[TEI.org]{org} \hyperref[TEI.orgName]{orgName} \hyperref[TEI.origDate]{origDate} \hyperref[TEI.origPlace]{origPlace} \hyperref[TEI.pb]{pb} \hyperref[TEI.pc]{pc} \hyperref[TEI.persName]{persName} \hyperref[TEI.phr]{phr} \hyperref[TEI.place]{place} \hyperref[TEI.placeName]{placeName} \hyperref[TEI.provenance]{provenance} \hyperref[TEI.ptr]{ptr} \hyperref[TEI.quote]{quote} \hyperref[TEI.ref]{ref} \hyperref[TEI.reg]{reg} \hyperref[TEI.region]{region} \hyperref[TEI.relatedItem]{relatedItem} \hyperref[TEI.restore]{restore} \hyperref[TEI.roleName]{roleName} \hyperref[TEI.rs]{rs} \hyperref[TEI.rubric]{rubric} \hyperref[TEI.s]{s} \hyperref[TEI.schemaRef]{schemaRef} \hyperref[TEI.seal]{seal} \hyperref[TEI.seg]{seg} \hyperref[TEI.settlement]{settlement} \hyperref[TEI.space]{space} \hyperref[TEI.stamp]{stamp} \hyperref[TEI.standOff]{standOff} \hyperref[TEI.state]{state} \hyperref[TEI.surface]{surface} \hyperref[TEI.surfaceGrp]{surfaceGrp} \hyperref[TEI.surname]{surname} \hyperref[TEI.table]{table} \hyperref[TEI.teiCorpus]{teiCorpus} \hyperref[TEI.term]{term} \hyperref[TEI.text]{text} \hyperref[TEI.time]{time} \hyperref[TEI.w]{w} \hyperref[TEI.zone]{zone}
    \item[{Attributs}]
  Attributs\hfil\\[-10pt]\begin{sansreflist}
    \item[@type]
  caractérise l'élément en utilisant n'importe quel système ou typologie de classification approprié.
\begin{reflist}
    \item[{Statut}]
  Optionel
    \item[{Type de données}]
  \hyperref[TEI.teidata.enumerated]{teidata.enumerated}
    \item[]\exampleFont {<\textbf{div}\hspace*{6pt}{type}="{verse}">}\mbox{}\newline 
\hspace*{6pt}{<\textbf{head}>}Night in Tarras{</\textbf{head}>}\mbox{}\newline 
\hspace*{6pt}{<\textbf{lg}\hspace*{6pt}{type}="{stanza}">}\mbox{}\newline 
\hspace*{6pt}\hspace*{6pt}{<\textbf{l}>}At evening tramping on the hot white road{</\textbf{l}>}\mbox{}\newline 
\hspace*{6pt}\hspace*{6pt}{<\textbf{l}>}…{</\textbf{l}>}\mbox{}\newline 
\hspace*{6pt}{</\textbf{lg}>}\mbox{}\newline 
\hspace*{6pt}{<\textbf{lg}\hspace*{6pt}{type}="{stanza}">}\mbox{}\newline 
\hspace*{6pt}\hspace*{6pt}{<\textbf{l}>}A wind sprang up from nowhere as the sky{</\textbf{l}>}\mbox{}\newline 
\hspace*{6pt}\hspace*{6pt}{<\textbf{l}>}…{</\textbf{l}>}\mbox{}\newline 
\hspace*{6pt}{</\textbf{lg}>}\mbox{}\newline 
{</\textbf{div}>}
\end{reflist}  
    \item[@subtype]
  (sous-type) fournit une sous-catégorisation de l'élément, si c'est nécessaire.
\begin{reflist}
    \item[{Statut}]
  Optionel
    \item[{Type de données}]
  \hyperref[TEI.teidata.enumerated]{teidata.enumerated}
    \item[{Note}]
  \par
L'attribut {\itshape subtype} peut être employé pour fournir une sous-classification pour cet élément, en plus de celle fournie par son propre attribut {\itshape type}.
\end{reflist}  
\end{sansreflist}  
    \item[{Schematron}]
   <sch:rule context="tei:*[@subtype]"> <sch:assert test="@type">The <sch:name/> element should not be categorized in detail with @subtype unless also categorized in general with @type</sch:assert> </sch:rule>
\end{reflist}  
\begin{reflist}
\item[]\begin{specHead}{TEI.att.written}{att.written}\index{att.written (attribute class)|oddindex}\index{hand=@hand!att.written (attribute class)|oddindex} provides an attribute to indicate the hand in which the textual content of an element was written in the source being transcribed.\end{specHead} 
    \item[{Module}]
  tei
    \item[{Membres}]
  \hyperref[TEI.att.damaged]{att.damaged}[\hyperref[TEI.damage]{damage} \hyperref[TEI.damageSpan]{damageSpan}] \hyperref[TEI.att.transcriptional]{att.transcriptional}[\hyperref[TEI.add]{add} \hyperref[TEI.addSpan]{addSpan} \hyperref[TEI.del]{del} \hyperref[TEI.delSpan]{delSpan} \hyperref[TEI.mod]{mod} \hyperref[TEI.redo]{redo} \hyperref[TEI.restore]{restore} \hyperref[TEI.retrace]{retrace} \hyperref[TEI.subst]{subst} \hyperref[TEI.substJoin]{substJoin} \hyperref[TEI.undo]{undo}] \hyperref[TEI.ab]{ab} \hyperref[TEI.div]{div} \hyperref[TEI.fw]{fw} \hyperref[TEI.head]{head} \hyperref[TEI.hi]{hi} \hyperref[TEI.label]{label} \hyperref[TEI.line]{line} \hyperref[TEI.note]{note} \hyperref[TEI.p]{p} \hyperref[TEI.seg]{seg} \hyperref[TEI.text]{text} \hyperref[TEI.zone]{zone}
    \item[{Attributs}]
  Attributs\hfil\\[-10pt]\begin{sansreflist}
    \item[@hand]
  signale la main de celui qui est intervenue.
\begin{reflist}
    \item[{Statut}]
  Optionel
    \item[{Type de données}]
  \hyperref[TEI.teidata.pointer]{teidata.pointer}
\end{reflist}  
\end{sansreflist}  
\end{reflist}  
\section[{Macros}]{Macros}\index{macro.limitedContent (macro)|oddindex}
\begin{reflist}
\item[]\begin{specHead}{TEI.macro.limitedContent}{macro.limitedContent} (contenu du paragraphe) définit le contenu des éléments textuels qui ne sont pas utilisés pour la transcription des contenus existants.\end{specHead} 
    \item[{Module}]
  tei
    \item[{Utilisé par}]
  \hyperref[TEI.desc]{desc} \hyperref[TEI.fDescr]{fDescr} \hyperref[TEI.figDesc]{figDesc} \hyperref[TEI.fsDescr]{fsDescr} \hyperref[TEI.meeting]{meeting} \hyperref[TEI.rendition]{rendition}
    \item[{Modèle de contenu}]
  \mbox{}\hfill\\[-10pt]\begin{Verbatim}[fontsize=\small]
<content>
 <alternate maxOccurs="unbounded"
  minOccurs="0">
  <textNode/>
  <classRef key="model.limitedPhrase"/>
  <classRef key="model.inter"/>
 </alternate>
</content>
    
\end{Verbatim}

    \item[{Declaration}]
  \mbox{}\hfill\\[-10pt]\begin{Verbatim}[fontsize=\small]
tei_macro.limitedContent =
   ( text | tei_model.limitedPhrase | tei_model.inter )*
\end{Verbatim}

\end{reflist}  \index{macro.paraContent (macro)|oddindex}
\begin{reflist}
\item[]\begin{specHead}{TEI.macro.paraContent}{macro.paraContent} (contenu de paragraphe.) définit le contenu de paragraphes et d' éléments semblables.\end{specHead} 
    \item[{Module}]
  tei
    \item[{Utilisé par}]
  \hyperref[TEI.ab]{ab} \hyperref[TEI.add]{add} \hyperref[TEI.corr]{corr} \hyperref[TEI.damage]{damage} \hyperref[TEI.del]{del} \hyperref[TEI.docEdition]{docEdition} \hyperref[TEI.emph]{emph} \hyperref[TEI.hi]{hi} \hyperref[TEI.mod]{mod} \hyperref[TEI.orig]{orig} \hyperref[TEI.p]{p} \hyperref[TEI.ref]{ref} \hyperref[TEI.reg]{reg} \hyperref[TEI.restore]{restore} \hyperref[TEI.retrace]{retrace} \hyperref[TEI.secl]{secl} \hyperref[TEI.seg]{seg} \hyperref[TEI.sic]{sic} \hyperref[TEI.supplied]{supplied} \hyperref[TEI.surplus]{surplus} \hyperref[TEI.title]{title} \hyperref[TEI.titlePart]{titlePart} \hyperref[TEI.unclear]{unclear}
    \item[{Modèle de contenu}]
  \mbox{}\hfill\\[-10pt]\begin{Verbatim}[fontsize=\small]
<content>
 <alternate maxOccurs="unbounded"
  minOccurs="0">
  <textNode/>
  <classRef key="model.gLike"/>
  <classRef key="model.phrase"/>
  <classRef key="model.inter"/>
  <classRef key="model.global"/>
  <elementRef key="lg"/>
  <classRef key="model.lLike"/>
 </alternate>
</content>
    
\end{Verbatim}

    \item[{Declaration}]
  \mbox{}\hfill\\[-10pt]\begin{Verbatim}[fontsize=\small]
tei_macro.paraContent =
   (
      text
    | tei_model.gLike    | tei_model.phrase    | tei_model.inter    | tei_model.global    | tei_lg    | tei_model.lLike   )*
\end{Verbatim}

\end{reflist}  \index{macro.phraseSeq (macro)|oddindex}
\begin{reflist}
\item[]\begin{specHead}{TEI.macro.phraseSeq}{macro.phraseSeq} (suite de syntagmes.) définit un ordre de données et d'éléments syntagmatiques.\end{specHead} 
    \item[{Module}]
  tei
    \item[{Utilisé par}]
  \hyperref[TEI.abbr]{abbr} \hyperref[TEI.addName]{addName} \hyperref[TEI.addrLine]{addrLine} \hyperref[TEI.affiliation]{affiliation} \hyperref[TEI.author]{author} \hyperref[TEI.biblScope]{biblScope} \hyperref[TEI.catchwords]{catchwords} \hyperref[TEI.citedRange]{citedRange} \hyperref[TEI.cl]{cl} \hyperref[TEI.colophon]{colophon} \hyperref[TEI.country]{country} \hyperref[TEI.distinct]{distinct} \hyperref[TEI.distributor]{distributor} \hyperref[TEI.docAuthor]{docAuthor} \hyperref[TEI.docDate]{docDate} \hyperref[TEI.edition]{edition} \hyperref[TEI.editor]{editor} \hyperref[TEI.email]{email} \hyperref[TEI.expan]{expan} \hyperref[TEI.explicit]{explicit} \hyperref[TEI.extent]{extent} \hyperref[TEI.finalRubric]{finalRubric} \hyperref[TEI.foreign]{foreign} \hyperref[TEI.forename]{forename} \hyperref[TEI.fw]{fw} \hyperref[TEI.genName]{genName} \hyperref[TEI.geogName]{geogName} \hyperref[TEI.gloss]{gloss} \hyperref[TEI.headItem]{headItem} \hyperref[TEI.headLabel]{headLabel} \hyperref[TEI.heraldry]{heraldry} \hyperref[TEI.incipit]{incipit} \hyperref[TEI.label]{label} \hyperref[TEI.material]{material} \hyperref[TEI.measure]{measure} \hyperref[TEI.mentioned]{mentioned} \hyperref[TEI.name]{name} \hyperref[TEI.nameLink]{nameLink} \hyperref[TEI.num]{num} \hyperref[TEI.objectType]{objectType} \hyperref[TEI.orgName]{orgName} \hyperref[TEI.origPlace]{origPlace} \hyperref[TEI.persName]{persName} \hyperref[TEI.phr]{phr} \hyperref[TEI.placeName]{placeName} \hyperref[TEI.pubPlace]{pubPlace} \hyperref[TEI.publisher]{publisher} \hyperref[TEI.region]{region} \hyperref[TEI.roleName]{roleName} \hyperref[TEI.rs]{rs} \hyperref[TEI.rubric]{rubric} \hyperref[TEI.s]{s} \hyperref[TEI.secFol]{secFol} \hyperref[TEI.settlement]{settlement} \hyperref[TEI.soCalled]{soCalled} \hyperref[TEI.speaker]{speaker} \hyperref[TEI.stamp]{stamp} \hyperref[TEI.street]{street} \hyperref[TEI.surname]{surname} \hyperref[TEI.term]{term} \hyperref[TEI.textLang]{textLang} \hyperref[TEI.watermark]{watermark}
    \item[{Modèle de contenu}]
  \mbox{}\hfill\\[-10pt]\begin{Verbatim}[fontsize=\small]
<content>
 <alternate maxOccurs="unbounded"
  minOccurs="0">
  <textNode/>
  <classRef key="model.gLike"/>
  <classRef key="model.phrase"/>
  <classRef key="model.global"/>
 </alternate>
</content>
    
\end{Verbatim}

    \item[{Declaration}]
  \mbox{}\hfill\\[-10pt]\begin{Verbatim}[fontsize=\small]
tei_macro.phraseSeq =
   ( text | tei_model.gLike | tei_model.phrase | tei_model.global )*
\end{Verbatim}

\end{reflist}  \index{macro.phraseSeq.limited (macro)|oddindex}
\begin{reflist}
\item[]\begin{specHead}{TEI.macro.phraseSeq.limited}{macro.phraseSeq.limited} (séquence d'expression délimitée) définit un ordre de données de caractère et ces éléments de niveau d'expression qui ne sont pas typiquement utilisées pour transcrire des documents existants.\end{specHead} 
    \item[{Module}]
  tei
    \item[{Utilisé par}]
  \hyperref[TEI.authority]{authority} \hyperref[TEI.classCode]{classCode} \hyperref[TEI.funder]{funder} \hyperref[TEI.language]{language} \hyperref[TEI.resp]{resp} \hyperref[TEI.span]{span}
    \item[{Modèle de contenu}]
  \mbox{}\hfill\\[-10pt]\begin{Verbatim}[fontsize=\small]
<content>
 <alternate maxOccurs="unbounded"
  minOccurs="0">
  <textNode/>
  <classRef key="model.limitedPhrase"/>
  <classRef key="model.global"/>
 </alternate>
</content>
    
\end{Verbatim}

    \item[{Declaration}]
  \mbox{}\hfill\\[-10pt]\begin{Verbatim}[fontsize=\small]
tei_macro.phraseSeq.limited =
   ( text | tei_model.limitedPhrase | tei_model.global )*
\end{Verbatim}

\end{reflist}  \index{macro.specialPara (macro)|oddindex}
\begin{reflist}
\item[]\begin{specHead}{TEI.macro.specialPara}{macro.specialPara} (contenu "spécial" de paragraphe) définit le modèle de contenu des éléments tels que des notes ou des items de liste, contenant soit une suite d'éléments de niveau composant soit qui ont la même structure qu'un paragraphe, contenant une suite d’éléments du niveau de l’expression et de niveau intermédiaire.\end{specHead} 
    \item[{Module}]
  tei
    \item[{Utilisé par}]
  \hyperref[TEI.accMat]{accMat} \hyperref[TEI.acquisition]{acquisition} \hyperref[TEI.additions]{additions} \hyperref[TEI.cell]{cell} \hyperref[TEI.change]{change} \hyperref[TEI.collation]{collation} \hyperref[TEI.condition]{condition} \hyperref[TEI.custEvent]{custEvent} \hyperref[TEI.decoNote]{decoNote} \hyperref[TEI.filiation]{filiation} \hyperref[TEI.foliation]{foliation} \hyperref[TEI.item]{item} \hyperref[TEI.layout]{layout} \hyperref[TEI.licence]{licence} \hyperref[TEI.metamark]{metamark} \hyperref[TEI.musicNotation]{musicNotation} \hyperref[TEI.note]{note} \hyperref[TEI.origin]{origin} \hyperref[TEI.provenance]{provenance} \hyperref[TEI.q]{q} \hyperref[TEI.quote]{quote} \hyperref[TEI.said]{said} \hyperref[TEI.signatures]{signatures} \hyperref[TEI.source]{source} \hyperref[TEI.stage]{stage} \hyperref[TEI.summary]{summary} \hyperref[TEI.support]{support} \hyperref[TEI.surrogates]{surrogates} \hyperref[TEI.typeNote]{typeNote}
    \item[{Modèle de contenu}]
  \mbox{}\hfill\\[-10pt]\begin{Verbatim}[fontsize=\small]
<content>
 <alternate maxOccurs="unbounded"
  minOccurs="0">
  <textNode/>
  <classRef key="model.gLike"/>
  <classRef key="model.phrase"/>
  <classRef key="model.inter"/>
  <classRef key="model.divPart"/>
  <classRef key="model.global"/>
 </alternate>
</content>
    
\end{Verbatim}

    \item[{Declaration}]
  \mbox{}\hfill\\[-10pt]\begin{Verbatim}[fontsize=\small]
tei_macro.specialPara =
   (
      text
    | tei_model.gLike    | tei_model.phrase    | tei_model.inter    | tei_model.divPart    | tei_model.global   )*
\end{Verbatim}

\end{reflist}  \index{macro.xtext (macro)|oddindex}
\begin{reflist}
\item[]\begin{specHead}{TEI.macro.xtext}{macro.xtext} (texte étendu) définit une suite de caractères et d'éléments gaiji\end{specHead} 
    \item[{Module}]
  tei
    \item[{Utilisé par}]
  \hyperref[TEI.c]{c} \hyperref[TEI.collection]{collection} \hyperref[TEI.depth]{depth} \hyperref[TEI.dim]{dim} \hyperref[TEI.ex]{ex} \hyperref[TEI.height]{height} \hyperref[TEI.institution]{institution} \hyperref[TEI.numeric]{numeric} \hyperref[TEI.repository]{repository} \hyperref[TEI.string]{string} \hyperref[TEI.width]{width}
    \item[{Modèle de contenu}]
  \mbox{}\hfill\\[-10pt]\begin{Verbatim}[fontsize=\small]
<content>
 <alternate maxOccurs="unbounded"
  minOccurs="0">
  <textNode/>
  <classRef key="model.gLike"/>
 </alternate>
</content>
    
\end{Verbatim}

    \item[{Declaration}]
  \fbox{\ttfamily tei\textunderscore macro.xtext = ( text | tei\textunderscore model.gLike )*} 
\end{reflist}  
\section[{Datatypes}]{Datatypes}
\begin{reflist}
\item[]\begin{specHead}{TEI.teidata.certainty}{teidata.certainty} Définit la gamme des valeurs d'attribut exprimant un degré de certitude\end{specHead} 
    \item[{Module}]
  tei
    \item[{Utilisé par}]
  \hyperref[TEI.teidata.probCert]{teidata.probCert}
    \item[{Modèle de contenu}]
  \mbox{}\hfill\\[-10pt]\begin{Verbatim}[fontsize=\small]
<content>
 <valList type="closed">
  <valItem ident="high"/>
  <valItem ident="medium"/>
  <valItem ident="low"/>
  <valItem ident="unknown"/>
 </valList>
</content>
    
\end{Verbatim}

    \item[{Declaration}]
  \mbox{}\hfill\\[-10pt]\begin{Verbatim}[fontsize=\small]
tei_teidata.certainty = "high" | "medium" | "low" | "unknown"
\end{Verbatim}

    \item[{Note}]
  \par
Le degré de certitude peut être exprimé par l'une des valeurs symboliques prédéfinies high, medium, ou low.
\end{reflist}  
\begin{reflist}
\item[]\begin{specHead}{TEI.teidata.count}{teidata.count} définit la gamme des valeurs des attributs exprimant une valeur entière et non négative utilisé pour des calculs.\end{specHead} 
    \item[{Module}]
  tei
    \item[{Utilisé par}]
  Elément: \begin{itemize}
\item \hyperref[TEI.handDesc]{handDesc}/@hands
\item \hyperref[TEI.layout]{layout}/@columns
\item \hyperref[TEI.layout]{layout}/@ruledLines
\item \hyperref[TEI.layout]{layout}/@writtenLines
\item \hyperref[TEI.table]{table}/@rows
\item \hyperref[TEI.table]{table}/@cols
\item \hyperref[TEI.term]{term}/@level
\item \hyperref[TEI.zone]{zone}/@rotate
\end{itemize} 
    \item[{Modèle de contenu}]
  \mbox{}\hfill\\[-10pt]\begin{Verbatim}[fontsize=\small]
<content>
 <dataRef name="nonNegativeInteger"/>
</content>
    
\end{Verbatim}

    \item[{Declaration}]
  \fbox{\ttfamily tei\textunderscore teidata.count = xsd:nonNegativeInteger} 
    \item[{Note}]
  \par
Seules des valeurs positives entières sont autorisées.
\end{reflist}  
\begin{reflist}
\item[]\begin{specHead}{TEI.teidata.duration.iso}{teidata.duration.iso} définit la gamme de valeurs d'attributs exprimant une durée temporaraire utilisant le norme ISO 8601.\end{specHead} 
    \item[{Module}]
  tei
    \item[{Utilisé par}]
  
    \item[{Modèle de contenu}]
  \mbox{}\hfill\\[-10pt]\begin{Verbatim}[fontsize=\small]
<content>
 <dataRef name="token"
  restriction="[0-9.,DHMPRSTWYZ/:+\-]+"/>
</content>
    
\end{Verbatim}

    \item[{Declaration}]
  \mbox{}\hfill\\[-10pt]\begin{Verbatim}[fontsize=\small]
tei_teidata.duration.iso = token { pattern = "[0-9.,DHMPRSTWYZ/:+\-]+" }
\end{Verbatim}

    \item[{Exemple}]
  \leavevmode\bgroup\exampleFont \begin{shaded}\noindent\mbox{}{<\textbf{time}\hspace*{6pt}{dur-iso}="{PT0,75H}">}trois quarts d'une heure{</\textbf{time}>}\end{shaded}\egroup 


    \item[{Exemple}]
  \leavevmode\bgroup\exampleFont \begin{shaded}\noindent\mbox{}{<\textbf{date}\hspace*{6pt}{dur-iso}="{P1,5D}">}une journee et demie{</\textbf{date}>}\end{shaded}\egroup 


    \item[{Exemple}]
  \leavevmode\bgroup\exampleFont \begin{shaded}\noindent\mbox{}{<\textbf{date}\hspace*{6pt}{dur-iso}="{P14D}">}une quinzaine{</\textbf{date}>}\end{shaded}\egroup 


    \item[{Exemple}]
  \leavevmode\bgroup\exampleFont \begin{shaded}\noindent\mbox{}{<\textbf{time}\hspace*{6pt}{dur-iso}="{PT0.02S}">}20 ms{</\textbf{time}>}\end{shaded}\egroup 


    \item[{Note}]
  \par
Une durée est exprimée par une suite de paires alphanumériques, précédée par la lettre P ; la lettre donne l'unité et peut être Y (année), M (mois), D (jour), H (heure), M (minute), ou S (seconde), dans cet ordre. Les nombres sont des entiers sans signe, à l'exception du dernier, qui peut comporter une décimale (en utilisant soit \texttt{.} soit \texttt{,} pour la virgule ; la dernière possibilité est préférable). Si un nombre est \textit{0}, alors la paire alphanumérique peut être omise. Si les paires alphanumériques H (heure), M (minute), ou S (seconde) sont présentes, alors le séparateur \texttt{T} doit précéder la première paire alphanumérique ‘time’.\par
Pour des détails complets, voir ISO 8601 \textit{Data elements and interchange formats — Information interchange — Representation of dates and times}.
\end{reflist}  
\begin{reflist}
\item[]\begin{specHead}{TEI.teidata.duration.w3c}{teidata.duration.w3c} définit la gamme des valeurs d'attributs exprimant une durée temporaraire utilisant les types de données W3C\end{specHead} 
    \item[{Module}]
  tei
    \item[{Utilisé par}]
  
    \item[{Modèle de contenu}]
  \fbox{\ttfamily <content>\newline
 <dataRef name="duration"/>\newline
</content>\newline
    } 
    \item[{Declaration}]
  \fbox{\ttfamily tei\textunderscore teidata.duration.w3c = xsd:duration} 
    \item[{Exemple}]
  \leavevmode\bgroup\exampleFont \begin{shaded}\noindent\mbox{}{<\textbf{time}\hspace*{6pt}{dur}="{PT45M}">}quarante-cinq minutes{</\textbf{time}>}\end{shaded}\egroup 


    \item[{Exemple}]
  \leavevmode\bgroup\exampleFont \begin{shaded}\noindent\mbox{}{<\textbf{date}\hspace*{6pt}{dur}="{P1DT12H}">}une journée et demie{</\textbf{date}>}\end{shaded}\egroup 


    \item[{Exemple}]
  \leavevmode\bgroup\exampleFont \begin{shaded}\noindent\mbox{}{<\textbf{date}\hspace*{6pt}{dur}="{P7D}">}une semaine{</\textbf{date}>}\end{shaded}\egroup 


    \item[{Exemple}]
  \leavevmode\bgroup\exampleFont \begin{shaded}\noindent\mbox{}{<\textbf{time}\hspace*{6pt}{dur}="{PT0.02S}">}20 ms{</\textbf{time}>}\end{shaded}\egroup 


    \item[{Note}]
  \par
Une durée est exprimée par une suite de paires alphanumériques, précédée par la lettre P ; la lettre donne l'unité et peut être Y (année), M (mois), D (jour), H (heure), M (minute), ou S (seconde), dans cet ordre. Les nombres sont des entiers non signés à l'exception du dernier, qui peut comporter une décimale (en utilisant soit \texttt{.} soit \texttt{,} pour la virgule ; la dernière possibilité est préférable). Si un nombre est \textit{0}, alors la paire alphanumérique peut être omise. Si les paires alphanumériques H (heure), M (minute), ou S (seconde) sont présentes, alors le séparateur \texttt{T} doit précéder la première paire alphanumérique ‘time’.\par
Pour des détails complets, voir \xref{http://www.w3.org/TR/2004/REC-xmlschema-2-20041028/\#duration}{W3C specification}.
\end{reflist}  
\begin{reflist}
\item[]\begin{specHead}{TEI.teidata.enumerated}{teidata.enumerated} définit la gamme de valeurs des attributs exprimant un nom XML extrait d'une liste de possibilités documentées\end{specHead} 
    \item[{Module}]
  tei
    \item[{Utilisé par}]
  Elément: \begin{itemize}
\item \hyperref[TEI.abbr]{abbr}/@type
\item \hyperref[TEI.affiliation]{affiliation}/@type
\item \hyperref[TEI.alt]{alt}/@mode
\item \hyperref[TEI.altGrp]{altGrp}/@mode
\item \hyperref[TEI.availability]{availability}/@status
\item \hyperref[TEI.correction]{correction}/@status
\item \hyperref[TEI.correction]{correction}/@method
\item \hyperref[TEI.dimensions]{dimensions}/@type
\item \hyperref[TEI.distinct]{distinct}/@type
\item \hyperref[TEI.divGen]{divGen}/@type
\item \hyperref[TEI.fs]{fs}/@type
\item \hyperref[TEI.fsDecl]{fsDecl}/@type
\item \hyperref[TEI.fsdLink]{fsdLink}/@type
\item \hyperref[TEI.fw]{fw}/@type
\item \hyperref[TEI.gap]{gap}/@reason
\item \hyperref[TEI.gap]{gap}/@agent
\item \hyperref[TEI.idno]{idno}/@type
\item \hyperref[TEI.join]{join}/@scope
\item \hyperref[TEI.list]{list}/@type
\item \hyperref[TEI.measure]{measure}/@type
\item \hyperref[TEI.num]{num}/@type
\item \hyperref[TEI.objectDesc]{objectDesc}/@form
\item \hyperref[TEI.pc]{pc}/@force
\item \hyperref[TEI.pc]{pc}/@unit
\item \hyperref[TEI.person]{person}/@role
\item \hyperref[TEI.person]{person}/@age
\item \hyperref[TEI.personGrp]{personGrp}/@role
\item \hyperref[TEI.personGrp]{personGrp}/@age
\item \hyperref[TEI.persona]{persona}/@role
\item \hyperref[TEI.persona]{persona}/@age
\item \hyperref[TEI.q]{q}/@type
\item \hyperref[TEI.rendition]{rendition}/@scope
\item \hyperref[TEI.space]{space}/@dim
\item \hyperref[TEI.stage]{stage}/@type
\item \hyperref[TEI.supportDesc]{supportDesc}/@material
\item \hyperref[TEI.surface]{surface}/@attachment
\item \hyperref[TEI.timeline]{timeline}/@unit
\item \hyperref[TEI.title]{title}/@type
\item \hyperref[TEI.title]{title}/@level
\item \hyperref[TEI.titlePage]{titlePage}/@type
\item \hyperref[TEI.titlePart]{titlePart}/@type
\item \hyperref[TEI.unclear]{unclear}/@agent
\item \hyperref[TEI.vColl]{vColl}/@org
\item \hyperref[TEI.vMerge]{vMerge}/@org
\item \hyperref[TEI.when]{when}/@unit
\end{itemize} 
    \item[{Modèle de contenu}]
  \fbox{\ttfamily <content>\newline
 <dataRef key="teidata.word"/>\newline
</content>\newline
    } 
    \item[{Declaration}]
  \fbox{\ttfamily tei\textunderscore teidata.enumerated = teidata.word} 
    \item[{Note}]
  \par
Les attributs utilisant ce type de données doivent contenir un mot qui suit les règles de définition d'un nom XML valide (voir \url{http://www.w3.org/TR/REC-xml/\#dt-name}): par exemple ils ne peuvent pas contenir des blancs ni commencer par des chiffres.\par
Normalement, la liste des possibilités documentées est fournie (ou exemplifiée) par une liste de valeurs dans la spécification de l'attribut associé, exprimée par un élément \texttt{<valList>}.
\end{reflist}  
\begin{reflist}
\item[]\begin{specHead}{TEI.teidata.interval}{teidata.interval} defines attribute values used to express an interval value.\end{specHead} 
    \item[{Module}]
  tei
    \item[{Utilisé par}]
  Elément: \begin{itemize}
\item \hyperref[TEI.timeline]{timeline}/@interval
\item \hyperref[TEI.when]{when}/@interval
\end{itemize} 
    \item[{Modèle de contenu}]
  \mbox{}\hfill\\[-10pt]\begin{Verbatim}[fontsize=\small]
<content>
 <alternate maxOccurs="1" minOccurs="1">
  <dataRef name="float"/>
  <valList>
   <valItem ident="regular"/>
   <valItem ident="irregular"/>
   <valItem ident="unknown"/>
  </valList>
 </alternate>
</content>
    
\end{Verbatim}

    \item[{Declaration}]
  \mbox{}\hfill\\[-10pt]\begin{Verbatim}[fontsize=\small]
tei_teidata.interval = xsd:float | ( "regular" | "irregular" | "unknown" )
\end{Verbatim}

\end{reflist}  
\begin{reflist}
\item[]\begin{specHead}{TEI.teidata.language}{teidata.language} définit la gamme des valeurs d'attributs exprimant une combinaison particulière du langage humain avec un système d'écriture.\end{specHead} 
    \item[{Module}]
  tei
    \item[{Utilisé par}]
  Elément: \begin{itemize}
\item \hyperref[TEI.language]{language}/@ident
\item \hyperref[TEI.textLang]{textLang}/@mainLang
\item \hyperref[TEI.textLang]{textLang}/@otherLangs
\end{itemize} 
    \item[{Modèle de contenu}]
  \mbox{}\hfill\\[-10pt]\begin{Verbatim}[fontsize=\small]
<content>
 <alternate maxOccurs="1" minOccurs="1">
  <dataRef name="language"/>
  <valList>
   <valItem ident=""/>
  </valList>
 </alternate>
</content>
    
\end{Verbatim}

    \item[{Declaration}]
  \fbox{\ttfamily tei\textunderscore teidata.language = xsd:language | ( "" )} 
    \item[{Note}]
  \par
Les valeurs pour cet attribut sont les ‘étiquettes’ de langue définies dans la norme \xref{https://tools.ietf.org/html/bcp47}{BCP 47}. Actuellement, la norme BCP 47 intègre les normes RFC 4646 et RFC 4647 ; à l'avenir, d'autres documents de l'IETF pourront leur succéder en tant que meilleure pratique.\par
Une ‘étiquette de langue’, pour la norme BCP 47, est formée par l'assemblage d'une suite de composants ou de \textit{sous-étiquettes} reliés par un trait d'union (\textit{-}, U+002D). L'étiquette est composée des sous-étiquettes suivantes, dans l'ordre indiqué. Chaque sous-étiquette est facultative, à l'exception de la première. Chacune ne peut avoir qu'une occurrence, sauf les quatrième et cinquième (variante et extension), qui sont répétables. \begin{description}

\item[{langue}]Code de langue enregistré par l'IANA. Il est presque toujours identique au code de langue alphabétique ISO 639-2, s'il y en a un. La liste des sous-étiquettes de langue enregistrées est disponible à : \url{http://www.iana.org/assignments/language-subtag-registry}Il est recommandé d'écrire ce code en minuscules.
\item[{écriture}]Code ISO 15924 pour l'écriture. Ces codes sont constitués de 4 lettres, et il est recommandé d'écrire la première lettre en majuscule, les trois autres en minuscules. La liste canonique des codes est maintenue par le Consortium Unicode, et elle est disponible à : \url{http://unicode.org/iso15924/iso15924-codes.html}. L'IETF recommande d'omettre ce code, sauf s'il est nécessaire pour établir une distinction.
\item[{région}]Soit un code de pays ISO 3166, soit un code de région UN M.49 enregistré par l'IANA (tous les codes de ce type ne sont pas enregistrés : par exemple, ne sont pas enregistrés les codes UN pour des regroupements économiques ou les codes de pays pour lesquels il existe déjà un code de pays alphabétique ISO 3166-2). Le premier est constitué de 2 lettres, et il est recommandé de l'écrire en majuscules. La liste des codes est disponible à : \url{http://www.iso.org/iso/en/prods-services/iso3166ma/02iso-3166-code-lists/index.html}. Le second est constitué de 3 chiffres ; la liste des codes est disponible à : \url{http://unstats.un.org/unsd/methods/m49/m49.htm}.
\item[{variante}]Variante enregistrée par l'IANA. Ces codes ‘sont utilisés pour indiquer des variantes additionnelles et bien établies, qui définissent une langue ou ses dialectes et qui ne sont pas couverts par d'autres sous-étiquettes existantes’.
\item[{extension}]Une extension a la forme d'une lettre unique, suivie d'un trait d'union, lui-même suivi de sous-étiquettes additionnelles. Ces dernières existent pour tenir compte d'une future extension de la norme BCP 47, mais à l'heure actuelle de telles extensions ne sont pas utilisées.
\item[{usage privé}]Une extension utilisant la sous-étiquette initiale de la lettre \textit{x} (i.e., commençant par \texttt{x-}) n'a pas d'autre signification que celle négociée entre les parties impliquées. Ces sous-étiquettes doivent être utilisées avec beaucoup de prudence, car elles interfèrent avec l'interopérabilité que l'utilisation de la norme RFC 4646 vise à promouvoir. Pour qu'un document qui utilise ces sous-étiquettes soit conforme à la TEI, un élément \hyperref[TEI.language]{<language>} correspondant doit être présent dans l'en-tête TEI.
\end{description} \par
Il y a deux exceptions au format ci-dessus. Premièrement, il y a des codes de langue dans le \xref{http://www.iana.org/assignments/language-subtag-registry}{registre de l'IANA} qui ne correspondent pas à la syntaxe ci-dessus, mais qui sont présents car ils ont été ‘hérités’ de spécifications antérieures.\par
En second lieu, une étiquette complète de langue peut consister seulement en une sous-étiquette d'usage privé. Ces étiquettes commencent par \texttt{x-} ; il n'est pas nécessaire qu'elles suivent les autres règles établies par l'IETF et acceptées par les présents Principes directeurs. Comme toutes les étiquettes de langue qui utilisent des sous-étiquettes d'usage privé, la langue en question doit être documentée dans un élément correspondant \hyperref[TEI.language]{<language>} dans l'en-tête TEI.\par
Les exemples incluent :\begin{description}

\item[{sn}]Shona
\item[{zh-TW}]Taïwanais
\item[{zh-Hant-HK}]Chinois de Hong Kong écrit dans l'écriture traditionnelle
\item[{en-SL}]Anglais parlé au Sierra Leone
\item[{pl}]Polonais
\item[{es-MX}]Espagnol parlé au Mexique
\item[{es-419}]Espagnol parlé en Amérique latine
\end{description} \par
La W3C Internationalization Activity a publié une introduction à la norme BCP 47 dont la lecture peut être utile : \xref{http://www.w3.org/International/articles/language-tags/Overview.en.php}{Language tags in HTML and XML}.
\end{reflist}  
\begin{reflist}
\item[]\begin{specHead}{TEI.teidata.name}{teidata.name} définit la gamme des valeurs d'attribut exprimant un nom XML\end{specHead} 
    \item[{Module}]
  tei
    \item[{Utilisé par}]
  Elément: \begin{itemize}
\item \hyperref[TEI.application]{application}/@ident
\item \hyperref[TEI.f]{f}/@name
\item \hyperref[TEI.fDecl]{fDecl}/@name
\item \hyperref[TEI.fsDecl]{fsDecl}/@baseTypes
\item \hyperref[TEI.index]{index}/@indexName
\item \hyperref[TEI.join]{join}/@result
\item \hyperref[TEI.joinGrp]{joinGrp}/@result
\end{itemize} 
    \item[{Modèle de contenu}]
  \fbox{\ttfamily <content>\newline
 <dataRef name="Name"/>\newline
</content>\newline
    } 
    \item[{Declaration}]
  \fbox{\ttfamily tei\textunderscore teidata.name = xsd:Name} 
    \item[{Note}]
  \par
Les attributs utilisant ce type de données doivent contenir un seul mot, qui suit les règles de définition d'un nom XML valide (voir \url{http://www.w3.org/TR/REC-xml/\#dt-name}) : par exemple ils ne peuvent contenir de blancs ou commencer par des chiffres.
\end{reflist}  
\begin{reflist}
\item[]\begin{specHead}{TEI.teidata.namespace}{teidata.namespace} définit la gamme des valeurs d'attributs exprimant une espace de noms XML tels qu'ils sont définis par le W3C.\end{specHead} 
    \item[{Module}]
  tei
    \item[{Utilisé par}]
  Elément: \begin{itemize}
\item \hyperref[TEI.namespace]{namespace}/@name
\end{itemize} 
    \item[{Modèle de contenu}]
  \fbox{\ttfamily <content>\newline
 <dataRef name="anyURI"/>\newline
</content>\newline
    } 
    \item[{Declaration}]
  \fbox{\ttfamily tei\textunderscore teidata.namespace = xsd:anyURI} 
    \item[{Note}]
  \par
La gamme des valeurs syntaxiquement valides est définie par \xref{http://www.ietf.org/rfc/rfc3986.txt}{RFC 3986 \textit{Uniform Resource Identifier (URI): Generic Syntax}}.
\end{reflist}  
\begin{reflist}
\item[]\begin{specHead}{TEI.teidata.numeric}{teidata.numeric} définit la gamme des valeurs d'attributs utilisées pour des valeurs numériques\end{specHead} 
    \item[{Module}]
  tei
    \item[{Utilisé par}]
  Elément: \begin{itemize}
\item \hyperref[TEI.num]{num}/@value
\item \hyperref[TEI.numeric]{numeric}/@value
\item \hyperref[TEI.numeric]{numeric}/@max
\end{itemize} 
    \item[{Modèle de contenu}]
  \mbox{}\hfill\\[-10pt]\begin{Verbatim}[fontsize=\small]
<content>
 <alternate maxOccurs="1" minOccurs="1">
  <dataRef name="double"/>
  <dataRef name="token"
   restriction="(\-?[\d]+/\-?[\d]+)"/>
  <dataRef name="decimal"/>
 </alternate>
</content>
    
\end{Verbatim}

    \item[{Declaration}]
  \mbox{}\hfill\\[-10pt]\begin{Verbatim}[fontsize=\small]
tei_teidata.numeric =
   xsd:double | token { pattern = "(\-?[\d]+/\-?[\d]+)" } | xsd:decimal
\end{Verbatim}

    \item[{Note}]
  \par
Toute valeur numérique, représentée en nombre décimal, notée en virgule flottante ou en fraction.\par
Pour représenter un nombre en virgule flottante, exprimé en notation scientifique, ‘E notation’, une variante de la ‘notation exponentielle’ peut être utilisée. Dans ce format, la valeur est exprimée par deux nombres séparés par la lettre E. Le premier facteur, le significande (parfois appelé mantisse) est donné sous forme décimale, tandis que le second est un entier. La valeur est obtenue en multipliant la mantisse par 10 fois le nombre indiqué par l'entier. Ainsi la valeur représentée en notation décimale 1000.0 pourrait être représentée en notation scientifique 10E3.\par
Une valeur exprimée en fraction est représentée par deux nombres entiers séparés par une barre oblique (/). Ainsi, la valeur représentée en notation décimale 0.5 pourrait être représentée en fraction par la chaîne de caractères 1/2.
\end{reflist}  
\begin{reflist}
\item[]\begin{specHead}{TEI.teidata.outputMeasurement}{teidata.outputMeasurement} définit la gamme de valeurs exprimant les dimensions d'un objet destiné à être affiché\end{specHead} 
    \item[{Module}]
  tei
    \item[{Utilisé par}]
  
    \item[{Modèle de contenu}]
  \mbox{}\hfill\\[-10pt]\begin{Verbatim}[fontsize=\small]
<content>
 <dataRef name="token"
  restriction="[\-+]?\d+(\.\d+)?(%|cm|mm|in|pt|pc|px|em|ex|gd|rem|vw|vh|vm)"/>
</content>
    
\end{Verbatim}

    \item[{Declaration}]
  \mbox{}\hfill\\[-10pt]\begin{Verbatim}[fontsize=\small]
tei_teidata.outputMeasurement =
   token
   {
      pattern = "[\-+]?\d+(\.\d+)?(%|cm|mm|in|pt|pc|px|em|ex|gd|rem|vw|vh|vm)"
   }
\end{Verbatim}

    \item[{Exemple}]
  \leavevmode\bgroup\exampleFont \begin{shaded}\noindent\mbox{}{<\textbf{figure}>}\mbox{}\newline 
\hspace*{6pt}{<\textbf{head}>}Le logo TEI{</\textbf{head}>}\mbox{}\newline 
\hspace*{6pt}{<\textbf{graphic}\hspace*{6pt}{height}="{600px}"\mbox{}\newline 
\hspace*{6pt}\hspace*{6pt}{url}="{http://www.tei-c.org/logos/TEI-600.jpg}"\hspace*{6pt}{width}="{600px}"/>}\mbox{}\newline 
{</\textbf{figure}>}\end{shaded}\egroup 


    \item[{Note}]
  \par
Ces valeurs peuvent être reportées directement sur des valeurs utilisées par XSL-FO et CSS. Pour les définitions des unités, voir ces spécifications ; à ce jour la liste la plus complète est dans un \xref{http://www.w3.org/TR/2005/WD-css3-values-20050726/\#numbers0}{CSS3 working draft}.
\end{reflist}  
\begin{reflist}
\item[]\begin{specHead}{TEI.teidata.pattern}{teidata.pattern} définit la gamme des valeurs d'attributs exprimant une expression régulière\end{specHead} 
    \item[{Module}]
  tei
    \item[{Utilisé par}]
  
    \item[{Modèle de contenu}]
  \fbox{\ttfamily <content>\newline
 <dataRef name="token"/>\newline
</content>\newline
    } 
    \item[{Declaration}]
  \fbox{\ttfamily tei\textunderscore teidata.pattern = token} 
    \item[{Note}]
  \par
‘Une expression régulière, souvent appelée \textit{modèle}, est une expression qui décrit un jeu de chaînes de caractères. Elles sont généralement utilisées pour donner une brève description d'un jeu, sans avoir à en lister tous les éléments. Par exemple, le jeu contenant les trois chaînes de caractères \textit{Handel}, \textit{Händel}, et \textit{Haendel} peut être décrit comme le modèle \texttt{H(ä|ae?)ndel} (ou on peut dire que \texttt{H(ä|ae?)ndel} \textit{équivaut à} chacune des trois chaînes)\xref{http://fr.wikipedia.org/wiki/Expression_rationnelle}{wikipedia}\xref{http://en.wikipedia.org/wiki/Regular_expression\#Basic_concepts}{wikipedia}’
\end{reflist}  
\begin{reflist}
\item[]\begin{specHead}{TEI.teidata.point}{teidata.point} defines the data type used to express a point in cartesian space.\end{specHead} 
    \item[{Module}]
  tei
    \item[{Utilisé par}]
  
    \item[{Modèle de contenu}]
  \mbox{}\hfill\\[-10pt]\begin{Verbatim}[fontsize=\small]
<content>
 <dataRef name="token"
  restriction="(\-?[0-9]+\.?[0-9]*,\-?[0-9]+\.?[0-9]*)"/>
</content>
    
\end{Verbatim}

    \item[{Declaration}]
  \mbox{}\hfill\\[-10pt]\begin{Verbatim}[fontsize=\small]
tei_teidata.point =
   token { pattern = "(\-?[0-9]+\.?[0-9]*,\-?[0-9]+\.?[0-9]*)" }
\end{Verbatim}

    \item[{Exemple}]
  \leavevmode\bgroup\exampleFont \begin{shaded}\noindent\mbox{}{<\textbf{facsimile}>}\mbox{}\newline 
\hspace*{6pt}{<\textbf{surface}\hspace*{6pt}{lrx}="{400}"\hspace*{6pt}{lry}="{280}"\hspace*{6pt}{ulx}="{0}"\hspace*{6pt}{uly}="{0}">}\mbox{}\newline 
\hspace*{6pt}\hspace*{6pt}{<\textbf{zone}\hspace*{6pt}{points}="{220,100 300,210 170,250 123,234}">}\mbox{}\newline 
\hspace*{6pt}\hspace*{6pt}\hspace*{6pt}{<\textbf{graphic}\hspace*{6pt}{url}="{handwriting.png }"/>}\mbox{}\newline 
\hspace*{6pt}\hspace*{6pt}{</\textbf{zone}>}\mbox{}\newline 
\hspace*{6pt}{</\textbf{surface}>}\mbox{}\newline 
{</\textbf{facsimile}>}\end{shaded}\egroup 


\end{reflist}  
\begin{reflist}
\item[]\begin{specHead}{TEI.teidata.pointer}{teidata.pointer} définit la gamme des valeurs d'attributs utilisées pour fournir un pointeur URI unique sur une autre ressource, soit dans le document courant, soit dans un autre document\end{specHead} 
    \item[{Module}]
  tei
    \item[{Utilisé par}]
  Elément: \begin{itemize}
\item \hyperref[TEI.alt]{alt}/@target
\item \hyperref[TEI.bibl]{bibl}/@scheme
\item \hyperref[TEI.change]{change}/@target
\item \hyperref[TEI.classCode]{classCode}/@scheme
\item \hyperref[TEI.date]{date}/@scheme
\item \hyperref[TEI.event]{event}/@where
\item \hyperref[TEI.f]{f}/@fVal
\item \hyperref[TEI.fs]{fs}/@feats
\item \hyperref[TEI.fsdLink]{fsdLink}/@target
\item \hyperref[TEI.gap]{gap}/@hand
\item \hyperref[TEI.geogName]{geogName}/@scheme
\item \hyperref[TEI.handShift]{handShift}/@new
\item \hyperref[TEI.keywords]{keywords}/@scheme
\item \hyperref[TEI.locus]{locus}/@scheme
\item \hyperref[TEI.locusGrp]{locusGrp}/@scheme
\item \hyperref[TEI.metamark]{metamark}/@target
\item \hyperref[TEI.note]{note}/@scheme
\item \hyperref[TEI.note]{note}/@targetEnd
\item \hyperref[TEI.orgName]{orgName}/@scheme
\item \hyperref[TEI.p]{p}/@scheme
\item \hyperref[TEI.persName]{persName}/@scheme
\item \hyperref[TEI.placeName]{placeName}/@scheme
\item \hyperref[TEI.publisher]{publisher}/@scheme
\item \hyperref[TEI.redo]{redo}/@target
\item \hyperref[TEI.ref]{ref}/@scheme
\item \hyperref[TEI.relatedItem]{relatedItem}/@target
\item \hyperref[TEI.space]{space}/@resp
\item \hyperref[TEI.span]{span}/@from
\item \hyperref[TEI.span]{span}/@to
\item \hyperref[TEI.timeline]{timeline}/@origin
\item \hyperref[TEI.unclear]{unclear}/@hand
\item \hyperref[TEI.undo]{undo}/@target
\item \hyperref[TEI.w]{w}/@lemmaRef
\item \hyperref[TEI.when]{when}/@since
\end{itemize} 
    \item[{Modèle de contenu}]
  \fbox{\ttfamily <content>\newline
 <dataRef name="anyURI"/>\newline
</content>\newline
    } 
    \item[{Declaration}]
  \fbox{\ttfamily tei\textunderscore teidata.pointer = xsd:anyURI} 
    \item[{Note}]
  \par
La gamme des valeurs valides syntaxiquement est définie par\xref{http://www.ietf.org/rfc/rfc3986.txt}{RFC 3986 \textit{Uniform Resource Identifier (URI): Generic Syntax}}
\end{reflist}  
\begin{reflist}
\item[]\begin{specHead}{TEI.teidata.probCert}{teidata.probCert} defines a range of attribute values which can be expressed either as a numeric probability or as a coded certainty value.\end{specHead} 
    \item[{Module}]
  tei
    \item[{Utilisé par}]
  
    \item[{Modèle de contenu}]
  \mbox{}\hfill\\[-10pt]\begin{Verbatim}[fontsize=\small]
<content>
 <alternate maxOccurs="1" minOccurs="1">
  <dataRef key="teidata.probability"/>
  <dataRef key="teidata.certainty"/>
 </alternate>
</content>
    
\end{Verbatim}

    \item[{Declaration}]
  \mbox{}\hfill\\[-10pt]\begin{Verbatim}[fontsize=\small]
tei_teidata.probCert = teidata.probability | teidata.certainty
\end{Verbatim}

\end{reflist}  
\begin{reflist}
\item[]\begin{specHead}{TEI.teidata.probability}{teidata.probability} définit la gamme des valeurs d'attributs exprimant une probabilité.\end{specHead} 
    \item[{Module}]
  tei
    \item[{Utilisé par}]
  \hyperref[TEI.teidata.probCert]{teidata.probCert}Elément: \begin{itemize}
\item \hyperref[TEI.alt]{alt}/@weights
\end{itemize} 
    \item[{Modèle de contenu}]
  \fbox{\ttfamily <content>\newline
 <dataRef name="double"/>\newline
</content>\newline
    } 
    \item[{Declaration}]
  \fbox{\ttfamily tei\textunderscore teidata.probability = xsd:double} 
    \item[{Note}]
  \par
Le degré de probabilité est exprimé par un nombre réel entre 0 et 1 ; 0 représentant \textit{certainement faux} et 1 \textit{certainement vrai}.
\end{reflist}  
\begin{reflist}
\item[]\begin{specHead}{TEI.teidata.replacement}{teidata.replacement} defines attribute values which contain a replacement template.\end{specHead} 
    \item[{Module}]
  tei
    \item[{Utilisé par}]
  
    \item[{Modèle de contenu}]
  \fbox{\ttfamily <content>\newline
 <textNode/>\newline
</content>\newline
    } 
    \item[{Declaration}]
  \fbox{\ttfamily tei\textunderscore teidata.replacement = text} 
\end{reflist}  
\begin{reflist}
\item[]\begin{specHead}{TEI.teidata.sex}{teidata.sex} définit la gamme des valeurs d'attributs employés pour identifier le sexe humain ou animal.\end{specHead} 
    \item[{Module}]
  tei
    \item[{Utilisé par}]
  Elément: \begin{itemize}
\item \hyperref[TEI.person]{person}/@sex
\item \hyperref[TEI.personGrp]{personGrp}/@sex
\item \hyperref[TEI.persona]{persona}/@sex
\end{itemize} 
    \item[{Modèle de contenu}]
  \fbox{\ttfamily <content>\newline
 <dataRef key="teidata.word"/>\newline
</content>\newline
    } 
    \item[{Declaration}]
  \fbox{\ttfamily tei\textunderscore teidata.sex = teidata.word} 
    \item[{Note}]
  \par
des valeurs sont celle de l' SO 5218:2004 \textit{identification des sexes humains.}; 0 : inconnu ; 1 : homme ; 2 : femme ; et 9 : non applicable.
\end{reflist}  
\begin{reflist}
\item[]\begin{specHead}{TEI.teidata.temporal.iso}{teidata.temporal.iso} définit la gamme des valeurs d'attribut qui sont capables d''exprimer une valeur temporelle comme une date, une période, ou une combinaison des deux qui se conforment au standard international \textit{Data elements and interchange formats – Information interchange – Representation of dates and times}\end{specHead} 
    \item[{Module}]
  tei
    \item[{Utilisé par}]
  
    \item[{Modèle de contenu}]
  \mbox{}\hfill\\[-10pt]\begin{Verbatim}[fontsize=\small]
<content>
 <alternate maxOccurs="1" minOccurs="1">
  <dataRef name="date"/>
  <dataRef name="gYear"/>
  <dataRef name="gMonth"/>
  <dataRef name="gDay"/>
  <dataRef name="gYearMonth"/>
  <dataRef name="gMonthDay"/>
  <dataRef name="time"/>
  <dataRef name="dateTime"/>
  <dataRef name="token"
   restriction="[0-9.,DHMPRSTWYZ/:+\-]+"/>
 </alternate>
</content>
    
\end{Verbatim}

    \item[{Declaration}]
  \mbox{}\hfill\\[-10pt]\begin{Verbatim}[fontsize=\small]
tei_teidata.temporal.iso =
   xsd:date
 | xsd:gYear
 | xsd:gMonth
 | xsd:gDay
 | xsd:gYearMonth
 | xsd:gMonthDay
 | xsd:time
 | xsd:dateTime
 | token { pattern = "[0-9.,DHMPRSTWYZ/:+\-]+" }
\end{Verbatim}

    \item[{Note}]
  \par
S'il est vraisemblable que la valeur utilisée soit destinée à être comparer à d’autres valeurs, alors une indication du fuseau horaire devrait toujours être incluse, et seule la représentation \textit{dateTime} devrait être employée.\par
Pour toutes les représentations pour lesquelles l’ISO 8601 décrit à la fois un format \textit{basique} et un format\textit{étendu}, ce guide d’encodage recommandande l’emploi du format \textit{étendu}.\par
Même si l’ ISO 8601 permet d’écrire à la fois \texttt{00:00} et \texttt{24:00} pour minuit, ce guide d’encodage déconseille vivement d’écrire \texttt{24:00}.
\end{reflist}  
\begin{reflist}
\item[]\begin{specHead}{TEI.teidata.temporal.w3c}{teidata.temporal.w3c} définit la gamme des valeurs d'attributs propre à exprimer une valeur temporelle comme une date, une période, ou une combinaison des deux conformément aux spécifications \textit{XML Schema Part 2: Datatypes Second Edition} du W3C.\end{specHead} 
    \item[{Module}]
  tei
    \item[{Utilisé par}]
  Elément: \begin{itemize}
\item \hyperref[TEI.docDate]{docDate}/@when
\item \hyperref[TEI.when]{when}/@absolute
\end{itemize} 
    \item[{Modèle de contenu}]
  \mbox{}\hfill\\[-10pt]\begin{Verbatim}[fontsize=\small]
<content>
 <alternate maxOccurs="1" minOccurs="1">
  <dataRef name="date"/>
  <dataRef name="gYear"/>
  <dataRef name="gMonth"/>
  <dataRef name="gDay"/>
  <dataRef name="gYearMonth"/>
  <dataRef name="gMonthDay"/>
  <dataRef name="time"/>
  <dataRef name="dateTime"/>
 </alternate>
</content>
    
\end{Verbatim}

    \item[{Declaration}]
  \mbox{}\hfill\\[-10pt]\begin{Verbatim}[fontsize=\small]
tei_teidata.temporal.w3c =
   xsd:date
 | xsd:gYear
 | xsd:gMonth
 | xsd:gDay
 | xsd:gYearMonth
 | xsd:gMonthDay
 | xsd:time
 | xsd:dateTime
\end{Verbatim}

    \item[{Note}]
  \par
S'il est probable que la valeur utilisée doive être comparée à d’autres, alors une indication de fuseau horaire sera toujours incluse, et seule la représentation de dateTime sera employée. 
\end{reflist}  
\begin{reflist}
\item[]\begin{specHead}{TEI.teidata.text}{teidata.text} définit la gamme des valeurs d'attributs exprimant une chaine de caracteres Unicode, y compris des espaces blancs.\end{specHead} 
    \item[{Module}]
  tei
    \item[{Utilisé par}]
  Elément: \begin{itemize}
\item \hyperref[TEI.distinct]{distinct}/@time
\item \hyperref[TEI.distinct]{distinct}/@space
\item \hyperref[TEI.distinct]{distinct}/@social
\item \hyperref[TEI.rendition]{rendition}/@selector
\item \hyperref[TEI.w]{w}/@lemma
\end{itemize} 
    \item[{Modèle de contenu}]
  \fbox{\ttfamily <content>\newline
 <dataRef name="string"/>\newline
</content>\newline
    } 
    \item[{Declaration}]
  \fbox{\ttfamily tei\textunderscore teidata.text = string} 
\end{reflist}  
\begin{reflist}
\item[]\begin{specHead}{TEI.teidata.truthValue}{teidata.truthValue} définit la gamme des valeurs d'attributs exprimant la vérité d'une proposition.\end{specHead} 
    \item[{Module}]
  tei
    \item[{Utilisé par}]
  Elément: \begin{itemize}
\item \hyperref[TEI.binary]{binary}/@value
\item \hyperref[TEI.fDecl]{fDecl}/@optional
\item \hyperref[TEI.note]{note}/@anchored
\item \hyperref[TEI.numeric]{numeric}/@trunc
\item \hyperref[TEI.pc]{pc}/@pre
\item \hyperref[TEI.surface]{surface}/@flipping
\end{itemize} 
    \item[{Modèle de contenu}]
  \fbox{\ttfamily <content>\newline
 <dataRef name="boolean"/>\newline
</content>\newline
    } 
    \item[{Declaration}]
  \fbox{\ttfamily tei\textunderscore teidata.truthValue = xsd:boolean} 
    \item[{Note}]
  \par
Ce type de données ne s'applique que dans les cas où l'incertitude est inappropriée ; c’est-à-dire si l'attribut concerné peut avoir une valeur autre que vrai ou faux, par ex. inconnu, ou inapplicable, il devrait alors y avoir la version étendue de ce type de données : data.xTruthValue.
\end{reflist}  
\begin{reflist}
\item[]\begin{specHead}{TEI.teidata.version}{teidata.version} définit la gamme des valeurs d'attribut exprimant un numéro de version TEI.\end{specHead} 
    \item[{Module}]
  tei
    \item[{Utilisé par}]
  Elément: \begin{itemize}
\item \hyperref[TEI.TEI]{TEI}/@version
\item \hyperref[TEI.teiCorpus]{teiCorpus}/@version
\end{itemize} 
    \item[{Modèle de contenu}]
  \mbox{}\hfill\\[-10pt]\begin{Verbatim}[fontsize=\small]
<content>
 <dataRef name="token"
  restriction="[\d]+(\.[\d]+){0,2}"/>
</content>
    
\end{Verbatim}

    \item[{Declaration}]
  \mbox{}\hfill\\[-10pt]\begin{Verbatim}[fontsize=\small]
tei_teidata.version = token { pattern = "[\d]+(\.[\d]+){0,2}" }
\end{Verbatim}

    \item[{Note}]
  \par
La valeur de cet attribut devrait suivre le format proposé par le Consortium Unicode pour identifier les versions (\url{http://unicode.org/versions/}). Un numéro de version ne contient que des chiffres et des points. Le numéro initiale identifie le numéro majeur de la version. Un deuxième et un troisième numéro sont aussi disponibles pour la numérotation facultative des versions mineures ou sous-mineurs.
\end{reflist}  
\begin{reflist}
\item[]\begin{specHead}{TEI.teidata.versionNumber}{teidata.versionNumber} defines the range of attribute values used for version numbers.\end{specHead} 
    \item[{Module}]
  tei
    \item[{Utilisé par}]
  Elément: \begin{itemize}
\item \hyperref[TEI.application]{application}/@version
\end{itemize} 
    \item[{Modèle de contenu}]
  \mbox{}\hfill\\[-10pt]\begin{Verbatim}[fontsize=\small]
<content>
 <dataRef name="token"
  restriction="[\d]+[a-z]*[\d]*(\.[\d]+[a-z]*[\d]*){0,3}"/>
</content>
    
\end{Verbatim}

    \item[{Declaration}]
  \mbox{}\hfill\\[-10pt]\begin{Verbatim}[fontsize=\small]
tei_teidata.versionNumber =
   token { pattern = "[\d]+[a-z]*[\d]*(\.[\d]+[a-z]*[\d]*){0,3}" }
\end{Verbatim}

\end{reflist}  
\begin{reflist}
\item[]\begin{specHead}{TEI.teidata.word}{teidata.word} définit la gamme des valeurs d'attributs exprimant un seul mot ou signe\end{specHead} 
    \item[{Module}]
  tei
    \item[{Utilisé par}]
  \hyperref[TEI.teidata.enumerated]{teidata.enumerated} \hyperref[TEI.teidata.sex]{teidata.sex}Elément: \begin{itemize}
\item \hyperref[TEI.binaryObject]{binaryObject}/@encoding
\item \hyperref[TEI.locus]{locus}/@from
\item \hyperref[TEI.locus]{locus}/@to
\item \hyperref[TEI.m]{m}/@baseForm
\item \hyperref[TEI.media]{media}/@mimeType
\item \hyperref[TEI.metamark]{metamark}/@function
\item \hyperref[TEI.org]{org}/@role
\item \hyperref[TEI.personGrp]{personGrp}/@size
\item \hyperref[TEI.secl]{secl}/@reason
\item \hyperref[TEI.supplied]{supplied}/@reason
\item \hyperref[TEI.surplus]{surplus}/@reason
\item \hyperref[TEI.symbol]{symbol}/@value
\item \hyperref[TEI.unclear]{unclear}/@reason
\item \hyperref[TEI.vLabel]{vLabel}/@name
\end{itemize} 
    \item[{Modèle de contenu}]
  \mbox{}\hfill\\[-10pt]\begin{Verbatim}[fontsize=\small]
<content>
 <dataRef name="token"
  restriction="(\p{L}|\p{N}|\p{P}|\p{S})+"/>
</content>
    
\end{Verbatim}

    \item[{Declaration}]
  \mbox{}\hfill\\[-10pt]\begin{Verbatim}[fontsize=\small]
tei_teidata.word = token { pattern = "(\p{L}|\p{N}|\p{P}|\p{S})+" }
\end{Verbatim}

    \item[{Note}]
  \par
Les attributs employant ce type de données doivent contenir un ‘mot’ simple ne contenant que des lettres, des chiffres, des signes de ponctuation, ou des symboles : ils ne peuvent donc pas inclure d’espace.
\end{reflist}  
\begin{reflist}
\item[]\begin{specHead}{TEI.teidata.xTruthValue}{teidata.xTruthValue} (extended truth value) définit la gamme des valeurs d'attributs exprimant une vérité potentiellement inconnue.\end{specHead} 
    \item[{Module}]
  tei
    \item[{Utilisé par}]
  Elément: \begin{itemize}
\item \hyperref[TEI.binding]{binding}/@contemporary
\item \hyperref[TEI.said]{said}/@aloud
\item \hyperref[TEI.said]{said}/@direct
\item \hyperref[TEI.seal]{seal}/@contemporary
\end{itemize} 
    \item[{Modèle de contenu}]
  \mbox{}\hfill\\[-10pt]\begin{Verbatim}[fontsize=\small]
<content>
 <alternate maxOccurs="1" minOccurs="1">
  <dataRef name="boolean"/>
  <valList>
   <valItem ident="unknown"/>
   <valItem ident="inapplicable"/>
  </valList>
 </alternate>
</content>
    
\end{Verbatim}

    \item[{Declaration}]
  \mbox{}\hfill\\[-10pt]\begin{Verbatim}[fontsize=\small]
tei_teidata.xTruthValue = xsd:boolean | ( "unknown" | "inapplicable" )
\end{Verbatim}

    \item[{Note}]
  \par
Dans le cas où l'incertitude n’est pas adaptée, employer plutot le type de données \textsf{data.TruthValue}.
\end{reflist}  
\begin{reflist}
\item[]\begin{specHead}{TEI.teidata.xmlName}{teidata.xmlName} defines attribute values which contain an XML name.\end{specHead} 
    \item[{Module}]
  tei
    \item[{Utilisé par}]
  Elément: \begin{itemize}
\item \hyperref[TEI.schemaRef]{schemaRef}/@key
\end{itemize} 
    \item[{Modèle de contenu}]
  \fbox{\ttfamily <content>\newline
 <dataRef name="NCName"/>\newline
</content>\newline
    } 
    \item[{Declaration}]
  \fbox{\ttfamily tei\textunderscore teidata.xmlName = xsd:NCName} 
\end{reflist}  
\begin{reflist}
\item[]\begin{specHead}{TEI.teidata.xpath}{teidata.xpath} defines attribute values which contain an XPath expression.\end{specHead} 
    \item[{Module}]
  tei
    \item[{Utilisé par}]
  
    \item[{Modèle de contenu}]
  \fbox{\ttfamily <content>\newline
 <textNode/>\newline
</content>\newline
    } 
    \item[{Declaration}]
  \fbox{\ttfamily tei\textunderscore teidata.xpath = text} 
\end{reflist}  
\end{document}
